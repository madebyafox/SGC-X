% Options for packages loaded elsewhere
\PassOptionsToPackage{unicode}{hyperref}
\PassOptionsToPackage{hyphens}{url}
\PassOptionsToPackage{dvipsnames,svgnames,x11names}{xcolor}
%
\documentclass[
  letterpaper,
  DIV=11,
  numbers=noendperiod]{scrreprt}
\usepackage{amsmath,amssymb}
\usepackage{lmodern}
\usepackage{iftex}
\ifPDFTeX
  \usepackage[T1]{fontenc}
  \usepackage[utf8]{inputenc}
  \usepackage{textcomp} % provide euro and other symbols
\else % if luatex or xetex
  \usepackage{unicode-math}
  \defaultfontfeatures{Scale=MatchLowercase}
  \defaultfontfeatures[\rmfamily]{Ligatures=TeX,Scale=1}
\fi
% Use upquote if available, for straight quotes in verbatim environments
\IfFileExists{upquote.sty}{\usepackage{upquote}}{}
\IfFileExists{microtype.sty}{% use microtype if available
  \usepackage[]{microtype}
  \UseMicrotypeSet[protrusion]{basicmath} % disable protrusion for tt fonts
}{}
\makeatletter
\@ifundefined{KOMAClassName}{% if non-KOMA class
  \IfFileExists{parskip.sty}{%
    \usepackage{parskip}
  }{% else
    \setlength{\parindent}{0pt}
    \setlength{\parskip}{6pt plus 2pt minus 1pt}}
}{% if KOMA class
  \KOMAoptions{parskip=half}}
\makeatother
\usepackage{xcolor}
\setlength{\emergencystretch}{3em} % prevent overfull lines
\setcounter{secnumdepth}{5}
% Make \paragraph and \subparagraph free-standing
\ifx\paragraph\undefined\else
  \let\oldparagraph\paragraph
  \renewcommand{\paragraph}[1]{\oldparagraph{#1}\mbox{}}
\fi
\ifx\subparagraph\undefined\else
  \let\oldsubparagraph\subparagraph
  \renewcommand{\subparagraph}[1]{\oldsubparagraph{#1}\mbox{}}
\fi

\usepackage{color}
\usepackage{fancyvrb}
\newcommand{\VerbBar}{|}
\newcommand{\VERB}{\Verb[commandchars=\\\{\}]}
\DefineVerbatimEnvironment{Highlighting}{Verbatim}{commandchars=\\\{\}}
% Add ',fontsize=\small' for more characters per line
\usepackage{framed}
\definecolor{shadecolor}{RGB}{241,243,245}
\newenvironment{Shaded}{\begin{snugshade}}{\end{snugshade}}
\newcommand{\AlertTok}[1]{\textcolor[rgb]{0.68,0.00,0.00}{#1}}
\newcommand{\AnnotationTok}[1]{\textcolor[rgb]{0.37,0.37,0.37}{#1}}
\newcommand{\AttributeTok}[1]{\textcolor[rgb]{0.40,0.45,0.13}{#1}}
\newcommand{\BaseNTok}[1]{\textcolor[rgb]{0.68,0.00,0.00}{#1}}
\newcommand{\BuiltInTok}[1]{\textcolor[rgb]{0.00,0.23,0.31}{#1}}
\newcommand{\CharTok}[1]{\textcolor[rgb]{0.13,0.47,0.30}{#1}}
\newcommand{\CommentTok}[1]{\textcolor[rgb]{0.37,0.37,0.37}{#1}}
\newcommand{\CommentVarTok}[1]{\textcolor[rgb]{0.37,0.37,0.37}{\textit{#1}}}
\newcommand{\ConstantTok}[1]{\textcolor[rgb]{0.56,0.35,0.01}{#1}}
\newcommand{\ControlFlowTok}[1]{\textcolor[rgb]{0.00,0.23,0.31}{#1}}
\newcommand{\DataTypeTok}[1]{\textcolor[rgb]{0.68,0.00,0.00}{#1}}
\newcommand{\DecValTok}[1]{\textcolor[rgb]{0.68,0.00,0.00}{#1}}
\newcommand{\DocumentationTok}[1]{\textcolor[rgb]{0.37,0.37,0.37}{\textit{#1}}}
\newcommand{\ErrorTok}[1]{\textcolor[rgb]{0.68,0.00,0.00}{#1}}
\newcommand{\ExtensionTok}[1]{\textcolor[rgb]{0.00,0.23,0.31}{#1}}
\newcommand{\FloatTok}[1]{\textcolor[rgb]{0.68,0.00,0.00}{#1}}
\newcommand{\FunctionTok}[1]{\textcolor[rgb]{0.28,0.35,0.67}{#1}}
\newcommand{\ImportTok}[1]{\textcolor[rgb]{0.00,0.46,0.62}{#1}}
\newcommand{\InformationTok}[1]{\textcolor[rgb]{0.37,0.37,0.37}{#1}}
\newcommand{\KeywordTok}[1]{\textcolor[rgb]{0.00,0.23,0.31}{#1}}
\newcommand{\NormalTok}[1]{\textcolor[rgb]{0.00,0.23,0.31}{#1}}
\newcommand{\OperatorTok}[1]{\textcolor[rgb]{0.37,0.37,0.37}{#1}}
\newcommand{\OtherTok}[1]{\textcolor[rgb]{0.00,0.23,0.31}{#1}}
\newcommand{\PreprocessorTok}[1]{\textcolor[rgb]{0.68,0.00,0.00}{#1}}
\newcommand{\RegionMarkerTok}[1]{\textcolor[rgb]{0.00,0.23,0.31}{#1}}
\newcommand{\SpecialCharTok}[1]{\textcolor[rgb]{0.37,0.37,0.37}{#1}}
\newcommand{\SpecialStringTok}[1]{\textcolor[rgb]{0.13,0.47,0.30}{#1}}
\newcommand{\StringTok}[1]{\textcolor[rgb]{0.13,0.47,0.30}{#1}}
\newcommand{\VariableTok}[1]{\textcolor[rgb]{0.07,0.07,0.07}{#1}}
\newcommand{\VerbatimStringTok}[1]{\textcolor[rgb]{0.13,0.47,0.30}{#1}}
\newcommand{\WarningTok}[1]{\textcolor[rgb]{0.37,0.37,0.37}{\textit{#1}}}

\providecommand{\tightlist}{%
  \setlength{\itemsep}{0pt}\setlength{\parskip}{0pt}}\usepackage{longtable,booktabs,array}
\usepackage{calc} % for calculating minipage widths
% Correct order of tables after \paragraph or \subparagraph
\usepackage{etoolbox}
\makeatletter
\patchcmd\longtable{\par}{\if@noskipsec\mbox{}\fi\par}{}{}
\makeatother
% Allow footnotes in longtable head/foot
\IfFileExists{footnotehyper.sty}{\usepackage{footnotehyper}}{\usepackage{footnote}}
\makesavenoteenv{longtable}
\usepackage{graphicx}
\makeatletter
\def\maxwidth{\ifdim\Gin@nat@width>\linewidth\linewidth\else\Gin@nat@width\fi}
\def\maxheight{\ifdim\Gin@nat@height>\textheight\textheight\else\Gin@nat@height\fi}
\makeatother
% Scale images if necessary, so that they will not overflow the page
% margins by default, and it is still possible to overwrite the defaults
% using explicit options in \includegraphics[width, height, ...]{}
\setkeys{Gin}{width=\maxwidth,height=\maxheight,keepaspectratio}
% Set default figure placement to htbp
\makeatletter
\def\fps@figure{htbp}
\makeatother
\newlength{\cslhangindent}
\setlength{\cslhangindent}{1.5em}
\newlength{\csllabelwidth}
\setlength{\csllabelwidth}{3em}
\newlength{\cslentryspacingunit} % times entry-spacing
\setlength{\cslentryspacingunit}{\parskip}
\newenvironment{CSLReferences}[2] % #1 hanging-ident, #2 entry spacing
 {% don't indent paragraphs
  \setlength{\parindent}{0pt}
  % turn on hanging indent if param 1 is 1
  \ifodd #1
  \let\oldpar\par
  \def\par{\hangindent=\cslhangindent\oldpar}
  \fi
  % set entry spacing
  \setlength{\parskip}{#2\cslentryspacingunit}
 }%
 {}
\usepackage{calc}
\newcommand{\CSLBlock}[1]{#1\hfill\break}
\newcommand{\CSLLeftMargin}[1]{\parbox[t]{\csllabelwidth}{#1}}
\newcommand{\CSLRightInline}[1]{\parbox[t]{\linewidth - \csllabelwidth}{#1}\break}
\newcommand{\CSLIndent}[1]{\hspace{\cslhangindent}#1}

\usepackage{booktabs}
\usepackage{longtable}
\usepackage{array}
\usepackage{multirow}
\usepackage{wrapfig}
\usepackage{float}
\usepackage{colortbl}
\usepackage{pdflscape}
\usepackage{tabu}
\usepackage{threeparttable}
\usepackage{threeparttablex}
\usepackage[normalem]{ulem}
\usepackage{makecell}
\usepackage{xcolor}
\KOMAoption{captions}{tableheading}
\makeatletter
\@ifpackageloaded{tcolorbox}{}{\usepackage[many]{tcolorbox}}
\@ifpackageloaded{fontawesome5}{}{\usepackage{fontawesome5}}
\definecolor{quarto-callout-color}{HTML}{909090}
\definecolor{quarto-callout-note-color}{HTML}{0758E5}
\definecolor{quarto-callout-important-color}{HTML}{CC1914}
\definecolor{quarto-callout-warning-color}{HTML}{EB9113}
\definecolor{quarto-callout-tip-color}{HTML}{00A047}
\definecolor{quarto-callout-caution-color}{HTML}{FC5300}
\definecolor{quarto-callout-color-frame}{HTML}{acacac}
\definecolor{quarto-callout-note-color-frame}{HTML}{4582ec}
\definecolor{quarto-callout-important-color-frame}{HTML}{d9534f}
\definecolor{quarto-callout-warning-color-frame}{HTML}{f0ad4e}
\definecolor{quarto-callout-tip-color-frame}{HTML}{02b875}
\definecolor{quarto-callout-caution-color-frame}{HTML}{fd7e14}
\makeatother
\makeatletter
\makeatother
\makeatletter
\@ifpackageloaded{caption}{}{\usepackage{caption}}
\AtBeginDocument{%
\ifdefined\contentsname
  \renewcommand*\contentsname{Table of contents}
\else
  \newcommand\contentsname{Table of contents}
\fi
\ifdefined\listfigurename
  \renewcommand*\listfigurename{List of Figures}
\else
  \newcommand\listfigurename{List of Figures}
\fi
\ifdefined\listtablename
  \renewcommand*\listtablename{List of Tables}
\else
  \newcommand\listtablename{List of Tables}
\fi
\ifdefined\figurename
  \renewcommand*\figurename{Figure}
\else
  \newcommand\figurename{Figure}
\fi
\ifdefined\tablename
  \renewcommand*\tablename{Table}
\else
  \newcommand\tablename{Table}
\fi
}
\@ifpackageloaded{float}{}{\usepackage{float}}
\floatstyle{ruled}
\@ifundefined{c@chapter}{\newfloat{codelisting}{h}{lop}}{\newfloat{codelisting}{h}{lop}[chapter]}
\floatname{codelisting}{Listing}
\newcommand*\listoflistings{\listof{codelisting}{List of Listings}}
\makeatother
\makeatletter
\@ifpackageloaded{caption}{}{\usepackage{caption}}
\@ifpackageloaded{subcaption}{}{\usepackage{subcaption}}
\makeatother
\makeatletter
\@ifpackageloaded{tcolorbox}{}{\usepackage[many]{tcolorbox}}
\makeatother
\makeatletter
\@ifundefined{shadecolor}{\definecolor{shadecolor}{rgb}{.97, .97, .97}}
\makeatother
\makeatletter
\makeatother
\ifLuaTeX
  \usepackage{selnolig}  % disable illegal ligatures
\fi
\IfFileExists{bookmark.sty}{\usepackage{bookmark}}{\usepackage{hyperref}}
\IfFileExists{xurl.sty}{\usepackage{xurl}}{} % add URL line breaks if available
\urlstyle{same} % disable monospaced font for URLs
\hypersetup{
  pdftitle={Response Rescoring},
  pdfauthor={Amy Rae Fox},
  colorlinks=true,
  linkcolor={blue},
  filecolor={Maroon},
  citecolor={Blue},
  urlcolor={Blue},
  pdfcreator={LaTeX via pandoc}}

\title{Response Rescoring}
\usepackage{etoolbox}
\makeatletter
\providecommand{\subtitle}[1]{% add subtitle to \maketitle
  \apptocmd{\@title}{\par {\large #1 \par}}{}{}
}
\makeatother
\subtitle{SGC3A-2-Response Rescoring}
\author{Amy Rae Fox}
\date{4/29/2022}

\begin{document}
\maketitle

\ifdefined\Shaded\renewenvironment{Shaded}{\begin{tcolorbox}[borderline west={3pt}{0pt}{shadecolor}, breakable, sharp corners, interior hidden, enhanced, boxrule=0pt, frame hidden]}{\end{tcolorbox}}\fi

\renewcommand*\contentsname{Table of contents}
{
\hypersetup{linkcolor=}
\setcounter{tocdepth}{2}
\tableofcontents
}
\hypertarget{status}{%
\chapter*{Status}\label{status}}
\addcontentsline{toc}{chapter}{Status}

\begin{itemize}
\tightlist
\item
  4/29/22 \textbar{} ported existing .Rmd analysis files to Quarto
  (.qmd) for sharing status w/ JMH CMW via web
\end{itemize}

\part{SGC3A}

\hypertarget{harmonization}{%
\chapter{Harmonization}\label{harmonization}}

\newpage

\emph{The purpose of this notebook is to harmonize data files for study
SGC\_3A.}

\begin{longtable}[]{@{}
  >{\raggedright\arraybackslash}p{(\columnwidth - 2\tabcolsep) * \real{0.7778}}
  >{\raggedright\arraybackslash}p{(\columnwidth - 2\tabcolsep) * \real{0.2222}}@{}}
\toprule()
\begin{minipage}[b]{\linewidth}\raggedright
Pre-Requisite
\end{minipage} & \begin{minipage}[b]{\linewidth}\raggedright
Followed By
\end{minipage} \\
\midrule()
\endhead
spring17\_clean\_data.Rmd spring18\_clean\_data.Rmd
fall21\_clean\_data.Rmd winter2022\_clean\_sgc3a.Rmd &
2\_sgc3A\_rescoring.qmd \\
\bottomrule()
\end{longtable}

\hypertarget{introduction}{%
\chapter{INTRODUCTION}\label{introduction}}

Data for study SGC\_3A were collected across four time periods,
interrupted by the Covid-19 pandemic.

\begin{longtable}[]{@{}ll@{}}
\toprule()
Period & Modality \\
\midrule()
\endhead
Fall 2017 & in person, SONA groups in computer lab \\
Spring 2018 & in person, SONA groups in computer lab \\
Fall 2021 & asynchronous, online, SONA \\
Winter 2022 & asynchronous, online, SONA \\
\bottomrule()
\end{longtable}

Data collected in Fall 2017, Spring 2018 constitute the original SGC\_3A
study, conducted in person. Data collected in Fall 2021, Winter 2022
constitute the web-based replication, conducted online (asynchronously).
In all cases, the experiment was administered via a web application.

\hypertarget{harmonization-1}{%
\chapter{HARMONIZATION}\label{harmonization-1}}

The underlying data structure of the stimulus web application changed
across the data collection period, resulting in slightly different data
files (i.e.~columns are not named consistently). In this section, we
combine the files from each data collection period into a single
\emph{harmonized} data file for analysis (one for participants, one for
items).

\hypertarget{participants}{%
\section{Participants}\label{participants}}

First we import participant-level data from each data collection period,
selecting only the columns relevant for analysis, and renaming columns
to be consistent across each file. The result is a single data frame
\texttt{df\_subjects} containing one row for each subject (across all
periods). Note that we \emph{are not} discarding any \emph{response}
data. Rather, we discard columns that are automatically recorded by the
stimulus web application and help the application run.

\emph{Note that we discard some columns representing scores calculated
in the stimulus engine. These scores were calculated differently across
collection periods, and so we discard them and recalculate scores in the
next analysis notebook.}

\begin{Shaded}
\begin{Highlighting}[]
\CommentTok{\#IMPORT PARTICIPANT DATA}

\CommentTok{\#set datafiles}
\NormalTok{fall17 }\OtherTok{\textless{}{-}} \StringTok{"data/session{-}level/fall17\_sgc3a\_participants.csv"}
\NormalTok{spring18 }\OtherTok{\textless{}{-}} \StringTok{"data/session{-}level/spring18\_sgc3a\_participants.csv"}
\NormalTok{fall21 }\OtherTok{\textless{}{-}} \StringTok{"data/session{-}level/fall21\_sgc3a\_participants.csv"}
\NormalTok{winter22 }\OtherTok{\textless{}{-}} \StringTok{"data/session{-}level/winter22\_sgc3a\_participants.rds"}

\CommentTok{\#read datafiles, set mode and term}
\NormalTok{df\_subjects\_fall17 }\OtherTok{\textless{}{-}} \FunctionTok{read.csv}\NormalTok{(fall17) }\SpecialCharTok{\%\textgreater{}\%} \FunctionTok{mutate}\NormalTok{(}\AttributeTok{mode =} \StringTok{"lab{-}synch"}\NormalTok{, }\AttributeTok{term =} \StringTok{"fall17"}\NormalTok{)}
\NormalTok{df\_subjects\_spring18 }\OtherTok{\textless{}{-}} \FunctionTok{read.csv}\NormalTok{(spring18) }\SpecialCharTok{\%\textgreater{}\%} \FunctionTok{mutate}\NormalTok{(}\AttributeTok{mode =} \StringTok{"lab{-}synch"}\NormalTok{, }\AttributeTok{term =} \StringTok{"spring18"}\NormalTok{)}
\NormalTok{df\_subjects\_fall21 }\OtherTok{\textless{}{-}} \FunctionTok{read.csv}\NormalTok{(fall21) }\SpecialCharTok{\%\textgreater{}\%} \FunctionTok{mutate}\NormalTok{(}\AttributeTok{mode =} \StringTok{"online{-}asynch"}\NormalTok{, }\AttributeTok{term =} \StringTok{"fall21"}\NormalTok{)}
\NormalTok{df\_subjects\_winter22 }\OtherTok{\textless{}{-}} \FunctionTok{read\_rds}\NormalTok{(winter22) }\CommentTok{\#use RDS file as it contains metadata}

\CommentTok{\#SAVE METADATA FROM WINTER, but no rows }
\NormalTok{df\_subjects }\OtherTok{\textless{}{-}}\NormalTok{ df\_subjects\_winter22 }\SpecialCharTok{\%\textgreater{}\%} \FunctionTok{filter}\NormalTok{(condition}\SpecialCharTok{==}\StringTok{\textquotesingle{}X\textquotesingle{}}\NormalTok{) }\SpecialCharTok{\%\textgreater{}\%} \FunctionTok{select}\NormalTok{(}
\NormalTok{  subject,condition,term,mode,}
\NormalTok{  gender,age,language, schoolyear, country,}
\NormalTok{  effort,difficulty,confidence,enjoyment,other,}
\NormalTok{  totaltime\_m,absolute\_score}
\NormalTok{)}

\CommentTok{\#reduce data collected using OLD webapp to useful columns}
\NormalTok{df\_subjects\_before }\OtherTok{\textless{}{-}} \FunctionTok{rbind}\NormalTok{(df\_subjects\_fall17, df\_subjects\_spring18, df\_subjects\_fall21) }\SpecialCharTok{\%\textgreater{}\%} 
  \CommentTok{\#rename and summarize some columns}
  \FunctionTok{mutate}\NormalTok{(}
    \AttributeTok{totaltime\_m =}\NormalTok{ totalTime }\SpecialCharTok{/} \DecValTok{1000} \SpecialCharTok{/} \DecValTok{60}\NormalTok{,  }
    \AttributeTok{absolute\_score =}\NormalTok{ triangular\_score,}
    \AttributeTok{language =}\NormalTok{ native\_language,}
    \AttributeTok{gender =}\NormalTok{ sex,}
    \AttributeTok{schoolyear =}\NormalTok{ year) }\SpecialCharTok{\%\textgreater{}\%} 
  \CommentTok{\#create placeholders for cols not collected until NEW webapp [for later rbind]}
  \FunctionTok{mutate}\NormalTok{(}
    \AttributeTok{effort =} \StringTok{"NULL"}\NormalTok{,}
    \AttributeTok{difficulty =} \StringTok{"NULL"}\NormalTok{,}
    \AttributeTok{confidence =} \StringTok{"NULL"}\NormalTok{,}
    \AttributeTok{enjoyment =} \StringTok{"NULL"}\NormalTok{,}
    \AttributeTok{other =} \StringTok{"NULL"}\NormalTok{,}
    \AttributeTok{disability =} \StringTok{"NULL"}
\NormalTok{  ) }\SpecialCharTok{\%\textgreater{}\%} 
  \CommentTok{\#select only columns we\textquotesingle{}ll be analyzing, discard others}
\NormalTok{  dplyr}\SpecialCharTok{::}\FunctionTok{select}\NormalTok{(subject, condition, term, mode, }
                \CommentTok{\#demographics}
\NormalTok{                gender, age, language, schoolyear, country,}
                \CommentTok{\#placeholder effort survey}
\NormalTok{                effort, difficulty, confidence, enjoyment, }
                \CommentTok{\#placeholder misc }
\NormalTok{                other, disability,}
                \CommentTok{\#response characteristics}
\NormalTok{                totaltime\_m, absolute\_score)}

\CommentTok{\#save \textquotesingle{}explanation\textquotesingle{} columns from winter22, which is actually a response to a free response item (Q16); was recorded with item\_level data in old webapp}
\NormalTok{df\_winter22\_q16 }\OtherTok{\textless{}{-}}\NormalTok{ df\_subjects\_winter22 }\SpecialCharTok{\%\textgreater{}\%} 
  \FunctionTok{select}\NormalTok{(subject, condition, term , mode, explanation) }\SpecialCharTok{\%\textgreater{}\%} 
  \FunctionTok{mutate}\NormalTok{(}
    \AttributeTok{q =} \DecValTok{16}\NormalTok{,}
    \AttributeTok{response =}\NormalTok{ explanation}
\NormalTok{  ) }\SpecialCharTok{\%\textgreater{}\%} \FunctionTok{select}\NormalTok{(}\SpecialCharTok{{-}}\NormalTok{explanation)}

\CommentTok{\#reduce data collected using NEW webapp to useful columns}
\NormalTok{df\_subjects\_winter22 }\OtherTok{\textless{}{-}}\NormalTok{ df\_subjects\_winter22 }\SpecialCharTok{\%\textgreater{}\%} 
  \FunctionTok{mutate}\NormalTok{(}\AttributeTok{score =}\NormalTok{ absolute\_score) }\SpecialCharTok{\%\textgreater{}\%} 
  \CommentTok{\#select only columns we\textquotesingle{}ll be analyzing, discard others}
\NormalTok{  dplyr}\SpecialCharTok{::}\FunctionTok{select}\NormalTok{( subject, condition, term, mode, }
                 \CommentTok{\#demographics}
\NormalTok{                 gender, age, language, schoolyear, country,}
                 \CommentTok{\#effort survey}
\NormalTok{                 effort, difficulty, confidence, enjoyment, }
                 \CommentTok{\#explanations}
\NormalTok{                 other,disability,}
                 \CommentTok{\#response characteristics}
\NormalTok{                 totaltime\_m, absolute\_score)}


\NormalTok{effort\_labels }\OtherTok{\textless{}{-}} \FunctionTok{c}\NormalTok{(}\StringTok{"I tried my best on each question"}\NormalTok{, }\StringTok{"I tried my best on most questions"}\NormalTok{)}

\CommentTok{\#combine dataframes from old and new webapps}
\NormalTok{df\_subjects }\OtherTok{\textless{}{-}} \FunctionTok{rbind}\NormalTok{(df\_subjects, df\_subjects\_winter22, df\_subjects\_before) }\SpecialCharTok{\%\textgreater{}\%} 
  \CommentTok{\#refactor factors}
  \FunctionTok{mutate}\NormalTok{ (}
    \AttributeTok{subject =} \FunctionTok{factor}\NormalTok{(subject),}
    \AttributeTok{condition =} \FunctionTok{factor}\NormalTok{(condition),}
    \AttributeTok{term =} \FunctionTok{factor}\NormalTok{(term),}
    \AttributeTok{mode =} \FunctionTok{factor}\NormalTok{(mode),}
    \AttributeTok{gender =} \FunctionTok{factor}\NormalTok{(gender),}
    \AttributeTok{schoolyear =} \FunctionTok{as.factor}\NormalTok{(schoolyear)}
\NormalTok{  )}

\CommentTok{\#FIX METADATA}
\CommentTok{\#Add metadata for columns that lost it [factors, for some reason!]}
\FunctionTok{var\_label}\NormalTok{(df\_subjects}\SpecialCharTok{$}\NormalTok{subject) }\OtherTok{\textless{}{-}} \StringTok{"ID of subject (randomly assigned in stimulus app)."}
\FunctionTok{var\_label}\NormalTok{(df\_subjects}\SpecialCharTok{$}\NormalTok{condition) }\OtherTok{\textless{}{-}} \StringTok{"ID indicates randomly assigned condition (111 {-}\textgreater{} control, 121 {-}\textgreater{} impasse)."}
\FunctionTok{var\_label}\NormalTok{(df\_subjects}\SpecialCharTok{$}\NormalTok{term) }\OtherTok{\textless{}{-}} \StringTok{"indicates if session was run with experimenter present or asynchronously"}
\FunctionTok{var\_label}\NormalTok{(df\_subjects}\SpecialCharTok{$}\NormalTok{mode) }\OtherTok{\textless{}{-}} \StringTok{"indicates mode in which the participant completed the study"}
\FunctionTok{var\_label}\NormalTok{(df\_subjects}\SpecialCharTok{$}\NormalTok{gender) }\OtherTok{\textless{}{-}} \StringTok{"What is your gender identity?"}
\FunctionTok{var\_label}\NormalTok{(df\_subjects}\SpecialCharTok{$}\NormalTok{schoolyear) }\OtherTok{\textless{}{-}} \StringTok{"What is your year in school?"}

\CommentTok{\#CLEANUP}
\FunctionTok{rm}\NormalTok{(df\_subjects\_fall17,df\_subjects\_fall21, df\_subjects\_spring18, df\_subjects\_winter22,df\_subjects\_before)}
\FunctionTok{rm}\NormalTok{(fall17,fall21,spring18,winter22)}
\end{Highlighting}
\end{Shaded}

\hypertarget{items}{%
\section{Items}\label{items}}

Next we import item-level data from each data collection period,
selecting only the columns relevant for analysis, and renaming columns
to be consistent across each file. The result is a single data frame
\texttt{df\_items} containing one row for each \emph{graph comprehension
task question} (qs=15) (across all periods). A second data frame
\texttt{df\_freeresponse} contains one row for each free response
strategy question (last question posed to participants in Winter2022)
Note that we \emph{do not} discard any \emph{response} data. Rather, we
\emph{do} discard several columns representing accuracy scores for
responses that were calculated in the stimulus engine. These scores were
calculated differently across collection periods, and so we discard them
and recalculate scores in the next analysis notebook. Original response
data are always preserved.

\begin{Shaded}
\begin{Highlighting}[]
\CommentTok{\#set datafiles}
\NormalTok{fall17 }\OtherTok{\textless{}{-}} \StringTok{"data/session{-}level/fall17\_sgc3a\_blocks.csv"}
\NormalTok{spring18 }\OtherTok{\textless{}{-}} \StringTok{"data/session{-}level/spring18\_sgc3a\_blocks.csv"}
\NormalTok{fall21 }\OtherTok{\textless{}{-}} \StringTok{"data/session{-}level/fall21\_sgc3a\_blocks.csv"}
\NormalTok{winter22 }\OtherTok{\textless{}{-}} \StringTok{"data/session{-}level/winter22\_sgc3a\_items.rds"}

\CommentTok{\#read datafiles, set mode and term}
\NormalTok{df\_items\_fall17 }\OtherTok{\textless{}{-}} \FunctionTok{read.csv}\NormalTok{(fall17) }\SpecialCharTok{\%\textgreater{}\%} \FunctionTok{mutate}\NormalTok{(}\AttributeTok{mode =} \StringTok{"lab{-}synch"}\NormalTok{, }\AttributeTok{term =} \StringTok{"fall17"}\NormalTok{)}
\NormalTok{df\_items\_spring18 }\OtherTok{\textless{}{-}} \FunctionTok{read.csv}\NormalTok{(spring18) }\SpecialCharTok{\%\textgreater{}\%} \FunctionTok{mutate}\NormalTok{(}\AttributeTok{mode =} \StringTok{"lab{-}synch"}\NormalTok{, }\AttributeTok{term =} \StringTok{"spring18"}\NormalTok{)}
\NormalTok{df\_items\_fall21 }\OtherTok{\textless{}{-}} \FunctionTok{read.csv}\NormalTok{(fall21) }\SpecialCharTok{\%\textgreater{}\%} \FunctionTok{mutate}\NormalTok{(}\AttributeTok{mode =} \StringTok{"online{-}asynch"}\NormalTok{, }\AttributeTok{term =} \StringTok{"fall21"}\NormalTok{)}
\NormalTok{df\_items\_winter22 }\OtherTok{\textless{}{-}} \FunctionTok{read\_rds}\NormalTok{(winter22) }\CommentTok{\#use RDS file as it contains metadata}

\CommentTok{\#get mapping being question \# and interval relation the question tests, that is encoded only in the winter22 data files}
\NormalTok{map\_relations }\OtherTok{\textless{}{-}}\NormalTok{ df\_items\_winter22 }\SpecialCharTok{\%\textgreater{}\%} \FunctionTok{group\_by}\NormalTok{(q) }\SpecialCharTok{\%\textgreater{}\%} \FunctionTok{select}\NormalTok{(q,relation) }\SpecialCharTok{\%\textgreater{}\%} \FunctionTok{unique}\NormalTok{()}


\CommentTok{\#SAVE METADATA FROM WINTER, but no rows }
\NormalTok{df\_items }\OtherTok{\textless{}{-}}\NormalTok{ df\_items\_winter22 }\SpecialCharTok{\%\textgreater{}\%} \FunctionTok{filter}\NormalTok{(condition}\SpecialCharTok{==}\StringTok{\textquotesingle{}X\textquotesingle{}}\NormalTok{) }\SpecialCharTok{\%\textgreater{}\%} \FunctionTok{select}\NormalTok{(}
\NormalTok{  subject,condition,term,mode,}
\NormalTok{  question, q, answer, correct, rt\_s}
\NormalTok{) }
  
\CommentTok{\#reduce data collected using old webapp}
\NormalTok{df\_items\_before }\OtherTok{\textless{}{-}} \FunctionTok{rbind}\NormalTok{(df\_items\_fall17, df\_items\_spring18, df\_items\_fall21) }\SpecialCharTok{\%\textgreater{}\%} 
  \FunctionTok{mutate}\NormalTok{(}\AttributeTok{rt\_s =}\NormalTok{ rt }\SpecialCharTok{/} \DecValTok{1000}\NormalTok{, }\AttributeTok{correct =} \FunctionTok{as.logical}\NormalTok{(correct)) }\SpecialCharTok{\%\textgreater{}\%} 
  \FunctionTok{select}\NormalTok{(subject, condition, term, mode, question, q, answer, correct, rt\_s) }
  
\CommentTok{\#reduce data collected using new webapp}
\NormalTok{df\_items\_winter22 }\OtherTok{\textless{}{-}}\NormalTok{ df\_items\_winter22 }\SpecialCharTok{\%\textgreater{}\%} 
  \FunctionTok{select}\NormalTok{(subject, condition, term, mode, question, q, answer, correct, rt\_s) }\SpecialCharTok{\%\textgreater{}\%} \CommentTok{\#unfactor before combine}
  \FunctionTok{mutate}\NormalTok{(}
    \AttributeTok{subject =} \FunctionTok{as.character}\NormalTok{(subject),}
    \AttributeTok{condition =} \FunctionTok{as.character}\NormalTok{(condition),}
    \AttributeTok{term =} \FunctionTok{as.character}\NormalTok{(term),}
    \AttributeTok{mode =} \FunctionTok{as.character}\NormalTok{(mode),}
    \AttributeTok{q =} \FunctionTok{as.integer}\NormalTok{(q),}
    \AttributeTok{correct =} \FunctionTok{as.logical}\NormalTok{(correct)}
\NormalTok{  )}

\CommentTok{\#combine dataframes from old and new webapps}
\NormalTok{df\_items }\OtherTok{\textless{}{-}} \FunctionTok{rbind}\NormalTok{(df\_items, df\_items\_winter22,df\_items\_before) }\SpecialCharTok{\%\textgreater{}\%} 
  \CommentTok{\#refactorize columns}
  \FunctionTok{mutate}\NormalTok{(}
    \AttributeTok{subject =} \FunctionTok{factor}\NormalTok{(subject),}
    \AttributeTok{condition =} \FunctionTok{factor}\NormalTok{(condition),}
    \AttributeTok{term =} \FunctionTok{factor}\NormalTok{(term),}
    \AttributeTok{mode =} \FunctionTok{factor}\NormalTok{(mode),}
    \AttributeTok{q =} \FunctionTok{as.integer}\NormalTok{(q)) }\SpecialCharTok{\%\textgreater{}\%} 
  \CommentTok{\#rename answer column to RESPONSE }
  \FunctionTok{rename}\NormalTok{(}\AttributeTok{response =}\NormalTok{ answer) }\SpecialCharTok{\%\textgreater{}\%} 
  \CommentTok{\#remove all commas and make as character string}
  \FunctionTok{mutate}\NormalTok{(}
    \AttributeTok{response =} \FunctionTok{str\_remove\_all}\NormalTok{(}\FunctionTok{as.character}\NormalTok{(response), }\StringTok{","}\NormalTok{),}
    \AttributeTok{num\_o =} \FunctionTok{str\_length}\NormalTok{(response)}
\NormalTok{  )}


\CommentTok{\#FIX METADATA}
\CommentTok{\#Add metadata for columns that lost it [factors, for some reason!]}
\FunctionTok{var\_label}\NormalTok{(df\_items}\SpecialCharTok{$}\NormalTok{subject) }\OtherTok{\textless{}{-}} \StringTok{"ID of subject (randomly assigned in stimulus app)."}
\FunctionTok{var\_label}\NormalTok{(df\_items}\SpecialCharTok{$}\NormalTok{condition) }\OtherTok{\textless{}{-}} \StringTok{"ID indicates randomly assigned condition (111 {-}\textgreater{} control, 121 {-}\textgreater{} impasse)."}
\FunctionTok{var\_label}\NormalTok{(df\_items}\SpecialCharTok{$}\NormalTok{term) }\OtherTok{\textless{}{-}} \StringTok{"indicates if session was run with experimenter present or asynchronously"}
\FunctionTok{var\_label}\NormalTok{(df\_items}\SpecialCharTok{$}\NormalTok{mode) }\OtherTok{\textless{}{-}} \StringTok{"indicates mode in which the participant completed the study"}
\FunctionTok{var\_label}\NormalTok{(df\_items}\SpecialCharTok{$}\NormalTok{q) }\OtherTok{\textless{}{-}} \StringTok{"Question Number (in order)"}
\FunctionTok{var\_label}\NormalTok{(df\_items}\SpecialCharTok{$}\NormalTok{correct) }\OtherTok{\textless{}{-}} \StringTok{"Is the response (strictly) correct? [dichotomous scoring]"}
\FunctionTok{var\_label}\NormalTok{(df\_items}\SpecialCharTok{$}\NormalTok{response) }\OtherTok{\textless{}{-}} \StringTok{"options (datapoints) selected by the subject"}
\FunctionTok{var\_label}\NormalTok{(df\_items}\SpecialCharTok{$}\NormalTok{num\_o) }\OtherTok{\textless{}{-}} \StringTok{"number of options selected by the subject"}

\CommentTok{\#HANDLE FREE RESPONSE QUESTION \#16 }
\CommentTok{\#save \textasciigrave{}free response\textasciigrave{} Q\#16 in its own dataframe}
\NormalTok{df\_freeresponse }\OtherTok{\textless{}{-}}\NormalTok{ df\_items }\SpecialCharTok{\%\textgreater{}\%} \FunctionTok{filter}\NormalTok{(q }\SpecialCharTok{==} \DecValTok{16}\NormalTok{) }\SpecialCharTok{\%\textgreater{}\%} \FunctionTok{select}\NormalTok{(}\SpecialCharTok{{-}}\NormalTok{question,}\SpecialCharTok{{-}}\NormalTok{correct,}\SpecialCharTok{{-}}\NormalTok{rt\_s,}\SpecialCharTok{{-}}\NormalTok{num\_o)}
\CommentTok{\#add data from wi22 [stored on subject data]}
\NormalTok{df\_freeresponse }\OtherTok{\textless{}{-}} \FunctionTok{rbind}\NormalTok{(df\_freeresponse, df\_winter22\_q16)}
\CommentTok{\#add question description}
\NormalTok{df\_freeresponse }\OtherTok{\textless{}{-}}\NormalTok{ df\_freeresponse }\SpecialCharTok{\%\textgreater{}\%} \FunctionTok{mutate}\NormalTok{(}
    \AttributeTok{question =} \StringTok{"Please describe how to determine what event(s) start at 12pm?"}\NormalTok{,}
    \AttributeTok{response =} \FunctionTok{as.character}\NormalTok{(response) }\CommentTok{\#doesn\textquotesingle{}t need to be factor}
\NormalTok{  ) }
\CommentTok{\#remove \textquotesingle{}free response\textquotesingle{} Q\#16 from df\_items}
\NormalTok{df\_items }\OtherTok{\textless{}{-}}\NormalTok{ df\_items }\SpecialCharTok{\%\textgreater{}\%} \FunctionTok{filter}\NormalTok{ (q }\SpecialCharTok{!=} \DecValTok{16}\NormalTok{)}

\CommentTok{\#CLEANUP}
\FunctionTok{rm}\NormalTok{(df\_items\_fall17,df\_items\_fall21, df\_items\_spring18, df\_items\_winter22, df\_items\_before, df\_winter22\_q16)}
\FunctionTok{rm}\NormalTok{(fall17,fall21,spring18,winter22, map\_relations)}
\end{Highlighting}
\end{Shaded}

\hypertarget{validation}{%
\subsection{Validation}\label{validation}}

Next, we validate that we have the complete number of item-level records
based on the number of subject-level records

\begin{Shaded}
\begin{Highlighting}[]
\CommentTok{\#the number of items should be equal to 15 x the number of subjects}
\FunctionTok{nrow}\NormalTok{(df\_items) }\SpecialCharTok{==} \DecValTok{15}\SpecialCharTok{*} \FunctionTok{nrow}\NormalTok{(df\_subjects) }\CommentTok{\#TRUE}
\end{Highlighting}
\end{Shaded}

\begin{verbatim}
[1] TRUE
\end{verbatim}

\begin{Shaded}
\begin{Highlighting}[]
\CommentTok{\#each subject should have 15 items}
\NormalTok{df\_items }\SpecialCharTok{\%\textgreater{}\%} \FunctionTok{group\_by}\NormalTok{(subject) }\SpecialCharTok{\%\textgreater{}\%} \FunctionTok{summarise}\NormalTok{(}\AttributeTok{n =} \FunctionTok{n}\NormalTok{()) }\SpecialCharTok{\%\textgreater{}\%} \FunctionTok{filter}\NormalTok{(n }\SpecialCharTok{!=} \DecValTok{15}\NormalTok{) }\SpecialCharTok{\%\textgreater{}\%} \FunctionTok{nrow}\NormalTok{() }\SpecialCharTok{==} \DecValTok{0}
\end{Highlighting}
\end{Shaded}

\begin{verbatim}
[1] TRUE
\end{verbatim}

\hypertarget{export}{%
\chapter{EXPORT}\label{export}}

Finally, we export the (session-harmonized) data for analysis, as CSVs,
and .RDS (includes metadata)

\begin{Shaded}
\begin{Highlighting}[]
\CommentTok{\#SAVE FILES}
\FunctionTok{write.csv}\NormalTok{(df\_subjects,}\StringTok{"data/sgc3a\_participants.csv"}\NormalTok{, }\AttributeTok{row.names =} \ConstantTok{FALSE}\NormalTok{)}
\FunctionTok{write.csv}\NormalTok{(df\_items,}\StringTok{"data/sgc3a\_items.csv"}\NormalTok{, }\AttributeTok{row.names =} \ConstantTok{FALSE}\NormalTok{)}
\FunctionTok{write.csv}\NormalTok{(df\_freeresponse,}\StringTok{"data/sgc3a\_items.csv"}\NormalTok{, }\AttributeTok{row.names =} \ConstantTok{FALSE}\NormalTok{)}

\CommentTok{\#SAVE R Data Structures }
\CommentTok{\#export R DATA STRUCTURES (include codebook metadata)}
\NormalTok{rio}\SpecialCharTok{::}\FunctionTok{export}\NormalTok{(df\_subjects, }\StringTok{"data/sgc3a\_participants.rds"}\NormalTok{) }\CommentTok{\# to R data structure file}
\NormalTok{rio}\SpecialCharTok{::}\FunctionTok{export}\NormalTok{(df\_items, }\StringTok{"data/sgc3a\_items.rds"}\NormalTok{) }\CommentTok{\# to R data structure file}
\end{Highlighting}
\end{Shaded}

\hypertarget{resources}{%
\chapter{RESOURCES}\label{resources}}

\begin{Shaded}
\begin{Highlighting}[]
\FunctionTok{sessionInfo}\NormalTok{()}
\end{Highlighting}
\end{Shaded}

\begin{verbatim}
R version 4.0.2 (2020-06-22)
Platform: x86_64-apple-darwin17.0 (64-bit)
Running under: macOS  10.16

Matrix products: default
BLAS:   /Library/Frameworks/R.framework/Versions/4.0/Resources/lib/libRblas.dylib
LAPACK: /Library/Frameworks/R.framework/Versions/4.0/Resources/lib/libRlapack.dylib

locale:
[1] en_US.UTF-8/en_US.UTF-8/en_US.UTF-8/C/en_US.UTF-8/en_US.UTF-8

attached base packages:
[1] stats     graphics  grDevices utils     datasets  methods   base     

other attached packages:
 [1] codebook_0.9.2  forcats_0.5.0   stringr_1.4.0   dplyr_1.0.2    
 [5] purrr_0.3.4     readr_1.4.0     tidyr_1.1.2     tibble_3.1.2   
 [9] ggplot2_3.3.5   tidyverse_1.3.0

loaded via a namespace (and not attached):
 [1] Rcpp_1.0.5        lubridate_1.7.9   assertthat_0.2.1  digest_0.6.27    
 [5] utf8_1.2.1        R6_2.5.0          cellranger_1.1.0  backports_1.2.1  
 [9] reprex_0.3.0      labelled_2.8.0    evaluate_0.14     httr_1.4.2       
[13] pillar_1.6.1      rlang_0.4.11      curl_4.3          readxl_1.3.1     
[17] rstudioapi_0.13   data.table_1.13.2 blob_1.2.1        rmarkdown_2.11   
[21] foreign_0.8-80    munsell_0.5.0     broom_0.7.12      compiler_4.0.2   
[25] modelr_0.1.8      xfun_0.29         pkgconfig_2.0.3   htmltools_0.5.2  
[29] tidyselect_1.1.0  rio_0.5.16        fansi_0.5.0       crayon_1.4.1     
[33] dbplyr_1.4.4      withr_2.4.2       grid_4.0.2        jsonlite_1.7.1   
[37] gtable_0.3.0      lifecycle_1.0.0   DBI_1.1.0         magrittr_2.0.1   
[41] scales_1.1.1      zip_2.1.1         cli_3.3.0         stringi_1.7.3    
[45] fs_1.5.0          xml2_1.3.2        ellipsis_0.3.2    generics_0.0.2   
[49] vctrs_0.3.8       openxlsx_4.2.3    tools_4.0.2       glue_1.6.2       
[53] hms_0.5.3         fastmap_1.1.0     yaml_2.2.1        colorspace_2.0-2 
[57] rvest_0.3.6       knitr_1.37        haven_2.3.1      
\end{verbatim}

\hypertarget{response-rescoring}{%
\chapter{Response Rescoring}\label{response-rescoring}}

\newpage

\emph{The purpose of this notebook is to re-score the response accuracy
data for the SGC\_3A study. This is required because the question type
on the graph comprehension task used a `Multiple Answer Multiple Choice'
design (MCMA).} Warning: this notebook takes several minutes to execute.

\begin{longtable}[]{@{}ll@{}}
\toprule()
Pre-Requisite & Followed By \\
\midrule()
\endhead
1\_sgc3a\_harmonize.qmd & 3\_sgc3A\_descriptives.qmd \\
\bottomrule()
\end{longtable}

\begin{Shaded}
\begin{Highlighting}[]
\CommentTok{\#read datafiles, set mode and term}
\NormalTok{df\_items }\OtherTok{\textless{}{-}} \FunctionTok{read\_rds}\NormalTok{(}\StringTok{\textquotesingle{}data/sgc3a\_items.rds\textquotesingle{}}\NormalTok{)}
\end{Highlighting}
\end{Shaded}

\hypertarget{introduction-1}{%
\chapter{INTRODUCTION}\label{introduction-1}}

The \emph{graph comprehension task} of study SGC 3A presents readers
with a graph, a question, and a series of checkboxes. Participants are
instructed to use the graph to answer the question, and respond by
selecting all the checkboxes that apply, where each checkbox corresponds
to a datapoint in the graph.

\begin{figure}

{\centering \includegraphics{analysis/SGC3A/static/img/sample_graphComprehensionTask.png}

}

\caption{\textbf{Figure 1. Sample Graph Comprehension (Question \# 6)}}

\end{figure}

In the psychological and education literatures on Tests \& Measures, the
format of this type of question is referred to as \emph{Multiple Choice
Multiple Answer} (MCMA) or \emph{Multiple Answer Multiple Choice}
(MAMC).\\
It has a number of properties that make it different from traditional
\emph{Single Answer Multiple Choice} (SAMC) questions, where the
respondent marks a single response from a number of options In
particular, there are a number of very different ways that MAMC
questions can be \emph{scored}.

Traditionally in SAMC questions, one point is given for selecting the
option designated as correct, and zero points given for marking any of
the alternative (i.e.~distractor) options. Responses on MAMC questions,
however might be partially correct (\(i\)), while responses on other
answer options within the same item might be incorrect (\(n – i\)). In
MAMC, it is not obvious how to allocate points when the respondent marks
a true-correct option (i.e.~options that \emph{should} be selected), as
well as one or more false-correct options (i.e.~options that
\emph{should not} be selected). Should partial credit be awarded? If so,
are options that respondents false-selected and false-unselected items
equally penalized?

Schmidt et. al (2021) performed a systematic literature review of
publications proposing MAMC (or equivalent) scoring schemes, ultimately
synthesizing over 80sources into 27 distinct scoring approaches. Upon
reviewing the benefits of tradeoffs of each approach, for this study we
choose utilize two of the schemes: \textbf{dichotomous scoring}
(Schmidt. et. al scheme \#1), and \textbf{partial scoring}
\([-1/q,0, +1/p]\) (Schmidt et. al.~scheme \#26), as well as a scaled
\textbf{discriminant score} that leverages partial scoring to
discriminate between strategy-specific patterns of response.

\hypertarget{response-encoding}{%
\section{Response Encoding}\label{response-encoding}}

First, we note that the question type evaluated by Schmidt et.
al.~(2021) is referred to as \emph{Multiple True-False} (MTF), a variant
of MAMC where respondents are presented with a question (stem) and
series of response options with True/False (e.g.~radio buttons) for
each. Depending on the implementation of the underlying instrument, it
may or may not be possible for respondents to \emph{not respond} to a
particular option (i.e.~leave the item `blank'). Although MTF questions
have a different underlying implementation (and potentially different
psychometric properties) they are identical in their mathematical
properties; that is, responses to a MAMC question of `select all that
apply' can be coded as a series of T/F responses to each response option

\begin{figure}

{\centering \includegraphics{analysis/SGC3A/static/img/MAMC-MTF.png}

}

\caption{\textbf{Figure 2. SAMC (vs) MAMC (vs) MTF}}

\end{figure}

In this example (Figure 2), we see an example of a question with four
response options (\(n=4\)) in each question type. In the \textbf{SAMC}
approach (at left), there are four possible responses, given explicitly
by the response options (respondent can select only one)
\((\text{number of possible responses} = n)\). With only four possible
responses, we cannot entirely discriminate between all combinations of
the underlying response variants we might be interested in, and must
always choose an `ideal subset' of possible distractors to present as
response options. In the MAMC (middle) and MTF (at right), the
\emph{same number of response options} (\(n=4\)) yield a much greater
number \((\text{number of possible responses} = 2^{n})\). We can also
see the equivalence between a MAMC and MTF format questions with the
same response options. Options the respondent \emph{selects} in MAMC are
can be coded as T, and options they leave \emph{unselected} can be coded
as F. Thus, for response options (ABCD), a response of {[}AB{]} can be
encoded as {[}TTFF{]}.

\begin{tcolorbox}[standard jigsaw, leftrule=.75mm, bottomrule=.15mm, colframe=quarto-callout-color-frame, rightrule=.15mm, toprule=.15mm, left=2mm, arc=.35mm, opacityback=0, colback=white]
In our analysis, we will transform the MAMC response string recorded for
the participant (given in column \texttt{response}), to an MTF encoding.
\end{tcolorbox}

\hypertarget{scoring-schemes}{%
\section{Scoring Schemes}\label{scoring-schemes}}

In the sections that follow, we use the terminology:

\textbf{Properties of the Stimulus-Question}

\[
\begin{align}
n &= \text{number of response options} \\  
  &= p + q \\ 
  p &= \text{number of true-select options (i.e. should be selected)} \\
  q &= \text{number of true-unselect options (i.e. should not be selected)} 
\end{align}
\]

\textbf{Properties of the Subject's Response}

\hypertarget{references}{%
\chapter*{References}\label{references}}
\addcontentsline{toc}{chapter}{References}

\hypertarget{refs}{}
\begin{CSLReferences}{0}{0}
\end{CSLReferences}



\end{document}
