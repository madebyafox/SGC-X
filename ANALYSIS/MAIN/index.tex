% Options for packages loaded elsewhere
\PassOptionsToPackage{unicode}{hyperref}
\PassOptionsToPackage{hyphens}{url}
\PassOptionsToPackage{dvipsnames,svgnames,x11names}{xcolor}
%
\documentclass[
  letterpaper,
  DIV=11,
  numbers=noendperiod]{scrreprt}
\usepackage{amsmath,amssymb}
\usepackage{lmodern}
\usepackage{iftex}
\ifPDFTeX
  \usepackage[T1]{fontenc}
  \usepackage[utf8]{inputenc}
  \usepackage{textcomp} % provide euro and other symbols
\else % if luatex or xetex
  \usepackage{unicode-math}
  \defaultfontfeatures{Scale=MatchLowercase}
  \defaultfontfeatures[\rmfamily]{Ligatures=TeX,Scale=1}
\fi
% Use upquote if available, for straight quotes in verbatim environments
\IfFileExists{upquote.sty}{\usepackage{upquote}}{}
\IfFileExists{microtype.sty}{% use microtype if available
  \usepackage[]{microtype}
  \UseMicrotypeSet[protrusion]{basicmath} % disable protrusion for tt fonts
}{}
\makeatletter
\@ifundefined{KOMAClassName}{% if non-KOMA class
  \IfFileExists{parskip.sty}{%
    \usepackage{parskip}
  }{% else
    \setlength{\parindent}{0pt}
    \setlength{\parskip}{6pt plus 2pt minus 1pt}}
}{% if KOMA class
  \KOMAoptions{parskip=half}}
\makeatother
\usepackage{xcolor}
\setlength{\emergencystretch}{3em} % prevent overfull lines
\setcounter{secnumdepth}{5}
% Make \paragraph and \subparagraph free-standing
\ifx\paragraph\undefined\else
  \let\oldparagraph\paragraph
  \renewcommand{\paragraph}[1]{\oldparagraph{#1}\mbox{}}
\fi
\ifx\subparagraph\undefined\else
  \let\oldsubparagraph\subparagraph
  \renewcommand{\subparagraph}[1]{\oldsubparagraph{#1}\mbox{}}
\fi

\usepackage{color}
\usepackage{fancyvrb}
\newcommand{\VerbBar}{|}
\newcommand{\VERB}{\Verb[commandchars=\\\{\}]}
\DefineVerbatimEnvironment{Highlighting}{Verbatim}{commandchars=\\\{\}}
% Add ',fontsize=\small' for more characters per line
\usepackage{framed}
\definecolor{shadecolor}{RGB}{241,243,245}
\newenvironment{Shaded}{\begin{snugshade}}{\end{snugshade}}
\newcommand{\AlertTok}[1]{\textcolor[rgb]{0.68,0.00,0.00}{#1}}
\newcommand{\AnnotationTok}[1]{\textcolor[rgb]{0.37,0.37,0.37}{#1}}
\newcommand{\AttributeTok}[1]{\textcolor[rgb]{0.40,0.45,0.13}{#1}}
\newcommand{\BaseNTok}[1]{\textcolor[rgb]{0.68,0.00,0.00}{#1}}
\newcommand{\BuiltInTok}[1]{\textcolor[rgb]{0.00,0.23,0.31}{#1}}
\newcommand{\CharTok}[1]{\textcolor[rgb]{0.13,0.47,0.30}{#1}}
\newcommand{\CommentTok}[1]{\textcolor[rgb]{0.37,0.37,0.37}{#1}}
\newcommand{\CommentVarTok}[1]{\textcolor[rgb]{0.37,0.37,0.37}{\textit{#1}}}
\newcommand{\ConstantTok}[1]{\textcolor[rgb]{0.56,0.35,0.01}{#1}}
\newcommand{\ControlFlowTok}[1]{\textcolor[rgb]{0.00,0.23,0.31}{#1}}
\newcommand{\DataTypeTok}[1]{\textcolor[rgb]{0.68,0.00,0.00}{#1}}
\newcommand{\DecValTok}[1]{\textcolor[rgb]{0.68,0.00,0.00}{#1}}
\newcommand{\DocumentationTok}[1]{\textcolor[rgb]{0.37,0.37,0.37}{\textit{#1}}}
\newcommand{\ErrorTok}[1]{\textcolor[rgb]{0.68,0.00,0.00}{#1}}
\newcommand{\ExtensionTok}[1]{\textcolor[rgb]{0.00,0.23,0.31}{#1}}
\newcommand{\FloatTok}[1]{\textcolor[rgb]{0.68,0.00,0.00}{#1}}
\newcommand{\FunctionTok}[1]{\textcolor[rgb]{0.28,0.35,0.67}{#1}}
\newcommand{\ImportTok}[1]{\textcolor[rgb]{0.00,0.46,0.62}{#1}}
\newcommand{\InformationTok}[1]{\textcolor[rgb]{0.37,0.37,0.37}{#1}}
\newcommand{\KeywordTok}[1]{\textcolor[rgb]{0.00,0.23,0.31}{#1}}
\newcommand{\NormalTok}[1]{\textcolor[rgb]{0.00,0.23,0.31}{#1}}
\newcommand{\OperatorTok}[1]{\textcolor[rgb]{0.37,0.37,0.37}{#1}}
\newcommand{\OtherTok}[1]{\textcolor[rgb]{0.00,0.23,0.31}{#1}}
\newcommand{\PreprocessorTok}[1]{\textcolor[rgb]{0.68,0.00,0.00}{#1}}
\newcommand{\RegionMarkerTok}[1]{\textcolor[rgb]{0.00,0.23,0.31}{#1}}
\newcommand{\SpecialCharTok}[1]{\textcolor[rgb]{0.37,0.37,0.37}{#1}}
\newcommand{\SpecialStringTok}[1]{\textcolor[rgb]{0.13,0.47,0.30}{#1}}
\newcommand{\StringTok}[1]{\textcolor[rgb]{0.13,0.47,0.30}{#1}}
\newcommand{\VariableTok}[1]{\textcolor[rgb]{0.07,0.07,0.07}{#1}}
\newcommand{\VerbatimStringTok}[1]{\textcolor[rgb]{0.13,0.47,0.30}{#1}}
\newcommand{\WarningTok}[1]{\textcolor[rgb]{0.37,0.37,0.37}{\textit{#1}}}

\providecommand{\tightlist}{%
  \setlength{\itemsep}{0pt}\setlength{\parskip}{0pt}}\usepackage{longtable,booktabs,array}
\usepackage{calc} % for calculating minipage widths
% Correct order of tables after \paragraph or \subparagraph
\usepackage{etoolbox}
\makeatletter
\patchcmd\longtable{\par}{\if@noskipsec\mbox{}\fi\par}{}{}
\makeatother
% Allow footnotes in longtable head/foot
\IfFileExists{footnotehyper.sty}{\usepackage{footnotehyper}}{\usepackage{footnote}}
\makesavenoteenv{longtable}
\usepackage{graphicx}
\makeatletter
\def\maxwidth{\ifdim\Gin@nat@width>\linewidth\linewidth\else\Gin@nat@width\fi}
\def\maxheight{\ifdim\Gin@nat@height>\textheight\textheight\else\Gin@nat@height\fi}
\makeatother
% Scale images if necessary, so that they will not overflow the page
% margins by default, and it is still possible to overwrite the defaults
% using explicit options in \includegraphics[width, height, ...]{}
\setkeys{Gin}{width=\maxwidth,height=\maxheight,keepaspectratio}
% Set default figure placement to htbp
\makeatletter
\def\fps@figure{htbp}
\makeatother
\newlength{\cslhangindent}
\setlength{\cslhangindent}{1.5em}
\newlength{\csllabelwidth}
\setlength{\csllabelwidth}{3em}
\newlength{\cslentryspacingunit} % times entry-spacing
\setlength{\cslentryspacingunit}{\parskip}
\newenvironment{CSLReferences}[2] % #1 hanging-ident, #2 entry spacing
 {% don't indent paragraphs
  \setlength{\parindent}{0pt}
  % turn on hanging indent if param 1 is 1
  \ifodd #1
  \let\oldpar\par
  \def\par{\hangindent=\cslhangindent\oldpar}
  \fi
  % set entry spacing
  \setlength{\parskip}{#2\cslentryspacingunit}
 }%
 {}
\usepackage{calc}
\newcommand{\CSLBlock}[1]{#1\hfill\break}
\newcommand{\CSLLeftMargin}[1]{\parbox[t]{\csllabelwidth}{#1}}
\newcommand{\CSLRightInline}[1]{\parbox[t]{\linewidth - \csllabelwidth}{#1}\break}
\newcommand{\CSLIndent}[1]{\hspace{\cslhangindent}#1}

\usepackage{booktabs}
\usepackage{longtable}
\usepackage{array}
\usepackage{multirow}
\usepackage{wrapfig}
\usepackage{float}
\usepackage{colortbl}
\usepackage{pdflscape}
\usepackage{tabu}
\usepackage{threeparttable}
\usepackage{threeparttablex}
\usepackage[normalem]{ulem}
\usepackage{makecell}
\usepackage{xcolor}
\KOMAoption{captions}{tableheading}
\makeatletter
\@ifpackageloaded{tcolorbox}{}{\usepackage[many]{tcolorbox}}
\@ifpackageloaded{fontawesome5}{}{\usepackage{fontawesome5}}
\definecolor{quarto-callout-color}{HTML}{909090}
\definecolor{quarto-callout-note-color}{HTML}{0758E5}
\definecolor{quarto-callout-important-color}{HTML}{CC1914}
\definecolor{quarto-callout-warning-color}{HTML}{EB9113}
\definecolor{quarto-callout-tip-color}{HTML}{00A047}
\definecolor{quarto-callout-caution-color}{HTML}{FC5300}
\definecolor{quarto-callout-color-frame}{HTML}{acacac}
\definecolor{quarto-callout-note-color-frame}{HTML}{4582ec}
\definecolor{quarto-callout-important-color-frame}{HTML}{d9534f}
\definecolor{quarto-callout-warning-color-frame}{HTML}{f0ad4e}
\definecolor{quarto-callout-tip-color-frame}{HTML}{02b875}
\definecolor{quarto-callout-caution-color-frame}{HTML}{fd7e14}
\makeatother
\makeatletter
\makeatother
\makeatletter
\@ifpackageloaded{caption}{}{\usepackage{caption}}
\AtBeginDocument{%
\ifdefined\contentsname
  \renewcommand*\contentsname{Table of contents}
\else
  \newcommand\contentsname{Table of contents}
\fi
\ifdefined\listfigurename
  \renewcommand*\listfigurename{List of Figures}
\else
  \newcommand\listfigurename{List of Figures}
\fi
\ifdefined\listtablename
  \renewcommand*\listtablename{List of Tables}
\else
  \newcommand\listtablename{List of Tables}
\fi
\ifdefined\figurename
  \renewcommand*\figurename{Figure}
\else
  \newcommand\figurename{Figure}
\fi
\ifdefined\tablename
  \renewcommand*\tablename{Table}
\else
  \newcommand\tablename{Table}
\fi
}
\@ifpackageloaded{float}{}{\usepackage{float}}
\floatstyle{ruled}
\@ifundefined{c@chapter}{\newfloat{codelisting}{h}{lop}}{\newfloat{codelisting}{h}{lop}[chapter]}
\floatname{codelisting}{Listing}
\newcommand*\listoflistings{\listof{codelisting}{List of Listings}}
\makeatother
\makeatletter
\@ifpackageloaded{caption}{}{\usepackage{caption}}
\@ifpackageloaded{subcaption}{}{\usepackage{subcaption}}
\makeatother
\makeatletter
\@ifpackageloaded{tcolorbox}{}{\usepackage[many]{tcolorbox}}
\makeatother
\makeatletter
\@ifundefined{shadecolor}{\definecolor{shadecolor}{rgb}{.97, .97, .97}}
\makeatother
\makeatletter
\makeatother
\ifLuaTeX
  \usepackage{selnolig}  % disable illegal ligatures
\fi
\IfFileExists{bookmark.sty}{\usepackage{bookmark}}{\usepackage{hyperref}}
\IfFileExists{xurl.sty}{\usepackage{xurl}}{} % add URL line breaks if available
\urlstyle{same} % disable monospaced font for URLs
\hypersetup{
  pdftitle={SGC-X},
  pdfauthor={Amy Rae Fox},
  colorlinks=true,
  linkcolor={blue},
  filecolor={Maroon},
  citecolor={Blue},
  urlcolor={Blue},
  pdfcreator={LaTeX via pandoc}}

\title{SGC-X}
\usepackage{etoolbox}
\makeatletter
\providecommand{\subtitle}[1]{% add subtitle to \maketitle
  \apptocmd{\@title}{\par {\large #1 \par}}{}{}
}
\makeatother
\subtitle{Study SGC5A \textbar{} 2 Response Scoring}
\author{Amy Rae Fox}
\date{}

\begin{document}
\maketitle

\ifdefined\Shaded\renewenvironment{Shaded}{\begin{tcolorbox}[enhanced, sharp corners, boxrule=0pt, interior hidden, breakable, borderline west={3pt}{0pt}{shadecolor}, frame hidden]}{\end{tcolorbox}}\fi

\renewcommand*\contentsname{Table of contents}
{
\hypersetup{linkcolor=}
\setcounter{tocdepth}{2}
\tableofcontents
}
\part{MAIN}

\hypertarget{sgc-3-the-insight-hypothesis}{%
\section*{SGC 3 --- The Insight
Hypothesis}\label{sgc-3-the-insight-hypothesis}}
\addcontentsline{toc}{section}{SGC 3 --- The Insight Hypothesis}

\hypertarget{SGC3}{}

\hypertarget{sgc-4-the-graphical-framework}{%
\section*{SGC 4 --- The Graphical
Framework}\label{sgc-4-the-graphical-framework}}
\addcontentsline{toc}{section}{SGC 4 --- The Graphical Framework}

\hypertarget{SGC4}{}

\hypertarget{sgc-5-the-effectiveness-of-interaction}{%
\section*{SGC 5 --- The Effectiveness of
Interaction}\label{sgc-5-the-effectiveness-of-interaction}}
\addcontentsline{toc}{section}{SGC 5 --- The Effectiveness of
Interaction}

\hypertarget{SGC5}{}

\newpage

\hypertarget{sec-scoring}{%
\chapter*{Scoring Strategy}\label{sec-scoring}}
\addcontentsline{toc}{chapter}{Scoring Strategy}

\emph{The purpose of this notebook is to describe the strategy for
assigning a score ( a measure of accuracy) to response data for the SGC
studies. This is required because the question type on the graph
comprehension task used a `Multiple Response' (MR) question design.
Here, we evaluate different approaches to scoring multiple response
questions, and transform raw item responses (e.g.~boxes ABC are checked)
to a measure of response accuracy. (Warning: this notebook takes several
minutes to execute.)}

\begin{Shaded}
\begin{Highlighting}[]
\FunctionTok{options}\NormalTok{(}\AttributeTok{scipen=}\DecValTok{1}\NormalTok{, }\AttributeTok{digits=}\DecValTok{3}\NormalTok{)}

\FunctionTok{library}\NormalTok{(kableExtra) }\CommentTok{\#printing tables }
\FunctionTok{library}\NormalTok{(ggformula) }\CommentTok{\#quick graphs}
\FunctionTok{library}\NormalTok{(pbapply) }\CommentTok{\#progress bar and time estimate for *apply fns}
\FunctionTok{library}\NormalTok{(Hmisc) }\CommentTok{\# \%nin\% operator}
\FunctionTok{library}\NormalTok{(tidyverse) }\CommentTok{\#ALL THE THINGS}
\end{Highlighting}
\end{Shaded}

\hypertarget{multiple-response-scoring}{%
\section*{MULTIPLE RESPONSE SCORING}\label{multiple-response-scoring}}
\addcontentsline{toc}{section}{MULTIPLE RESPONSE SCORING}

The \emph{graph comprehension task} of the SGC studies presents readers
with a graph, a question, and a series of checkboxes. Participants are
instructed to use the graph to answer the question, and respond by
selecting all the checkboxes that apply, where each checkbox corresponds
to a datapoint in the graph.

\begin{figure}

{\centering \includegraphics{analysis/utils/img/sample_graphComprehensionTask.png}

}

\caption{\textbf{Figure 1. Sample Graph Comprehension (Question \# 6)}}

\end{figure}

In the psychology and education literatures on Tests \& Measures, the
format of this type of question is referred to as Multiple Response
(MR), (also: Multiple Choice Multiple Answer (MCMA) and Multiple Answer
Multiple Choice (MAMC)). It has a number of properties that make it
different from traditional Single Answer Multiple Choice (SAMC)
questions, where the respondent marks a single response from a number of
options. In particular, there are a number of very different ways that
MAMC questions can be \emph{scored}.

In tranditional SAMC format questions, one point is given for selecting
the option designated as correct, and zero points given for marking any
of the alternative (i.e.~distractor) options. Individual response
options on MAMC questions, however might be partially correct (\(i\)),
while responses on other answer options within the same item might be
incorrect (\(n – i\)). In MR, it is not obvious how to allocate points
when the respondent marks a true-correct option (i.e.~options that
\emph{should} be selected, denoted \(p\)), as well as one or more
false-correct options (i.e.~options that \emph{should not} be selected,
denoted \(q\)). Should partial credit be awarded? If so, are options
that respondents false-selected and false-unselected items equally
penalized?

Schmidt et al. (2021) performed a systematic literature review of
publications proposing MAMC (or equivalent) scoring schemes, ultimately
synthesizing over 80 sources into 27 distinct scoring approaches. Upon
reviewing the benefits of trade-offs of each approach, for this study we
choose utilize two of the schemes: \textbf{dichotomous scoring} (
Schmidt et al. (2021) scheme \#1), and \textbf{partial scoring}
\([-1/q,0, +1/p]\) ( Schmidt et al. (2021) scheme \#26), as well as a
scaled \textbf{discriminant score} that leverages partial scoring to
discriminate between strategy-specific patterns of response.

\hypertarget{response-encoding}{%
\subsection*{Response Encoding}\label{response-encoding}}
\addcontentsline{toc}{subsection}{Response Encoding}

First, we note that the question type evaluated by Schmidt et al. (2021)
is referred to as \emph{Multiple True-False} (MTF), a variant of MAMC
where respondents are presented with a question (stem) and series of
response options with True/False (e.g.~radio buttons) for each.
Depending on the implementation of the underlying instrument, it may or
may not be possible for respondents to \emph{not respond} to a
particular option (i.e.~leave the item `blank'). Although MTF questions
have a different underlying implementation (and potentially different
psychometric properties) they are identical in their mathematical
properties; that is, responses to a MAMC question of `select all that
apply' can be coded as a series of T/F responses to each response option

\begin{figure}

{\centering \includegraphics{analysis/utils/img/MAMC-MTF.png}

}

\caption{\label{fig-ItemTypes}\textbf{Figure 2. SAMC (vs) MAMC (vs)
MTF}}

\end{figure}

In this example (Figure~\ref{fig-ItemTypes}), we see an example of a
question with four response options (\(n=4\)) in each question type. In
the \textbf{SAMC} approach (at left), there are four possible responses,
given explicitly by the response options (respondent can select only
one) \((\text{number of possible responses} = n)\). With only four
possible responses, we cannot entirely discriminate between all
combinations of the underlying response variants we might be interested
in, and must always choose an `ideal subset' of possible distractors to
present as response options. In the MAMC (middle) and MTF (at right),
the \emph{same number of response options} (\(n=4\)) yield a much
greater number \((\text{number of possible responses} = 2^{n})\). We can
also see the equivalence between a MAMC and MTF format questions with
the same response options. Options the respondent \emph{selects} in MAMC
are can be coded as T, and options they leave \emph{unselected} can be
coded as F. Thus, for response options (ABCD), a response of {[}AB{]}
can also be encoded as {[}TTFF{]}.

\hypertarget{scoring-schemes}{%
\subsection*{Scoring Schemes}\label{scoring-schemes}}
\addcontentsline{toc}{subsection}{Scoring Schemes}

In the sections that follow, we use the terminology:

\textbf{Properties of the Stimulus-Question}

\begin{align}
n &= \text{number of response options} \\  
  &= p + q \\ 
  p &= \text{number of true-select options (i.e. should be selected)} \\
  q &= \text{number of true-unselect options (i.e. should not be selected)} 
\end{align}

\textbf{Properties of the Subject's Response}

\begin{align} 
i &= \text{number of options in correct state}, (0 ≤ i ≤ n) \\ 
f &= \text{resulting score} 
\end{align}

\hypertarget{sec-absolute-scoring}{%
\subsubsection*{Dichotomous Scoring}\label{sec-absolute-scoring}}
\addcontentsline{toc}{subsubsection}{Dichotomous Scoring}

\textbf{Dichotomous Scoring} is the strictest scoring scheme, where a
response only receives points if it is \emph{exactly} correct, meaning
the respondent includes \emph{only correct-select} options, and does
select any additional (i.e.~incorrect-select) options that should not be
selected. This is also known as \emph{all or nothing scoring}, and
importantly, it ignores any partial knowledge that a participant may be
expressing through their choice of options. They may select some but not
all of the correct-select options, and one or more but not all of the
correct-unselect items, but receive the same score as a respondent
selects none of the correct-select options, or all of the
correct-unselect options. In this sense, dichotomous scoring tells us
\emph{only} about perfect knowledge, and ignores any indication of
partial knowledge the respondent may be indicating through their
selection of response options.

\textbf{In Dichotomous Scoring}

\begin{itemize}
\tightlist
\item
  score for the question is either 0 or 1
\item
  full credit is only given if all responses are correct; otherwise no
  credit
\item
  does not account for \emph{partial knowledge}. - with increasing
  number of response options, scoring becomes stricter as each statement
  must be marked correctly.
\end{itemize}

The algorithm for \textbf{dichotomous scoring} is given by:

\begin{gather*}
f = 
\begin{cases}
  1, \text{if } i = n \\    
  0, \text{otherwise}    
\end{cases}
\end{gather*} \text{where } 0 \le i \le n

\begin{Shaded}
\begin{Highlighting}[]
\NormalTok{f\_dichom }\OtherTok{\textless{}{-}} \ControlFlowTok{function}\NormalTok{(i, n) \{}
 
  \CommentTok{\# print(paste("i is :",i," n is:",n)) }
  
  \CommentTok{\#if (n == 0 ) return error }
  \FunctionTok{ifelse}\NormalTok{( (n }\SpecialCharTok{==} \DecValTok{0}\NormalTok{), }\FunctionTok{print}\NormalTok{(}\StringTok{"ERROR n can\textquotesingle{}t be 0"}\NormalTok{), }\StringTok{""}\NormalTok{)}
  
  \CommentTok{\#if (i \textgreater{} n ) return error }
  \FunctionTok{ifelse}\NormalTok{( (i }\SpecialCharTok{\textgreater{}}\NormalTok{ n), }\FunctionTok{print}\NormalTok{(}\StringTok{"i n can\textquotesingle{}t \textgreater{} n"}\NormalTok{), }\StringTok{""}\NormalTok{)}
  
  \CommentTok{\#if (i==n) return 1, else 0}
  \FunctionTok{return}\NormalTok{ (}\FunctionTok{ifelse}\NormalTok{( (i}\SpecialCharTok{==}\NormalTok{n), }\DecValTok{1}\NormalTok{ , }\DecValTok{0}\NormalTok{))}
 
\NormalTok{\}}
\end{Highlighting}
\end{Shaded}

\hypertarget{partial-scoring--1n-1n}{%
\subsubsection*{Partial Scoring {[}-1/n,
+1/n{]}}\label{partial-scoring--1n-1n}}
\addcontentsline{toc}{subsubsection}{Partial Scoring {[}-1/n, +1/n{]}}

\textbf{Partial Scoring} refers to a class or scoring schemes that award
the respondent partial credit depending on pattern of options they
select. Schmidt et al. (2021) identify twenty-six different partial
credit scoring schemes in the literature, varying in the range of
possible scores, and the relative weighting of incorrectly selected (vs)
incorrectly unselected options.

A particularly elegant approach to partial scoring is referred to as the
\([-1/n, +1/n]\) approach ( Schmidt et al. (2021) \#17). This approach
is appealing in the context of SGC3A, because it: (1) takes into account
all information provided by the respondent: the pattern of what the
select, and choose not to select.

\textbf{In Partial Scoring} \([-1/n, +1/n]\):

\begin{itemize}
\tightlist
\item
  Scores range from {[}-1, +1{]}
\item
  One point is awarded if all options are \emph{correct}
\item
  One point point is subtracted if all options are \emph{incorrect}.
\item
  Intermediate results are credited as fractions accordingly (\(+1/n\)
  for each correct, \(-1/n\) for each incorrect)
\item
  This results in \emph{at chance performance} (i.e.~half of the given
  options marked correctly), being awarded 0 points are awarded
\end{itemize}

This scoring is more consistent with the motivating theory that
Triangular Graph readers start out with an incorrect (i.e.~orthogonal,
cartesian) interpretation of the coordinate system, and transition to a
correct (i.e.~triangular) interpretation. But the first step in making
this transition is realizing the cartesian interpretation \emph{is
incorrect}, which may yield blank responses where the respondent is
essentially saying, `there is no correct answer to this question'.

Schmidt et al. (2021) describe the \emph{Partial} \({[-1/n, +1/n]}\)
scoring scheme as the \emph{only} scoring method (of the 27 described)
where respondents' scoring results can be interpreted as a percentage of
their true knowledge. One important drawback of this method is that a
respondent may receive credit (a great deal of credit, depending on the
number of answer options n) even if she did not select \emph{any}
options. In the case (such as ours) where there are many more response
options \(n\) than there are options meant to be selected \(p\), this
partial scoring algorithm poses a challenge because the respondent can
achieve an almost completely perfect score by selecting a small number
of options that should not be selected.

The algorithm for \textbf{partial scoring}\([-1/n, +1/n]\) is given by:

\begin{align}
f &= (1/n * i) - (1/n * (n-i)) \\
&= (2i - n)/{n} 
\end{align}

\begin{Shaded}
\begin{Highlighting}[]
\NormalTok{f\_partialN }\OtherTok{\textless{}{-}} \ControlFlowTok{function}\NormalTok{(i, n) \{}

\CommentTok{\# print(paste("i is :",i," n is:",n))}

\CommentTok{\#if(n==0) return error}
\FunctionTok{ifelse}\NormalTok{((n}\SpecialCharTok{==}\DecValTok{0}\NormalTok{),}\FunctionTok{print}\NormalTok{(}\StringTok{"ERROR: n should not be 0"}\NormalTok{),}\StringTok{""}\NormalTok{)}

\CommentTok{\#if(i \textgreater{}n ) return error}
\FunctionTok{ifelse}\NormalTok{((i }\SpecialCharTok{\textgreater{}}\NormalTok{ n),}\FunctionTok{print}\NormalTok{(}\StringTok{"ERROR: i CANNOT BE GREATER THAN n"}\NormalTok{),}\StringTok{""}\NormalTok{)}

\FunctionTok{return}\NormalTok{ ((}\DecValTok{2}\SpecialCharTok{*}\NormalTok{i }\SpecialCharTok{{-}}\NormalTok{ n) }\SpecialCharTok{/}\NormalTok{ n) }
\NormalTok{\}}
\end{Highlighting}
\end{Shaded}

\hypertarget{sec-partialp}{%
\subsubsection*{Partial Scoring {[}-1/q, +1/p{]}}\label{sec-partialp}}
\addcontentsline{toc}{subsubsection}{Partial Scoring {[}-1/q, +1/p{]}}

One drawback of the Partial Scoring \([-1/n, +1/n]\) approach is that
treats the choice to select, and choice to not select options as equally
indicative of the respondent's understanding. That is to say,
incorrectly selecting one particular option is no more or less
informative than incorrectly not-selecting a different item. This
represents an important difference between MAMC (i.e.~``select all
correct options'') vs MTF (i.e.~``Mark each option as true or false'')
questions.

In our study, the selection of any particular option (remember options
represent data points on the stimulus graph) is indicative of a
particular interpretation of the stimulus. Incorrectly selecting an
option indicates an interpretation of the graph with respect to that
particular option. Alternatively, failing to select a correct option
\emph{might} mean the individual has a different interpretation, or that
they failed to find \emph{all} the data points consistent with the
interpretation.

For this reason, we consider another alternative Partial Scoring scheme
that takes into consideration only the selected statements, without
penalizing statements incorrectly \emph{not selected}. (See Schmidt et
al. (2021) method \#26; also referred to as the Morgan-Method) This
partial scoring scheme takes into consideration that the most
effort-free (or `default') response for any given item is the null, or
blank response. Blank responses indicate \emph{no understanding},
perhaps \emph{confusion}, or refusal to answer. These lack of responses
are awarded zero credit. Whereas taking the action to select an
\emph{incorrect} option is effortful, and is indicative of
\emph{incorrect understanding}.

\textbf{Partial Scoring} \([-1/q, +1/p]\):

\begin{itemize}
\tightlist
\item
  awards +1/p for each correctly selected option (\(p_s\)), and
  subtracts \(1/(n-p) = 1/q\) for each incorrectly selected option
  (\(q_s\))
\item
  only considers \emph{selected} options; does not penalize nor reward
  \emph{unselected} options
\end{itemize}

\textbf{Properties of Item}

\begin{align}
p &= \text{number of true-select options (i.e. should be selected)} \\
q &= \text{number of true-unselect options (i.e. should not be selected)} \\
n &= \text{number of options} \: ( n = p + q)
\end{align}

\textbf{Properties of Response}

\begin{align}
p_s &= \text{number of true-select options selected (i.e. number of correctly checked options)}\\
q_s &= \text{number of true-unselect options selected (i.e. number of incorrectly checked options }
\end{align}

The algorithm for \textbf{partial scoring} \([-1/q, +1/p]\) is given by:

\begin{align}
f &= (p_s / p) - ({q_s}/{q}) \\
\end{align}

\begin{Shaded}
\begin{Highlighting}[]
\NormalTok{f\_partialP }\OtherTok{\textless{}{-}} \ControlFlowTok{function}\NormalTok{(t,p,f,q) \{}

  \CommentTok{\#t = number of correct{-}selected options}
  \CommentTok{\#p = number of true options}
  \CommentTok{\#f = number of incorrect{-}selected options}
  \CommentTok{\#q = number of false options}
  \CommentTok{\#n = number of options + p + q}
  
  \FunctionTok{ifelse}\NormalTok{( (p }\SpecialCharTok{==} \DecValTok{0}\NormalTok{), }\FunctionTok{return}\NormalTok{(}\ConstantTok{NA}\NormalTok{), }\StringTok{""}\NormalTok{) }\CommentTok{\#handle empty response set gracefully by returning nothing rather than 0}
  \FunctionTok{ifelse}\NormalTok{( (p }\SpecialCharTok{!=} \DecValTok{0}\NormalTok{), }\FunctionTok{return}\NormalTok{( (t }\SpecialCharTok{/}\NormalTok{ p) }\SpecialCharTok{{-}}\NormalTok{ (f}\SpecialCharTok{/}\NormalTok{q)), }\StringTok{""}\NormalTok{)}
\NormalTok{\}}
\end{Highlighting}
\end{Shaded}

\hypertarget{comparison-of-schemes}{%
\subsection*{Comparison of Schemes}\label{comparison-of-schemes}}
\addcontentsline{toc}{subsection}{Comparison of Schemes}

Which scoring scheme is most appropriate for the goals of the graph
comprehension task?

Consider the following example:

\emph{For a question with} \(n = 5\) response options (data points A, B,
C, D and E) with a correct response of A, the schemes under
consideration yield the following scores:

\begin{Shaded}
\begin{Highlighting}[]
\NormalTok{title }\OtherTok{\textless{}{-}} \StringTok{"Comparison of Scoring Schemes for n = 5 options [ A,B,C,D,E ]"}

\NormalTok{correct }\OtherTok{\textless{}{-}} \FunctionTok{c}\NormalTok{( }\StringTok{"A\_\_\_\_"}\NormalTok{,  }
              \StringTok{"A\_\_\_\_"}\NormalTok{,      }
              \StringTok{"A\_\_\_\_"}\NormalTok{,        }
              \StringTok{"A\_\_\_\_"}\NormalTok{,        }
              \StringTok{"A\_\_\_\_"}\NormalTok{,      }
              \StringTok{"A\_\_\_\_"}\NormalTok{,      }
              \StringTok{"A\_\_\_\_"}\NormalTok{,      }
              \StringTok{"A\_\_\_\_"}\NormalTok{,      }
              \StringTok{"A\_\_\_\_"}\NormalTok{ ) }

\NormalTok{response }\OtherTok{\textless{}{-}} \FunctionTok{c}\NormalTok{(}\StringTok{"A\_\_\_\_"}\NormalTok{,  }
              \StringTok{"AB\_\_\_"}\NormalTok{,      }
              \StringTok{"A\_\_\_E"}\NormalTok{,      }
              \StringTok{"AB\_\_E"}\NormalTok{,        }
              \StringTok{"\_\_\_\_E"}\NormalTok{,}
              \StringTok{"\_\_\_DE"}\NormalTok{,}
              \StringTok{"\_BCDE"}\NormalTok{,      }
              \StringTok{"ABCDE"}\NormalTok{,      }
              \StringTok{"\_\_\_\_\_"}\NormalTok{ )}

\NormalTok{i }\OtherTok{\textless{}{-}} \FunctionTok{c}\NormalTok{(        }\DecValTok{5}\NormalTok{,       }
               \DecValTok{4}\NormalTok{,              }
               \DecValTok{4}\NormalTok{,              }
               \DecValTok{3}\NormalTok{,               }
               
               \DecValTok{3}\NormalTok{,}
               \DecValTok{2}\NormalTok{,}
               \DecValTok{0}\NormalTok{,}
               \DecValTok{1}\NormalTok{,}
               \DecValTok{4}\NormalTok{)}

\NormalTok{abs }\OtherTok{\textless{}{-}} \FunctionTok{c}\NormalTok{(}\FunctionTok{f\_dichom}\NormalTok{(}\DecValTok{5}\NormalTok{,}\DecValTok{5}\NormalTok{), }
         \FunctionTok{f\_dichom}\NormalTok{(}\DecValTok{4}\NormalTok{,}\DecValTok{5}\NormalTok{), }
         \FunctionTok{f\_dichom}\NormalTok{(}\DecValTok{4}\NormalTok{,}\DecValTok{5}\NormalTok{), }
         \FunctionTok{f\_dichom}\NormalTok{(}\DecValTok{3}\NormalTok{,}\DecValTok{5}\NormalTok{), }
         
         \FunctionTok{f\_dichom}\NormalTok{(}\DecValTok{3}\NormalTok{,}\DecValTok{5}\NormalTok{), }
         \FunctionTok{f\_dichom}\NormalTok{(}\DecValTok{2}\NormalTok{,}\DecValTok{5}\NormalTok{),}
         \FunctionTok{f\_dichom}\NormalTok{(}\DecValTok{0}\NormalTok{,}\DecValTok{5}\NormalTok{),}
         \FunctionTok{f\_dichom}\NormalTok{(}\DecValTok{1}\NormalTok{,}\DecValTok{5}\NormalTok{),}
         \FunctionTok{f\_dichom}\NormalTok{(}\DecValTok{4}\NormalTok{,}\DecValTok{5}\NormalTok{))}

\NormalTok{partial1 }\OtherTok{\textless{}{-}} \FunctionTok{c}\NormalTok{(}\FunctionTok{f\_partialN}\NormalTok{(}\DecValTok{5}\NormalTok{,}\DecValTok{5}\NormalTok{), }
              \FunctionTok{f\_partialN}\NormalTok{(}\DecValTok{4}\NormalTok{,}\DecValTok{5}\NormalTok{), }
              \FunctionTok{f\_partialN}\NormalTok{(}\DecValTok{4}\NormalTok{,}\DecValTok{5}\NormalTok{), }
              \FunctionTok{f\_partialN}\NormalTok{(}\DecValTok{3}\NormalTok{,}\DecValTok{5}\NormalTok{), }
              
              \FunctionTok{f\_partialN}\NormalTok{(}\DecValTok{3}\NormalTok{,}\DecValTok{5}\NormalTok{), }
              \FunctionTok{f\_partialN}\NormalTok{(}\DecValTok{2}\NormalTok{,}\DecValTok{5}\NormalTok{),}
              \FunctionTok{f\_partialN}\NormalTok{(}\DecValTok{0}\NormalTok{,}\DecValTok{5}\NormalTok{),}
              \FunctionTok{f\_partialN}\NormalTok{(}\DecValTok{1}\NormalTok{,}\DecValTok{5}\NormalTok{),}
              \FunctionTok{f\_partialN}\NormalTok{(}\DecValTok{4}\NormalTok{,}\DecValTok{5}\NormalTok{))}

\NormalTok{partial2 }\OtherTok{\textless{}{-}} \FunctionTok{c}\NormalTok{(}\FunctionTok{f\_partialP}\NormalTok{(}\DecValTok{1}\NormalTok{,}\DecValTok{1}\NormalTok{,}\DecValTok{0}\NormalTok{,}\DecValTok{4}\NormalTok{), }
              \FunctionTok{f\_partialP}\NormalTok{(}\DecValTok{1}\NormalTok{,}\DecValTok{1}\NormalTok{,}\DecValTok{1}\NormalTok{,}\DecValTok{4}\NormalTok{), }
              \FunctionTok{f\_partialP}\NormalTok{(}\DecValTok{1}\NormalTok{,}\DecValTok{1}\NormalTok{,}\DecValTok{1}\NormalTok{,}\DecValTok{4}\NormalTok{), }
              \FunctionTok{f\_partialP}\NormalTok{(}\DecValTok{1}\NormalTok{,}\DecValTok{1}\NormalTok{,}\DecValTok{2}\NormalTok{,}\DecValTok{4}\NormalTok{), }
              
              \FunctionTok{f\_partialP}\NormalTok{(}\DecValTok{0}\NormalTok{,}\DecValTok{1}\NormalTok{,}\DecValTok{1}\NormalTok{,}\DecValTok{4}\NormalTok{),}
              \FunctionTok{f\_partialP}\NormalTok{(}\DecValTok{0}\NormalTok{,}\DecValTok{1}\NormalTok{,}\DecValTok{2}\NormalTok{,}\DecValTok{4}\NormalTok{),}
              \FunctionTok{f\_partialP}\NormalTok{(}\DecValTok{0}\NormalTok{,}\DecValTok{1}\NormalTok{,}\DecValTok{4}\NormalTok{,}\DecValTok{4}\NormalTok{),}
              \FunctionTok{f\_partialP}\NormalTok{(}\DecValTok{1}\NormalTok{,}\DecValTok{1}\NormalTok{,}\DecValTok{4}\NormalTok{,}\DecValTok{4}\NormalTok{), }
              \FunctionTok{f\_partialP}\NormalTok{(}\DecValTok{0}\NormalTok{,}\DecValTok{1}\NormalTok{,}\DecValTok{0}\NormalTok{,}\DecValTok{4}\NormalTok{))}

\NormalTok{names }\OtherTok{=} \FunctionTok{c}\NormalTok{(    }\StringTok{"Correct Answer"}\NormalTok{,}
              \StringTok{"Response"}\NormalTok{,}
              \StringTok{"i "}\NormalTok{,}
              \StringTok{"Dichotomous"}\NormalTok{,}
              \StringTok{"Partial [{-}1/n, +1/n]"}\NormalTok{,}
              \StringTok{"Partial[{-}1/q, +1/p]"}\NormalTok{)}

\NormalTok{dt }\OtherTok{\textless{}{-}} \FunctionTok{data.frame}\NormalTok{(correct, response, i, abs, partial1 , partial2)}

\FunctionTok{kbl}\NormalTok{(dt, }\AttributeTok{col.names =}\NormalTok{ names, }\AttributeTok{caption =}\NormalTok{ title, }\AttributeTok{digits=}\DecValTok{3}\NormalTok{) }\SpecialCharTok{\%\textgreater{}\%}
  \FunctionTok{kable\_classic}\NormalTok{() }\SpecialCharTok{\%\textgreater{}\%}
    \FunctionTok{add\_header\_above}\NormalTok{(}\FunctionTok{c}\NormalTok{(}\StringTok{"Response Scenario "} \OtherTok{=} \DecValTok{3}\NormalTok{, }\StringTok{"Scores"} \OtherTok{=} \DecValTok{3}\NormalTok{)) }\SpecialCharTok{\%\textgreater{}\%} 
    \FunctionTok{pack\_rows}\NormalTok{(}\StringTok{"Perfect Response"}\NormalTok{, }\DecValTok{1}\NormalTok{, }\DecValTok{1}\NormalTok{) }\SpecialCharTok{\%\textgreater{}\%}
    \FunctionTok{pack\_rows}\NormalTok{(}\StringTok{"Correct + Extra Incorrect Selections"}\NormalTok{, }\DecValTok{2}\NormalTok{, }\DecValTok{4}\NormalTok{) }\SpecialCharTok{\%\textgreater{}\%}
    \FunctionTok{pack\_rows}\NormalTok{(}\StringTok{"Only Incorrect Selections"}\NormalTok{, }\DecValTok{5}\NormalTok{, }\DecValTok{6}\NormalTok{) }\SpecialCharTok{\%\textgreater{}\%}
    \FunctionTok{pack\_rows}\NormalTok{(}\StringTok{"Completely Inverse Response "}\NormalTok{, }\DecValTok{7}\NormalTok{, }\DecValTok{7}\NormalTok{) }\SpecialCharTok{\%\textgreater{}\%}
    \FunctionTok{pack\_rows}\NormalTok{(}\StringTok{"Selected ALL or NONE"}\NormalTok{, }\DecValTok{8}\NormalTok{, }\DecValTok{9}\NormalTok{) }\SpecialCharTok{\%\textgreater{}\%}
    \FunctionTok{footnote}\NormalTok{(}\AttributeTok{general =} \FunctionTok{paste}\NormalTok{(}\StringTok{"i = number of options in correct state; \_ indicates option not selected"}\NormalTok{),}
           \AttributeTok{general\_title =} \StringTok{"Note: "}\NormalTok{,}\AttributeTok{footnote\_as\_chunk =}\NormalTok{ T)}
\end{Highlighting}
\end{Shaded}

\begin{table}

\caption{Comparison of Scoring Schemes for n = 5 options [ A,B,C,D,E ]}
\centering
\begin{tabular}[t]{l|l|r|r|r|r}
\hline
\multicolumn{3}{c|}{Response Scenario } & \multicolumn{3}{c}{Scores} \\
\cline{1-3} \cline{4-6}
Correct Answer & Response & i  & Dichotomous & Partial [-1/n, +1/n] & Partial[-1/q, +1/p]\\
\hline
\multicolumn{6}{l}{\textbf{Perfect Response}}\\
\hline
\hspace{1em}A\_\_\_\_ & A\_\_\_\_ & 5 & 1 & 1.0 & 1.00\\
\hline
\multicolumn{6}{l}{\textbf{Correct + Extra Incorrect Selections}}\\
\hline
\hspace{1em}A\_\_\_\_ & AB\_\_\_ & 4 & 0 & 0.6 & 0.75\\
\hline
\hspace{1em}A\_\_\_\_ & A\_\_\_E & 4 & 0 & 0.6 & 0.75\\
\hline
\hspace{1em}A\_\_\_\_ & AB\_\_E & 3 & 0 & 0.2 & 0.50\\
\hline
\multicolumn{6}{l}{\textbf{Only Incorrect Selections}}\\
\hline
\hspace{1em}A\_\_\_\_ & \_\_\_\_E & 3 & 0 & 0.2 & -0.25\\
\hline
\hspace{1em}A\_\_\_\_ & \_\_\_DE & 2 & 0 & -0.2 & -0.50\\
\hline
\multicolumn{6}{l}{\textbf{Completely Inverse Response }}\\
\hline
\hspace{1em}A\_\_\_\_ & \_BCDE & 0 & 0 & -1.0 & -1.00\\
\hline
\multicolumn{6}{l}{\textbf{Selected ALL or NONE}}\\
\hline
\hspace{1em}A\_\_\_\_ & ABCDE & 1 & 0 & -0.6 & 0.00\\
\hline
\hspace{1em}A\_\_\_\_ & \_\_\_\_\_ & 4 & 0 & 0.6 & 0.00\\
\hline
\multicolumn{6}{l}{\rule{0pt}{1em}\textit{Note: } i = number of options in correct state; \_ indicates option not selected}\\
\end{tabular}
\end{table}

\begin{Shaded}
\begin{Highlighting}[]
\CommentTok{\#cleanup}
\FunctionTok{rm}\NormalTok{(dt, abs, correct,i,names,partial1,partial2,response,title)}
\end{Highlighting}
\end{Shaded}

\begin{itemize}
\item
  We see that in the Dichotomous scheme, only the precisely correct
  response (row 1) yields a score other than zero. This scheme does now
  allow us to differentiate between different response patters.
\item
  The Partial \([-1/n, +1/n]\) scheme yields a range from \([-1,1]\),
  differentiating between responses. However, a blank response (bottom
  row) receives the same score (0.6) as the selection of the correct
  option + 1 incorrect option (row 2), which is problematic with for the
  goals of this study, where we need to differentiate between states of
  confusion or uncertainty yielding blank responses and the intentional
  selection of incorrect items.
\item
  The Partial \([-1/q, +1/p]\) scheme also yields a range of scores from
  \([-1,1]\). A blank response (bottom row) yields the same score
  (\(0\)) as the selection of \emph{all} answer options (row 7); both
  are patterns of behavior we would expect to see if a respondent is
  confused or uncertain that there is a correct answer to the question.
\end{itemize}

Next we consider an example from our study, with \(n = 15\) options and
\(p = 1\) correct option to be selected.

\begin{Shaded}
\begin{Highlighting}[]
\NormalTok{title }\OtherTok{\textless{}{-}} \StringTok{"Comparison of Scoring Schemes for SGC3 with n=15 and p=1 options [A,B...N,O]  "}

\NormalTok{correct }\OtherTok{\textless{}{-}} \FunctionTok{c}\NormalTok{( }\StringTok{"A\_\_\_\_"}\NormalTok{,  }
              \StringTok{"A\_\_\_\_"}\NormalTok{,      }
              \StringTok{"A\_\_\_\_"}\NormalTok{,      }
              \StringTok{"A\_\_\_\_"}\NormalTok{,        }
              \StringTok{"A\_\_\_\_"}\NormalTok{,      }
              \StringTok{"A\_\_\_\_"}\NormalTok{,      }
              \StringTok{"A\_\_\_\_"}\NormalTok{,      }
              \StringTok{"A\_\_\_\_"}\NormalTok{,      }
              \StringTok{"A\_\_\_\_"}\NormalTok{ ) }

\NormalTok{response }\OtherTok{\textless{}{-}} \FunctionTok{c}\NormalTok{(}\StringTok{"A\_\_...\_\_"}\NormalTok{,  }
              \StringTok{"AB\_...\_\_"}\NormalTok{,      }
              \StringTok{"A\_\_...\_O"}\NormalTok{,      }
              \StringTok{"AB\_...\_O"}\NormalTok{,        }
              \StringTok{"\_\_\_...\_O"}\NormalTok{,      }
              \StringTok{"\_\_\_...NO"}\NormalTok{,      }
              \StringTok{"\_BC...NO"}\NormalTok{,}
              \StringTok{"ABC...NO"}\NormalTok{,      }
              \StringTok{"\_\_\_...\_\_"}\NormalTok{ )}

\NormalTok{i }\OtherTok{\textless{}{-}} \FunctionTok{c}\NormalTok{(        }\DecValTok{15}\NormalTok{,       }
               \DecValTok{14}\NormalTok{,              }
               \DecValTok{14}\NormalTok{,              }
               \DecValTok{13}\NormalTok{,}
               \DecValTok{13}\NormalTok{,               }
               \DecValTok{12}\NormalTok{,          }
               \DecValTok{0}\NormalTok{,}
               \DecValTok{1}\NormalTok{,}
               \DecValTok{14}\NormalTok{)}

\NormalTok{abs }\OtherTok{\textless{}{-}} \FunctionTok{c}\NormalTok{(}\FunctionTok{f\_dichom}\NormalTok{(}\DecValTok{15}\NormalTok{,}\DecValTok{15}\NormalTok{), }
         \FunctionTok{f\_dichom}\NormalTok{(}\DecValTok{14}\NormalTok{,}\DecValTok{15}\NormalTok{), }
         \FunctionTok{f\_dichom}\NormalTok{(}\DecValTok{14}\NormalTok{,}\DecValTok{15}\NormalTok{), }
         \FunctionTok{f\_dichom}\NormalTok{(}\DecValTok{13}\NormalTok{,}\DecValTok{15}\NormalTok{), }
         \FunctionTok{f\_dichom}\NormalTok{(}\DecValTok{13}\NormalTok{,}\DecValTok{15}\NormalTok{),}
         \FunctionTok{f\_dichom}\NormalTok{(}\DecValTok{12}\NormalTok{,}\DecValTok{15}\NormalTok{),}
         \FunctionTok{f\_dichom}\NormalTok{(}\DecValTok{0}\NormalTok{,}\DecValTok{15}\NormalTok{),}
         \FunctionTok{f\_dichom}\NormalTok{(}\DecValTok{1}\NormalTok{,}\DecValTok{15}\NormalTok{),}
         \FunctionTok{f\_dichom}\NormalTok{(}\DecValTok{14}\NormalTok{,}\DecValTok{15}\NormalTok{))}

\NormalTok{partial1 }\OtherTok{\textless{}{-}} \FunctionTok{c}\NormalTok{(}\FunctionTok{f\_partialN}\NormalTok{(}\DecValTok{15}\NormalTok{,}\DecValTok{15}\NormalTok{), }
              \FunctionTok{f\_partialN}\NormalTok{(}\DecValTok{14}\NormalTok{,}\DecValTok{15}\NormalTok{), }
              \FunctionTok{f\_partialN}\NormalTok{(}\DecValTok{14}\NormalTok{,}\DecValTok{15}\NormalTok{), }
              \FunctionTok{f\_partialN}\NormalTok{(}\DecValTok{13}\NormalTok{,}\DecValTok{15}\NormalTok{), }
              \FunctionTok{f\_partialN}\NormalTok{(}\DecValTok{13}\NormalTok{,}\DecValTok{15}\NormalTok{),}
              \FunctionTok{f\_partialN}\NormalTok{(}\DecValTok{12}\NormalTok{,}\DecValTok{15}\NormalTok{),}
              \FunctionTok{f\_partialN}\NormalTok{(}\DecValTok{0}\NormalTok{,}\DecValTok{15}\NormalTok{),}
              \FunctionTok{f\_partialN}\NormalTok{(}\DecValTok{1}\NormalTok{,}\DecValTok{15}\NormalTok{),}
              \FunctionTok{f\_partialN}\NormalTok{(}\DecValTok{14}\NormalTok{,}\DecValTok{15}\NormalTok{))}

\NormalTok{partial2 }\OtherTok{\textless{}{-}} \FunctionTok{c}\NormalTok{(}\FunctionTok{f\_partialP}\NormalTok{(}\DecValTok{1}\NormalTok{,}\DecValTok{1}\NormalTok{,}\DecValTok{0}\NormalTok{,}\DecValTok{14}\NormalTok{), }
              \FunctionTok{f\_partialP}\NormalTok{(}\DecValTok{1}\NormalTok{,}\DecValTok{1}\NormalTok{,}\DecValTok{1}\NormalTok{,}\DecValTok{14}\NormalTok{), }
              \FunctionTok{f\_partialP}\NormalTok{(}\DecValTok{1}\NormalTok{,}\DecValTok{1}\NormalTok{,}\DecValTok{1}\NormalTok{,}\DecValTok{14}\NormalTok{), }
              \FunctionTok{f\_partialP}\NormalTok{(}\DecValTok{1}\NormalTok{,}\DecValTok{1}\NormalTok{,}\DecValTok{2}\NormalTok{,}\DecValTok{14}\NormalTok{), }
              \FunctionTok{f\_partialP}\NormalTok{(}\DecValTok{0}\NormalTok{,}\DecValTok{1}\NormalTok{,}\DecValTok{1}\NormalTok{,}\DecValTok{14}\NormalTok{),}
              \FunctionTok{f\_partialP}\NormalTok{(}\DecValTok{0}\NormalTok{,}\DecValTok{1}\NormalTok{,}\DecValTok{2}\NormalTok{,}\DecValTok{14}\NormalTok{),}
              \FunctionTok{f\_partialP}\NormalTok{(}\DecValTok{0}\NormalTok{,}\DecValTok{1}\NormalTok{,}\DecValTok{14}\NormalTok{,}\DecValTok{14}\NormalTok{),}
              \FunctionTok{f\_partialP}\NormalTok{(}\DecValTok{1}\NormalTok{,}\DecValTok{1}\NormalTok{,}\DecValTok{14}\NormalTok{,}\DecValTok{14}\NormalTok{), }
              \FunctionTok{f\_partialP}\NormalTok{(}\DecValTok{0}\NormalTok{,}\DecValTok{1}\NormalTok{,}\DecValTok{0}\NormalTok{,}\DecValTok{14}\NormalTok{))}

\NormalTok{names }\OtherTok{=} \FunctionTok{c}\NormalTok{(    }\StringTok{"Correct Answer"}\NormalTok{,}
              \StringTok{"Response"}\NormalTok{,}
              \StringTok{"$i$ "}\NormalTok{,}
              \StringTok{"Dichotomous"}\NormalTok{,}
              \StringTok{"Partial [{-}1/n, +1/n]"}\NormalTok{,}
              \StringTok{"Partial [{-}1/q, +1/p]"}\NormalTok{)}

\NormalTok{dt }\OtherTok{\textless{}{-}} \FunctionTok{data.frame}\NormalTok{(correct, response, i, abs, partial1 , partial2)}

\FunctionTok{kbl}\NormalTok{(dt, }\AttributeTok{col.names =}\NormalTok{ names, }\AttributeTok{caption =}\NormalTok{ title, }\AttributeTok{digits=}\DecValTok{3}\NormalTok{) }\SpecialCharTok{\%\textgreater{}\%}
  \FunctionTok{kable\_classic}\NormalTok{() }\SpecialCharTok{\%\textgreater{}\%}
    \FunctionTok{add\_header\_above}\NormalTok{(}\FunctionTok{c}\NormalTok{(}\StringTok{"Response Scenario "} \OtherTok{=} \DecValTok{3}\NormalTok{, }\StringTok{"Scores"} \OtherTok{=} \DecValTok{3}\NormalTok{)) }\SpecialCharTok{\%\textgreater{}\%} 
    \FunctionTok{pack\_rows}\NormalTok{(}\StringTok{"Perfect Response"}\NormalTok{, }\DecValTok{1}\NormalTok{, }\DecValTok{1}\NormalTok{) }\SpecialCharTok{\%\textgreater{}\%}
    \FunctionTok{pack\_rows}\NormalTok{(}\StringTok{"Correct + Extra Incorrect Selections"}\NormalTok{, }\DecValTok{2}\NormalTok{, }\DecValTok{4}\NormalTok{) }\SpecialCharTok{\%\textgreater{}\%}
    \FunctionTok{pack\_rows}\NormalTok{(}\StringTok{"Only Incorrect Selections"}\NormalTok{, }\DecValTok{5}\NormalTok{, }\DecValTok{6}\NormalTok{) }\SpecialCharTok{\%\textgreater{}\%}
    \FunctionTok{pack\_rows}\NormalTok{(}\StringTok{"Completely Inverse Response "}\NormalTok{, }\DecValTok{7}\NormalTok{, }\DecValTok{7}\NormalTok{) }\SpecialCharTok{\%\textgreater{}\%}
    \FunctionTok{pack\_rows}\NormalTok{(}\StringTok{"Selected ALL or NONE"}\NormalTok{, }\DecValTok{8}\NormalTok{, }\DecValTok{9}\NormalTok{) }\SpecialCharTok{\%\textgreater{}\%}
    \FunctionTok{footnote}\NormalTok{(}\AttributeTok{general =} \FunctionTok{paste}\NormalTok{(}\StringTok{"i = number of options in correct state; \_ indicates option not selected"}\NormalTok{),}
           \AttributeTok{general\_title =} \StringTok{"Note: "}\NormalTok{,}\AttributeTok{footnote\_as\_chunk =}\NormalTok{ T)}
\end{Highlighting}
\end{Shaded}

\begin{table}

\caption{Comparison of Scoring Schemes for SGC3 with n=15 and p=1 options [A,B...N,O]  }
\centering
\begin{tabular}[t]{l|l|r|r|r|r}
\hline
\multicolumn{3}{c|}{Response Scenario } & \multicolumn{3}{c}{Scores} \\
\cline{1-3} \cline{4-6}
Correct Answer & Response & \$i\$  & Dichotomous & Partial [-1/n, +1/n] & Partial [-1/q, +1/p]\\
\hline
\multicolumn{6}{l}{\textbf{Perfect Response}}\\
\hline
\hspace{1em}A\_\_\_\_ & A\_\_...\_\_ & 15 & 1 & 1.000 & 1.000\\
\hline
\multicolumn{6}{l}{\textbf{Correct + Extra Incorrect Selections}}\\
\hline
\hspace{1em}A\_\_\_\_ & AB\_...\_\_ & 14 & 0 & 0.867 & 0.929\\
\hline
\hspace{1em}A\_\_\_\_ & A\_\_...\_O & 14 & 0 & 0.867 & 0.929\\
\hline
\hspace{1em}A\_\_\_\_ & AB\_...\_O & 13 & 0 & 0.733 & 0.857\\
\hline
\multicolumn{6}{l}{\textbf{Only Incorrect Selections}}\\
\hline
\hspace{1em}A\_\_\_\_ & \_\_\_...\_O & 13 & 0 & 0.733 & -0.071\\
\hline
\hspace{1em}A\_\_\_\_ & \_\_\_...NO & 12 & 0 & 0.600 & -0.143\\
\hline
\multicolumn{6}{l}{\textbf{Completely Inverse Response }}\\
\hline
\hspace{1em}A\_\_\_\_ & \_BC...NO & 0 & 0 & -1.000 & -1.000\\
\hline
\multicolumn{6}{l}{\textbf{Selected ALL or NONE}}\\
\hline
\hspace{1em}A\_\_\_\_ & ABC...NO & 1 & 0 & -0.867 & 0.000\\
\hline
\hspace{1em}A\_\_\_\_ & \_\_\_...\_\_ & 14 & 0 & 0.867 & 0.000\\
\hline
\multicolumn{6}{l}{\rule{0pt}{1em}\textit{Note: } i = number of options in correct state; \_ indicates option not selected}\\
\end{tabular}
\end{table}

\begin{Shaded}
\begin{Highlighting}[]
\CommentTok{\#cleanup}
\FunctionTok{rm}\NormalTok{(dt, abs, correct,i,names,partial1,partial2,response,title)}
\end{Highlighting}
\end{Shaded}

Here again we see that the Partial \([-1/q, +1/p]\) scheme allows us
differentiate between patterns of responses, in a way that is more
sensible for the goals of the SGC3 graph comprehension task.

\begin{tcolorbox}[standard jigsaw, bottomrule=.15mm, toprule=.15mm, colback=white, arc=.35mm, colframe=quarto-callout-color-frame, rightrule=.15mm, leftrule=.75mm, opacityback=0, left=2mm]
The Partial \([-1/q, +1/p]\) scheme is more appropriate for scoring the
graph comprehension task than the Partial \([-1/n, +1/n]\) scheme
because it allows us to differentially penalize incorrectly
\emph{selected} and incorrectly \emph{not selected} answer options.
\end{tcolorbox}

\hypertarget{sec-scoringOverview}{%
\section*{SCORING SGC DATA}\label{sec-scoringOverview}}
\addcontentsline{toc}{section}{SCORING SGC DATA}

In SGC\_3A we are fundamentally interested in understanding how a
participant interprets the presented graph (stimulus). The \textbf{graph
comprehension task} asks them to select the data points in the graph
that meet the criteria posed in the question. To assess a participant's
performance, for each question (q=15) we will calculate the following
scores:

\emph{An overall, strict score:}\\
1. \textbf{Absolute Score} : using dichotomous scoring referencing true
(Triangular) answer. (see 1.2)

\emph{Sub-scores, for each alternative graph interpretation}\\
2. \textbf{Triangular Score} : using partial scoring {[}-1/q, +1/p{]}
referencing true (Triangular) answer key.

3. \textbf{Orthogonal Score} : using partial scoring {[}-1/q, +1/p{]}
referencing (incorrect Orthogonal) answer key.

Based on prior observational studies where we observed emergence of
other alternative interpretations (i.e.~transitional interpretations) we
also calculate subscores for these alternatives.

4. \textbf{Tversky Score} : using partial scoring {[}-1/q, +1/p{]}
referencing (incorrect connecting-lines strategy) answer key.

5. \textbf{Satisficing Score} : using partial scoring {[}-1/q, +1/p{]}
referencing (incorrect satisficing strategy) answer key.

For each study in the SGC project, MR data will be scored by following
these steps:

(1) Preparing answer keys: For each dataset+question set combination, an
answer key is that defines the `correct' answer set under \emph{each}
interpretation of the graph (i.e.~a triangular answer, an orthogonal
answer, etc).

(2) Calculate strategy scores: Using the strategy specific answer keys,
an interpretation subscore is calculated for each response for each
interpretation.

(3) Interpretation classification: The interpretation subscores are
compared in order to classify each response as a particular
interpretation. If no classification can be made, the response is
classified as `?'.

(4) Calculate Absolute and Scaled Scores: Two final scores are
calculated for each response; an Absolute score that indicates if the
response was precisely correct according to the triangular
interpretation, and a Scaled score that assigns a numeric value to the
interpretation given by the response (ranging from -1 to +1)

\hypertarget{sec-scoring-keys}{%
\subsection*{1. Prepare Answer Keys}\label{sec-scoring-keys}}
\addcontentsline{toc}{subsection}{1. Prepare Answer Keys}

We start by importing three answer keys: (1) Q1 - Q5 {[}control
condition{]}, (2) Q1-Q5 {[}impasse condition{]}, (3) Q6-15. Separate
answer keys by condition are required for Q1-Q5 because the stimuli for
each condition visualize a different underlying dataset (i.e.~the graphs
show datapoints in different positions). Q6-Q15 are identical across
conditions. Each answer key includes a row for each question, and a
column defining the subset of response options that correspond to
different graph interpretations.

\begin{Shaded}
\begin{Highlighting}[]
\CommentTok{\# imac = "/Users/amyraefox/Code/SGC{-}Scaffolding\_Graph\_Comprehension/SGC{-}X/ANALYSIS/MAIN"}
\CommentTok{\# setwd(imac)}

\CommentTok{\#LOAD INDIVIDUAL KEY FILES }
\NormalTok{key\_111\_raw }\OtherTok{\textless{}{-}} \FunctionTok{read\_csv}\NormalTok{(}\StringTok{\textquotesingle{}analysis/utils/keys/SGCX\_scaffold\_111\_key.csv\textquotesingle{}}\NormalTok{) }\SpecialCharTok{\%\textgreater{}\%} \FunctionTok{mutate}\NormalTok{(}\AttributeTok{condition =} \StringTok{"DEFAULT"}\NormalTok{, }\AttributeTok{phase =} \StringTok{"scaffold"}\NormalTok{)}
\NormalTok{key\_121\_raw }\OtherTok{\textless{}{-}} \FunctionTok{read\_csv}\NormalTok{(}\StringTok{\textquotesingle{}analysis/utils/keys/SGCX\_scaffold\_121\_key.csv\textquotesingle{}}\NormalTok{)}\SpecialCharTok{\%\textgreater{}\%} \FunctionTok{mutate}\NormalTok{(}\AttributeTok{condition =} \DecValTok{121}\NormalTok{, }\AttributeTok{phase =} \StringTok{"scaffold"}\NormalTok{)}
\NormalTok{cs }\OtherTok{=} \FunctionTok{rep}\NormalTok{(}\StringTok{\textquotesingle{}c\textquotesingle{}}\NormalTok{, }\DecValTok{23}\NormalTok{) }\SpecialCharTok{\%\textgreater{}\%} \FunctionTok{str\_c}\NormalTok{(}\AttributeTok{collapse=}\StringTok{""}\NormalTok{) }\CommentTok{\#create column spec }
\NormalTok{key\_test\_raw }\OtherTok{\textless{}{-}} \FunctionTok{read\_csv}\NormalTok{(}\StringTok{\textquotesingle{}analysis/utils/keys/SGCX\_test\_key.csv\textquotesingle{}}\NormalTok{, }\AttributeTok{col\_types =}\NormalTok{ cs)}\SpecialCharTok{\%\textgreater{}\%} \FunctionTok{mutate}\NormalTok{(}\AttributeTok{condition =} \StringTok{"DEFAULT"}\NormalTok{, }\AttributeTok{phase =} \StringTok{"test"}\NormalTok{) }

\CommentTok{\#JOIN THEM}
\NormalTok{keys\_raw }\OtherTok{\textless{}{-}} \FunctionTok{rbind}\NormalTok{(key\_111\_raw, key\_121\_raw, key\_test\_raw )}

\CommentTok{\#CLEANUP}
\FunctionTok{rm}\NormalTok{(key\_111\_raw, key\_121\_raw, key\_test\_raw)}
\end{Highlighting}
\end{Shaded}

In order to calculate scores using the \([-1/q, +1/p]\) algorithm, we
need to define the subset of all response options (set N) that should be
selected (set P) and should not be selected (set Q). In order to
calculate subscores for each graph interpretation (i.e.~triangular,
orthogonal, tversky) we must define these sets independently for each
interpretation. For each question, the \texttt{keys\_raw} dataframe
gives us set N (all response options), and a set P (options that should
be selected) for \emph{each interpretation}. From these we must derive
set Q for each interpretation.

\begin{itemize}
\item
  SET \(N\), all response options (superset) . This set is the same
  across all interpretations (a property of the question) and is given
  in the answer key.
\item
  SET \(P\), \(P \subset N\) , the subset of options that
  \textbf{should} be selected (rewarded as +1/p) . This set differs by
  interpretation, and is given in the answer key.
\item
  SET \(A, A \subset N, A \sqcup P\) , the subset of options that
  \textbf{should not} be selected, \emph{but if they are, aren't
  penalized} (i.e.~these options are ignored. Not rewarded, nor
  penalized). These include any options referenced in the question
  (i.e.~select shifts that start at the same time as X; don't penalize
  if they also select `X'), as well as options within 0.5hr offset from
  the data point to accommodate reasonable visual errors. This set
  differs by interpretation, and is given in the answer key (columns
  \texttt{REF\_POINT} and \texttt{\_also}).
\item
  SET \(Q\), the subset of options that \textbf{should not be selected}
  and are penalized (as -1/q). This set differs by interpretation and is
  not given in the answer key. We can derive set Q for each
  interpretation by \(Q = N - (P \cup A)\)
\end{itemize}

The next step in scoring is preparing interpretation-specific answer
keys that specify sets N, P, A and Q for each question.

\hypertarget{triangular-key}{%
\subsubsection*{Triangular Key}\label{triangular-key}}
\addcontentsline{toc}{subsubsection}{Triangular Key}

First we construct a key set based on the `Triangular' interpretation
(i.e.~the actually correct answers).

\begin{Shaded}
\begin{Highlighting}[]
\NormalTok{verify\_tri }\OtherTok{=} \FunctionTok{c}\NormalTok{() }\CommentTok{\#sanity check}
\DocumentationTok{\#\#——————————————————————————————————————————————————————————————————————}
\DocumentationTok{\#\#CONSTRUCT TRIANGULAR KEY SET}
\DocumentationTok{\#\#——————————————————————————————————————————————————————————————————————}
\CommentTok{\#1. DEFINE SETS N, P, A}
\NormalTok{keys\_tri }\OtherTok{\textless{}{-}}\NormalTok{ keys\_raw }\SpecialCharTok{\%\textgreater{}\%} 
  \FunctionTok{select}\NormalTok{(Q, condition, OPTIONS, TRIANGULAR, TRI\_allow, REF\_POINT) }\SpecialCharTok{\%\textgreater{}\%} 
  \FunctionTok{mutate}\NormalTok{(}
    
    \CommentTok{\#replace NAs }
    \AttributeTok{TRI\_allow =} \FunctionTok{str\_replace\_na}\NormalTok{(TRI\_allow,}\StringTok{""}\NormalTok{),}
    \AttributeTok{REF\_POINT =} \FunctionTok{str\_replace\_na}\NormalTok{(REF\_POINT,}\StringTok{""}\NormalTok{),}
    
    \CommentTok{\#P options that SHOULD be selected (rewarded)}
    \AttributeTok{set\_p =}\NormalTok{ TRIANGULAR,}
    \AttributeTok{set\_p =} \FunctionTok{str\_replace\_na}\NormalTok{(set\_p,}\StringTok{""}\NormalTok{),}\CommentTok{\#replace na if empty}
    \AttributeTok{n\_p =} \FunctionTok{nchar}\NormalTok{(set\_p), }\CommentTok{\#number of true{-}select options}
    
    \CommentTok{\#A options that are ignored if selected }
    \AttributeTok{set\_a =} \FunctionTok{str\_c}\NormalTok{(TRI\_allow,REF\_POINT, }\AttributeTok{sep=}\StringTok{""}\NormalTok{),}
    \AttributeTok{set\_a =} \FunctionTok{str\_replace\_na}\NormalTok{(set\_a,}\StringTok{""}\NormalTok{),}\CommentTok{\#replace na if empty}
    \AttributeTok{n\_a =} \FunctionTok{nchar}\NormalTok{(set\_a),}
    
    \CommentTok{\#N store all answr options (superset)}
    \AttributeTok{set\_n =}\NormalTok{ OPTIONS,  }
    \AttributeTok{n\_n =} \FunctionTok{nchar}\NormalTok{(set\_n)}
  
\NormalTok{) }\SpecialCharTok{\%\textgreater{}\%} \FunctionTok{select}\NormalTok{(Q, condition, set\_n, set\_p, set\_a, n\_n, n\_p, n\_a)}

\CommentTok{\#2. DEFINE SETS N, P, A}
\ControlFlowTok{for}\NormalTok{ (x }\ControlFlowTok{in} \DecValTok{1}\SpecialCharTok{:}\FunctionTok{nrow}\NormalTok{(keys\_tri)) \{}
  
  \CommentTok{\#UNWIND STRINGS FOR SETDIFF}
  \CommentTok{\#n all answer options}
\NormalTok{  N }\OtherTok{=}\NormalTok{ keys\_tri[x,}\StringTok{\textquotesingle{}set\_n\textquotesingle{}}\NormalTok{] }\SpecialCharTok{\%\textgreater{}\%} \FunctionTok{pull}\NormalTok{(set\_n) }\SpecialCharTok{\%\textgreater{}\%} \FunctionTok{strsplit}\NormalTok{(}\StringTok{""}\NormalTok{) }\SpecialCharTok{\%\textgreater{}\%} \FunctionTok{unlist}\NormalTok{()}
  \CommentTok{\#p correct{-}select answer options}
\NormalTok{  P }\OtherTok{=}\NormalTok{ keys\_tri[x,}\StringTok{\textquotesingle{}set\_p\textquotesingle{}}\NormalTok{] }\SpecialCharTok{\%\textgreater{}\%} \FunctionTok{pull}\NormalTok{(set\_p) }\SpecialCharTok{\%\textgreater{}\%} \FunctionTok{strsplit}\NormalTok{(}\StringTok{""}\NormalTok{) }\SpecialCharTok{\%\textgreater{}\%} \FunctionTok{unlist}\NormalTok{()}
  \CommentTok{\#a ignore{-}select answer options (should not be selected, but if they are, don\textquotesingle{}t penalize)}
\NormalTok{  A }\OtherTok{=}\NormalTok{ keys\_tri[x,}\StringTok{\textquotesingle{}set\_a\textquotesingle{}}\NormalTok{] }\SpecialCharTok{\%\textgreater{}\%} \FunctionTok{pull}\NormalTok{(set\_a) }\SpecialCharTok{\%\textgreater{}\%} \FunctionTok{strsplit}\NormalTok{(}\StringTok{""}\NormalTok{) }\SpecialCharTok{\%\textgreater{}\%} \FunctionTok{unlist}\NormalTok{()}
  
  \CommentTok{\#Q = N {-} (P+A)}
  \CommentTok{\#answers that are penalized (at {-}1/q) if selected }
\NormalTok{  s }\OtherTok{=} \FunctionTok{union}\NormalTok{(P,A) }\CommentTok{\#rewarded plus ignored }
\NormalTok{  s }\OtherTok{=} \FunctionTok{str\_replace\_na}\NormalTok{(s,}\StringTok{""}\NormalTok{)}
  \CommentTok{\# s = union(s,X) \# + trapdoor }
\NormalTok{  Q }\OtherTok{=} \FunctionTok{setdiff}\NormalTok{(N,s) }\CommentTok{\# = penalized at {-}1/q when selected }
  
  \CommentTok{\#save set to dataframe}
\NormalTok{  Q }\OtherTok{=} \FunctionTok{str\_c}\NormalTok{(Q, }\AttributeTok{collapse=}\StringTok{""}\NormalTok{)}
\NormalTok{  n\_q }\OtherTok{=} \FunctionTok{nchar}\NormalTok{(Q)}
\NormalTok{  keys\_tri[x,}\StringTok{\textquotesingle{}set\_q\textquotesingle{}}\NormalTok{] }\OtherTok{=}\NormalTok{ Q}
\NormalTok{  keys\_tri[x,}\StringTok{\textquotesingle{}n\_q\textquotesingle{}}\NormalTok{] }\OtherTok{=}\NormalTok{ n\_q}
  
  \CommentTok{\#verify each element in N is included in one and only one of P,A,Q}
\NormalTok{  tempunion }\OtherTok{=} \FunctionTok{union}\NormalTok{(s,Q) }\SpecialCharTok{\%\textgreater{}\%} \FunctionTok{str\_c}\NormalTok{(}\AttributeTok{collapse=}\StringTok{""}\NormalTok{) }\SpecialCharTok{\%\textgreater{}\%} \FunctionTok{strsplit}\NormalTok{(}\StringTok{""}\NormalTok{) }\SpecialCharTok{\%\textgreater{}\%} \FunctionTok{unlist}\NormalTok{()}
\NormalTok{  N }\OtherTok{=}\NormalTok{ N }\SpecialCharTok{\%\textgreater{}\%} \FunctionTok{str\_c}\NormalTok{(}\AttributeTok{collapse=}\StringTok{""}\NormalTok{) }\SpecialCharTok{\%\textgreater{}\%} \FunctionTok{strsplit}\NormalTok{(}\StringTok{""}\NormalTok{) }\SpecialCharTok{\%\textgreater{}\%} \FunctionTok{unlist}\NormalTok{()}
\NormalTok{  verify\_tri[x] }\OtherTok{=} \FunctionTok{setequal}\NormalTok{(tempunion,N)}
\NormalTok{\}}

\CommentTok{\#3. reorder cols for ease of use}
\NormalTok{keys\_tri }\OtherTok{\textless{}{-}}\NormalTok{ keys\_tri }\SpecialCharTok{\%\textgreater{}\%} \FunctionTok{select}\NormalTok{(Q, condition, set\_n, set\_p, set\_a, set\_q, n\_n, n\_p, n\_a, n\_q) }\SpecialCharTok{\%\textgreater{}\%} \FunctionTok{mutate}\NormalTok{(}\AttributeTok{verify =}\NormalTok{ n\_p }\SpecialCharTok{+}\NormalTok{ n\_a }\SpecialCharTok{+}\NormalTok{ n\_q)}

\CommentTok{\#4. replace condition 111 with "general" to accomodate other conditions [only 121 is special]}
\NormalTok{keys\_tri }\OtherTok{\textless{}{-}}\NormalTok{ keys\_tri }\SpecialCharTok{\%\textgreater{}\%} \FunctionTok{mutate}\NormalTok{(}
  \AttributeTok{condition =} \FunctionTok{replace}\NormalTok{(condition, condition }\SpecialCharTok{!=} \StringTok{"121"}\NormalTok{, }\StringTok{"DEFAULT"}\NormalTok{)}
\NormalTok{)}

\CommentTok{\#cleanup }
\FunctionTok{rm}\NormalTok{(N,A,P,Q,n\_q,s,x,tempunion)}
\end{Highlighting}
\end{Shaded}

This leaves us a dataframe \texttt{keys\_tri} that define the sets of
response options consistent with the triangular graph interpretation.

To verify we have generated the correct sets, we verify that for each
question, each option in N is included in either set P, A or Q (once and
only once).

TRUE, TRUE, TRUE, TRUE, TRUE, TRUE, TRUE, TRUE, TRUE, TRUE, TRUE, TRUE,
TRUE, TRUE, TRUE, TRUE, TRUE, TRUE, TRUE, TRUE

\hypertarget{orthogonal-key}{%
\subsubsection*{Orthogonal Key}\label{orthogonal-key}}
\addcontentsline{toc}{subsubsection}{Orthogonal Key}

Next we construct a key set based on the `Orthogonal' interpretation.

\begin{Shaded}
\begin{Highlighting}[]
\NormalTok{verify\_orth }\OtherTok{=} \FunctionTok{c}\NormalTok{() }\CommentTok{\#sanity check}
\DocumentationTok{\#\#——————————————————————————————————————————————————————————————————————}
\DocumentationTok{\#\#CONSTRUCT ORTHOGONAL KEY SET}
\DocumentationTok{\#\#——————————————————————————————————————————————————————————————————————}
\CommentTok{\#1. DEFINE SETS N, P, A}
\NormalTok{keys\_orth }\OtherTok{\textless{}{-}}\NormalTok{ keys\_raw }\SpecialCharTok{\%\textgreater{}\%} 
  \FunctionTok{select}\NormalTok{(Q, condition, OPTIONS, ORTHOGONAL, ORTH\_allow, REF\_POINT) }\SpecialCharTok{\%\textgreater{}\%} 
  \FunctionTok{mutate}\NormalTok{(}
    
    \CommentTok{\#replace NAs }
    \AttributeTok{ORTH\_allow =} \FunctionTok{str\_replace\_na}\NormalTok{(ORTH\_allow,}\StringTok{""}\NormalTok{),}
    \AttributeTok{REF\_POINT =} \FunctionTok{str\_replace\_na}\NormalTok{(REF\_POINT,}\StringTok{""}\NormalTok{),}
    
    \CommentTok{\#P options that SHOULD be selected (rewarded)}
    \AttributeTok{set\_p =}\NormalTok{ ORTHOGONAL,}
    \AttributeTok{set\_p =} \FunctionTok{str\_replace\_na}\NormalTok{(set\_p,}\StringTok{""}\NormalTok{),}\CommentTok{\#replace na if empty}
    \AttributeTok{n\_p =} \FunctionTok{nchar}\NormalTok{(set\_p), }\CommentTok{\#number of true{-}select options}
    
    \CommentTok{\#A options that are ignored if selected }
    \AttributeTok{set\_a =} \FunctionTok{str\_c}\NormalTok{(ORTH\_allow,REF\_POINT, }\AttributeTok{sep=}\StringTok{""}\NormalTok{),}
    \AttributeTok{set\_a =} \FunctionTok{str\_replace\_na}\NormalTok{(set\_a,}\StringTok{""}\NormalTok{), }\CommentTok{\#replace na if empty}
    \AttributeTok{n\_a =} \FunctionTok{nchar}\NormalTok{(set\_a),}
    
    \CommentTok{\#N store all answer options (superset)}
    \AttributeTok{set\_n =}\NormalTok{ OPTIONS,  }
    \AttributeTok{n\_n =} \FunctionTok{nchar}\NormalTok{(set\_n)}
  
\NormalTok{) }\SpecialCharTok{\%\textgreater{}\%} \FunctionTok{select}\NormalTok{(Q, condition, set\_n, set\_p, set\_a, n\_n, n\_p, n\_a)}

\CommentTok{\#2. DO THE STUFF THAT\textquotesingle{}S EASIER IN A LOOP}
\ControlFlowTok{for}\NormalTok{ (x }\ControlFlowTok{in} \DecValTok{1}\SpecialCharTok{:}\FunctionTok{nrow}\NormalTok{(keys\_orth)) \{}
  
  \CommentTok{\#UNWIND STRINGS FOR SETDIFF}
  \CommentTok{\#n all answer options}
\NormalTok{  N }\OtherTok{=}\NormalTok{ keys\_orth[x,}\StringTok{\textquotesingle{}set\_n\textquotesingle{}}\NormalTok{] }\SpecialCharTok{\%\textgreater{}\%} \FunctionTok{pull}\NormalTok{(set\_n) }\SpecialCharTok{\%\textgreater{}\%} \FunctionTok{strsplit}\NormalTok{(}\StringTok{""}\NormalTok{) }\SpecialCharTok{\%\textgreater{}\%} \FunctionTok{unlist}\NormalTok{()}
  \CommentTok{\#p correct{-}select answer options}
\NormalTok{  P }\OtherTok{=}\NormalTok{ keys\_orth[x,}\StringTok{\textquotesingle{}set\_p\textquotesingle{}}\NormalTok{] }\SpecialCharTok{\%\textgreater{}\%} \FunctionTok{pull}\NormalTok{(set\_p) }\SpecialCharTok{\%\textgreater{}\%} \FunctionTok{strsplit}\NormalTok{(}\StringTok{""}\NormalTok{) }\SpecialCharTok{\%\textgreater{}\%} \FunctionTok{unlist}\NormalTok{()}
  \CommentTok{\#a ignore{-}select answer options (should not be selected, but if they are, don\textquotesingle{}t penalize)}
\NormalTok{  A }\OtherTok{=}\NormalTok{ keys\_orth[x,}\StringTok{\textquotesingle{}set\_a\textquotesingle{}}\NormalTok{] }\SpecialCharTok{\%\textgreater{}\%} \FunctionTok{pull}\NormalTok{(set\_a) }\SpecialCharTok{\%\textgreater{}\%} \FunctionTok{strsplit}\NormalTok{(}\StringTok{""}\NormalTok{) }\SpecialCharTok{\%\textgreater{}\%} \FunctionTok{unlist}\NormalTok{() }
 
  \CommentTok{\#Q = N {-} (P+A)}
  \CommentTok{\#answers that are penalized (at {-}1/q) if selected }
\NormalTok{  s }\OtherTok{=} \FunctionTok{union}\NormalTok{(P,A) }\CommentTok{\#rewarded plus ignored }
\NormalTok{  s }\OtherTok{=} \FunctionTok{str\_replace\_na}\NormalTok{(s,}\StringTok{""}\NormalTok{)}
  \CommentTok{\# print(s)}
  \CommentTok{\# s = union(s,X) \# + trapdoor }
\NormalTok{  Q }\OtherTok{=} \FunctionTok{setdiff}\NormalTok{(N,s) }\CommentTok{\# = penalized at {-}1/q when selected }
  \CommentTok{\#save set to dataframe}
\NormalTok{  Q }\OtherTok{=} \FunctionTok{str\_c}\NormalTok{(Q, }\AttributeTok{collapse=}\StringTok{""}\NormalTok{)}
\NormalTok{  n\_q }\OtherTok{=} \FunctionTok{nchar}\NormalTok{(Q)}
\NormalTok{  keys\_orth[x,}\StringTok{\textquotesingle{}set\_q\textquotesingle{}}\NormalTok{] }\OtherTok{=}\NormalTok{ Q}
\NormalTok{  keys\_orth[x,}\StringTok{\textquotesingle{}n\_q\textquotesingle{}}\NormalTok{] }\OtherTok{=}\NormalTok{ n\_q}
  
  \CommentTok{\#verify each element in N is included in one and only one of P,A,Q}
\NormalTok{  tempunion }\OtherTok{=} \FunctionTok{union}\NormalTok{(s,Q) }\SpecialCharTok{\%\textgreater{}\%} \FunctionTok{str\_c}\NormalTok{(}\AttributeTok{collapse=}\StringTok{""}\NormalTok{) }\SpecialCharTok{\%\textgreater{}\%} \FunctionTok{strsplit}\NormalTok{(}\StringTok{""}\NormalTok{) }\SpecialCharTok{\%\textgreater{}\%} \FunctionTok{unlist}\NormalTok{()}
  \CommentTok{\# print(tempunion)}
\NormalTok{  N }\OtherTok{=}\NormalTok{ N }\SpecialCharTok{\%\textgreater{}\%} \FunctionTok{str\_c}\NormalTok{(}\AttributeTok{collapse=}\StringTok{""}\NormalTok{) }\SpecialCharTok{\%\textgreater{}\%} \FunctionTok{strsplit}\NormalTok{(}\StringTok{""}\NormalTok{) }\SpecialCharTok{\%\textgreater{}\%} \FunctionTok{unlist}\NormalTok{()}

\NormalTok{  verify\_orth[x] }\OtherTok{=} \FunctionTok{setequal}\NormalTok{(tempunion,N)}
\NormalTok{\}}

\CommentTok{\#3. reorder cols for ease of use}
\NormalTok{keys\_orth }\OtherTok{\textless{}{-}}\NormalTok{ keys\_orth }\SpecialCharTok{\%\textgreater{}\%} \FunctionTok{select}\NormalTok{(Q, condition, set\_n, set\_p, set\_a, set\_q, n\_n, n\_p, n\_a, n\_q) }\SpecialCharTok{\%\textgreater{}\%} \FunctionTok{mutate}\NormalTok{(}\AttributeTok{verify =}\NormalTok{ n\_p }\SpecialCharTok{+}\NormalTok{ n\_a }\SpecialCharTok{+}\NormalTok{ n\_q)}

\CommentTok{\#4. replace condition 111 with "general" to accomodate other conditions [only 121 is special]}
\NormalTok{keys\_orth }\OtherTok{\textless{}{-}}\NormalTok{ keys\_orth }\SpecialCharTok{\%\textgreater{}\%} \FunctionTok{mutate}\NormalTok{(}
  \AttributeTok{condition =} \FunctionTok{replace}\NormalTok{(condition, condition }\SpecialCharTok{!=} \StringTok{"121"}\NormalTok{, }\StringTok{"DEFAULT"}\NormalTok{)}
\NormalTok{)}

\CommentTok{\#cleanup}
\FunctionTok{rm}\NormalTok{(A, N, n\_q, P, Q, s, tempunion, x, cs)}
\end{Highlighting}
\end{Shaded}

This leaves us a dataframe \texttt{keys\_orth} that define the sets of
response options consistent with the orthogonal graph interpretation.

To verify we have generated the correct sets, we verify that for each
question, each response in N is included in either set P, A or Q (once
and only once).

TRUE, TRUE, TRUE, TRUE, TRUE, TRUE, TRUE, TRUE, TRUE, TRUE, TRUE, TRUE,
TRUE, TRUE, TRUE, TRUE, TRUE, TRUE, TRUE, TRUE

\hypertarget{tversky-keys}{%
\subsubsection*{Tversky Keys}\label{tversky-keys}}
\addcontentsline{toc}{subsubsection}{Tversky Keys}

Next we construct the key set based on a partial-understanding strategy
we refer to as `Tversky'. We use the label Tversky as shorthand for a
partial interpretation of the coordinate system where subjects select a
set of responses that lay along a connecting line from the referenced
data point or referenced time for that item. The term is named for
Barbara Tversky based on her work on graphical primitives (e.g.~``lines
connect, arrows direct, boxes contain'').

\begin{Shaded}
\begin{Highlighting}[]
\NormalTok{verify\_max }\OtherTok{=} \FunctionTok{c}\NormalTok{() }\CommentTok{\#sanity check}
\DocumentationTok{\#\#——————————————————————————————————————————————————————————————————————}
\DocumentationTok{\#\#CONSTRUCT TVERSKY KEY SET for TVERSKY{-}MAX}
\DocumentationTok{\#\#——————————————————————————————————————————————————————————————————————}
\CommentTok{\#1. DEFINE SETS N, P, A}
\NormalTok{keys\_tversky\_max }\OtherTok{\textless{}{-}}\NormalTok{ keys\_raw }\SpecialCharTok{\%\textgreater{}\%} 
  \FunctionTok{select}\NormalTok{(Q, condition, OPTIONS, REF\_POINT, TV\_max, TV\_max\_allow) }\SpecialCharTok{\%\textgreater{}\%} 
  \FunctionTok{mutate}\NormalTok{(}
  
    \CommentTok{\#replace NAs }
    \AttributeTok{REF\_POINT =} \FunctionTok{str\_replace\_na}\NormalTok{(REF\_POINT,}\StringTok{""}\NormalTok{),}
    \AttributeTok{TV\_max =} \FunctionTok{str\_replace\_na}\NormalTok{(TV\_max,}\StringTok{""}\NormalTok{),}
    \AttributeTok{TV\_max\_allow =} \FunctionTok{str\_replace\_na}\NormalTok{(TV\_max\_allow,}\StringTok{""}\NormalTok{),}
    
    \CommentTok{\#P options that SHOULD be selected (rewarded)}
    \AttributeTok{set\_p =}\NormalTok{ TV\_max,}
    \AttributeTok{set\_p =} \FunctionTok{str\_replace\_na}\NormalTok{(set\_p,}\StringTok{""}\NormalTok{), }\CommentTok{\#replace na if empty}
    \AttributeTok{n\_p =} \FunctionTok{nchar}\NormalTok{(set\_p), }\CommentTok{\#number of true{-}select options}
    
    \CommentTok{\#A options that are ignored if selected }
    \AttributeTok{set\_a =} \FunctionTok{str\_c}\NormalTok{(TV\_max\_allow,REF\_POINT, }\AttributeTok{sep=}\StringTok{""}\NormalTok{),}
    \AttributeTok{set\_a =} \FunctionTok{str\_replace\_na}\NormalTok{(set\_a,}\StringTok{""}\NormalTok{), }\CommentTok{\#replace na if empty}
    \AttributeTok{n\_a =} \FunctionTok{nchar}\NormalTok{(set\_a),}
    
    \CommentTok{\#N store all answr options (superset)}
    \AttributeTok{set\_n =}\NormalTok{ OPTIONS,  }
    \AttributeTok{n\_n =} \FunctionTok{nchar}\NormalTok{(set\_n)}
  
\NormalTok{) }\SpecialCharTok{\%\textgreater{}\%} \FunctionTok{select}\NormalTok{(Q, condition, set\_n, set\_p, set\_a, n\_n, n\_p, n\_a)}

\CommentTok{\#2. DO THE STUFF THAT\textquotesingle{}S EASIER IN A LOOP}
\ControlFlowTok{for}\NormalTok{ (x }\ControlFlowTok{in} \DecValTok{1}\SpecialCharTok{:}\FunctionTok{nrow}\NormalTok{(keys\_tversky\_max)) \{}
  
  \CommentTok{\#UNWIND STRINGS FOR SETDIFF}
  \CommentTok{\#n all answer options}
\NormalTok{  N }\OtherTok{=}\NormalTok{ keys\_tversky\_max[x,}\StringTok{\textquotesingle{}set\_n\textquotesingle{}}\NormalTok{] }\SpecialCharTok{\%\textgreater{}\%} \FunctionTok{pull}\NormalTok{(set\_n) }\SpecialCharTok{\%\textgreater{}\%} \FunctionTok{strsplit}\NormalTok{(}\StringTok{""}\NormalTok{) }\SpecialCharTok{\%\textgreater{}\%} \FunctionTok{unlist}\NormalTok{()}
  \CommentTok{\#p correct{-}select answer options}
\NormalTok{  P }\OtherTok{=}\NormalTok{ keys\_tversky\_max[x,}\StringTok{\textquotesingle{}set\_p\textquotesingle{}}\NormalTok{] }\SpecialCharTok{\%\textgreater{}\%} \FunctionTok{pull}\NormalTok{(set\_p) }\SpecialCharTok{\%\textgreater{}\%} \FunctionTok{strsplit}\NormalTok{(}\StringTok{""}\NormalTok{) }\SpecialCharTok{\%\textgreater{}\%} \FunctionTok{unlist}\NormalTok{()}
  \CommentTok{\#a ignore{-}select answer options (should not be selected, but if they are, don\textquotesingle{}t penalize)}
\NormalTok{  A }\OtherTok{=}\NormalTok{ keys\_tversky\_max[x,}\StringTok{\textquotesingle{}set\_a\textquotesingle{}}\NormalTok{] }\SpecialCharTok{\%\textgreater{}\%} \FunctionTok{pull}\NormalTok{(set\_a) }\SpecialCharTok{\%\textgreater{}\%} \FunctionTok{strsplit}\NormalTok{(}\StringTok{""}\NormalTok{) }\SpecialCharTok{\%\textgreater{}\%} \FunctionTok{unlist}\NormalTok{()}
  
  \CommentTok{\#Q = N {-} (P+A)}
  \CommentTok{\#answers that are penalized (at {-}1/q) if selected }
\NormalTok{  s }\OtherTok{=} \FunctionTok{union}\NormalTok{(P,A) }\CommentTok{\#rewarded plus ignored }
\NormalTok{  s }\OtherTok{=} \FunctionTok{str\_replace\_na}\NormalTok{(s,}\StringTok{""}\NormalTok{)}
  \CommentTok{\# s = union(s,X) \# + trapdoor }
\NormalTok{  Q }\OtherTok{=} \FunctionTok{setdiff}\NormalTok{(N,s) }\CommentTok{\# = penalized at {-}1/q when selected }
  
  \CommentTok{\#save set to dataframe}
\NormalTok{  Q }\OtherTok{=} \FunctionTok{str\_c}\NormalTok{(Q, }\AttributeTok{collapse=}\StringTok{""}\NormalTok{)}
\NormalTok{  n\_q }\OtherTok{=} \FunctionTok{nchar}\NormalTok{(Q)}
\NormalTok{  keys\_tversky\_max[x,}\StringTok{\textquotesingle{}set\_q\textquotesingle{}}\NormalTok{] }\OtherTok{=}\NormalTok{ Q}
\NormalTok{  keys\_tversky\_max[x,}\StringTok{\textquotesingle{}n\_q\textquotesingle{}}\NormalTok{] }\OtherTok{=}\NormalTok{ n\_q}
  
  \CommentTok{\#verify each element in N is included in one and only one of P,A,Q}
\NormalTok{  tempunion }\OtherTok{=} \FunctionTok{union}\NormalTok{(s,Q) }\SpecialCharTok{\%\textgreater{}\%} \FunctionTok{str\_c}\NormalTok{(}\AttributeTok{collapse=}\StringTok{""}\NormalTok{) }\SpecialCharTok{\%\textgreater{}\%} \FunctionTok{strsplit}\NormalTok{(}\StringTok{""}\NormalTok{) }\SpecialCharTok{\%\textgreater{}\%} \FunctionTok{unlist}\NormalTok{()}
\NormalTok{  N }\OtherTok{=}\NormalTok{ N }\SpecialCharTok{\%\textgreater{}\%} \FunctionTok{str\_c}\NormalTok{(}\AttributeTok{collapse=}\StringTok{""}\NormalTok{) }\SpecialCharTok{\%\textgreater{}\%} \FunctionTok{strsplit}\NormalTok{(}\StringTok{""}\NormalTok{) }\SpecialCharTok{\%\textgreater{}\%} \FunctionTok{unlist}\NormalTok{()}
\NormalTok{  verify\_max[x] }\OtherTok{=} \FunctionTok{setequal}\NormalTok{(tempunion,N)}
\NormalTok{\}}

\CommentTok{\#3. reorder cols for ease of use}
\NormalTok{keys\_tversky\_max }\OtherTok{\textless{}{-}}\NormalTok{ keys\_tversky\_max }\SpecialCharTok{\%\textgreater{}\%} \FunctionTok{select}\NormalTok{(Q, condition, set\_n, set\_p, set\_a, set\_q, n\_n, n\_p, n\_a, n\_q) }\SpecialCharTok{\%\textgreater{}\%} \FunctionTok{mutate}\NormalTok{(}\AttributeTok{verify =}\NormalTok{ n\_p }\SpecialCharTok{+}\NormalTok{ n\_a }\SpecialCharTok{+}\NormalTok{ n\_q)}

\CommentTok{\#4. replace condition 111 with "general" to accomodate other conditions [only 121 is special]}
\NormalTok{keys\_tversky\_max }\OtherTok{\textless{}{-}}\NormalTok{ keys\_tversky\_max }\SpecialCharTok{\%\textgreater{}\%} \FunctionTok{mutate}\NormalTok{(}
  \AttributeTok{condition =} \FunctionTok{replace}\NormalTok{(condition, condition }\SpecialCharTok{!=} \StringTok{"121"}\NormalTok{, }\StringTok{"DEFAULT"}\NormalTok{)}
\NormalTok{)}

\NormalTok{verify\_tversky\_start }\OtherTok{=} \FunctionTok{c}\NormalTok{() }\CommentTok{\#sanity check}
\DocumentationTok{\#\#——————————————————————————————————————————————————————————————————————}
\DocumentationTok{\#\#CONSTRUCT TVERSKY KEY SET for TVERSKY{-}START}
\DocumentationTok{\#\#——————————————————————————————————————————————————————————————————————}
\CommentTok{\#1. DEFINE SETS N, P, A}
\NormalTok{keys\_tversky\_start }\OtherTok{\textless{}{-}}\NormalTok{ keys\_raw }\SpecialCharTok{\%\textgreater{}\%} 
  \FunctionTok{select}\NormalTok{(Q, condition, OPTIONS, REF\_POINT, TV\_start, TV\_start\_allow) }\SpecialCharTok{\%\textgreater{}\%} 
  \FunctionTok{mutate}\NormalTok{(}
  
    \CommentTok{\#replace NAs }
    \AttributeTok{REF\_POINT =} \FunctionTok{str\_replace\_na}\NormalTok{(REF\_POINT,}\StringTok{""}\NormalTok{),}
    \AttributeTok{TV\_start =} \FunctionTok{str\_replace\_na}\NormalTok{(TV\_start,}\StringTok{""}\NormalTok{),}
    \AttributeTok{TV\_start\_allow =} \FunctionTok{str\_replace\_na}\NormalTok{(TV\_start\_allow,}\StringTok{""}\NormalTok{),}
    
    \CommentTok{\#P options that SHOULD be selected (rewarded)}
    \AttributeTok{set\_p =}\NormalTok{ TV\_start,}
    \AttributeTok{set\_p =} \FunctionTok{str\_replace\_na}\NormalTok{(set\_p,}\StringTok{""}\NormalTok{), }\CommentTok{\#replace na if empty}
    \AttributeTok{n\_p =} \FunctionTok{nchar}\NormalTok{(set\_p), }\CommentTok{\#number of true{-}select options}
    
    \CommentTok{\#A options that are ignored if selected }
    \AttributeTok{set\_a =} \FunctionTok{str\_c}\NormalTok{(TV\_start\_allow,REF\_POINT, }\AttributeTok{sep=}\StringTok{""}\NormalTok{),}
    \AttributeTok{set\_a =} \FunctionTok{str\_replace\_na}\NormalTok{(set\_a,}\StringTok{""}\NormalTok{), }\CommentTok{\#replace na if empty}
    \AttributeTok{n\_a =} \FunctionTok{nchar}\NormalTok{(set\_a),}
    
    \CommentTok{\#N store all answr options (superset)}
    \AttributeTok{set\_n =}\NormalTok{ OPTIONS,  }
    \AttributeTok{n\_n =} \FunctionTok{nchar}\NormalTok{(set\_n)}
  
\NormalTok{) }\SpecialCharTok{\%\textgreater{}\%} \FunctionTok{select}\NormalTok{(Q, condition, set\_n, set\_p, set\_a, n\_n, n\_p, n\_a)}

\CommentTok{\#2. DO THE STUFF THAT\textquotesingle{}S EASIER IN A LOOP}
\ControlFlowTok{for}\NormalTok{ (x }\ControlFlowTok{in} \DecValTok{1}\SpecialCharTok{:}\FunctionTok{nrow}\NormalTok{(keys\_tversky\_start)) \{}
  
  \CommentTok{\#UNWIND STRINGS FOR SETDIFF}
  \CommentTok{\#n all answer options}
\NormalTok{  N }\OtherTok{=}\NormalTok{ keys\_tversky\_start[x,}\StringTok{\textquotesingle{}set\_n\textquotesingle{}}\NormalTok{] }\SpecialCharTok{\%\textgreater{}\%} \FunctionTok{pull}\NormalTok{(set\_n) }\SpecialCharTok{\%\textgreater{}\%} \FunctionTok{strsplit}\NormalTok{(}\StringTok{""}\NormalTok{) }\SpecialCharTok{\%\textgreater{}\%} \FunctionTok{unlist}\NormalTok{()}
  \CommentTok{\#p correct{-}select answer options}
\NormalTok{  P }\OtherTok{=}\NormalTok{ keys\_tversky\_start[x,}\StringTok{\textquotesingle{}set\_p\textquotesingle{}}\NormalTok{] }\SpecialCharTok{\%\textgreater{}\%} \FunctionTok{pull}\NormalTok{(set\_p) }\SpecialCharTok{\%\textgreater{}\%} \FunctionTok{strsplit}\NormalTok{(}\StringTok{""}\NormalTok{) }\SpecialCharTok{\%\textgreater{}\%} \FunctionTok{unlist}\NormalTok{()}
  \CommentTok{\#a ignore{-}select answer options (should not be selected, but if they are, don\textquotesingle{}t penalize)}
\NormalTok{  A }\OtherTok{=}\NormalTok{ keys\_tversky\_start[x,}\StringTok{\textquotesingle{}set\_a\textquotesingle{}}\NormalTok{] }\SpecialCharTok{\%\textgreater{}\%} \FunctionTok{pull}\NormalTok{(set\_a) }\SpecialCharTok{\%\textgreater{}\%} \FunctionTok{strsplit}\NormalTok{(}\StringTok{""}\NormalTok{) }\SpecialCharTok{\%\textgreater{}\%} \FunctionTok{unlist}\NormalTok{()}
  
  \CommentTok{\#Q = N {-} (P+A)}
  \CommentTok{\#answers that are penalized (at {-}1/q) if selected }
\NormalTok{  s }\OtherTok{=} \FunctionTok{union}\NormalTok{(P,A) }\CommentTok{\#rewarded plus ignored }
\NormalTok{  s }\OtherTok{=} \FunctionTok{str\_replace\_na}\NormalTok{(s,}\StringTok{""}\NormalTok{)}
  \CommentTok{\# s = union(s,X) \# + trapdoor }
\NormalTok{  Q }\OtherTok{=} \FunctionTok{setdiff}\NormalTok{(N,s) }\CommentTok{\# = penalized at {-}1/q when selected }
  
  \CommentTok{\#save set to dataframe}
\NormalTok{  Q }\OtherTok{=} \FunctionTok{str\_c}\NormalTok{(Q, }\AttributeTok{collapse=}\StringTok{""}\NormalTok{)}
\NormalTok{  n\_q }\OtherTok{=} \FunctionTok{nchar}\NormalTok{(Q)}
\NormalTok{  keys\_tversky\_start[x,}\StringTok{\textquotesingle{}set\_q\textquotesingle{}}\NormalTok{] }\OtherTok{=}\NormalTok{ Q}
\NormalTok{  keys\_tversky\_start[x,}\StringTok{\textquotesingle{}n\_q\textquotesingle{}}\NormalTok{] }\OtherTok{=}\NormalTok{ n\_q}
  
  \CommentTok{\#verify each element in N is included in one and only one of P,A,Q}
\NormalTok{  tempunion }\OtherTok{=} \FunctionTok{union}\NormalTok{(s,Q) }\SpecialCharTok{\%\textgreater{}\%} \FunctionTok{str\_c}\NormalTok{(}\AttributeTok{collapse=}\StringTok{""}\NormalTok{) }\SpecialCharTok{\%\textgreater{}\%} \FunctionTok{strsplit}\NormalTok{(}\StringTok{""}\NormalTok{) }\SpecialCharTok{\%\textgreater{}\%} \FunctionTok{unlist}\NormalTok{()}
\NormalTok{  N }\OtherTok{=}\NormalTok{ N }\SpecialCharTok{\%\textgreater{}\%} \FunctionTok{str\_c}\NormalTok{(}\AttributeTok{collapse=}\StringTok{""}\NormalTok{) }\SpecialCharTok{\%\textgreater{}\%} \FunctionTok{strsplit}\NormalTok{(}\StringTok{""}\NormalTok{) }\SpecialCharTok{\%\textgreater{}\%} \FunctionTok{unlist}\NormalTok{()}
\NormalTok{  verify\_tversky\_start[x] }\OtherTok{=} \FunctionTok{setequal}\NormalTok{(tempunion,N)}
\NormalTok{\}}

\CommentTok{\#3. reorder cols for ease of use}
\NormalTok{keys\_tversky\_start }\OtherTok{\textless{}{-}}\NormalTok{ keys\_tversky\_start }\SpecialCharTok{\%\textgreater{}\%} \FunctionTok{select}\NormalTok{(Q, condition, set\_n, set\_p, set\_a, set\_q, n\_n, n\_p, n\_a, n\_q) }\SpecialCharTok{\%\textgreater{}\%} \FunctionTok{mutate}\NormalTok{(}\AttributeTok{verify =}\NormalTok{ n\_p }\SpecialCharTok{+}\NormalTok{ n\_a }\SpecialCharTok{+}\NormalTok{ n\_q)}

\CommentTok{\#4. replace condition 111 with "general" to accomodate other conditions [only 121 is special]}
\NormalTok{keys\_tversky\_start }\OtherTok{\textless{}{-}}\NormalTok{ keys\_tversky\_start }\SpecialCharTok{\%\textgreater{}\%} \FunctionTok{mutate}\NormalTok{(}
  \AttributeTok{condition =} \FunctionTok{replace}\NormalTok{(condition, condition }\SpecialCharTok{!=} \StringTok{"121"}\NormalTok{, }\StringTok{"DEFAULT"}\NormalTok{)}
\NormalTok{)}

\NormalTok{verify\_tversky\_end }\OtherTok{=} \FunctionTok{c}\NormalTok{() }\CommentTok{\#sanity check}
\DocumentationTok{\#\#——————————————————————————————————————————————————————————————————————}
\DocumentationTok{\#\#CONSTRUCT TVERSKY KEY SET for TVERSKY{-}}\RegionMarkerTok{END}
\DocumentationTok{\#\#——————————————————————————————————————————————————————————————————————}
\CommentTok{\#1. DEFINE SETS N, P, A}
\NormalTok{keys\_tversky\_end }\OtherTok{\textless{}{-}}\NormalTok{ keys\_raw }\SpecialCharTok{\%\textgreater{}\%} 
  \FunctionTok{select}\NormalTok{(Q, condition, OPTIONS, REF\_POINT, TV\_end, TV\_end\_allow) }\SpecialCharTok{\%\textgreater{}\%} 
  \FunctionTok{mutate}\NormalTok{(}
  
    \CommentTok{\#replace NAs }
    \AttributeTok{REF\_POINT =} \FunctionTok{str\_replace\_na}\NormalTok{(REF\_POINT,}\StringTok{""}\NormalTok{),}
    \AttributeTok{TV\_end =} \FunctionTok{str\_replace\_na}\NormalTok{(TV\_end,}\StringTok{""}\NormalTok{),}
    \AttributeTok{TV\_end\_allow =} \FunctionTok{str\_replace\_na}\NormalTok{(TV\_end\_allow,}\StringTok{""}\NormalTok{),}
    
    \CommentTok{\#P options that SHOULD be selected (rewarded)}
    \AttributeTok{set\_p =}\NormalTok{ TV\_end,}
    \AttributeTok{set\_p =} \FunctionTok{str\_replace\_na}\NormalTok{(set\_p,}\StringTok{""}\NormalTok{), }\CommentTok{\#replace na if empty}
    \AttributeTok{n\_p =} \FunctionTok{nchar}\NormalTok{(set\_p), }\CommentTok{\#number of true{-}select options}
    
    \CommentTok{\#A options that are ignored if selected }
    \AttributeTok{set\_a =} \FunctionTok{str\_c}\NormalTok{(TV\_end\_allow,REF\_POINT, }\AttributeTok{sep=}\StringTok{""}\NormalTok{),}
    \AttributeTok{set\_a =} \FunctionTok{str\_replace\_na}\NormalTok{(set\_a,}\StringTok{""}\NormalTok{), }\CommentTok{\#replace na if empty}
    \AttributeTok{n\_a =} \FunctionTok{nchar}\NormalTok{(set\_a),}
    
    \CommentTok{\#N store all answr options (superset)}
    \AttributeTok{set\_n =}\NormalTok{ OPTIONS,  }
    \AttributeTok{n\_n =} \FunctionTok{nchar}\NormalTok{(set\_n)}
  
\NormalTok{) }\SpecialCharTok{\%\textgreater{}\%} \FunctionTok{select}\NormalTok{(Q, condition, set\_n, set\_p, set\_a, n\_n, n\_p, n\_a)}

\CommentTok{\#2. DO THE STUFF THAT\textquotesingle{}S EASIER IN A LOOP}
\ControlFlowTok{for}\NormalTok{ (x }\ControlFlowTok{in} \DecValTok{1}\SpecialCharTok{:}\FunctionTok{nrow}\NormalTok{(keys\_tversky\_end)) \{}
  
  \CommentTok{\#UNWIND STRINGS FOR SETDIFF}
  \CommentTok{\#n all answer options}
\NormalTok{  N }\OtherTok{=}\NormalTok{ keys\_tversky\_end[x,}\StringTok{\textquotesingle{}set\_n\textquotesingle{}}\NormalTok{] }\SpecialCharTok{\%\textgreater{}\%} \FunctionTok{pull}\NormalTok{(set\_n) }\SpecialCharTok{\%\textgreater{}\%} \FunctionTok{strsplit}\NormalTok{(}\StringTok{""}\NormalTok{) }\SpecialCharTok{\%\textgreater{}\%} \FunctionTok{unlist}\NormalTok{()}
  \CommentTok{\#p correct{-}select answer options}
\NormalTok{  P }\OtherTok{=}\NormalTok{ keys\_tversky\_end[x,}\StringTok{\textquotesingle{}set\_p\textquotesingle{}}\NormalTok{] }\SpecialCharTok{\%\textgreater{}\%} \FunctionTok{pull}\NormalTok{(set\_p) }\SpecialCharTok{\%\textgreater{}\%} \FunctionTok{strsplit}\NormalTok{(}\StringTok{""}\NormalTok{) }\SpecialCharTok{\%\textgreater{}\%} \FunctionTok{unlist}\NormalTok{()}
  \CommentTok{\#a ignore{-}select answer options (should not be selected, but if they are, don\textquotesingle{}t penalize)}
\NormalTok{  A }\OtherTok{=}\NormalTok{ keys\_tversky\_end[x,}\StringTok{\textquotesingle{}set\_a\textquotesingle{}}\NormalTok{] }\SpecialCharTok{\%\textgreater{}\%} \FunctionTok{pull}\NormalTok{(set\_a) }\SpecialCharTok{\%\textgreater{}\%} \FunctionTok{strsplit}\NormalTok{(}\StringTok{""}\NormalTok{) }\SpecialCharTok{\%\textgreater{}\%} \FunctionTok{unlist}\NormalTok{()}
  
  \CommentTok{\#Q = N {-} (P+A)}
  \CommentTok{\#answers that are penalized (at {-}1/q) if selected }
\NormalTok{  s }\OtherTok{=} \FunctionTok{union}\NormalTok{(P,A) }\CommentTok{\#rewarded plus ignored }
\NormalTok{  s }\OtherTok{=} \FunctionTok{str\_replace\_na}\NormalTok{(s,}\StringTok{""}\NormalTok{)}
  \CommentTok{\# s = union(s,X) \# + trapdoor }
\NormalTok{  Q }\OtherTok{=} \FunctionTok{setdiff}\NormalTok{(N,s) }\CommentTok{\# = penalized at {-}1/q when selected }
  
  \CommentTok{\#save set to dataframe}
\NormalTok{  Q }\OtherTok{=} \FunctionTok{str\_c}\NormalTok{(Q, }\AttributeTok{collapse=}\StringTok{""}\NormalTok{)}
\NormalTok{  n\_q }\OtherTok{=} \FunctionTok{nchar}\NormalTok{(Q)}
\NormalTok{  keys\_tversky\_end[x,}\StringTok{\textquotesingle{}set\_q\textquotesingle{}}\NormalTok{] }\OtherTok{=}\NormalTok{ Q}
\NormalTok{  keys\_tversky\_end[x,}\StringTok{\textquotesingle{}n\_q\textquotesingle{}}\NormalTok{] }\OtherTok{=}\NormalTok{ n\_q}
  
  \CommentTok{\#verify each element in N is included in one and only one of P,A,Q}
\NormalTok{  tempunion }\OtherTok{=} \FunctionTok{union}\NormalTok{(s,Q) }\SpecialCharTok{\%\textgreater{}\%} \FunctionTok{str\_c}\NormalTok{(}\AttributeTok{collapse=}\StringTok{""}\NormalTok{) }\SpecialCharTok{\%\textgreater{}\%} \FunctionTok{strsplit}\NormalTok{(}\StringTok{""}\NormalTok{) }\SpecialCharTok{\%\textgreater{}\%} \FunctionTok{unlist}\NormalTok{()}
\NormalTok{  N }\OtherTok{=}\NormalTok{ N }\SpecialCharTok{\%\textgreater{}\%} \FunctionTok{str\_c}\NormalTok{(}\AttributeTok{collapse=}\StringTok{""}\NormalTok{) }\SpecialCharTok{\%\textgreater{}\%} \FunctionTok{strsplit}\NormalTok{(}\StringTok{""}\NormalTok{) }\SpecialCharTok{\%\textgreater{}\%} \FunctionTok{unlist}\NormalTok{()}
\NormalTok{  verify\_tversky\_end[x] }\OtherTok{=} \FunctionTok{setequal}\NormalTok{(tempunion,N)}
\NormalTok{\}}

\CommentTok{\#3. reorder cols for ease of use}
\NormalTok{keys\_tversky\_end }\OtherTok{\textless{}{-}}\NormalTok{ keys\_tversky\_end }\SpecialCharTok{\%\textgreater{}\%} \FunctionTok{select}\NormalTok{(Q, condition, set\_n, set\_p, set\_a, set\_q, n\_n, n\_p, n\_a, n\_q) }\SpecialCharTok{\%\textgreater{}\%} \FunctionTok{mutate}\NormalTok{(}\AttributeTok{verify =}\NormalTok{ n\_p }\SpecialCharTok{+}\NormalTok{ n\_a }\SpecialCharTok{+}\NormalTok{ n\_q)}

\CommentTok{\#4. replace condition 111 with "general" to accomodate other conditions [only 121 is special]}
\NormalTok{keys\_tversky\_end }\OtherTok{\textless{}{-}}\NormalTok{ keys\_tversky\_end }\SpecialCharTok{\%\textgreater{}\%} \FunctionTok{mutate}\NormalTok{(}
  \AttributeTok{condition =} \FunctionTok{replace}\NormalTok{(condition, condition }\SpecialCharTok{!=} \StringTok{"121"}\NormalTok{, }\StringTok{"DEFAULT"}\NormalTok{)}
\NormalTok{)}

\NormalTok{verify\_tversky\_duration }\OtherTok{=} \FunctionTok{c}\NormalTok{()}
\DocumentationTok{\#\#——————————————————————————————————————————————————————————————————————}
\DocumentationTok{\#\#CONSTRUCT TVERSKY KEY SET for TVERSKY{-}DURATION}
\DocumentationTok{\#\#——————————————————————————————————————————————————————————————————————}
\CommentTok{\#1. DEFINE SETS N, P, A}
\NormalTok{keys\_tversky\_duration }\OtherTok{\textless{}{-}}\NormalTok{ keys\_raw }\SpecialCharTok{\%\textgreater{}\%} 
  \FunctionTok{select}\NormalTok{(Q, condition, OPTIONS, REF\_POINT, TV\_dur, TV\_dur\_allow) }\SpecialCharTok{\%\textgreater{}\%} 
  \FunctionTok{mutate}\NormalTok{(}
  
    \CommentTok{\#replace NAs }
    \AttributeTok{REF\_POINT =} \FunctionTok{str\_replace\_na}\NormalTok{(REF\_POINT,}\StringTok{""}\NormalTok{),}
    \AttributeTok{TV\_dur =} \FunctionTok{str\_replace\_na}\NormalTok{(TV\_dur,}\StringTok{""}\NormalTok{),}
    \AttributeTok{TV\_dur\_allow =} \FunctionTok{str\_replace\_na}\NormalTok{(TV\_dur\_allow,}\StringTok{""}\NormalTok{),}
    
    \CommentTok{\#P options that SHOULD be selected (rewarded)}
    \AttributeTok{set\_p =}\NormalTok{ TV\_dur,}
    \AttributeTok{set\_p =} \FunctionTok{str\_replace\_na}\NormalTok{(set\_p,}\StringTok{""}\NormalTok{), }\CommentTok{\#replace na if empty}
    \AttributeTok{n\_p =} \FunctionTok{nchar}\NormalTok{(set\_p), }\CommentTok{\#number of true{-}select options}
    
    \CommentTok{\#A options that are ignored if selected }
    \AttributeTok{set\_a =} \FunctionTok{str\_c}\NormalTok{(TV\_dur\_allow,REF\_POINT, }\AttributeTok{sep=}\StringTok{""}\NormalTok{),}
    \AttributeTok{set\_a =} \FunctionTok{str\_replace\_na}\NormalTok{(set\_a,}\StringTok{""}\NormalTok{), }\CommentTok{\#replace na if empty}
    \AttributeTok{n\_a =} \FunctionTok{nchar}\NormalTok{(set\_a),}
    
    \CommentTok{\#N store all answr options (superset)}
    \AttributeTok{set\_n =}\NormalTok{ OPTIONS,  }
    \AttributeTok{n\_n =} \FunctionTok{nchar}\NormalTok{(set\_n)}
  
\NormalTok{) }\SpecialCharTok{\%\textgreater{}\%} \FunctionTok{select}\NormalTok{(Q, condition, set\_n, set\_p, set\_a, n\_n, n\_p, n\_a)}

\CommentTok{\#2. DO THE STUFF THAT\textquotesingle{}S EASIER IN A LOOP}
\ControlFlowTok{for}\NormalTok{ (x }\ControlFlowTok{in} \DecValTok{1}\SpecialCharTok{:}\FunctionTok{nrow}\NormalTok{(keys\_tversky\_duration)) \{}
  
  \CommentTok{\#UNWIND STRINGS FOR SETDIFF}
  \CommentTok{\#n all answer options}
\NormalTok{  N }\OtherTok{=}\NormalTok{ keys\_tversky\_duration[x,}\StringTok{\textquotesingle{}set\_n\textquotesingle{}}\NormalTok{] }\SpecialCharTok{\%\textgreater{}\%} \FunctionTok{pull}\NormalTok{(set\_n) }\SpecialCharTok{\%\textgreater{}\%} \FunctionTok{strsplit}\NormalTok{(}\StringTok{""}\NormalTok{) }\SpecialCharTok{\%\textgreater{}\%} \FunctionTok{unlist}\NormalTok{()}
  \CommentTok{\#p correct{-}select answer options}
\NormalTok{  P }\OtherTok{=}\NormalTok{ keys\_tversky\_duration[x,}\StringTok{\textquotesingle{}set\_p\textquotesingle{}}\NormalTok{] }\SpecialCharTok{\%\textgreater{}\%} \FunctionTok{pull}\NormalTok{(set\_p) }\SpecialCharTok{\%\textgreater{}\%} \FunctionTok{strsplit}\NormalTok{(}\StringTok{""}\NormalTok{) }\SpecialCharTok{\%\textgreater{}\%} \FunctionTok{unlist}\NormalTok{()}
  \CommentTok{\#a ignore{-}select answer options (should not be selected, but if they are, don\textquotesingle{}t penalize)}
\NormalTok{  A }\OtherTok{=}\NormalTok{ keys\_tversky\_duration[x,}\StringTok{\textquotesingle{}set\_a\textquotesingle{}}\NormalTok{] }\SpecialCharTok{\%\textgreater{}\%} \FunctionTok{pull}\NormalTok{(set\_a) }\SpecialCharTok{\%\textgreater{}\%} \FunctionTok{strsplit}\NormalTok{(}\StringTok{""}\NormalTok{) }\SpecialCharTok{\%\textgreater{}\%} \FunctionTok{unlist}\NormalTok{()}
  
  \CommentTok{\#Q = N {-} (P+A)}
  \CommentTok{\#answers that are penalized (at {-}1/q) if selected }
\NormalTok{  s }\OtherTok{=} \FunctionTok{union}\NormalTok{(P,A) }\CommentTok{\#rewarded plus ignored }
\NormalTok{  s }\OtherTok{=} \FunctionTok{str\_replace\_na}\NormalTok{(s,}\StringTok{""}\NormalTok{)}
  \CommentTok{\# s = union(s,X) \# + trapdoor }
\NormalTok{  Q }\OtherTok{=} \FunctionTok{setdiff}\NormalTok{(N,s) }\CommentTok{\# = penalized at {-}1/q when selected }
  
  \CommentTok{\#save set to dataframe}
\NormalTok{  Q }\OtherTok{=} \FunctionTok{str\_c}\NormalTok{(Q, }\AttributeTok{collapse=}\StringTok{""}\NormalTok{)}
\NormalTok{  n\_q }\OtherTok{=} \FunctionTok{nchar}\NormalTok{(Q)}
\NormalTok{  keys\_tversky\_duration[x,}\StringTok{\textquotesingle{}set\_q\textquotesingle{}}\NormalTok{] }\OtherTok{=}\NormalTok{ Q}
\NormalTok{  keys\_tversky\_duration[x,}\StringTok{\textquotesingle{}n\_q\textquotesingle{}}\NormalTok{] }\OtherTok{=}\NormalTok{ n\_q}
  
  \CommentTok{\#verify each element in N is included in one and only one of P,A,Q}
\NormalTok{  tempunion }\OtherTok{=} \FunctionTok{union}\NormalTok{(s,Q) }\SpecialCharTok{\%\textgreater{}\%} \FunctionTok{str\_c}\NormalTok{(}\AttributeTok{collapse=}\StringTok{""}\NormalTok{) }\SpecialCharTok{\%\textgreater{}\%} \FunctionTok{strsplit}\NormalTok{(}\StringTok{""}\NormalTok{) }\SpecialCharTok{\%\textgreater{}\%} \FunctionTok{unlist}\NormalTok{()}
\NormalTok{  N }\OtherTok{=}\NormalTok{ N }\SpecialCharTok{\%\textgreater{}\%} \FunctionTok{str\_c}\NormalTok{(}\AttributeTok{collapse=}\StringTok{""}\NormalTok{) }\SpecialCharTok{\%\textgreater{}\%} \FunctionTok{strsplit}\NormalTok{(}\StringTok{""}\NormalTok{) }\SpecialCharTok{\%\textgreater{}\%} \FunctionTok{unlist}\NormalTok{()}
\NormalTok{  verify\_tversky\_duration[x] }\OtherTok{=} \FunctionTok{setequal}\NormalTok{(tempunion,N)}
\NormalTok{\}}

\CommentTok{\#3. reorder cols for ease of use}
\NormalTok{keys\_tversky\_duration }\OtherTok{\textless{}{-}}\NormalTok{ keys\_tversky\_duration }\SpecialCharTok{\%\textgreater{}\%} \FunctionTok{select}\NormalTok{(Q, condition, set\_n, set\_p, set\_a, set\_q, n\_n, n\_p, n\_a, n\_q) }\SpecialCharTok{\%\textgreater{}\%} \FunctionTok{mutate}\NormalTok{(}\AttributeTok{verify =}\NormalTok{ n\_p }\SpecialCharTok{+}\NormalTok{ n\_a }\SpecialCharTok{+}\NormalTok{ n\_q)}

\CommentTok{\#4. replace condition 111 with "general" to accomodate other conditions [only 121 is special]}
\NormalTok{keys\_tversky\_duration }\OtherTok{\textless{}{-}}\NormalTok{ keys\_tversky\_duration }\SpecialCharTok{\%\textgreater{}\%} \FunctionTok{mutate}\NormalTok{(}
  \AttributeTok{condition =} \FunctionTok{replace}\NormalTok{(condition, condition }\SpecialCharTok{!=} \StringTok{"121"}\NormalTok{, }\StringTok{"DEFAULT"}\NormalTok{)}
\NormalTok{)}

\CommentTok{\#cleanup}
\FunctionTok{rm}\NormalTok{(A, N, n\_q, P, Q, s, tempunion, x)}
\end{Highlighting}
\end{Shaded}

This leaves us four dataframes, each corresponding to a different
variant of a `lines connecting to reference point' strategy.\\
- \texttt{keys\_tversky\_max} : the superset of lines connecting options
- \texttt{keys\_tversky\_start} : lines connecting to the rightward
diagonal (start time) of the reference point -
\texttt{keys\_tversky\_end}: lines connecting to the leftward diagonal
(end time) of the reference point - \texttt{keys\_tversky\_duration}:
lines connecting to the horizontal y-intercept (duration) of the
reference point

To verify we have generated the correct sets, we verify that for each
question, each response in N is included in either set P, A or Q (once
and only once).

TRUE, TRUE, TRUE, TRUE, TRUE, TRUE, TRUE, TRUE, TRUE, TRUE, TRUE, TRUE,
TRUE, TRUE, TRUE, TRUE, TRUE, TRUE, TRUE, TRUE

TRUE, TRUE, TRUE, TRUE, TRUE, TRUE, TRUE, TRUE, TRUE, TRUE, TRUE, TRUE,
TRUE, TRUE, TRUE, TRUE, TRUE, TRUE, TRUE, TRUE

TRUE, TRUE, TRUE, TRUE, TRUE, TRUE, TRUE, TRUE, TRUE, TRUE, TRUE, TRUE,
TRUE, TRUE, TRUE, TRUE, TRUE, TRUE, TRUE, TRUE

TRUE, TRUE, TRUE, TRUE, TRUE, TRUE, TRUE, TRUE, TRUE, TRUE, TRUE, TRUE,
TRUE, TRUE, TRUE, TRUE, TRUE, TRUE, TRUE, TRUE

\hypertarget{satisficing-key}{%
\subsubsection*{Satisficing Key}\label{satisficing-key}}
\addcontentsline{toc}{subsubsection}{Satisficing Key}

Next we construct two keys based on the `Satisficing' strategy.
Satisficing involves selecting any data points within 0.5hr visual
offset of the orthogonal interpretation of the graph (because no
orthogonal response option is available). One key represents selecting a
point slightly to the left of the orthogonal, and the other key
represents selecting a point slightly to the right of the orthogonal.
The ``Satisficing'' strategy involves the reader selecting data points
\emph{nearest} to the orthogonal projection from the reference point in
the question. We observe this strategy in some readers when there is no
orthogonal response available (i.e.~in the impasse condition), so they
select the points nearest to the projection (i.e.~``close enough'').

\begin{Shaded}
\begin{Highlighting}[]
\NormalTok{verify\_satisfice\_right }\OtherTok{=} \FunctionTok{c}\NormalTok{() }\CommentTok{\#sanity check}
\DocumentationTok{\#\#——————————————————————————————————————————————————————————————————————}
\DocumentationTok{\#\#CONSTRUCT SATISFICE RIGHT KEY SET}
\DocumentationTok{\#\#——————————————————————————————————————————————————————————————————————}
\CommentTok{\#1. DEFINE SETS N, P, A}
\NormalTok{keys\_satisfice\_right }\OtherTok{\textless{}{-}}\NormalTok{ keys\_raw }\SpecialCharTok{\%\textgreater{}\%} 
  \FunctionTok{select}\NormalTok{(Q, condition, OPTIONS, SATISFICE\_right, REF\_POINT) }\SpecialCharTok{\%\textgreater{}\%} 
  \FunctionTok{mutate}\NormalTok{(}
    \CommentTok{\#replace NAs}
    \AttributeTok{REF\_POINT =} \FunctionTok{str\_replace\_na}\NormalTok{(REF\_POINT,}\StringTok{""}\NormalTok{),}

    \CommentTok{\#P options that SHOULD be selected (rewarded)}
    \AttributeTok{set\_p =}\NormalTok{ SATISFICE\_right,}
    \AttributeTok{set\_p =} \FunctionTok{str\_replace\_na}\NormalTok{(set\_p,}\StringTok{""}\NormalTok{), }\CommentTok{\#replace na if empty}
    \AttributeTok{n\_p =} \FunctionTok{nchar}\NormalTok{(set\_p), }\CommentTok{\#number of true{-}select options}

    \CommentTok{\#A options that are ignored if selected}
    \AttributeTok{set\_a =} \FunctionTok{str\_c}\NormalTok{(REF\_POINT, }\AttributeTok{sep=}\StringTok{""}\NormalTok{),}
    \AttributeTok{set\_a =} \FunctionTok{str\_replace\_na}\NormalTok{(set\_a,}\StringTok{""}\NormalTok{), }\CommentTok{\#replace na if empty}
    \AttributeTok{n\_a =} \FunctionTok{nchar}\NormalTok{(set\_a),}

    \CommentTok{\#N store all answr options (superset)}
    \AttributeTok{set\_n =}\NormalTok{ OPTIONS,  }
    \AttributeTok{n\_n =} \FunctionTok{nchar}\NormalTok{(set\_n)}
  
\NormalTok{) }\SpecialCharTok{\%\textgreater{}\%} \FunctionTok{select}\NormalTok{(Q, condition, set\_n, set\_p, set\_a, n\_n, n\_p, n\_a)}

\CommentTok{\#2. DO THE STUFF THAT\textquotesingle{}S EASIER IN A LOOP}
\ControlFlowTok{for}\NormalTok{ (x }\ControlFlowTok{in} \DecValTok{1}\SpecialCharTok{:}\FunctionTok{nrow}\NormalTok{(keys\_satisfice\_right)) \{}
  
  \CommentTok{\#UNWIND STRINGS FOR SETDIFF}
  \CommentTok{\#n all answer options}
\NormalTok{  N }\OtherTok{=}\NormalTok{ keys\_satisfice\_right[x,}\StringTok{\textquotesingle{}set\_n\textquotesingle{}}\NormalTok{] }\SpecialCharTok{\%\textgreater{}\%} \FunctionTok{pull}\NormalTok{(set\_n) }\SpecialCharTok{\%\textgreater{}\%} \FunctionTok{strsplit}\NormalTok{(}\StringTok{""}\NormalTok{) }\SpecialCharTok{\%\textgreater{}\%} \FunctionTok{unlist}\NormalTok{()}
  \CommentTok{\#p correct{-}select answer options}
\NormalTok{  P }\OtherTok{=}\NormalTok{ keys\_satisfice\_right[x,}\StringTok{\textquotesingle{}set\_p\textquotesingle{}}\NormalTok{] }\SpecialCharTok{\%\textgreater{}\%} \FunctionTok{pull}\NormalTok{(set\_p) }\SpecialCharTok{\%\textgreater{}\%} \FunctionTok{strsplit}\NormalTok{(}\StringTok{""}\NormalTok{) }\SpecialCharTok{\%\textgreater{}\%} \FunctionTok{unlist}\NormalTok{()}
  \CommentTok{\#a ignore{-}select answer options (should not be selected, but if they are, don\textquotesingle{}t penalize)}
\NormalTok{  A }\OtherTok{=}\NormalTok{ keys\_satisfice\_right[x,}\StringTok{\textquotesingle{}set\_a\textquotesingle{}}\NormalTok{] }\SpecialCharTok{\%\textgreater{}\%} \FunctionTok{pull}\NormalTok{(set\_a) }\SpecialCharTok{\%\textgreater{}\%} \FunctionTok{strsplit}\NormalTok{(}\StringTok{""}\NormalTok{) }\SpecialCharTok{\%\textgreater{}\%} \FunctionTok{unlist}\NormalTok{() }
  
  \CommentTok{\#Q = N {-} (P+A)}
  \CommentTok{\#answers that are penalized (at {-}1/q) if selected }
\NormalTok{  s }\OtherTok{=} \FunctionTok{union}\NormalTok{(P,A) }\CommentTok{\#rewarded plus ignored }
\NormalTok{  s }\OtherTok{=} \FunctionTok{str\_replace\_na}\NormalTok{(s,}\StringTok{""}\NormalTok{)}
  \CommentTok{\# print(s)}
  \CommentTok{\# s = union(s,X) \# + trapdoor }
\NormalTok{  Q }\OtherTok{=} \FunctionTok{setdiff}\NormalTok{(N,s) }\CommentTok{\# = penalized at {-}1/q when selected }
  \CommentTok{\#save set to data frame}
\NormalTok{  Q }\OtherTok{=} \FunctionTok{str\_c}\NormalTok{(Q, }\AttributeTok{collapse=}\StringTok{""}\NormalTok{)}
\NormalTok{  n\_q }\OtherTok{=} \FunctionTok{nchar}\NormalTok{(Q)}
\NormalTok{  keys\_satisfice\_right[x,}\StringTok{\textquotesingle{}set\_q\textquotesingle{}}\NormalTok{] }\OtherTok{=}\NormalTok{ Q}
\NormalTok{  keys\_satisfice\_right[x,}\StringTok{\textquotesingle{}n\_q\textquotesingle{}}\NormalTok{] }\OtherTok{=}\NormalTok{ n\_q}
  
  \CommentTok{\#verify each element in N is included in one and only one of P,A,Q}
\NormalTok{  tempunion }\OtherTok{=} \FunctionTok{union}\NormalTok{(s,Q) }\SpecialCharTok{\%\textgreater{}\%} \FunctionTok{str\_c}\NormalTok{(}\AttributeTok{collapse=}\StringTok{""}\NormalTok{) }\SpecialCharTok{\%\textgreater{}\%} \FunctionTok{strsplit}\NormalTok{(}\StringTok{""}\NormalTok{) }\SpecialCharTok{\%\textgreater{}\%} \FunctionTok{unlist}\NormalTok{()}
\NormalTok{  N }\OtherTok{=}\NormalTok{ N }\SpecialCharTok{\%\textgreater{}\%} \FunctionTok{str\_c}\NormalTok{(}\AttributeTok{collapse=}\StringTok{""}\NormalTok{) }\SpecialCharTok{\%\textgreater{}\%} \FunctionTok{strsplit}\NormalTok{(}\StringTok{""}\NormalTok{) }\SpecialCharTok{\%\textgreater{}\%} \FunctionTok{unlist}\NormalTok{()}
\NormalTok{  verify\_satisfice\_right[x] }\OtherTok{=} \FunctionTok{setequal}\NormalTok{(tempunion,N)}
\NormalTok{\}}

\CommentTok{\#3. reorder cols for ease of use}
\NormalTok{keys\_satisfice\_right }\OtherTok{\textless{}{-}}\NormalTok{ keys\_satisfice\_right }\SpecialCharTok{\%\textgreater{}\%} \FunctionTok{select}\NormalTok{(Q, condition, set\_n, set\_p, set\_a, set\_q, n\_n, n\_p, n\_a, n\_q)}\SpecialCharTok{\%\textgreater{}\%} \FunctionTok{mutate}\NormalTok{(}\AttributeTok{verify =}\NormalTok{ n\_p }\SpecialCharTok{+}\NormalTok{ n\_a }\SpecialCharTok{+}\NormalTok{ n\_q)}

\CommentTok{\#4. replace condition 111 with "general" to accomodate other conditions [only 121 is special]}
\NormalTok{keys\_satisfice\_right }\OtherTok{\textless{}{-}}\NormalTok{ keys\_satisfice\_right }\SpecialCharTok{\%\textgreater{}\%} \FunctionTok{mutate}\NormalTok{(}
  \AttributeTok{condition =} \FunctionTok{replace}\NormalTok{(condition, condition }\SpecialCharTok{!=} \StringTok{"121"}\NormalTok{, }\StringTok{"DEFAULT"}\NormalTok{)}
\NormalTok{)}

\CommentTok{\#cleanup}
\FunctionTok{rm}\NormalTok{(A, N, n\_q, P, Q, s, tempunion, x)}



\NormalTok{verify\_satisfice\_left }\OtherTok{=} \FunctionTok{c}\NormalTok{() }\CommentTok{\#sanity check}
\DocumentationTok{\#\#——————————————————————————————————————————————————————————————————————}
\DocumentationTok{\#\#CONSTRUCT SATISFICE left KEY SET}
\DocumentationTok{\#\#——————————————————————————————————————————————————————————————————————}
\CommentTok{\#1. DEFINE SETS N, P, A}
\NormalTok{keys\_satisfice\_left }\OtherTok{\textless{}{-}}\NormalTok{ keys\_raw }\SpecialCharTok{\%\textgreater{}\%} 
  \FunctionTok{select}\NormalTok{(Q, condition, OPTIONS, SATISFICE\_left, REF\_POINT) }\SpecialCharTok{\%\textgreater{}\%} 
  \FunctionTok{mutate}\NormalTok{(}
    
    \CommentTok{\#replace NAs }
    \AttributeTok{REF\_POINT =} \FunctionTok{str\_replace\_na}\NormalTok{(REF\_POINT,}\StringTok{""}\NormalTok{),}
    
    \CommentTok{\#P options that SHOULD be selected (rewarded)}
    \AttributeTok{set\_p =}\NormalTok{ SATISFICE\_left,}
    \AttributeTok{set\_p =} \FunctionTok{str\_replace\_na}\NormalTok{(set\_p,}\StringTok{""}\NormalTok{), }\CommentTok{\#replace na if empty}
    \AttributeTok{n\_p =} \FunctionTok{nchar}\NormalTok{(set\_p), }\CommentTok{\#number of true{-}select options}
    
    \CommentTok{\#A options that are ignored if selected }
    \AttributeTok{set\_a =} \FunctionTok{str\_c}\NormalTok{(REF\_POINT, }\AttributeTok{sep=}\StringTok{""}\NormalTok{),}
    \AttributeTok{set\_a =} \FunctionTok{str\_replace\_na}\NormalTok{(set\_a,}\StringTok{""}\NormalTok{), }\CommentTok{\#replace na if empty}
    \AttributeTok{n\_a =} \FunctionTok{nchar}\NormalTok{(set\_a),}
    
    \CommentTok{\#N store all answr options (superset)}
    \AttributeTok{set\_n =}\NormalTok{ OPTIONS,  }
    \AttributeTok{n\_n =} \FunctionTok{nchar}\NormalTok{(set\_n)}
  
\NormalTok{) }\SpecialCharTok{\%\textgreater{}\%} \FunctionTok{select}\NormalTok{(Q, condition, set\_n, set\_p, set\_a, n\_n, n\_p, n\_a)}

\CommentTok{\#2. DO THE STUFF THAT\textquotesingle{}S EASIER IN A LOOP}
\ControlFlowTok{for}\NormalTok{ (x }\ControlFlowTok{in} \DecValTok{1}\SpecialCharTok{:}\FunctionTok{nrow}\NormalTok{(keys\_satisfice\_left)) \{}
  
  \CommentTok{\#UNWIND STRINGS FOR SETDIFF}
  \CommentTok{\#n all answer options}
\NormalTok{  N }\OtherTok{=}\NormalTok{ keys\_satisfice\_left[x,}\StringTok{\textquotesingle{}set\_n\textquotesingle{}}\NormalTok{] }\SpecialCharTok{\%\textgreater{}\%} \FunctionTok{pull}\NormalTok{(set\_n) }\SpecialCharTok{\%\textgreater{}\%} \FunctionTok{strsplit}\NormalTok{(}\StringTok{""}\NormalTok{) }\SpecialCharTok{\%\textgreater{}\%} \FunctionTok{unlist}\NormalTok{()}
  \CommentTok{\#p correct{-}select answer options}
\NormalTok{  P }\OtherTok{=}\NormalTok{ keys\_satisfice\_left[x,}\StringTok{\textquotesingle{}set\_p\textquotesingle{}}\NormalTok{] }\SpecialCharTok{\%\textgreater{}\%} \FunctionTok{pull}\NormalTok{(set\_p) }\SpecialCharTok{\%\textgreater{}\%} \FunctionTok{strsplit}\NormalTok{(}\StringTok{""}\NormalTok{) }\SpecialCharTok{\%\textgreater{}\%} \FunctionTok{unlist}\NormalTok{()}
  \CommentTok{\#a ignore{-}select answer options (should not be selected, but if they are, don\textquotesingle{}t penalize)}
\NormalTok{  A }\OtherTok{=}\NormalTok{ keys\_satisfice\_left[x,}\StringTok{\textquotesingle{}set\_a\textquotesingle{}}\NormalTok{] }\SpecialCharTok{\%\textgreater{}\%} \FunctionTok{pull}\NormalTok{(set\_a) }\SpecialCharTok{\%\textgreater{}\%} \FunctionTok{strsplit}\NormalTok{(}\StringTok{""}\NormalTok{) }\SpecialCharTok{\%\textgreater{}\%} \FunctionTok{unlist}\NormalTok{() }
  
  \CommentTok{\#Q = N {-} (P+A)}
  \CommentTok{\#answers that are penalized (at {-}1/q) if selected }
\NormalTok{  s }\OtherTok{=} \FunctionTok{union}\NormalTok{(P,A) }\CommentTok{\#rewarded plus ignored }
\NormalTok{  s }\OtherTok{=} \FunctionTok{str\_replace\_na}\NormalTok{(s,}\StringTok{""}\NormalTok{)}
  \CommentTok{\# print(s)}
  \CommentTok{\# s = union(s,X) \# + trapdoor }
\NormalTok{  Q }\OtherTok{=} \FunctionTok{setdiff}\NormalTok{(N,s) }\CommentTok{\# = penalized at {-}1/q when selected }
  \CommentTok{\#save set to data frame}
\NormalTok{  Q }\OtherTok{=} \FunctionTok{str\_c}\NormalTok{(Q, }\AttributeTok{collapse=}\StringTok{""}\NormalTok{)}
\NormalTok{  n\_q }\OtherTok{=} \FunctionTok{nchar}\NormalTok{(Q)}
\NormalTok{  keys\_satisfice\_left[x,}\StringTok{\textquotesingle{}set\_q\textquotesingle{}}\NormalTok{] }\OtherTok{=}\NormalTok{ Q}
\NormalTok{  keys\_satisfice\_left[x,}\StringTok{\textquotesingle{}n\_q\textquotesingle{}}\NormalTok{] }\OtherTok{=}\NormalTok{ n\_q}
  
  \CommentTok{\#verify each element in N is included in one and only one of P,A,Q}
\NormalTok{  tempunion }\OtherTok{=} \FunctionTok{union}\NormalTok{(s,Q) }\SpecialCharTok{\%\textgreater{}\%} \FunctionTok{str\_c}\NormalTok{(}\AttributeTok{collapse=}\StringTok{""}\NormalTok{) }\SpecialCharTok{\%\textgreater{}\%} \FunctionTok{strsplit}\NormalTok{(}\StringTok{""}\NormalTok{) }\SpecialCharTok{\%\textgreater{}\%} \FunctionTok{unlist}\NormalTok{()}
\NormalTok{  N }\OtherTok{=}\NormalTok{ N }\SpecialCharTok{\%\textgreater{}\%} \FunctionTok{str\_c}\NormalTok{(}\AttributeTok{collapse=}\StringTok{""}\NormalTok{) }\SpecialCharTok{\%\textgreater{}\%} \FunctionTok{strsplit}\NormalTok{(}\StringTok{""}\NormalTok{) }\SpecialCharTok{\%\textgreater{}\%} \FunctionTok{unlist}\NormalTok{()}
\NormalTok{  verify\_satisfice\_left[x] }\OtherTok{=} \FunctionTok{setequal}\NormalTok{(tempunion,N)}
\NormalTok{\}}

\CommentTok{\#3. reorder cols for ease of use}
\NormalTok{keys\_satisfice\_left }\OtherTok{\textless{}{-}}\NormalTok{ keys\_satisfice\_left }\SpecialCharTok{\%\textgreater{}\%} \FunctionTok{select}\NormalTok{(Q, condition, set\_n, set\_p, set\_a, set\_q, n\_n, n\_p, n\_a, n\_q)}\SpecialCharTok{\%\textgreater{}\%} \FunctionTok{mutate}\NormalTok{(}\AttributeTok{verify =}\NormalTok{ n\_p }\SpecialCharTok{+}\NormalTok{ n\_a }\SpecialCharTok{+}\NormalTok{ n\_q)}

\CommentTok{\#4. replace condition 111 with "general" to accomodate other conditions [only 121 is special]}
\NormalTok{keys\_satisfice\_left }\OtherTok{\textless{}{-}}\NormalTok{ keys\_satisfice\_left }\SpecialCharTok{\%\textgreater{}\%} \FunctionTok{mutate}\NormalTok{(}
  \AttributeTok{condition =} \FunctionTok{replace}\NormalTok{(condition, condition }\SpecialCharTok{!=} \StringTok{"121"}\NormalTok{, }\StringTok{"DEFAULT"}\NormalTok{)}
\NormalTok{)}

\CommentTok{\#cleanup}
\FunctionTok{rm}\NormalTok{(A, N, n\_q, P, Q, s, tempunion, x)}
\end{Highlighting}
\end{Shaded}

This leaves us a dataframe \texttt{keys\_satisfice} that define the sets
of response options consistent with the orthogonal graph interpretation.

To verify we have generated the correct sets, we verify that for each
question, each response in N is included in either set P, A or Q (once
and only once).

TRUE, TRUE, TRUE, TRUE, TRUE, TRUE, TRUE, TRUE, TRUE, TRUE, TRUE, TRUE,
TRUE, TRUE, TRUE, TRUE, TRUE, TRUE, TRUE, TRUE TRUE, TRUE, TRUE, TRUE,
TRUE, TRUE, TRUE, TRUE, TRUE, TRUE, TRUE, TRUE, TRUE, TRUE, TRUE, TRUE,
TRUE, TRUE, TRUE, TRUE

\begin{Shaded}
\begin{Highlighting}[]
\CommentTok{\#cleanup}
\FunctionTok{rm}\NormalTok{(verify\_tri, verify\_orth, verify\_max, verify\_tversky\_duration, verify\_tversky\_end, verify\_tversky\_start, verify\_satisfice\_right, verify\_satisfice\_left)}
\end{Highlighting}
\end{Shaded}

Finally, we need to clean up and generalize our answer keys to
accommodate the experimental conditions for Study SGC4-SGC5. In both of
these studies the answer set (and underlying graphed data set) are
identical, the conditions differ only based on the structure of the
gridlines or marks used to represent the data, or interactive mode of
the answer format.

\begin{Shaded}
\begin{Highlighting}[]
\CommentTok{\# imac = "/Users/amyraefox/Code/SGC{-}Scaffolding\_Graph\_Comprehension/SGC{-}X/ANALYSIS/MAIN"}
\CommentTok{\# setwd(imac)}

\CommentTok{\#SAVE KEYS FOR FUTURE USE}
\FunctionTok{write.csv}\NormalTok{(keys\_raw,}\StringTok{"analysis/utils/keys/parsed\_keys/keys\_raw"}\NormalTok{, }\AttributeTok{row.names =} \ConstantTok{FALSE}\NormalTok{)}
\FunctionTok{write.csv}\NormalTok{(keys\_orth,}\StringTok{"analysis/utils/keys/parsed\_keys/keys\_orth"}\NormalTok{, }\AttributeTok{row.names =} \ConstantTok{FALSE}\NormalTok{)}
\FunctionTok{write.csv}\NormalTok{(keys\_tri,}\StringTok{"analysis/utils/keys/parsed\_keys/keys\_tri"}\NormalTok{, }\AttributeTok{row.names =} \ConstantTok{FALSE}\NormalTok{)}
\FunctionTok{write.csv}\NormalTok{(keys\_satisfice\_left,}\StringTok{"analysis/utils/keys/parsed\_keys/keys\_satisfice\_left"}\NormalTok{, }\AttributeTok{row.names =} \ConstantTok{FALSE}\NormalTok{)}
\FunctionTok{write.csv}\NormalTok{(keys\_satisfice\_right,}\StringTok{"analysis/utils/keys/parsed\_keys/keys\_satisfice\_right"}\NormalTok{, }\AttributeTok{row.names =} \ConstantTok{FALSE}\NormalTok{)}
\FunctionTok{write.csv}\NormalTok{(keys\_tversky\_duration,}\StringTok{"analysis/utils/keys/parsed\_keys/keys\_tversky\_duration"}\NormalTok{, }\AttributeTok{row.names =} \ConstantTok{FALSE}\NormalTok{)}
\FunctionTok{write.csv}\NormalTok{(keys\_tversky\_end,}\StringTok{"analysis/utils/keys/parsed\_keys/keys\_tversky\_end"}\NormalTok{, }\AttributeTok{row.names =} \ConstantTok{FALSE}\NormalTok{)}
\FunctionTok{write.csv}\NormalTok{(keys\_tversky\_max,}\StringTok{"analysis/utils/keys/parsed\_keys/keys\_tversky\_max"}\NormalTok{, }\AttributeTok{row.names =} \ConstantTok{FALSE}\NormalTok{)}
\FunctionTok{write.csv}\NormalTok{(keys\_tversky\_start,}\StringTok{"analysis/utils/keys/parsed\_keys/keys\_tversky\_start"}\NormalTok{, }\AttributeTok{row.names =} \ConstantTok{FALSE}\NormalTok{)}
\end{Highlighting}
\end{Shaded}

\hypertarget{sec-scoring-subscores}{%
\subsection*{2. Calculate Subscores}\label{sec-scoring-subscores}}
\addcontentsline{toc}{subsection}{2. Calculate Subscores}

Next, we import the item-level response data. For each row in the item
level dataset (indicating the response to a single question-item for a
single subject), we compare the raw response
\texttt{df\_items\$response} with the answer keys in each interpretation
(e.g.~\texttt{keys\_orth}, \texttt{keys\_tri}, etc\ldots), then using
those sets, determine the number of correctly selected items(p) and
incorrectly selected items (q), which in turn are used to calculate
partial{[}-1/q, +1/p{]} scores for each interpretation. The resulting
scores are then stored on each item in \texttt{df\_items}, and can be
used to determine which graph interpretation the subject held.

Specifically, the following scores are calculated for each item:

\textbf{Interpretation Subscores}

\begin{itemize}
\tightlist
\item
  \texttt{score\_TRI} How consistent is the response with the
  \textbf{triangular} interpretation?
\item
  \texttt{score\_ORTH} How consistent is the response with the
  \textbf{orthogonal} interpretation?
\item
  \texttt{score\_SATISFICE} is calculated by taking the maximum value of
  :

  \begin{itemize}
  \tightlist
  \item
    \texttt{score\_SAT\_left} How consistent is the response with the
    \textbf{(left side) Satisficing} interpretation?
  \item
    \texttt{score\_SAT\_right} How consistent is the response with the
    \textbf{(right side) Satisficing} interpretation?
  \end{itemize}
\item
  \texttt{score\_TVERSKY} is calculated by taking the maximum value of:

  \begin{itemize}
  \tightlist
  \item
    \texttt{score\_TV\_max} How consistent is the response with the
    \textbf{(maximal) Tversky} interpretation?
  \item
    \texttt{score\_TV\_start} How consistent is the response with the
    \textbf{(start-time) Tversky} interpretation?
  \item
    \texttt{score\_TV\_end} How consistent is the response with the
    \textbf{(end-time) Tversky} interpretation?
  \item
    \texttt{score\_TV\_duration} How consistent is the response with the
    \textbf{(duration) Tversky} interpretation?
  \end{itemize}
\item
  \texttt{score\_REF} Did the response select only the \textbf{reference
  point}?
\item
  \texttt{score\_BOTH} How consistent is the response with \textbf{both}
  the orthogonal and triangular interpretations?
\end{itemize}

\textbf{Absolute Scores}

\begin{itemize}
\tightlist
\item
  \texttt{score\_ABS} Is the response strictly correct? (triangular
  interpretation)
\item
  \texttt{score\_niceABS} Is the response strictly correct? (triangular
  interpretation, not penalizing ref points). This is a more generous
  version of the Absolute score that does not penalize the participant
  if in addition to the correct answer \emph{in addition to} they also
  select the data point referenced in the question.
\end{itemize}

To facilitate scoring, we import the following helper functions in each
scoring script.

\begin{Shaded}
\begin{Highlighting}[]
\CommentTok{\# \#HACK WD FOR LOCAL RUNNING?}
\NormalTok{imac }\OtherTok{=} \StringTok{"/Users/amyraefox/Code/SGC{-}Scaffolding\_Graph\_Comprehension/SGC{-}X/ANALYSIS/MAIN"}
\FunctionTok{setwd}\NormalTok{(imac)}

\FunctionTok{source}\NormalTok{(}\StringTok{"analysis/utils/scoring.R"}\NormalTok{)}

\FunctionTok{print}\NormalTok{(calc\_subscore)}
\end{Highlighting}
\end{Shaded}

\begin{verbatim}
function (question, cond, response, keyframe) 
{
    if (cond == 121 & question < 6) {
        p = keyframe %>% filter(Q == question) %>% filter(condition == 
            "121") %>% select(set_p) %>% pull(set_p) %>% str_split("") %>% 
            unlist()
        q = keyframe %>% filter(Q == question) %>% filter(condition == 
            "121") %>% select(set_q) %>% pull(set_q) %>% str_split("") %>% 
            unlist()
        pn = keyframe %>% filter(Q == question) %>% filter(condition == 
            "121") %>% select(n_p)
        qn = keyframe %>% filter(Q == question) %>% filter(condition == 
            "121") %>% select(n_q)
    }
    else {
        p = keyframe %>% filter(Q == question) %>% filter(condition == 
            "DEFAULT") %>% select(set_p) %>% pull(set_p) %>% 
            str_split("") %>% unlist()
        q = keyframe %>% filter(Q == question) %>% filter(condition == 
            "DEFAULT") %>% select(set_q) %>% pull(set_q) %>% 
            str_split("") %>% unlist()
        pn = keyframe %>% filter(Q == question) %>% filter(condition == 
            "DEFAULT") %>% select(n_p)
        qn = keyframe %>% filter(Q == question) %>% filter(condition == 
            "DEFAULT") %>% select(n_q)
    }
    if (response != "") {
        response = response %>% str_split("") %>% unlist()
    }
    ps = length(intersect(response, p))
    qs = length(intersect(response, q))
    x = f_partialP(ps, pn, qs, qn) %>% unlist() %>% as.numeric()
    rm(p, q, pn, qn, ps, qs)
    return(x)
}
\end{verbatim}

\hypertarget{sec-scoring-interpretation}{%
\subsection*{3. Derive
Interpretation}\label{sec-scoring-interpretation}}
\addcontentsline{toc}{subsection}{3. Derive Interpretation}

Next, we use the interpretation subscores to classify the response as a
particular interpretation. This classification algorithm : (1) First
decides if the response matches one or more `special' situations (blank
response, reference point response, both ORTH and TRI) (2) If response
doesn't match a special situation, it compares the individual subscores,
and subscores and decides if they are \emph{discriminant} (i.e.~are the
scores different enough to make a prediction). A discriminant threshold
of 0.5pts (on a scale from -1 to +1 is used) (2) If the variance in
subscores surpasses the threshold, the interpretation is classified
based on the highest subscore (TRIANGULAR, ORTHOGONAL, TVERSKY,
SATISFICE) (3) If the variance does not surpass the threshold, the
interpretation is labelled as ``?'', indicating it cannot be classified,
and is of an unknown interpretation.

The final output is called \texttt{interpretation}.

\begin{Shaded}
\begin{Highlighting}[]
\FunctionTok{print}\NormalTok{(derive\_interpretation)}
\end{Highlighting}
\end{Shaded}

\begin{verbatim}
function (df) 
{
    threshold_range = 0.5
    threshold_frenzy = 4
    for (x in 1:nrow(df)) {
        t = df[x, ] %>% dplyr::select(score_TV_max, score_TV_start, 
            score_TV_end, score_TV_duration)
        t.long = gather(t, score, value, 1:4)
        t.long[t.long == ""] = NA
        if (length(unique(t.long$value)) == 1) {
            if (is.na(unique(t.long$value))) {
                df[x, "score_TVERSKY"] = NA
                df[x, "tv_type"] = NA
            }
        }
        else {
            df[x, "score_TVERSKY"] = as.numeric(max(t.long$value, 
                na.rm = TRUE))
            df[x, "tv_type"] = t.long[which.max(t.long$value), 
                "score"]
        }
        t = df[x, ] %>% dplyr::select(score_SAT_left, score_SAT_right)
        t.long = gather(t, score, value, 1:2)
        t.long[t.long == ""] = NA
        if (length(unique(t.long$value)) == 1) {
            if (is.na(unique(t.long$value))) {
                df[x, "score_SATISFICE"] = NA
                df[x, "sat_type"] = NA
            }
        }
        else {
            df[x, "score_SATISFICE"] = as.numeric(max(t.long$value, 
                na.rm = TRUE))
            df[x, "sat_type"] = t.long[which.max(t.long$value), 
                "score"]
        }
        t = df[x, ] %>% dplyr::select(score_TRI, score_TVERSKY, 
            score_SATISFICE, score_ORTH)
        t.long = gather(t, score, value, 1:4)
        t.long[t.long == ""] = NA
        df[x, "top_score"] = as.numeric(max(t.long$value, na.rm = TRUE))
        df[x, "top_type"] = t.long[which.max(t.long$value), "score"]
        r = as.numeric(range(t.long$value, na.rm = TRUE))
        r = diff(r)
        df[x, "range"] = r
        if (r < threshold_range) {
            df[x, "best"] = "?"
        }
        else {
            p = df[x, "top_type"]
            if (p == "score_TRI") {
                df[x, "best"] = "Triangular"
            }
            else if (p == "score_ORTH") {
                df[x, "best"] = "Orthogonal"
            }
            else if (p == "score_TVERSKY") {
                df[x, "best"] = "Tversky"
            }
            else if (p == "score_SATISFICE") {
                df[x, "best"] = "Satisfice"
            }
        }
        if (!is.na(df[x, "score_BOTH"])) {
            if (df[x, "score_BOTH"] == 1) {
                df[x, "best"] = "both tri + orth"
            }
        }
        if (df[x, "num_o"] == 0) {
            df[x, "best"] = "blank"
        }
        if (df[x, "num_o"] > threshold_frenzy) {
            df[x, "best"] = "frenzy"
        }
        if (!is.na(df[x, "score_REF"])) {
            if (df[x, "score_REF"] == 1) {
                df[x, "best"] = "reference"
            }
        }
    }
    rm(t, t.long, x, r, p)
    rm(threshold_frenzy, threshold_range)
    df$int2 <- factor(df$best, levels = c("Triangular", "Tversky", 
        "Satisfice", "Orthogonal", "reference", "both tri + orth", 
        "blank", "frenzy", "?"))
    df$interpretation <- factor(df$best, levels = c("Orthogonal", 
        "Satisfice", "frenzy", "?", "reference", "blank", "both tri + orth", 
        "Tversky", "Triangular"))
    df$high_interpretation <- fct_collapse(df$interpretation, 
        orthogonal = c("Satisfice", "Orthogonal"), neg.trans = c("frenzy", 
            "?"), neutral = c("reference", "blank"), pos.trans = c("Tversky", 
            "both tri + orth"), triangular = "Triangular")
    df$tv_type = as.factor(df$tv_type)
    df$top_type = as.factor(df$top_type)
    df$high_interpretation = factor(df$high_interpretation, levels = c("orthogonal", 
        "neg.trans", "neutral", "pos.trans", "triangular"))
    df <- df %>% dplyr::select(-best)
    return(df)
}
\end{verbatim}

\hypertarget{sec-scoring-scaledScore}{%
\subsection*{4. Derive Scaled Score}\label{sec-scoring-scaledScore}}
\addcontentsline{toc}{subsection}{4. Derive Scaled Score}

The \texttt{interpretation} response variable gives us the finest grain
indication of the reader's understanding of the graph for a particular
question. However, it is a categorical variable, which poses a challenge
for analyzing cumulative performance at the subject level. To address
this challenge, we derive a \emph{scaled\_score} that converts each
possible interpretation to a numeric value on a scale from -1 to +1.
This scaling takes advantage of the observation that each interpretation
can be positioned along a spectrum of understanding from completely
incorrect (orthogonal) to completely correct (triangular). Alternative
interpretations lay somewhere between.

Specifically, we assign the following values to each interpretation:

\begin{itemize}
\tightlist
\item
  (-1) : ORTHOGONAL, SATISFICE (satisfice represents an attempt at an
  orthogonal answer when none is available)
\item
  (-0.5): ? (some alternative that cannot be identified; but meaningful
  that it is not orthogonal)
\item
  (0): REFERENCE POINT, BLANK (indicates the individual thinks there is
  no answer, recognizes that ORTHOGONAL cannot be correct, but does not
  conceive of triangular)
\item
  (+0.5) TVERSKY, BOTH TRI + ORTH (indicates that they ``see'' a
  triangular response, but lack certainty and also select the ORTHOGONAL
  response)
\item
  (+1) TRIANGULAR +1
\end{itemize}

\begin{Shaded}
\begin{Highlighting}[]
\FunctionTok{print}\NormalTok{(calc\_scaled)}
\end{Highlighting}
\end{Shaded}

\begin{verbatim}
function (v) 
{
    v <- recode(v, Orthogonal = -1, Satisfice = -1, frenzy = -0.5, 
        `?` = -0.5, reference = 0, blank = 0, `both tri + orth` = 0.5, 
        Tversky = 0.5, Triangular = 1)
    return(v)
}
\end{verbatim}

\hypertarget{sec-scoring-summarise}{%
\subsection*{5. Summarize by Subject}\label{sec-scoring-summarise}}
\addcontentsline{toc}{subsection}{5. Summarize by Subject}

The final step in the scoring procedure is to summarise the item-level
scores by subject, and save certain summaries to the subject-level
record. We also construct two long-format dataframes containing
cummulative progress scores (the point-in-time {[}absolute, scaled{]}
scores for each subject on each question).

\begin{Shaded}
\begin{Highlighting}[]
\FunctionTok{print}\NormalTok{(summarise\_bySubject)}
\end{Highlighting}
\end{Shaded}

\begin{verbatim}
function (subjects, items) 
{
    subjects_summary <- items %>% filter(q %nin% c(6, 9)) %>% 
        group_by(subject) %>% dplyr::summarise(subject = as.character(subject), 
        s_TRI = sum(score_TRI, na.rm = TRUE), s_ORTH = sum(score_ORTH, 
            na.rm = TRUE), s_TVERSKY = sum(score_TVERSKY, na.rm = TRUE), 
        s_SATISFICE = sum(score_SATISFICE, na.rm = TRUE), s_REF = sum(score_REF, 
            na.rm = TRUE), s_ABS = sum(score_ABS, na.rm = TRUE), 
        s_NABS = sum(score_niceABS, na.rm = TRUE), s_SCALED = sum(score_SCALED, 
            na.rm = TRUE), DV_percent_NABS = s_NABS/13, rt_m = sum(rt_s)/60, 
        item_avg_rt = mean(rt_s), item_min_rt = min(rt_s), item_max_rt = max(rt_s), 
        item_n_TRI = sum(interpretation == "Triangular"), item_n_ORTH = sum(interpretation == 
            "Orthogonal"), item_n_TV = sum(interpretation == 
            "Tversky"), item_n_SAT = sum(interpretation == "Satisfice"), 
        item_n_OTHER = sum(interpretation %nin% c("Triangular", 
            "Orthogonal", "Tversky", "Satisfice")), item_n_POS = sum(high_interpretation == 
            "pos.trans"), item_n_NEG = sum(high_interpretation == 
            "neg.trans"), item_n_NEUTRAL = sum(high_interpretation == 
            "neutral")) %>% arrange(subject) %>% slice(1L)
    subjects_q1 <- items %>% filter(q == 1) %>% mutate(item_q1_NABS = score_niceABS, 
        item_q1_SCALED = score_SCALED, item_q1_interpretation = interpretation, 
        item_q1_rt = rt_s, ) %>% dplyr::select(subject, item_q1_NABS, 
        item_q1_SCALED, item_q1_interpretation, item_q1_rt) %>% 
        arrange(subject)
    subjects_q5 <- items %>% filter(q == 5) %>% mutate(item_q5_NABS = score_niceABS, 
        item_q5_SCALED = score_SCALED, item_q5_interpretation = interpretation, 
        item_q5_rt = rt_s, ) %>% dplyr::select(subject, item_q5_NABS, 
        item_q5_SCALED, item_q5_interpretation, item_q5_rt) %>% 
        arrange(subject)
    subjects_q7 <- items %>% filter(q == 7) %>% mutate(item_q7_NABS = score_niceABS, 
        item_q7_interpretation = interpretation, item_q7_rt = rt_s, 
        ) %>% dplyr::select(subject, item_q7_NABS, item_q7_interpretation, 
        item_q7_rt) %>% arrange(subject)
    subjects_q15 <- items %>% filter(q == 15) %>% mutate(item_q15_NABS = score_niceABS, 
        item_q15_interpretation = interpretation, item_q15_rt = rt_s, 
        ) %>% dplyr::select(subject, item_q15_NABS, item_q15_interpretation, 
        item_q15_rt) %>% arrange(subject)
    subjects_scaffold <- items %>% filter(q < 6) %>% group_by(subject) %>% 
        dplyr::summarise(item_scaffold_NABS = sum(score_niceABS), 
            item_scaffold_SCALED = sum(score_SCALED), item_scaffold_rt = sum(rt_s)/60) %>% 
        dplyr::select(subject, item_scaffold_NABS, item_scaffold_SCALED, 
            item_scaffold_rt) %>% arrange(subject)
    subjects_test <- items %>% filter(q %nin% c(1, 2, 3, 4, 5, 
        6, 9)) %>% group_by(subject) %>% dplyr::summarise(item_test_NABS = sum(score_niceABS), 
        item_test_SCALED = sum(score_SCALED), item_test_rt = sum(rt_s)/60) %>% 
        dplyr::select(subject, item_test_NABS, item_test_SCALED, 
            item_test_rt) %>% arrange(subject)
    print(unique(subjects_summary$subject == subjects$subject))
    print(unique(subjects_summary$subject == subjects_q1$subject))
    print(unique(subjects_summary$subject == subjects_q5$subject))
    print(unique(subjects_summary$subject == subjects_q7$subject))
    print(unique(subjects_summary$subject == subjects_q15$subject))
    print(unique(subjects_summary$subject == subjects_scaffold$subject))
    print(unique(subjects_summary$subject == subjects_test$subject))
    x = merge(subjects, subjects_summary, by.x = "subject", by.y = "subject")
    x = merge(x, subjects_q1)
    x = merge(x, subjects_q5)
    x = merge(x, subjects_q7)
    x = merge(x, subjects_q15)
    x = merge(x, subjects_scaffold)
    x = merge(x, subjects_test)
    subjects <- x
    rm(subjects_q1, subjects_q5, subjects_q7, subjects_q15, subjects_scaffold, 
        subjects_test, subjects_summary, x)
    return(subjects)
}
\end{verbatim}

\begin{Shaded}
\begin{Highlighting}[]
\FunctionTok{print}\NormalTok{(progress\_Absolute)}
\end{Highlighting}
\end{Shaded}

\begin{verbatim}
function (items) 
{
    x <- items %>% filter(q %nin% c(6, 9)) %>% dplyr::select(subject, 
        mode, pretty_condition, q, score_niceABS)
    wide <- x %>% pivot_wider(names_from = q, names_glue = "q_{q}", 
        values_from = score_niceABS)
    wide$c1 = wide$q_1
    wide$c2 = wide$c1 + wide$q_2
    wide$c3 = wide$c2 + wide$q_3
    wide$c4 = wide$c3 + wide$q_4
    wide$c5 = wide$c4 + wide$q_5
    wide$c6 = wide$c5 + wide$q_7
    wide$c7 = wide$c6 + wide$q_8
    wide$c8 = wide$c7 + wide$q_10
    wide$c9 = wide$c8 + wide$q_11
    wide$c10 = wide$c9 + wide$q_12
    wide$c11 = wide$c10 + wide$q_13
    wide$c12 = wide$c11 + wide$q_14
    wide$c13 = wide$c12 + wide$q_15
    wide <- wide %>% dplyr::select(subject, mode, pretty_condition, 
        c1, c2, c3, c4, c5, c6, c7, c8, c9, c10, c11, c12, c13)
    df_absolute_progress <- wide %>% pivot_longer(cols = c1:c13, 
        names_to = "question", names_pattern = "c(.*)", values_to = "score")
    df_absolute_progress$question <- as.integer(df_absolute_progress$question)
    rm(x, wide)
    return(df_absolute_progress)
}
\end{verbatim}

\begin{Shaded}
\begin{Highlighting}[]
\FunctionTok{print}\NormalTok{(progress\_Scaled)}
\end{Highlighting}
\end{Shaded}

\begin{verbatim}
function (items) 
{
    x <- items %>% filter(q %nin% c(6, 9)) %>% select(subject, 
        mode, pretty_condition, q, score_SCALED)
    wide <- x %>% pivot_wider(names_from = q, names_glue = "q_{q}", 
        values_from = score_SCALED)
    wide$c1 = wide$q_1
    wide$c2 = wide$c1 + wide$q_2
    wide$c3 = wide$c2 + wide$q_3
    wide$c4 = wide$c3 + wide$q_4
    wide$c5 = wide$c4 + wide$q_5
    wide$c6 = wide$c5 + wide$q_7
    wide$c7 = wide$c6 + wide$q_8
    wide$c8 = wide$c7 + wide$q_10
    wide$c9 = wide$c8 + wide$q_11
    wide$c10 = wide$c9 + wide$q_12
    wide$c11 = wide$c10 + wide$q_13
    wide$c12 = wide$c11 + wide$q_14
    wide$c13 = wide$c12 + wide$q_15
    wide <- wide %>% select(subject, mode, pretty_condition, 
        c1, c2, c3, c4, c5, c6, c7, c8, c9, c10, c11, c12, c13)
    df_scaled_progress <- wide %>% pivot_longer(cols = c1:c13, 
        names_to = "question", names_pattern = "c(.*)", values_to = "score")
    df_scaled_progress$question <- as.integer(df_scaled_progress$question)
    rm(x, wide)
    return(df_scaled_progress)
}
\end{verbatim}

\part{SGC3A}

\newpage

\hypertarget{sec-SGC3A-introduction}{%
\chapter{Introduction}\label{sec-SGC3A-introduction}}

\textbf{In Study 3A we explore the extent to which confronting a learner
with an implicit obstacle (a mental impasse) influences their
interpretation of the underlying coordinate system.} This is a
hypothesis that emerged from analysis of Study 2, leading us to suspect
that \emph{presenting a learner with a situation that induces a state of
impasse will increase the probability that learners experience a moment
of insight, and in turn restructure their interpretation of the
coordinate system.}

\begin{longtable}[]{@{}
  >{\raggedright\arraybackslash}p{(\columnwidth - 2\tabcolsep) * \real{0.4000}}
  >{\raggedright\arraybackslash}p{(\columnwidth - 2\tabcolsep) * \real{0.6000}}@{}}
\caption{\textbf{SGC3A Study Conditions}}\tabularnewline
\toprule()
\endhead
\includegraphics{analysis/utils/img/111.png} &
\begin{minipage}[t]{\linewidth}\raggedright
\textbf{Control-Condition}\\
Demo:
\href{https://limitless-plains-85018.herokuapp.com/?study=SGC3A\&condition=111\&session=WEB-DEMO}{111}\strut
\end{minipage} \\
\includegraphics{analysis/utils/img/121.png} &
\begin{minipage}[t]{\linewidth}\raggedright
\textbf{Impasse-Condition}\\
Demo:
\href{https://limitless-plains-85018.herokuapp.com/?study=SGC3A\&condition=121\&session=WEB-DEMO}{121}\strut
\end{minipage} \\
\bottomrule()
\end{longtable}

In the context of Study 2, an impasse state was (unintentionally)
induced when the combination of question + data set yielded no available
answer in the incorrect (cartesian) interpretation of the graph. In
Study 3A, we test this hypothesis by comparing performance between a
(treatment) group receiving impasse-inducing questions followed by
normal questions, and a non-impasse control.

\begin{figure}

\begin{minipage}[t]{0.50\linewidth}

{\centering 

\raisebox{-\height}{

\includegraphics{analysis/SGC3A/static/stimuli/nonimpasse.png}

}

}

\subcaption{\label{fig-control}}
\end{minipage}%
%
\begin{minipage}[t]{0.50\linewidth}

{\centering 

\raisebox{-\height}{

\includegraphics{analysis/SGC3A/static/stimuli/impasse.png}

}

}

\subcaption{\label{fig-impasse}}
\end{minipage}%

\caption{\label{fig-manipulation}Posing a mental impasse}

\end{figure}

\begin{Shaded}
\begin{Highlighting}[]
\FunctionTok{library}\NormalTok{(codebook) }\CommentTok{\#data dictionary}
\FunctionTok{library}\NormalTok{(tidyverse) }\CommentTok{\#ALL THE THINGS}
\FunctionTok{library}\NormalTok{(kableExtra) }\CommentTok{\#tables}

\CommentTok{\#set some output options}
\FunctionTok{library}\NormalTok{(dplyr, }\AttributeTok{warn.conflicts =} \ConstantTok{FALSE}\NormalTok{)}
\FunctionTok{options}\NormalTok{(}\AttributeTok{dplyr.summarise.inform =} \ConstantTok{FALSE}\NormalTok{)}
\FunctionTok{options}\NormalTok{(}\AttributeTok{scipen=}\DecValTok{1}\NormalTok{, }\AttributeTok{digits=}\DecValTok{3}\NormalTok{)}
\end{Highlighting}
\end{Shaded}

\begin{Shaded}
\begin{Highlighting}[]
\CommentTok{\# }\AlertTok{HACK}\CommentTok{ WD FOR LOCAL RUNNING?}
\CommentTok{\# imac = "/Users/amyraefox/Code/SGC{-}Scaffolding\_Graph\_Comprehension/SGC{-}X/ANALYSIS/MAIN"}
\CommentTok{\# \# mbp = "/Users/amyfox/Sites/RESEARCH/SGC—Scaffolding Graph Comprehension/SGC{-}X/ANALYSIS/MAIN"}
\CommentTok{\# setwd(imac)}

\CommentTok{\#IMPORT DATA }
\NormalTok{df\_subjects }\OtherTok{\textless{}{-}} \FunctionTok{read\_rds}\NormalTok{(}\StringTok{\textquotesingle{}analysis/SGC3A/data/1{-}study{-}level/sgc3a\_participants.rds\textquotesingle{}}\NormalTok{)}
\end{Highlighting}
\end{Shaded}

\begin{Shaded}
\begin{Highlighting}[]
\NormalTok{title }\OtherTok{=} \StringTok{"Participants by Condition and Data Collection Period"}
\NormalTok{cols }\OtherTok{=} \FunctionTok{c}\NormalTok{(}\StringTok{"Control Condition"}\NormalTok{,}\StringTok{"Impasse Condition"}\NormalTok{,}\StringTok{"Total for Period"}\NormalTok{)}
\NormalTok{cont }\OtherTok{\textless{}{-}} \FunctionTok{table}\NormalTok{(df\_subjects}\SpecialCharTok{$}\NormalTok{term, df\_subjects}\SpecialCharTok{$}\NormalTok{condition)}
\NormalTok{cont }\SpecialCharTok{\%\textgreater{}\%} \FunctionTok{addmargins}\NormalTok{() }\SpecialCharTok{\%\textgreater{}\%} \FunctionTok{kbl}\NormalTok{(}\AttributeTok{caption =}\NormalTok{ title, }\AttributeTok{col.names =}\NormalTok{ cols) }\SpecialCharTok{\%\textgreater{}\%}  \FunctionTok{kable\_classic}\NormalTok{()}
\end{Highlighting}
\end{Shaded}

\begin{table}

\caption{Participants by Condition and Data Collection Period}
\centering
\begin{tabular}[t]{l|r|r|r}
\hline
  & Control Condition & Impasse Condition & Total for Period\\
\hline
fall17 & 27 & 27 & 54\\
\hline
spring18 & 35 & 37 & 72\\
\hline
fall21 & 68 & 71 & 139\\
\hline
winter22 & 28 & 37 & 65\\
\hline
Sum & 158 & 172 & 330\\
\hline
\end{tabular}
\end{table}

\hypertarget{hypotheses}{%
\subsection{Hypotheses}\label{hypotheses}}

\textbf{Experimental Hypothesis}\\
\emph{Learners posed with scenario designed to evoke a mental impasse
will be more likely to correct interpret the graph.}

\begin{itemize}
\tightlist
\item
  H1A \textbar{} Learners in the IMPASSE condition will score higher on
  the TEST Phase than learners in CONTROL.
\item
  H1B \textbar{} Learners in the IMPASSE condition will be more likely
  to correctly answer the first question than learners in CONTROL.
\item
  H1C \textbar{} Learners in the IMPASSE condition will spend more time
  on the first question than learners in CONTROL.
\end{itemize}

\textbf{Null Hypothesis}\\
\emph{No significant differences in performance will exist between
learners in the IMPASSE and CONTROL conditions.}

\textbf{Exploratory Questions}

\begin{itemize}
\tightlist
\item
  Consistency \textbar{} How consistent are learners in their
  interpretation of the graph? Do they adopt an interpretation on the
  first question and hold constant? Or do they change interpretations
  from question to question? Are there any interpretations that serve as
  `absorbing states' (i.e.~once encountered, the learner does not exist
  this state).
\item
  Time Course of Exploration \textbar{} What is the relationship between
  response accuracy (and interpretation) and time spent on each item?
\item
  Can exploration strategies be derived from mouse cursor activity?
\end{itemize}

\hypertarget{methods}{%
\section{METHODS}\label{methods}}

\hypertarget{design}{%
\subsection{Design}\label{design}}

We employed a mixed design with 1 between-subjects factor with 2 levels
(Scaffold: control, impasse) and 15 items (within-subjects factor).

Independent Variables:

\begin{itemize}
\tightlist
\item
  B-S (Scaffold: control,impasse)
\item
  W-S (Item x 15)
\end{itemize}

Dependent Variables:

\begin{itemize}
\tightlist
\item
  Response Accuracy : Is the response triangular-correct?
\item
  Response Interpretation : (derived) With which interpretation of the
  graph is the subject's response on an individual question consistent?
\item
  Response Latency : Time from stimulus onset to clicking `Submit'
  button: time in (s)
\end{itemize}

\hypertarget{materials}{%
\subsection{Materials}\label{materials}}

Stimuli consisted of a series of 15 graph comprehension questions, each
testing a different combination of time interval relations, to be read
from a Triangular-Model graph. Figure~\ref{fig-sample}. The list of
questions can be found \href{static/stimuli/sgcx_questions.csv}{here}.

\begin{figure}

{\centering \includegraphics{analysis/SGC3A/static/stimuli/sample_task.png}

}

\caption{\label{fig-sample}Sample Question (Q=1) for Graph Comprehension
Task}

\end{figure}

Note that across both control and impasse conditions, both the question,
response options and graph structure were identical. The experimental
manipulation (posing a mental impasse) was accomplished by changing the
position of datapoints in the impasse-condition graph, such that for any
given question, there was no available response option if the reader
were to interpret the graph as cartesian (making an orthogonal rather
than diagonal projection from the x-axis.)

\emph{The green line indicates the ideal-scanpath to the correct
(triangular) answer to the first question, and the red line indicates
the (incorrect) orthogonal interpretation. In the IMPASSE figure (at
right), there are no data points that intersect the red line. We
hypothesize that this presents the reader with an obstacle, at which
point they are forced to confront their interpretation of the coordinate
system and (ideally) develop a new strategy.}

\begin{figure}

{\centering \includegraphics{analysis/SGC3A/static/stimuli/3A_conditions.png}

}

\caption{\label{fig-conditions}Sample Question (Q=1) graphs for each
condition}

\end{figure}

\hypertarget{procedure}{%
\subsection{Procedure}\label{procedure}}

Participants completed the study via a web-browser.

(1) Upon starting, they submitted informed consent, before reading task
instructions.

(2) Participants were introduced to a scenario in which they were to
play the role of a project manager, scheduling shifts for a group of
employees. The schedule of the employees was presented in a
TriangularModel (TM) graph, and they would be answering question about
the schedule.

(3) Then participants completed an experimental block of 15 items.

(3A) The first five items in the task are defined as the SCAFFOLDING
block. In the IMPASSE condition, the first five questions included an
IMPASSE problem state. For participants in the CONTROL condition, the
dataset was structure such that there was always an available
`orthogonal answer' for the first 5 questions.

(3B) The remaining 10 items are defined as the TESTING block. In both
conditions, these questions were not structured as impasse
(i.e.~contained an available orthogonal answer)

(4) Following the experimental block, participants answered a
free-response question about their strategy for reading the graph,
followed by a demographic questionnaire and debrief.

\hypertarget{sample}{%
\subsection{Sample}\label{sample}}

Data was collected by convenience sample of a university subject pool.
Initial data (Fall 2017, Spring 2018) were collected in-person, with
large groups of students simultaneously completing the study
(independently) in a computer lab. In Fall 2021 and Winter 2022 we
collected additional data to replicate results in a remote format
(students completing the study asynchronously on their own computers).

\hypertarget{analysis}{%
\section{ANALYSIS}\label{analysis}}

\hypertarget{sec-SGC3A-harmonize}{%
\subsection{Data Preparation}\label{sec-SGC3A-harmonize}}

Data were collected via a custom web application and stored in a NoSQL
database. The following exclusion criteria were applied during data
cleaning:

\begin{itemize}
\tightlist
\item
  completion status : ``success'' ; subject must have finished all parts
  of the study, including demographic questionnaire
\item
  session ID: {[}in list{]} ; subject must have been assigned to valid
  data collection session (discard testing and piloting data)
\item
  browser interaction violations \textless{} 3; subject must have fewer
  than 3 violations of non-allowed browser interactions (i.e.~resizing
  window, leaving browser tab or leaving fullscreen mode)
\item
  self-rated effort \textgreater{} 2; subjects who reported, ``not
  trying hard/rushing through questions'' or ``started out trying hard
  but giving up at some point'' were excluded from analysis.
\item
  attention check ==TRUE ; subjects who failed to answer a mid-study
  attention check question (Graph Comprehension Task Question \#6) are
  excluded
\end{itemize}

Before analysis, data files from individual data collection periods are
harmonized into a common data format.

\begin{longtable}[]{@{}
  >{\raggedright\arraybackslash}p{(\columnwidth - 2\tabcolsep) * \real{0.8382}}
  >{\raggedright\arraybackslash}p{(\columnwidth - 2\tabcolsep) * \real{0.1618}}@{}}
\toprule()
\begin{minipage}[b]{\linewidth}\raggedright
Pre-Requisite
\end{minipage} & \begin{minipage}[b]{\linewidth}\raggedright
Followed By
\end{minipage} \\
\midrule()
\endhead
spring17\_clean\_data.Rmd spring18\_clean\_data.Rmd
fall21\_clean\_data.Rmd winter2022\_clean\_sgc3a.Rmd &
2\_sgc3A\_scoring.qmd \\
\bottomrule()
\end{longtable}

Data for study SGC\_3A were collected across four time periods,
interrupted by the Covid-19 pandemic.

\begin{longtable}[]{@{}ll@{}}
\toprule()
Period & Modality \\
\midrule()
\endhead
Fall 2017 & in person, SONA groups in computer lab \\
Spring 2018 & in person, SONA groups in computer lab \\
Fall 2021 & asynchronous, online, SONA \\
Winter 2022 & asynchronous, online, SONA \\
\bottomrule()
\end{longtable}

Data collected in Fall 2017, Spring 2018 constitute the original SGC\_3A
study, conducted in person. Data collected in Fall 2021, Winter 2022
constitute the web-based replication, conducted online (asynchronously).
In all cases, the experiment was administered via a web application.

The underlying data structure of the stimulus web application changed
across the data collection period, resulting in slightly different data
files (i.e.~columns are not named consistently). In this section, we
combine the files from each data collection period into a single
\emph{harmonized} data file for analysis (one for participants, one for
items).

\hypertarget{participants}{%
\subsubsection{Participants}\label{participants}}

First we import participant-level data from each data collection period,
selecting only the columns relevant for analysis, and renaming columns
to be consistent across each file. The result is a single data frame
\texttt{df\_subjects} containing one row for each subject (across all
periods). Note that we \emph{are not} discarding any \emph{response}
data. Rather, we discard columns that are automatically recorded by the
stimulus web application and help the application run.

\emph{Note that we discard some columns representing scores calculated
in the stimulus engine. These scores were calculated differently across
collection periods, and so we discard them and recalculate scores in the
next analysis notebook. No raw data (responses and response times) are
discarded, only algorithmically-derived scores for the responses.}

\begin{Shaded}
\begin{Highlighting}[]
\CommentTok{\#IMPORT PARTICIPANT DATA}

\CommentTok{\# }\AlertTok{HACK}\CommentTok{ WD FOR LOCAL RUNNING?}
\CommentTok{\# imac = "/Users/amyraefox/Code/SGC{-}Scaffolding\_Graph\_Comprehension/SGC{-}X/ANALYSIS/MAIN"}
\CommentTok{\# \# mbp = "/Users/amyfox/Sites/RESEARCH/SGC—Scaffolding Graph Comprehension/SGC{-}X/ANALYSIS/MAIN"}
\CommentTok{\# setwd(imac)}

\CommentTok{\#set datafiles}
\NormalTok{fall17 }\OtherTok{\textless{}{-}} \StringTok{"analysis/SGC3A/data/0{-}session{-}level/fall17\_sgc3a\_participants.csv"}
\NormalTok{spring18 }\OtherTok{\textless{}{-}} \StringTok{"analysis/SGC3A/data/0{-}session{-}level/spring18\_sgc3a\_participants.csv"}
\NormalTok{fall21 }\OtherTok{\textless{}{-}} \StringTok{"analysis/SGC3A/data/0{-}session{-}level/fall21\_sgc3a\_participants.csv"}
\NormalTok{winter22 }\OtherTok{\textless{}{-}} \StringTok{"analysis/SGC3A/data/0{-}session{-}level/winter22\_sgc3a\_participants.rds"}

\CommentTok{\#read datafiles, set mode and term}
\NormalTok{df\_subjects\_fall17 }\OtherTok{\textless{}{-}} \FunctionTok{read\_csv}\NormalTok{(fall17) }\SpecialCharTok{\%\textgreater{}\%} \FunctionTok{mutate}\NormalTok{(}\AttributeTok{mode =} \StringTok{"lab{-}synch"}\NormalTok{, }\AttributeTok{term =} \StringTok{"fall17"}\NormalTok{)}
\NormalTok{df\_subjects\_spring18 }\OtherTok{\textless{}{-}} \FunctionTok{read\_csv}\NormalTok{(spring18) }\SpecialCharTok{\%\textgreater{}\%} \FunctionTok{mutate}\NormalTok{(}\AttributeTok{mode =} \StringTok{"lab{-}synch"}\NormalTok{, }\AttributeTok{term =} \StringTok{"spring18"}\NormalTok{)}
\NormalTok{df\_subjects\_fall21 }\OtherTok{\textless{}{-}} \FunctionTok{read\_csv}\NormalTok{(fall21) }\SpecialCharTok{\%\textgreater{}\%} \FunctionTok{mutate}\NormalTok{(}\AttributeTok{mode =} \StringTok{"asynch"}\NormalTok{, }\AttributeTok{term =} \StringTok{"fall21"}\NormalTok{)}
\NormalTok{df\_subjects\_winter22 }\OtherTok{\textless{}{-}} \FunctionTok{read\_rds}\NormalTok{(winter22) }\CommentTok{\#use RDS file as it contains metadata}

\CommentTok{\#SAVE METADATA FROM WINTER, but no rows }
\NormalTok{df\_subjects }\OtherTok{\textless{}{-}}\NormalTok{ df\_subjects\_winter22 }\SpecialCharTok{\%\textgreater{}\%} \FunctionTok{filter}\NormalTok{(condition}\SpecialCharTok{==}\StringTok{\textquotesingle{}X\textquotesingle{}}\NormalTok{) }\SpecialCharTok{\%\textgreater{}\%} 
\NormalTok{  dplyr}\SpecialCharTok{::}\FunctionTok{select}\NormalTok{(}
\NormalTok{  subject,condition,term,mode,}
\NormalTok{  gender,age,language, schoolyear, country,}
\NormalTok{  effort,difficulty,confidence,enjoyment,other,}
\NormalTok{  totaltime\_m, }
  \CommentTok{\# absolute\_score, \#drop absolute score as this is re{-}scored [though should be the same]}
  \CommentTok{\#exploratory factors}
\NormalTok{  violations, browser, width, height}
\NormalTok{)}

\CommentTok{\#reduce data collected using OLD webapp to useful columns}
\NormalTok{df\_subjects\_before }\OtherTok{\textless{}{-}} \FunctionTok{rbind}\NormalTok{(df\_subjects\_fall17, df\_subjects\_spring18, df\_subjects\_fall21) }\SpecialCharTok{\%\textgreater{}\%} 
  \CommentTok{\#rename and summarize some columns}
  \FunctionTok{mutate}\NormalTok{(}
    \AttributeTok{totaltime\_m =}\NormalTok{ totalTime }\SpecialCharTok{/} \DecValTok{1000} \SpecialCharTok{/} \DecValTok{60}\NormalTok{,  }
    \AttributeTok{absolute\_score =}\NormalTok{ triangular\_score,}
    \AttributeTok{language =}\NormalTok{ native\_language,}
    \AttributeTok{gender =}\NormalTok{ sex,}
    \AttributeTok{schoolyear =}\NormalTok{ year) }\SpecialCharTok{\%\textgreater{}\%} 
  \CommentTok{\#create placeholders for cols not collected until NEW webapp [for later rbind]}
  \FunctionTok{mutate}\NormalTok{(}
    \AttributeTok{effort =} \StringTok{"NULL"}\NormalTok{,}
    \AttributeTok{difficulty =} \StringTok{"NULL"}\NormalTok{,}
    \AttributeTok{confidence =} \StringTok{"NULL"}\NormalTok{,}
    \AttributeTok{enjoyment =} \StringTok{"NULL"}\NormalTok{,}
    \AttributeTok{other =} \StringTok{"NULL"}\NormalTok{,}
    \AttributeTok{disability =} \StringTok{"NULL"}\NormalTok{,}
    \AttributeTok{violations =} \StringTok{"NULL"}\NormalTok{,}
    \AttributeTok{browser =} \StringTok{"NULL"}\NormalTok{,}
    \AttributeTok{width =} \StringTok{"NULL"}\NormalTok{,}
    \AttributeTok{height =} \StringTok{"NULL"}
\NormalTok{  ) }\SpecialCharTok{\%\textgreater{}\%} 
  \CommentTok{\#select only columns we\textquotesingle{}ll be analyzing, discard others}
\NormalTok{  dplyr}\SpecialCharTok{::}\FunctionTok{select}\NormalTok{(subject, condition, term, mode, }
                \CommentTok{\#demographics}
\NormalTok{                gender, age, language, schoolyear, country,}
                \CommentTok{\#placeholder effort survey}
\NormalTok{                effort, difficulty, confidence, enjoyment, }
                \CommentTok{\#placeholder misc }
\NormalTok{                other, disability,}
                \CommentTok{\#response characteristics}
\NormalTok{                totaltime\_m, }
                \CommentTok{\# absolute\_score, \#drop absolute score as this is re{-}scored [though should be the same]}
                \CommentTok{\#exploratory factors}
\NormalTok{                violations, browser, width, height)}

\CommentTok{\#save \textquotesingle{}explanation\textquotesingle{} columns from winter22, which is actually a response to a free response item (Q16); was recorded with item\_level data in old webapp}
\NormalTok{df\_winter22\_q16 }\OtherTok{\textless{}{-}}\NormalTok{ df\_subjects\_winter22 }\SpecialCharTok{\%\textgreater{}\%} 
\NormalTok{  dplyr}\SpecialCharTok{::}\FunctionTok{select}\NormalTok{(subject, condition, term , mode, explanation) }\SpecialCharTok{\%\textgreater{}\%} 
  \FunctionTok{mutate}\NormalTok{(}
    \AttributeTok{q =} \DecValTok{16}\NormalTok{,}
    \AttributeTok{response =}\NormalTok{ explanation}
\NormalTok{  ) }\SpecialCharTok{\%\textgreater{}\%}\NormalTok{ dplyr}\SpecialCharTok{::}\FunctionTok{select}\NormalTok{(}\SpecialCharTok{{-}}\NormalTok{explanation)}

\CommentTok{\#reduce data collected using NEW webapp to useful columns}
\NormalTok{df\_subjects\_winter22 }\OtherTok{\textless{}{-}}\NormalTok{ df\_subjects\_winter22 }\SpecialCharTok{\%\textgreater{}\%} 
  \FunctionTok{mutate}\NormalTok{(}\AttributeTok{score =}\NormalTok{ absolute\_score) }\SpecialCharTok{\%\textgreater{}\%} 
  \CommentTok{\#select only columns we\textquotesingle{}ll be analyzing, discard others}
\NormalTok{  dplyr}\SpecialCharTok{::}\FunctionTok{select}\NormalTok{( subject, condition, term, mode, }
                 \CommentTok{\#demographics}
\NormalTok{                 gender, age, language, schoolyear, country,}
                 \CommentTok{\#effort survey}
\NormalTok{                 effort, difficulty, confidence, enjoyment, }
                 \CommentTok{\#explanations}
\NormalTok{                 other,disability,}
                 \CommentTok{\#response characteristics}
\NormalTok{                 totaltime\_m, }
                 \CommentTok{\# absolute\_score, \#drop absolute score as this is re{-}scored [though should be the same]}
                 \CommentTok{\#exploratory factors }
\NormalTok{                 violations, browser, width, height)}

\NormalTok{effort\_labels }\OtherTok{\textless{}{-}} \FunctionTok{c}\NormalTok{(}\StringTok{"I tried my best on each question"}\NormalTok{, }\StringTok{"I tried my best on most questions"}\NormalTok{)}

\CommentTok{\#combine dataframes from old and new webapps}
\NormalTok{df\_subjects }\OtherTok{\textless{}{-}} \FunctionTok{rbind}\NormalTok{(df\_subjects, df\_subjects\_winter22, df\_subjects\_before) }\SpecialCharTok{\%\textgreater{}\%} 
  \CommentTok{\#refactor factors}
  \FunctionTok{mutate}\NormalTok{ (}
    \AttributeTok{subject =} \FunctionTok{factor}\NormalTok{(subject),}
    \AttributeTok{condition =} \FunctionTok{factor}\NormalTok{(condition),}
    \AttributeTok{pretty\_condition =} \FunctionTok{recode\_factor}\NormalTok{(condition, }\StringTok{"111"} \OtherTok{=} \StringTok{"control"}\NormalTok{, }\StringTok{"121"} \OtherTok{=}  \StringTok{"impasse"}\NormalTok{),}
    \AttributeTok{pretty\_mode =} \FunctionTok{recode\_factor}\NormalTok{(mode, }\StringTok{"lab{-}synch"} \OtherTok{=} \StringTok{"laboratory"}\NormalTok{, }\StringTok{"asynch"} \OtherTok{=}  \StringTok{"online{-}replication"}\NormalTok{),}
    \AttributeTok{term =} \FunctionTok{factor}\NormalTok{(term, }\AttributeTok{levels=} \FunctionTok{c}\NormalTok{(}\StringTok{"fall17"}\NormalTok{,}\StringTok{"spring18"}\NormalTok{,}\StringTok{"fall21"}\NormalTok{,}\StringTok{"winter22"}\NormalTok{)),}
    \AttributeTok{mode =} \FunctionTok{factor}\NormalTok{(mode, }\AttributeTok{levels=}\FunctionTok{c}\NormalTok{(}\StringTok{"lab{-}synch"}\NormalTok{,}\StringTok{"asynch"}\NormalTok{)),}
    \AttributeTok{gender =} \FunctionTok{factor}\NormalTok{(gender),}
    \AttributeTok{schoolyear =} \FunctionTok{factor}\NormalTok{(schoolyear, }\AttributeTok{levels=}\FunctionTok{c}\NormalTok{(}\StringTok{"First"}\NormalTok{,}\StringTok{"Second"}\NormalTok{,}\StringTok{"Third"}\NormalTok{,}\StringTok{"Fourth"}\NormalTok{,}\StringTok{"Fifth"}\NormalTok{,}\StringTok{"Other"}\NormalTok{))}
\NormalTok{  )}

\CommentTok{\#FIX METADATA}
\CommentTok{\#Add metadata for columns that lost it [factors, for some reason!]}
\FunctionTok{var\_label}\NormalTok{(df\_subjects}\SpecialCharTok{$}\NormalTok{subject) }\OtherTok{\textless{}{-}} \StringTok{"ID of subject (randomly assigned in stimulus app)."}
\FunctionTok{var\_label}\NormalTok{(df\_subjects}\SpecialCharTok{$}\NormalTok{condition) }\OtherTok{\textless{}{-}} \StringTok{"ID indicates randomly assigned condition (111 {-}\textgreater{} control, 121 {-}\textgreater{} impasse)."}
\FunctionTok{var\_label}\NormalTok{(df\_subjects}\SpecialCharTok{$}\NormalTok{term) }\OtherTok{\textless{}{-}} \StringTok{"indicates if session was run with experimenter present or asynchronously"}
\FunctionTok{var\_label}\NormalTok{(df\_subjects}\SpecialCharTok{$}\NormalTok{mode) }\OtherTok{\textless{}{-}} \StringTok{"indicates mode in which the participant completed the study"}
\FunctionTok{var\_label}\NormalTok{(df\_subjects}\SpecialCharTok{$}\NormalTok{gender) }\OtherTok{\textless{}{-}} \StringTok{"What is your gender identity?"}
\FunctionTok{var\_label}\NormalTok{(df\_subjects}\SpecialCharTok{$}\NormalTok{schoolyear) }\OtherTok{\textless{}{-}} \StringTok{"What is your year in school?"}

\CommentTok{\#CLEANUP}
\FunctionTok{rm}\NormalTok{(df\_subjects\_fall17,df\_subjects\_fall21, df\_subjects\_spring18, df\_subjects\_winter22,df\_subjects\_before)}
\FunctionTok{rm}\NormalTok{(fall17,fall21,spring18,winter22)}
\end{Highlighting}
\end{Shaded}

\hypertarget{items}{%
\subsubsection{Items}\label{items}}

Next we import item-level data from each data collection period,
selecting only the columns relevant for analysis, and renaming columns
to be consistent across each file. The result is a single data frame
\texttt{df\_items} containing one row for each \emph{graph comprehension
task question} (qs=15) (across all periods). A second data frame
\texttt{df\_freeresponse} contains one row for each free response
strategy question (last question posed to participants in Winter2022)
Note that we \emph{do not} discard any \emph{response} data. Rather, we
\emph{do} discard several columns representing accuracy scores for
responses that were calculated in the stimulus engine. These scores were
calculated differently across collection periods, and so we discard them
and recalculate scores in the next analysis notebook. Original response
data are always preserved.

\begin{Shaded}
\begin{Highlighting}[]
\CommentTok{\# }\AlertTok{HACK}\CommentTok{ WD FOR LOCAL RUNNING?}
\CommentTok{\#  imac = "/Users/amyraefox/Code/SGC{-}Scaffolding\_Graph\_Comprehension/SGC{-}X/ANALYSIS/MAIN"}
\CommentTok{\# \#mbp = "/Users/amyfox/Sites/RESEARCH/SGC—Scaffolding Graph Comprehension/SGC{-}X/ANALYSIS/MAIN"}
\CommentTok{\# setwd(imac)}

\CommentTok{\#set datafiles}
\NormalTok{fall17 }\OtherTok{\textless{}{-}} \StringTok{"analysis/SGC3A/data/0{-}session{-}level/fall17\_sgc3a\_blocks.csv"}
\NormalTok{spring18 }\OtherTok{\textless{}{-}} \StringTok{"analysis/SGC3A/data/0{-}session{-}level/spring18\_sgc3a\_blocks.csv"}
\NormalTok{fall21 }\OtherTok{\textless{}{-}} \StringTok{"analysis/SGC3A/data/0{-}session{-}level/fall21\_sgc3a\_blocks.csv"}
\NormalTok{winter22 }\OtherTok{\textless{}{-}} \StringTok{"analysis/SGC3A/data/0{-}session{-}level/winter22\_sgc3a\_items.rds"}

\CommentTok{\#read datafiles, set mode and term}
\NormalTok{df\_items\_fall17 }\OtherTok{\textless{}{-}} \FunctionTok{read\_csv}\NormalTok{(fall17) }\SpecialCharTok{\%\textgreater{}\%} \FunctionTok{mutate}\NormalTok{(}\AttributeTok{mode =} \StringTok{"lab{-}synch"}\NormalTok{, }\AttributeTok{term =} \StringTok{"fall17"}\NormalTok{)}
\NormalTok{df\_items\_spring18 }\OtherTok{\textless{}{-}} \FunctionTok{read\_csv}\NormalTok{(spring18) }\SpecialCharTok{\%\textgreater{}\%} \FunctionTok{mutate}\NormalTok{(}\AttributeTok{mode =} \StringTok{"lab{-}synch"}\NormalTok{, }\AttributeTok{term =} \StringTok{"spring18"}\NormalTok{)}
\NormalTok{df\_items\_fall21 }\OtherTok{\textless{}{-}} \FunctionTok{read\_csv}\NormalTok{(fall21) }\SpecialCharTok{\%\textgreater{}\%} \FunctionTok{mutate}\NormalTok{(}\AttributeTok{mode =} \StringTok{"asynch"}\NormalTok{, }\AttributeTok{term =} \StringTok{"fall21"}\NormalTok{)}
\NormalTok{df\_items\_winter22 }\OtherTok{\textless{}{-}} \FunctionTok{read\_rds}\NormalTok{(winter22) }\CommentTok{\#use RDS file as it contains metadata}

\CommentTok{\#get mapping being question \# and interval relation the question tests, that is encoded only in the winter22 data files}
\NormalTok{map\_relations }\OtherTok{\textless{}{-}}\NormalTok{ df\_items\_winter22 }\SpecialCharTok{\%\textgreater{}\%} \FunctionTok{group\_by}\NormalTok{(q) }\SpecialCharTok{\%\textgreater{}\%} \FunctionTok{select}\NormalTok{(q,relation) }\SpecialCharTok{\%\textgreater{}\%} \FunctionTok{unique}\NormalTok{()}

\CommentTok{\#SAVE METADATA FROM WINTER, but no rows }
\NormalTok{df\_items }\OtherTok{\textless{}{-}}\NormalTok{ df\_items\_winter22 }\SpecialCharTok{\%\textgreater{}\%} \FunctionTok{filter}\NormalTok{(condition}\SpecialCharTok{==}\StringTok{\textquotesingle{}X\textquotesingle{}}\NormalTok{) }\SpecialCharTok{\%\textgreater{}\%} \FunctionTok{select}\NormalTok{(}
\NormalTok{  subject,condition,term,mode,}
\NormalTok{  question, q, answer, correct, rt\_s}
\NormalTok{) }
  
\CommentTok{\#reduce data collected using old webapp}
\NormalTok{df\_items\_before }\OtherTok{\textless{}{-}} \FunctionTok{rbind}\NormalTok{(df\_items\_fall17, df\_items\_spring18, df\_items\_fall21) }\SpecialCharTok{\%\textgreater{}\%} 
  \FunctionTok{mutate}\NormalTok{(}\AttributeTok{rt\_s =}\NormalTok{ rt }\SpecialCharTok{/} \DecValTok{1000}\NormalTok{, }\AttributeTok{correct =} \FunctionTok{as.logical}\NormalTok{(correct)) }\SpecialCharTok{\%\textgreater{}\%} 
  \FunctionTok{select}\NormalTok{(subject, condition, term, mode, question, q, answer, correct, rt\_s) }
  
\CommentTok{\#reduce data collected using new webapp}
\NormalTok{df\_items\_winter22 }\OtherTok{\textless{}{-}}\NormalTok{ df\_items\_winter22 }\SpecialCharTok{\%\textgreater{}\%} 
  \FunctionTok{select}\NormalTok{(subject, condition, term, mode, question, q, answer, correct, rt\_s) }\SpecialCharTok{\%\textgreater{}\%} \CommentTok{\#unfactor before combine}
  \FunctionTok{mutate}\NormalTok{(}
    \AttributeTok{subject =} \FunctionTok{as.character}\NormalTok{(subject),}
    \AttributeTok{condition =} \FunctionTok{as.character}\NormalTok{(condition),}
    \AttributeTok{term =} \FunctionTok{as.character}\NormalTok{(term),}
    \AttributeTok{mode =} \FunctionTok{as.character}\NormalTok{(mode),}
    \AttributeTok{q =} \FunctionTok{as.integer}\NormalTok{(q),}
    \AttributeTok{correct =} \FunctionTok{as.logical}\NormalTok{(correct)}
\NormalTok{  )}

\CommentTok{\#combine dataframes from old and new webapps}
\NormalTok{df\_items }\OtherTok{\textless{}{-}} \FunctionTok{rbind}\NormalTok{(df\_items, df\_items\_winter22,df\_items\_before) }\SpecialCharTok{\%\textgreater{}\%} 
  \CommentTok{\#refactorize columns}
  \FunctionTok{mutate}\NormalTok{(}
    \AttributeTok{subject =} \FunctionTok{factor}\NormalTok{(subject),}
    \AttributeTok{condition =} \FunctionTok{factor}\NormalTok{(condition),}
    \AttributeTok{term =} \FunctionTok{factor}\NormalTok{(term, }\AttributeTok{levels=} \FunctionTok{c}\NormalTok{(}\StringTok{"fall17"}\NormalTok{,}\StringTok{"spring18"}\NormalTok{,}\StringTok{"fall21"}\NormalTok{,}\StringTok{"winter22"}\NormalTok{)),}
    \AttributeTok{mode =} \FunctionTok{factor}\NormalTok{(mode, }\AttributeTok{levels=}\FunctionTok{c}\NormalTok{(}\StringTok{"lab{-}synch"}\NormalTok{,}\StringTok{"asynch"}\NormalTok{)),}
    \AttributeTok{q =} \FunctionTok{as.integer}\NormalTok{(q)) }\SpecialCharTok{\%\textgreater{}\%} 
  \CommentTok{\#rename answer column to RESPONSE }
  \FunctionTok{rename}\NormalTok{(}\AttributeTok{response =}\NormalTok{ answer) }\SpecialCharTok{\%\textgreater{}\%} 
  \CommentTok{\#remove all commas and make as character string}
  \FunctionTok{mutate}\NormalTok{(}
    \AttributeTok{response =} \FunctionTok{str\_remove\_all}\NormalTok{(}\FunctionTok{as.character}\NormalTok{(response), }\StringTok{","}\NormalTok{),}
    \AttributeTok{num\_o =} \FunctionTok{str\_length}\NormalTok{(response)}
\NormalTok{  ) }\SpecialCharTok{\%\textgreater{}\%} 
  \CommentTok{\# handle NA values (why are some empty responses blank and others NA?) }
  \FunctionTok{mutate}\NormalTok{(}
    \AttributeTok{response =} \FunctionTok{replace\_na}\NormalTok{(response, }\StringTok{""}\NormalTok{),}
    \AttributeTok{num\_o =} \FunctionTok{replace\_na}\NormalTok{(num\_o, }\DecValTok{0}\NormalTok{)}
\NormalTok{  )}


\CommentTok{\#FIX METADATA}
\CommentTok{\#Add metadata for columns that lost it [factors, for some reason!]}
\FunctionTok{var\_label}\NormalTok{(df\_items}\SpecialCharTok{$}\NormalTok{subject) }\OtherTok{\textless{}{-}} \StringTok{"ID of subject (randomly assigned in stimulus app)."}
\FunctionTok{var\_label}\NormalTok{(df\_items}\SpecialCharTok{$}\NormalTok{condition) }\OtherTok{\textless{}{-}} \StringTok{"ID indicates randomly assigned condition (111 {-}\textgreater{} control, 121 {-}\textgreater{} impasse)."}
\FunctionTok{var\_label}\NormalTok{(df\_items}\SpecialCharTok{$}\NormalTok{term) }\OtherTok{\textless{}{-}} \StringTok{"indicates if session was run with experimenter present or asynchronously"}
\FunctionTok{var\_label}\NormalTok{(df\_items}\SpecialCharTok{$}\NormalTok{mode) }\OtherTok{\textless{}{-}} \StringTok{"indicates mode in which the participant completed the study"}
\FunctionTok{var\_label}\NormalTok{(df\_items}\SpecialCharTok{$}\NormalTok{q) }\OtherTok{\textless{}{-}} \StringTok{"Question Number (in order)"}
\FunctionTok{var\_label}\NormalTok{(df\_items}\SpecialCharTok{$}\NormalTok{correct) }\OtherTok{\textless{}{-}} \StringTok{"Is the response (strictly) correct? [dichotomous scoring]"}
\FunctionTok{var\_label}\NormalTok{(df\_items}\SpecialCharTok{$}\NormalTok{response) }\OtherTok{\textless{}{-}} \StringTok{"options (datapoints) selected by the subject"}
\FunctionTok{var\_label}\NormalTok{(df\_items}\SpecialCharTok{$}\NormalTok{num\_o) }\OtherTok{\textless{}{-}} \StringTok{"number of options selected by the subject"}

\CommentTok{\#HANDLE FREE RESPONSE QUESTION \#16 }
\CommentTok{\#save \textasciigrave{}free response\textasciigrave{} Q\#16 in its own dataframe}
\NormalTok{df\_freeresponse }\OtherTok{\textless{}{-}}\NormalTok{ df\_items }\SpecialCharTok{\%\textgreater{}\%} \FunctionTok{filter}\NormalTok{(q }\SpecialCharTok{==} \DecValTok{16}\NormalTok{) }\SpecialCharTok{\%\textgreater{}\%} \FunctionTok{select}\NormalTok{(}\SpecialCharTok{{-}}\NormalTok{question,}\SpecialCharTok{{-}}\NormalTok{correct,}\SpecialCharTok{{-}}\NormalTok{rt\_s,}\SpecialCharTok{{-}}\NormalTok{num\_o)}
\CommentTok{\#add data from wi22 [stored on subject data]}
\NormalTok{df\_freeresponse }\OtherTok{\textless{}{-}} \FunctionTok{rbind}\NormalTok{(df\_freeresponse, df\_winter22\_q16)}
\CommentTok{\#add question description}
\NormalTok{df\_freeresponse }\OtherTok{\textless{}{-}}\NormalTok{ df\_freeresponse }\SpecialCharTok{\%\textgreater{}\%} \FunctionTok{mutate}\NormalTok{(}
    \AttributeTok{question =} \StringTok{"Please describe how to determine what event(s) start at 12pm?"}\NormalTok{,}
    \AttributeTok{response =} \FunctionTok{as.character}\NormalTok{(response) }\CommentTok{\#doesn\textquotesingle{}t need to be factor}
\NormalTok{  ) }
\CommentTok{\#remove \textquotesingle{}free response\textquotesingle{} Q\#16 from df\_items}
\NormalTok{df\_items }\OtherTok{\textless{}{-}}\NormalTok{ df\_items }\SpecialCharTok{\%\textgreater{}\%} \FunctionTok{filter}\NormalTok{ (q }\SpecialCharTok{!=} \DecValTok{16}\NormalTok{)}

\CommentTok{\#add back pretty condition }
\NormalTok{df\_items }\OtherTok{\textless{}{-}}\NormalTok{ df\_items }\SpecialCharTok{\%\textgreater{}\%} \FunctionTok{mutate}\NormalTok{(}
  \AttributeTok{pretty\_condition =} \FunctionTok{recode\_factor}\NormalTok{(condition, }\StringTok{"111"} \OtherTok{=} \StringTok{"control"}\NormalTok{, }\StringTok{"121"} \OtherTok{=}  \StringTok{"impasse"}\NormalTok{),}
  \AttributeTok{pretty\_mode =} \FunctionTok{recode\_factor}\NormalTok{(mode, }\StringTok{"lab{-}synch"} \OtherTok{=} \StringTok{"laboratory"}\NormalTok{, }\StringTok{"asynch"} \OtherTok{=}  \StringTok{"online{-}replication"}\NormalTok{)}
\NormalTok{) }

\CommentTok{\#CLEANUP}
\FunctionTok{rm}\NormalTok{(df\_items\_fall17,df\_items\_fall21, df\_items\_spring18, df\_items\_winter22, df\_items\_before, df\_winter22\_q16)}
\FunctionTok{rm}\NormalTok{(fall17,fall21,spring18,winter22, map\_relations)}
\end{Highlighting}
\end{Shaded}

\hypertarget{validation}{%
\subsubsection{Validation}\label{validation}}

Next, we validate that we have the complete number of item-level records
based on the number of subject-level records

\begin{Shaded}
\begin{Highlighting}[]
\CommentTok{\#the number of items should be equal to 15 x the number of subjects}
\FunctionTok{nrow}\NormalTok{(df\_items) }\SpecialCharTok{==} \DecValTok{15}\SpecialCharTok{*} \FunctionTok{nrow}\NormalTok{(df\_subjects) }\CommentTok{\#TRUE}
\end{Highlighting}
\end{Shaded}

\begin{verbatim}
[1] TRUE
\end{verbatim}

\begin{Shaded}
\begin{Highlighting}[]
\CommentTok{\#each subject should have 15 items}
\NormalTok{df\_items }\SpecialCharTok{\%\textgreater{}\%} \FunctionTok{group\_by}\NormalTok{(subject) }\SpecialCharTok{\%\textgreater{}\%} \FunctionTok{summarise}\NormalTok{(}\AttributeTok{n =} \FunctionTok{n}\NormalTok{()) }\SpecialCharTok{\%\textgreater{}\%} \FunctionTok{filter}\NormalTok{(n }\SpecialCharTok{!=} \DecValTok{15}\NormalTok{) }\SpecialCharTok{\%\textgreater{}\%} \FunctionTok{nrow}\NormalTok{() }\SpecialCharTok{==} \DecValTok{0}
\end{Highlighting}
\end{Shaded}

\begin{verbatim}
[1] TRUE
\end{verbatim}

\hypertarget{export}{%
\subsubsection{Export}\label{export}}

Finally, we export the (session-harmonized) data for analysis, as CSVs,
and .RDS (includes metadata)

\begin{Shaded}
\begin{Highlighting}[]
\CommentTok{\# }\AlertTok{HACK}\CommentTok{ WD FOR LOCAL RUNNING?}
\CommentTok{\# imac = "/Users/amyraefox/Code/SGC{-}Scaffolding\_Graph\_Comprehension/SGC{-}X/ANALYSIS/MAIN"}
\CommentTok{\# \#mbp = "/Users/amyfox/Sites/RESEARCH/SGC—Scaffolding Graph Comprehension/SGC{-}X/ANALYSIS/MAIN"}
\CommentTok{\# setwd(imac)}

\CommentTok{\#SAVE FILES}
\FunctionTok{write.csv}\NormalTok{(df\_subjects,}\StringTok{"analysis/SGC3A/data/1{-}study{-}level/sgc3a\_participants.csv"}\NormalTok{, }\AttributeTok{row.names =} \ConstantTok{FALSE}\NormalTok{)}
\FunctionTok{write.csv}\NormalTok{(df\_items,}\StringTok{"analysis/SGC3A/data/1{-}study{-}level/sgc3a\_items.csv"}\NormalTok{, }\AttributeTok{row.names =} \ConstantTok{FALSE}\NormalTok{)}
\FunctionTok{write.csv}\NormalTok{(df\_freeresponse,}\StringTok{"analysis/SGC3A/data/1{-}study{-}level/sgc3a\_freeresponse.csv"}\NormalTok{, }\AttributeTok{row.names =} \ConstantTok{FALSE}\NormalTok{)}

\CommentTok{\#SAVE R Data Structures }
\CommentTok{\#export R DATA STRUCTURES (include codebook metadata)}
\NormalTok{rio}\SpecialCharTok{::}\FunctionTok{export}\NormalTok{(df\_subjects, }\StringTok{"analysis/SGC3A/data/1{-}study{-}level/sgc3a\_participants.rds"}\NormalTok{) }\CommentTok{\# to R data structure file}
\NormalTok{rio}\SpecialCharTok{::}\FunctionTok{export}\NormalTok{(df\_items, }\StringTok{"analysis/SGC3A/data/1{-}study{-}level/sgc3a\_items.rds"}\NormalTok{) }\CommentTok{\# to R data structure file}
\end{Highlighting}
\end{Shaded}

\hypertarget{resources}{%
\section{RESOURCES}\label{resources}}

\begin{Shaded}
\begin{Highlighting}[]
\FunctionTok{sessionInfo}\NormalTok{()}
\end{Highlighting}
\end{Shaded}

\begin{verbatim}
R version 4.2.1 (2022-06-23)
Platform: x86_64-apple-darwin17.0 (64-bit)
Running under: macOS Big Sur ... 10.16

Matrix products: default
BLAS:   /Library/Frameworks/R.framework/Versions/4.2/Resources/lib/libRblas.0.dylib
LAPACK: /Library/Frameworks/R.framework/Versions/4.2/Resources/lib/libRlapack.dylib

locale:
[1] en_US.UTF-8/en_US.UTF-8/en_US.UTF-8/C/en_US.UTF-8/en_US.UTF-8

attached base packages:
[1] stats     graphics  grDevices utils     datasets  methods   base     

other attached packages:
 [1] kableExtra_1.3.4 forcats_0.5.1    stringr_1.4.0    dplyr_1.0.9     
 [5] purrr_0.3.4      readr_2.1.2      tidyr_1.2.0      tibble_3.1.7    
 [9] ggplot2_3.3.6    tidyverse_1.3.1  codebook_0.9.2  

loaded via a namespace (and not attached):
 [1] Rcpp_1.0.8.3      svglite_2.1.0     lubridate_1.8.0   assertthat_0.2.1 
 [5] digest_0.6.29     utf8_1.2.2        R6_2.5.1          cellranger_1.1.0 
 [9] backports_1.4.1   reprex_2.0.1      labelled_2.9.1    evaluate_0.15    
[13] httr_1.4.3        pillar_1.7.0      rlang_1.0.3       curl_4.3.2       
[17] readxl_1.4.0      data.table_1.14.2 rstudioapi_0.13   rmarkdown_2.14   
[21] webshot_0.5.3     foreign_0.8-82    bit_4.0.4         munsell_0.5.0    
[25] broom_0.8.0       compiler_4.2.1    modelr_0.1.8      xfun_0.31        
[29] pkgconfig_2.0.3   systemfonts_1.0.4 htmltools_0.5.2   tidyselect_1.1.2 
[33] rio_0.5.29        fansi_1.0.3       viridisLite_0.4.0 crayon_1.5.1     
[37] tzdb_0.3.0        dbplyr_2.2.1      withr_2.5.0       grid_4.2.1       
[41] jsonlite_1.8.0    gtable_0.3.0      lifecycle_1.0.1   DBI_1.1.3        
[45] magrittr_2.0.3    scales_1.2.0      zip_2.2.0         cli_3.3.0        
[49] stringi_1.7.6     vroom_1.5.7       fs_1.5.2          xml2_1.3.3       
[53] ellipsis_0.3.2    generics_0.1.2    vctrs_0.4.1       openxlsx_4.2.5   
[57] tools_4.2.1       bit64_4.0.5       glue_1.6.2        hms_1.1.1        
[61] parallel_4.2.1    fastmap_1.1.0     yaml_2.3.5        colorspace_2.0-3 
[65] rvest_1.0.2       knitr_1.39        haven_2.5.0      
\end{verbatim}

\newpage

\hypertarget{sec-SGC3A-scoring}{%
\chapter{Response Scoring}\label{sec-SGC3A-scoring}}

\textbf{TODO}

\begin{itemize}
\tightlist
\item
  finish item level response exploration
\item
  TODO: generate heat maps of Q9. Same answer but very different optimal
  operation paths!
\item
  see individual item level todos on response exploration
\end{itemize}

\emph{The purpose of this notebook is to score (assign a measure of
accuracy) to response data for the SGC3A study. This is required because
the question type on the graph comprehension task used a `Multiple
Response' (MR) question design. Here, we evaluate different approaches
to scoring multiple response questions, and transform raw item responses
(e.g.~boxes ABC are checked) to a measure of response accuracy.
(Warning: this notebook takes several minutes to execute.)} To review
the strategy behind Multiple Response scoring for the SGC project, refer
to section \textbf{?@sec-scoring}.

\begin{longtable}[]{@{}l@{}}
\toprule()
Pre-Requisite \\
\midrule()
\endhead
1\_sgc3A\_harmonize.qmd \\
\bottomrule()
\end{longtable}

\begin{Shaded}
\begin{Highlighting}[]
\FunctionTok{options}\NormalTok{(}\AttributeTok{scipen=}\DecValTok{1}\NormalTok{, }\AttributeTok{digits=}\DecValTok{3}\NormalTok{)}

\FunctionTok{library}\NormalTok{(kableExtra) }\CommentTok{\#printing tables }
\FunctionTok{library}\NormalTok{(ggformula) }\CommentTok{\#quick graphs}
\FunctionTok{library}\NormalTok{(pbapply) }\CommentTok{\#progress bar and time estimate for *apply fns}
\FunctionTok{library}\NormalTok{(Hmisc) }\CommentTok{\# \%nin\% operator}
\FunctionTok{library}\NormalTok{(tidyverse) }\CommentTok{\#ALL THE THINGS}
\end{Highlighting}
\end{Shaded}

\hypertarget{score-sgc-data}{%
\section{SCORE SGC DATA}\label{score-sgc-data}}

To review the strategy behind Multiple Response scoring for the SGC
project, refer to section \textbf{?@sec-scoring}.

In SGC we are fundamentally interested in understanding how a
participant interprets the presented graph (stimulus). The \textbf{graph
comprehension task} asks them to select the data points in the graph
that meet the criteria posed in the question. To assess a participant's
performance, for each question (q=15) we will calculate the following
scores:

\emph{An overall, strict score:}\\
1. \textbf{Absolute Score} : using dichotomous scoring referencing true
(Triangular) answer. (see 1.2)

\emph{Sub-scores, for each alternative graph interpretation}\\
2. \textbf{Triangular Score} : using partial scoring {[}-1/q, +1/p{]}
referencing true (Triangular) answer key.

3. \textbf{Orthogonal Score} : using partial scoring {[}-1/q, +1/p{]}
referencing (incorrect Orthogonal) answer key.

Based on prior observational studies where we observed emergence of
other alternative interpretations (i.e.~transitional interpretations) we
also calculate subscores for these alternatives.

4. \textbf{Tversky Score} : using partial scoring {[}-1/q, +1/p{]}
referencing (incorrect connecting-lines strategy) answer key. 5.
\textbf{Satisficing Score} : using partial scoring {[}-1/q, +1/p{]}
referencing (incorrect satisficing strategy) answer key.

\hypertarget{sec-SGC3A-keys}{%
\subsection{Prepare Answer Keys}\label{sec-SGC3A-keys}}

We start by importing three answer keys: (1) Q1 - Q5 {[}control
condition{]}, (2) Q1-Q5 {[}impasse condition{]}, (3) Q6-15. Separate
answer keys by condition are required for Q1-Q5 because the stimuli for
each condition visualize a different underlying dataset (i.e.~the graphs
show datapoints in different positions). Q6-Q15 are identical across
conditions. Each answer key includes a row for each question, and a
column defining the subset of response options that correspond to
different graph interpretations.

\begin{Shaded}
\begin{Highlighting}[]
\CommentTok{\# \#HACK WD FOR LOCAL RUNNING?}
\CommentTok{\#imac = "/Users/amyraefox/Code/SGC{-}Scaffolding\_Graph\_Comprehension/SGC{-}X/ANALYSIS/MAIN"}
\CommentTok{\#setwd(imac)}

\CommentTok{\#SAVE KEYS FOR FUTURE USE}
\NormalTok{keys\_raw }\OtherTok{\textless{}{-}}  \FunctionTok{read\_csv}\NormalTok{(}\StringTok{"analysis/utils/keys/parsed\_keys/keys\_raw"}\NormalTok{)}
\NormalTok{keys\_orth }\OtherTok{\textless{}{-}}  \FunctionTok{read\_csv}\NormalTok{(}\StringTok{"analysis/utils/keys/parsed\_keys/keys\_orth"}\NormalTok{)}
\NormalTok{keys\_tri }\OtherTok{\textless{}{-}}  \FunctionTok{read\_csv}\NormalTok{(}\StringTok{"analysis/utils/keys/parsed\_keys/keys\_tri"}\NormalTok{)}
\NormalTok{keys\_satisfice\_left }\OtherTok{\textless{}{-}}  \FunctionTok{read\_csv}\NormalTok{(}\StringTok{"analysis/utils/keys/parsed\_keys/keys\_satisfice\_left"}\NormalTok{)}
\NormalTok{keys\_satisfice\_right }\OtherTok{\textless{}{-}}  \FunctionTok{read\_csv}\NormalTok{(}\StringTok{"analysis/utils/keys/parsed\_keys/keys\_satisfice\_right"}\NormalTok{)}
\NormalTok{keys\_tversky\_duration }\OtherTok{\textless{}{-}}  \FunctionTok{read\_csv}\NormalTok{(}\StringTok{"analysis/utils/keys/parsed\_keys/keys\_tversky\_duration"}\NormalTok{)}
\NormalTok{keys\_tversky\_end }\OtherTok{\textless{}{-}}  \FunctionTok{read\_csv}\NormalTok{(}\StringTok{"analysis/utils/keys/parsed\_keys/keys\_tversky\_end"}\NormalTok{)}
\NormalTok{keys\_tversky\_max }\OtherTok{\textless{}{-}}  \FunctionTok{read\_csv}\NormalTok{(}\StringTok{"analysis/utils/keys/parsed\_keys/keys\_tversky\_max"}\NormalTok{)}
\NormalTok{keys\_tversky\_start }\OtherTok{\textless{}{-}}  \FunctionTok{read\_csv}\NormalTok{(}\StringTok{"analysis/utils/keys/parsed\_keys/keys\_tversky\_start"}\NormalTok{)}
\end{Highlighting}
\end{Shaded}

\hypertarget{sec-SGC3A-subscores}{%
\subsection{Calculate Subscores}\label{sec-SGC3A-subscores}}

Next, we import the item-level response data. For each row in the item
level dataset (indicating the response to a single question-item for a
single subject), we compare the raw response
\texttt{df\_items\$response} with the answer keys in each interpretation
(e.g.~\texttt{keys\_orth}, \texttt{keys\_tri}, etc\ldots), then using
those sets, determine the number of correctly selected items(p) and
incorrectly selected items (q), which in turn are used to calculate
partial{[}-1/q, +1/p{]} scores for each interpretation. The resulting
scores are then stored on each item in \texttt{df\_items}, and can be
used to determine which graph interpretation the subject held.

Specifically, the following scores are calculated for each item:

\textbf{Interpretation Subscores}

\begin{itemize}
\tightlist
\item
  \texttt{score\_TRI} How consistent is the response with the
  \textbf{triangular}interpretation?
\item
  \texttt{score\_ORTH} How consistent is the response with the
  \textbf{orthogonal}interpretation?
\item
  \texttt{score\_SATISFICE} is calculated by taking the maximum value of
  :

  \begin{itemize}
  \tightlist
  \item
    \texttt{score\_SAT\_left} How consistent is the response with the
    \textbf{(left side) Satisficing} interpretation?
  \item
    \texttt{score\_SAT\_right} How consistent is the response with the
    \textbf{(right side) Satisficing} interpretation
  \end{itemize}
\item
  \texttt{score\_TVERSKY} is calculated by taking the maximum value of:

  \begin{itemize}
  \tightlist
  \item
    \texttt{score\_TV\_max} How consistent is the response with the
    \textbf{(maximal) Tversky} interpretation?
  \item
    \texttt{score\_TV\_start} How consistent is the response with the
    \textbf{(start-time) Tversky} interpretation?
  \item
    \texttt{score\_TV\_end} How consistent is the response with the
    \textbf{(end-time) Tversky} interpretation?
  \item
    \texttt{score\_TV\_duration} How consistent is the response with the
    \textbf{(duration) Tversky} interpretation?
  \end{itemize}
\item
  \texttt{score\_REF} Did the response select only the \textbf{reference
  point}?
\item
  \texttt{score\_BOTH} How consistent is the response with \textbf{both}
  the orthogonal and triangular interpretations?
\end{itemize}

\textbf{Absolute Scores}

\begin{itemize}
\tightlist
\item
  \texttt{score\_ABS} Is the response strictly correct? (triangular
  interpretation)
\item
  \texttt{score\_niceABS} Is the response strictly correct? (triangular
  interpretation, not penalizing ref points). This is a more generous
  version of the Absolute score that does not penalize the participant
  if in addition to the correct answer \emph{in addition to} they also
  select the data point referenced in the question.
\end{itemize}

\begin{Shaded}
\begin{Highlighting}[]
\CommentTok{\#HACK WD FOR LOCAL RUNNING?}
\NormalTok{imac }\OtherTok{=} \StringTok{"/Users/amyraefox/Code/SGC{-}Scaffolding\_Graph\_Comprehension/SGC{-}X/ANALYSIS/MAIN"}
\FunctionTok{setwd}\NormalTok{(imac)}

\CommentTok{\#backup \textless{}{-} read\_rds(\textquotesingle{}analysis/SGC3A/data/1{-}study{-}level/sgc3a\_items.rds\textquotesingle{}) \#for troubleshooting only}
\NormalTok{df\_items }\OtherTok{\textless{}{-}} \FunctionTok{read\_rds}\NormalTok{(}\StringTok{\textquotesingle{}analysis/SGC3A/data/1{-}study{-}level/sgc3a\_items.rds\textquotesingle{}}\NormalTok{)}
\end{Highlighting}
\end{Shaded}

\begin{Shaded}
\begin{Highlighting}[]
\CommentTok{\# \#HACK WD FOR LOCAL RUNNING?}
\CommentTok{\# imac = "/Users/amyraefox/Code/SGC{-}Scaffolding\_Graph\_Comprehension/SGC{-}X/ANALYSIS/MAIN"}
\CommentTok{\# setwd(imac)}

\FunctionTok{source}\NormalTok{(}\StringTok{"analysis/utils/scoring.R"}\NormalTok{)}
\end{Highlighting}
\end{Shaded}

\emph{note: this cell takes approximately 30 minutes to run on the full
df\_items dataframe with 4950 records}

\begin{Shaded}
\begin{Highlighting}[]
\CommentTok{\#RUN THIS \textless{}OR\textgreater{} THE CALCULATE{-}SCORES{-}FORLOOP [not both]}

\CommentTok{\#ALPHEBETIZE RESPONSE}
\NormalTok{df\_items}\SpecialCharTok{$}\NormalTok{response }\OtherTok{=} \FunctionTok{pbmapply}\NormalTok{(reorder\_inplace, df\_items}\SpecialCharTok{$}\NormalTok{response)}

\CommentTok{\#STRATEGY PARTIAL{-}SUBSCORES}
\NormalTok{df\_items}\SpecialCharTok{$}\NormalTok{score\_TRI }\OtherTok{=} \FunctionTok{pbmapply}\NormalTok{(calc\_subscore, df\_items}\SpecialCharTok{$}\NormalTok{q, df\_items}\SpecialCharTok{$}\NormalTok{condition, df\_items}\SpecialCharTok{$}\NormalTok{response, }\AttributeTok{MoreArgs =} \FunctionTok{list}\NormalTok{(}\AttributeTok{keyframe =}\NormalTok{ keys\_tri))}
\NormalTok{df\_items}\SpecialCharTok{$}\NormalTok{score\_ORTH }\OtherTok{=} \FunctionTok{pbmapply}\NormalTok{(calc\_subscore, df\_items}\SpecialCharTok{$}\NormalTok{q, df\_items}\SpecialCharTok{$}\NormalTok{condition, df\_items}\SpecialCharTok{$}\NormalTok{response, }\AttributeTok{MoreArgs =} \FunctionTok{list}\NormalTok{(}\AttributeTok{keyframe =}\NormalTok{ keys\_orth))}
\NormalTok{df\_items}\SpecialCharTok{$}\NormalTok{score\_SAT\_left }\OtherTok{=} \FunctionTok{pbmapply}\NormalTok{(calc\_subscore, df\_items}\SpecialCharTok{$}\NormalTok{q, df\_items}\SpecialCharTok{$}\NormalTok{condition, df\_items}\SpecialCharTok{$}\NormalTok{response, }\AttributeTok{MoreArgs =} \FunctionTok{list}\NormalTok{(}\AttributeTok{keyframe =}\NormalTok{ keys\_satisfice\_left))}
\NormalTok{df\_items}\SpecialCharTok{$}\NormalTok{score\_SAT\_right }\OtherTok{=} \FunctionTok{pbmapply}\NormalTok{(calc\_subscore, df\_items}\SpecialCharTok{$}\NormalTok{q, df\_items}\SpecialCharTok{$}\NormalTok{condition, df\_items}\SpecialCharTok{$}\NormalTok{response, }\AttributeTok{MoreArgs =} \FunctionTok{list}\NormalTok{(}\AttributeTok{keyframe =}\NormalTok{ keys\_satisfice\_right))}
\NormalTok{df\_items}\SpecialCharTok{$}\NormalTok{score\_TV\_max }\OtherTok{=} \FunctionTok{pbmapply}\NormalTok{(calc\_subscore, df\_items}\SpecialCharTok{$}\NormalTok{q, df\_items}\SpecialCharTok{$}\NormalTok{condition, df\_items}\SpecialCharTok{$}\NormalTok{response, }\AttributeTok{MoreArgs =} \FunctionTok{list}\NormalTok{(}\AttributeTok{keyframe =}\NormalTok{ keys\_tversky\_max))}
\NormalTok{df\_items}\SpecialCharTok{$}\NormalTok{score\_TV\_start }\OtherTok{=} \FunctionTok{pbmapply}\NormalTok{(calc\_subscore, df\_items}\SpecialCharTok{$}\NormalTok{q, df\_items}\SpecialCharTok{$}\NormalTok{condition, df\_items}\SpecialCharTok{$}\NormalTok{response, }\AttributeTok{MoreArgs =} \FunctionTok{list}\NormalTok{(}\AttributeTok{keyframe =}\NormalTok{ keys\_tversky\_start))}
\NormalTok{df\_items}\SpecialCharTok{$}\NormalTok{score\_TV\_end }\OtherTok{=} \FunctionTok{pbmapply}\NormalTok{(calc\_subscore, df\_items}\SpecialCharTok{$}\NormalTok{q, df\_items}\SpecialCharTok{$}\NormalTok{condition, df\_items}\SpecialCharTok{$}\NormalTok{response, }\AttributeTok{MoreArgs =} \FunctionTok{list}\NormalTok{(}\AttributeTok{keyframe =}\NormalTok{ keys\_tversky\_end))}
\NormalTok{df\_items}\SpecialCharTok{$}\NormalTok{score\_TV\_duration }\OtherTok{=} \FunctionTok{pbmapply}\NormalTok{(calc\_subscore, df\_items}\SpecialCharTok{$}\NormalTok{q, df\_items}\SpecialCharTok{$}\NormalTok{condition, df\_items}\SpecialCharTok{$}\NormalTok{response, }\AttributeTok{MoreArgs =} \FunctionTok{list}\NormalTok{(}\AttributeTok{keyframe =}\NormalTok{ keys\_tversky\_duration))}

\CommentTok{\#SPECIAL ABSOLUTE SUBSCORES}
\NormalTok{df\_items}\SpecialCharTok{$}\NormalTok{score\_REF }\OtherTok{=} \FunctionTok{pbmapply}\NormalTok{(calc\_refscore, df\_items}\SpecialCharTok{$}\NormalTok{q, df\_items}\SpecialCharTok{$}\NormalTok{response)}
\NormalTok{df\_items}\SpecialCharTok{$}\NormalTok{score\_BOTH }\OtherTok{=} \FunctionTok{as.integer}\NormalTok{((df\_items}\SpecialCharTok{$}\NormalTok{score\_TRI }\SpecialCharTok{==} \DecValTok{1}\NormalTok{) }\SpecialCharTok{\&}\NormalTok{ (df\_items}\SpecialCharTok{$}\NormalTok{score\_ORTH }\SpecialCharTok{==}\DecValTok{1}\NormalTok{))}

\CommentTok{\#ABSOLUTE SCORES}
\NormalTok{df\_items}\SpecialCharTok{$}\NormalTok{score\_ABS }\OtherTok{=} \FunctionTok{as.integer}\NormalTok{(df\_items}\SpecialCharTok{$}\NormalTok{correct) }
\NormalTok{df\_items}\SpecialCharTok{$}\NormalTok{score\_niceABS  }\OtherTok{\textless{}{-}} \FunctionTok{as.integer}\NormalTok{((df\_items}\SpecialCharTok{$}\NormalTok{score\_TRI }\SpecialCharTok{==} \DecValTok{1}\NormalTok{)) }\CommentTok{\#tri doesn\textquotesingle{}t penalize ref or ve{-}area}
\end{Highlighting}
\end{Shaded}

\hypertarget{sec-SGC3A-interpretation}{%
\subsection{Derive Interpretation}\label{sec-SGC3A-interpretation}}

Finally, we use the interpretation subscores to classify the response as
a particular interpretation. This classification algorithm : (1) First
decides if the response matches one or more `special' situations (blank
response, reference point response, both ORTH and TRI) (2) If response
doesn't match a special situation, it compares the individual subscores,
and subscores and decides if they are \emph{discriminant} (i.e.~are the
scores different enough to make a prediction). A discriminant threshold
of 0.5pts (on a scale from -1 to +1 is used) (2) If the variance in
subscores surpasses the threshold, the interpretation is classified
based on the highest subscore (TRIANGULAR, ORTHOGONAL, TVERSKY,
SATISFICE) (3) If the variance does not surpass the threshold, the
interpretation is labelled as ``?'', indicating it cannot be classified,
and is of an unknown interpretation.

The final output is called \texttt{interpretation}.

\begin{Shaded}
\begin{Highlighting}[]
\CommentTok{\#stoopid extra copying for troubleshooting safety}
\NormalTok{temp }\OtherTok{\textless{}{-}}\NormalTok{ df\_items }
\NormalTok{temp }\OtherTok{\textless{}{-}} \FunctionTok{derive\_interpretation}\NormalTok{(temp)}
\NormalTok{df\_items }\OtherTok{\textless{}{-}}\NormalTok{ temp }
\end{Highlighting}
\end{Shaded}

\hypertarget{sec-SGC3A-scaledScore}{%
\subsection{Derive Scaled Score}\label{sec-SGC3A-scaledScore}}

The \texttt{interpretation} response variable gives us the finest grain
indication of the reader's understanding of the graph for a particular
question. However, it is a categorical variable, which poses a challenge
for analyzing cumulative performance at the subject level. To address
this challenge, we derive a \emph{scaled\_score} that converts each
possible interpretation to a numeric value on a scale from -1 to +1.
This scaling takes advantage of the observation that each interpretation
can be positioned along a spectrum of understanding from completely
incorrect (orthogonal) to completely correct (triangular). Alternative
interpretations lay somewhere between.

Specifically, we assign the following values to each interpretation:

\begin{itemize}
\tightlist
\item
  (-1) : ORTHOGONAL, SATISFICE (satisfice represents an attempt at an
  orthogonal answer when none is available)
\item
  (-0.5): ? (some alternative that cannot be identified; but meaningful
  that it is not orthogonal)
\item
  (0): REFERENCE POINT, BLANK (indicates the individual thinks there is
  no answer, recognizes that ORTHOGONAL cannot be correct, but does not
  conceive of triangular)
\item
  (+0.5) TVERSKY, BOTH TRI + ORTH (indicates that they ``see'' a
  triangular response, but lack certainty and also select the ORTHOGONAL
  response)
\item
  (+1) TRIANGULAR +1
\end{itemize}

\begin{Shaded}
\begin{Highlighting}[]
\NormalTok{df\_items}\SpecialCharTok{$}\NormalTok{score\_SCALED }\OtherTok{\textless{}{-}} \FunctionTok{calc\_scaled}\NormalTok{(df\_items}\SpecialCharTok{$}\NormalTok{interpretation)}
\end{Highlighting}
\end{Shaded}

\hypertarget{summarize-by-subject}{%
\section{SUMMARIZE BY SUBJECT}\label{summarize-by-subject}}

Next, we summarize the item level scores by subject in order to
calculate cummulative subscores to be stored on the subject record.

\begin{Shaded}
\begin{Highlighting}[]
\CommentTok{\# \#HACK WD FOR LOCAL RUNNING?}
\NormalTok{imac }\OtherTok{=} \StringTok{"/Users/amyraefox/Code/SGC{-}Scaffolding\_Graph\_Comprehension/SGC{-}X/ANALYSIS/MAIN"}
\FunctionTok{setwd}\NormalTok{(imac)}

\CommentTok{\#import subjects}
\NormalTok{df\_subjects }\OtherTok{\textless{}{-}} \FunctionTok{read\_rds}\NormalTok{(}\StringTok{\textquotesingle{}analysis/SGC3A/data/1{-}study{-}level/sgc3a\_participants.rds\textquotesingle{}}\NormalTok{) }\SpecialCharTok{\%\textgreater{}\%} \FunctionTok{mutate}\NormalTok{(}\AttributeTok{subject =} \FunctionTok{as.character}\NormalTok{(subject)) }\SpecialCharTok{\%\textgreater{}\%} \FunctionTok{arrange}\NormalTok{(subject)}

\CommentTok{\#make temporary copies for testing safety}
\NormalTok{s }\OtherTok{=}\NormalTok{ df\_subjects}
\NormalTok{i }\OtherTok{=}\NormalTok{ df\_items }

\CommentTok{\#summarize}
\NormalTok{test\_subs }\OtherTok{\textless{}{-}} \FunctionTok{summarise\_bySubject}\NormalTok{(s,i)}
\end{Highlighting}
\end{Shaded}

\begin{verbatim}
`summarise()` has grouped output by 'subject'. You can override using the
`.groups` argument.
\end{verbatim}

\begin{verbatim}
[1] TRUE
[1] TRUE
[1] TRUE
[1] TRUE
[1] TRUE
[1] TRUE
[1] TRUE
\end{verbatim}

\begin{Shaded}
\begin{Highlighting}[]
\NormalTok{df\_subjects }\OtherTok{\textless{}{-}}\NormalTok{ test\_subs}
\end{Highlighting}
\end{Shaded}

We also summarize absolute and scaled score progress at each question in
the task, to explore cumulative performance over the task.

\begin{Shaded}
\begin{Highlighting}[]
\CommentTok{\#GET ABSOLUTE PROGRESS }
\NormalTok{df\_absolute\_progress }\OtherTok{\textless{}{-}} \FunctionTok{progress\_Absolute}\NormalTok{(df\_items)}

\CommentTok{\#GET SCALED PROGRESS}
\NormalTok{df\_scaled\_progress }\OtherTok{\textless{}{-}} \FunctionTok{progress\_Scaled}\NormalTok{(df\_items)}
\end{Highlighting}
\end{Shaded}

\hypertarget{explore-distributions}{%
\section{EXPLORE DISTRIBUTIONS}\label{explore-distributions}}

\begin{Shaded}
\begin{Highlighting}[]
\FunctionTok{options}\NormalTok{(}\AttributeTok{repr.plot.width =}\DecValTok{9}\NormalTok{, }\AttributeTok{repr.plot.height =}\DecValTok{12}\NormalTok{)}

\CommentTok{\#create temp data frame for visualizations}
\NormalTok{df }\OtherTok{=}\NormalTok{ df\_items }\SpecialCharTok{\%\textgreater{}\%} \FunctionTok{filter}\NormalTok{ (q }\SpecialCharTok{\%nin\%} \FunctionTok{c}\NormalTok{(}\DecValTok{6}\NormalTok{,}\DecValTok{9}\NormalTok{)) }\SpecialCharTok{\%\textgreater{}\%} \FunctionTok{mutate}\NormalTok{(}
  \AttributeTok{score\_niceABS =} \FunctionTok{as.factor}\NormalTok{(score\_niceABS),}
  \AttributeTok{pretty\_interpretation =} \FunctionTok{factor}\NormalTok{(interpretation,}
    \AttributeTok{levels =} \FunctionTok{c}\NormalTok{(}\StringTok{"Orthogonal"}\NormalTok{, }\StringTok{"Satisfice"}\NormalTok{, }
               \StringTok{"frenzy"}\NormalTok{,}\StringTok{"?"}\NormalTok{,}
                \StringTok{"reference"}\NormalTok{,}\StringTok{"blank"}\NormalTok{,}
                \StringTok{"Tversky"}\NormalTok{, }\StringTok{"both tri + orth"}\NormalTok{,}
               \StringTok{"Triangular"}\NormalTok{ ))}
\NormalTok{  )}
\end{Highlighting}
\end{Shaded}

\hypertarget{absolute-score}{%
\subsection{Absolute Score}\label{absolute-score}}

\begin{Shaded}
\begin{Highlighting}[]
\CommentTok{\#DISTRIBUTION ABSOLUTE SCORE FULL }\AlertTok{TASK}
\FunctionTok{gf\_props}\NormalTok{(}\SpecialCharTok{\textasciitilde{}}\NormalTok{score\_niceABS, }\AttributeTok{fill =} \SpecialCharTok{\textasciitilde{}}\NormalTok{pretty\_condition, }\AttributeTok{position =} \FunctionTok{position\_dodge}\NormalTok{(), }\AttributeTok{data =}\NormalTok{ df) }\SpecialCharTok{+}
  \FunctionTok{labs}\NormalTok{( }\AttributeTok{x =} \StringTok{"Absolute Score"}\NormalTok{, }
        \AttributeTok{title =} \StringTok{"Distribution of Absolute Score (all Items)"}\NormalTok{,}
        \AttributeTok{subtitle =} \FunctionTok{paste}\NormalTok{(}\StringTok{"Impasse Condition (blue) yields more correct responses across the entire task"}\NormalTok{),}
        \AttributeTok{y =} \StringTok{"Proportion of Items"}\NormalTok{) }\SpecialCharTok{+}
  \FunctionTok{scale\_fill\_discrete}\NormalTok{(}\AttributeTok{name =} \StringTok{"Condition"}\NormalTok{) }\SpecialCharTok{+}  
  \FunctionTok{theme\_minimal}\NormalTok{()}
\end{Highlighting}
\end{Shaded}

\begin{figure}[H]

{\centering \includegraphics{analysis/SGC3A/2_sgc3A_scoring_files/figure-pdf/DISTR-ABSCORE-1.pdf}

}

\end{figure}

\begin{Shaded}
\begin{Highlighting}[]
\CommentTok{\#DISTRIBUTION ABSOLUTE SCORE BY ITEM}
\FunctionTok{gf\_props}\NormalTok{(}\SpecialCharTok{\textasciitilde{}}\NormalTok{score\_niceABS, }\AttributeTok{fill =} \SpecialCharTok{\textasciitilde{}}\NormalTok{pretty\_condition, }\AttributeTok{position =} \FunctionTok{position\_dodge}\NormalTok{(), }\AttributeTok{data =}\NormalTok{ df)  }\SpecialCharTok{\%\textgreater{}\%} 
  \FunctionTok{gf\_facet\_grid}\NormalTok{(pretty\_condition}\SpecialCharTok{\textasciitilde{}}\NormalTok{q) }\SpecialCharTok{+} 
  \FunctionTok{labs}\NormalTok{( }\AttributeTok{x =} \StringTok{"Absolute Score"}\NormalTok{, }
        \AttributeTok{title =} \StringTok{"Distribution of Absolute Score (by Item)"}\NormalTok{,}
        \AttributeTok{subtitle =} \StringTok{"Impasse Condition (blue) yields more correct responses on each item"}\NormalTok{,}
        \AttributeTok{y =} \StringTok{"Proprition of Subjects"}\NormalTok{) }\SpecialCharTok{+}
  \FunctionTok{scale\_fill\_discrete}\NormalTok{(}\AttributeTok{name =} \StringTok{"Condition"}\NormalTok{) }\SpecialCharTok{+}  
  \FunctionTok{theme\_minimal}\NormalTok{()}
\end{Highlighting}
\end{Shaded}

\begin{figure}[H]

{\centering \includegraphics{analysis/SGC3A/2_sgc3A_scoring_files/figure-pdf/DISTR-ABSCORE-2.pdf}

}

\end{figure}

\begin{Shaded}
\begin{Highlighting}[]
\CommentTok{\#DISTRIBUTION ABSOLUTE SCORE BY SUBJECT}
\FunctionTok{gf\_props}\NormalTok{(}\SpecialCharTok{\textasciitilde{}}\NormalTok{s\_NABS, }\AttributeTok{fill =} \SpecialCharTok{\textasciitilde{}}\NormalTok{pretty\_condition, }\AttributeTok{position =} \FunctionTok{position\_dodge}\NormalTok{(), }\AttributeTok{data =}\NormalTok{ df\_subjects) }\SpecialCharTok{\%\textgreater{}\%} 
\FunctionTok{gf\_facet\_grid}\NormalTok{(pretty\_condition }\SpecialCharTok{\textasciitilde{}}\NormalTok{. )}\SpecialCharTok{+}
  \FunctionTok{labs}\NormalTok{( }\AttributeTok{x =} \StringTok{"Total Absolute Score"}\NormalTok{, }
        \AttributeTok{title =} \StringTok{"Distribution of Total Absolute Score (by Subject)"}\NormalTok{,}
        \AttributeTok{subtitle =} \StringTok{"Impasse Condition (blue) yields higher total absolute scores"}\NormalTok{,}
        \AttributeTok{y =} \StringTok{"Proportion of Subjects"}\NormalTok{) }\SpecialCharTok{+}
  \FunctionTok{scale\_fill\_discrete}\NormalTok{(}\AttributeTok{name =} \StringTok{"Condition"}\NormalTok{) }\SpecialCharTok{+}  
  \FunctionTok{theme\_minimal}\NormalTok{() }\SpecialCharTok{+} \FunctionTok{theme}\NormalTok{(}\AttributeTok{legend.position =} \StringTok{"blank"}\NormalTok{)}
\end{Highlighting}
\end{Shaded}

\begin{figure}[H]

{\centering \includegraphics{analysis/SGC3A/2_sgc3A_scoring_files/figure-pdf/DISTR-ABSCORE-3.pdf}

}

\end{figure}

\begin{Shaded}
\begin{Highlighting}[]
\CommentTok{\#DISTRIBUTION ABSOLUTE SCORE }\AlertTok{TEST}\CommentTok{ PHASE}
\FunctionTok{gf\_histogram}\NormalTok{(}\SpecialCharTok{\textasciitilde{}}\NormalTok{item\_test\_NABS, }\AttributeTok{fill =} \SpecialCharTok{\textasciitilde{}}\NormalTok{pretty\_condition, }\AttributeTok{data =}\NormalTok{ df\_subjects) }\SpecialCharTok{\%\textgreater{}\%} 
  \FunctionTok{gf\_facet\_wrap}\NormalTok{(}\SpecialCharTok{\textasciitilde{}}\NormalTok{pretty\_condition) }\SpecialCharTok{+} 
  \FunctionTok{labs}\NormalTok{( }\AttributeTok{x =} \StringTok{"Absolute Score in TEST Phase"}\NormalTok{, }
        \AttributeTok{title =} \StringTok{"Distribution of TEST PHASE Absolute Score (all Items)"}\NormalTok{,}
        \AttributeTok{subtitle =} \FunctionTok{paste}\NormalTok{(}\StringTok{""}\NormalTok{),}
        \AttributeTok{y =} \StringTok{"Proportion of Items"}\NormalTok{) }\SpecialCharTok{+}
  \FunctionTok{scale\_fill\_discrete}\NormalTok{(}\AttributeTok{name =} \StringTok{"Condition"}\NormalTok{) }\SpecialCharTok{+}  
  \FunctionTok{theme\_minimal}\NormalTok{()}
\end{Highlighting}
\end{Shaded}

\begin{figure}[H]

{\centering \includegraphics{analysis/SGC3A/2_sgc3A_scoring_files/figure-pdf/DISTR-ABSCORE-4.pdf}

}

\end{figure}

\hypertarget{scaled-score}{%
\subsection{Scaled Score}\label{scaled-score}}

\begin{Shaded}
\begin{Highlighting}[]
\FunctionTok{options}\NormalTok{(}\AttributeTok{repr.plot.width =}\DecValTok{9}\NormalTok{, }\AttributeTok{repr.plot.height =}\DecValTok{12}\NormalTok{)}

\CommentTok{\#DISTRIBUTION SCALED SCORE FULL }\AlertTok{TASK}
\FunctionTok{gf\_props}\NormalTok{(}\SpecialCharTok{\textasciitilde{}}\NormalTok{score\_SCALED, }\AttributeTok{fill =} \SpecialCharTok{\textasciitilde{}}\NormalTok{pretty\_condition, }\AttributeTok{position =} \FunctionTok{position\_dodge}\NormalTok{(), }\AttributeTok{data =}\NormalTok{ df) }\SpecialCharTok{+}
  \FunctionTok{labs}\NormalTok{( }\AttributeTok{x =} \StringTok{"Scaled Score"}\NormalTok{, }
        \AttributeTok{title =} \StringTok{"Distribution of Scaled Score (all Items)"}\NormalTok{,}
        \AttributeTok{subtitle =} \StringTok{"Impasse Condition (blue) yields higher scaled scores across the entire task"}\NormalTok{,}
        \AttributeTok{y =} \StringTok{"Proportion of Items"}\NormalTok{) }\SpecialCharTok{+}
  \FunctionTok{scale\_fill\_discrete}\NormalTok{(}\AttributeTok{name =} \StringTok{"Condition"}\NormalTok{) }\SpecialCharTok{+}  
  \FunctionTok{theme\_minimal}\NormalTok{()}
\end{Highlighting}
\end{Shaded}

\begin{figure}[H]

{\centering \includegraphics{analysis/SGC3A/2_sgc3A_scoring_files/figure-pdf/DISTR-SCALEDSCORE-1.pdf}

}

\end{figure}

\begin{Shaded}
\begin{Highlighting}[]
\CommentTok{\#DISTRIBUTION SCALED SCORE BY ITEM}
\FunctionTok{gf\_props}\NormalTok{(}\SpecialCharTok{\textasciitilde{}}\NormalTok{score\_SCALED, }\AttributeTok{fill =} \SpecialCharTok{\textasciitilde{}}\NormalTok{pretty\_condition, }\AttributeTok{position =} \FunctionTok{position\_dodge}\NormalTok{(), }\AttributeTok{data =}\NormalTok{ df)  }\SpecialCharTok{\%\textgreater{}\%} 
  \FunctionTok{gf\_facet\_grid}\NormalTok{(q}\SpecialCharTok{\textasciitilde{}}\NormalTok{pretty\_condition) }\SpecialCharTok{+} 
  \FunctionTok{labs}\NormalTok{( }\AttributeTok{x =} \StringTok{"Scaled Score"}\NormalTok{, }
        \AttributeTok{title =} \StringTok{"Distribution of Scaled Score (by Item)"}\NormalTok{,}
        \AttributeTok{subtitle =} \StringTok{"Impasse Condition (blue) yields higher scaled scores on each item"}\NormalTok{,}
        \AttributeTok{y =} \StringTok{"Proportion of Subjects"}\NormalTok{) }\SpecialCharTok{+}
  \FunctionTok{scale\_fill\_discrete}\NormalTok{(}\AttributeTok{name =} \StringTok{"Condition"}\NormalTok{) }\SpecialCharTok{+}  \FunctionTok{scale\_y\_continuous}\NormalTok{(}\AttributeTok{breaks=}\FunctionTok{c}\NormalTok{(}\DecValTok{0}\NormalTok{,}\FloatTok{0.5}\NormalTok{)) }\SpecialCharTok{+} 
  \FunctionTok{theme\_minimal}\NormalTok{() }\SpecialCharTok{+} \FunctionTok{theme}\NormalTok{(}\AttributeTok{legend.position=}\StringTok{"blank"}\NormalTok{)}
\end{Highlighting}
\end{Shaded}

\begin{figure}[H]

{\centering \includegraphics{analysis/SGC3A/2_sgc3A_scoring_files/figure-pdf/DISTR-SCALEDSCORE-2.pdf}

}

\end{figure}

\begin{Shaded}
\begin{Highlighting}[]
\CommentTok{\#DISTRIBUTION SCALED SCORE BY SUBJECT}
\FunctionTok{gf\_props}\NormalTok{(}\SpecialCharTok{\textasciitilde{}}\NormalTok{s\_SCALED, }\AttributeTok{fill =} \SpecialCharTok{\textasciitilde{}}\NormalTok{pretty\_condition, }\AttributeTok{data =}\NormalTok{ df\_subjects)  }\SpecialCharTok{\%\textgreater{}\%} 
  \FunctionTok{gf\_facet\_grid}\NormalTok{(pretty\_condition }\SpecialCharTok{\textasciitilde{}}\NormalTok{. )}\SpecialCharTok{+}
  \FunctionTok{labs}\NormalTok{( }\AttributeTok{x =} \StringTok{"Total Scaled Score"}\NormalTok{, }
        \AttributeTok{title =} \StringTok{"Distribution of Total Scaled Score (by Subject)"}\NormalTok{,}
        \AttributeTok{subtitle =} \StringTok{"Impasse Condition (blue) yields higher cumulative scaled scores"}\NormalTok{,}
        \AttributeTok{y =} \StringTok{"Number of Subjects"}\NormalTok{) }\SpecialCharTok{+}
  \FunctionTok{scale\_fill\_discrete}\NormalTok{(}\AttributeTok{name =} \StringTok{"Condition"}\NormalTok{) }\SpecialCharTok{+}  
  \FunctionTok{theme\_minimal}\NormalTok{()}
\end{Highlighting}
\end{Shaded}

\begin{figure}[H]

{\centering \includegraphics{analysis/SGC3A/2_sgc3A_scoring_files/figure-pdf/DISTR-SCALEDSCORE-3.pdf}

}

\end{figure}

\begin{Shaded}
\begin{Highlighting}[]
\CommentTok{\#DISTRIBUTION SCALED SCORE }\AlertTok{TEST}\CommentTok{ PHASE}
\FunctionTok{gf\_histogram}\NormalTok{(}\SpecialCharTok{\textasciitilde{}}\NormalTok{item\_test\_SCALED, }\AttributeTok{fill =} \SpecialCharTok{\textasciitilde{}}\NormalTok{pretty\_condition, }\AttributeTok{data =}\NormalTok{ df\_subjects) }\SpecialCharTok{\%\textgreater{}\%} 
  \FunctionTok{gf\_facet\_wrap}\NormalTok{(}\SpecialCharTok{\textasciitilde{}}\NormalTok{pretty\_condition) }\SpecialCharTok{+} 
  \FunctionTok{labs}\NormalTok{( }\AttributeTok{x =} \StringTok{"Scaled Score in TEST Phase"}\NormalTok{, }
        \AttributeTok{title =} \StringTok{"Distribution of TEST PHASE Scaled Score (all Items)"}\NormalTok{,}
        \AttributeTok{subtitle =} \FunctionTok{paste}\NormalTok{(}\StringTok{""}\NormalTok{),}
        \AttributeTok{y =} \StringTok{"Proportion of Items"}\NormalTok{) }\SpecialCharTok{+}
  \FunctionTok{scale\_fill\_discrete}\NormalTok{(}\AttributeTok{name =} \StringTok{"Condition"}\NormalTok{) }\SpecialCharTok{+}  
  \FunctionTok{theme\_minimal}\NormalTok{()}
\end{Highlighting}
\end{Shaded}

\begin{figure}[H]

{\centering \includegraphics{analysis/SGC3A/2_sgc3A_scoring_files/figure-pdf/DISTR-SCALEDSCORE-4.pdf}

}

\end{figure}

\begin{itemize}
\tightlist
\item
  TODO: INVESTIGATE if some of the scores assigned to 0 should be
  assigned to -0.5 to balance
\item
  TODO: INVESTIGATE DISTRIBUTIONS of each subscore type
\end{itemize}

\hypertarget{interpretations}{%
\subsection{Interpretations}\label{interpretations}}

\begin{Shaded}
\begin{Highlighting}[]
\CommentTok{\#DISTRIBUTION OF INTERPRETATION}
\FunctionTok{gf\_props}\NormalTok{(}\SpecialCharTok{\textasciitilde{}}\NormalTok{pretty\_interpretation, }\AttributeTok{fill =} \SpecialCharTok{\textasciitilde{}}\NormalTok{pretty\_condition, }\AttributeTok{data =}\NormalTok{ df) }\SpecialCharTok{\%\textgreater{}\%} 
  \FunctionTok{gf\_facet\_grid}\NormalTok{( pretty\_condition }\SpecialCharTok{\textasciitilde{}}\NormalTok{ ., }\AttributeTok{labeller =}\NormalTok{ label\_both) }\SpecialCharTok{+} 
  \FunctionTok{labs}\NormalTok{( }\AttributeTok{title =} \StringTok{"Distribution of Interpretations (across Task)"}\NormalTok{,}
        \AttributeTok{x =} \StringTok{"Graph Interpretation"}\NormalTok{,}
        \AttributeTok{y =} \StringTok{"Proportion of Responses"}\NormalTok{,}
        \AttributeTok{subtitle =} \StringTok{"Impasse condition (blue) yields fewer Orthogonal and more Triangular responses"}\NormalTok{) }\SpecialCharTok{+} 
  \FunctionTok{theme\_minimal}\NormalTok{() }\SpecialCharTok{+} \FunctionTok{theme}\NormalTok{(}\AttributeTok{legend.position =} \StringTok{"blank"}\NormalTok{)}
\end{Highlighting}
\end{Shaded}

\begin{figure}[H]

{\centering \includegraphics{analysis/SGC3A/2_sgc3A_scoring_files/figure-pdf/DISTR-INTERPRETATIONS-1.pdf}

}

\end{figure}

\begin{Shaded}
\begin{Highlighting}[]
\CommentTok{\#DISTRIBUTION OF INTERPRETATION ACROSS ITEMS}
\FunctionTok{gf\_propsh}\NormalTok{(}\SpecialCharTok{\textasciitilde{}}\NormalTok{ pretty\_interpretation, }\AttributeTok{fill =} \SpecialCharTok{\textasciitilde{}}\NormalTok{pretty\_condition, }\AttributeTok{data =}\NormalTok{ df) }\SpecialCharTok{\%\textgreater{}\%} 
  \FunctionTok{gf\_facet\_grid}\NormalTok{( pretty\_condition}\SpecialCharTok{\textasciitilde{}}\NormalTok{q) }\SpecialCharTok{+} 
  \FunctionTok{labs}\NormalTok{( }\AttributeTok{title =} \StringTok{"Distribution of Interpretations (by Item)"}\NormalTok{,}
        \AttributeTok{subtitle =} \StringTok{"Impasse condition (blue) yields more Triangular responses on each question"}\NormalTok{,}
        \AttributeTok{y =} \StringTok{"Interpretation"}\NormalTok{, }\AttributeTok{x =} \StringTok{"Proportion of Subjects"}\NormalTok{) }\SpecialCharTok{+} \FunctionTok{theme\_minimal}\NormalTok{() }\SpecialCharTok{+} \FunctionTok{theme}\NormalTok{(}\AttributeTok{legend.position =} \StringTok{"blank"}\NormalTok{)}
\end{Highlighting}
\end{Shaded}

\begin{figure}[H]

{\centering \includegraphics{analysis/SGC3A/2_sgc3A_scoring_files/figure-pdf/DISTR-INTERPRETATIONS-2.pdf}

}

\end{figure}

\begin{Shaded}
\begin{Highlighting}[]
\CommentTok{\#DISTRIBUTION OF INTERPRETATION TYPE ACROSS ITEMS}
\FunctionTok{gf\_propsh}\NormalTok{(}\SpecialCharTok{\textasciitilde{}}\NormalTok{ high\_interpretation, }\AttributeTok{fill =} \SpecialCharTok{\textasciitilde{}}\NormalTok{pretty\_condition, }\AttributeTok{data =}\NormalTok{ df) }\SpecialCharTok{\%\textgreater{}\%} 
  \FunctionTok{gf\_facet\_grid}\NormalTok{( pretty\_condition}\SpecialCharTok{\textasciitilde{}}\NormalTok{q) }\SpecialCharTok{+} 
  \FunctionTok{labs}\NormalTok{( }\AttributeTok{title =} \StringTok{"Distribution of Interpretations (by Item)"}\NormalTok{,}
        \AttributeTok{subtitle =} \StringTok{"Impasse condition (blue) yields more positive trending responses on each question"}\NormalTok{,}
        \AttributeTok{y =} \StringTok{"Interpretation"}\NormalTok{, }\AttributeTok{x =} \StringTok{"Proportion of Subjects"}\NormalTok{) }\SpecialCharTok{+} \FunctionTok{theme\_minimal}\NormalTok{() }\SpecialCharTok{+} \FunctionTok{theme}\NormalTok{(}\AttributeTok{legend.position =} \StringTok{"blank"}\NormalTok{)}
\end{Highlighting}
\end{Shaded}

\begin{figure}[H]

{\centering \includegraphics{analysis/SGC3A/2_sgc3A_scoring_files/figure-pdf/DISTR-INTERPRETATIONS-3.pdf}

}

\end{figure}

\hypertarget{progress-over-task}{%
\subsection{Progress over Task}\label{progress-over-task}}

\begin{Shaded}
\begin{Highlighting}[]
\CommentTok{\#VISUALIZE progress over time ABSOLUTE score }
\FunctionTok{ggplot}\NormalTok{(}\AttributeTok{data =}\NormalTok{ df\_absolute\_progress, }\FunctionTok{aes}\NormalTok{(}\AttributeTok{x =}\NormalTok{ question, }\AttributeTok{y =}\NormalTok{ score, }\AttributeTok{group =}\NormalTok{ subject, }\AttributeTok{alpha =} \FloatTok{0.01}\NormalTok{, }\AttributeTok{color =}\NormalTok{ pretty\_condition)) }\SpecialCharTok{+} 
 \FunctionTok{geom\_line}\NormalTok{(}\AttributeTok{position=}\FunctionTok{position\_jitter}\NormalTok{(}\AttributeTok{w=}\FloatTok{0.15}\NormalTok{, }\AttributeTok{h=}\FloatTok{0.15}\NormalTok{), }\AttributeTok{size=}\FloatTok{0.1}\NormalTok{) }\SpecialCharTok{+}
 \FunctionTok{facet\_wrap}\NormalTok{(}\SpecialCharTok{\textasciitilde{}}\NormalTok{pretty\_condition) }\SpecialCharTok{+} 
 \FunctionTok{labs}\NormalTok{ (}\AttributeTok{title =} \StringTok{"Cumulative Absolute Score over sequence of task"}\NormalTok{, }\AttributeTok{x =} \StringTok{"Question"}\NormalTok{ , }\AttributeTok{y =} \StringTok{"Cumulative Absolute Score"}\NormalTok{) }\SpecialCharTok{+} 
 \FunctionTok{scale\_x\_continuous}\NormalTok{(}\AttributeTok{breaks =} \FunctionTok{c}\NormalTok{(}\DecValTok{1}\NormalTok{,}\DecValTok{2}\NormalTok{,}\DecValTok{3}\NormalTok{,}\DecValTok{4}\NormalTok{,}\DecValTok{5}\NormalTok{,}\DecValTok{6}\NormalTok{,}\DecValTok{7}\NormalTok{,}\DecValTok{8}\NormalTok{,}\DecValTok{9}\NormalTok{,}\DecValTok{10}\NormalTok{,}\DecValTok{11}\NormalTok{,}\DecValTok{12}\NormalTok{,}\DecValTok{13}\NormalTok{)) }\SpecialCharTok{+}
 \FunctionTok{theme\_minimal}\NormalTok{() }\SpecialCharTok{+} \FunctionTok{theme}\NormalTok{(}\AttributeTok{legend.position =} \StringTok{"blank"}\NormalTok{)}
\end{Highlighting}
\end{Shaded}

\begin{figure}[H]

{\centering \includegraphics{analysis/SGC3A/2_sgc3A_scoring_files/figure-pdf/VIZ-PROGRESS-1.pdf}

}

\end{figure}

\begin{Shaded}
\begin{Highlighting}[]
\CommentTok{\#VISUALIZE progress over time SCALED score }
\FunctionTok{ggplot}\NormalTok{(}\AttributeTok{data =}\NormalTok{ df\_scaled\_progress, }\FunctionTok{aes}\NormalTok{(}\AttributeTok{x =}\NormalTok{ question, }\AttributeTok{y =}\NormalTok{ score, }\AttributeTok{group =}\NormalTok{ subject, }\AttributeTok{alpha =} \FloatTok{0.01}\NormalTok{, }\AttributeTok{color =}\NormalTok{ pretty\_condition)) }\SpecialCharTok{+} 
 \FunctionTok{geom\_line}\NormalTok{(}\AttributeTok{position=}\FunctionTok{position\_jitter}\NormalTok{(}\AttributeTok{w=}\FloatTok{0.15}\NormalTok{, }\AttributeTok{h=}\FloatTok{0.15}\NormalTok{), }\AttributeTok{size=}\FloatTok{0.1}\NormalTok{) }\SpecialCharTok{+}
 \FunctionTok{facet\_wrap}\NormalTok{(}\SpecialCharTok{\textasciitilde{}}\NormalTok{pretty\_condition) }\SpecialCharTok{+} 
 \FunctionTok{labs}\NormalTok{ (}\AttributeTok{title =} \StringTok{"Cumulative Scaled Score over sequence of task"}\NormalTok{, }\AttributeTok{x =} \StringTok{"Question"}\NormalTok{ , }\AttributeTok{y =} \StringTok{"Cumulative Scaled Score"}\NormalTok{) }\SpecialCharTok{+} 
 \FunctionTok{scale\_x\_continuous}\NormalTok{(}\AttributeTok{breaks =} \FunctionTok{c}\NormalTok{(}\DecValTok{1}\NormalTok{,}\DecValTok{2}\NormalTok{,}\DecValTok{3}\NormalTok{,}\DecValTok{4}\NormalTok{,}\DecValTok{5}\NormalTok{,}\DecValTok{6}\NormalTok{,}\DecValTok{7}\NormalTok{,}\DecValTok{8}\NormalTok{,}\DecValTok{9}\NormalTok{,}\DecValTok{10}\NormalTok{,}\DecValTok{11}\NormalTok{,}\DecValTok{12}\NormalTok{,}\DecValTok{13}\NormalTok{)) }\SpecialCharTok{+}
 \FunctionTok{theme\_minimal}\NormalTok{() }\SpecialCharTok{+} \FunctionTok{theme}\NormalTok{(}\AttributeTok{legend.position =} \StringTok{"blank"}\NormalTok{)}
\end{Highlighting}
\end{Shaded}

\begin{figure}[H]

{\centering \includegraphics{analysis/SGC3A/2_sgc3A_scoring_files/figure-pdf/VIZ-PROGRESS-2.pdf}

}

\end{figure}

\hypertarget{interpretation-subscores}{%
\subsection{Interpretation Subscores}\label{interpretation-subscores}}

\begin{Shaded}
\begin{Highlighting}[]
\FunctionTok{gf\_density}\NormalTok{(}\SpecialCharTok{\textasciitilde{}}\NormalTok{ s\_TRI, }\AttributeTok{fill =} \SpecialCharTok{\textasciitilde{}}\NormalTok{pretty\_condition, }\AttributeTok{data =}\NormalTok{ df\_subjects) }\SpecialCharTok{\%\textgreater{}\%} 
  \FunctionTok{gf\_facet\_wrap}\NormalTok{( }\SpecialCharTok{\textasciitilde{}}\NormalTok{ pretty\_condition) }\SpecialCharTok{+} 
  \FunctionTok{labs}\NormalTok{( }\AttributeTok{title =} \StringTok{"Distribution of Total Triangular Score"}\NormalTok{,}
        \AttributeTok{subtitle =} \StringTok{"Impasse shifts density toward higher Triagular scores"}\NormalTok{,}
        \AttributeTok{x =} \StringTok{"Item Triangular Score"}\NormalTok{, }\AttributeTok{y =} \StringTok{"Proportion of Subjects"}\NormalTok{) }\SpecialCharTok{+} 
        \FunctionTok{theme\_minimal}\NormalTok{() }\SpecialCharTok{+} \FunctionTok{theme}\NormalTok{(}\AttributeTok{legend.position =} \StringTok{"blank"}\NormalTok{)}
\end{Highlighting}
\end{Shaded}

\begin{figure}[H]

{\centering \includegraphics{analysis/SGC3A/2_sgc3A_scoring_files/figure-pdf/DIST-SUBSCORES-1.pdf}

}

\end{figure}

\begin{Shaded}
\begin{Highlighting}[]
\FunctionTok{gf\_density}\NormalTok{(}\SpecialCharTok{\textasciitilde{}}\NormalTok{ s\_ORTH, }\AttributeTok{fill =} \SpecialCharTok{\textasciitilde{}}\NormalTok{pretty\_condition, }\AttributeTok{data =}\NormalTok{ df\_subjects) }\SpecialCharTok{\%\textgreater{}\%} 
  \FunctionTok{gf\_facet\_wrap}\NormalTok{( }\SpecialCharTok{\textasciitilde{}}\NormalTok{ pretty\_condition) }\SpecialCharTok{+} 
  \FunctionTok{labs}\NormalTok{( }\AttributeTok{title =} \StringTok{"Distribution of Total Orthogonal Score"}\NormalTok{,}
        \AttributeTok{subtitle =} \StringTok{"Impasse shifts density toward lower Orthogonal scores"}\NormalTok{,}
        \AttributeTok{x =} \StringTok{"Item Orthogonal Score"}\NormalTok{, }\AttributeTok{y =} \StringTok{"Proportion of Subjects"}\NormalTok{) }\SpecialCharTok{+} 
        \FunctionTok{theme\_minimal}\NormalTok{() }\SpecialCharTok{+} \FunctionTok{theme}\NormalTok{(}\AttributeTok{legend.position =} \StringTok{"blank"}\NormalTok{)}
\end{Highlighting}
\end{Shaded}

\begin{figure}[H]

{\centering \includegraphics{analysis/SGC3A/2_sgc3A_scoring_files/figure-pdf/DIST-SUBSCORES-2.pdf}

}

\end{figure}

\begin{Shaded}
\begin{Highlighting}[]
\FunctionTok{gf\_density}\NormalTok{(}\SpecialCharTok{\textasciitilde{}}\NormalTok{ s\_TVERSKY, }\AttributeTok{fill =} \SpecialCharTok{\textasciitilde{}}\NormalTok{pretty\_condition, }\AttributeTok{data =}\NormalTok{ df\_subjects) }\SpecialCharTok{\%\textgreater{}\%} 
  \FunctionTok{gf\_facet\_wrap}\NormalTok{( }\SpecialCharTok{\textasciitilde{}}\NormalTok{ pretty\_condition) }\SpecialCharTok{+} 
  \FunctionTok{labs}\NormalTok{( }\AttributeTok{title =} \StringTok{"Distribution of Total Tversky Score"}\NormalTok{,}
        \AttributeTok{subtitle =} \StringTok{"Impasse shifts density toward higher Tversky scores"}\NormalTok{,}
        \AttributeTok{x =} \StringTok{"Item Orthogonal Score"}\NormalTok{, }\AttributeTok{y =} \StringTok{"Proportion of Subjects"}\NormalTok{) }\SpecialCharTok{+} 
        \FunctionTok{theme\_minimal}\NormalTok{() }\SpecialCharTok{+} \FunctionTok{theme}\NormalTok{(}\AttributeTok{legend.position =} \StringTok{"blank"}\NormalTok{)}
\end{Highlighting}
\end{Shaded}

\begin{figure}[H]

{\centering \includegraphics{analysis/SGC3A/2_sgc3A_scoring_files/figure-pdf/DIST-SUBSCORES-3.pdf}

}

\end{figure}

\begin{Shaded}
\begin{Highlighting}[]
\FunctionTok{gf\_histogram}\NormalTok{(}\SpecialCharTok{\textasciitilde{}}\NormalTok{ s\_SATISFICE, }\AttributeTok{fill =} \SpecialCharTok{\textasciitilde{}}\NormalTok{pretty\_condition, }\AttributeTok{data =}\NormalTok{ df\_subjects) }\SpecialCharTok{\%\textgreater{}\%} 
  \FunctionTok{gf\_facet\_wrap}\NormalTok{( }\SpecialCharTok{\textasciitilde{}}\NormalTok{ pretty\_condition) }\SpecialCharTok{+} 
  \FunctionTok{labs}\NormalTok{( }\AttributeTok{title =} \StringTok{"Distribution of Total Satisfice Score"}\NormalTok{,}
        \AttributeTok{subtitle =} \StringTok{"Satisficing only occurs in impasse, when no orthogonal response is available"}\NormalTok{,}
        \AttributeTok{x =} \StringTok{"Item Orthogonal Score"}\NormalTok{, }\AttributeTok{y =} \StringTok{"Proportion of Subjects"}\NormalTok{) }\SpecialCharTok{+} 
        \FunctionTok{theme\_minimal}\NormalTok{() }\SpecialCharTok{+} \FunctionTok{theme}\NormalTok{(}\AttributeTok{legend.position =} \StringTok{"blank"}\NormalTok{)}
\end{Highlighting}
\end{Shaded}

\begin{figure}[H]

{\centering \includegraphics{analysis/SGC3A/2_sgc3A_scoring_files/figure-pdf/DIST-SUBSCORES-4.pdf}

}

\end{figure}

\hypertarget{explore-responses}{%
\section{EXPLORE RESPONSES}\label{explore-responses}}

In this section we explore responses given by participants to each
particular item in the graph comprehension task, indicate how each
response was scored, and what interpretation of the graph is indicated
by different responses.

\hypertarget{scaffold-phase}{%
\subsection{Scaffold Phase}\label{scaffold-phase}}

The first five questions constitute the `scaffold' (or learning) phase,
where participants see a different version of the stimulus (specifically
a different dataset is visualized) invoking a different experimental
condition.

\hypertarget{question-1}{%
\subsubsection{Question \#1}\label{question-1}}

\hypertarget{q1.-control-condition}{%
\paragraph{Q1. Control Condition}\label{q1.-control-condition}}

We start by exploring the range of response options checked by
participants on Question 1, for those assigned to the control
(non-impasse) condition (\texttt{condition} = 111).

\begin{figure}

{\centering \includegraphics{analysis/SGC3A/static/questions/Q1_111.png}

}

\caption{\label{fig-Q1-111}Question 1 --- Control Condition}

\end{figure}

\begin{Shaded}
\begin{Highlighting}[]
\NormalTok{q }\OtherTok{\textless{}{-}}\NormalTok{ keys\_raw }\SpecialCharTok{\%\textgreater{}\%} \FunctionTok{filter}\NormalTok{(condition }\SpecialCharTok{==} \StringTok{"DEFAULT"}\NormalTok{) }\SpecialCharTok{\%\textgreater{}\%} \FunctionTok{filter}\NormalTok{(Q}\SpecialCharTok{==}\DecValTok{1}\NormalTok{)}
\NormalTok{ignore }\OtherTok{\textless{}{-}}\NormalTok{ q }\SpecialCharTok{\%\textgreater{}\%} \FunctionTok{select}\NormalTok{(}\StringTok{"REF\_POINT"}\NormalTok{)}
\NormalTok{answers }\OtherTok{\textless{}{-}}\NormalTok{ q }\SpecialCharTok{\%\textgreater{}\%} \FunctionTok{select}\NormalTok{(}\StringTok{"TRIANGULAR"}\NormalTok{, }\StringTok{"ORTHOGONAL"}\NormalTok{, }\StringTok{"SATISFICE\_left"}\NormalTok{, }\StringTok{"SATISFICE\_right"}\NormalTok{,}\StringTok{"TV\_max"}\NormalTok{,}\StringTok{"TV\_start"}\NormalTok{, }\StringTok{"TV\_end"}\NormalTok{, }\StringTok{"TV\_dur"}\NormalTok{) }\SpecialCharTok{\%\textgreater{}\%} \FunctionTok{unlist}\NormalTok{()}
\NormalTok{ves }\OtherTok{\textless{}{-}}\NormalTok{ q }\SpecialCharTok{\%\textgreater{}\%} \FunctionTok{mutate}\NormalTok{(}
  \AttributeTok{SATISFICE\_left\_allow =} \StringTok{""}\NormalTok{,}
  \AttributeTok{SATISFICE\_right\_allow =} \StringTok{""}
\NormalTok{) }\SpecialCharTok{\%\textgreater{}\%} \FunctionTok{select}\NormalTok{(}\StringTok{"TRI\_allow"}\NormalTok{, }\StringTok{"ORTH\_allow"}\NormalTok{, }\StringTok{"SATISFICE\_left\_allow"}\NormalTok{,}\StringTok{"SATISFICE\_right\_allow"}\NormalTok{, }\StringTok{"TV\_max\_allow"}\NormalTok{,}\StringTok{"TV\_start\_allow"}\NormalTok{,}\StringTok{"TV\_end\_allow"}\NormalTok{, }\StringTok{"TV\_dur\_allow"}\NormalTok{)}\SpecialCharTok{\%\textgreater{}\%} \FunctionTok{unlist}\NormalTok{() }
\NormalTok{options }\OtherTok{\textless{}{-}}\NormalTok{ q }\SpecialCharTok{\%\textgreater{}\%} \FunctionTok{select}\NormalTok{(}\StringTok{"OPTIONS"}\NormalTok{)}
\NormalTok{question }\OtherTok{=}\NormalTok{ q }\SpecialCharTok{\%\textgreater{}\%}  \FunctionTok{select}\NormalTok{(}\StringTok{"TEXT"}\NormalTok{)}
\NormalTok{scores }\OtherTok{\textless{}{-}} \FunctionTok{c}\NormalTok{(}\StringTok{"Triangular"}\NormalTok{, }\StringTok{"Orthgonal"}\NormalTok{, }\StringTok{"Satisficing [left]"}\NormalTok{, }\StringTok{"Satisficing [right]"}\NormalTok{, }\StringTok{"Tversky [maximal]"}\NormalTok{, }\StringTok{"Tversky [start diagonal]"}\NormalTok{, }
            \StringTok{"Tversky [end diagonal]"}\NormalTok{, }\StringTok{"Tversky [duration line]"}\NormalTok{)}
\NormalTok{d }\OtherTok{=} \FunctionTok{tibble}\NormalTok{(}\AttributeTok{interpretation =}\NormalTok{ scores, }\AttributeTok{answer =}\NormalTok{ answers, }\AttributeTok{allowed=}\NormalTok{ves)}
\NormalTok{d}\SpecialCharTok{$}\NormalTok{answer }\OtherTok{\textless{}{-}} \FunctionTok{replace\_na}\NormalTok{(d}\SpecialCharTok{$}\NormalTok{answer, }\StringTok{""}\NormalTok{)}
\NormalTok{d}\SpecialCharTok{$}\NormalTok{allowed }\OtherTok{\textless{}{-}} \FunctionTok{replace\_na}\NormalTok{(d}\SpecialCharTok{$}\NormalTok{allowed, }\StringTok{""}\NormalTok{)}

\NormalTok{title }\OtherTok{=} \FunctionTok{paste}\NormalTok{(}\StringTok{"Answer Key | Q1 Control Condition : "}\NormalTok{, question)}
\NormalTok{cols }\OtherTok{=} \FunctionTok{c}\NormalTok{(}\StringTok{"interpretation"}\NormalTok{, }\StringTok{"answer"}\NormalTok{,}\StringTok{"not penalized"}\NormalTok{)}

\NormalTok{d }\SpecialCharTok{\%\textgreater{}\%} \FunctionTok{kbl}\NormalTok{(}\AttributeTok{caption =}\NormalTok{ title, }\AttributeTok{col.names =}\NormalTok{ cols) }\SpecialCharTok{\%\textgreater{}\%} \FunctionTok{kable\_classic}\NormalTok{() }\SpecialCharTok{\%\textgreater{}\%} 
  \FunctionTok{footnote}\NormalTok{(}\AttributeTok{general =} \FunctionTok{paste}\NormalTok{(}\StringTok{"15 response options: "}\NormalTok{, options), }\AttributeTok{general\_title =} \StringTok{"Note: "}\NormalTok{,}\AttributeTok{footnote\_as\_chunk =}\NormalTok{ T) }
\end{Highlighting}
\end{Shaded}

\begin{table}

\caption{Answer Key | Q1 Control Condition :  Which shift(s) start at 11 am?}
\centering
\begin{tabular}[t]{l|l|l}
\hline
interpretation & answer & not penalized\\
\hline
Triangular & F & Z\\
\hline
Orthgonal & A & OI\\
\hline
Satisficing [left] &  & \\
\hline
Satisficing [right] &  & \\
\hline
Tversky [maximal] & CF & Z\\
\hline
Tversky [start diagonal] & F & Z\\
\hline
Tversky [end diagonal] & C & \\
\hline
Tversky [duration line] &  & \\
\hline
\multicolumn{3}{l}{\rule{0pt}{1em}\textit{Note: } 15 response options:  AIKGXJDBCHUZOFE}\\
\end{tabular}
\end{table}

Here we summarize the distinct response options given by participants on
this item. Each letter in \texttt{response} indicates a checkbox
selected by the participant (See Figure~\ref{fig-Q1-111} ). \texttt{n}
indicates the number of participants who gave this response, while
\texttt{interpretation} indicates the \emph{graph interpretation} most
consistent with that response. At the right of this table are the
Absolute, followed by Partial Credit subscores for each response. NA
indicates that there is no score calculated (occurs when there is no
subset of response options that accord with that interpretation for this
question).

Notice that for this Question, the \emph{Triangular} answer is the same
as the \emph{Tversky {[}start diagonal{]}} answer. In fact, for most
questions, one of the Tversky sub-types will match the correct response.

\begin{Shaded}
\begin{Highlighting}[]
\NormalTok{title }\OtherTok{\textless{}{-}} \StringTok{"Frequency of Selected Response Options for Question \#1 (Control Condition)"}
\NormalTok{names }\OtherTok{=} \FunctionTok{c}\NormalTok{(}\StringTok{"response"}\NormalTok{,}\StringTok{"n"}\NormalTok{,}\StringTok{"interpretation"}\NormalTok{,}\StringTok{"absolute"}\NormalTok{,}\StringTok{"tri"}\NormalTok{,}\StringTok{"tversky"}\NormalTok{,}\StringTok{"satisfice"}\NormalTok{,}\StringTok{"orthogonal"}\NormalTok{, }\StringTok{"scaled score"}\NormalTok{)}

\NormalTok{df\_items }\SpecialCharTok{\%\textgreater{}\%} \FunctionTok{filter}\NormalTok{(q }\SpecialCharTok{==} \DecValTok{1} \SpecialCharTok{\&}\NormalTok{ condition }\SpecialCharTok{==} \DecValTok{111}\NormalTok{) }\SpecialCharTok{\%\textgreater{}\%} \FunctionTok{group\_by}\NormalTok{(response) }\SpecialCharTok{\%\textgreater{}\%} 
\NormalTok{  dplyr}\SpecialCharTok{::}\FunctionTok{summarise}\NormalTok{( }\AttributeTok{count =} \FunctionTok{n}\NormalTok{(), }
                    \AttributeTok{nice =} \FunctionTok{unique}\NormalTok{(score\_niceABS),}
                    \AttributeTok{triangular =} \FunctionTok{unique}\NormalTok{(score\_TRI), }
                    \AttributeTok{orthogonal =}  \FunctionTok{unique}\NormalTok{(score\_ORTH),}
                    \AttributeTok{satisficing =}  \FunctionTok{unique}\NormalTok{(score\_SATISFICE),}
                    \AttributeTok{tversky =} \FunctionTok{unique}\NormalTok{(score\_TVERSKY),}
                    \AttributeTok{interpretation =} \FunctionTok{unique}\NormalTok{(int2),}
                    \AttributeTok{scaled =} \FunctionTok{unique}\NormalTok{(score\_SCALED)) }\SpecialCharTok{\%\textgreater{}\%} 
  \FunctionTok{arrange}\NormalTok{(interpretation, }\FunctionTok{desc}\NormalTok{(count)) }\SpecialCharTok{\%\textgreater{}\%} 
  \FunctionTok{select}\NormalTok{(response, count, interpretation, nice, }
\NormalTok{         triangular, tversky, satisficing, orthogonal, scaled) }\SpecialCharTok{\%\textgreater{}\%} 
  \FunctionTok{kbl}\NormalTok{(}\AttributeTok{caption =}\NormalTok{ title, }\AttributeTok{col.names =}\NormalTok{ names) }\SpecialCharTok{\%\textgreater{}\%}  \FunctionTok{kable\_classic}\NormalTok{() }\SpecialCharTok{\%\textgreater{}\%} 
  \FunctionTok{add\_header\_above}\NormalTok{(}\FunctionTok{c}\NormalTok{(}\StringTok{" "} \OtherTok{=} \DecValTok{3}\NormalTok{, }\StringTok{"Strict Score"} \OtherTok{=} \DecValTok{1}\NormalTok{, }\StringTok{"Interpretation Scores"}\OtherTok{=}\DecValTok{4}\NormalTok{, }\StringTok{"Discriminant"}\OtherTok{=}\DecValTok{1}\NormalTok{)) }\SpecialCharTok{\%\textgreater{}\%}
  \FunctionTok{pack\_rows}\NormalTok{(}\StringTok{"Triangular"}\NormalTok{, }\DecValTok{1}\NormalTok{, }\DecValTok{1}\NormalTok{) }\SpecialCharTok{\%\textgreater{}\%} 
  \FunctionTok{pack\_rows}\NormalTok{(}\StringTok{"Lines{-}Connect"}\NormalTok{, }\DecValTok{2}\NormalTok{, }\DecValTok{2}\NormalTok{) }\SpecialCharTok{\%\textgreater{}\%} 
  \FunctionTok{pack\_rows}\NormalTok{(}\StringTok{"Orthogonal"}\NormalTok{, }\DecValTok{3}\NormalTok{, }\DecValTok{3}\NormalTok{) }\SpecialCharTok{\%\textgreater{}\%} 
  \FunctionTok{pack\_rows}\NormalTok{(}\StringTok{"Other"}\NormalTok{, }\DecValTok{4}\NormalTok{, }\DecValTok{4}\NormalTok{)  }\SpecialCharTok{\%\textgreater{}\%} 
  \FunctionTok{pack\_rows}\NormalTok{(}\StringTok{"Unknown"}\NormalTok{, }\DecValTok{5}\NormalTok{, }\DecValTok{7}\NormalTok{)  }\SpecialCharTok{\%\textgreater{}\%} 
  \FunctionTok{footnote}\NormalTok{(}\AttributeTok{general =} \StringTok{"n = number of responses in sample"}\NormalTok{, }
           \AttributeTok{general\_title =} \StringTok{"Note: "}\NormalTok{,}\AttributeTok{footnote\_as\_chunk =}\NormalTok{ T) }
\end{Highlighting}
\end{Shaded}

\textbackslash begin\{table\}

\textbackslash caption\{\label{tab:Q1-CONTROL-RESPONSES}Frequency of
Selected Response Options for Question \#1 (Control Condition)\}
\centering

\begin{tabular}[t]{l|r|l|r|r|r|r|r|r}
\hline
\multicolumn{3}{c|}{ } & \multicolumn{1}{c|}{Strict Score} & \multicolumn{4}{c|}{Interpretation Scores} & \multicolumn{1}{c}{Discriminant} \\
\cline{4-4} \cline{5-8} \cline{9-9}
response & n & interpretation & absolute & tri & tversky & satisfice & orthogonal & scaled score\\
\hline
\multicolumn{9}{l}{\textbf{Triangular}}\\
\hline
\hspace{1em}F & 22 & Triangular & 1 & 1.000 & 1.000 & NA & -0.083 & 1.0\\
\hline
\multicolumn{9}{l}{\textbf{Lines-Connect}}\\
\hline
\hspace{1em}CF & 3 & Tversky & 0 & 0.923 & 1.000 & NA & -0.167 & 0.5\\
\hline
\multicolumn{9}{l}{\textbf{Orthogonal}}\\
\hline
\hspace{1em}A & 129 & Orthogonal & 0 & -0.077 & -0.071 & NA & 1.000 & -1.0\\
\hline
\multicolumn{9}{l}{\textbf{Other}}\\
\hline
\hspace{1em}AF & 1 & ? & 0 & 0.923 & 0.923 & NA & 0.917 & -0.5\\
\hline
\multicolumn{9}{l}{\textbf{Unknown}}\\
\hline
\hspace{1em}DIJ & 1 & ? & 0 & -0.231 & -0.214 & NA & -0.167 & -0.5\\
\hline
\hspace{1em}X & 1 & ? & 0 & -0.077 & -0.071 & NA & -0.083 & -0.5\\
\hline
\hspace{1em}Z & 1 & ? & 0 & 0.000 & 0.000 & NA & -0.083 & -0.5\\
\hline
\multicolumn{9}{l}{\rule{0pt}{1em}\textit{Note: } n = number of responses in sample}\\
\end{tabular}

\textbackslash end\{table\}

We see that nearly all of the subjects selected a response consistent
with one of the identified interpretations. Responses that do not accord
with any interpretation are indicated as \texttt{?} .

\begin{longtable}[]{@{}
  >{\raggedright\arraybackslash}p{(\columnwidth - 2\tabcolsep) * \real{0.2500}}
  >{\raggedright\arraybackslash}p{(\columnwidth - 2\tabcolsep) * \real{0.7500}}@{}}
\toprule()
\begin{minipage}[b]{\linewidth}\raggedright
Which shifts start at 11am?
\end{minipage} & \begin{minipage}[b]{\linewidth}\raggedright
\end{minipage} \\
\midrule()
\endhead
\includegraphics[width=5.20833in,height=\textheight]{analysis/SGC3A/static/interpretations/Q1_111_A.png}
& \begin{minipage}[t]{\linewidth}\raggedright
\textbf{Response: A}

\begin{itemize}
\item
  indicates an \textbf{orthogonal} (incorrect) interpretation of the
  coordinate system
\item
  Consistent with the reader identifying the reference point (11am) on
  the x-axis, \emph{projecting an invisible orthogonal line upward}, and
  locating data point \textbf{A}.
\end{itemize}
\end{minipage} \\
\includegraphics[width=5.20833in,height=\textheight]{analysis/SGC3A/static/interpretations/Q1_111_F.png}
& \begin{minipage}[t]{\linewidth}\raggedright
\textbf{Response: F}

\begin{itemize}
\item
  indicates the \textbf{triangular} (correct) interpretation of the
  coordinate system
\item
  Consistent with the reader identifying the reference point (11am) on
  the x-axis, and following the right-diagonal gridline, identifying
  data point \textbf{F}.
\end{itemize}
\end{minipage} \\
\includegraphics[width=5.20833in,height=\textheight]{analysis/SGC3A/static/interpretations/Q1_111_CF.png}
& \begin{minipage}[t]{\linewidth}\raggedright
\textbf{Response: C, F}

\begin{itemize}
\item
  indicates a \textbf{maximal-Tversky} strategy following connecting
  lines
\item
  Consistent with the reader identifying the reference point (11am) on
  the x-axis, and following \emph{both} the right-diagonal and
  left-diagonal gridlines, identifying both datapoints F and C.
\end{itemize}
\end{minipage} \\
\includegraphics[width=5.20833in,height=\textheight]{analysis/SGC3A/static/interpretations/Q1_111_AF.png}
& \begin{minipage}[t]{\linewidth}\raggedright
\textbf{Response: A , F}

\begin{itemize}
\item
  The reader selects both triangular and orthogonal-consistent data
  points
\item
  Possibly indicates uncertainty or confusion
\end{itemize}
\end{minipage} \\
\bottomrule()
\end{longtable}

Three responses were given that were not consistent with any of the
identified interpretations. \emph{Note that options highlighted in light
grey are considered within the range of `visual error', defined by 0.5hr
offset from the interpretation-specific projection.}

\begin{longtable}[]{@{}
  >{\raggedright\arraybackslash}p{(\columnwidth - 4\tabcolsep) * \real{0.3000}}
  >{\raggedright\arraybackslash}p{(\columnwidth - 4\tabcolsep) * \real{0.3000}}
  >{\raggedright\arraybackslash}p{(\columnwidth - 4\tabcolsep) * \real{0.3000}}@{}}
\toprule()
\begin{minipage}[b]{\linewidth}\raggedright
\textbf{D I J}
\end{minipage} & \begin{minipage}[b]{\linewidth}\raggedright
\textbf{X}
\end{minipage} & \begin{minipage}[b]{\linewidth}\raggedright
\textbf{Z}

TODO find this person, did they subsequently give tri answers?
\end{minipage} \\
\midrule()
\endhead
\includegraphics[width=5.20833in,height=\textheight]{analysis/SGC3A/static/interpretations/Q1_111_IJD.png}
& \includegraphics{analysis/SGC3A/static/interpretations/Q1_111_X.png} &
\includegraphics{analysis/SGC3A/static/interpretations/Q1_111_Z.png} \\
\bottomrule()
\end{longtable}

\hypertarget{q1.-impasse-condition}{%
\paragraph{Q1. Impasse Condition}\label{q1.-impasse-condition}}

Next we explore the range of response options checked by participants on
Question 1, for those assigned to the control (non-impasse) condition
(\texttt{condition} = 111).

\begin{figure}

{\centering \includegraphics{analysis/SGC3A/static/questions/Q1_121.png}

}

\caption{\label{fig-Q1-121}Question 1 --- Impasse Condition}

\end{figure}

\begin{Shaded}
\begin{Highlighting}[]
\NormalTok{q }\OtherTok{\textless{}{-}}\NormalTok{ keys\_raw }\SpecialCharTok{\%\textgreater{}\%} \FunctionTok{filter}\NormalTok{(condition }\SpecialCharTok{==} \DecValTok{121}\NormalTok{) }\SpecialCharTok{\%\textgreater{}\%} \FunctionTok{filter}\NormalTok{(Q}\SpecialCharTok{==}\DecValTok{1}\NormalTok{)}
\NormalTok{ignore }\OtherTok{\textless{}{-}}\NormalTok{ q }\SpecialCharTok{\%\textgreater{}\%} \FunctionTok{select}\NormalTok{(}\StringTok{"REF\_POINT"}\NormalTok{)}
\NormalTok{answers }\OtherTok{\textless{}{-}}\NormalTok{ q }\SpecialCharTok{\%\textgreater{}\%} \FunctionTok{select}\NormalTok{(}\StringTok{"TRIANGULAR"}\NormalTok{, }\StringTok{"ORTHOGONAL"}\NormalTok{, }\StringTok{"SATISFICE\_left"}\NormalTok{, }\StringTok{"SATISFICE\_right"}\NormalTok{,}\StringTok{"TV\_max"}\NormalTok{,}\StringTok{"TV\_start"}\NormalTok{, }\StringTok{"TV\_end"}\NormalTok{, }\StringTok{"TV\_dur"}\NormalTok{) }\SpecialCharTok{\%\textgreater{}\%} \FunctionTok{unlist}\NormalTok{()}
\NormalTok{ves }\OtherTok{\textless{}{-}}\NormalTok{ q }\SpecialCharTok{\%\textgreater{}\%} \FunctionTok{mutate}\NormalTok{(}
  \AttributeTok{SATISFICE\_left\_allow =} \StringTok{""}\NormalTok{,}
  \AttributeTok{SATISFICE\_right\_allow =} \StringTok{""}
\NormalTok{) }\SpecialCharTok{\%\textgreater{}\%} \FunctionTok{select}\NormalTok{(}\StringTok{"TRI\_allow"}\NormalTok{, }\StringTok{"ORTH\_allow"}\NormalTok{, }\StringTok{"SATISFICE\_left\_allow"}\NormalTok{,}\StringTok{"SATISFICE\_right\_allow"}\NormalTok{, }\StringTok{"TV\_max\_allow"}\NormalTok{,}\StringTok{"TV\_start\_allow"}\NormalTok{,}\StringTok{"TV\_end\_allow"}\NormalTok{, }\StringTok{"TV\_dur\_allow"}\NormalTok{)}\SpecialCharTok{\%\textgreater{}\%} \FunctionTok{unlist}\NormalTok{() }
\NormalTok{options }\OtherTok{\textless{}{-}}\NormalTok{ q }\SpecialCharTok{\%\textgreater{}\%} \FunctionTok{select}\NormalTok{(}\StringTok{"OPTIONS"}\NormalTok{)}
\NormalTok{question }\OtherTok{=}\NormalTok{ q }\SpecialCharTok{\%\textgreater{}\%}  \FunctionTok{select}\NormalTok{(}\StringTok{"TEXT"}\NormalTok{)}
\NormalTok{scores }\OtherTok{\textless{}{-}} \FunctionTok{c}\NormalTok{(}\StringTok{"Triangular"}\NormalTok{, }\StringTok{"Orthgonal"}\NormalTok{, }\StringTok{"Satisficing [left]"}\NormalTok{, }\StringTok{"Satisficing [right]"}\NormalTok{, }\StringTok{"Tversky [maximal]"}\NormalTok{, }\StringTok{"Tversky [start diagonal]"}\NormalTok{, }
            \StringTok{"Tversky [end diagonal]"}\NormalTok{, }\StringTok{"Tversky [duration line]"}\NormalTok{)}
\NormalTok{d }\OtherTok{=} \FunctionTok{tibble}\NormalTok{(}\AttributeTok{interpretation =}\NormalTok{ scores, }\AttributeTok{answer =}\NormalTok{ answers, }\AttributeTok{allowed=}\NormalTok{ves)}
\NormalTok{d}\SpecialCharTok{$}\NormalTok{answer }\OtherTok{\textless{}{-}} \FunctionTok{replace\_na}\NormalTok{(d}\SpecialCharTok{$}\NormalTok{answer, }\StringTok{""}\NormalTok{)}
\NormalTok{d}\SpecialCharTok{$}\NormalTok{allowed }\OtherTok{\textless{}{-}} \FunctionTok{replace\_na}\NormalTok{(d}\SpecialCharTok{$}\NormalTok{allowed, }\StringTok{""}\NormalTok{)}

\NormalTok{title }\OtherTok{=} \FunctionTok{paste}\NormalTok{(}\StringTok{"Answer Key | Q1 Impasse Condition : "}\NormalTok{, question)}
\NormalTok{cols }\OtherTok{=} \FunctionTok{c}\NormalTok{(}\StringTok{"interpretation"}\NormalTok{, }\StringTok{"answer"}\NormalTok{,}\StringTok{"not penalized"}\NormalTok{)}

\NormalTok{d }\SpecialCharTok{\%\textgreater{}\%} \FunctionTok{kbl}\NormalTok{(}\AttributeTok{caption =}\NormalTok{ title, }\AttributeTok{col.names =}\NormalTok{ cols) }\SpecialCharTok{\%\textgreater{}\%} \FunctionTok{kable\_classic}\NormalTok{() }\SpecialCharTok{\%\textgreater{}\%} 
  \FunctionTok{footnote}\NormalTok{(}\AttributeTok{general =} \FunctionTok{paste}\NormalTok{(}\StringTok{"15 response options: "}\NormalTok{, options), }\AttributeTok{general\_title =} \StringTok{"Note: "}\NormalTok{,}\AttributeTok{footnote\_as\_chunk =}\NormalTok{ T) }
\end{Highlighting}
\end{Shaded}

\begin{table}

\caption{Answer Key | Q1 Impasse Condition :  Which shift(s) start at 11 am?}
\centering
\begin{tabular}[t]{l|l|l}
\hline
interpretation & answer & not penalized\\
\hline
Triangular & F & \\
\hline
Orthgonal &  & \\
\hline
Satisficing [left] & O & \\
\hline
Satisficing [right] & AI & \\
\hline
Tversky [maximal] & CF & \\
\hline
Tversky [start diagonal] & F & \\
\hline
Tversky [end diagonal] & C & \\
\hline
Tversky [duration line] &  & \\
\hline
\multicolumn{3}{l}{\rule{0pt}{1em}\textit{Note: } 15 response options:  AIKGXJDBCHUZOFE}\\
\end{tabular}
\end{table}

Notice that there \textbf{is no orthogonal answer} for this question.
This is the purpose of the impasse condition, to remove the possibility
of selecting the orthogonal answer, we expect learners will be more
likely to restructure their understanding of the coordinate system, and
arrive at a correct (triangular) interpretation.

\begin{Shaded}
\begin{Highlighting}[]
\NormalTok{title }\OtherTok{\textless{}{-}} \StringTok{"Frequency of Selected Response Options for Question \#1 (Impasse Condition)"}
\NormalTok{names }\OtherTok{=} \FunctionTok{c}\NormalTok{(}\StringTok{"response"}\NormalTok{,}\StringTok{"n"}\NormalTok{,}\StringTok{"interpretation"}\NormalTok{,}\StringTok{"absolute"}\NormalTok{,}\StringTok{"tri"}\NormalTok{,}\StringTok{"tversky"}\NormalTok{,}\StringTok{"satisfice"}\NormalTok{,}\StringTok{"orthogonal"}\NormalTok{, }\StringTok{"scaled score"}\NormalTok{)}

\NormalTok{df\_items }\SpecialCharTok{\%\textgreater{}\%} \FunctionTok{filter}\NormalTok{(q }\SpecialCharTok{==} \DecValTok{1} \SpecialCharTok{\&}\NormalTok{ condition }\SpecialCharTok{==} \DecValTok{121}\NormalTok{) }\SpecialCharTok{\%\textgreater{}\%} \FunctionTok{group\_by}\NormalTok{(response) }\SpecialCharTok{\%\textgreater{}\%} 
\NormalTok{  dplyr}\SpecialCharTok{::}\FunctionTok{summarise}\NormalTok{( }\AttributeTok{count =} \FunctionTok{n}\NormalTok{(), }
                    \AttributeTok{nice =} \FunctionTok{unique}\NormalTok{(score\_niceABS),}
                    \AttributeTok{triangular =} \FunctionTok{unique}\NormalTok{(score\_TRI), }
                    \AttributeTok{orthogonal =}  \FunctionTok{unique}\NormalTok{(score\_ORTH),}
                    \AttributeTok{satisficing =}  \FunctionTok{unique}\NormalTok{(score\_SATISFICE),}
                    \AttributeTok{tversky =} \FunctionTok{unique}\NormalTok{(score\_TVERSKY),}
                    \AttributeTok{interpretation =} \FunctionTok{unique}\NormalTok{(int2),}
                    \AttributeTok{scaled =} \FunctionTok{unique}\NormalTok{(score\_SCALED)) }\SpecialCharTok{\%\textgreater{}\%} 
  \FunctionTok{arrange}\NormalTok{(interpretation, }\FunctionTok{desc}\NormalTok{(count)) }\SpecialCharTok{\%\textgreater{}\%} 
  \FunctionTok{select}\NormalTok{(response, count, interpretation, nice, }
\NormalTok{         triangular, tversky, satisficing, orthogonal, scaled) }\SpecialCharTok{\%\textgreater{}\%} 
  \FunctionTok{kbl}\NormalTok{(}\AttributeTok{caption =}\NormalTok{ title, }\AttributeTok{col.names =}\NormalTok{ names) }\SpecialCharTok{\%\textgreater{}\%}  \FunctionTok{kable\_classic}\NormalTok{() }\SpecialCharTok{\%\textgreater{}\%} 
  \FunctionTok{add\_header\_above}\NormalTok{(}\FunctionTok{c}\NormalTok{(}\StringTok{" "} \OtherTok{=} \DecValTok{3}\NormalTok{, }\StringTok{"Strict Score"} \OtherTok{=} \DecValTok{1}\NormalTok{, }\StringTok{"Interpretation Scores"}\OtherTok{=}\DecValTok{4}\NormalTok{, }\StringTok{"Discriminant"}\OtherTok{=}\DecValTok{1}\NormalTok{)) }\SpecialCharTok{\%\textgreater{}\%}
  \FunctionTok{pack\_rows}\NormalTok{(}\StringTok{"Triangular"}\NormalTok{, }\DecValTok{1}\NormalTok{, }\DecValTok{1}\NormalTok{) }\SpecialCharTok{\%\textgreater{}\%} 
  \FunctionTok{pack\_rows}\NormalTok{(}\StringTok{"Lines{-}Connect"}\NormalTok{, }\DecValTok{2}\NormalTok{, }\DecValTok{4}\NormalTok{) }\SpecialCharTok{\%\textgreater{}\%} 
  \FunctionTok{pack\_rows}\NormalTok{(}\StringTok{"Satisfice"}\NormalTok{, }\DecValTok{5}\NormalTok{, }\DecValTok{9}\NormalTok{) }\SpecialCharTok{\%\textgreater{}\%} 
  \FunctionTok{pack\_rows}\NormalTok{(}\StringTok{"Other"}\NormalTok{, }\DecValTok{10}\NormalTok{, }\DecValTok{10}\NormalTok{) }\SpecialCharTok{\%\textgreater{}\%} 
  \FunctionTok{pack\_rows}\NormalTok{(}\StringTok{"Unknown"}\NormalTok{, }\DecValTok{11}\NormalTok{, }\DecValTok{12}\NormalTok{) }\SpecialCharTok{\%\textgreater{}\%} 
  \FunctionTok{footnote}\NormalTok{(}\AttributeTok{general =} \StringTok{"n = number of responses in sample"}\NormalTok{, }
           \AttributeTok{general\_title =} \StringTok{"Note: "}\NormalTok{,}\AttributeTok{footnote\_as\_chunk =}\NormalTok{ T) }
\end{Highlighting}
\end{Shaded}

\textbackslash begin\{table\}

\textbackslash caption\{\label{tab:Q1-IMPASSE-RESPONSES}Frequency of
Selected Response Options for Question \#1 (Impasse Condition)\}
\centering

\begin{tabular}[t]{l|r|l|r|r|r|r|r|r}
\hline
\multicolumn{3}{c|}{ } & \multicolumn{1}{c|}{Strict Score} & \multicolumn{4}{c|}{Interpretation Scores} & \multicolumn{1}{c}{Discriminant} \\
\cline{4-4} \cline{5-8} \cline{9-9}
response & n & interpretation & absolute & tri & tversky & satisfice & orthogonal & scaled score\\
\hline
\multicolumn{9}{l}{\textbf{Triangular}}\\
\hline
\hspace{1em}F & 49 & Triangular & 1 & 1.000 & 1.000 & -0.071 & NA & 1.0\\
\hline
\multicolumn{9}{l}{\textbf{Lines-Connect}}\\
\hline
\hspace{1em}CF & 14 & Tversky & 0 & 0.929 & 1.000 & -0.143 & NA & 0.5\\
\hline
\hspace{1em}C & 3 & Tversky & 0 & -0.071 & 1.000 & -0.071 & NA & 0.5\\
\hline
\hspace{1em}CO & 1 & Tversky & 0 & -0.143 & 0.929 & 0.929 & NA & 0.5\\
\hline
\multicolumn{9}{l}{\textbf{Satisfice}}\\
\hline
\hspace{1em}O & 28 & Satisfice & 0 & -0.071 & -0.071 & 1.000 & NA & -1.0\\
\hline
\hspace{1em}AI & 9 & Satisfice & 0 & -0.143 & -0.143 & 1.000 & NA & -1.0\\
\hline
\hspace{1em}A & 4 & Satisfice & 0 & -0.071 & -0.071 & 0.500 & NA & -1.0\\
\hline
\hspace{1em}AO & 2 & Satisfice & 0 & -0.143 & -0.143 & 0.929 & NA & -1.0\\
\hline
\hspace{1em}I & 2 & Satisfice & 0 & -0.071 & -0.071 & 0.500 & NA & -1.0\\
\hline
\multicolumn{9}{l}{\textbf{Other}}\\
\hline
\hspace{1em} & 57 & blank & 0 & 0.000 & 0.000 & NA & NA & 0.0\\
\hline
\multicolumn{9}{l}{\textbf{Unknown}}\\
\hline
\hspace{1em}E & 2 & ? & 0 & -0.071 & -0.071 & -0.071 & NA & -0.5\\
\hline
\hspace{1em}X & 1 & ? & 0 & -0.071 & -0.071 & -0.071 & NA & -0.5\\
\hline
\multicolumn{9}{l}{\rule{0pt}{1em}\textit{Note: } n = number of responses in sample}\\
\end{tabular}

\textbackslash end\{table\}

We see that nearly all of the subjects selected a response consistent
with one of the identified interpretations. Responses that do not accord
with any interpretation are indicated as \texttt{?} .

TODO ADJUST `both' to select for both tri/satisfice or both tri/orth

\begin{longtable}[]{@{}
  >{\raggedright\arraybackslash}p{(\columnwidth - 2\tabcolsep) * \real{0.2500}}
  >{\raggedright\arraybackslash}p{(\columnwidth - 2\tabcolsep) * \real{0.7500}}@{}}
\toprule()
\begin{minipage}[b]{\linewidth}\raggedright
Which shifts start at 11am?
\end{minipage} & \begin{minipage}[b]{\linewidth}\raggedright
\end{minipage} \\
\midrule()
\endhead
\includegraphics{analysis/SGC3A/static/interpretations/Q1_121_F.png} &
\begin{minipage}[t]{\linewidth}\raggedright
\textbf{Response: F}

\begin{itemize}
\item
  indicates the \textbf{triangular} (correct) interpretation of the
  coordinate system
\item
  Consistent with the reader identifying the reference point (11am) on
  the x-axis, and following the right-diagonal gridline, identifying
  data point \textbf{F}.
\end{itemize}
\end{minipage} \\
\includegraphics[width=5.20833in,height=\textheight]{analysis/SGC3A/static/interpretations/Q1_111_CF.png}
& \begin{minipage}[t]{\linewidth}\raggedright
\textbf{Response: {[}C, F{]}}

\begin{itemize}
\item
  indicates a \textbf{maximal-Tversky} strategy following connecting
  lines
\item
  Consistent with the reader identifying the reference point (11am) on
  the x-axis, and following \emph{both} the right-diagonal and
  left-diagonal gridlines, identifying both datapoints F and C gridline.
\end{itemize}
\end{minipage} \\
\includegraphics{analysis/SGC3A/static/interpretations/Q1_121_SATISFICE.png}
& \begin{minipage}[t]{\linewidth}\raggedright
\textbf{Responses: {[}AOI{]}}

\begin{itemize}
\item
  indicates a \textbf{satisficing} strategy
\item
  Consistent with the reader identifying the datapoints nearest to the
  orthogonal projection from the reference point point
\end{itemize}
\end{minipage} \\
\bottomrule()
\end{longtable}

Two responses were given that were not consistent with any of the
identified interpretations.

\begin{longtable}[]{@{}
  >{\raggedright\arraybackslash}p{(\columnwidth - 0\tabcolsep) * \real{0.3000}}@{}}
\toprule()
\begin{minipage}[b]{\linewidth}\raggedright
\textbf{{[}E{]},{[}X{]}}
\end{minipage} \\
\midrule()
\endhead
\includegraphics{analysis/SGC3A/static/interpretations/Q1_121_EX.png} \\
\bottomrule()
\end{longtable}

\begin{Shaded}
\begin{Highlighting}[]
\FunctionTok{gf\_dhistogram}\NormalTok{(}\SpecialCharTok{\textasciitilde{}}\NormalTok{ score\_niceABS, }\AttributeTok{fill =} \SpecialCharTok{\textasciitilde{}}\NormalTok{condition, }\AttributeTok{data =}\NormalTok{ df\_items }\SpecialCharTok{\%\textgreater{}\%} \FunctionTok{filter}\NormalTok{(q }\SpecialCharTok{==}\DecValTok{1}\NormalTok{)) }\SpecialCharTok{\%\textgreater{}\%} 
  \FunctionTok{gf\_facet\_grid}\NormalTok{( condition }\SpecialCharTok{\textasciitilde{}}\NormalTok{ ., }\AttributeTok{labeller =}\NormalTok{ label\_both) }\SpecialCharTok{+} 
  \FunctionTok{labs}\NormalTok{( }\AttributeTok{x =} \StringTok{"Scaled Item Score"}\NormalTok{, }\AttributeTok{title =} \StringTok{"Distribution of Scaled Scores | Q1 "}\NormalTok{) }\SpecialCharTok{+} 
  \FunctionTok{theme\_minimal}\NormalTok{() }\SpecialCharTok{+} \FunctionTok{theme}\NormalTok{(}\AttributeTok{legend.position =} \StringTok{"blank"}\NormalTok{)}
\end{Highlighting}
\end{Shaded}

\begin{figure}[H]

{\centering \includegraphics{analysis/SGC3A/2_sgc3A_scoring_files/figure-pdf/Q1-distribution-1.pdf}

}

\end{figure}

\begin{Shaded}
\begin{Highlighting}[]
\FunctionTok{gf\_props}\NormalTok{(}\SpecialCharTok{\textasciitilde{}}\NormalTok{interpretation, }\AttributeTok{fill =} \SpecialCharTok{\textasciitilde{}}\NormalTok{condition, }\AttributeTok{data =}\NormalTok{ df\_items }\SpecialCharTok{\%\textgreater{}\%} \FunctionTok{filter}\NormalTok{(q }\SpecialCharTok{==}\DecValTok{1}\NormalTok{)) }\SpecialCharTok{\%\textgreater{}\%} 
  \FunctionTok{gf\_facet\_grid}\NormalTok{( condition }\SpecialCharTok{\textasciitilde{}}\NormalTok{ ., }\AttributeTok{labeller =}\NormalTok{ label\_both) }\SpecialCharTok{+} 
  \FunctionTok{labs}\NormalTok{( }\AttributeTok{x =} \StringTok{"Interpretation"}\NormalTok{, }\AttributeTok{title =} \StringTok{"Distribution of Interpretations | Q1 "}\NormalTok{) }\SpecialCharTok{+} 
  \FunctionTok{theme\_minimal}\NormalTok{() }\SpecialCharTok{+} \FunctionTok{theme}\NormalTok{(}\AttributeTok{legend.position =} \StringTok{"blank"}\NormalTok{)}
\end{Highlighting}
\end{Shaded}

\begin{figure}[H]

{\centering \includegraphics{analysis/SGC3A/2_sgc3A_scoring_files/figure-pdf/Q1-distribution-2.pdf}

}

\end{figure}

\hypertarget{question-2}{%
\subsubsection{Question \#2}\label{question-2}}

\hypertarget{q2.-control-condition}{%
\paragraph{Q2. Control Condition}\label{q2.-control-condition}}

\begin{figure}

{\centering \includegraphics{analysis/SGC3A/static/questions/Q2_111.png}

}

\caption{\label{fig-Q2-111}Q2---Control Condition}

\end{figure}

\begin{Shaded}
\begin{Highlighting}[]
\NormalTok{q }\OtherTok{\textless{}{-}}\NormalTok{ keys\_raw }\SpecialCharTok{\%\textgreater{}\%} \FunctionTok{filter}\NormalTok{(condition }\SpecialCharTok{==} \StringTok{"DEFAULT"}\NormalTok{) }\SpecialCharTok{\%\textgreater{}\%} \FunctionTok{filter}\NormalTok{(Q}\SpecialCharTok{==}\DecValTok{2}\NormalTok{)}
\NormalTok{ignore }\OtherTok{\textless{}{-}}\NormalTok{ q }\SpecialCharTok{\%\textgreater{}\%} \FunctionTok{select}\NormalTok{(}\StringTok{"REF\_POINT"}\NormalTok{)}
\NormalTok{answers }\OtherTok{\textless{}{-}}\NormalTok{ q }\SpecialCharTok{\%\textgreater{}\%} \FunctionTok{select}\NormalTok{(}\StringTok{"TRIANGULAR"}\NormalTok{, }\StringTok{"ORTHOGONAL"}\NormalTok{, }\StringTok{"SATISFICE\_left"}\NormalTok{, }\StringTok{"SATISFICE\_right"}\NormalTok{,}\StringTok{"TV\_max"}\NormalTok{,}\StringTok{"TV\_start"}\NormalTok{, }\StringTok{"TV\_end"}\NormalTok{, }\StringTok{"TV\_dur"}\NormalTok{) }\SpecialCharTok{\%\textgreater{}\%} \FunctionTok{unlist}\NormalTok{()}
\NormalTok{ves }\OtherTok{\textless{}{-}}\NormalTok{ q }\SpecialCharTok{\%\textgreater{}\%} \FunctionTok{mutate}\NormalTok{(}
  \AttributeTok{SATISFICE\_left\_allow =} \StringTok{""}\NormalTok{,}
  \AttributeTok{SATISFICE\_right\_allow =} \StringTok{""}
\NormalTok{) }\SpecialCharTok{\%\textgreater{}\%} \FunctionTok{select}\NormalTok{(}\StringTok{"TRI\_allow"}\NormalTok{, }\StringTok{"ORTH\_allow"}\NormalTok{, }\StringTok{"SATISFICE\_left\_allow"}\NormalTok{,}\StringTok{"SATISFICE\_right\_allow"}\NormalTok{, }\StringTok{"TV\_max\_allow"}\NormalTok{,}\StringTok{"TV\_start\_allow"}\NormalTok{,}\StringTok{"TV\_end\_allow"}\NormalTok{, }\StringTok{"TV\_dur\_allow"}\NormalTok{)}\SpecialCharTok{\%\textgreater{}\%} \FunctionTok{unlist}\NormalTok{() }
\NormalTok{options }\OtherTok{\textless{}{-}}\NormalTok{ q }\SpecialCharTok{\%\textgreater{}\%} \FunctionTok{select}\NormalTok{(}\StringTok{"OPTIONS"}\NormalTok{)}
\NormalTok{question }\OtherTok{=}\NormalTok{ q }\SpecialCharTok{\%\textgreater{}\%}  \FunctionTok{select}\NormalTok{(}\StringTok{"TEXT"}\NormalTok{)}
\NormalTok{scores }\OtherTok{\textless{}{-}} \FunctionTok{c}\NormalTok{(}\StringTok{"Triangular"}\NormalTok{, }\StringTok{"Orthgonal"}\NormalTok{, }\StringTok{"Satisficing [left]"}\NormalTok{, }\StringTok{"Satisficing [right]"}\NormalTok{, }\StringTok{"Tversky [maximal]"}\NormalTok{, }\StringTok{"Tversky [start diagonal]"}\NormalTok{, }
            \StringTok{"Tversky [end diagonal]"}\NormalTok{, }\StringTok{"Tversky [duration line]"}\NormalTok{)}
\NormalTok{d }\OtherTok{=} \FunctionTok{tibble}\NormalTok{(}\AttributeTok{interpretation =}\NormalTok{ scores, }\AttributeTok{answer =}\NormalTok{ answers, }\AttributeTok{allowed=}\NormalTok{ves)}
\NormalTok{d}\SpecialCharTok{$}\NormalTok{answer }\OtherTok{\textless{}{-}} \FunctionTok{replace\_na}\NormalTok{(d}\SpecialCharTok{$}\NormalTok{answer, }\StringTok{""}\NormalTok{)}
\NormalTok{d}\SpecialCharTok{$}\NormalTok{allowed }\OtherTok{\textless{}{-}} \FunctionTok{replace\_na}\NormalTok{(d}\SpecialCharTok{$}\NormalTok{allowed, }\StringTok{""}\NormalTok{)}

\NormalTok{title }\OtherTok{=} \FunctionTok{paste}\NormalTok{(}\StringTok{"Answer Key | Q2 Control Condition : "}\NormalTok{, question)}
\NormalTok{cols }\OtherTok{=} \FunctionTok{c}\NormalTok{(}\StringTok{"interpretation"}\NormalTok{, }\StringTok{"answer"}\NormalTok{,}\StringTok{"not penalized"}\NormalTok{)}

\NormalTok{d }\SpecialCharTok{\%\textgreater{}\%} \FunctionTok{kbl}\NormalTok{(}\AttributeTok{caption =}\NormalTok{ title, }\AttributeTok{col.names =}\NormalTok{ cols) }\SpecialCharTok{\%\textgreater{}\%} \FunctionTok{kable\_classic}\NormalTok{() }\SpecialCharTok{\%\textgreater{}\%} 
  \FunctionTok{footnote}\NormalTok{(}\AttributeTok{general =} \FunctionTok{paste}\NormalTok{(}\StringTok{"15 response options: "}\NormalTok{, options), }\AttributeTok{general\_title =} \StringTok{"Note: "}\NormalTok{,}\AttributeTok{footnote\_as\_chunk =}\NormalTok{ T) }
\end{Highlighting}
\end{Shaded}

\begin{table}

\caption{Answer Key | Q2 Control Condition :  Which shift(s) start at the same time as D?}
\centering
\begin{tabular}[t]{l|l|l}
\hline
interpretation & answer & not penalized\\
\hline
Triangular & K & Z\\
\hline
Orthgonal & E & G\\
\hline
Satisficing [left] &  & \\
\hline
Satisficing [right] &  & \\
\hline
Tversky [maximal] & AKJX & Z\\
\hline
Tversky [start diagonal] & AK & Z\\
\hline
Tversky [end diagonal] & X & \\
\hline
Tversky [duration line] & J & \\
\hline
\multicolumn{3}{l}{\rule{0pt}{1em}\textit{Note: } 15 response options:  AIKGXJDBCHUZOFE}\\
\end{tabular}
\end{table}

\begin{Shaded}
\begin{Highlighting}[]
\NormalTok{title }\OtherTok{\textless{}{-}} \StringTok{"Frequency of Selected Response Options for Question \#2 (Control Condition)"}
\NormalTok{names }\OtherTok{=} \FunctionTok{c}\NormalTok{(}\StringTok{"response"}\NormalTok{,}\StringTok{"n"}\NormalTok{,}\StringTok{"interpretation"}\NormalTok{,}\StringTok{"absolute"}\NormalTok{,}\StringTok{"tri"}\NormalTok{,}\StringTok{"tversky"}\NormalTok{,}\StringTok{"satisfice"}\NormalTok{,}\StringTok{"orthogonal"}\NormalTok{, }\StringTok{"scaled score"}\NormalTok{)}

\NormalTok{df\_items }\SpecialCharTok{\%\textgreater{}\%} \FunctionTok{filter}\NormalTok{(q }\SpecialCharTok{==} \DecValTok{2} \SpecialCharTok{\&}\NormalTok{ condition }\SpecialCharTok{==} \DecValTok{111}\NormalTok{) }\SpecialCharTok{\%\textgreater{}\%} \FunctionTok{group\_by}\NormalTok{(response) }\SpecialCharTok{\%\textgreater{}\%} 
\NormalTok{  dplyr}\SpecialCharTok{::}\FunctionTok{summarise}\NormalTok{( }\AttributeTok{count =} \FunctionTok{n}\NormalTok{(), }
                    \AttributeTok{nice =} \FunctionTok{unique}\NormalTok{(score\_niceABS),}
                    \AttributeTok{triangular =} \FunctionTok{unique}\NormalTok{(score\_TRI), }
                    \AttributeTok{orthogonal =}  \FunctionTok{unique}\NormalTok{(score\_ORTH),}
                    \AttributeTok{satisficing =}  \FunctionTok{unique}\NormalTok{(score\_SATISFICE),}
                    \AttributeTok{tversky =} \FunctionTok{unique}\NormalTok{(score\_TVERSKY),}
                    \AttributeTok{interpretation =} \FunctionTok{unique}\NormalTok{(int2),}
                    \AttributeTok{scaled =} \FunctionTok{unique}\NormalTok{(score\_SCALED)) }\SpecialCharTok{\%\textgreater{}\%} 
  \FunctionTok{arrange}\NormalTok{(interpretation, }\FunctionTok{desc}\NormalTok{(count)) }\SpecialCharTok{\%\textgreater{}\%} 
  \FunctionTok{select}\NormalTok{(response, count, interpretation, nice, }
\NormalTok{         triangular, tversky, satisficing, orthogonal, scaled) }\SpecialCharTok{\%\textgreater{}\%} 
  \FunctionTok{kbl}\NormalTok{(}\AttributeTok{caption =}\NormalTok{ title, }\AttributeTok{col.names =}\NormalTok{ names) }\SpecialCharTok{\%\textgreater{}\%}  \FunctionTok{kable\_classic}\NormalTok{() }\SpecialCharTok{\%\textgreater{}\%} 
  \FunctionTok{add\_header\_above}\NormalTok{(}\FunctionTok{c}\NormalTok{(}\StringTok{" "} \OtherTok{=} \DecValTok{3}\NormalTok{, }\StringTok{"Strict Score"} \OtherTok{=} \DecValTok{1}\NormalTok{, }\StringTok{"Interpretation Scores"}\OtherTok{=}\DecValTok{4}\NormalTok{, }\StringTok{"Discriminant"}\OtherTok{=}\DecValTok{1}\NormalTok{)) }\SpecialCharTok{\%\textgreater{}\%}
  \FunctionTok{pack\_rows}\NormalTok{(}\StringTok{"Triangular"}\NormalTok{, }\DecValTok{1}\NormalTok{, }\DecValTok{2}\NormalTok{) }\SpecialCharTok{\%\textgreater{}\%}
  \FunctionTok{pack\_rows}\NormalTok{(}\StringTok{"Lines{-}Connect"}\NormalTok{, }\DecValTok{3}\NormalTok{, }\DecValTok{4}\NormalTok{) }\SpecialCharTok{\%\textgreater{}\%} 
  \FunctionTok{pack\_rows}\NormalTok{(}\StringTok{"Orthogonal"}\NormalTok{, }\DecValTok{5}\NormalTok{, }\DecValTok{7}\NormalTok{) }\SpecialCharTok{\%\textgreater{}\%}
  \FunctionTok{pack\_rows}\NormalTok{(}\StringTok{"Other"}\NormalTok{, }\DecValTok{8}\NormalTok{, }\DecValTok{8}\NormalTok{)  }\SpecialCharTok{\%\textgreater{}\%} 
  \FunctionTok{pack\_rows}\NormalTok{(}\StringTok{"Unknown"}\NormalTok{, }\DecValTok{9}\NormalTok{, }\DecValTok{10}\NormalTok{)  }\SpecialCharTok{\%\textgreater{}\%} 
  \FunctionTok{footnote}\NormalTok{(}\AttributeTok{general =} \StringTok{"n = number of responses in sample"}\NormalTok{, }
           \AttributeTok{general\_title =} \StringTok{"Note: "}\NormalTok{,}\AttributeTok{footnote\_as\_chunk =}\NormalTok{ T) }
\end{Highlighting}
\end{Shaded}

\textbackslash begin\{table\}

\textbackslash caption\{\label{tab:Q2-CONTROL-RESPONSES}Frequency of
Selected Response Options for Question \#2 (Control Condition)\}
\centering

\begin{tabular}[t]{l|r|l|r|r|r|r|r|r}
\hline
\multicolumn{3}{c|}{ } & \multicolumn{1}{c|}{Strict Score} & \multicolumn{4}{c|}{Interpretation Scores} & \multicolumn{1}{c}{Discriminant} \\
\cline{4-4} \cline{5-8} \cline{9-9}
response & n & interpretation & absolute & tri & tversky & satisfice & orthogonal & scaled score\\
\hline
\multicolumn{9}{l}{\textbf{Triangular}}\\
\hline
\hspace{1em}K & 24 & Triangular & 1 & 1.000 & 0.500 & NA & -0.083 & 1.0\\
\hline
\hspace{1em}DK & 1 & Triangular & 1 & 1.000 & 0.500 & NA & -0.083 & 1.0\\
\hline
\multicolumn{9}{l}{\textbf{Lines-Connect}}\\
\hline
\hspace{1em}J & 4 & Tversky & 0 & -0.083 & 1.000 & NA & -0.083 & 0.5\\
\hline
\hspace{1em}AK & 1 & Tversky & 0 & 0.917 & 1.000 & NA & -0.167 & 0.5\\
\hline
\multicolumn{9}{l}{\textbf{Orthogonal}}\\
\hline
\hspace{1em}E & 121 & Orthogonal & 0 & -0.083 & -0.077 & NA & 1.000 & -1.0\\
\hline
\hspace{1em}DE & 3 & Orthogonal & 0 & -0.083 & -0.077 & NA & 1.000 & -1.0\\
\hline
\hspace{1em}EG & 1 & Orthogonal & 0 & -0.167 & -0.154 & NA & 1.000 & -1.0\\
\hline
\multicolumn{9}{l}{\textbf{Other}}\\
\hline
\hspace{1em}D & 1 & reference & 0 & 0.000 & NA & NA & 0.000 & 0.0\\
\hline
\multicolumn{9}{l}{\textbf{Unknown}}\\
\hline
\hspace{1em}B & 1 & ? & 0 & -0.083 & -0.077 & NA & -0.083 & -0.5\\
\hline
\hspace{1em}C & 1 & ? & 0 & -0.083 & -0.077 & NA & -0.083 & -0.5\\
\hline
\multicolumn{9}{l}{\rule{0pt}{1em}\textit{Note: } n = number of responses in sample}\\
\end{tabular}

\textbackslash end\{table\}

Again, we see that most subjects selected a response consistent with one
of the identified interpretations. (note, when the question stem
includes a data point rather than time as reference, we do not penalize
respondents for selecting the reference data point \emph{in addition} to
an interpretation consistent response. For example, in this question, we
do not penalize respondents for selecting option D, the reference point
in the question. )

\begin{longtable}[]{@{}
  >{\raggedright\arraybackslash}p{(\columnwidth - 2\tabcolsep) * \real{0.2500}}
  >{\raggedright\arraybackslash}p{(\columnwidth - 2\tabcolsep) * \real{0.7500}}@{}}
\toprule()
\endhead
\textbf{Which shift(s) start at the same time as D?} & \\
\includegraphics[width=5.20833in,height=\textheight]{analysis/SGC3A/static/interpretations/Q2_111_ORTH.png}
& \begin{minipage}[t]{\linewidth}\raggedright
\textbf{Reponse: E} (also EG, DE)

\begin{itemize}
\item
  indicates an \textbf{orthogonal} (incorrect) interpretation of the
  coordinate system
\item
  Consistent with the reader identifying the reference point (D) on the
  graph, \emph{projecting an invisible orthogonal line through it}, and
  locating data point \textbf{E}.
\end{itemize}
\end{minipage} \\
\includegraphics[width=5.20833in,height=\textheight]{analysis/SGC3A/static/interpretations/Q2_111_TRI.png}
& \begin{minipage}[t]{\linewidth}\raggedright
\textbf{Response: K} (also KD)

\begin{itemize}
\item
  indicates an \textbf{triangular} (correct) interpretation of the
  coordinate system
\item
  Consistent with the reader identifying the reference point (D) on the
  graph, and following its \emph{descending-leftward} \emph{diagonal
  gridline}, and locating data point \textbf{K}.
\end{itemize}
\end{minipage} \\
\includegraphics[width=5.20833in,height=\textheight]{analysis/SGC3A/static/interpretations/Q2_111_AK.png}
& \begin{minipage}[t]{\linewidth}\raggedright
\textbf{Response: AK}

\begin{itemize}
\tightlist
\item
  indicates an \textbf{Tversky} strategy following connecting lines
\item
  Consistent with the reader identifying the reference point (D) on the
  graph, and following its \emph{descending-leftward} \emph{diagonal
  gridline}, and locating data point \textbf{K} then \emph{continuing}
  \emph{along the connecting ascending leftward diagonal} locating data
  point A.
\end{itemize}
\end{minipage} \\
\includegraphics[width=5.20833in,height=\textheight]{analysis/SGC3A/static/interpretations/Q2_111_J.png}
& \begin{minipage}[t]{\linewidth}\raggedright
\textbf{Response: J}

\begin{itemize}
\tightlist
\item
  indicates an \textbf{Tversky} strategy following connecting lines
\item
  Consistent with the reader identifying the reference point (D) on the
  graph, and following its horizontal gridline to the y-axis, locating
  data point J.
\end{itemize}
\end{minipage} \\
\includegraphics[width=5.20833in,height=\textheight]{analysis/SGC3A/static/interpretations/Q2_111_D.png}
& \begin{minipage}[t]{\linewidth}\raggedright
\textbf{Response: D}

\begin{itemize}
\tightlist
\item
  the reader selected only the \textbf{reference point}
\item
  Consistent with the reader identifying the reference point (D) on the
  graph
\item
  Possibly indicates uncertainty or confusion
\end{itemize}
\end{minipage} \\
\bottomrule()
\end{longtable}

\begin{longtable}[]{@{}
  >{\raggedright\arraybackslash}p{(\columnwidth - 2\tabcolsep) * \real{0.3000}}
  >{\raggedright\arraybackslash}p{(\columnwidth - 2\tabcolsep) * \real{0.3000}}@{}}
\toprule()
\begin{minipage}[b]{\linewidth}\raggedright
B
\end{minipage} & \begin{minipage}[b]{\linewidth}\raggedright
C
\end{minipage} \\
\midrule()
\endhead
\includegraphics[width=5.20833in,height=\textheight]{analysis/SGC3A/static/interpretations/Q2_111_B.png}
&
\includegraphics{analysis/SGC3A/static/interpretations/Q2_111_C.png} \\
\bottomrule()
\end{longtable}

\hypertarget{q2.-impasse-condition}{%
\paragraph{Q2. Impasse Condition}\label{q2.-impasse-condition}}

\begin{figure}

{\centering \includegraphics{analysis/SGC3A/static/questions/Q2_121.png}

}

\caption{\label{fig-Q2-121}Q2---Impasse Condition}

\end{figure}

\begin{Shaded}
\begin{Highlighting}[]
\NormalTok{q }\OtherTok{\textless{}{-}}\NormalTok{ keys\_raw }\SpecialCharTok{\%\textgreater{}\%} \FunctionTok{filter}\NormalTok{(condition }\SpecialCharTok{==} \DecValTok{121}\NormalTok{) }\SpecialCharTok{\%\textgreater{}\%} \FunctionTok{filter}\NormalTok{(Q}\SpecialCharTok{==}\DecValTok{2}\NormalTok{)}
\NormalTok{ignore }\OtherTok{\textless{}{-}}\NormalTok{ q }\SpecialCharTok{\%\textgreater{}\%} \FunctionTok{select}\NormalTok{(}\StringTok{"REF\_POINT"}\NormalTok{)}
\NormalTok{answers }\OtherTok{\textless{}{-}}\NormalTok{ q }\SpecialCharTok{\%\textgreater{}\%} \FunctionTok{select}\NormalTok{(}\StringTok{"TRIANGULAR"}\NormalTok{, }\StringTok{"ORTHOGONAL"}\NormalTok{, }\StringTok{"SATISFICE\_left"}\NormalTok{, }\StringTok{"SATISFICE\_right"}\NormalTok{,}\StringTok{"TV\_max"}\NormalTok{,}\StringTok{"TV\_start"}\NormalTok{, }\StringTok{"TV\_end"}\NormalTok{, }\StringTok{"TV\_dur"}\NormalTok{) }\SpecialCharTok{\%\textgreater{}\%} \FunctionTok{unlist}\NormalTok{()}
\NormalTok{ves }\OtherTok{\textless{}{-}}\NormalTok{ q }\SpecialCharTok{\%\textgreater{}\%} \FunctionTok{mutate}\NormalTok{(}
  \AttributeTok{SATISFICE\_left\_allow =} \StringTok{""}\NormalTok{,}
  \AttributeTok{SATISFICE\_right\_allow =} \StringTok{""}
\NormalTok{) }\SpecialCharTok{\%\textgreater{}\%} \FunctionTok{select}\NormalTok{(}\StringTok{"TRI\_allow"}\NormalTok{, }\StringTok{"ORTH\_allow"}\NormalTok{, }\StringTok{"SATISFICE\_left\_allow"}\NormalTok{,}\StringTok{"SATISFICE\_right\_allow"}\NormalTok{, }\StringTok{"TV\_max\_allow"}\NormalTok{,}\StringTok{"TV\_start\_allow"}\NormalTok{,}\StringTok{"TV\_end\_allow"}\NormalTok{, }\StringTok{"TV\_dur\_allow"}\NormalTok{)}\SpecialCharTok{\%\textgreater{}\%} \FunctionTok{unlist}\NormalTok{() }
\NormalTok{options }\OtherTok{\textless{}{-}}\NormalTok{ q }\SpecialCharTok{\%\textgreater{}\%} \FunctionTok{select}\NormalTok{(}\StringTok{"OPTIONS"}\NormalTok{)}
\NormalTok{question }\OtherTok{=}\NormalTok{ q }\SpecialCharTok{\%\textgreater{}\%}  \FunctionTok{select}\NormalTok{(}\StringTok{"TEXT"}\NormalTok{)}
\NormalTok{scores }\OtherTok{\textless{}{-}} \FunctionTok{c}\NormalTok{(}\StringTok{"Triangular"}\NormalTok{, }\StringTok{"Orthgonal"}\NormalTok{, }\StringTok{"Satisficing [left]"}\NormalTok{, }\StringTok{"Satisficing [right]"}\NormalTok{, }\StringTok{"Tversky [maximal]"}\NormalTok{, }\StringTok{"Tversky [start diagonal]"}\NormalTok{, }
            \StringTok{"Tversky [end diagonal]"}\NormalTok{, }\StringTok{"Tversky [duration line]"}\NormalTok{)}
\NormalTok{d }\OtherTok{=} \FunctionTok{tibble}\NormalTok{(}\AttributeTok{interpretation =}\NormalTok{ scores, }\AttributeTok{answer =}\NormalTok{ answers, }\AttributeTok{allowed=}\NormalTok{ves)}
\NormalTok{d}\SpecialCharTok{$}\NormalTok{answer }\OtherTok{\textless{}{-}} \FunctionTok{replace\_na}\NormalTok{(d}\SpecialCharTok{$}\NormalTok{answer, }\StringTok{""}\NormalTok{)}
\NormalTok{d}\SpecialCharTok{$}\NormalTok{allowed }\OtherTok{\textless{}{-}} \FunctionTok{replace\_na}\NormalTok{(d}\SpecialCharTok{$}\NormalTok{allowed, }\StringTok{""}\NormalTok{)}

\NormalTok{title }\OtherTok{=} \FunctionTok{paste}\NormalTok{(}\StringTok{"Answer Key | Q2 Impasse Condition : "}\NormalTok{, question)}
\NormalTok{cols }\OtherTok{=} \FunctionTok{c}\NormalTok{(}\StringTok{"interpretation"}\NormalTok{, }\StringTok{"answer"}\NormalTok{,}\StringTok{"not penalized"}\NormalTok{)}

\NormalTok{d }\SpecialCharTok{\%\textgreater{}\%} \FunctionTok{kbl}\NormalTok{(}\AttributeTok{caption =}\NormalTok{ title, }\AttributeTok{col.names =}\NormalTok{ cols) }\SpecialCharTok{\%\textgreater{}\%} \FunctionTok{kable\_classic}\NormalTok{() }\SpecialCharTok{\%\textgreater{}\%} 
  \FunctionTok{footnote}\NormalTok{(}\AttributeTok{general =} \FunctionTok{paste}\NormalTok{(}\StringTok{"15 response options: "}\NormalTok{, options), }\AttributeTok{general\_title =} \StringTok{"Note: "}\NormalTok{,}\AttributeTok{footnote\_as\_chunk =}\NormalTok{ T) }
\end{Highlighting}
\end{Shaded}

\begin{table}

\caption{Answer Key | Q2 Impasse Condition :  Which shift(s) start at the same time as D?}
\centering
\begin{tabular}[t]{l|l|l}
\hline
interpretation & answer & not penalized\\
\hline
Triangular & K & Z\\
\hline
Orthgonal &  & \\
\hline
Satisficing [left] &  & \\
\hline
Satisficing [right] & G & \\
\hline
Tversky [maximal] & JKE & Z\\
\hline
Tversky [start diagonal] & K & Z\\
\hline
Tversky [end diagonal] & E & \\
\hline
Tversky [duration line] & J & \\
\hline
\multicolumn{3}{l}{\rule{0pt}{1em}\textit{Note: } 15 response options:  AIKGXJDBCHUZOFE}\\
\end{tabular}
\end{table}

\begin{Shaded}
\begin{Highlighting}[]
\NormalTok{title }\OtherTok{\textless{}{-}} \StringTok{"Frequency of Selected Response Options for Question \#2 (Impasse Condition)"}
\NormalTok{names }\OtherTok{=} \FunctionTok{c}\NormalTok{(}\StringTok{"response"}\NormalTok{,}\StringTok{"n"}\NormalTok{,}\StringTok{"interpretation"}\NormalTok{,}\StringTok{"absolute"}\NormalTok{,}\StringTok{"tri"}\NormalTok{,}\StringTok{"tversky"}\NormalTok{,}\StringTok{"satisfice"}\NormalTok{,}\StringTok{"orthogonal"}\NormalTok{, }\StringTok{"scaled score"}\NormalTok{)}

\NormalTok{df\_items }\SpecialCharTok{\%\textgreater{}\%} \FunctionTok{filter}\NormalTok{(q }\SpecialCharTok{==} \DecValTok{2} \SpecialCharTok{\&}\NormalTok{ condition }\SpecialCharTok{==} \DecValTok{121}\NormalTok{) }\SpecialCharTok{\%\textgreater{}\%} \FunctionTok{group\_by}\NormalTok{(response) }\SpecialCharTok{\%\textgreater{}\%} 
\NormalTok{  dplyr}\SpecialCharTok{::}\FunctionTok{summarise}\NormalTok{( }\AttributeTok{count =} \FunctionTok{n}\NormalTok{(), }
                    \AttributeTok{nice =} \FunctionTok{unique}\NormalTok{(score\_niceABS),}
                    \AttributeTok{triangular =} \FunctionTok{unique}\NormalTok{(score\_TRI), }
                    \AttributeTok{orthogonal =}  \FunctionTok{unique}\NormalTok{(score\_ORTH),}
                    \AttributeTok{satisficing =}  \FunctionTok{unique}\NormalTok{(score\_SATISFICE),}
                    \AttributeTok{tversky =} \FunctionTok{unique}\NormalTok{(score\_TVERSKY),}
                    \AttributeTok{interpretation =} \FunctionTok{unique}\NormalTok{(int2),}
                    \AttributeTok{scaled =} \FunctionTok{unique}\NormalTok{(score\_SCALED)) }\SpecialCharTok{\%\textgreater{}\%} 
  \FunctionTok{arrange}\NormalTok{(interpretation, }\FunctionTok{desc}\NormalTok{(count)) }\SpecialCharTok{\%\textgreater{}\%} 
  \FunctionTok{select}\NormalTok{(response, count, interpretation, nice, }
\NormalTok{         triangular, tversky, satisficing, orthogonal, scaled) }\SpecialCharTok{\%\textgreater{}\%} 
  \FunctionTok{kbl}\NormalTok{(}\AttributeTok{caption =}\NormalTok{ title, }\AttributeTok{col.names =}\NormalTok{ names) }\SpecialCharTok{\%\textgreater{}\%}  \FunctionTok{kable\_classic}\NormalTok{() }\SpecialCharTok{\%\textgreater{}\%} 
  \FunctionTok{add\_header\_above}\NormalTok{(}\FunctionTok{c}\NormalTok{(}\StringTok{" "} \OtherTok{=} \DecValTok{3}\NormalTok{, }\StringTok{"Strict Score"} \OtherTok{=} \DecValTok{1}\NormalTok{, }\StringTok{"Interpretation Scores"}\OtherTok{=}\DecValTok{4}\NormalTok{, }\StringTok{"Discriminant"}\OtherTok{=}\DecValTok{1}\NormalTok{)) }\SpecialCharTok{\%\textgreater{}\%}
  \FunctionTok{pack\_rows}\NormalTok{(}\StringTok{"Triangular"}\NormalTok{, }\DecValTok{1}\NormalTok{, }\DecValTok{2}\NormalTok{) }\SpecialCharTok{\%\textgreater{}\%}
  \FunctionTok{pack\_rows}\NormalTok{(}\StringTok{"Lines{-}Connect"}\NormalTok{, }\DecValTok{3}\NormalTok{, }\DecValTok{10}\NormalTok{) }\SpecialCharTok{\%\textgreater{}\%} 
  \FunctionTok{pack\_rows}\NormalTok{(}\StringTok{"Satisfice"}\NormalTok{, }\DecValTok{11}\NormalTok{, }\DecValTok{12}\NormalTok{) }\SpecialCharTok{\%\textgreater{}\%}
  \FunctionTok{pack\_rows}\NormalTok{(}\StringTok{"Other"}\NormalTok{, }\DecValTok{13}\NormalTok{, }\DecValTok{16}\NormalTok{)  }\SpecialCharTok{\%\textgreater{}\%} 
  \FunctionTok{pack\_rows}\NormalTok{(}\StringTok{"Unknown"}\NormalTok{, }\DecValTok{17}\NormalTok{, }\DecValTok{18}\NormalTok{)  }\SpecialCharTok{\%\textgreater{}\%} 
  \FunctionTok{footnote}\NormalTok{(}\AttributeTok{general =} \StringTok{"n = number of responses in sample"}\NormalTok{, }
           \AttributeTok{general\_title =} \StringTok{"Note: "}\NormalTok{,}\AttributeTok{footnote\_as\_chunk =}\NormalTok{ T) }
\end{Highlighting}
\end{Shaded}

\textbackslash begin\{table\}

\textbackslash caption\{\label{tab:Q2-IMPASSE-RESPONSES}Frequency of
Selected Response Options for Question \#2 (Impasse Condition)\}
\centering

\begin{tabular}[t]{l|r|l|r|r|r|r|r|r}
\hline
\multicolumn{3}{c|}{ } & \multicolumn{1}{c|}{Strict Score} & \multicolumn{4}{c|}{Interpretation Scores} & \multicolumn{1}{c}{Discriminant} \\
\cline{4-4} \cline{5-8} \cline{9-9}
response & n & interpretation & absolute & tri & tversky & satisfice & orthogonal & scaled score\\
\hline
\multicolumn{9}{l}{\textbf{Triangular}}\\
\hline
\hspace{1em}K & 69 & Triangular & 1 & 1.000 & 1.000 & -0.077 & NA & 1.0\\
\hline
\hspace{1em}DK & 1 & Triangular & 1 & 1.000 & 1.000 & -0.077 & NA & 1.0\\
\hline
\multicolumn{9}{l}{\textbf{Lines-Connect}}\\
\hline
\hspace{1em}J & 12 & Tversky & 0 & -0.083 & 1.000 & -0.077 & NA & 0.5\\
\hline
\hspace{1em}EK & 3 & Tversky & 0 & 0.917 & 0.923 & -0.154 & NA & 0.5\\
\hline
\hspace{1em}EX & 2 & Tversky & 0 & -0.167 & 0.923 & -0.154 & NA & 0.5\\
\hline
\hspace{1em}BEG & 1 & Tversky & 0 & -0.250 & 0.846 & 0.846 & NA & 0.5\\
\hline
\hspace{1em}E & 1 & Tversky & 0 & -0.083 & 1.000 & -0.077 & NA & 0.5\\
\hline
\hspace{1em}EKX & 1 & Tversky & 0 & 0.833 & 0.846 & -0.231 & NA & 0.5\\
\hline
\hspace{1em}HJZ & 1 & Tversky & 0 & -0.167 & 0.846 & -0.231 & NA & 0.5\\
\hline
\hspace{1em}JK & 1 & Tversky & 0 & 0.917 & 0.923 & -0.154 & NA & 0.5\\
\hline
\multicolumn{9}{l}{\textbf{Satisfice}}\\
\hline
\hspace{1em}G & 19 & Satisfice & 0 & -0.083 & -0.077 & 1.000 & NA & -1.0\\
\hline
\hspace{1em}BG & 2 & Satisfice & 0 & -0.167 & -0.154 & 0.923 & NA & -1.0\\
\hline
\multicolumn{9}{l}{\textbf{Other}}\\
\hline
\hspace{1em}D & 7 & reference & 0 & 0.000 & NA & 0.000 & NA & 0.0\\
\hline
\hspace{1em} & 43 & blank & 0 & 0.000 & NA & 0.000 & NA & 0.0\\
\hline
\hspace{1em}ACDFHIJKOUXZ & 1 & frenzy & 0 & 0.250 & 0.250 & -0.846 & NA & -0.5\\
\hline
\hspace{1em}BEGKUZ & 1 & frenzy & 0 & 0.667 & 0.667 & 0.615 & NA & -0.5\\
\hline
\multicolumn{9}{l}{\textbf{Unknown}}\\
\hline
\hspace{1em}C & 6 & ? & 0 & -0.083 & -0.077 & -0.077 & NA & -0.5\\
\hline
\hspace{1em}FO & 1 & ? & 0 & -0.167 & -0.154 & -0.154 & NA & -0.5\\
\hline
\multicolumn{9}{l}{\rule{0pt}{1em}\textit{Note: } n = number of responses in sample}\\
\end{tabular}

\textbackslash end\{table\}

\begin{Shaded}
\begin{Highlighting}[]
\FunctionTok{gf\_dhistogram}\NormalTok{(}\SpecialCharTok{\textasciitilde{}}\NormalTok{ score\_niceABS, }\AttributeTok{fill =} \SpecialCharTok{\textasciitilde{}}\NormalTok{condition, }\AttributeTok{data =}\NormalTok{ df\_items }\SpecialCharTok{\%\textgreater{}\%} \FunctionTok{filter}\NormalTok{(q }\SpecialCharTok{==}\DecValTok{2}\NormalTok{)) }\SpecialCharTok{\%\textgreater{}\%} 
  \FunctionTok{gf\_facet\_grid}\NormalTok{( condition }\SpecialCharTok{\textasciitilde{}}\NormalTok{ ., }\AttributeTok{labeller =}\NormalTok{ label\_both) }\SpecialCharTok{+} 
  \FunctionTok{labs}\NormalTok{( }\AttributeTok{x =} \StringTok{"Scaled Item Score"}\NormalTok{, }\AttributeTok{title =} \StringTok{"Distribution of Scaled Scores | Q2 "}\NormalTok{) }\SpecialCharTok{+} 
  \FunctionTok{theme\_minimal}\NormalTok{() }\SpecialCharTok{+} \FunctionTok{theme}\NormalTok{(}\AttributeTok{legend.position =} \StringTok{"blank"}\NormalTok{)}
\end{Highlighting}
\end{Shaded}

\begin{figure}[H]

{\centering \includegraphics{analysis/SGC3A/2_sgc3A_scoring_files/figure-pdf/Q2-distribution-1.pdf}

}

\end{figure}

\begin{Shaded}
\begin{Highlighting}[]
\FunctionTok{gf\_props}\NormalTok{(}\SpecialCharTok{\textasciitilde{}}\NormalTok{interpretation, }\AttributeTok{fill =} \SpecialCharTok{\textasciitilde{}}\NormalTok{condition, }\AttributeTok{data =}\NormalTok{ df\_items }\SpecialCharTok{\%\textgreater{}\%} \FunctionTok{filter}\NormalTok{(q }\SpecialCharTok{==}\DecValTok{2}\NormalTok{)) }\SpecialCharTok{\%\textgreater{}\%} 
  \FunctionTok{gf\_facet\_grid}\NormalTok{( condition }\SpecialCharTok{\textasciitilde{}}\NormalTok{ ., }\AttributeTok{labeller =}\NormalTok{ label\_both) }\SpecialCharTok{+} 
  \FunctionTok{labs}\NormalTok{( }\AttributeTok{x =} \StringTok{"Interpretation"}\NormalTok{, }\AttributeTok{title =} \StringTok{"Distribution of Interpretations | Q2 "}\NormalTok{) }\SpecialCharTok{+} 
  \FunctionTok{theme\_minimal}\NormalTok{() }\SpecialCharTok{+} \FunctionTok{theme}\NormalTok{(}\AttributeTok{legend.position =} \StringTok{"blank"}\NormalTok{)}
\end{Highlighting}
\end{Shaded}

\begin{figure}[H]

{\centering \includegraphics{analysis/SGC3A/2_sgc3A_scoring_files/figure-pdf/Q2-distribution-2.pdf}

}

\end{figure}

\hypertarget{question-3}{%
\subsubsection{Question \#3}\label{question-3}}

\hypertarget{q3.-control-condition}{%
\paragraph{Q3. Control Condition}\label{q3.-control-condition}}

\begin{figure}

{\centering \includegraphics{analysis/SGC3A/static/questions/Q3_111.png}

}

\caption{\label{fig-Q3-111}Q3---Control Condition}

\end{figure}

\begin{Shaded}
\begin{Highlighting}[]
\NormalTok{q }\OtherTok{\textless{}{-}}\NormalTok{ keys\_raw }\SpecialCharTok{\%\textgreater{}\%} \FunctionTok{filter}\NormalTok{(condition }\SpecialCharTok{==} \StringTok{"DEFAULT"}\NormalTok{) }\SpecialCharTok{\%\textgreater{}\%} \FunctionTok{filter}\NormalTok{(Q}\SpecialCharTok{==}\DecValTok{3}\NormalTok{)}
\NormalTok{ignore }\OtherTok{\textless{}{-}}\NormalTok{ q }\SpecialCharTok{\%\textgreater{}\%} \FunctionTok{select}\NormalTok{(}\StringTok{"REF\_POINT"}\NormalTok{)}
\NormalTok{answers }\OtherTok{\textless{}{-}}\NormalTok{ q }\SpecialCharTok{\%\textgreater{}\%} \FunctionTok{select}\NormalTok{(}\StringTok{"TRIANGULAR"}\NormalTok{, }\StringTok{"ORTHOGONAL"}\NormalTok{, }\StringTok{"SATISFICE\_left"}\NormalTok{, }\StringTok{"SATISFICE\_right"}\NormalTok{,}\StringTok{"TV\_max"}\NormalTok{,}\StringTok{"TV\_start"}\NormalTok{, }\StringTok{"TV\_end"}\NormalTok{, }\StringTok{"TV\_dur"}\NormalTok{) }\SpecialCharTok{\%\textgreater{}\%} \FunctionTok{unlist}\NormalTok{()}
\NormalTok{ves }\OtherTok{\textless{}{-}}\NormalTok{ q }\SpecialCharTok{\%\textgreater{}\%} \FunctionTok{mutate}\NormalTok{(}
  \AttributeTok{SATISFICE\_left\_allow =} \StringTok{""}\NormalTok{,}
  \AttributeTok{SATISFICE\_right\_allow =} \StringTok{""}
\NormalTok{) }\SpecialCharTok{\%\textgreater{}\%} \FunctionTok{select}\NormalTok{(}\StringTok{"TRI\_allow"}\NormalTok{, }\StringTok{"ORTH\_allow"}\NormalTok{, }\StringTok{"SATISFICE\_left\_allow"}\NormalTok{,}\StringTok{"SATISFICE\_right\_allow"}\NormalTok{, }\StringTok{"TV\_max\_allow"}\NormalTok{,}\StringTok{"TV\_start\_allow"}\NormalTok{,}\StringTok{"TV\_end\_allow"}\NormalTok{, }\StringTok{"TV\_dur\_allow"}\NormalTok{)}\SpecialCharTok{\%\textgreater{}\%} \FunctionTok{unlist}\NormalTok{() }
\NormalTok{options }\OtherTok{\textless{}{-}}\NormalTok{ q }\SpecialCharTok{\%\textgreater{}\%} \FunctionTok{select}\NormalTok{(}\StringTok{"OPTIONS"}\NormalTok{)}
\NormalTok{question }\OtherTok{=}\NormalTok{ q }\SpecialCharTok{\%\textgreater{}\%}  \FunctionTok{select}\NormalTok{(}\StringTok{"TEXT"}\NormalTok{)}
\NormalTok{scores }\OtherTok{\textless{}{-}} \FunctionTok{c}\NormalTok{(}\StringTok{"Triangular"}\NormalTok{, }\StringTok{"Orthgonal"}\NormalTok{, }\StringTok{"Satisficing [left]"}\NormalTok{, }\StringTok{"Satisficing [right]"}\NormalTok{, }\StringTok{"Tversky [maximal]"}\NormalTok{, }\StringTok{"Tversky [start diagonal]"}\NormalTok{, }
            \StringTok{"Tversky [end diagonal]"}\NormalTok{, }\StringTok{"Tversky [duration line]"}\NormalTok{)}
\NormalTok{d }\OtherTok{=} \FunctionTok{tibble}\NormalTok{(}\AttributeTok{interpretation =}\NormalTok{ scores, }\AttributeTok{answer =}\NormalTok{ answers, }\AttributeTok{allowed=}\NormalTok{ves)}
\NormalTok{d}\SpecialCharTok{$}\NormalTok{answer }\OtherTok{\textless{}{-}} \FunctionTok{replace\_na}\NormalTok{(d}\SpecialCharTok{$}\NormalTok{answer, }\StringTok{""}\NormalTok{)}
\NormalTok{d}\SpecialCharTok{$}\NormalTok{allowed }\OtherTok{\textless{}{-}} \FunctionTok{replace\_na}\NormalTok{(d}\SpecialCharTok{$}\NormalTok{allowed, }\StringTok{""}\NormalTok{)}

\NormalTok{title }\OtherTok{=} \FunctionTok{paste}\NormalTok{(}\StringTok{"Answer Key | Q3 Control Condition : "}\NormalTok{, question)}
\NormalTok{cols }\OtherTok{=} \FunctionTok{c}\NormalTok{(}\StringTok{"interpretation"}\NormalTok{, }\StringTok{"answer"}\NormalTok{,}\StringTok{"not penalized"}\NormalTok{)}

\NormalTok{d }\SpecialCharTok{\%\textgreater{}\%} \FunctionTok{kbl}\NormalTok{(}\AttributeTok{caption =}\NormalTok{ title, }\AttributeTok{col.names =}\NormalTok{ cols) }\SpecialCharTok{\%\textgreater{}\%} \FunctionTok{kable\_classic}\NormalTok{() }\SpecialCharTok{\%\textgreater{}\%} 
  \FunctionTok{footnote}\NormalTok{(}\AttributeTok{general =} \FunctionTok{paste}\NormalTok{(}\StringTok{"15 response options: "}\NormalTok{, options), }\AttributeTok{general\_title =} \StringTok{"Note: "}\NormalTok{,}\AttributeTok{footnote\_as\_chunk =}\NormalTok{ T) }
\end{Highlighting}
\end{Shaded}

\begin{table}

\caption{Answer Key | Q3 Control Condition :  Which shift(s) begin when C ends?}
\centering
\begin{tabular}[t]{l|l|l}
\hline
interpretation & answer & not penalized\\
\hline
Triangular & F & Z\\
\hline
Orthgonal & Z & FIO\\
\hline
Satisficing [left] &  & \\
\hline
Satisficing [right] &  & \\
\hline
Tversky [maximal] & AUBFOJ & \\
\hline
Tversky [start diagonal] & OJ & \\
\hline
Tversky [end diagonal] & F & Z\\
\hline
Tversky [duration line] & AUB & \\
\hline
\multicolumn{3}{l}{\rule{0pt}{1em}\textit{Note: } 15 response options:  AIKGXJDBCHUZOFE}\\
\end{tabular}
\end{table}

\begin{Shaded}
\begin{Highlighting}[]
\NormalTok{title }\OtherTok{\textless{}{-}} \StringTok{"Frequency of Selected Response Options for Question \#3 (Control Condition)"}
\NormalTok{names }\OtherTok{=} \FunctionTok{c}\NormalTok{(}\StringTok{"response"}\NormalTok{,}\StringTok{"n"}\NormalTok{,}\StringTok{"interpretation"}\NormalTok{,}\StringTok{"absolute"}\NormalTok{,}\StringTok{"tri"}\NormalTok{,}\StringTok{"tversky"}\NormalTok{,}\StringTok{"satisfice"}\NormalTok{,}\StringTok{"orthogonal"}\NormalTok{, }\StringTok{"scaled score"}\NormalTok{)}

\NormalTok{df\_items }\SpecialCharTok{\%\textgreater{}\%} \FunctionTok{filter}\NormalTok{(q }\SpecialCharTok{==} \DecValTok{3} \SpecialCharTok{\&}\NormalTok{ condition }\SpecialCharTok{==} \DecValTok{111}\NormalTok{) }\SpecialCharTok{\%\textgreater{}\%} \FunctionTok{group\_by}\NormalTok{(response) }\SpecialCharTok{\%\textgreater{}\%} 
\NormalTok{  dplyr}\SpecialCharTok{::}\FunctionTok{summarise}\NormalTok{( }\AttributeTok{count =} \FunctionTok{n}\NormalTok{(), }
                    \AttributeTok{nice =} \FunctionTok{unique}\NormalTok{(score\_niceABS),}
                    \AttributeTok{triangular =} \FunctionTok{unique}\NormalTok{(score\_TRI), }
                    \AttributeTok{orthogonal =}  \FunctionTok{unique}\NormalTok{(score\_ORTH),}
                    \AttributeTok{satisficing =}  \FunctionTok{unique}\NormalTok{(score\_SATISFICE),}
                    \AttributeTok{tversky =} \FunctionTok{unique}\NormalTok{(score\_TVERSKY),}
                    \AttributeTok{interpretation =} \FunctionTok{unique}\NormalTok{(int2),}
                    \AttributeTok{scaled =} \FunctionTok{unique}\NormalTok{(score\_SCALED)) }\SpecialCharTok{\%\textgreater{}\%} 
  \FunctionTok{arrange}\NormalTok{(interpretation, }\FunctionTok{desc}\NormalTok{(count)) }\SpecialCharTok{\%\textgreater{}\%} 
  \FunctionTok{select}\NormalTok{(response, count, interpretation, nice, }
\NormalTok{         triangular, tversky, satisficing, orthogonal, scaled) }\SpecialCharTok{\%\textgreater{}\%} 
  \FunctionTok{kbl}\NormalTok{(}\AttributeTok{caption =}\NormalTok{ title, }\AttributeTok{col.names =}\NormalTok{ names) }\SpecialCharTok{\%\textgreater{}\%}  \FunctionTok{kable\_classic}\NormalTok{() }\SpecialCharTok{\%\textgreater{}\%} 
  \FunctionTok{add\_header\_above}\NormalTok{(}\FunctionTok{c}\NormalTok{(}\StringTok{" "} \OtherTok{=} \DecValTok{3}\NormalTok{, }\StringTok{"Strict Score"} \OtherTok{=} \DecValTok{1}\NormalTok{, }\StringTok{"Interpretation Scores"}\OtherTok{=}\DecValTok{4}\NormalTok{, }\StringTok{"Discriminant"}\OtherTok{=}\DecValTok{1}\NormalTok{)) }\SpecialCharTok{\%\textgreater{}\%}
  \FunctionTok{pack\_rows}\NormalTok{(}\StringTok{"Triangular"}\NormalTok{, }\DecValTok{1}\NormalTok{, }\DecValTok{2}\NormalTok{) }\SpecialCharTok{\%\textgreater{}\%} 
  \FunctionTok{pack\_rows}\NormalTok{(}\StringTok{"Lines{-}Connect"}\NormalTok{, }\DecValTok{3}\NormalTok{, }\DecValTok{7}\NormalTok{) }\SpecialCharTok{\%\textgreater{}\%} 
  \FunctionTok{pack\_rows}\NormalTok{(}\StringTok{"Orthogonal"}\NormalTok{, }\DecValTok{8}\NormalTok{, }\DecValTok{8}\NormalTok{) }\SpecialCharTok{\%\textgreater{}\%} 
  \FunctionTok{pack\_rows}\NormalTok{(}\StringTok{"Other"}\NormalTok{, }\DecValTok{9}\NormalTok{, }\DecValTok{10}\NormalTok{) }\SpecialCharTok{\%\textgreater{}\%} 
  \FunctionTok{pack\_rows}\NormalTok{(}\StringTok{"Unknown"}\NormalTok{, }\DecValTok{11}\NormalTok{, }\DecValTok{17}\NormalTok{)  }
\end{Highlighting}
\end{Shaded}

\textbackslash begin\{table\}

\textbackslash caption\{\label{tab:Q3-CONTROL-RESPONSES}Frequency of
Selected Response Options for Question \#3 (Control Condition)\}
\centering

\begin{tabular}[t]{l|r|l|r|r|r|r|r|r}
\hline
\multicolumn{3}{c|}{ } & \multicolumn{1}{c|}{Strict Score} & \multicolumn{4}{c|}{Interpretation Scores} & \multicolumn{1}{c}{Discriminant} \\
\cline{4-4} \cline{5-8} \cline{9-9}
response & n & interpretation & absolute & tri & tversky & satisfice & orthogonal & scaled score\\
\hline
\multicolumn{9}{l}{\textbf{Triangular}}\\
\hline
\hspace{1em}F & 24 & Triangular & 1 & 1.000 & 1.000 & NA & 0.0 & 1.0\\
\hline
\hspace{1em}EFK & 1 & Triangular & 0 & 0.833 & 0.833 & NA & -0.2 & 1.0\\
\hline
\multicolumn{9}{l}{\textbf{Lines-Connect}}\\
\hline
\hspace{1em}ABU & 4 & Tversky & 0 & -0.250 & 1.000 & NA & -0.3 & 0.5\\
\hline
\hspace{1em}O & 3 & Tversky & 0 & -0.083 & 0.500 & NA & 0.0 & 0.5\\
\hline
\hspace{1em}JO & 2 & Tversky & 0 & -0.167 & 1.000 & NA & -0.1 & 0.5\\
\hline
\hspace{1em}DJO & 1 & Tversky & 0 & -0.250 & 0.917 & NA & -0.2 & 0.5\\
\hline
\hspace{1em}KO & 1 & Tversky & 0 & -0.167 & 0.417 & NA & -0.1 & 0.5\\
\hline
\multicolumn{9}{l}{\textbf{Orthogonal}}\\
\hline
\hspace{1em}Z & 94 & Orthogonal & 0 & 0.000 & 0.000 & NA & 1.0 & -1.0\\
\hline
\multicolumn{9}{l}{\textbf{Other}}\\
\hline
\hspace{1em}C & 1 & reference & 0 & 0.000 & NA & NA & 0.0 & 0.0\\
\hline
\hspace{1em}ABDEFGHIJKOUXZ & 1 & frenzy & 0 & 0.000 & NA & NA & 0.0 & -0.5\\
\hline
\multicolumn{9}{l}{\textbf{Unknown}}\\
\hline
\hspace{1em}A & 18 & ? & 0 & -0.083 & 0.333 & NA & -0.1 & -0.5\\
\hline
\hspace{1em}K & 3 & ? & 0 & -0.083 & -0.083 & NA & -0.1 & -0.5\\
\hline
\hspace{1em}AH & 1 & ? & 0 & -0.167 & 0.242 & NA & -0.2 & -0.5\\
\hline
\hspace{1em}DE & 1 & ? & 0 & -0.167 & -0.167 & NA & -0.2 & -0.5\\
\hline
\hspace{1em}E & 1 & ? & 0 & -0.083 & -0.083 & NA & -0.1 & -0.5\\
\hline
\hspace{1em}EU & 1 & ? & 0 & -0.167 & 0.242 & NA & -0.2 & -0.5\\
\hline
\hspace{1em}U & 1 & ? & 0 & -0.083 & 0.333 & NA & -0.1 & -0.5\\
\hline
\end{tabular}

\textbackslash end\{table\}

TODO

\begin{itemize}
\tightlist
\item
  address RESPONSE FKE which is classified as Triangular but doesn't
  seem to fit this interpretation?
\item
  Should O,K be considered Tvresky ?
\item
  consider adding trapdoor on n\_q, such that score is penalized (OR
  interpretation is not predicted?) if the Ss selects more than 1 extra
  options, or is missing more than 2 options?
\item
  LEFT OFF HERE
\end{itemize}

\begin{longtable}[]{@{}
  >{\raggedright\arraybackslash}p{(\columnwidth - 2\tabcolsep) * \real{0.2500}}
  >{\raggedright\arraybackslash}p{(\columnwidth - 2\tabcolsep) * \real{0.7500}}@{}}
\toprule()
\begin{minipage}[b]{\linewidth}\raggedright
What shift(s) begin when C ends?
\end{minipage} & \begin{minipage}[b]{\linewidth}\raggedright
\end{minipage} \\
\midrule()
\endhead
\includegraphics{analysis/SGC3A/static/interpretations/Q3_111_Z.png} &
\begin{minipage}[t]{\linewidth}\raggedright
\textbf{Response: Z}

\begin{itemize}
\tightlist
\item
  indicates an \textbf{orthogonal} (incorrect) interpretation of the
  coordinate system
\item
  Consistent with the reader identifying the reference point (C) then
  using the duration encoded on the y-axis (2) , project along the
  horizontal gridline by two hours, and then \emph{project an invisible
  orthogonal line through that time (12PM)} locating data point
  \textbf{Z}.
\end{itemize}
\end{minipage} \\
\includegraphics{analysis/SGC3A/static/interpretations/Q3_111_F.png} &
\begin{minipage}[t]{\linewidth}\raggedright
\textbf{Response: F}

\begin{itemize}
\tightlist
\item
  indicates a (correct) \textbf{triangular} interpretation of the
  coordinate system
\item
  Consistent with the reader identifying the reference point (C) on the
  graph, and following the \emph{descending gridline} to the x-axis to
  identify the end-time (11AM) and then following the \emph{ascending
  gridline} to identify datapoints starting at 11AM and locating data
  point \textbf{F}.
\end{itemize}
\end{minipage} \\
\includegraphics{analysis/SGC3A/static/interpretations/Q3_111_AUB.png} &
\begin{minipage}[t]{\linewidth}\raggedright
\textbf{Response: AUB (also A)}

\begin{itemize}
\tightlist
\item
  indicates a Tversky strategy following connecting lines (duration)
\item
  Consistent with the reader identifying the reference point (C) on the
  graph, and following the \emph{horizontal y-axis gridline} and
  locating data points \textbf{A U B}.
\end{itemize}
\end{minipage} \\
\includegraphics{analysis/SGC3A/.pdf}\includegraphics{analysis/SGC3A/static/interpretations/Q3_111_OJ.png}
& \begin{minipage}[t]{\linewidth}\raggedright
\textbf{Response: OJ}

\begin{itemize}
\tightlist
\item
  indicates a Tversky strategy following connecting lines (start-time)
\item
  Consistent with the reader identifying the reference point (C) on the
  graph, and following the \emph{ascending diagonal gridline} and
  locating data points \textbf{O J}.
\end{itemize}
\end{minipage} \\
\includegraphics{analysis/SGC3A/static/interpretations/Q3_111_C.png} &
\begin{minipage}[t]{\linewidth}\raggedright
\textbf{Response: C}

\begin{itemize}
\tightlist
\item
  the participant selected the point referenced in the question
\item
  possibly indicates confusion or uncertainty
\end{itemize}
\end{minipage} \\
\includegraphics{analysis/SGC3A/static/interpretations/Q3_111_frenzy.png}
& \begin{minipage}[t]{\linewidth}\raggedright
\textbf{Response: AIOZFHJXKUDEGB}

\begin{itemize}
\tightlist
\item
  the participant selects \emph{all} (or nearly all) the data points
\item
  possibly indicates confusion or uncertainty
\end{itemize}
\end{minipage} \\
\bottomrule()
\end{longtable}

Six responses (from 9 participants) appear inconsistent with any
interpretation.

\begin{longtable}[]{@{}
  >{\raggedright\arraybackslash}p{(\columnwidth - 4\tabcolsep) * \real{0.3000}}
  >{\raggedright\arraybackslash}p{(\columnwidth - 4\tabcolsep) * \real{0.3000}}
  >{\raggedright\arraybackslash}p{(\columnwidth - 4\tabcolsep) * \real{0.3000}}@{}}
\toprule()
\begin{minipage}[b]{\linewidth}\raggedright
K (\textbf{n=3)}
\end{minipage} & \begin{minipage}[b]{\linewidth}\raggedright
AH (n=1)
\end{minipage} & \begin{minipage}[b]{\linewidth}\raggedright
DE (n=1)
\end{minipage} \\
\midrule()
\endhead
\includegraphics{analysis/SGC3A/.pdf}\includegraphics{analysis/SGC3A/static/interpretations/Q3_111_K.png}
& \includegraphics{analysis/SGC3A/static/interpretations/Q3_111_AH.png}
&
\includegraphics{analysis/SGC3A/static/interpretations/Q3_111_ED.png} \\
\textbf{UE (n=1)} & \textbf{U (n=1)} & \textbf{E (n=1)} \\
\includegraphics{analysis/SGC3A/static/interpretations/Q3_111_UE.png} &
\includegraphics{analysis/SGC3A/static/interpretations/Q3_111_U.png} &
\includegraphics{analysis/SGC3A/static/interpretations/Q3_111_E.png} \\
\bottomrule()
\end{longtable}

\hypertarget{q3.-impasse-condition}{%
\paragraph{Q3. Impasse Condition}\label{q3.-impasse-condition}}

\begin{figure}

{\centering \includegraphics{analysis/SGC3A/static/questions/Q3_121.png}

}

\caption{\label{fig-Q3-121}Q3---Impasse Condition}

\end{figure}

\begin{Shaded}
\begin{Highlighting}[]
\NormalTok{q }\OtherTok{\textless{}{-}}\NormalTok{ keys\_raw }\SpecialCharTok{\%\textgreater{}\%} \FunctionTok{filter}\NormalTok{(condition }\SpecialCharTok{==} \DecValTok{121}\NormalTok{) }\SpecialCharTok{\%\textgreater{}\%} \FunctionTok{filter}\NormalTok{(Q}\SpecialCharTok{==}\DecValTok{3}\NormalTok{)}
\NormalTok{ignore }\OtherTok{\textless{}{-}}\NormalTok{ q }\SpecialCharTok{\%\textgreater{}\%} \FunctionTok{select}\NormalTok{(}\StringTok{"REF\_POINT"}\NormalTok{)}
\NormalTok{answers }\OtherTok{\textless{}{-}}\NormalTok{ q }\SpecialCharTok{\%\textgreater{}\%} \FunctionTok{select}\NormalTok{(}\StringTok{"TRIANGULAR"}\NormalTok{, }\StringTok{"ORTHOGONAL"}\NormalTok{, }\StringTok{"SATISFICE\_left"}\NormalTok{, }\StringTok{"SATISFICE\_right"}\NormalTok{,}\StringTok{"TV\_max"}\NormalTok{,}\StringTok{"TV\_start"}\NormalTok{, }\StringTok{"TV\_end"}\NormalTok{, }\StringTok{"TV\_dur"}\NormalTok{) }\SpecialCharTok{\%\textgreater{}\%} \FunctionTok{unlist}\NormalTok{()}
\NormalTok{ves }\OtherTok{\textless{}{-}}\NormalTok{ q }\SpecialCharTok{\%\textgreater{}\%} \FunctionTok{mutate}\NormalTok{(}
  \AttributeTok{SATISFICE\_left\_allow =} \StringTok{""}\NormalTok{,}
  \AttributeTok{SATISFICE\_right\_allow =} \StringTok{""}
\NormalTok{) }\SpecialCharTok{\%\textgreater{}\%} \FunctionTok{select}\NormalTok{(}\StringTok{"TRI\_allow"}\NormalTok{, }\StringTok{"ORTH\_allow"}\NormalTok{, }\StringTok{"SATISFICE\_left\_allow"}\NormalTok{,}\StringTok{"SATISFICE\_right\_allow"}\NormalTok{, }\StringTok{"TV\_max\_allow"}\NormalTok{,}\StringTok{"TV\_start\_allow"}\NormalTok{,}\StringTok{"TV\_end\_allow"}\NormalTok{, }\StringTok{"TV\_dur\_allow"}\NormalTok{)}\SpecialCharTok{\%\textgreater{}\%} \FunctionTok{unlist}\NormalTok{() }
\NormalTok{options }\OtherTok{\textless{}{-}}\NormalTok{ q }\SpecialCharTok{\%\textgreater{}\%} \FunctionTok{select}\NormalTok{(}\StringTok{"OPTIONS"}\NormalTok{)}
\NormalTok{question }\OtherTok{=}\NormalTok{ q }\SpecialCharTok{\%\textgreater{}\%}  \FunctionTok{select}\NormalTok{(}\StringTok{"TEXT"}\NormalTok{)}
\NormalTok{scores }\OtherTok{\textless{}{-}} \FunctionTok{c}\NormalTok{(}\StringTok{"Triangular"}\NormalTok{, }\StringTok{"Orthgonal"}\NormalTok{, }\StringTok{"Satisficing [left]"}\NormalTok{, }\StringTok{"Satisficing [right]"}\NormalTok{, }\StringTok{"Tversky [maximal]"}\NormalTok{, }\StringTok{"Tversky [start diagonal]"}\NormalTok{, }
            \StringTok{"Tversky [end diagonal]"}\NormalTok{, }\StringTok{"Tversky [duration line]"}\NormalTok{)}
\NormalTok{d }\OtherTok{=} \FunctionTok{tibble}\NormalTok{(}\AttributeTok{interpretation =}\NormalTok{ scores, }\AttributeTok{answer =}\NormalTok{ answers, }\AttributeTok{allowed=}\NormalTok{ves)}
\NormalTok{d}\SpecialCharTok{$}\NormalTok{answer }\OtherTok{\textless{}{-}} \FunctionTok{replace\_na}\NormalTok{(d}\SpecialCharTok{$}\NormalTok{answer, }\StringTok{""}\NormalTok{)}
\NormalTok{d}\SpecialCharTok{$}\NormalTok{allowed }\OtherTok{\textless{}{-}} \FunctionTok{replace\_na}\NormalTok{(d}\SpecialCharTok{$}\NormalTok{allowed, }\StringTok{""}\NormalTok{)}

\NormalTok{title }\OtherTok{=} \FunctionTok{paste}\NormalTok{(}\StringTok{"Answer Key | Q3 Impasse Condition : "}\NormalTok{, question)}
\NormalTok{cols }\OtherTok{=} \FunctionTok{c}\NormalTok{(}\StringTok{"interpretation"}\NormalTok{, }\StringTok{"answer"}\NormalTok{,}\StringTok{"not penalized"}\NormalTok{)}

\NormalTok{d }\SpecialCharTok{\%\textgreater{}\%} \FunctionTok{kbl}\NormalTok{(}\AttributeTok{caption =}\NormalTok{ title, }\AttributeTok{col.names =}\NormalTok{ cols) }\SpecialCharTok{\%\textgreater{}\%} \FunctionTok{kable\_classic}\NormalTok{() }\SpecialCharTok{\%\textgreater{}\%} 
  \FunctionTok{footnote}\NormalTok{(}\AttributeTok{general =} \FunctionTok{paste}\NormalTok{(}\StringTok{"15 response options: "}\NormalTok{, options), }\AttributeTok{general\_title =} \StringTok{"Note: "}\NormalTok{,}\AttributeTok{footnote\_as\_chunk =}\NormalTok{ T) }
\end{Highlighting}
\end{Shaded}

\begin{table}

\caption{Answer Key | Q3 Impasse Condition :  Which shift(s) begin when C ends?}
\centering
\begin{tabular}[t]{l|l|l}
\hline
interpretation & answer & not penalized\\
\hline
Triangular & F & \\
\hline
Orthgonal &  & \\
\hline
Satisficing [left] & AI & \\
\hline
Satisficing [right] & F & \\
\hline
Tversky [maximal] & BJ & \\
\hline
Tversky [start diagonal] & J & \\
\hline
Tversky [end diagonal] &  & \\
\hline
Tversky [duration line] & B & \\
\hline
\multicolumn{3}{l}{\rule{0pt}{1em}\textit{Note: } 15 response options:  AIKGXJDBCHUZOFE}\\
\end{tabular}
\end{table}

TODO investigate these responses 17 at O?

\begin{Shaded}
\begin{Highlighting}[]
\NormalTok{title }\OtherTok{\textless{}{-}} \StringTok{"Frequency of Selected Response Options for Question \#3 (Impasse Condition)"}
\NormalTok{names }\OtherTok{=} \FunctionTok{c}\NormalTok{(}\StringTok{"response"}\NormalTok{,}\StringTok{"n"}\NormalTok{,}\StringTok{"interpretation"}\NormalTok{,}\StringTok{"absolute"}\NormalTok{,}\StringTok{"tri"}\NormalTok{,}\StringTok{"tversky"}\NormalTok{,}\StringTok{"satisfice"}\NormalTok{,}\StringTok{"orthogonal"}\NormalTok{, }\StringTok{"scaled score"}\NormalTok{)}

\NormalTok{df\_items }\SpecialCharTok{\%\textgreater{}\%} \FunctionTok{filter}\NormalTok{(q }\SpecialCharTok{==} \DecValTok{3} \SpecialCharTok{\&}\NormalTok{ condition }\SpecialCharTok{==} \DecValTok{121}\NormalTok{) }\SpecialCharTok{\%\textgreater{}\%} \FunctionTok{group\_by}\NormalTok{(response) }\SpecialCharTok{\%\textgreater{}\%} 
\NormalTok{  dplyr}\SpecialCharTok{::}\FunctionTok{summarise}\NormalTok{( }\AttributeTok{count =} \FunctionTok{n}\NormalTok{(), }
                    \AttributeTok{nice =} \FunctionTok{unique}\NormalTok{(score\_niceABS),}
                    \AttributeTok{triangular =} \FunctionTok{unique}\NormalTok{(score\_TRI), }
                    \AttributeTok{orthogonal =}  \FunctionTok{unique}\NormalTok{(score\_ORTH),}
                    \AttributeTok{satisficing =}  \FunctionTok{unique}\NormalTok{(score\_SATISFICE),}
                    \AttributeTok{tversky =} \FunctionTok{unique}\NormalTok{(score\_TVERSKY),}
                    \AttributeTok{interpretation =} \FunctionTok{unique}\NormalTok{(int2),}
                    \AttributeTok{scaled =} \FunctionTok{unique}\NormalTok{(score\_SCALED)) }\SpecialCharTok{\%\textgreater{}\%} 
  \FunctionTok{arrange}\NormalTok{(interpretation, }\FunctionTok{desc}\NormalTok{(count)) }\SpecialCharTok{\%\textgreater{}\%} 
  \FunctionTok{select}\NormalTok{(response, count, interpretation, nice, }
\NormalTok{         triangular, tversky, satisficing, orthogonal, scaled) }\SpecialCharTok{\%\textgreater{}\%} 
  \FunctionTok{kbl}\NormalTok{(}\AttributeTok{caption =}\NormalTok{ title, }\AttributeTok{col.names =}\NormalTok{ names) }\SpecialCharTok{\%\textgreater{}\%}  \FunctionTok{kable\_classic}\NormalTok{() }\SpecialCharTok{\%\textgreater{}\%} 
  \FunctionTok{add\_header\_above}\NormalTok{(}\FunctionTok{c}\NormalTok{(}\StringTok{" "} \OtherTok{=} \DecValTok{3}\NormalTok{, }\StringTok{"Strict Score"} \OtherTok{=} \DecValTok{1}\NormalTok{, }\StringTok{"Interpretation Scores"}\OtherTok{=}\DecValTok{4}\NormalTok{, }\StringTok{"Discriminant"}\OtherTok{=}\DecValTok{1}\NormalTok{)) }\SpecialCharTok{\%\textgreater{}\%}
  \FunctionTok{pack\_rows}\NormalTok{(}\StringTok{"Triangular"}\NormalTok{, }\DecValTok{1}\NormalTok{, }\DecValTok{5}\NormalTok{) }\SpecialCharTok{\%\textgreater{}\%} 
  \FunctionTok{pack\_rows}\NormalTok{(}\StringTok{"Lines{-}Connect"}\NormalTok{, }\DecValTok{3}\NormalTok{, }\DecValTok{5}\NormalTok{) }\SpecialCharTok{\%\textgreater{}\%} 
  \FunctionTok{pack\_rows}\NormalTok{(}\StringTok{"Satisfice"}\NormalTok{, }\DecValTok{6}\NormalTok{, }\DecValTok{15}\NormalTok{) }\SpecialCharTok{\%\textgreater{}\%} 
  \FunctionTok{pack\_rows}\NormalTok{(}\StringTok{"Other"}\NormalTok{, }\DecValTok{16}\NormalTok{, }\DecValTok{21}\NormalTok{) }\SpecialCharTok{\%\textgreater{}\%} 
  \FunctionTok{pack\_rows}\NormalTok{(}\StringTok{"Unknown"}\NormalTok{, }\DecValTok{22}\NormalTok{, }\DecValTok{29}\NormalTok{) }
\end{Highlighting}
\end{Shaded}

\textbackslash begin\{table\}

\textbackslash caption\{\label{tab:Q3-IMPASSE-RESPONSES}Frequency of
Selected Response Options for Question \#3 (Impasse Condition)\}
\centering

\begin{tabular}[t]{l|r|l|r|r|r|r|r|r}
\hline
\multicolumn{3}{c|}{ } & \multicolumn{1}{c|}{Strict Score} & \multicolumn{4}{c|}{Interpretation Scores} & \multicolumn{1}{c}{Discriminant} \\
\cline{4-4} \cline{5-8} \cline{9-9}
response & n & interpretation & absolute & tri & tversky & satisfice & orthogonal & scaled score\\
\hline
\multicolumn{9}{l}{\textbf{Triangular}}\\
\hline
\hspace{1em}F & 61 & Triangular & 1 & 1.000 & -0.077 & 1.000 & NA & 1.0\\
\hline
\hspace{1em}AF & 5 & Triangular & 0 & 0.923 & -0.154 & 0.923 & NA & 1.0\\
\hline
\multicolumn{9}{l}{\textbf{Lines-Connect}}\\
\hline
\hspace{1em}\hspace{1em}AFG & 1 & Triangular & 0 & 0.846 & -0.231 & 0.846 & NA & 1.0\\
\hline
\hspace{1em}\hspace{1em}B & 8 & Tversky & 0 & -0.077 & 1.000 & -0.077 & NA & 0.5\\
\hline
\hspace{1em}\hspace{1em}J & 3 & Tversky & 0 & -0.077 & 1.000 & -0.077 & NA & 0.5\\
\hline
\multicolumn{9}{l}{\textbf{Satisfice}}\\
\hline
\hspace{1em}BE & 1 & Tversky & 0 & -0.154 & 0.923 & -0.154 & NA & 0.5\\
\hline
\hspace{1em}BJ & 1 & Tversky & 0 & -0.154 & 1.000 & -0.154 & NA & 0.5\\
\hline
\hspace{1em}HJZ & 1 & Tversky & 0 & -0.231 & 0.846 & -0.231 & NA & 0.5\\
\hline
\hspace{1em}A & 7 & Satisfice & 0 & -0.077 & -0.077 & 0.500 & NA & -1.0\\
\hline
\hspace{1em}AH & 5 & Satisfice & 0 & -0.154 & -0.154 & 0.417 & NA & -1.0\\
\hline
\hspace{1em}AI & 3 & Satisfice & 0 & -0.154 & -0.154 & 1.000 & NA & -1.0\\
\hline
\hspace{1em}AOU & 3 & Satisfice & 0 & -0.231 & -0.231 & 0.333 & NA & -1.0\\
\hline
\hspace{1em}AFI & 2 & Satisfice & 0 & 0.846 & -0.231 & 0.917 & NA & -1.0\\
\hline
\hspace{1em}AIO & 2 & Satisfice & 0 & -0.231 & -0.231 & 0.917 & NA & -1.0\\
\hline
\hspace{1em}AO & 2 & Satisfice & 0 & -0.154 & -0.154 & 0.417 & NA & -1.0\\
\hline
\multicolumn{9}{l}{\textbf{Other}}\\
\hline
\hspace{1em}C & 2 & reference & 0 & 0.000 & 0.000 & NA & NA & 0.0\\
\hline
\hspace{1em} & 36 & blank & 0 & 0.000 & 0.000 & NA & NA & 0.0\\
\hline
\hspace{1em}ABDEFGHIJKOUXZ & 1 & frenzy & 0 & 0.000 & 0.000 & NA & NA & -0.5\\
\hline
\hspace{1em}ABDEFGHJKUZ & 1 & frenzy & 0 & 0.231 & 0.250 & 0.231 & NA & -0.5\\
\hline
\hspace{1em}BDEFGHJKU & 1 & frenzy & 0 & 0.385 & 0.417 & 0.385 & NA & -0.5\\
\hline
\hspace{1em}BDEFGHJKUXZ & 1 & frenzy & 0 & 0.231 & 0.250 & 0.231 & NA & -0.5\\
\hline
\multicolumn{9}{l}{\textbf{Unknown}}\\
\hline
\hspace{1em}O & 17 & ? & 0 & -0.077 & -0.077 & -0.077 & NA & -0.5\\
\hline
\hspace{1em}DK & 2 & ? & 0 & -0.154 & -0.154 & -0.154 & NA & -0.5\\
\hline
\hspace{1em}FJZ & 1 & ? & 0 & 0.846 & 0.846 & 0.846 & NA & -0.5\\
\hline
\hspace{1em}K & 1 & ? & 0 & -0.077 & -0.077 & -0.077 & NA & -0.5\\
\hline
\hspace{1em}KO & 1 & ? & 0 & -0.154 & -0.154 & -0.154 & NA & -0.5\\
\hline
\hspace{1em}U & 1 & ? & 0 & -0.077 & -0.077 & -0.077 & NA & -0.5\\
\hline
\hspace{1em}X & 1 & ? & 0 & -0.077 & -0.077 & -0.077 & NA & -0.5\\
\hline
\hspace{1em}Z & 1 & ? & 0 & -0.077 & -0.077 & -0.077 & NA & -0.5\\
\hline
\end{tabular}

\textbackslash end\{table\}

\begin{Shaded}
\begin{Highlighting}[]
\FunctionTok{gf\_dhistogram}\NormalTok{(}\SpecialCharTok{\textasciitilde{}}\NormalTok{ score\_niceABS, }\AttributeTok{fill =} \SpecialCharTok{\textasciitilde{}}\NormalTok{condition, }\AttributeTok{data =}\NormalTok{ df\_items }\SpecialCharTok{\%\textgreater{}\%} \FunctionTok{filter}\NormalTok{(q }\SpecialCharTok{==}\DecValTok{3}\NormalTok{)) }\SpecialCharTok{\%\textgreater{}\%} 
  \FunctionTok{gf\_facet\_grid}\NormalTok{( condition }\SpecialCharTok{\textasciitilde{}}\NormalTok{ ., }\AttributeTok{labeller =}\NormalTok{ label\_both) }\SpecialCharTok{+} 
  \FunctionTok{labs}\NormalTok{( }\AttributeTok{x =} \StringTok{"Scaled Item Score"}\NormalTok{, }\AttributeTok{title =} \StringTok{"Distribution of Scaled Scores | Q3 "}\NormalTok{) }\SpecialCharTok{+} 
  \FunctionTok{theme\_minimal}\NormalTok{() }\SpecialCharTok{+} \FunctionTok{theme}\NormalTok{(}\AttributeTok{legend.position =} \StringTok{"blank"}\NormalTok{)}
\end{Highlighting}
\end{Shaded}

\begin{figure}[H]

{\centering \includegraphics{analysis/SGC3A/2_sgc3A_scoring_files/figure-pdf/Q3-distribution-1.pdf}

}

\end{figure}

\begin{Shaded}
\begin{Highlighting}[]
\FunctionTok{gf\_props}\NormalTok{(}\SpecialCharTok{\textasciitilde{}}\NormalTok{interpretation, }\AttributeTok{fill =} \SpecialCharTok{\textasciitilde{}}\NormalTok{condition, }\AttributeTok{data =}\NormalTok{ df\_items }\SpecialCharTok{\%\textgreater{}\%} \FunctionTok{filter}\NormalTok{(q }\SpecialCharTok{==}\DecValTok{3}\NormalTok{)) }\SpecialCharTok{\%\textgreater{}\%} 
  \FunctionTok{gf\_facet\_grid}\NormalTok{( condition }\SpecialCharTok{\textasciitilde{}}\NormalTok{ ., }\AttributeTok{labeller =}\NormalTok{ label\_both) }\SpecialCharTok{+} 
  \FunctionTok{labs}\NormalTok{( }\AttributeTok{x =} \StringTok{"Interpretation"}\NormalTok{, }\AttributeTok{title =} \StringTok{"Distribution of Interpretations | Q3 "}\NormalTok{) }\SpecialCharTok{+} 
  \FunctionTok{theme\_minimal}\NormalTok{() }\SpecialCharTok{+} \FunctionTok{theme}\NormalTok{(}\AttributeTok{legend.position =} \StringTok{"blank"}\NormalTok{)}
\end{Highlighting}
\end{Shaded}

\begin{figure}[H]

{\centering \includegraphics{analysis/SGC3A/2_sgc3A_scoring_files/figure-pdf/Q3-distribution-2.pdf}

}

\end{figure}

\hypertarget{question-4}{%
\subsubsection{Question \#4}\label{question-4}}

{[}PLACEHOLDER --- NOT YET CONSIDERED THIS QUESTION{]}

\hypertarget{q4.-control-condition}{%
\subparagraph{Q4. Control Condition}\label{q4.-control-condition}}

\begin{figure}

{\centering \includegraphics{analysis/SGC3A/static/questions/Q4_111.png}

}

\caption{\label{fig-Q4-111}Q4---Control Condition}

\end{figure}

\begin{Shaded}
\begin{Highlighting}[]
\NormalTok{q }\OtherTok{\textless{}{-}}\NormalTok{ keys\_raw }\SpecialCharTok{\%\textgreater{}\%} \FunctionTok{filter}\NormalTok{(condition }\SpecialCharTok{==} \StringTok{"DEFAULT"}\NormalTok{) }\SpecialCharTok{\%\textgreater{}\%} \FunctionTok{filter}\NormalTok{(Q}\SpecialCharTok{==}\DecValTok{4}\NormalTok{)}
\NormalTok{ignore }\OtherTok{\textless{}{-}}\NormalTok{ q }\SpecialCharTok{\%\textgreater{}\%} \FunctionTok{select}\NormalTok{(}\StringTok{"REF\_POINT"}\NormalTok{)}
\NormalTok{answers }\OtherTok{\textless{}{-}}\NormalTok{ q }\SpecialCharTok{\%\textgreater{}\%} \FunctionTok{select}\NormalTok{(}\StringTok{"TRIANGULAR"}\NormalTok{, }\StringTok{"ORTHOGONAL"}\NormalTok{, }\StringTok{"SATISFICE\_left"}\NormalTok{, }\StringTok{"SATISFICE\_right"}\NormalTok{,}\StringTok{"TV\_max"}\NormalTok{,}\StringTok{"TV\_start"}\NormalTok{, }\StringTok{"TV\_end"}\NormalTok{, }\StringTok{"TV\_dur"}\NormalTok{) }\SpecialCharTok{\%\textgreater{}\%} \FunctionTok{unlist}\NormalTok{()}
\NormalTok{ves }\OtherTok{\textless{}{-}}\NormalTok{ q }\SpecialCharTok{\%\textgreater{}\%} \FunctionTok{mutate}\NormalTok{(}
  \AttributeTok{SATISFICE\_left\_allow =} \StringTok{""}\NormalTok{,}
  \AttributeTok{SATISFICE\_right\_allow =} \StringTok{""}
\NormalTok{) }\SpecialCharTok{\%\textgreater{}\%} \FunctionTok{select}\NormalTok{(}\StringTok{"TRI\_allow"}\NormalTok{, }\StringTok{"ORTH\_allow"}\NormalTok{, }\StringTok{"SATISFICE\_left\_allow"}\NormalTok{,}\StringTok{"SATISFICE\_right\_allow"}\NormalTok{, }\StringTok{"TV\_max\_allow"}\NormalTok{,}\StringTok{"TV\_start\_allow"}\NormalTok{,}\StringTok{"TV\_end\_allow"}\NormalTok{, }\StringTok{"TV\_dur\_allow"}\NormalTok{)}\SpecialCharTok{\%\textgreater{}\%} \FunctionTok{unlist}\NormalTok{() }
\NormalTok{options }\OtherTok{\textless{}{-}}\NormalTok{ q }\SpecialCharTok{\%\textgreater{}\%} \FunctionTok{select}\NormalTok{(}\StringTok{"OPTIONS"}\NormalTok{)}
\NormalTok{question }\OtherTok{=}\NormalTok{ q }\SpecialCharTok{\%\textgreater{}\%}  \FunctionTok{select}\NormalTok{(}\StringTok{"TEXT"}\NormalTok{)}
\NormalTok{scores }\OtherTok{\textless{}{-}} \FunctionTok{c}\NormalTok{(}\StringTok{"Triangular"}\NormalTok{, }\StringTok{"Orthgonal"}\NormalTok{, }\StringTok{"Satisficing [left]"}\NormalTok{, }\StringTok{"Satisficing [right]"}\NormalTok{, }\StringTok{"Tversky [maximal]"}\NormalTok{, }\StringTok{"Tversky [start diagonal]"}\NormalTok{, }
            \StringTok{"Tversky [end diagonal]"}\NormalTok{, }\StringTok{"Tversky [duration line]"}\NormalTok{)}
\NormalTok{d }\OtherTok{=} \FunctionTok{tibble}\NormalTok{(}\AttributeTok{interpretation =}\NormalTok{ scores, }\AttributeTok{answer =}\NormalTok{ answers, }\AttributeTok{allowed=}\NormalTok{ves)}
\NormalTok{d}\SpecialCharTok{$}\NormalTok{answer }\OtherTok{\textless{}{-}} \FunctionTok{replace\_na}\NormalTok{(d}\SpecialCharTok{$}\NormalTok{answer, }\StringTok{""}\NormalTok{)}
\NormalTok{d}\SpecialCharTok{$}\NormalTok{allowed }\OtherTok{\textless{}{-}} \FunctionTok{replace\_na}\NormalTok{(d}\SpecialCharTok{$}\NormalTok{allowed, }\StringTok{""}\NormalTok{)}

\NormalTok{title }\OtherTok{=} \FunctionTok{paste}\NormalTok{(}\StringTok{"Answer Key | Q4 Control Condition : "}\NormalTok{, question)}
\NormalTok{cols }\OtherTok{=} \FunctionTok{c}\NormalTok{(}\StringTok{"interpretation"}\NormalTok{, }\StringTok{"answer"}\NormalTok{,}\StringTok{"not penalized"}\NormalTok{)}

\NormalTok{d }\SpecialCharTok{\%\textgreater{}\%} \FunctionTok{kbl}\NormalTok{(}\AttributeTok{caption =}\NormalTok{ title, }\AttributeTok{col.names =}\NormalTok{ cols) }\SpecialCharTok{\%\textgreater{}\%} \FunctionTok{kable\_classic}\NormalTok{() }\SpecialCharTok{\%\textgreater{}\%} 
  \FunctionTok{footnote}\NormalTok{(}\AttributeTok{general =} \FunctionTok{paste}\NormalTok{(}\StringTok{"15 response options: "}\NormalTok{, options), }\AttributeTok{general\_title =} \StringTok{"Note: "}\NormalTok{,}\AttributeTok{footnote\_as\_chunk =}\NormalTok{ T) }
\end{Highlighting}
\end{Shaded}

\begin{table}

\caption{Answer Key | Q4 Control Condition :  Which shift(s) end at 4 pm?}
\centering
\begin{tabular}[t]{l|l|l}
\hline
interpretation & answer & not penalized\\
\hline
Triangular & H & \\
\hline
Orthgonal & U & OF\\
\hline
Satisficing [left] &  & \\
\hline
Satisficing [right] &  & \\
\hline
Tversky [maximal] & BH & \\
\hline
Tversky [start diagonal] & B & \\
\hline
Tversky [end diagonal] & H & \\
\hline
Tversky [duration line] &  & \\
\hline
\multicolumn{3}{l}{\rule{0pt}{1em}\textit{Note: } 15 response options:  AIKGXJDBCHUZOFE}\\
\end{tabular}
\end{table}

\begin{Shaded}
\begin{Highlighting}[]
\NormalTok{title }\OtherTok{\textless{}{-}} \StringTok{"Frequency of Selected Response Options for Question \#4 (Control Condition)"}
\NormalTok{names }\OtherTok{=} \FunctionTok{c}\NormalTok{(}\StringTok{"response"}\NormalTok{,}\StringTok{"n"}\NormalTok{,}\StringTok{"interpretation"}\NormalTok{,}\StringTok{"absolute"}\NormalTok{,}\StringTok{"tri"}\NormalTok{,}\StringTok{"tversky"}\NormalTok{,}\StringTok{"satisfice"}\NormalTok{,}\StringTok{"orthogonal"}\NormalTok{, }\StringTok{"scaled score"}\NormalTok{)}

\NormalTok{df\_items }\SpecialCharTok{\%\textgreater{}\%} \FunctionTok{filter}\NormalTok{(q }\SpecialCharTok{==} \DecValTok{4} \SpecialCharTok{\&}\NormalTok{ condition }\SpecialCharTok{==} \DecValTok{111}\NormalTok{) }\SpecialCharTok{\%\textgreater{}\%} \FunctionTok{group\_by}\NormalTok{(response) }\SpecialCharTok{\%\textgreater{}\%} 
\NormalTok{  dplyr}\SpecialCharTok{::}\FunctionTok{summarise}\NormalTok{( }\AttributeTok{count =} \FunctionTok{n}\NormalTok{(), }
                    \AttributeTok{nice =} \FunctionTok{unique}\NormalTok{(score\_niceABS),}
                    \AttributeTok{triangular =} \FunctionTok{unique}\NormalTok{(score\_TRI), }
                    \AttributeTok{orthogonal =}  \FunctionTok{unique}\NormalTok{(score\_ORTH),}
                    \AttributeTok{satisficing =}  \FunctionTok{unique}\NormalTok{(score\_SATISFICE),}
                    \AttributeTok{tversky =} \FunctionTok{unique}\NormalTok{(score\_TVERSKY),}
                    \AttributeTok{interpretation =} \FunctionTok{unique}\NormalTok{(int2),}
                    \AttributeTok{scaled =} \FunctionTok{unique}\NormalTok{(score\_SCALED)) }\SpecialCharTok{\%\textgreater{}\%} 
  \FunctionTok{arrange}\NormalTok{(interpretation, }\FunctionTok{desc}\NormalTok{(count)) }\SpecialCharTok{\%\textgreater{}\%} 
  \FunctionTok{select}\NormalTok{(response, count, interpretation, nice, }
\NormalTok{         triangular, tversky, satisficing, orthogonal, scaled) }\SpecialCharTok{\%\textgreater{}\%} 
  \FunctionTok{kbl}\NormalTok{(}\AttributeTok{caption =}\NormalTok{ title, }\AttributeTok{col.names =}\NormalTok{ names) }\SpecialCharTok{\%\textgreater{}\%}  \FunctionTok{kable\_classic}\NormalTok{() }\SpecialCharTok{\%\textgreater{}\%} 
  \FunctionTok{add\_header\_above}\NormalTok{(}\FunctionTok{c}\NormalTok{(}\StringTok{" "} \OtherTok{=} \DecValTok{3}\NormalTok{, }\StringTok{"Strict Score"} \OtherTok{=} \DecValTok{1}\NormalTok{, }\StringTok{"Interpretation Scores"}\OtherTok{=}\DecValTok{4}\NormalTok{, }\StringTok{"Discriminant"}\OtherTok{=}\DecValTok{1}\NormalTok{)) }\SpecialCharTok{\%\textgreater{}\%}
  \FunctionTok{pack\_rows}\NormalTok{(}\StringTok{"Triangular"}\NormalTok{, }\DecValTok{1}\NormalTok{, }\DecValTok{2}\NormalTok{) }\SpecialCharTok{\%\textgreater{}\%} 
  \FunctionTok{pack\_rows}\NormalTok{(}\StringTok{"Lines{-}Connect"}\NormalTok{, }\DecValTok{3}\NormalTok{, }\DecValTok{3}\NormalTok{) }\SpecialCharTok{\%\textgreater{}\%} 
  \FunctionTok{pack\_rows}\NormalTok{(}\StringTok{"Orthogonal"}\NormalTok{, }\DecValTok{4}\NormalTok{, }\DecValTok{8}\NormalTok{) }\SpecialCharTok{\%\textgreater{}\%} 
  \FunctionTok{pack\_rows}\NormalTok{(}\StringTok{"Other"}\NormalTok{, }\DecValTok{9}\NormalTok{, }\DecValTok{10}\NormalTok{) }\SpecialCharTok{\%\textgreater{}\%} 
  \FunctionTok{pack\_rows}\NormalTok{(}\StringTok{"Unknown"}\NormalTok{, }\DecValTok{11}\NormalTok{, }\DecValTok{16}\NormalTok{) }
\end{Highlighting}
\end{Shaded}

\textbackslash begin\{table\}

\textbackslash caption\{\label{tab:Q4-CONTROL-RESPONSES}Frequency of
Selected Response Options for Question \#4 (Control Condition)\}
\centering

\begin{tabular}[t]{l|r|l|r|r|r|r|r|r}
\hline
\multicolumn{3}{c|}{ } & \multicolumn{1}{c|}{Strict Score} & \multicolumn{4}{c|}{Interpretation Scores} & \multicolumn{1}{c}{Discriminant} \\
\cline{4-4} \cline{5-8} \cline{9-9}
response & n & interpretation & absolute & tri & tversky & satisfice & orthogonal & scaled score\\
\hline
\multicolumn{9}{l}{\textbf{Triangular}}\\
\hline
\hspace{1em}H & 29 & Triangular & 1 & 1.000 & 1.000 & NA & -0.083 & 1.0\\
\hline
\hspace{1em}AH & 1 & Triangular & 0 & 0.929 & 0.929 & NA & -0.167 & 1.0\\
\hline
\multicolumn{9}{l}{\textbf{Lines-Connect}}\\
\hline
\hspace{1em}B & 3 & Tversky & 0 & -0.071 & 1.000 & NA & -0.083 & 0.5\\
\hline
\multicolumn{9}{l}{\textbf{Orthogonal}}\\
\hline
\hspace{1em}U & 87 & Orthogonal & 0 & -0.071 & -0.071 & NA & 1.000 & -1.0\\
\hline
\hspace{1em}FU & 2 & Orthogonal & 0 & -0.143 & -0.143 & NA & 1.000 & -1.0\\
\hline
\hspace{1em}DEOU & 1 & Orthogonal & 0 & -0.286 & -0.286 & NA & 0.833 & -1.0\\
\hline
\hspace{1em}DEU & 1 & Orthogonal & 0 & -0.214 & -0.214 & NA & 0.833 & -1.0\\
\hline
\hspace{1em}KU & 1 & Orthogonal & 0 & -0.143 & -0.143 & NA & 0.917 & -1.0\\
\hline
\multicolumn{9}{l}{\textbf{Other}}\\
\hline
\hspace{1em} & 6 & blank & 0 & 0.000 & 0.000 & NA & 0.000 & 0.0\\
\hline
\hspace{1em}ACFHIJKOUXZ & 1 & frenzy & 0 & 0.286 & 0.286 & NA & 0.333 & -0.5\\
\hline
\multicolumn{9}{l}{\textbf{Unknown}}\\
\hline
\hspace{1em}DE & 14 & ? & 0 & -0.143 & -0.143 & NA & -0.167 & -0.5\\
\hline
\hspace{1em}E & 6 & ? & 0 & -0.071 & -0.071 & NA & -0.083 & -0.5\\
\hline
\hspace{1em}K & 2 & ? & 0 & -0.071 & -0.071 & NA & -0.083 & -0.5\\
\hline
\hspace{1em}O & 2 & ? & 0 & -0.071 & -0.071 & NA & 0.000 & -0.5\\
\hline
\hspace{1em}D & 1 & ? & 0 & -0.071 & -0.071 & NA & -0.083 & -0.5\\
\hline
\hspace{1em}G & 1 & ? & 0 & -0.071 & -0.071 & NA & -0.083 & -0.5\\
\hline
\end{tabular}

\textbackslash end\{table\}

\begin{longtable}[]{@{}
  >{\raggedright\arraybackslash}p{(\columnwidth - 2\tabcolsep) * \real{0.5000}}
  >{\raggedright\arraybackslash}p{(\columnwidth - 2\tabcolsep) * \real{0.5000}}@{}}
\caption{TBL4 test}\tabularnewline
\toprule()
\begin{minipage}[b]{\linewidth}\raggedright
Orthogonal
\end{minipage} & \begin{minipage}[b]{\linewidth}\raggedright
Orthogonal-LinesConnecting
\end{minipage} \\
\midrule()
\endfirsthead
\toprule()
\begin{minipage}[b]{\linewidth}\raggedright
Orthogonal
\end{minipage} & \begin{minipage}[b]{\linewidth}\raggedright
Orthogonal-LinesConnecting
\end{minipage} \\
\midrule()
\endhead
\includegraphics{analysis/SGC3A/static/interpretations/Q4_111_ORTH.png}
\textbar{} & \\
If the subject calculates end time for each data point (using duration
on the y axis), they find that an (incorrect) projection of point U `end
time' intersects with the (incorrect) orthogonal projection of 4:00PM. &
Alternatively, some subjects selected points E and D which intersect
with an orthogonal projection from 4:00pm. We call this an
'orthogonal-lines connect'' strategy, because it (incorrectly) adapts
the orthogonal procedure for finding events that \emph{start} at 4:00pm
in order to find those that \emph{end} at 4:00pm, thus selecting any
data point with an orthogonal intersection with 4:00pm. \\
\bottomrule()
\end{longtable}

\hypertarget{q4.-impasse-condition}{%
\subparagraph{Q4. Impasse Condition}\label{q4.-impasse-condition}}

\begin{figure}

{\centering \includegraphics{analysis/SGC3A/static/questions/Q4_121.png}

}

\caption{\label{fig-Q4-121}Q4---Impasse Condition}

\end{figure}

\begin{Shaded}
\begin{Highlighting}[]
\NormalTok{q }\OtherTok{\textless{}{-}}\NormalTok{ keys\_raw }\SpecialCharTok{\%\textgreater{}\%} \FunctionTok{filter}\NormalTok{(condition }\SpecialCharTok{==} \DecValTok{121}\NormalTok{) }\SpecialCharTok{\%\textgreater{}\%} \FunctionTok{filter}\NormalTok{(Q}\SpecialCharTok{==}\DecValTok{4}\NormalTok{)}
\NormalTok{ignore }\OtherTok{\textless{}{-}}\NormalTok{ q }\SpecialCharTok{\%\textgreater{}\%} \FunctionTok{select}\NormalTok{(}\StringTok{"REF\_POINT"}\NormalTok{)}
\NormalTok{answers }\OtherTok{\textless{}{-}}\NormalTok{ q }\SpecialCharTok{\%\textgreater{}\%} \FunctionTok{select}\NormalTok{(}\StringTok{"TRIANGULAR"}\NormalTok{, }\StringTok{"ORTHOGONAL"}\NormalTok{, }\StringTok{"SATISFICE\_left"}\NormalTok{, }\StringTok{"SATISFICE\_right"}\NormalTok{,}\StringTok{"TV\_max"}\NormalTok{,}\StringTok{"TV\_start"}\NormalTok{, }\StringTok{"TV\_end"}\NormalTok{, }\StringTok{"TV\_dur"}\NormalTok{) }\SpecialCharTok{\%\textgreater{}\%} \FunctionTok{unlist}\NormalTok{()}
\NormalTok{ves }\OtherTok{\textless{}{-}}\NormalTok{ q }\SpecialCharTok{\%\textgreater{}\%} \FunctionTok{mutate}\NormalTok{(}
  \AttributeTok{SATISFICE\_left\_allow =} \StringTok{""}\NormalTok{,}
  \AttributeTok{SATISFICE\_right\_allow =} \StringTok{""}
\NormalTok{) }\SpecialCharTok{\%\textgreater{}\%} \FunctionTok{select}\NormalTok{(}\StringTok{"TRI\_allow"}\NormalTok{, }\StringTok{"ORTH\_allow"}\NormalTok{, }\StringTok{"SATISFICE\_left\_allow"}\NormalTok{,}\StringTok{"SATISFICE\_right\_allow"}\NormalTok{, }\StringTok{"TV\_max\_allow"}\NormalTok{,}\StringTok{"TV\_start\_allow"}\NormalTok{,}\StringTok{"TV\_end\_allow"}\NormalTok{, }\StringTok{"TV\_dur\_allow"}\NormalTok{)}\SpecialCharTok{\%\textgreater{}\%} \FunctionTok{unlist}\NormalTok{() }
\NormalTok{options }\OtherTok{\textless{}{-}}\NormalTok{ q }\SpecialCharTok{\%\textgreater{}\%} \FunctionTok{select}\NormalTok{(}\StringTok{"OPTIONS"}\NormalTok{)}
\NormalTok{question }\OtherTok{=}\NormalTok{ q }\SpecialCharTok{\%\textgreater{}\%}  \FunctionTok{select}\NormalTok{(}\StringTok{"TEXT"}\NormalTok{)}
\NormalTok{scores }\OtherTok{\textless{}{-}} \FunctionTok{c}\NormalTok{(}\StringTok{"Triangular"}\NormalTok{, }\StringTok{"Orthgonal"}\NormalTok{, }\StringTok{"Satisficing [left]"}\NormalTok{, }\StringTok{"Satisficing [right]"}\NormalTok{, }\StringTok{"Tversky [maximal]"}\NormalTok{, }\StringTok{"Tversky [start diagonal]"}\NormalTok{, }
            \StringTok{"Tversky [end diagonal]"}\NormalTok{, }\StringTok{"Tversky [duration line]"}\NormalTok{)}
\NormalTok{d }\OtherTok{=} \FunctionTok{tibble}\NormalTok{(}\AttributeTok{interpretation =}\NormalTok{ scores, }\AttributeTok{answer =}\NormalTok{ answers, }\AttributeTok{allowed=}\NormalTok{ves)}
\NormalTok{d}\SpecialCharTok{$}\NormalTok{answer }\OtherTok{\textless{}{-}} \FunctionTok{replace\_na}\NormalTok{(d}\SpecialCharTok{$}\NormalTok{answer, }\StringTok{""}\NormalTok{)}
\NormalTok{d}\SpecialCharTok{$}\NormalTok{allowed }\OtherTok{\textless{}{-}} \FunctionTok{replace\_na}\NormalTok{(d}\SpecialCharTok{$}\NormalTok{allowed, }\StringTok{""}\NormalTok{)}

\NormalTok{title }\OtherTok{=} \FunctionTok{paste}\NormalTok{(}\StringTok{"Answer Key | Q4 Impasse Condition : "}\NormalTok{, question)}
\NormalTok{cols }\OtherTok{=} \FunctionTok{c}\NormalTok{(}\StringTok{"interpretation"}\NormalTok{, }\StringTok{"answer"}\NormalTok{,}\StringTok{"not penalized"}\NormalTok{)}

\NormalTok{d }\SpecialCharTok{\%\textgreater{}\%} \FunctionTok{kbl}\NormalTok{(}\AttributeTok{caption =}\NormalTok{ title, }\AttributeTok{col.names =}\NormalTok{ cols) }\SpecialCharTok{\%\textgreater{}\%} \FunctionTok{kable\_classic}\NormalTok{() }\SpecialCharTok{\%\textgreater{}\%} 
  \FunctionTok{footnote}\NormalTok{(}\AttributeTok{general =} \FunctionTok{paste}\NormalTok{(}\StringTok{"15 response options: "}\NormalTok{, options), }\AttributeTok{general\_title =} \StringTok{"Note: "}\NormalTok{,}\AttributeTok{footnote\_as\_chunk =}\NormalTok{ T) }
\end{Highlighting}
\end{Shaded}

\begin{table}

\caption{Answer Key | Q4 Impasse Condition :  Which shift(s) end at 4 pm?}
\centering
\begin{tabular}[t]{l|l|l}
\hline
interpretation & answer & not penalized\\
\hline
Triangular & H & \\
\hline
Orthgonal &  & \\
\hline
Satisficing [left] & FO & \\
\hline
Satisficing [right] &  & \\
\hline
Tversky [maximal] & BH & \\
\hline
Tversky [start diagonal] & B & \\
\hline
Tversky [end diagonal] & H & \\
\hline
Tversky [duration line] &  & \\
\hline
\multicolumn{3}{l}{\rule{0pt}{1em}\textit{Note: } 15 response options:  AIKGXJDBCHUZOFE}\\
\end{tabular}
\end{table}

TODO investigate D? add to tversky or orth?

\begin{Shaded}
\begin{Highlighting}[]
\NormalTok{title }\OtherTok{\textless{}{-}} \StringTok{"Frequency of Selected Response Options for Question \#4 (Impasse Condition)"}
\NormalTok{names }\OtherTok{=} \FunctionTok{c}\NormalTok{(}\StringTok{"response"}\NormalTok{,}\StringTok{"n"}\NormalTok{,}\StringTok{"interpretation"}\NormalTok{,}\StringTok{"absolute"}\NormalTok{,}\StringTok{"tri"}\NormalTok{,}\StringTok{"tversky"}\NormalTok{,}\StringTok{"satisfice"}\NormalTok{,}\StringTok{"orthogonal"}\NormalTok{, }\StringTok{"scaled score"}\NormalTok{)}

\NormalTok{df\_items }\SpecialCharTok{\%\textgreater{}\%} \FunctionTok{filter}\NormalTok{(q }\SpecialCharTok{==} \DecValTok{4} \SpecialCharTok{\&}\NormalTok{ condition }\SpecialCharTok{==} \DecValTok{121}\NormalTok{) }\SpecialCharTok{\%\textgreater{}\%} \FunctionTok{group\_by}\NormalTok{(response) }\SpecialCharTok{\%\textgreater{}\%} 
\NormalTok{  dplyr}\SpecialCharTok{::}\FunctionTok{summarise}\NormalTok{( }\AttributeTok{count =} \FunctionTok{n}\NormalTok{(), }
                    \AttributeTok{nice =} \FunctionTok{unique}\NormalTok{(score\_niceABS),}
                    \AttributeTok{triangular =} \FunctionTok{unique}\NormalTok{(score\_TRI), }
                    \AttributeTok{orthogonal =}  \FunctionTok{unique}\NormalTok{(score\_ORTH),}
                    \AttributeTok{satisficing =}  \FunctionTok{unique}\NormalTok{(score\_SATISFICE),}
                    \AttributeTok{tversky =} \FunctionTok{unique}\NormalTok{(score\_TVERSKY),}
                    \AttributeTok{interpretation =} \FunctionTok{unique}\NormalTok{(int2),}
                    \AttributeTok{scaled =} \FunctionTok{unique}\NormalTok{(score\_SCALED)) }\SpecialCharTok{\%\textgreater{}\%} 
  \FunctionTok{arrange}\NormalTok{(interpretation, }\FunctionTok{desc}\NormalTok{(count)) }\SpecialCharTok{\%\textgreater{}\%} 
  \FunctionTok{select}\NormalTok{(response, count, interpretation, nice, }
\NormalTok{         triangular, tversky, satisficing, orthogonal, scaled) }\SpecialCharTok{\%\textgreater{}\%} 
  \FunctionTok{kbl}\NormalTok{(}\AttributeTok{caption =}\NormalTok{ title, }\AttributeTok{col.names =}\NormalTok{ names) }\SpecialCharTok{\%\textgreater{}\%}  \FunctionTok{kable\_classic}\NormalTok{() }\SpecialCharTok{\%\textgreater{}\%} 
  \FunctionTok{add\_header\_above}\NormalTok{(}\FunctionTok{c}\NormalTok{(}\StringTok{" "} \OtherTok{=} \DecValTok{3}\NormalTok{, }\StringTok{"Strict Score"} \OtherTok{=} \DecValTok{1}\NormalTok{, }\StringTok{"Interpretation Scores"}\OtherTok{=}\DecValTok{4}\NormalTok{, }\StringTok{"Discriminant"}\OtherTok{=}\DecValTok{1}\NormalTok{)) }\SpecialCharTok{\%\textgreater{}\%}
  \FunctionTok{pack\_rows}\NormalTok{(}\StringTok{"Triangular"}\NormalTok{, }\DecValTok{1}\NormalTok{, }\DecValTok{2}\NormalTok{) }\SpecialCharTok{\%\textgreater{}\%} 
  \FunctionTok{pack\_rows}\NormalTok{(}\StringTok{"Lines{-}Connect"}\NormalTok{, }\DecValTok{3}\NormalTok{, }\DecValTok{6}\NormalTok{) }\SpecialCharTok{\%\textgreater{}\%} 
  \FunctionTok{pack\_rows}\NormalTok{(}\StringTok{"Satisfice"}\NormalTok{, }\DecValTok{7}\NormalTok{, }\DecValTok{10}\NormalTok{) }\SpecialCharTok{\%\textgreater{}\%} 
  \FunctionTok{pack\_rows}\NormalTok{(}\StringTok{"Other"}\NormalTok{, }\DecValTok{11}\NormalTok{, }\DecValTok{12}\NormalTok{) }\SpecialCharTok{\%\textgreater{}\%} 
  \FunctionTok{pack\_rows}\NormalTok{(}\StringTok{"Unknown"}\NormalTok{, }\DecValTok{13}\NormalTok{, }\DecValTok{19}\NormalTok{) }
\end{Highlighting}
\end{Shaded}

\textbackslash begin\{table\}

\textbackslash caption\{\label{tab:Q4-IMPASSE-RESPONSES}Frequency of
Selected Response Options for Question \#4 (Impasse Condition)\}
\centering

\begin{tabular}[t]{l|r|l|r|r|r|r|r|r}
\hline
\multicolumn{3}{c|}{ } & \multicolumn{1}{c|}{Strict Score} & \multicolumn{4}{c|}{Interpretation Scores} & \multicolumn{1}{c}{Discriminant} \\
\cline{4-4} \cline{5-8} \cline{9-9}
response & n & interpretation & absolute & tri & tversky & satisfice & orthogonal & scaled score\\
\hline
\multicolumn{9}{l}{\textbf{Triangular}}\\
\hline
\hspace{1em}H & 64 & Triangular & 1 & 1.000 & 1.000 & -0.077 & NA & 1.0\\
\hline
\hspace{1em}DH & 1 & Triangular & 0 & 0.929 & 0.929 & -0.154 & NA & 1.0\\
\hline
\multicolumn{9}{l}{\textbf{Lines-Connect}}\\
\hline
\hspace{1em}B & 6 & Tversky & 0 & -0.071 & 1.000 & -0.077 & NA & 0.5\\
\hline
\hspace{1em}BD & 2 & Tversky & 0 & -0.143 & 0.929 & -0.154 & NA & 0.5\\
\hline
\hspace{1em}BH & 2 & Tversky & 0 & 0.929 & 1.000 & -0.154 & NA & 0.5\\
\hline
\hspace{1em}BDEG & 1 & Tversky & 0 & -0.286 & 0.786 & -0.308 & NA & 0.5\\
\hline
\multicolumn{9}{l}{\textbf{Satisfice}}\\
\hline
\hspace{1em}O & 11 & Satisfice & 0 & -0.071 & -0.071 & 0.500 & NA & -1.0\\
\hline
\hspace{1em}F & 8 & Satisfice & 0 & -0.071 & -0.071 & 0.500 & NA & -1.0\\
\hline
\hspace{1em}FO & 7 & Satisfice & 0 & -0.143 & -0.143 & 1.000 & NA & -1.0\\
\hline
\hspace{1em}AFG & 1 & Satisfice & 0 & -0.214 & -0.214 & 0.346 & NA & -1.0\\
\hline
\multicolumn{9}{l}{\textbf{Other}}\\
\hline
\hspace{1em} & 20 & blank & 0 & 0.000 & 0.000 & 0.000 & NA & 0.0\\
\hline
\hspace{1em}ACFHIJKOUZ & 1 & frenzy & 0 & 0.357 & 0.357 & 0.385 & NA & -0.5\\
\hline
\multicolumn{9}{l}{\textbf{Unknown}}\\
\hline
\hspace{1em}D & 35 & ? & 0 & -0.071 & -0.071 & -0.077 & NA & -0.5\\
\hline
\hspace{1em}A & 5 & ? & 0 & -0.071 & -0.071 & -0.077 & NA & -0.5\\
\hline
\hspace{1em}K & 3 & ? & 0 & -0.071 & -0.071 & -0.077 & NA & -0.5\\
\hline
\hspace{1em}G & 2 & ? & 0 & -0.071 & -0.071 & -0.077 & NA & -0.5\\
\hline
\hspace{1em}AI & 1 & ? & 0 & -0.143 & -0.143 & -0.154 & NA & -0.5\\
\hline
\hspace{1em}DK & 1 & ? & 0 & -0.143 & -0.143 & -0.154 & NA & -0.5\\
\hline
\hspace{1em}J & 1 & ? & 0 & -0.071 & -0.071 & -0.077 & NA & -0.5\\
\hline
\end{tabular}

\textbackslash end\{table\}

\begin{Shaded}
\begin{Highlighting}[]
\FunctionTok{gf\_dhistogram}\NormalTok{(}\SpecialCharTok{\textasciitilde{}}\NormalTok{ score\_niceABS, }\AttributeTok{fill =} \SpecialCharTok{\textasciitilde{}}\NormalTok{condition, }\AttributeTok{data =}\NormalTok{ df\_items }\SpecialCharTok{\%\textgreater{}\%} \FunctionTok{filter}\NormalTok{(q }\SpecialCharTok{==}\DecValTok{4}\NormalTok{)) }\SpecialCharTok{\%\textgreater{}\%} 
  \FunctionTok{gf\_facet\_grid}\NormalTok{( condition }\SpecialCharTok{\textasciitilde{}}\NormalTok{ ., }\AttributeTok{labeller =}\NormalTok{ label\_both) }\SpecialCharTok{+} 
  \FunctionTok{labs}\NormalTok{( }\AttributeTok{x =} \StringTok{"Scaled Item Score"}\NormalTok{, }\AttributeTok{title =} \StringTok{"Distribution of Scaled Scores | Q4 "}\NormalTok{) }\SpecialCharTok{+} 
  \FunctionTok{theme\_minimal}\NormalTok{() }\SpecialCharTok{+} \FunctionTok{theme}\NormalTok{(}\AttributeTok{legend.position =} \StringTok{"blank"}\NormalTok{)}
\end{Highlighting}
\end{Shaded}

\begin{figure}[H]

{\centering \includegraphics{analysis/SGC3A/2_sgc3A_scoring_files/figure-pdf/Q4-distribution-1.pdf}

}

\end{figure}

\begin{Shaded}
\begin{Highlighting}[]
\FunctionTok{gf\_props}\NormalTok{(}\SpecialCharTok{\textasciitilde{}}\NormalTok{interpretation, }\AttributeTok{fill =} \SpecialCharTok{\textasciitilde{}}\NormalTok{condition, }\AttributeTok{data =}\NormalTok{ df\_items }\SpecialCharTok{\%\textgreater{}\%} \FunctionTok{filter}\NormalTok{(q }\SpecialCharTok{==}\DecValTok{4}\NormalTok{)) }\SpecialCharTok{\%\textgreater{}\%} 
  \FunctionTok{gf\_facet\_grid}\NormalTok{( condition }\SpecialCharTok{\textasciitilde{}}\NormalTok{ ., }\AttributeTok{labeller =}\NormalTok{ label\_both) }\SpecialCharTok{+} 
  \FunctionTok{labs}\NormalTok{( }\AttributeTok{x =} \StringTok{"Interpretation"}\NormalTok{, }\AttributeTok{title =} \StringTok{"Distribution of Interpretations | Q4 "}\NormalTok{) }\SpecialCharTok{+} 
  \FunctionTok{theme\_minimal}\NormalTok{() }\SpecialCharTok{+} \FunctionTok{theme}\NormalTok{(}\AttributeTok{legend.position =} \StringTok{"blank"}\NormalTok{)}
\end{Highlighting}
\end{Shaded}

\begin{figure}[H]

{\centering \includegraphics{analysis/SGC3A/2_sgc3A_scoring_files/figure-pdf/Q4-distribution-2.pdf}

}

\end{figure}

\hypertarget{question-5}{%
\subsubsection{Question \#5}\label{question-5}}

\hypertarget{q5.-control-condition}{%
\subparagraph{Q5. Control Condition}\label{q5.-control-condition}}

\begin{figure}

{\centering \includegraphics{analysis/SGC3A/static/questions/Q5_111.png}

}

\caption{\label{fig-Q5-111}Q5---Control Condition}

\end{figure}

\begin{Shaded}
\begin{Highlighting}[]
\NormalTok{q }\OtherTok{\textless{}{-}}\NormalTok{ keys\_raw }\SpecialCharTok{\%\textgreater{}\%} \FunctionTok{filter}\NormalTok{(condition }\SpecialCharTok{==} \StringTok{"DEFAULT"}\NormalTok{) }\SpecialCharTok{\%\textgreater{}\%} \FunctionTok{filter}\NormalTok{(Q}\SpecialCharTok{==}\DecValTok{5}\NormalTok{)}
\NormalTok{ignore }\OtherTok{\textless{}{-}}\NormalTok{ q }\SpecialCharTok{\%\textgreater{}\%} \FunctionTok{select}\NormalTok{(}\StringTok{"REF\_POINT"}\NormalTok{)}
\NormalTok{answers }\OtherTok{\textless{}{-}}\NormalTok{ q }\SpecialCharTok{\%\textgreater{}\%} \FunctionTok{select}\NormalTok{(}\StringTok{"TRIANGULAR"}\NormalTok{, }\StringTok{"ORTHOGONAL"}\NormalTok{, }\StringTok{"SATISFICE\_left"}\NormalTok{, }\StringTok{"SATISFICE\_right"}\NormalTok{,}\StringTok{"TV\_max"}\NormalTok{,}\StringTok{"TV\_start"}\NormalTok{, }\StringTok{"TV\_end"}\NormalTok{, }\StringTok{"TV\_dur"}\NormalTok{) }\SpecialCharTok{\%\textgreater{}\%} \FunctionTok{unlist}\NormalTok{()}
\NormalTok{ves }\OtherTok{\textless{}{-}}\NormalTok{ q }\SpecialCharTok{\%\textgreater{}\%} \FunctionTok{mutate}\NormalTok{(}
  \AttributeTok{SATISFICE\_left\_allow =} \StringTok{""}\NormalTok{,}
  \AttributeTok{SATISFICE\_right\_allow =} \StringTok{""}
\NormalTok{) }\SpecialCharTok{\%\textgreater{}\%} \FunctionTok{select}\NormalTok{(}\StringTok{"TRI\_allow"}\NormalTok{, }\StringTok{"ORTH\_allow"}\NormalTok{, }\StringTok{"SATISFICE\_left\_allow"}\NormalTok{,}\StringTok{"SATISFICE\_right\_allow"}\NormalTok{, }\StringTok{"TV\_max\_allow"}\NormalTok{,}\StringTok{"TV\_start\_allow"}\NormalTok{,}\StringTok{"TV\_end\_allow"}\NormalTok{, }\StringTok{"TV\_dur\_allow"}\NormalTok{)}\SpecialCharTok{\%\textgreater{}\%} \FunctionTok{unlist}\NormalTok{() }
\NormalTok{options }\OtherTok{\textless{}{-}}\NormalTok{ q }\SpecialCharTok{\%\textgreater{}\%} \FunctionTok{select}\NormalTok{(}\StringTok{"OPTIONS"}\NormalTok{)}
\NormalTok{question }\OtherTok{=}\NormalTok{ q }\SpecialCharTok{\%\textgreater{}\%}  \FunctionTok{select}\NormalTok{(}\StringTok{"TEXT"}\NormalTok{)}
\NormalTok{scores }\OtherTok{\textless{}{-}} \FunctionTok{c}\NormalTok{(}\StringTok{"Triangular"}\NormalTok{, }\StringTok{"Orthgonal"}\NormalTok{, }\StringTok{"Satisficing [left]"}\NormalTok{, }\StringTok{"Satisficing [right]"}\NormalTok{, }\StringTok{"Tversky [maximal]"}\NormalTok{, }\StringTok{"Tversky [start diagonal]"}\NormalTok{, }
            \StringTok{"Tversky [end diagonal]"}\NormalTok{, }\StringTok{"Tversky [duration line]"}\NormalTok{)}
\NormalTok{d }\OtherTok{=} \FunctionTok{tibble}\NormalTok{(}\AttributeTok{interpretation =}\NormalTok{ scores, }\AttributeTok{answer =}\NormalTok{ answers, }\AttributeTok{allowed=}\NormalTok{ves)}
\NormalTok{d}\SpecialCharTok{$}\NormalTok{answer }\OtherTok{\textless{}{-}} \FunctionTok{replace\_na}\NormalTok{(d}\SpecialCharTok{$}\NormalTok{answer, }\StringTok{""}\NormalTok{)}
\NormalTok{d}\SpecialCharTok{$}\NormalTok{allowed }\OtherTok{\textless{}{-}} \FunctionTok{replace\_na}\NormalTok{(d}\SpecialCharTok{$}\NormalTok{allowed, }\StringTok{""}\NormalTok{)}

\NormalTok{title }\OtherTok{=} \FunctionTok{paste}\NormalTok{(}\StringTok{"Answer Key | Q5 Control Condition : "}\NormalTok{, question)}
\NormalTok{cols }\OtherTok{=} \FunctionTok{c}\NormalTok{(}\StringTok{"interpretation"}\NormalTok{, }\StringTok{"answer"}\NormalTok{,}\StringTok{"not penalized"}\NormalTok{)}

\NormalTok{d }\SpecialCharTok{\%\textgreater{}\%} \FunctionTok{kbl}\NormalTok{(}\AttributeTok{caption =}\NormalTok{ title, }\AttributeTok{col.names =}\NormalTok{ cols) }\SpecialCharTok{\%\textgreater{}\%} \FunctionTok{kable\_classic}\NormalTok{() }\SpecialCharTok{\%\textgreater{}\%} 
  \FunctionTok{footnote}\NormalTok{(}\AttributeTok{general =} \FunctionTok{paste}\NormalTok{(}\StringTok{"15 response options: "}\NormalTok{, options), }\AttributeTok{general\_title =} \StringTok{"Note: "}\NormalTok{,}\AttributeTok{footnote\_as\_chunk =}\NormalTok{ T) }
\end{Highlighting}
\end{Shaded}

\begin{table}

\caption{Answer Key | Q5 Control Condition :  Coffee breaks happen halfway through a shift. Which shift(s) share a break with I?}
\centering
\begin{tabular}[t]{l|l|l}
\hline
interpretation & answer & not penalized\\
\hline
Triangular & O & AZ\\
\hline
Orthgonal & U & \\
\hline
Satisficing [left] &  & \\
\hline
Satisficing [right] &  & \\
\hline
Tversky [maximal] & UGX & AZKD\\
\hline
Tversky [start diagonal] & X & \\
\hline
Tversky [end diagonal] & UG & \\
\hline
Tversky [duration line] &  & \\
\hline
\multicolumn{3}{l}{\rule{0pt}{1em}\textit{Note: } 15 response options:  AIKGXJDBCHUZOFE}\\
\end{tabular}
\end{table}

\begin{Shaded}
\begin{Highlighting}[]
\NormalTok{title }\OtherTok{\textless{}{-}} \StringTok{"Frequency of Selected Response Options for Question \#5 (Control Condition)"}
\NormalTok{names }\OtherTok{=} \FunctionTok{c}\NormalTok{(}\StringTok{"response"}\NormalTok{,}\StringTok{"n"}\NormalTok{,}\StringTok{"interpretation"}\NormalTok{,}\StringTok{"absolute"}\NormalTok{,}\StringTok{"tri"}\NormalTok{,}\StringTok{"tversky"}\NormalTok{,}\StringTok{"satisfice"}\NormalTok{,}\StringTok{"orthogonal"}\NormalTok{, }\StringTok{"scaled score"}\NormalTok{)}

\NormalTok{df\_items }\SpecialCharTok{\%\textgreater{}\%} \FunctionTok{filter}\NormalTok{(q }\SpecialCharTok{==} \DecValTok{5} \SpecialCharTok{\&}\NormalTok{ condition }\SpecialCharTok{==} \DecValTok{111}\NormalTok{) }\SpecialCharTok{\%\textgreater{}\%} \FunctionTok{group\_by}\NormalTok{(response) }\SpecialCharTok{\%\textgreater{}\%} 
\NormalTok{  dplyr}\SpecialCharTok{::}\FunctionTok{summarise}\NormalTok{( }\AttributeTok{count =} \FunctionTok{n}\NormalTok{(), }
                    \AttributeTok{nice =} \FunctionTok{unique}\NormalTok{(score\_niceABS),}
                    \AttributeTok{triangular =} \FunctionTok{unique}\NormalTok{(score\_TRI), }
                    \AttributeTok{orthogonal =}  \FunctionTok{unique}\NormalTok{(score\_ORTH),}
                    \AttributeTok{satisficing =}  \FunctionTok{unique}\NormalTok{(score\_SATISFICE),}
                    \AttributeTok{tversky =} \FunctionTok{unique}\NormalTok{(score\_TVERSKY),}
                    \AttributeTok{interpretation =} \FunctionTok{unique}\NormalTok{(int2),}
                    \AttributeTok{scaled =} \FunctionTok{unique}\NormalTok{(score\_SCALED)) }\SpecialCharTok{\%\textgreater{}\%} 
  \FunctionTok{arrange}\NormalTok{(interpretation, }\FunctionTok{desc}\NormalTok{(count)) }\SpecialCharTok{\%\textgreater{}\%} 
  \FunctionTok{select}\NormalTok{(response, count, interpretation, nice, }
\NormalTok{         triangular, tversky, satisficing, orthogonal, scaled) }\SpecialCharTok{\%\textgreater{}\%} 
  \FunctionTok{kbl}\NormalTok{(}\AttributeTok{caption =}\NormalTok{ title, }\AttributeTok{col.names =}\NormalTok{ names) }\SpecialCharTok{\%\textgreater{}\%}  \FunctionTok{kable\_classic}\NormalTok{() }\SpecialCharTok{\%\textgreater{}\%} 
  \FunctionTok{add\_header\_above}\NormalTok{(}\FunctionTok{c}\NormalTok{(}\StringTok{" "} \OtherTok{=} \DecValTok{3}\NormalTok{, }\StringTok{"Strict Score"} \OtherTok{=} \DecValTok{1}\NormalTok{, }\StringTok{"Interpretation Scores"}\OtherTok{=}\DecValTok{4}\NormalTok{, }\StringTok{"Discriminant"}\OtherTok{=}\DecValTok{1}\NormalTok{)) }\SpecialCharTok{\%\textgreater{}\%}
  \FunctionTok{pack\_rows}\NormalTok{(}\StringTok{"Triangular"}\NormalTok{, }\DecValTok{1}\NormalTok{, }\DecValTok{4}\NormalTok{) }\SpecialCharTok{\%\textgreater{}\%} 
  \FunctionTok{pack\_rows}\NormalTok{(}\StringTok{"Lines{-}Connect"}\NormalTok{, }\DecValTok{5}\NormalTok{, }\DecValTok{7}\NormalTok{) }\SpecialCharTok{\%\textgreater{}\%} 
  \FunctionTok{pack\_rows}\NormalTok{(}\StringTok{"Orthogonal"}\NormalTok{, }\DecValTok{8}\NormalTok{, }\DecValTok{9}\NormalTok{) }\SpecialCharTok{\%\textgreater{}\%} 
  \FunctionTok{pack\_rows}\NormalTok{(}\StringTok{"Other"}\NormalTok{, }\DecValTok{10}\NormalTok{, }\DecValTok{11}\NormalTok{) }\SpecialCharTok{\%\textgreater{}\%} 
  \FunctionTok{pack\_rows}\NormalTok{(}\StringTok{"Unknown"}\NormalTok{, }\DecValTok{12}\NormalTok{, }\DecValTok{22}\NormalTok{) }
\end{Highlighting}
\end{Shaded}

\textbackslash begin\{table\}

\textbackslash caption\{\label{tab:Q5-CONTROL-RESPONSES}Frequency of
Selected Response Options for Question \#5 (Control Condition)\}
\centering

\begin{tabular}[t]{l|r|l|r|r|r|r|r|r}
\hline
\multicolumn{3}{c|}{ } & \multicolumn{1}{c|}{Strict Score} & \multicolumn{4}{c|}{Interpretation Scores} & \multicolumn{1}{c}{Discriminant} \\
\cline{4-4} \cline{5-8} \cline{9-9}
response & n & interpretation & absolute & tri & tversky & satisfice & orthogonal & scaled score\\
\hline
\multicolumn{9}{l}{\textbf{Triangular}}\\
\hline
\hspace{1em}O & 50 & Triangular & 1 & 1.000 & -0.077 & NA & -0.077 & 1.0\\
\hline
\hspace{1em}FO & 3 & Triangular & 0 & 0.909 & -0.154 & NA & -0.154 & 1.0\\
\hline
\hspace{1em}HO & 1 & Triangular & 0 & 0.909 & -0.154 & NA & -0.154 & 1.0\\
\hline
\hspace{1em}KO & 1 & Triangular & 0 & 0.909 & -0.143 & NA & -0.154 & 1.0\\
\hline
\multicolumn{9}{l}{\textbf{Lines-Connect}}\\
\hline
\hspace{1em}FG & 2 & Tversky & 0 & -0.182 & 0.417 & NA & -0.154 & 0.5\\
\hline
\hspace{1em}G & 1 & Tversky & 0 & -0.091 & 0.500 & NA & -0.077 & 0.5\\
\hline
\hspace{1em}X & 1 & Tversky & 0 & -0.091 & 1.000 & NA & -0.077 & 0.5\\
\hline
\multicolumn{9}{l}{\textbf{Orthogonal}}\\
\hline
\hspace{1em}U & 64 & Orthogonal & 0 & -0.091 & 0.500 & NA & 1.000 & -1.0\\
\hline
\hspace{1em}HU & 1 & Orthogonal & 0 & -0.182 & 0.417 & NA & 0.923 & -1.0\\
\hline
\multicolumn{9}{l}{\textbf{Other}}\\
\hline
\hspace{1em}I & 1 & reference & 0 & 0.000 & 0.000 & NA & 0.000 & 0.0\\
\hline
\hspace{1em} & 6 & blank & 0 & 0.000 & 0.000 & NA & 0.000 & 0.0\\
\hline
\multicolumn{9}{l}{\textbf{Unknown}}\\
\hline
\hspace{1em}F & 10 & ? & 0 & -0.091 & -0.077 & NA & -0.077 & -0.5\\
\hline
\hspace{1em}H & 3 & ? & 0 & -0.091 & -0.077 & NA & -0.077 & -0.5\\
\hline
\hspace{1em}J & 3 & ? & 0 & -0.091 & -0.077 & NA & -0.077 & -0.5\\
\hline
\hspace{1em}B & 2 & ? & 0 & -0.091 & -0.077 & NA & -0.077 & -0.5\\
\hline
\hspace{1em}DJ & 2 & ? & 0 & -0.182 & -0.143 & NA & -0.154 & -0.5\\
\hline
\hspace{1em}K & 2 & ? & 0 & -0.091 & 0.000 & NA & -0.077 & -0.5\\
\hline
\hspace{1em}C & 1 & ? & 0 & -0.091 & -0.077 & NA & -0.077 & -0.5\\
\hline
\hspace{1em}DEHJ & 1 & ? & 0 & -0.364 & -0.308 & NA & -0.308 & -0.5\\
\hline
\hspace{1em}FK & 1 & ? & 0 & -0.182 & -0.143 & NA & -0.154 & -0.5\\
\hline
\hspace{1em}HJ & 1 & ? & 0 & -0.182 & -0.154 & NA & -0.154 & -0.5\\
\hline
\hspace{1em}Z & 1 & ? & 0 & 0.000 & 0.000 & NA & -0.077 & -0.5\\
\hline
\end{tabular}

\textbackslash end\{table\}

TODO note the compelling cases of internal inconsistency (HJDE)

\hypertarget{q5.-impasse-condition}{%
\paragraph{Q5. Impasse Condition}\label{q5.-impasse-condition}}

\begin{figure}

{\centering \includegraphics{analysis/SGC3A/static/questions/Q5_121.png}

}

\caption{\label{fig-Q5-121}Q5---Impasse Condition}

\end{figure}

\begin{Shaded}
\begin{Highlighting}[]
\NormalTok{q }\OtherTok{\textless{}{-}}\NormalTok{ keys\_raw }\SpecialCharTok{\%\textgreater{}\%} \FunctionTok{filter}\NormalTok{(condition }\SpecialCharTok{==} \DecValTok{121}\NormalTok{) }\SpecialCharTok{\%\textgreater{}\%} \FunctionTok{filter}\NormalTok{(Q}\SpecialCharTok{==}\DecValTok{5}\NormalTok{)}
\NormalTok{ignore }\OtherTok{\textless{}{-}}\NormalTok{ q }\SpecialCharTok{\%\textgreater{}\%} \FunctionTok{select}\NormalTok{(}\StringTok{"REF\_POINT"}\NormalTok{)}
\NormalTok{answers }\OtherTok{\textless{}{-}}\NormalTok{ q }\SpecialCharTok{\%\textgreater{}\%} \FunctionTok{select}\NormalTok{(}\StringTok{"TRIANGULAR"}\NormalTok{, }\StringTok{"ORTHOGONAL"}\NormalTok{, }\StringTok{"SATISFICE\_left"}\NormalTok{, }\StringTok{"SATISFICE\_right"}\NormalTok{,}\StringTok{"TV\_max"}\NormalTok{,}\StringTok{"TV\_start"}\NormalTok{, }\StringTok{"TV\_end"}\NormalTok{, }\StringTok{"TV\_dur"}\NormalTok{) }\SpecialCharTok{\%\textgreater{}\%} \FunctionTok{unlist}\NormalTok{()}
\NormalTok{ves }\OtherTok{\textless{}{-}}\NormalTok{ q }\SpecialCharTok{\%\textgreater{}\%} \FunctionTok{mutate}\NormalTok{(}
  \AttributeTok{SATISFICE\_left\_allow =} \StringTok{""}\NormalTok{,}
  \AttributeTok{SATISFICE\_right\_allow =} \StringTok{""}
\NormalTok{) }\SpecialCharTok{\%\textgreater{}\%} \FunctionTok{select}\NormalTok{(}\StringTok{"TRI\_allow"}\NormalTok{, }\StringTok{"ORTH\_allow"}\NormalTok{, }\StringTok{"SATISFICE\_left\_allow"}\NormalTok{,}\StringTok{"SATISFICE\_right\_allow"}\NormalTok{, }\StringTok{"TV\_max\_allow"}\NormalTok{,}\StringTok{"TV\_start\_allow"}\NormalTok{,}\StringTok{"TV\_end\_allow"}\NormalTok{, }\StringTok{"TV\_dur\_allow"}\NormalTok{)}\SpecialCharTok{\%\textgreater{}\%} \FunctionTok{unlist}\NormalTok{() }
\NormalTok{options }\OtherTok{\textless{}{-}}\NormalTok{ q }\SpecialCharTok{\%\textgreater{}\%} \FunctionTok{select}\NormalTok{(}\StringTok{"OPTIONS"}\NormalTok{)}
\NormalTok{question }\OtherTok{=}\NormalTok{ q }\SpecialCharTok{\%\textgreater{}\%}  \FunctionTok{select}\NormalTok{(}\StringTok{"TEXT"}\NormalTok{)}
\NormalTok{scores }\OtherTok{\textless{}{-}} \FunctionTok{c}\NormalTok{(}\StringTok{"Triangular"}\NormalTok{, }\StringTok{"Orthgonal"}\NormalTok{, }\StringTok{"Satisficing [left]"}\NormalTok{, }\StringTok{"Satisficing [right]"}\NormalTok{, }\StringTok{"Tversky [maximal]"}\NormalTok{, }\StringTok{"Tversky [start diagonal]"}\NormalTok{, }
            \StringTok{"Tversky [end diagonal]"}\NormalTok{, }\StringTok{"Tversky [duration line]"}\NormalTok{)}
\NormalTok{d }\OtherTok{=} \FunctionTok{tibble}\NormalTok{(}\AttributeTok{interpretation =}\NormalTok{ scores, }\AttributeTok{answer =}\NormalTok{ answers, }\AttributeTok{allowed=}\NormalTok{ves)}
\NormalTok{d}\SpecialCharTok{$}\NormalTok{answer }\OtherTok{\textless{}{-}} \FunctionTok{replace\_na}\NormalTok{(d}\SpecialCharTok{$}\NormalTok{answer, }\StringTok{""}\NormalTok{)}
\NormalTok{d}\SpecialCharTok{$}\NormalTok{allowed }\OtherTok{\textless{}{-}} \FunctionTok{replace\_na}\NormalTok{(d}\SpecialCharTok{$}\NormalTok{allowed, }\StringTok{""}\NormalTok{)}

\NormalTok{title }\OtherTok{=} \FunctionTok{paste}\NormalTok{(}\StringTok{"Answer Key | Q5 Impasse Condition : "}\NormalTok{, question)}
\NormalTok{cols }\OtherTok{=} \FunctionTok{c}\NormalTok{(}\StringTok{"interpretation"}\NormalTok{, }\StringTok{"answer"}\NormalTok{,}\StringTok{"not penalized"}\NormalTok{)}

\NormalTok{d }\SpecialCharTok{\%\textgreater{}\%} \FunctionTok{kbl}\NormalTok{(}\AttributeTok{caption =}\NormalTok{ title, }\AttributeTok{col.names =}\NormalTok{ cols) }\SpecialCharTok{\%\textgreater{}\%} \FunctionTok{kable\_classic}\NormalTok{() }\SpecialCharTok{\%\textgreater{}\%} 
  \FunctionTok{footnote}\NormalTok{(}\AttributeTok{general =} \FunctionTok{paste}\NormalTok{(}\StringTok{"15 response options: "}\NormalTok{, options), }\AttributeTok{general\_title =} \StringTok{"Note: "}\NormalTok{,}\AttributeTok{footnote\_as\_chunk =}\NormalTok{ T) }
\end{Highlighting}
\end{Shaded}

\begin{table}

\caption{Answer Key | Q5 Impasse Condition :  Coffee breaks happen halfway through a shift.</br> Which shift(s) share a break with I?}
\centering
\begin{tabular}[t]{l|l|l}
\hline
interpretation & answer & not penalized\\
\hline
Triangular & A & \\
\hline
Orthgonal &  & \\
\hline
Satisficing [left] & K & \\
\hline
Satisficing [right] &  & \\
\hline
Tversky [maximal] & OX & \\
\hline
Tversky [start diagonal] & OX & \\
\hline
Tversky [end diagonal] &  & \\
\hline
Tversky [duration line] &  & \\
\hline
\multicolumn{3}{l}{\rule{0pt}{1em}\textit{Note: } 15 response options:  AIKGXJDBCHUZOFE}\\
\end{tabular}
\end{table}

\begin{Shaded}
\begin{Highlighting}[]
\NormalTok{title }\OtherTok{\textless{}{-}} \StringTok{"Frequency of Selected Response Options for Question \#5 (Control Condition)"}
\NormalTok{names }\OtherTok{=} \FunctionTok{c}\NormalTok{(}\StringTok{"response"}\NormalTok{,}\StringTok{"n"}\NormalTok{,}\StringTok{"interpretation"}\NormalTok{,}\StringTok{"absolute"}\NormalTok{,}\StringTok{"tri"}\NormalTok{,}\StringTok{"tversky"}\NormalTok{,}\StringTok{"satisfice"}\NormalTok{,}\StringTok{"orthogonal"}\NormalTok{, }\StringTok{"scaled score"}\NormalTok{)}

\NormalTok{df\_items }\SpecialCharTok{\%\textgreater{}\%} \FunctionTok{filter}\NormalTok{(q }\SpecialCharTok{==} \DecValTok{5} \SpecialCharTok{\&}\NormalTok{ condition }\SpecialCharTok{==} \DecValTok{121}\NormalTok{) }\SpecialCharTok{\%\textgreater{}\%} \FunctionTok{group\_by}\NormalTok{(response) }\SpecialCharTok{\%\textgreater{}\%} 
\NormalTok{  dplyr}\SpecialCharTok{::}\FunctionTok{summarise}\NormalTok{( }\AttributeTok{count =} \FunctionTok{n}\NormalTok{(), }
                    \AttributeTok{nice =} \FunctionTok{unique}\NormalTok{(score\_niceABS),}
                    \AttributeTok{triangular =} \FunctionTok{unique}\NormalTok{(score\_TRI), }
                    \AttributeTok{orthogonal =}  \FunctionTok{unique}\NormalTok{(score\_ORTH),}
                    \AttributeTok{satisficing =}  \FunctionTok{unique}\NormalTok{(score\_SATISFICE),}
                    \AttributeTok{tversky =} \FunctionTok{unique}\NormalTok{(score\_TVERSKY),}
                    \AttributeTok{interpretation =} \FunctionTok{unique}\NormalTok{(int2),}
                    \AttributeTok{scaled =} \FunctionTok{unique}\NormalTok{(score\_SCALED)) }\SpecialCharTok{\%\textgreater{}\%} 
  \FunctionTok{arrange}\NormalTok{(interpretation, }\FunctionTok{desc}\NormalTok{(count)) }\SpecialCharTok{\%\textgreater{}\%} 
  \FunctionTok{select}\NormalTok{(response, count, interpretation, nice, }
\NormalTok{         triangular, tversky, satisficing, orthogonal, scaled) }\SpecialCharTok{\%\textgreater{}\%} 
  \FunctionTok{kbl}\NormalTok{(}\AttributeTok{caption =}\NormalTok{ title, }\AttributeTok{col.names =}\NormalTok{ names) }\SpecialCharTok{\%\textgreater{}\%}  \FunctionTok{kable\_classic}\NormalTok{() }\SpecialCharTok{\%\textgreater{}\%} 
  \FunctionTok{add\_header\_above}\NormalTok{(}\FunctionTok{c}\NormalTok{(}\StringTok{" "} \OtherTok{=} \DecValTok{3}\NormalTok{, }\StringTok{"Strict Score"} \OtherTok{=} \DecValTok{1}\NormalTok{, }\StringTok{"Interpretation Scores"}\OtherTok{=}\DecValTok{4}\NormalTok{, }\StringTok{"Discriminant"}\OtherTok{=}\DecValTok{1}\NormalTok{)) }\SpecialCharTok{\%\textgreater{}\%}
  \FunctionTok{pack\_rows}\NormalTok{(}\StringTok{"Triangular"}\NormalTok{, }\DecValTok{1}\NormalTok{, }\DecValTok{7}\NormalTok{) }\SpecialCharTok{\%\textgreater{}\%} 
  \FunctionTok{pack\_rows}\NormalTok{(}\StringTok{"Lines{-}Connect"}\NormalTok{, }\DecValTok{8}\NormalTok{, }\DecValTok{13}\NormalTok{) }\SpecialCharTok{\%\textgreater{}\%} 
  \FunctionTok{pack\_rows}\NormalTok{(}\StringTok{"Orthogonal"}\NormalTok{, }\DecValTok{14}\NormalTok{, }\DecValTok{16}\NormalTok{) }\SpecialCharTok{\%\textgreater{}\%} 
  \FunctionTok{pack\_rows}\NormalTok{(}\StringTok{"Other"}\NormalTok{, }\DecValTok{17}\NormalTok{, }\DecValTok{21}\NormalTok{) }\SpecialCharTok{\%\textgreater{}\%} 
  \FunctionTok{pack\_rows}\NormalTok{(}\StringTok{"Unknown"}\NormalTok{, }\DecValTok{22}\NormalTok{, }\DecValTok{31}\NormalTok{) }
\end{Highlighting}
\end{Shaded}

\textbackslash begin\{table\}

\textbackslash caption\{\label{tab:Q5-IMPASSE-RESPONSES}Frequency of
Selected Response Options for Question \#5 (Control Condition)\}
\centering

\begin{tabular}[t]{l|r|l|r|r|r|r|r|r}
\hline
\multicolumn{3}{c|}{ } & \multicolumn{1}{c|}{Strict Score} & \multicolumn{4}{c|}{Interpretation Scores} & \multicolumn{1}{c}{Discriminant} \\
\cline{4-4} \cline{5-8} \cline{9-9}
response & n & interpretation & absolute & tri & tversky & satisfice & orthogonal & scaled score\\
\hline
\multicolumn{9}{l}{\textbf{Triangular}}\\
\hline
\hspace{1em}A & 83 & Triangular & 1 & 1.000 & -0.083 & -0.077 & NA & 1.0\\
\hline
\hspace{1em}AFG & 5 & Triangular & 0 & 0.846 & -0.250 & -0.231 & NA & 1.0\\
\hline
\hspace{1em}AF & 4 & Triangular & 0 & 0.923 & -0.167 & -0.154 & NA & 1.0\\
\hline
\hspace{1em}AO & 2 & Triangular & 0 & 0.923 & 0.417 & -0.154 & NA & 1.0\\
\hline
\hspace{1em}AI & 1 & Triangular & 1 & 1.000 & -0.083 & -0.077 & NA & 1.0\\
\hline
\hspace{1em}AU & 1 & Triangular & 0 & 0.923 & -0.167 & -0.154 & NA & 1.0\\
\hline
\hspace{1em}AZ & 1 & Triangular & 0 & 0.923 & -0.167 & -0.154 & NA & 1.0\\
\hline
\multicolumn{9}{l}{\textbf{Lines-Connect}}\\
\hline
\hspace{1em}O & 6 & Tversky & 0 & -0.077 & 0.500 & -0.077 & NA & 0.5\\
\hline
\hspace{1em}CO & 1 & Tversky & 0 & -0.154 & 0.417 & -0.154 & NA & 0.5\\
\hline
\hspace{1em}JO & 1 & Tversky & 0 & -0.154 & 0.417 & -0.154 & NA & 0.5\\
\hline
\hspace{1em}OX & 1 & Tversky & 0 & -0.154 & 1.000 & -0.154 & NA & 0.5\\
\hline
\hspace{1em}UXZ & 1 & Tversky & 0 & -0.231 & 0.333 & -0.231 & NA & 0.5\\
\hline
\hspace{1em}X & 1 & Tversky & 0 & -0.077 & 0.500 & -0.077 & NA & 0.5\\
\hline
\multicolumn{9}{l}{\textbf{Orthogonal}}\\
\hline
\hspace{1em}K & 5 & Satisfice & 0 & -0.077 & -0.083 & 1.000 & NA & -1.0\\
\hline
\hspace{1em}HK & 3 & Satisfice & 0 & -0.154 & -0.167 & 0.923 & NA & -1.0\\
\hline
\hspace{1em}HKUZ & 1 & Satisfice & 0 & -0.308 & -0.333 & 0.769 & NA & -1.0\\
\hline
\multicolumn{9}{l}{\textbf{Other}}\\
\hline
\hspace{1em}I & 2 & reference & 0 & 0.000 & 0.000 & 0.000 & NA & 0.0\\
\hline
\hspace{1em} & 24 & blank & 0 & 0.000 & 0.000 & 0.000 & NA & 0.0\\
\hline
\hspace{1em}ABCFGUZ & 1 & frenzy & 0 & 0.538 & -0.583 & -0.538 & NA & -0.5\\
\hline
\hspace{1em}ACDEFHIJKOUXZ & 1 & frenzy & 0 & 0.154 & 0.167 & 0.154 & NA & -0.5\\
\hline
\hspace{1em}FHJKX & 1 & frenzy & 0 & -0.385 & 0.167 & 0.692 & NA & -0.5\\
\hline
\multicolumn{9}{l}{\textbf{Unknown}}\\
\hline
\hspace{1em}H & 11 & ? & 0 & -0.077 & -0.083 & -0.077 & NA & -0.5\\
\hline
\hspace{1em}J & 3 & ? & 0 & -0.077 & -0.083 & -0.077 & NA & -0.5\\
\hline
\hspace{1em}C & 2 & ? & 0 & -0.077 & -0.083 & -0.077 & NA & -0.5\\
\hline
\hspace{1em}DJ & 2 & ? & 0 & -0.154 & -0.167 & -0.154 & NA & -0.5\\
\hline
\hspace{1em}F & 2 & ? & 0 & -0.077 & -0.083 & -0.077 & NA & -0.5\\
\hline
\hspace{1em}FU & 2 & ? & 0 & -0.154 & -0.167 & -0.154 & NA & -0.5\\
\hline
\hspace{1em}DG & 1 & ? & 0 & -0.154 & -0.167 & -0.154 & NA & -0.5\\
\hline
\hspace{1em}FHZ & 1 & ? & 0 & -0.231 & -0.250 & -0.231 & NA & -0.5\\
\hline
\hspace{1em}U & 1 & ? & 0 & -0.077 & -0.083 & -0.077 & NA & -0.5\\
\hline
\hspace{1em}Z & 1 & ? & 0 & -0.077 & -0.083 & -0.077 & NA & -0.5\\
\hline
\end{tabular}

\textbackslash end\{table\}

\begin{Shaded}
\begin{Highlighting}[]
\FunctionTok{gf\_dhistogram}\NormalTok{(}\SpecialCharTok{\textasciitilde{}}\NormalTok{ score\_niceABS, }\AttributeTok{fill =} \SpecialCharTok{\textasciitilde{}}\NormalTok{condition, }\AttributeTok{data =}\NormalTok{ df\_items }\SpecialCharTok{\%\textgreater{}\%} \FunctionTok{filter}\NormalTok{(q }\SpecialCharTok{==}\DecValTok{5}\NormalTok{)) }\SpecialCharTok{\%\textgreater{}\%} 
  \FunctionTok{gf\_facet\_grid}\NormalTok{( condition }\SpecialCharTok{\textasciitilde{}}\NormalTok{ ., }\AttributeTok{labeller =}\NormalTok{ label\_both) }\SpecialCharTok{+} 
  \FunctionTok{labs}\NormalTok{( }\AttributeTok{x =} \StringTok{"Scaled Item Score"}\NormalTok{, }\AttributeTok{title =} \StringTok{"Distribution of Scaled Scores | Q5 "}\NormalTok{) }\SpecialCharTok{+} 
  \FunctionTok{theme\_minimal}\NormalTok{() }\SpecialCharTok{+} \FunctionTok{theme}\NormalTok{(}\AttributeTok{legend.position =} \StringTok{"blank"}\NormalTok{)}
\end{Highlighting}
\end{Shaded}

\begin{figure}[H]

{\centering \includegraphics{analysis/SGC3A/2_sgc3A_scoring_files/figure-pdf/Q5-distribution-1.pdf}

}

\end{figure}

\begin{Shaded}
\begin{Highlighting}[]
\FunctionTok{gf\_props}\NormalTok{(}\SpecialCharTok{\textasciitilde{}}\NormalTok{interpretation, }\AttributeTok{fill =} \SpecialCharTok{\textasciitilde{}}\NormalTok{condition, }\AttributeTok{data =}\NormalTok{ df\_items }\SpecialCharTok{\%\textgreater{}\%} \FunctionTok{filter}\NormalTok{(q }\SpecialCharTok{==}\DecValTok{5}\NormalTok{)) }\SpecialCharTok{\%\textgreater{}\%} 
  \FunctionTok{gf\_facet\_grid}\NormalTok{( condition }\SpecialCharTok{\textasciitilde{}}\NormalTok{ ., }\AttributeTok{labeller =}\NormalTok{ label\_both) }\SpecialCharTok{+} 
  \FunctionTok{labs}\NormalTok{( }\AttributeTok{x =} \StringTok{"Interpretation"}\NormalTok{, }\AttributeTok{title =} \StringTok{"Distribution of Interpretations | Q5 "}\NormalTok{) }\SpecialCharTok{+} 
  \FunctionTok{theme\_minimal}\NormalTok{() }\SpecialCharTok{+} \FunctionTok{theme}\NormalTok{(}\AttributeTok{legend.position =} \StringTok{"blank"}\NormalTok{)}
\end{Highlighting}
\end{Shaded}

\begin{figure}[H]

{\centering \includegraphics{analysis/SGC3A/2_sgc3A_scoring_files/figure-pdf/Q5-distribution-2.pdf}

}

\end{figure}

\hypertarget{testing-phase}{%
\subsection{Testing Phase}\label{testing-phase}}

The following 10 questions were the same for both conditions.

\hypertarget{question-6-nondiscrim}{%
\subsubsection{Question \#6 NONDISCRIM}\label{question-6-nondiscrim}}

\begin{figure}

{\centering \includegraphics{analysis/SGC3A/static/questions/Q6.png}

}

\caption{\label{fig-Q6}Q6-Question}

\end{figure}

\begin{Shaded}
\begin{Highlighting}[]
\NormalTok{q }\OtherTok{\textless{}{-}}\NormalTok{ keys\_raw }\SpecialCharTok{\%\textgreater{}\%} \FunctionTok{filter}\NormalTok{(Q}\SpecialCharTok{==}\DecValTok{6}\NormalTok{)}
\NormalTok{ignore }\OtherTok{\textless{}{-}}\NormalTok{ q }\SpecialCharTok{\%\textgreater{}\%} \FunctionTok{select}\NormalTok{(}\StringTok{"REF\_POINT"}\NormalTok{)}
\NormalTok{answers }\OtherTok{\textless{}{-}}\NormalTok{ q }\SpecialCharTok{\%\textgreater{}\%} \FunctionTok{select}\NormalTok{(}\StringTok{"TRIANGULAR"}\NormalTok{, }\StringTok{"ORTHOGONAL"}\NormalTok{, }\StringTok{"SATISFICE\_left"}\NormalTok{, }\StringTok{"SATISFICE\_right"}\NormalTok{,}\StringTok{"TV\_max"}\NormalTok{,}\StringTok{"TV\_start"}\NormalTok{, }\StringTok{"TV\_end"}\NormalTok{, }\StringTok{"TV\_dur"}\NormalTok{) }\SpecialCharTok{\%\textgreater{}\%} \FunctionTok{unlist}\NormalTok{()}
\NormalTok{ves }\OtherTok{\textless{}{-}}\NormalTok{ q }\SpecialCharTok{\%\textgreater{}\%} \FunctionTok{mutate}\NormalTok{(}
  \AttributeTok{SATISFICE\_left\_allow =} \StringTok{""}\NormalTok{,}
  \AttributeTok{SATISFICE\_right\_allow =} \StringTok{""}
\NormalTok{) }\SpecialCharTok{\%\textgreater{}\%} \FunctionTok{select}\NormalTok{(}\StringTok{"TRI\_allow"}\NormalTok{, }\StringTok{"ORTH\_allow"}\NormalTok{, }\StringTok{"SATISFICE\_left\_allow"}\NormalTok{,}\StringTok{"SATISFICE\_right\_allow"}\NormalTok{, }\StringTok{"TV\_max\_allow"}\NormalTok{,}\StringTok{"TV\_start\_allow"}\NormalTok{,}\StringTok{"TV\_end\_allow"}\NormalTok{, }\StringTok{"TV\_dur\_allow"}\NormalTok{)}\SpecialCharTok{\%\textgreater{}\%} \FunctionTok{unlist}\NormalTok{()}
\NormalTok{options }\OtherTok{\textless{}{-}}\NormalTok{ q }\SpecialCharTok{\%\textgreater{}\%} \FunctionTok{select}\NormalTok{(}\StringTok{"OPTIONS"}\NormalTok{)}
\NormalTok{question }\OtherTok{=}\NormalTok{ q }\SpecialCharTok{\%\textgreater{}\%}  \FunctionTok{select}\NormalTok{(}\StringTok{"TEXT"}\NormalTok{)}
\NormalTok{scores }\OtherTok{\textless{}{-}} \FunctionTok{c}\NormalTok{(}\StringTok{"Triangular"}\NormalTok{, }\StringTok{"Orthgonal"}\NormalTok{, }\StringTok{"Satisficing [left]"}\NormalTok{, }\StringTok{"Satisficing [right]"}\NormalTok{, }\StringTok{"Tversky [maximal]"}\NormalTok{, }\StringTok{"Tversky [start diagonal]"}\NormalTok{,}
            \StringTok{"Tversky [end diagonal]"}\NormalTok{, }\StringTok{"Tversky [duration line]"}\NormalTok{)}
\NormalTok{d }\OtherTok{=} \FunctionTok{tibble}\NormalTok{(}\AttributeTok{interpretation =}\NormalTok{ scores, }\AttributeTok{answer =}\NormalTok{ answers, }\AttributeTok{allowed=}\NormalTok{ves)}
\NormalTok{d}\SpecialCharTok{$}\NormalTok{answer }\OtherTok{\textless{}{-}} \FunctionTok{replace\_na}\NormalTok{(d}\SpecialCharTok{$}\NormalTok{answer, }\StringTok{""}\NormalTok{)}
\NormalTok{d}\SpecialCharTok{$}\NormalTok{allowed }\OtherTok{\textless{}{-}} \FunctionTok{replace\_na}\NormalTok{(d}\SpecialCharTok{$}\NormalTok{allowed, }\StringTok{""}\NormalTok{)}

\NormalTok{title }\OtherTok{=} \FunctionTok{paste}\NormalTok{(}\StringTok{"Answer Key | Q : "}\NormalTok{, question)}
\NormalTok{cols }\OtherTok{=} \FunctionTok{c}\NormalTok{(}\StringTok{"interpretation"}\NormalTok{, }\StringTok{"answer"}\NormalTok{,}\StringTok{"not penalized"}\NormalTok{)}

\NormalTok{d }\SpecialCharTok{\%\textgreater{}\%} \FunctionTok{kbl}\NormalTok{(}\AttributeTok{caption =}\NormalTok{ title, }\AttributeTok{col.names =}\NormalTok{ cols) }\SpecialCharTok{\%\textgreater{}\%} \FunctionTok{kable\_classic}\NormalTok{() }\SpecialCharTok{\%\textgreater{}\%}
  \FunctionTok{footnote}\NormalTok{(}\AttributeTok{general =} \FunctionTok{paste}\NormalTok{(}\StringTok{"15 response options: "}\NormalTok{, options), }\AttributeTok{general\_title =} \StringTok{"Note: "}\NormalTok{,}\AttributeTok{footnote\_as\_chunk =}\NormalTok{ T)}
\end{Highlighting}
\end{Shaded}

\begin{table}

\caption{Answer Key | Q :  Which shift(s) are six hours long?}
\centering
\begin{tabular}[t]{l|l|l}
\hline
interpretation & answer & not penalized\\
\hline
Triangular & EG & \\
\hline
Orthgonal & EG & \\
\hline
Satisficing [left] &  & \\
\hline
Satisficing [right] &  & \\
\hline
Tversky [maximal] &  & \\
\hline
Tversky [start diagonal] &  & \\
\hline
Tversky [end diagonal] &  & \\
\hline
Tversky [duration line] &  & \\
\hline
\multicolumn{3}{l}{\rule{0pt}{1em}\textit{Note: } 15 response options:  ABCDEFGHIJKLMNOPZX}\\
\end{tabular}
\end{table}

TODO discuss non discriminant

\begin{Shaded}
\begin{Highlighting}[]
\NormalTok{title }\OtherTok{\textless{}{-}} \StringTok{"Frequency of Selected Response Options for Question \#6"}
\NormalTok{names }\OtherTok{=} \FunctionTok{c}\NormalTok{(}\StringTok{"response"}\NormalTok{,}\StringTok{"n"}\NormalTok{,}\StringTok{"interpretation"}\NormalTok{,}\StringTok{"absolute"}\NormalTok{,}\StringTok{"tri"}\NormalTok{,}\StringTok{"tversky"}\NormalTok{,}\StringTok{"satisfice"}\NormalTok{,}\StringTok{"orthogonal"}\NormalTok{, }\StringTok{"scaled score"}\NormalTok{)}

\NormalTok{df\_items }\SpecialCharTok{\%\textgreater{}\%} \FunctionTok{filter}\NormalTok{(q }\SpecialCharTok{==} \DecValTok{6}\NormalTok{) }\SpecialCharTok{\%\textgreater{}\%} \FunctionTok{group\_by}\NormalTok{(response) }\SpecialCharTok{\%\textgreater{}\%}
\NormalTok{  dplyr}\SpecialCharTok{::}\FunctionTok{summarise}\NormalTok{( }\AttributeTok{count =} \FunctionTok{n}\NormalTok{(),}
                    \AttributeTok{nice =} \FunctionTok{unique}\NormalTok{(score\_niceABS),}
                    \AttributeTok{triangular =} \FunctionTok{unique}\NormalTok{(score\_TRI),}
                    \AttributeTok{orthogonal =}  \FunctionTok{unique}\NormalTok{(score\_ORTH),}
                    \AttributeTok{satisficing =}  \FunctionTok{unique}\NormalTok{(score\_SATISFICE),}
                    \AttributeTok{tversky =} \FunctionTok{unique}\NormalTok{(score\_TVERSKY),}
                    \AttributeTok{interpretation =} \FunctionTok{unique}\NormalTok{(int2),}
                    \AttributeTok{scaled =} \FunctionTok{unique}\NormalTok{(score\_SCALED)) }\SpecialCharTok{\%\textgreater{}\%}
  \FunctionTok{arrange}\NormalTok{(interpretation, }\FunctionTok{desc}\NormalTok{(count)) }\SpecialCharTok{\%\textgreater{}\%}
  \FunctionTok{select}\NormalTok{(response, count, interpretation, nice,}
\NormalTok{         triangular, tversky, satisficing, orthogonal, scaled) }\SpecialCharTok{\%\textgreater{}\%}
  \FunctionTok{kbl}\NormalTok{(}\AttributeTok{caption =}\NormalTok{ title, }\AttributeTok{col.names =}\NormalTok{ names) }\SpecialCharTok{\%\textgreater{}\%}  \FunctionTok{kable\_classic}\NormalTok{() }\SpecialCharTok{\%\textgreater{}\%}
  \FunctionTok{add\_header\_above}\NormalTok{(}\FunctionTok{c}\NormalTok{(}\StringTok{" "} \OtherTok{=} \DecValTok{3}\NormalTok{, }\StringTok{"Strict Score"} \OtherTok{=} \DecValTok{1}\NormalTok{, }\StringTok{"Interpretation Scores"}\OtherTok{=}\DecValTok{4}\NormalTok{, }\StringTok{"Discriminant"}\OtherTok{=}\DecValTok{1}\NormalTok{)) }
\end{Highlighting}
\end{Shaded}

\textbackslash begin\{table\}

\textbackslash caption\{\label{tab:Q6-RESPONSES}Frequency of Selected
Response Options for Question \#6\} \centering

\begin{tabular}[t]{l|r|l|r|r|r|r|r|r}
\hline
\multicolumn{3}{c|}{ } & \multicolumn{1}{c|}{Strict Score} & \multicolumn{4}{c|}{Interpretation Scores} & \multicolumn{1}{c}{Discriminant} \\
\cline{4-4} \cline{5-8} \cline{9-9}
response & n & interpretation & absolute & tri & tversky & satisfice & orthogonal & scaled score\\
\hline
EG & 330 & both tri + orth & 1 & 1 & NA & NA & 1 & 0.5\\
\hline
\end{tabular}

\textbackslash end\{table\}

\begin{Shaded}
\begin{Highlighting}[]
\FunctionTok{gf\_dhistogram}\NormalTok{(}\SpecialCharTok{\textasciitilde{}}\NormalTok{ score\_niceABS, }\AttributeTok{fill =} \SpecialCharTok{\textasciitilde{}}\NormalTok{condition, }\AttributeTok{data =}\NormalTok{ df\_items }\SpecialCharTok{\%\textgreater{}\%} \FunctionTok{filter}\NormalTok{(q }\SpecialCharTok{==}\DecValTok{6}\NormalTok{)) }\SpecialCharTok{\%\textgreater{}\%} 
  \FunctionTok{gf\_facet\_grid}\NormalTok{( condition }\SpecialCharTok{\textasciitilde{}}\NormalTok{ ., }\AttributeTok{labeller =}\NormalTok{ label\_both) }\SpecialCharTok{+} 
  \FunctionTok{labs}\NormalTok{( }\AttributeTok{x =} \StringTok{"Scaled Item Score"}\NormalTok{, }\AttributeTok{title =} \StringTok{"Distribution of Scaled Scores | Q6 "}\NormalTok{) }\SpecialCharTok{+} 
  \FunctionTok{theme\_minimal}\NormalTok{() }\SpecialCharTok{+} \FunctionTok{theme}\NormalTok{(}\AttributeTok{legend.position =} \StringTok{"blank"}\NormalTok{)}
\end{Highlighting}
\end{Shaded}

\begin{figure}[H]

{\centering \includegraphics{analysis/SGC3A/2_sgc3A_scoring_files/figure-pdf/Q6-distribution-1.pdf}

}

\end{figure}

\begin{Shaded}
\begin{Highlighting}[]
\FunctionTok{gf\_props}\NormalTok{(}\SpecialCharTok{\textasciitilde{}}\NormalTok{interpretation, }\AttributeTok{fill =} \SpecialCharTok{\textasciitilde{}}\NormalTok{condition, }\AttributeTok{data =}\NormalTok{ df\_items }\SpecialCharTok{\%\textgreater{}\%} \FunctionTok{filter}\NormalTok{(q }\SpecialCharTok{==}\DecValTok{6}\NormalTok{)) }\SpecialCharTok{\%\textgreater{}\%} 
  \FunctionTok{gf\_facet\_grid}\NormalTok{( condition }\SpecialCharTok{\textasciitilde{}}\NormalTok{ ., }\AttributeTok{labeller =}\NormalTok{ label\_both) }\SpecialCharTok{+} 
  \FunctionTok{labs}\NormalTok{( }\AttributeTok{x =} \StringTok{"Interpretation"}\NormalTok{, }\AttributeTok{title =} \StringTok{"Distribution of Interpretations | Q6 "}\NormalTok{) }\SpecialCharTok{+} 
  \FunctionTok{theme\_minimal}\NormalTok{() }\SpecialCharTok{+} \FunctionTok{theme}\NormalTok{(}\AttributeTok{legend.position =} \StringTok{"blank"}\NormalTok{)}
\end{Highlighting}
\end{Shaded}

\begin{figure}[H]

{\centering \includegraphics{analysis/SGC3A/2_sgc3A_scoring_files/figure-pdf/Q6-distribution-2.pdf}

}

\end{figure}

\hypertarget{question-7}{%
\subsubsection{Question \#7}\label{question-7}}

\begin{figure}

{\centering \includegraphics{analysis/SGC3A/static/questions/Q7.png}

}

\caption{\label{fig-Q7}Q7-Question}

\end{figure}

\begin{Shaded}
\begin{Highlighting}[]
\NormalTok{q }\OtherTok{\textless{}{-}}\NormalTok{ keys\_raw }\SpecialCharTok{\%\textgreater{}\%} \FunctionTok{filter}\NormalTok{(Q}\SpecialCharTok{==}\DecValTok{7}\NormalTok{)}
\NormalTok{ignore }\OtherTok{\textless{}{-}}\NormalTok{ q }\SpecialCharTok{\%\textgreater{}\%} \FunctionTok{select}\NormalTok{(}\StringTok{"REF\_POINT"}\NormalTok{)}
\NormalTok{answers }\OtherTok{\textless{}{-}}\NormalTok{ q }\SpecialCharTok{\%\textgreater{}\%} \FunctionTok{select}\NormalTok{(}\StringTok{"TRIANGULAR"}\NormalTok{, }\StringTok{"ORTHOGONAL"}\NormalTok{, }\StringTok{"SATISFICE\_left"}\NormalTok{, }\StringTok{"SATISFICE\_right"}\NormalTok{,}\StringTok{"TV\_max"}\NormalTok{,}\StringTok{"TV\_start"}\NormalTok{, }\StringTok{"TV\_end"}\NormalTok{, }\StringTok{"TV\_dur"}\NormalTok{) }\SpecialCharTok{\%\textgreater{}\%} \FunctionTok{unlist}\NormalTok{()}
\NormalTok{ves }\OtherTok{\textless{}{-}}\NormalTok{ q }\SpecialCharTok{\%\textgreater{}\%} \FunctionTok{mutate}\NormalTok{(}
  \AttributeTok{SATISFICE\_left\_allow =} \StringTok{""}\NormalTok{,}
  \AttributeTok{SATISFICE\_right\_allow =} \StringTok{""}
\NormalTok{) }\SpecialCharTok{\%\textgreater{}\%} \FunctionTok{select}\NormalTok{(}\StringTok{"TRI\_allow"}\NormalTok{, }\StringTok{"ORTH\_allow"}\NormalTok{, }\StringTok{"SATISFICE\_left\_allow"}\NormalTok{,}\StringTok{"SATISFICE\_right\_allow"}\NormalTok{, }\StringTok{"TV\_max\_allow"}\NormalTok{,}\StringTok{"TV\_start\_allow"}\NormalTok{,}\StringTok{"TV\_end\_allow"}\NormalTok{, }\StringTok{"TV\_dur\_allow"}\NormalTok{)}\SpecialCharTok{\%\textgreater{}\%} \FunctionTok{unlist}\NormalTok{()}
\NormalTok{options }\OtherTok{\textless{}{-}}\NormalTok{ q }\SpecialCharTok{\%\textgreater{}\%} \FunctionTok{select}\NormalTok{(}\StringTok{"OPTIONS"}\NormalTok{)}
\NormalTok{question }\OtherTok{=}\NormalTok{ q }\SpecialCharTok{\%\textgreater{}\%}  \FunctionTok{select}\NormalTok{(}\StringTok{"TEXT"}\NormalTok{)}
\NormalTok{scores }\OtherTok{\textless{}{-}} \FunctionTok{c}\NormalTok{(}\StringTok{"Triangular"}\NormalTok{, }\StringTok{"Orthgonal"}\NormalTok{, }\StringTok{"Satisficing [left]"}\NormalTok{, }\StringTok{"Satisficing [right]"}\NormalTok{, }\StringTok{"Tversky [maximal]"}\NormalTok{, }\StringTok{"Tversky [start diagonal]"}\NormalTok{,}
            \StringTok{"Tversky [end diagonal]"}\NormalTok{, }\StringTok{"Tversky [duration line]"}\NormalTok{)}
\NormalTok{d }\OtherTok{=} \FunctionTok{tibble}\NormalTok{(}\AttributeTok{interpretation =}\NormalTok{ scores, }\AttributeTok{answer =}\NormalTok{ answers, }\AttributeTok{allowed=}\NormalTok{ves)}
\NormalTok{d}\SpecialCharTok{$}\NormalTok{answer }\OtherTok{\textless{}{-}} \FunctionTok{replace\_na}\NormalTok{(d}\SpecialCharTok{$}\NormalTok{answer, }\StringTok{""}\NormalTok{)}
\NormalTok{d}\SpecialCharTok{$}\NormalTok{allowed }\OtherTok{\textless{}{-}} \FunctionTok{replace\_na}\NormalTok{(d}\SpecialCharTok{$}\NormalTok{allowed, }\StringTok{""}\NormalTok{)}

\NormalTok{title }\OtherTok{=} \FunctionTok{paste}\NormalTok{(}\StringTok{"Answer Key | Q : "}\NormalTok{, question)}
\NormalTok{cols }\OtherTok{=} \FunctionTok{c}\NormalTok{(}\StringTok{"interpretation"}\NormalTok{, }\StringTok{"answer"}\NormalTok{,}\StringTok{"not penalized"}\NormalTok{)}

\NormalTok{d }\SpecialCharTok{\%\textgreater{}\%} \FunctionTok{kbl}\NormalTok{(}\AttributeTok{caption =}\NormalTok{ title, }\AttributeTok{col.names =}\NormalTok{ cols) }\SpecialCharTok{\%\textgreater{}\%} \FunctionTok{kable\_classic}\NormalTok{() }\SpecialCharTok{\%\textgreater{}\%}
  \FunctionTok{footnote}\NormalTok{(}\AttributeTok{general =} \FunctionTok{paste}\NormalTok{(}\StringTok{"15 response options: "}\NormalTok{, options), }\AttributeTok{general\_title =} \StringTok{"Note: "}\NormalTok{,}\AttributeTok{footnote\_as\_chunk =}\NormalTok{ T)}
\end{Highlighting}
\end{Shaded}

\begin{table}

\caption{Answer Key | Q :  Which 2 shifts less than 5 hours long start at the same time?}
\centering
\begin{tabular}[t]{l|l|l}
\hline
interpretation & answer & not penalized\\
\hline
Triangular & OX & \\
\hline
Orthgonal & FB & M\\
\hline
Satisficing [left] &  & \\
\hline
Satisficing [right] &  & \\
\hline
Tversky [maximal] & IJZNCHOX & \\
\hline
Tversky [start diagonal] & OX & \\
\hline
Tversky [end diagonal] & IJZN & \\
\hline
Tversky [duration line] & CH & \\
\hline
\multicolumn{3}{l}{\rule{0pt}{1em}\textit{Note: } 15 response options:  ABCDEFGHIJKLMNOPZX}\\
\end{tabular}
\end{table}

\begin{Shaded}
\begin{Highlighting}[]
\NormalTok{title }\OtherTok{\textless{}{-}} \StringTok{"Frequency of Selected Response Options for Question \#7"}
\NormalTok{names }\OtherTok{=} \FunctionTok{c}\NormalTok{(}\StringTok{"response"}\NormalTok{,}\StringTok{"n"}\NormalTok{,}\StringTok{"interpretation"}\NormalTok{,}\StringTok{"absolute"}\NormalTok{,}\StringTok{"tri"}\NormalTok{,}\StringTok{"tversky"}\NormalTok{,}\StringTok{"satisfice"}\NormalTok{,}\StringTok{"orthogonal"}\NormalTok{, }\StringTok{"scaled score"}\NormalTok{)}

\NormalTok{df\_items }\SpecialCharTok{\%\textgreater{}\%} \FunctionTok{filter}\NormalTok{(q }\SpecialCharTok{==} \DecValTok{7}\NormalTok{) }\SpecialCharTok{\%\textgreater{}\%} \FunctionTok{group\_by}\NormalTok{(response) }\SpecialCharTok{\%\textgreater{}\%}
\NormalTok{  dplyr}\SpecialCharTok{::}\FunctionTok{summarise}\NormalTok{( }\AttributeTok{count =} \FunctionTok{n}\NormalTok{(),}
                    \AttributeTok{nice =} \FunctionTok{unique}\NormalTok{(score\_niceABS),}
                    \AttributeTok{triangular =} \FunctionTok{unique}\NormalTok{(score\_TRI),}
                    \AttributeTok{orthogonal =}  \FunctionTok{unique}\NormalTok{(score\_ORTH),}
                    \AttributeTok{satisficing =}  \FunctionTok{unique}\NormalTok{(score\_SATISFICE),}
                    \AttributeTok{tversky =} \FunctionTok{unique}\NormalTok{(score\_TVERSKY),}
                    \AttributeTok{interpretation =} \FunctionTok{unique}\NormalTok{(int2),}
                    \AttributeTok{scaled =} \FunctionTok{unique}\NormalTok{(score\_SCALED)) }\SpecialCharTok{\%\textgreater{}\%}
  \FunctionTok{arrange}\NormalTok{(interpretation, }\FunctionTok{desc}\NormalTok{(count)) }\SpecialCharTok{\%\textgreater{}\%}
  \FunctionTok{select}\NormalTok{(response, count, interpretation, nice,}
\NormalTok{         triangular, tversky, satisficing, orthogonal, scaled) }\SpecialCharTok{\%\textgreater{}\%}
  \FunctionTok{kbl}\NormalTok{(}\AttributeTok{caption =}\NormalTok{ title, }\AttributeTok{col.names =}\NormalTok{ names) }\SpecialCharTok{\%\textgreater{}\%}  \FunctionTok{kable\_classic}\NormalTok{() }\SpecialCharTok{\%\textgreater{}\%}
  \FunctionTok{add\_header\_above}\NormalTok{(}\FunctionTok{c}\NormalTok{(}\StringTok{" "} \OtherTok{=} \DecValTok{3}\NormalTok{, }\StringTok{"Strict Score"} \OtherTok{=} \DecValTok{1}\NormalTok{, }\StringTok{"Interpretation Scores"}\OtherTok{=}\DecValTok{4}\NormalTok{, }\StringTok{"Discriminant"}\OtherTok{=}\DecValTok{1}\NormalTok{)) }\SpecialCharTok{\%\textgreater{}\%}
  \FunctionTok{pack\_rows}\NormalTok{(}\StringTok{"Triangular"}\NormalTok{, }\DecValTok{1}\NormalTok{, }\DecValTok{5}\NormalTok{) }\SpecialCharTok{\%\textgreater{}\%}
  \FunctionTok{pack\_rows}\NormalTok{(}\StringTok{"Lines{-}Connect"}\NormalTok{, }\DecValTok{6}\NormalTok{, }\DecValTok{9}\NormalTok{) }\SpecialCharTok{\%\textgreater{}\%}
  \FunctionTok{pack\_rows}\NormalTok{(}\StringTok{"Orthogonal"}\NormalTok{, }\DecValTok{10}\NormalTok{, }\DecValTok{13}\NormalTok{) }\SpecialCharTok{\%\textgreater{}\%}
  \FunctionTok{pack\_rows}\NormalTok{(}\StringTok{"Other"}\NormalTok{, }\DecValTok{14}\NormalTok{, }\DecValTok{14}\NormalTok{) }\SpecialCharTok{\%\textgreater{}\%}
  \FunctionTok{pack\_rows}\NormalTok{(}\StringTok{"Unknown"}\NormalTok{, }\DecValTok{15}\NormalTok{, }\DecValTok{17}\NormalTok{)}
\end{Highlighting}
\end{Shaded}

\textbackslash begin\{table\}

\textbackslash caption\{\label{tab:Q7-RESPONSES}Frequency of Selected
Response Options for Question \#7\} \centering

\begin{tabular}[t]{l|r|l|r|r|r|r|r|r}
\hline
\multicolumn{3}{c|}{ } & \multicolumn{1}{c|}{Strict Score} & \multicolumn{4}{c|}{Interpretation Scores} & \multicolumn{1}{c}{Discriminant} \\
\cline{4-4} \cline{5-8} \cline{9-9}
response & n & interpretation & absolute & tri & tversky & satisfice & orthogonal & scaled score\\
\hline
\multicolumn{9}{l}{\textbf{Triangular}}\\
\hline
\hspace{1em}OX & 93 & Triangular & 1 & 1.000 & 1.000 & NA & -0.133 & 1.0\\
\hline
\hspace{1em}MO & 2 & Triangular & 0 & 0.438 & 0.438 & NA & -0.067 & 1.0\\
\hline
\hspace{1em}AX & 1 & Triangular & 0 & 0.438 & 0.438 & NA & -0.133 & 1.0\\
\hline
\hspace{1em}MOX & 1 & Triangular & 0 & 0.938 & 0.938 & NA & -0.133 & 1.0\\
\hline
\hspace{1em}MX & 1 & Triangular & 0 & 0.438 & 0.438 & NA & -0.067 & 1.0\\
\hline
\multicolumn{9}{l}{\textbf{Lines-Connect}}\\
\hline
\hspace{1em}IJ & 3 & Tversky & 0 & -0.125 & 0.500 & NA & -0.133 & 0.5\\
\hline
\hspace{1em}CH & 1 & Tversky & 0 & -0.125 & 1.000 & NA & -0.133 & 0.5\\
\hline
\hspace{1em}DJNX & 1 & Tversky & 0 & 0.312 & 0.357 & NA & -0.267 & 0.5\\
\hline
\hspace{1em}HK & 1 & Tversky & 0 & -0.125 & 0.438 & NA & -0.133 & 0.5\\
\hline
\multicolumn{9}{l}{\textbf{Orthogonal}}\\
\hline
\hspace{1em}BF & 203 & Orthogonal & 0 & -0.125 & -0.125 & NA & 1.000 & -1.0\\
\hline
\hspace{1em}FZ & 16 & Orthogonal & 0 & -0.125 & 0.179 & NA & 0.433 & -1.0\\
\hline
\hspace{1em}B & 1 & Orthogonal & 0 & -0.062 & -0.062 & NA & 0.500 & -1.0\\
\hline
\hspace{1em}F & 1 & Orthogonal & 0 & -0.062 & -0.062 & NA & 0.500 & -1.0\\
\hline
\multicolumn{9}{l}{\textbf{Other}}\\
\hline
\hspace{1em} & 2 & blank & 0 & 0.000 & NA & NA & 0.000 & 0.0\\
\hline
\multicolumn{9}{l}{\textbf{Unknown}}\\
\hline
\hspace{1em}GK & 1 & ? & 0 & -0.125 & -0.125 & NA & -0.133 & -0.5\\
\hline
\hspace{1em}JM & 1 & ? & 0 & -0.125 & 0.179 & NA & -0.067 & -0.5\\
\hline
\hspace{1em}KM & 1 & ? & 0 & -0.125 & -0.125 & NA & -0.067 & -0.5\\
\hline
\end{tabular}

\textbackslash end\{table\}

\begin{Shaded}
\begin{Highlighting}[]
\FunctionTok{gf\_dhistogram}\NormalTok{(}\SpecialCharTok{\textasciitilde{}}\NormalTok{ score\_niceABS, }\AttributeTok{fill =} \SpecialCharTok{\textasciitilde{}}\NormalTok{condition, }\AttributeTok{data =}\NormalTok{ df\_items }\SpecialCharTok{\%\textgreater{}\%} \FunctionTok{filter}\NormalTok{(q }\SpecialCharTok{==} \DecValTok{7}\NormalTok{)) }\SpecialCharTok{\%\textgreater{}\%} 
  \FunctionTok{gf\_facet\_grid}\NormalTok{( condition }\SpecialCharTok{\textasciitilde{}}\NormalTok{ ., }\AttributeTok{labeller =}\NormalTok{ label\_both) }\SpecialCharTok{+} 
  \FunctionTok{labs}\NormalTok{( }\AttributeTok{x =} \StringTok{"Scaled Item Score"}\NormalTok{, }\AttributeTok{title =} \StringTok{"Distribution of Scaled Scores | Q7 "}\NormalTok{) }\SpecialCharTok{+} 
  \FunctionTok{theme\_minimal}\NormalTok{() }\SpecialCharTok{+} \FunctionTok{theme}\NormalTok{(}\AttributeTok{legend.position =} \StringTok{"blank"}\NormalTok{)}
\end{Highlighting}
\end{Shaded}

\begin{figure}[H]

{\centering \includegraphics{analysis/SGC3A/2_sgc3A_scoring_files/figure-pdf/Q7-distribution-1.pdf}

}

\end{figure}

\begin{Shaded}
\begin{Highlighting}[]
\FunctionTok{gf\_props}\NormalTok{(}\SpecialCharTok{\textasciitilde{}}\NormalTok{interpretation, }\AttributeTok{fill =} \SpecialCharTok{\textasciitilde{}}\NormalTok{condition, }\AttributeTok{data =}\NormalTok{ df\_items }\SpecialCharTok{\%\textgreater{}\%} \FunctionTok{filter}\NormalTok{(q }\SpecialCharTok{==} \DecValTok{7}\NormalTok{)) }\SpecialCharTok{\%\textgreater{}\%} 
  \FunctionTok{gf\_facet\_grid}\NormalTok{( condition }\SpecialCharTok{\textasciitilde{}}\NormalTok{ ., }\AttributeTok{labeller =}\NormalTok{ label\_both) }\SpecialCharTok{+} 
  \FunctionTok{labs}\NormalTok{( }\AttributeTok{x =} \StringTok{"Interpretation"}\NormalTok{, }\AttributeTok{title =} \StringTok{"Distribution of Interpretations | Q7 "}\NormalTok{) }\SpecialCharTok{+} 
  \FunctionTok{theme\_minimal}\NormalTok{() }\SpecialCharTok{+} \FunctionTok{theme}\NormalTok{(}\AttributeTok{legend.position =} \StringTok{"blank"}\NormalTok{)}
\end{Highlighting}
\end{Shaded}

\begin{figure}[H]

{\centering \includegraphics{analysis/SGC3A/2_sgc3A_scoring_files/figure-pdf/Q7-distribution-2.pdf}

}

\end{figure}

\hypertarget{question-8}{%
\subsubsection{Question \#8}\label{question-8}}

\begin{figure}

{\centering \includegraphics{analysis/SGC3A/static/questions/Q8.png}

}

\caption{\label{fig-Q8}Q8-Question}

\end{figure}

\begin{Shaded}
\begin{Highlighting}[]
\NormalTok{q }\OtherTok{\textless{}{-}}\NormalTok{ keys\_raw }\SpecialCharTok{\%\textgreater{}\%} \FunctionTok{filter}\NormalTok{(Q}\SpecialCharTok{==}\DecValTok{8}\NormalTok{)}
\NormalTok{ignore }\OtherTok{\textless{}{-}}\NormalTok{ q }\SpecialCharTok{\%\textgreater{}\%} \FunctionTok{select}\NormalTok{(}\StringTok{"REF\_POINT"}\NormalTok{)}
\NormalTok{answers }\OtherTok{\textless{}{-}}\NormalTok{ q }\SpecialCharTok{\%\textgreater{}\%} \FunctionTok{select}\NormalTok{(}\StringTok{"TRIANGULAR"}\NormalTok{, }\StringTok{"ORTHOGONAL"}\NormalTok{, }\StringTok{"SATISFICE\_left"}\NormalTok{, }\StringTok{"SATISFICE\_right"}\NormalTok{,}\StringTok{"TV\_max"}\NormalTok{,}\StringTok{"TV\_start"}\NormalTok{, }\StringTok{"TV\_end"}\NormalTok{, }\StringTok{"TV\_dur"}\NormalTok{) }\SpecialCharTok{\%\textgreater{}\%} \FunctionTok{unlist}\NormalTok{()}
\NormalTok{ves }\OtherTok{\textless{}{-}}\NormalTok{ q }\SpecialCharTok{\%\textgreater{}\%} \FunctionTok{mutate}\NormalTok{(}
  \AttributeTok{SATISFICE\_left\_allow =} \StringTok{""}\NormalTok{,}
  \AttributeTok{SATISFICE\_right\_allow =} \StringTok{""}
\NormalTok{) }\SpecialCharTok{\%\textgreater{}\%} \FunctionTok{select}\NormalTok{(}\StringTok{"TRI\_allow"}\NormalTok{, }\StringTok{"ORTH\_allow"}\NormalTok{, }\StringTok{"SATISFICE\_left\_allow"}\NormalTok{,}\StringTok{"SATISFICE\_right\_allow"}\NormalTok{, }\StringTok{"TV\_max\_allow"}\NormalTok{,}\StringTok{"TV\_start\_allow"}\NormalTok{,}\StringTok{"TV\_end\_allow"}\NormalTok{, }\StringTok{"TV\_dur\_allow"}\NormalTok{)}\SpecialCharTok{\%\textgreater{}\%} \FunctionTok{unlist}\NormalTok{()}
\NormalTok{options }\OtherTok{\textless{}{-}}\NormalTok{ q }\SpecialCharTok{\%\textgreater{}\%} \FunctionTok{select}\NormalTok{(}\StringTok{"OPTIONS"}\NormalTok{)}
\NormalTok{question }\OtherTok{=}\NormalTok{ q }\SpecialCharTok{\%\textgreater{}\%}  \FunctionTok{select}\NormalTok{(}\StringTok{"TEXT"}\NormalTok{)}
\NormalTok{scores }\OtherTok{\textless{}{-}} \FunctionTok{c}\NormalTok{(}\StringTok{"Triangular"}\NormalTok{, }\StringTok{"Orthgonal"}\NormalTok{, }\StringTok{"Satisficing [left]"}\NormalTok{, }\StringTok{"Satisficing [right]"}\NormalTok{, }\StringTok{"Tversky [maximal]"}\NormalTok{, }\StringTok{"Tversky [start diagonal]"}\NormalTok{,}
            \StringTok{"Tversky [end diagonal]"}\NormalTok{, }\StringTok{"Tversky [duration line]"}\NormalTok{)}
\NormalTok{d }\OtherTok{=} \FunctionTok{tibble}\NormalTok{(}\AttributeTok{interpretation =}\NormalTok{ scores, }\AttributeTok{answer =}\NormalTok{ answers, }\AttributeTok{allowed=}\NormalTok{ves)}
\NormalTok{d}\SpecialCharTok{$}\NormalTok{answer }\OtherTok{\textless{}{-}} \FunctionTok{replace\_na}\NormalTok{(d}\SpecialCharTok{$}\NormalTok{answer, }\StringTok{""}\NormalTok{)}
\NormalTok{d}\SpecialCharTok{$}\NormalTok{allowed }\OtherTok{\textless{}{-}} \FunctionTok{replace\_na}\NormalTok{(d}\SpecialCharTok{$}\NormalTok{allowed, }\StringTok{""}\NormalTok{)}

\NormalTok{title }\OtherTok{=} \FunctionTok{paste}\NormalTok{(}\StringTok{"Answer Key | Q: "}\NormalTok{, question)}
\NormalTok{cols }\OtherTok{=} \FunctionTok{c}\NormalTok{(}\StringTok{"interpretation"}\NormalTok{, }\StringTok{"answer"}\NormalTok{,}\StringTok{"not penalized"}\NormalTok{)}

\NormalTok{d }\SpecialCharTok{\%\textgreater{}\%} \FunctionTok{kbl}\NormalTok{(}\AttributeTok{caption =}\NormalTok{ title, }\AttributeTok{col.names =}\NormalTok{ cols) }\SpecialCharTok{\%\textgreater{}\%} \FunctionTok{kable\_classic}\NormalTok{() }\SpecialCharTok{\%\textgreater{}\%}
  \FunctionTok{footnote}\NormalTok{(}\AttributeTok{general =} \FunctionTok{paste}\NormalTok{(}\StringTok{"15 response options: "}\NormalTok{, options), }\AttributeTok{general\_title =} \StringTok{"Note: "}\NormalTok{,}\AttributeTok{footnote\_as\_chunk =}\NormalTok{ T)}
\end{Highlighting}
\end{Shaded}

\begin{table}

\caption{Answer Key | Q:  Which shift(s) under 7 hours long starts before B starts, and ends after X ends?}
\centering
\begin{tabular}[t]{l|l|l}
\hline
interpretation & answer & not penalized\\
\hline
Triangular & G & \\
\hline
Orthgonal & E & \\
\hline
Satisficing [left] &  & \\
\hline
Satisficing [right] &  & \\
\hline
Tversky [maximal] &  & \\
\hline
Tversky [start diagonal] &  & \\
\hline
Tversky [end diagonal] &  & \\
\hline
Tversky [duration line] &  & \\
\hline
\multicolumn{3}{l}{\rule{0pt}{1em}\textit{Note: } 15 response options:  ABCDEFGHIJKLMNOPZX}\\
\end{tabular}
\end{table}

\begin{Shaded}
\begin{Highlighting}[]
\NormalTok{title }\OtherTok{\textless{}{-}} \StringTok{"Frequency of Selected Response Options for Question \#8"}
\NormalTok{names }\OtherTok{=} \FunctionTok{c}\NormalTok{(}\StringTok{"response"}\NormalTok{,}\StringTok{"n"}\NormalTok{,}\StringTok{"interpretation"}\NormalTok{,}\StringTok{"absolute"}\NormalTok{,}\StringTok{"tri"}\NormalTok{,}\StringTok{"tversky"}\NormalTok{,}\StringTok{"satisfice"}\NormalTok{,}\StringTok{"orthogonal"}\NormalTok{, }\StringTok{"scaled score"}\NormalTok{)}

\NormalTok{df\_items }\SpecialCharTok{\%\textgreater{}\%} \FunctionTok{filter}\NormalTok{(q }\SpecialCharTok{==} \DecValTok{8}\NormalTok{) }\SpecialCharTok{\%\textgreater{}\%} \FunctionTok{group\_by}\NormalTok{(response) }\SpecialCharTok{\%\textgreater{}\%}
\NormalTok{  dplyr}\SpecialCharTok{::}\FunctionTok{summarise}\NormalTok{( }\AttributeTok{count =} \FunctionTok{n}\NormalTok{(),}
                    \AttributeTok{nice =} \FunctionTok{unique}\NormalTok{(score\_niceABS),}
                    \AttributeTok{triangular =} \FunctionTok{unique}\NormalTok{(score\_TRI),}
                    \AttributeTok{orthogonal =}  \FunctionTok{unique}\NormalTok{(score\_ORTH),}
                    \AttributeTok{satisficing =}  \FunctionTok{unique}\NormalTok{(score\_SATISFICE),}
                    \AttributeTok{tversky =} \FunctionTok{unique}\NormalTok{(score\_TVERSKY),}
                    \AttributeTok{interpretation =} \FunctionTok{unique}\NormalTok{(int2),}
                    \AttributeTok{scaled =} \FunctionTok{unique}\NormalTok{(score\_SCALED)) }\SpecialCharTok{\%\textgreater{}\%}
  \FunctionTok{arrange}\NormalTok{(interpretation, }\FunctionTok{desc}\NormalTok{(count)) }\SpecialCharTok{\%\textgreater{}\%}
  \FunctionTok{select}\NormalTok{(response, count, interpretation, nice,}
\NormalTok{         triangular, tversky, satisficing, orthogonal, scaled) }\SpecialCharTok{\%\textgreater{}\%}
  \FunctionTok{kbl}\NormalTok{(}\AttributeTok{caption =}\NormalTok{ title, }\AttributeTok{col.names =}\NormalTok{ names) }\SpecialCharTok{\%\textgreater{}\%}  \FunctionTok{kable\_classic}\NormalTok{() }\SpecialCharTok{\%\textgreater{}\%}
  \FunctionTok{add\_header\_above}\NormalTok{(}\FunctionTok{c}\NormalTok{(}\StringTok{" "} \OtherTok{=} \DecValTok{3}\NormalTok{, }\StringTok{"Strict Score"} \OtherTok{=} \DecValTok{1}\NormalTok{, }\StringTok{"Interpretation Scores"}\OtherTok{=}\DecValTok{4}\NormalTok{, }\StringTok{"Discriminant"}\OtherTok{=}\DecValTok{1}\NormalTok{)) }\SpecialCharTok{\%\textgreater{}\%}
  \FunctionTok{pack\_rows}\NormalTok{(}\StringTok{"Triangular"}\NormalTok{, }\DecValTok{1}\NormalTok{, }\DecValTok{10}\NormalTok{) }\SpecialCharTok{\%\textgreater{}\%}
  \FunctionTok{pack\_rows}\NormalTok{(}\StringTok{"Orthogonal"}\NormalTok{, }\DecValTok{11}\NormalTok{, }\DecValTok{16}\NormalTok{) }\SpecialCharTok{\%\textgreater{}\%}
  \FunctionTok{pack\_rows}\NormalTok{(}\StringTok{"Other"}\NormalTok{, }\DecValTok{17}\NormalTok{, }\DecValTok{21}\NormalTok{) }\SpecialCharTok{\%\textgreater{}\%}
  \FunctionTok{pack\_rows}\NormalTok{(}\StringTok{"Unknown"}\NormalTok{, }\DecValTok{22}\NormalTok{, }\DecValTok{45}\NormalTok{)}
\end{Highlighting}
\end{Shaded}

\textbackslash begin\{table\}

\textbackslash caption\{\label{tab:Q8-RESPONSES}Frequency of Selected
Response Options for Question \#8\} \centering

\begin{tabular}[t]{l|r|l|r|r|r|r|r|r}
\hline
\multicolumn{3}{c|}{ } & \multicolumn{1}{c|}{Strict Score} & \multicolumn{4}{c|}{Interpretation Scores} & \multicolumn{1}{c}{Discriminant} \\
\cline{4-4} \cline{5-8} \cline{9-9}
response & n & interpretation & absolute & tri & tversky & satisfice & orthogonal & scaled score\\
\hline
\multicolumn{9}{l}{\textbf{Triangular}}\\
\hline
\hspace{1em}G & 64 & Triangular & 1 & 1.000 & NA & NA & -0.067 & 1.0\\
\hline
\hspace{1em}AGK & 4 & Triangular & 0 & 0.867 & NA & NA & -0.200 & 1.0\\
\hline
\hspace{1em}CG & 3 & Triangular & 0 & 0.933 & NA & NA & -0.133 & 1.0\\
\hline
\hspace{1em}FG & 3 & Triangular & 0 & 0.933 & NA & NA & -0.133 & 1.0\\
\hline
\hspace{1em}AG & 2 & Triangular & 0 & 0.933 & NA & NA & -0.133 & 1.0\\
\hline
\hspace{1em}CFGO & 2 & Triangular & 0 & 0.800 & NA & NA & -0.267 & 1.0\\
\hline
\hspace{1em}ACGP & 1 & Triangular & 0 & 0.800 & NA & NA & -0.267 & 1.0\\
\hline
\hspace{1em}CFG & 1 & Triangular & 0 & 0.867 & NA & NA & -0.200 & 1.0\\
\hline
\hspace{1em}CGM & 1 & Triangular & 0 & 0.867 & NA & NA & -0.200 & 1.0\\
\hline
\hspace{1em}GM & 1 & Triangular & 0 & 0.933 & NA & NA & -0.133 & 1.0\\
\hline
\multicolumn{9}{l}{\textbf{Orthogonal}}\\
\hline
\hspace{1em}E & 157 & Orthogonal & 0 & -0.067 & NA & NA & 1.000 & -1.0\\
\hline
\hspace{1em}EIJ & 5 & Orthogonal & 0 & -0.200 & NA & NA & 0.867 & -1.0\\
\hline
\hspace{1em}EFIJ & 3 & Orthogonal & 0 & -0.267 & NA & NA & 0.800 & -1.0\\
\hline
\hspace{1em}EF & 2 & Orthogonal & 0 & -0.133 & NA & NA & 0.933 & -1.0\\
\hline
\hspace{1em}EI & 2 & Orthogonal & 0 & -0.133 & NA & NA & 0.933 & -1.0\\
\hline
\hspace{1em}EFI & 1 & Orthogonal & 0 & -0.200 & NA & NA & 0.867 & -1.0\\
\hline
\multicolumn{9}{l}{\textbf{Other}}\\
\hline
\hspace{1em} & 12 & blank & 0 & 0.000 & NA & NA & 0.000 & 0.0\\
\hline
\hspace{1em}DEHIJNOZ & 2 & frenzy & 0 & -0.533 & NA & NA & 0.533 & -0.5\\
\hline
\hspace{1em}EFGIJ & 2 & frenzy & 0 & 0.733 & NA & NA & 0.733 & -0.5\\
\hline
\hspace{1em}CDGHLNOXZ & 1 & frenzy & 0 & 0.533 & NA & NA & -0.533 & -0.5\\
\hline
\hspace{1em}DEIJN & 1 & frenzy & 0 & -0.333 & NA & NA & 0.733 & -0.5\\
\hline
\multicolumn{9}{l}{\textbf{Unknown}}\\
\hline
\hspace{1em}IJ & 17 & ? & 0 & -0.133 & NA & NA & -0.133 & -0.5\\
\hline
\hspace{1em}I & 7 & ? & 0 & -0.067 & NA & NA & -0.067 & -0.5\\
\hline
\hspace{1em}EFG & 3 & ? & 0 & 0.867 & NA & NA & 0.867 & -0.5\\
\hline
\hspace{1em}J & 3 & ? & 0 & -0.067 & NA & NA & -0.067 & -0.5\\
\hline
\hspace{1em}O & 3 & ? & 0 & -0.067 & NA & NA & -0.067 & -0.5\\
\hline
\hspace{1em}A & 2 & ? & 0 & -0.067 & NA & NA & -0.067 & -0.5\\
\hline
\hspace{1em}AK & 2 & ? & 0 & -0.133 & NA & NA & -0.133 & -0.5\\
\hline
\hspace{1em}C & 2 & ? & 0 & -0.067 & NA & NA & -0.067 & -0.5\\
\hline
\hspace{1em}DN & 2 & ? & 0 & -0.133 & NA & NA & -0.133 & -0.5\\
\hline
\hspace{1em}F & 2 & ? & 0 & -0.067 & NA & NA & -0.067 & -0.5\\
\hline
\hspace{1em}IJM & 2 & ? & 0 & -0.200 & NA & NA & -0.200 & -0.5\\
\hline
\hspace{1em}L & 2 & ? & 0 & -0.067 & NA & NA & -0.067 & -0.5\\
\hline
\hspace{1em}M & 2 & ? & 0 & -0.067 & NA & NA & -0.067 & -0.5\\
\hline
\hspace{1em}CM & 1 & ? & 0 & -0.133 & NA & NA & -0.133 & -0.5\\
\hline
\hspace{1em}CX & 1 & ? & 0 & -0.067 & NA & NA & -0.067 & -0.5\\
\hline
\hspace{1em}D & 1 & ? & 0 & -0.067 & NA & NA & -0.067 & -0.5\\
\hline
\hspace{1em}DHNZ & 1 & ? & 0 & -0.267 & NA & NA & -0.267 & -0.5\\
\hline
\hspace{1em}DIJN & 1 & ? & 0 & -0.267 & NA & NA & -0.267 & -0.5\\
\hline
\hspace{1em}EFGI & 1 & ? & 0 & 0.800 & NA & NA & 0.800 & -0.5\\
\hline
\hspace{1em}HLO & 1 & ? & 0 & -0.200 & NA & NA & -0.200 & -0.5\\
\hline
\hspace{1em}IO & 1 & ? & 0 & -0.133 & NA & NA & -0.133 & -0.5\\
\hline
\hspace{1em}JM & 1 & ? & 0 & -0.133 & NA & NA & -0.133 & -0.5\\
\hline
\hspace{1em}KL & 1 & ? & 0 & -0.133 & NA & NA & -0.133 & -0.5\\
\hline
\hspace{1em}N & 1 & ? & 0 & -0.067 & NA & NA & -0.067 & -0.5\\
\hline
\end{tabular}

\textbackslash end\{table\}

\begin{Shaded}
\begin{Highlighting}[]
\FunctionTok{gf\_dhistogram}\NormalTok{(}\SpecialCharTok{\textasciitilde{}}\NormalTok{ score\_niceABS, }\AttributeTok{fill =} \SpecialCharTok{\textasciitilde{}}\NormalTok{condition, }\AttributeTok{data =}\NormalTok{ df\_items }\SpecialCharTok{\%\textgreater{}\%} \FunctionTok{filter}\NormalTok{(q }\SpecialCharTok{==} \DecValTok{8}\NormalTok{)) }\SpecialCharTok{\%\textgreater{}\%} 
  \FunctionTok{gf\_facet\_grid}\NormalTok{( condition }\SpecialCharTok{\textasciitilde{}}\NormalTok{ ., }\AttributeTok{labeller =}\NormalTok{ label\_both) }\SpecialCharTok{+} 
  \FunctionTok{labs}\NormalTok{( }\AttributeTok{x =} \StringTok{"Scaled Item Score"}\NormalTok{, }\AttributeTok{title =} \StringTok{"Distribution of Scaled Scores | Q8 "}\NormalTok{) }\SpecialCharTok{+} 
  \FunctionTok{theme\_minimal}\NormalTok{() }\SpecialCharTok{+} \FunctionTok{theme}\NormalTok{(}\AttributeTok{legend.position =} \StringTok{"blank"}\NormalTok{)}
\end{Highlighting}
\end{Shaded}

\begin{figure}[H]

{\centering \includegraphics{analysis/SGC3A/2_sgc3A_scoring_files/figure-pdf/Q8-distribution-1.pdf}

}

\end{figure}

\begin{Shaded}
\begin{Highlighting}[]
\FunctionTok{gf\_props}\NormalTok{(}\SpecialCharTok{\textasciitilde{}}\NormalTok{interpretation, }\AttributeTok{fill =} \SpecialCharTok{\textasciitilde{}}\NormalTok{condition, }\AttributeTok{data =}\NormalTok{ df\_items }\SpecialCharTok{\%\textgreater{}\%} \FunctionTok{filter}\NormalTok{(q }\SpecialCharTok{==} \DecValTok{8}\NormalTok{)) }\SpecialCharTok{\%\textgreater{}\%} 
  \FunctionTok{gf\_facet\_grid}\NormalTok{( condition }\SpecialCharTok{\textasciitilde{}}\NormalTok{ ., }\AttributeTok{labeller =}\NormalTok{ label\_both) }\SpecialCharTok{+} 
  \FunctionTok{labs}\NormalTok{( }\AttributeTok{x =} \StringTok{"Interpretation"}\NormalTok{, }\AttributeTok{title =} \StringTok{"Distribution of Interpretations | Q8 "}\NormalTok{) }\SpecialCharTok{+} 
  \FunctionTok{theme\_minimal}\NormalTok{() }\SpecialCharTok{+} \FunctionTok{theme}\NormalTok{(}\AttributeTok{legend.position =} \StringTok{"blank"}\NormalTok{)}
\end{Highlighting}
\end{Shaded}

\begin{figure}[H]

{\centering \includegraphics{analysis/SGC3A/2_sgc3A_scoring_files/figure-pdf/Q8-distribution-2.pdf}

}

\end{figure}

\hypertarget{question-9-nondiscrim}{%
\subsubsection{Question \#9 NONDISCRIM}\label{question-9-nondiscrim}}

\begin{figure}

{\centering \includegraphics{analysis/SGC3A/static/questions/Q9.png}

}

\caption{\label{fig-Q9}Q9-Question}

\end{figure}

\begin{Shaded}
\begin{Highlighting}[]
\NormalTok{q }\OtherTok{\textless{}{-}}\NormalTok{ keys\_raw }\SpecialCharTok{\%\textgreater{}\%} \FunctionTok{filter}\NormalTok{(Q}\SpecialCharTok{==}\DecValTok{9}\NormalTok{)}
\NormalTok{ignore }\OtherTok{\textless{}{-}}\NormalTok{ q }\SpecialCharTok{\%\textgreater{}\%} \FunctionTok{select}\NormalTok{(}\StringTok{"REF\_POINT"}\NormalTok{)}
\NormalTok{answers }\OtherTok{\textless{}{-}}\NormalTok{ q }\SpecialCharTok{\%\textgreater{}\%} \FunctionTok{select}\NormalTok{(}\StringTok{"TRIANGULAR"}\NormalTok{, }\StringTok{"ORTHOGONAL"}\NormalTok{, }\StringTok{"SATISFICE\_left"}\NormalTok{, }\StringTok{"SATISFICE\_right"}\NormalTok{,}\StringTok{"TV\_max"}\NormalTok{,}\StringTok{"TV\_start"}\NormalTok{, }\StringTok{"TV\_end"}\NormalTok{, }\StringTok{"TV\_dur"}\NormalTok{) }\SpecialCharTok{\%\textgreater{}\%} \FunctionTok{unlist}\NormalTok{()}
\NormalTok{ves }\OtherTok{\textless{}{-}}\NormalTok{ q }\SpecialCharTok{\%\textgreater{}\%} \FunctionTok{mutate}\NormalTok{(}
  \AttributeTok{SATISFICE\_left\_allow =} \StringTok{""}\NormalTok{,}
  \AttributeTok{SATISFICE\_right\_allow =} \StringTok{""}
\NormalTok{) }\SpecialCharTok{\%\textgreater{}\%} \FunctionTok{select}\NormalTok{(}\StringTok{"TRI\_allow"}\NormalTok{, }\StringTok{"ORTH\_allow"}\NormalTok{, }\StringTok{"SATISFICE\_left\_allow"}\NormalTok{,}\StringTok{"SATISFICE\_right\_allow"}\NormalTok{, }\StringTok{"TV\_max\_allow"}\NormalTok{,}\StringTok{"TV\_start\_allow"}\NormalTok{,}\StringTok{"TV\_end\_allow"}\NormalTok{, }\StringTok{"TV\_dur\_allow"}\NormalTok{)}\SpecialCharTok{\%\textgreater{}\%} \FunctionTok{unlist}\NormalTok{()}
\NormalTok{options }\OtherTok{\textless{}{-}}\NormalTok{ q }\SpecialCharTok{\%\textgreater{}\%} \FunctionTok{select}\NormalTok{(}\StringTok{"OPTIONS"}\NormalTok{)}
\NormalTok{question }\OtherTok{=}\NormalTok{ q }\SpecialCharTok{\%\textgreater{}\%}  \FunctionTok{select}\NormalTok{(}\StringTok{"TEXT"}\NormalTok{)}
\NormalTok{scores }\OtherTok{\textless{}{-}} \FunctionTok{c}\NormalTok{(}\StringTok{"Triangular"}\NormalTok{, }\StringTok{"Orthgonal"}\NormalTok{, }\StringTok{"Satisficing [left]"}\NormalTok{, }\StringTok{"Satisficing [right]"}\NormalTok{, }\StringTok{"Tversky [maximal]"}\NormalTok{, }\StringTok{"Tversky [start diagonal]"}\NormalTok{,}
            \StringTok{"Tversky [end diagonal]"}\NormalTok{, }\StringTok{"Tversky [duration line]"}\NormalTok{)}
\NormalTok{d }\OtherTok{=} \FunctionTok{tibble}\NormalTok{(}\AttributeTok{interpretation =}\NormalTok{ scores, }\AttributeTok{answer =}\NormalTok{ answers, }\AttributeTok{allowed=}\NormalTok{ves)}
\NormalTok{d}\SpecialCharTok{$}\NormalTok{answer }\OtherTok{\textless{}{-}} \FunctionTok{replace\_na}\NormalTok{(d}\SpecialCharTok{$}\NormalTok{answer, }\StringTok{""}\NormalTok{)}
\NormalTok{d}\SpecialCharTok{$}\NormalTok{allowed }\OtherTok{\textless{}{-}} \FunctionTok{replace\_na}\NormalTok{(d}\SpecialCharTok{$}\NormalTok{allowed, }\StringTok{""}\NormalTok{)}

\NormalTok{title }\OtherTok{=} \FunctionTok{paste}\NormalTok{(}\StringTok{"Answer Key | Q : "}\NormalTok{, question)}
\NormalTok{cols }\OtherTok{=} \FunctionTok{c}\NormalTok{(}\StringTok{"interpretation"}\NormalTok{, }\StringTok{"answer"}\NormalTok{,}\StringTok{"not penalized"}\NormalTok{)}

\NormalTok{d }\SpecialCharTok{\%\textgreater{}\%} \FunctionTok{kbl}\NormalTok{(}\AttributeTok{caption =}\NormalTok{ title, }\AttributeTok{col.names =}\NormalTok{ cols) }\SpecialCharTok{\%\textgreater{}\%} \FunctionTok{kable\_classic}\NormalTok{() }\SpecialCharTok{\%\textgreater{}\%}
  \FunctionTok{footnote}\NormalTok{(}\AttributeTok{general =} \FunctionTok{paste}\NormalTok{(}\StringTok{"15 response options: "}\NormalTok{, options), }\AttributeTok{general\_title =} \StringTok{"Note: "}\NormalTok{,}\AttributeTok{footnote\_as\_chunk =}\NormalTok{ T)}
\end{Highlighting}
\end{Shaded}

\begin{table}

\caption{Answer Key | Q :  Which shift(s) begins before J begins and ends during B?}
\centering
\begin{tabular}[t]{l|l|l}
\hline
interpretation & answer & not penalized\\
\hline
Triangular & I & \\
\hline
Orthgonal & I & \\
\hline
Satisficing [left] &  & \\
\hline
Satisficing [right] &  & \\
\hline
Tversky [maximal] &  & \\
\hline
Tversky [start diagonal] &  & \\
\hline
Tversky [end diagonal] &  & \\
\hline
Tversky [duration line] &  & \\
\hline
\multicolumn{3}{l}{\rule{0pt}{1em}\textit{Note: } 15 response options:  ABCDEFGHIJKLMNOPZX}\\
\end{tabular}
\end{table}

\begin{Shaded}
\begin{Highlighting}[]
\NormalTok{title }\OtherTok{\textless{}{-}} \StringTok{"Frequency of Selected Response Options for Question \#9"}
\NormalTok{names }\OtherTok{=} \FunctionTok{c}\NormalTok{(}\StringTok{"response"}\NormalTok{,}\StringTok{"n"}\NormalTok{,}\StringTok{"interpretation"}\NormalTok{,}\StringTok{"absolute"}\NormalTok{,}\StringTok{"tri"}\NormalTok{,}\StringTok{"tversky"}\NormalTok{,}\StringTok{"satisfice"}\NormalTok{,}\StringTok{"orthogonal"}\NormalTok{, }\StringTok{"scaled score"}\NormalTok{)}

\NormalTok{df\_items }\SpecialCharTok{\%\textgreater{}\%} \FunctionTok{filter}\NormalTok{(q }\SpecialCharTok{==}\DecValTok{9}\NormalTok{) }\SpecialCharTok{\%\textgreater{}\%} \FunctionTok{group\_by}\NormalTok{(response) }\SpecialCharTok{\%\textgreater{}\%}
\NormalTok{  dplyr}\SpecialCharTok{::}\FunctionTok{summarise}\NormalTok{( }\AttributeTok{count =} \FunctionTok{n}\NormalTok{(),}
                    \AttributeTok{nice =} \FunctionTok{unique}\NormalTok{(score\_niceABS),}
                    \AttributeTok{triangular =} \FunctionTok{unique}\NormalTok{(score\_TRI),}
                    \AttributeTok{orthogonal =}  \FunctionTok{unique}\NormalTok{(score\_ORTH),}
                    \AttributeTok{satisficing =}  \FunctionTok{unique}\NormalTok{(score\_SATISFICE),}
                    \AttributeTok{tversky =} \FunctionTok{unique}\NormalTok{(score\_TVERSKY),}
                    \AttributeTok{interpretation =} \FunctionTok{unique}\NormalTok{(int2),}
                    \AttributeTok{scaled =} \FunctionTok{unique}\NormalTok{(score\_SCALED)) }\SpecialCharTok{\%\textgreater{}\%}
  \FunctionTok{arrange}\NormalTok{(interpretation, }\FunctionTok{desc}\NormalTok{(count)) }\SpecialCharTok{\%\textgreater{}\%}
  \FunctionTok{select}\NormalTok{(response, count, interpretation, nice,}
\NormalTok{         triangular, tversky, satisficing, orthogonal, scaled) }\SpecialCharTok{\%\textgreater{}\%}
  \FunctionTok{kbl}\NormalTok{(}\AttributeTok{caption =}\NormalTok{ title, }\AttributeTok{col.names =}\NormalTok{ names) }\SpecialCharTok{\%\textgreater{}\%}  \FunctionTok{kable\_classic}\NormalTok{() }\SpecialCharTok{\%\textgreater{}\%}
  \FunctionTok{add\_header\_above}\NormalTok{(}\FunctionTok{c}\NormalTok{(}\StringTok{" "} \OtherTok{=} \DecValTok{3}\NormalTok{, }\StringTok{"Strict Score"} \OtherTok{=} \DecValTok{1}\NormalTok{, }\StringTok{"Interpretation Scores"}\OtherTok{=}\DecValTok{4}\NormalTok{, }\StringTok{"Discriminant"}\OtherTok{=}\DecValTok{1}\NormalTok{)) }\SpecialCharTok{\%\textgreater{}\%}
  \FunctionTok{pack\_rows}\NormalTok{(}\StringTok{"Other"}\NormalTok{, }\DecValTok{1}\NormalTok{, }\DecValTok{2}\NormalTok{) }\SpecialCharTok{\%\textgreater{}\%}
  \FunctionTok{pack\_rows}\NormalTok{(}\StringTok{"Unknown"}\NormalTok{, }\DecValTok{3}\NormalTok{, }\DecValTok{19}\NormalTok{)}
\end{Highlighting}
\end{Shaded}

\textbackslash begin\{table\}

\textbackslash caption\{\label{tab:Q9-RESPONSES}Frequency of Selected
Response Options for Question \#9\} \centering

\begin{tabular}[t]{l|r|l|r|r|r|r|r|r}
\hline
\multicolumn{3}{c|}{ } & \multicolumn{1}{c|}{Strict Score} & \multicolumn{4}{c|}{Interpretation Scores} & \multicolumn{1}{c}{Discriminant} \\
\cline{4-4} \cline{5-8} \cline{9-9}
response & n & interpretation & absolute & tri & tversky & satisfice & orthogonal & scaled score\\
\hline
\multicolumn{9}{l}{\textbf{Other}}\\
\hline
\hspace{1em}I & 247 & both tri + orth & 1 & 1.000 & NA & NA & 1.000 & 0.5\\
\hline
\hspace{1em}IJ & 1 & both tri + orth & 1 & 1.000 & NA & NA & 1.000 & 0.5\\
\hline
\multicolumn{9}{l}{\textbf{Unknown}}\\
\hline
\hspace{1em} & 23 & blank & 0 & 0.000 & NA & NA & 0.000 & 0.0\\
\hline
\hspace{1em}E & 29 & ? & 0 & -0.067 & NA & NA & -0.067 & -0.5\\
\hline
\hspace{1em}F & 6 & ? & 0 & -0.067 & NA & NA & -0.067 & -0.5\\
\hline
\hspace{1em}M & 4 & ? & 0 & -0.067 & NA & NA & -0.067 & -0.5\\
\hline
\hspace{1em}EI & 3 & ? & 0 & 0.933 & NA & NA & 0.933 & -0.5\\
\hline
\hspace{1em}FI & 3 & ? & 0 & 0.933 & NA & NA & 0.933 & -0.5\\
\hline
\hspace{1em}J & 3 & ? & 0 & 0.000 & NA & NA & 0.000 & -0.5\\
\hline
\hspace{1em}K & 2 & ? & 0 & -0.067 & NA & NA & -0.067 & -0.5\\
\hline
\hspace{1em}AGN & 1 & ? & 0 & -0.200 & NA & NA & -0.200 & -0.5\\
\hline
\hspace{1em}B & 1 & ? & 0 & 0.000 & NA & NA & 0.000 & -0.5\\
\hline
\hspace{1em}C & 1 & ? & 0 & -0.067 & NA & NA & -0.067 & -0.5\\
\hline
\hspace{1em}CHO & 1 & ? & 0 & -0.200 & NA & NA & -0.200 & -0.5\\
\hline
\hspace{1em}D & 1 & ? & 0 & -0.067 & NA & NA & -0.067 & -0.5\\
\hline
\hspace{1em}DK & 1 & ? & 0 & -0.133 & NA & NA & -0.133 & -0.5\\
\hline
\hspace{1em}IM & 1 & ? & 0 & 0.933 & NA & NA & 0.933 & -0.5\\
\hline
\hspace{1em}IO & 1 & ? & 0 & 0.933 & NA & NA & 0.933 & -0.5\\
\hline
\hspace{1em}X & 1 & ? & 0 & -0.067 & NA & NA & -0.067 & -0.5\\
\hline
\end{tabular}

\textbackslash end\{table\}

\begin{Shaded}
\begin{Highlighting}[]
\FunctionTok{gf\_dhistogram}\NormalTok{(}\SpecialCharTok{\textasciitilde{}}\NormalTok{ score\_niceABS, }\AttributeTok{fill =} \SpecialCharTok{\textasciitilde{}}\NormalTok{condition, }\AttributeTok{data =}\NormalTok{ df\_items }\SpecialCharTok{\%\textgreater{}\%} \FunctionTok{filter}\NormalTok{(q }\SpecialCharTok{==} \DecValTok{9}\NormalTok{)) }\SpecialCharTok{\%\textgreater{}\%} 
  \FunctionTok{gf\_facet\_grid}\NormalTok{( condition }\SpecialCharTok{\textasciitilde{}}\NormalTok{ ., }\AttributeTok{labeller =}\NormalTok{ label\_both) }\SpecialCharTok{+} 
  \FunctionTok{labs}\NormalTok{( }\AttributeTok{x =} \StringTok{"Scaled Item Score"}\NormalTok{, }\AttributeTok{title =} \StringTok{"Distribution of Scaled Scores | Q9 "}\NormalTok{) }\SpecialCharTok{+} 
  \FunctionTok{theme\_minimal}\NormalTok{() }\SpecialCharTok{+} \FunctionTok{theme}\NormalTok{(}\AttributeTok{legend.position =} \StringTok{"blank"}\NormalTok{)}
\end{Highlighting}
\end{Shaded}

\begin{figure}[H]

{\centering \includegraphics{analysis/SGC3A/2_sgc3A_scoring_files/figure-pdf/Q9-distribution-1.pdf}

}

\end{figure}

\begin{Shaded}
\begin{Highlighting}[]
\FunctionTok{gf\_props}\NormalTok{(}\SpecialCharTok{\textasciitilde{}}\NormalTok{interpretation, }\AttributeTok{fill =} \SpecialCharTok{\textasciitilde{}}\NormalTok{condition, }\AttributeTok{data =}\NormalTok{ df\_items }\SpecialCharTok{\%\textgreater{}\%} \FunctionTok{filter}\NormalTok{(q }\SpecialCharTok{==} \DecValTok{9}\NormalTok{)) }\SpecialCharTok{\%\textgreater{}\%} 
  \FunctionTok{gf\_facet\_grid}\NormalTok{( condition }\SpecialCharTok{\textasciitilde{}}\NormalTok{ ., }\AttributeTok{labeller =}\NormalTok{ label\_both) }\SpecialCharTok{+} 
  \FunctionTok{labs}\NormalTok{( }\AttributeTok{x =} \StringTok{"Interpretation"}\NormalTok{, }\AttributeTok{title =} \StringTok{"Distribution of Interpretations | Q9 "}\NormalTok{) }\SpecialCharTok{+} 
  \FunctionTok{theme\_minimal}\NormalTok{() }\SpecialCharTok{+} \FunctionTok{theme}\NormalTok{(}\AttributeTok{legend.position =} \StringTok{"blank"}\NormalTok{)}
\end{Highlighting}
\end{Shaded}

\begin{figure}[H]

{\centering \includegraphics{analysis/SGC3A/2_sgc3A_scoring_files/figure-pdf/Q9-distribution-2.pdf}

}

\end{figure}

\hypertarget{question-10}{%
\subsubsection{Question \#10}\label{question-10}}

\begin{figure}

{\centering \includegraphics{analysis/SGC3A/static/questions/Q10.png}

}

\caption{\label{fig-Q10}Q10-Question}

\end{figure}

\begin{Shaded}
\begin{Highlighting}[]
\NormalTok{q }\OtherTok{\textless{}{-}}\NormalTok{ keys\_raw }\SpecialCharTok{\%\textgreater{}\%} \FunctionTok{filter}\NormalTok{(Q}\SpecialCharTok{==}\DecValTok{10}\NormalTok{)}
\NormalTok{ignore }\OtherTok{\textless{}{-}}\NormalTok{ q }\SpecialCharTok{\%\textgreater{}\%} \FunctionTok{select}\NormalTok{(}\StringTok{"REF\_POINT"}\NormalTok{)}
\NormalTok{answers }\OtherTok{\textless{}{-}}\NormalTok{ q }\SpecialCharTok{\%\textgreater{}\%} \FunctionTok{select}\NormalTok{(}\StringTok{"TRIANGULAR"}\NormalTok{, }\StringTok{"ORTHOGONAL"}\NormalTok{, }\StringTok{"SATISFICE\_left"}\NormalTok{, }\StringTok{"SATISFICE\_right"}\NormalTok{,}\StringTok{"TV\_max"}\NormalTok{,}\StringTok{"TV\_start"}\NormalTok{, }\StringTok{"TV\_end"}\NormalTok{, }\StringTok{"TV\_dur"}\NormalTok{) }\SpecialCharTok{\%\textgreater{}\%} \FunctionTok{unlist}\NormalTok{()}
\NormalTok{ves }\OtherTok{\textless{}{-}}\NormalTok{ q }\SpecialCharTok{\%\textgreater{}\%} \FunctionTok{mutate}\NormalTok{(}
  \AttributeTok{SATISFICE\_left\_allow =} \StringTok{""}\NormalTok{,}
  \AttributeTok{SATISFICE\_right\_allow =} \StringTok{""}
\NormalTok{) }\SpecialCharTok{\%\textgreater{}\%} \FunctionTok{select}\NormalTok{(}\StringTok{"TRI\_allow"}\NormalTok{, }\StringTok{"ORTH\_allow"}\NormalTok{, }\StringTok{"SATISFICE\_left\_allow"}\NormalTok{,}\StringTok{"SATISFICE\_right\_allow"}\NormalTok{, }\StringTok{"TV\_max\_allow"}\NormalTok{,}\StringTok{"TV\_start\_allow"}\NormalTok{,}\StringTok{"TV\_end\_allow"}\NormalTok{, }\StringTok{"TV\_dur\_allow"}\NormalTok{)}\SpecialCharTok{\%\textgreater{}\%} \FunctionTok{unlist}\NormalTok{()}
\NormalTok{options }\OtherTok{\textless{}{-}}\NormalTok{ q }\SpecialCharTok{\%\textgreater{}\%} \FunctionTok{select}\NormalTok{(}\StringTok{"OPTIONS"}\NormalTok{)}
\NormalTok{question }\OtherTok{=}\NormalTok{ q }\SpecialCharTok{\%\textgreater{}\%}  \FunctionTok{select}\NormalTok{(}\StringTok{"TEXT"}\NormalTok{)}
\NormalTok{scores }\OtherTok{\textless{}{-}} \FunctionTok{c}\NormalTok{(}\StringTok{"Triangular"}\NormalTok{, }\StringTok{"Orthgonal"}\NormalTok{, }\StringTok{"Satisficing [left]"}\NormalTok{, }\StringTok{"Satisficing [right]"}\NormalTok{, }\StringTok{"Tversky [maximal]"}\NormalTok{, }\StringTok{"Tversky [start diagonal]"}\NormalTok{,}
            \StringTok{"Tversky [end diagonal]"}\NormalTok{, }\StringTok{"Tversky [duration line]"}\NormalTok{)}
\NormalTok{d }\OtherTok{=} \FunctionTok{tibble}\NormalTok{(}\AttributeTok{interpretation =}\NormalTok{ scores, }\AttributeTok{answer =}\NormalTok{ answers, }\AttributeTok{allowed=}\NormalTok{ves)}
\NormalTok{d}\SpecialCharTok{$}\NormalTok{answer }\OtherTok{\textless{}{-}} \FunctionTok{replace\_na}\NormalTok{(d}\SpecialCharTok{$}\NormalTok{answer, }\StringTok{""}\NormalTok{)}
\NormalTok{d}\SpecialCharTok{$}\NormalTok{allowed }\OtherTok{\textless{}{-}} \FunctionTok{replace\_na}\NormalTok{(d}\SpecialCharTok{$}\NormalTok{allowed, }\StringTok{""}\NormalTok{)}

\NormalTok{title }\OtherTok{=} \FunctionTok{paste}\NormalTok{(}\StringTok{"Answer Key | Q : "}\NormalTok{, question)}
\NormalTok{cols }\OtherTok{=} \FunctionTok{c}\NormalTok{(}\StringTok{"interpretation"}\NormalTok{, }\StringTok{"answer"}\NormalTok{,}\StringTok{"not penalized"}\NormalTok{)}

\NormalTok{d }\SpecialCharTok{\%\textgreater{}\%} \FunctionTok{kbl}\NormalTok{(}\AttributeTok{caption =}\NormalTok{ title, }\AttributeTok{col.names =}\NormalTok{ cols) }\SpecialCharTok{\%\textgreater{}\%} \FunctionTok{kable\_classic}\NormalTok{() }\SpecialCharTok{\%\textgreater{}\%}
  \FunctionTok{footnote}\NormalTok{(}\AttributeTok{general =} \FunctionTok{paste}\NormalTok{(}\StringTok{"15 response options: "}\NormalTok{, options), }\AttributeTok{general\_title =} \StringTok{"Note: "}\NormalTok{,}\AttributeTok{footnote\_as\_chunk =}\NormalTok{ T)}
\end{Highlighting}
\end{Shaded}

\begin{table}

\caption{Answer Key | Q :  Which shift(s) end at the same time as F?}
\centering
\begin{tabular}[t]{l|l|l}
\hline
interpretation & answer & not penalized\\
\hline
Triangular & E & \\
\hline
Orthgonal & X & \\
\hline
Satisficing [left] &  & \\
\hline
Satisficing [right] &  & \\
\hline
Tversky [maximal] & EGZ & \\
\hline
Tversky [start diagonal] & G & \\
\hline
Tversky [end diagonal] & E & \\
\hline
Tversky [duration line] & Z & \\
\hline
\multicolumn{3}{l}{\rule{0pt}{1em}\textit{Note: } 15 response options:  ABCDEFGHIJKLMNOPZX}\\
\end{tabular}
\end{table}

\begin{Shaded}
\begin{Highlighting}[]
\NormalTok{title }\OtherTok{\textless{}{-}} \StringTok{"Frequency of Selected Response Options for Question \#10"}
\NormalTok{names }\OtherTok{=} \FunctionTok{c}\NormalTok{(}\StringTok{"response"}\NormalTok{,}\StringTok{"n"}\NormalTok{,}\StringTok{"interpretation"}\NormalTok{,}\StringTok{"absolute"}\NormalTok{,}\StringTok{"tri"}\NormalTok{,}\StringTok{"tversky"}\NormalTok{,}\StringTok{"satisfice"}\NormalTok{,}\StringTok{"orthogonal"}\NormalTok{, }\StringTok{"scaled score"}\NormalTok{)}

\NormalTok{df\_items }\SpecialCharTok{\%\textgreater{}\%} \FunctionTok{filter}\NormalTok{(q }\SpecialCharTok{==} \DecValTok{10}\NormalTok{) }\SpecialCharTok{\%\textgreater{}\%} \FunctionTok{group\_by}\NormalTok{(response) }\SpecialCharTok{\%\textgreater{}\%}
\NormalTok{  dplyr}\SpecialCharTok{::}\FunctionTok{summarise}\NormalTok{( }\AttributeTok{count =} \FunctionTok{n}\NormalTok{(),}
                    \AttributeTok{nice =} \FunctionTok{unique}\NormalTok{(score\_niceABS),}
                    \AttributeTok{triangular =} \FunctionTok{unique}\NormalTok{(score\_TRI),}
                    \AttributeTok{orthogonal =}  \FunctionTok{unique}\NormalTok{(score\_ORTH),}
                    \AttributeTok{satisficing =}  \FunctionTok{unique}\NormalTok{(score\_SATISFICE),}
                    \AttributeTok{tversky =} \FunctionTok{unique}\NormalTok{(score\_TVERSKY),}
                    \AttributeTok{interpretation =} \FunctionTok{unique}\NormalTok{(int2),}
                    \AttributeTok{scaled =} \FunctionTok{unique}\NormalTok{(score\_SCALED)) }\SpecialCharTok{\%\textgreater{}\%}
  \FunctionTok{arrange}\NormalTok{(interpretation, }\FunctionTok{desc}\NormalTok{(count)) }\SpecialCharTok{\%\textgreater{}\%}
  \FunctionTok{select}\NormalTok{(response, count, interpretation, nice,}
\NormalTok{         triangular, tversky, satisficing, orthogonal, scaled) }\SpecialCharTok{\%\textgreater{}\%}
  \FunctionTok{kbl}\NormalTok{(}\AttributeTok{caption =}\NormalTok{ title, }\AttributeTok{col.names =}\NormalTok{ names) }\SpecialCharTok{\%\textgreater{}\%}  \FunctionTok{kable\_classic}\NormalTok{() }\SpecialCharTok{\%\textgreater{}\%}
  \FunctionTok{add\_header\_above}\NormalTok{(}\FunctionTok{c}\NormalTok{(}\StringTok{" "} \OtherTok{=} \DecValTok{3}\NormalTok{, }\StringTok{"Strict Score"} \OtherTok{=} \DecValTok{1}\NormalTok{, }\StringTok{"Interpretation Scores"}\OtherTok{=}\DecValTok{4}\NormalTok{, }\StringTok{"Discriminant"}\OtherTok{=}\DecValTok{1}\NormalTok{)) }\SpecialCharTok{\%\textgreater{}\%}
  \FunctionTok{pack\_rows}\NormalTok{(}\StringTok{"Triangular"}\NormalTok{, }\DecValTok{1}\NormalTok{, }\DecValTok{2}\NormalTok{) }\SpecialCharTok{\%\textgreater{}\%}
  \FunctionTok{pack\_rows}\NormalTok{(}\StringTok{"Lines{-}Connect"}\NormalTok{, }\DecValTok{3}\NormalTok{, }\DecValTok{7}\NormalTok{) }\SpecialCharTok{\%\textgreater{}\%}
  \FunctionTok{pack\_rows}\NormalTok{(}\StringTok{"Orthogonal"}\NormalTok{, }\DecValTok{8}\NormalTok{, }\DecValTok{11}\NormalTok{) }\SpecialCharTok{\%\textgreater{}\%}
  \FunctionTok{pack\_rows}\NormalTok{(}\StringTok{"Other"}\NormalTok{, }\DecValTok{12}\NormalTok{, }\DecValTok{14}\NormalTok{) }\SpecialCharTok{\%\textgreater{}\%}
  \FunctionTok{pack\_rows}\NormalTok{(}\StringTok{"Unknown"}\NormalTok{, }\DecValTok{15}\NormalTok{, }\DecValTok{27}\NormalTok{)}
\end{Highlighting}
\end{Shaded}

\textbackslash begin\{table\}

\textbackslash caption\{\label{tab:Q10-RESPONSES}Frequency of Selected
Response Options for Question \#10\} \centering

\begin{tabular}[t]{l|r|l|r|r|r|r|r|r}
\hline
\multicolumn{3}{c|}{ } & \multicolumn{1}{c|}{Strict Score} & \multicolumn{4}{c|}{Interpretation Scores} & \multicolumn{1}{c}{Discriminant} \\
\cline{4-4} \cline{5-8} \cline{9-9}
response & n & interpretation & absolute & tri & tversky & satisfice & orthogonal & scaled score\\
\hline
\multicolumn{9}{l}{\textbf{Triangular}}\\
\hline
\hspace{1em}E & 103 & Triangular & 1 & 1.000 & 1.000 & NA & -0.062 & 1.0\\
\hline
\hspace{1em}EF & 1 & Triangular & 1 & 1.000 & 1.000 & NA & -0.062 & 1.0\\
\hline
\multicolumn{9}{l}{\textbf{Lines-Connect}}\\
\hline
\hspace{1em}Z & 23 & Tversky & 0 & -0.062 & 1.000 & NA & -0.062 & 0.5\\
\hline
\hspace{1em}XZ & 2 & Tversky & 0 & -0.125 & 0.938 & NA & 0.938 & 0.5\\
\hline
\hspace{1em}CG & 1 & Tversky & 0 & -0.125 & 0.938 & NA & -0.125 & 0.5\\
\hline
\hspace{1em}G & 1 & Tversky & 0 & -0.062 & 1.000 & NA & -0.062 & 0.5\\
\hline
\hspace{1em}HLPZ & 1 & Tversky & 0 & -0.250 & 0.812 & NA & -0.250 & 0.5\\
\hline
\multicolumn{9}{l}{\textbf{Orthogonal}}\\
\hline
\hspace{1em}X & 139 & Orthogonal & 0 & -0.062 & -0.062 & NA & 1.000 & -1.0\\
\hline
\hspace{1em}BX & 2 & Orthogonal & 0 & -0.125 & -0.125 & NA & 0.938 & -1.0\\
\hline
\hspace{1em}FX & 2 & Orthogonal & 0 & -0.062 & -0.062 & NA & 1.000 & -1.0\\
\hline
\hspace{1em}AMX & 1 & Orthogonal & 0 & -0.188 & -0.188 & NA & 0.875 & -1.0\\
\hline
\multicolumn{9}{l}{\textbf{Other}}\\
\hline
\hspace{1em}F & 1 & reference & 0 & 0.000 & NA & NA & 0.000 & 0.0\\
\hline
\hspace{1em} & 6 & blank & 0 & 0.000 & NA & NA & 0.000 & 0.0\\
\hline
\hspace{1em}CEGIO & 1 & frenzy & 0 & 0.750 & 0.750 & NA & -0.312 & -0.5\\
\hline
\multicolumn{9}{l}{\textbf{Unknown}}\\
\hline
\hspace{1em}B & 27 & ? & 0 & -0.062 & -0.062 & NA & -0.062 & -0.5\\
\hline
\hspace{1em}J & 6 & ? & 0 & -0.062 & -0.062 & NA & -0.062 & -0.5\\
\hline
\hspace{1em}IJ & 2 & ? & 0 & -0.125 & -0.125 & NA & -0.125 & -0.5\\
\hline
\hspace{1em}P & 2 & ? & 0 & -0.062 & -0.062 & NA & -0.062 & -0.5\\
\hline
\hspace{1em}BO & 1 & ? & 0 & -0.125 & -0.125 & NA & -0.125 & -0.5\\
\hline
\hspace{1em}C & 1 & ? & 0 & -0.062 & -0.062 & NA & -0.062 & -0.5\\
\hline
\hspace{1em}H & 1 & ? & 0 & -0.062 & -0.062 & NA & -0.062 & -0.5\\
\hline
\hspace{1em}HLP & 1 & ? & 0 & -0.188 & -0.188 & NA & -0.188 & -0.5\\
\hline
\hspace{1em}I & 1 & ? & 0 & -0.062 & -0.062 & NA & -0.062 & -0.5\\
\hline
\hspace{1em}JM & 1 & ? & 0 & -0.125 & -0.125 & NA & -0.125 & -0.5\\
\hline
\hspace{1em}K & 1 & ? & 0 & -0.062 & -0.062 & NA & -0.062 & -0.5\\
\hline
\hspace{1em}L & 1 & ? & 0 & -0.062 & -0.062 & NA & -0.062 & -0.5\\
\hline
\hspace{1em}O & 1 & ? & 0 & -0.062 & -0.062 & NA & -0.062 & -0.5\\
\hline
\end{tabular}

\textbackslash end\{table\}

\begin{Shaded}
\begin{Highlighting}[]
\FunctionTok{gf\_dhistogram}\NormalTok{(}\SpecialCharTok{\textasciitilde{}}\NormalTok{ score\_niceABS, }\AttributeTok{fill =} \SpecialCharTok{\textasciitilde{}}\NormalTok{condition, }\AttributeTok{data =}\NormalTok{ df\_items }\SpecialCharTok{\%\textgreater{}\%} \FunctionTok{filter}\NormalTok{(q }\SpecialCharTok{==} \DecValTok{10}\NormalTok{)) }\SpecialCharTok{\%\textgreater{}\%} 
  \FunctionTok{gf\_facet\_grid}\NormalTok{( condition }\SpecialCharTok{\textasciitilde{}}\NormalTok{ ., }\AttributeTok{labeller =}\NormalTok{ label\_both) }\SpecialCharTok{+} 
  \FunctionTok{labs}\NormalTok{( }\AttributeTok{x =} \StringTok{"Scaled Item Score"}\NormalTok{, }\AttributeTok{title =} \StringTok{"Distribution of Scaled Scores | Q10 "}\NormalTok{) }\SpecialCharTok{+} 
  \FunctionTok{theme\_minimal}\NormalTok{() }\SpecialCharTok{+} \FunctionTok{theme}\NormalTok{(}\AttributeTok{legend.position =} \StringTok{"blank"}\NormalTok{)}
\end{Highlighting}
\end{Shaded}

\begin{figure}[H]

{\centering \includegraphics{analysis/SGC3A/2_sgc3A_scoring_files/figure-pdf/Q10-distribution-1.pdf}

}

\end{figure}

\begin{Shaded}
\begin{Highlighting}[]
\FunctionTok{gf\_props}\NormalTok{(}\SpecialCharTok{\textasciitilde{}}\NormalTok{interpretation, }\AttributeTok{fill =} \SpecialCharTok{\textasciitilde{}}\NormalTok{condition, }\AttributeTok{data =}\NormalTok{ df\_items }\SpecialCharTok{\%\textgreater{}\%} \FunctionTok{filter}\NormalTok{(q }\SpecialCharTok{==} \DecValTok{10}\NormalTok{)) }\SpecialCharTok{\%\textgreater{}\%} 
  \FunctionTok{gf\_facet\_grid}\NormalTok{( condition }\SpecialCharTok{\textasciitilde{}}\NormalTok{ ., }\AttributeTok{labeller =}\NormalTok{ label\_both) }\SpecialCharTok{+} 
  \FunctionTok{labs}\NormalTok{( }\AttributeTok{x =} \StringTok{"Interpretation"}\NormalTok{, }\AttributeTok{title =} \StringTok{"Distribution of Interpretations | Q10 "}\NormalTok{) }\SpecialCharTok{+} 
  \FunctionTok{theme\_minimal}\NormalTok{() }\SpecialCharTok{+} \FunctionTok{theme}\NormalTok{(}\AttributeTok{legend.position =} \StringTok{"blank"}\NormalTok{)}
\end{Highlighting}
\end{Shaded}

\begin{figure}[H]

{\centering \includegraphics{analysis/SGC3A/2_sgc3A_scoring_files/figure-pdf/Q10-distribution-2.pdf}

}

\end{figure}

\hypertarget{question-11}{%
\subsubsection{Question \#11}\label{question-11}}

\begin{figure}

{\centering \includegraphics{analysis/SGC3A/static/questions/Q11.png}

}

\caption{\label{fig-Q11}Q11-Question}

\end{figure}

\begin{Shaded}
\begin{Highlighting}[]
\NormalTok{q }\OtherTok{\textless{}{-}}\NormalTok{ keys\_raw }\SpecialCharTok{\%\textgreater{}\%} \FunctionTok{filter}\NormalTok{(Q}\SpecialCharTok{==}\DecValTok{11}\NormalTok{)}
\NormalTok{ignore }\OtherTok{\textless{}{-}}\NormalTok{ q }\SpecialCharTok{\%\textgreater{}\%} \FunctionTok{select}\NormalTok{(}\StringTok{"REF\_POINT"}\NormalTok{)}
\NormalTok{answers }\OtherTok{\textless{}{-}}\NormalTok{ q }\SpecialCharTok{\%\textgreater{}\%} \FunctionTok{select}\NormalTok{(}\StringTok{"TRIANGULAR"}\NormalTok{, }\StringTok{"ORTHOGONAL"}\NormalTok{, }\StringTok{"SATISFICE\_left"}\NormalTok{, }\StringTok{"SATISFICE\_right"}\NormalTok{,}\StringTok{"TV\_max"}\NormalTok{,}\StringTok{"TV\_start"}\NormalTok{, }\StringTok{"TV\_end"}\NormalTok{, }\StringTok{"TV\_dur"}\NormalTok{) }\SpecialCharTok{\%\textgreater{}\%} \FunctionTok{unlist}\NormalTok{()}
\NormalTok{ves }\OtherTok{\textless{}{-}}\NormalTok{ q }\SpecialCharTok{\%\textgreater{}\%} \FunctionTok{mutate}\NormalTok{(}
  \AttributeTok{SATISFICE\_left\_allow =} \StringTok{""}\NormalTok{,}
  \AttributeTok{SATISFICE\_right\_allow =} \StringTok{""}
\NormalTok{) }\SpecialCharTok{\%\textgreater{}\%} \FunctionTok{select}\NormalTok{(}\StringTok{"TRI\_allow"}\NormalTok{, }\StringTok{"ORTH\_allow"}\NormalTok{, }\StringTok{"SATISFICE\_left\_allow"}\NormalTok{,}\StringTok{"SATISFICE\_right\_allow"}\NormalTok{, }\StringTok{"TV\_max\_allow"}\NormalTok{,}\StringTok{"TV\_start\_allow"}\NormalTok{,}\StringTok{"TV\_end\_allow"}\NormalTok{, }\StringTok{"TV\_dur\_allow"}\NormalTok{)}\SpecialCharTok{\%\textgreater{}\%} \FunctionTok{unlist}\NormalTok{()}
\NormalTok{options }\OtherTok{\textless{}{-}}\NormalTok{ q }\SpecialCharTok{\%\textgreater{}\%} \FunctionTok{select}\NormalTok{(}\StringTok{"OPTIONS"}\NormalTok{)}
\NormalTok{question }\OtherTok{=}\NormalTok{ q }\SpecialCharTok{\%\textgreater{}\%}  \FunctionTok{select}\NormalTok{(}\StringTok{"TEXT"}\NormalTok{)}
\NormalTok{scores }\OtherTok{\textless{}{-}} \FunctionTok{c}\NormalTok{(}\StringTok{"Triangular"}\NormalTok{, }\StringTok{"Orthgonal"}\NormalTok{, }\StringTok{"Satisficing [left]"}\NormalTok{, }\StringTok{"Satisficing [right]"}\NormalTok{, }\StringTok{"Tversky [maximal]"}\NormalTok{, }\StringTok{"Tversky [start diagonal]"}\NormalTok{,}
            \StringTok{"Tversky [end diagonal]"}\NormalTok{, }\StringTok{"Tversky [duration line]"}\NormalTok{)}
\NormalTok{d }\OtherTok{=} \FunctionTok{tibble}\NormalTok{(}\AttributeTok{interpretation =}\NormalTok{ scores, }\AttributeTok{answer =}\NormalTok{ answers, }\AttributeTok{allowed=}\NormalTok{ves)}
\NormalTok{d}\SpecialCharTok{$}\NormalTok{answer }\OtherTok{\textless{}{-}} \FunctionTok{replace\_na}\NormalTok{(d}\SpecialCharTok{$}\NormalTok{answer, }\StringTok{""}\NormalTok{)}
\NormalTok{d}\SpecialCharTok{$}\NormalTok{allowed }\OtherTok{\textless{}{-}} \FunctionTok{replace\_na}\NormalTok{(d}\SpecialCharTok{$}\NormalTok{allowed, }\StringTok{""}\NormalTok{)}

\NormalTok{title }\OtherTok{=} \FunctionTok{paste}\NormalTok{(}\StringTok{"Answer Key | Q : "}\NormalTok{, question)}
\NormalTok{cols }\OtherTok{=} \FunctionTok{c}\NormalTok{(}\StringTok{"interpretation"}\NormalTok{, }\StringTok{"answer"}\NormalTok{,}\StringTok{"not penalized"}\NormalTok{)}

\NormalTok{d }\SpecialCharTok{\%\textgreater{}\%} \FunctionTok{kbl}\NormalTok{(}\AttributeTok{caption =}\NormalTok{ title, }\AttributeTok{col.names =}\NormalTok{ cols) }\SpecialCharTok{\%\textgreater{}\%} \FunctionTok{kable\_classic}\NormalTok{() }\SpecialCharTok{\%\textgreater{}\%}
  \FunctionTok{footnote}\NormalTok{(}\AttributeTok{general =} \FunctionTok{paste}\NormalTok{(}\StringTok{"15 response options: "}\NormalTok{, options), }\AttributeTok{general\_title =} \StringTok{"Note: "}\NormalTok{,}\AttributeTok{footnote\_as\_chunk =}\NormalTok{ T)}
\end{Highlighting}
\end{Shaded}

\begin{table}

\caption{Answer Key | Q :  Which shift(s) start at 12pm?}
\centering
\begin{tabular}[t]{l|l|l}
\hline
interpretation & answer & not penalized\\
\hline
Triangular & ML & \\
\hline
Orthgonal & FB & \\
\hline
Satisficing [left] &  & \\
\hline
Satisficing [right] &  & \\
\hline
Tversky [maximal] &  & \\
\hline
Tversky [start diagonal] &  & \\
\hline
Tversky [end diagonal] &  & \\
\hline
Tversky [duration line] &  & \\
\hline
\multicolumn{3}{l}{\rule{0pt}{1em}\textit{Note: } 15 response options:  ABCDEFGHIJKLMNOPZX}\\
\end{tabular}
\end{table}

\begin{Shaded}
\begin{Highlighting}[]
\NormalTok{title }\OtherTok{\textless{}{-}} \StringTok{"Frequency of Selected Response Options for Question \#11"}
\NormalTok{names }\OtherTok{=} \FunctionTok{c}\NormalTok{(}\StringTok{"response"}\NormalTok{,}\StringTok{"n"}\NormalTok{,}\StringTok{"interpretation"}\NormalTok{,}\StringTok{"absolute"}\NormalTok{,}\StringTok{"tri"}\NormalTok{,}\StringTok{"tversky"}\NormalTok{,}\StringTok{"satisfice"}\NormalTok{,}\StringTok{"orthogonal"}\NormalTok{, }\StringTok{"scaled score"}\NormalTok{)}

\NormalTok{df\_items }\SpecialCharTok{\%\textgreater{}\%} \FunctionTok{filter}\NormalTok{(q }\SpecialCharTok{==} \DecValTok{11}\NormalTok{) }\SpecialCharTok{\%\textgreater{}\%} \FunctionTok{group\_by}\NormalTok{(response) }\SpecialCharTok{\%\textgreater{}\%}
\NormalTok{  dplyr}\SpecialCharTok{::}\FunctionTok{summarise}\NormalTok{( }\AttributeTok{count =} \FunctionTok{n}\NormalTok{(),}
                    \AttributeTok{nice =} \FunctionTok{unique}\NormalTok{(score\_niceABS),}
                    \AttributeTok{triangular =} \FunctionTok{unique}\NormalTok{(score\_TRI),}
                    \AttributeTok{orthogonal =}  \FunctionTok{unique}\NormalTok{(score\_ORTH),}
                    \AttributeTok{satisficing =}  \FunctionTok{unique}\NormalTok{(score\_SATISFICE),}
                    \AttributeTok{tversky =} \FunctionTok{unique}\NormalTok{(score\_TVERSKY),}
                    \AttributeTok{interpretation =} \FunctionTok{unique}\NormalTok{(int2),}
                    \AttributeTok{scaled =} \FunctionTok{unique}\NormalTok{(score\_SCALED)) }\SpecialCharTok{\%\textgreater{}\%}
  \FunctionTok{arrange}\NormalTok{(interpretation, }\FunctionTok{desc}\NormalTok{(count)) }\SpecialCharTok{\%\textgreater{}\%}
  \FunctionTok{select}\NormalTok{(response, count, interpretation, nice,}
\NormalTok{         triangular, tversky, satisficing, orthogonal, scaled) }\SpecialCharTok{\%\textgreater{}\%}
  \FunctionTok{kbl}\NormalTok{(}\AttributeTok{caption =}\NormalTok{ title, }\AttributeTok{col.names =}\NormalTok{ names) }\SpecialCharTok{\%\textgreater{}\%}  \FunctionTok{kable\_classic}\NormalTok{() }\SpecialCharTok{\%\textgreater{}\%}
  \FunctionTok{add\_header\_above}\NormalTok{(}\FunctionTok{c}\NormalTok{(}\StringTok{" "} \OtherTok{=} \DecValTok{3}\NormalTok{, }\StringTok{"Strict Score"} \OtherTok{=} \DecValTok{1}\NormalTok{, }\StringTok{"Interpretation Scores"}\OtherTok{=}\DecValTok{4}\NormalTok{, }\StringTok{"Discriminant"}\OtherTok{=}\DecValTok{1}\NormalTok{)) }\SpecialCharTok{\%\textgreater{}\%}
  \FunctionTok{pack\_rows}\NormalTok{(}\StringTok{"Triangular"}\NormalTok{, }\DecValTok{1}\NormalTok{, }\DecValTok{4}\NormalTok{) }\SpecialCharTok{\%\textgreater{}\%}
  \FunctionTok{pack\_rows}\NormalTok{(}\StringTok{"Orthogonal"}\NormalTok{, }\DecValTok{5}\NormalTok{, }\DecValTok{9}\NormalTok{) }\SpecialCharTok{\%\textgreater{}\%}
  \FunctionTok{pack\_rows}\NormalTok{(}\StringTok{"Other"}\NormalTok{, }\DecValTok{10}\NormalTok{, }\DecValTok{12}\NormalTok{) }\SpecialCharTok{\%\textgreater{}\%}
  \FunctionTok{pack\_rows}\NormalTok{(}\StringTok{"Unknown"}\NormalTok{, }\DecValTok{13}\NormalTok{, }\DecValTok{17}\NormalTok{)}
\end{Highlighting}
\end{Shaded}

\textbackslash begin\{table\}

\textbackslash caption\{\label{tab:Q11-RESPONSES}Frequency of Selected
Response Options for Question \#11\} \centering

\begin{tabular}[t]{l|r|l|r|r|r|r|r|r}
\hline
\multicolumn{3}{c|}{ } & \multicolumn{1}{c|}{Strict Score} & \multicolumn{4}{c|}{Interpretation Scores} & \multicolumn{1}{c}{Discriminant} \\
\cline{4-4} \cline{5-8} \cline{9-9}
response & n & interpretation & absolute & tri & tversky & satisfice & orthogonal & scaled score\\
\hline
\multicolumn{9}{l}{\textbf{Triangular}}\\
\hline
\hspace{1em}LM & 99 & Triangular & 1 & 1.000 & NA & NA & -0.125 & 1.0\\
\hline
\hspace{1em}M & 7 & Triangular & 0 & 0.500 & NA & NA & -0.062 & 1.0\\
\hline
\hspace{1em}BLM & 2 & Triangular & 0 & 0.938 & NA & NA & 0.375 & 1.0\\
\hline
\hspace{1em}EKM & 1 & Triangular & 0 & 0.375 & NA & NA & -0.188 & 1.0\\
\hline
\multicolumn{9}{l}{\textbf{Orthogonal}}\\
\hline
\hspace{1em}BF & 201 & Orthogonal & 0 & -0.125 & NA & NA & 1.000 & -1.0\\
\hline
\hspace{1em}B & 4 & Orthogonal & 0 & -0.062 & NA & NA & 0.500 & -1.0\\
\hline
\hspace{1em}F & 2 & Orthogonal & 0 & -0.062 & NA & NA & 0.500 & -1.0\\
\hline
\hspace{1em}BFXZ & 1 & Orthogonal & 0 & -0.250 & NA & NA & 0.875 & -1.0\\
\hline
\hspace{1em}BH & 1 & Orthogonal & 0 & -0.125 & NA & NA & 0.438 & -1.0\\
\hline
\multicolumn{9}{l}{\textbf{Other}}\\
\hline
\hspace{1em} & 4 & blank & 0 & 0.000 & NA & NA & 0.000 & 0.0\\
\hline
\hspace{1em}ACDGHKLMNOPXZ & 1 & frenzy & 0 & 0.312 & NA & NA & -0.812 & -0.5\\
\hline
\hspace{1em}DHLMNOXZ & 1 & frenzy & 0 & 0.625 & NA & NA & -0.500 & -0.5\\
\hline
\multicolumn{9}{l}{\textbf{Unknown}}\\
\hline
\hspace{1em}J & 2 & ? & 0 & -0.062 & NA & NA & -0.062 & -0.5\\
\hline
\hspace{1em}CX & 1 & ? & 0 & -0.125 & NA & NA & -0.125 & -0.5\\
\hline
\hspace{1em}N & 1 & ? & 0 & -0.062 & NA & NA & -0.062 & -0.5\\
\hline
\hspace{1em}X & 1 & ? & 0 & -0.062 & NA & NA & -0.062 & -0.5\\
\hline
\hspace{1em}XZ & 1 & ? & 0 & -0.125 & NA & NA & -0.125 & -0.5\\
\hline
\end{tabular}

\textbackslash end\{table\}

\begin{Shaded}
\begin{Highlighting}[]
\FunctionTok{gf\_dhistogram}\NormalTok{(}\SpecialCharTok{\textasciitilde{}}\NormalTok{ score\_niceABS, }\AttributeTok{fill =} \SpecialCharTok{\textasciitilde{}}\NormalTok{condition, }\AttributeTok{data =}\NormalTok{ df\_items }\SpecialCharTok{\%\textgreater{}\%} \FunctionTok{filter}\NormalTok{(q }\SpecialCharTok{==} \DecValTok{11}\NormalTok{)) }\SpecialCharTok{\%\textgreater{}\%} 
  \FunctionTok{gf\_facet\_grid}\NormalTok{( condition }\SpecialCharTok{\textasciitilde{}}\NormalTok{ ., }\AttributeTok{labeller =}\NormalTok{ label\_both) }\SpecialCharTok{+} 
  \FunctionTok{labs}\NormalTok{( }\AttributeTok{x =} \StringTok{"Scaled Item Score"}\NormalTok{, }\AttributeTok{title =} \StringTok{"Distribution of Scaled Scores | Q11 "}\NormalTok{) }\SpecialCharTok{+} 
  \FunctionTok{theme\_minimal}\NormalTok{() }\SpecialCharTok{+} \FunctionTok{theme}\NormalTok{(}\AttributeTok{legend.position =} \StringTok{"blank"}\NormalTok{)}
\end{Highlighting}
\end{Shaded}

\begin{figure}[H]

{\centering \includegraphics{analysis/SGC3A/2_sgc3A_scoring_files/figure-pdf/Q11-distribution-1.pdf}

}

\end{figure}

\begin{Shaded}
\begin{Highlighting}[]
\FunctionTok{gf\_props}\NormalTok{(}\SpecialCharTok{\textasciitilde{}}\NormalTok{interpretation, }\AttributeTok{fill =} \SpecialCharTok{\textasciitilde{}}\NormalTok{condition, }\AttributeTok{data =}\NormalTok{ df\_items }\SpecialCharTok{\%\textgreater{}\%} \FunctionTok{filter}\NormalTok{(q }\SpecialCharTok{==} \DecValTok{11}\NormalTok{)) }\SpecialCharTok{\%\textgreater{}\%}
  \FunctionTok{gf\_facet\_grid}\NormalTok{( condition }\SpecialCharTok{\textasciitilde{}}\NormalTok{ ., }\AttributeTok{labeller =}\NormalTok{ label\_both) }\SpecialCharTok{+} 
  \FunctionTok{labs}\NormalTok{( }\AttributeTok{x =} \StringTok{"Interpretation"}\NormalTok{, }\AttributeTok{title =} \StringTok{"Distribution of Interpretations | Q11 "}\NormalTok{) }\SpecialCharTok{+} 
  \FunctionTok{theme\_minimal}\NormalTok{() }\SpecialCharTok{+} \FunctionTok{theme}\NormalTok{(}\AttributeTok{legend.position =} \StringTok{"blank"}\NormalTok{)}
\end{Highlighting}
\end{Shaded}

\begin{figure}[H]

{\centering \includegraphics{analysis/SGC3A/2_sgc3A_scoring_files/figure-pdf/Q11-distribution-2.pdf}

}

\end{figure}

\hypertarget{question-12}{%
\subsubsection{Question \#12}\label{question-12}}

\begin{figure}

{\centering \includegraphics{analysis/SGC3A/static/questions/Q12.png}

}

\caption{\label{fig-Q12}Q12-Question}

\end{figure}

\begin{Shaded}
\begin{Highlighting}[]
\NormalTok{q }\OtherTok{\textless{}{-}}\NormalTok{ keys\_raw }\SpecialCharTok{\%\textgreater{}\%} \FunctionTok{filter}\NormalTok{(Q}\SpecialCharTok{==}\DecValTok{12}\NormalTok{)}
\NormalTok{ignore }\OtherTok{\textless{}{-}}\NormalTok{ q }\SpecialCharTok{\%\textgreater{}\%} \FunctionTok{select}\NormalTok{(}\StringTok{"REF\_POINT"}\NormalTok{)}
\NormalTok{answers }\OtherTok{\textless{}{-}}\NormalTok{ q }\SpecialCharTok{\%\textgreater{}\%} \FunctionTok{select}\NormalTok{(}\StringTok{"TRIANGULAR"}\NormalTok{, }\StringTok{"ORTHOGONAL"}\NormalTok{, }\StringTok{"SATISFICE\_left"}\NormalTok{, }\StringTok{"SATISFICE\_right"}\NormalTok{,}\StringTok{"TV\_max"}\NormalTok{,}\StringTok{"TV\_start"}\NormalTok{, }\StringTok{"TV\_end"}\NormalTok{, }\StringTok{"TV\_dur"}\NormalTok{) }\SpecialCharTok{\%\textgreater{}\%} \FunctionTok{unlist}\NormalTok{()}
\NormalTok{ves }\OtherTok{\textless{}{-}}\NormalTok{ q }\SpecialCharTok{\%\textgreater{}\%} \FunctionTok{mutate}\NormalTok{(}
  \AttributeTok{SATISFICE\_left\_allow =} \StringTok{""}\NormalTok{,}
  \AttributeTok{SATISFICE\_right\_allow =} \StringTok{""}
\NormalTok{) }\SpecialCharTok{\%\textgreater{}\%} \FunctionTok{select}\NormalTok{(}\StringTok{"TRI\_allow"}\NormalTok{, }\StringTok{"ORTH\_allow"}\NormalTok{, }\StringTok{"SATISFICE\_left\_allow"}\NormalTok{,}\StringTok{"SATISFICE\_right\_allow"}\NormalTok{, }\StringTok{"TV\_max\_allow"}\NormalTok{,}\StringTok{"TV\_start\_allow"}\NormalTok{,}\StringTok{"TV\_end\_allow"}\NormalTok{, }\StringTok{"TV\_dur\_allow"}\NormalTok{)}\SpecialCharTok{\%\textgreater{}\%} \FunctionTok{unlist}\NormalTok{()}
\NormalTok{options }\OtherTok{\textless{}{-}}\NormalTok{ q }\SpecialCharTok{\%\textgreater{}\%} \FunctionTok{select}\NormalTok{(}\StringTok{"OPTIONS"}\NormalTok{)}
\NormalTok{question }\OtherTok{=}\NormalTok{ q }\SpecialCharTok{\%\textgreater{}\%}  \FunctionTok{select}\NormalTok{(}\StringTok{"TEXT"}\NormalTok{)}
\NormalTok{scores }\OtherTok{\textless{}{-}} \FunctionTok{c}\NormalTok{(}\StringTok{"Triangular"}\NormalTok{, }\StringTok{"Orthgonal"}\NormalTok{, }\StringTok{"Satisficing [left]"}\NormalTok{, }\StringTok{"Satisficing [right]"}\NormalTok{, }\StringTok{"Tversky [maximal]"}\NormalTok{, }\StringTok{"Tversky [start diagonal]"}\NormalTok{,}
            \StringTok{"Tversky [end diagonal]"}\NormalTok{, }\StringTok{"Tversky [duration line]"}\NormalTok{)}
\NormalTok{d }\OtherTok{=} \FunctionTok{tibble}\NormalTok{(}\AttributeTok{interpretation =}\NormalTok{ scores, }\AttributeTok{answer =}\NormalTok{ answers, }\AttributeTok{allowed=}\NormalTok{ves)}
\NormalTok{d}\SpecialCharTok{$}\NormalTok{answer }\OtherTok{\textless{}{-}} \FunctionTok{replace\_na}\NormalTok{(d}\SpecialCharTok{$}\NormalTok{answer, }\StringTok{""}\NormalTok{)}
\NormalTok{d}\SpecialCharTok{$}\NormalTok{allowed }\OtherTok{\textless{}{-}} \FunctionTok{replace\_na}\NormalTok{(d}\SpecialCharTok{$}\NormalTok{allowed, }\StringTok{""}\NormalTok{)}

\NormalTok{title }\OtherTok{=} \FunctionTok{paste}\NormalTok{(}\StringTok{"Answer Key | Q : "}\NormalTok{, question)}
\NormalTok{cols }\OtherTok{=} \FunctionTok{c}\NormalTok{(}\StringTok{"interpretation"}\NormalTok{, }\StringTok{"answer"}\NormalTok{,}\StringTok{"not penalized"}\NormalTok{)}

\NormalTok{d }\SpecialCharTok{\%\textgreater{}\%} \FunctionTok{kbl}\NormalTok{(}\AttributeTok{caption =}\NormalTok{ title, }\AttributeTok{col.names =}\NormalTok{ cols) }\SpecialCharTok{\%\textgreater{}\%} \FunctionTok{kable\_classic}\NormalTok{() }\SpecialCharTok{\%\textgreater{}\%}
  \FunctionTok{footnote}\NormalTok{(}\AttributeTok{general =} \FunctionTok{paste}\NormalTok{(}\StringTok{"15 response options: "}\NormalTok{, options), }\AttributeTok{general\_title =} \StringTok{"Note: "}\NormalTok{,}\AttributeTok{footnote\_as\_chunk =}\NormalTok{ T)}
\end{Highlighting}
\end{Shaded}

\begin{table}

\caption{Answer Key | Q :  Which shift(s) start at the same time as F?}
\centering
\begin{tabular}[t]{l|l|l}
\hline
interpretation & answer & not penalized\\
\hline
Triangular & G & \\
\hline
Orthgonal & B & \\
\hline
Satisficing [left] &  & \\
\hline
Satisficing [right] &  & \\
\hline
Tversky [maximal] & GZ & \\
\hline
Tversky [start diagonal] & G & \\
\hline
Tversky [end diagonal] &  & \\
\hline
Tversky [duration line] & Z & \\
\hline
\multicolumn{3}{l}{\rule{0pt}{1em}\textit{Note: } 15 response options:  ABCDEFGHIJKLMNOPZX}\\
\end{tabular}
\end{table}

\begin{Shaded}
\begin{Highlighting}[]
\NormalTok{title }\OtherTok{\textless{}{-}} \StringTok{"Frequency of Selected Response Options for Question \#12"}
\NormalTok{names }\OtherTok{=} \FunctionTok{c}\NormalTok{(}\StringTok{"response"}\NormalTok{,}\StringTok{"n"}\NormalTok{,}\StringTok{"interpretation"}\NormalTok{,}\StringTok{"absolute"}\NormalTok{,}\StringTok{"tri"}\NormalTok{,}\StringTok{"tversky"}\NormalTok{,}\StringTok{"satisfice"}\NormalTok{,}\StringTok{"orthogonal"}\NormalTok{, }\StringTok{"scaled score"}\NormalTok{)}

\NormalTok{df\_items }\SpecialCharTok{\%\textgreater{}\%} \FunctionTok{filter}\NormalTok{(q }\SpecialCharTok{==} \DecValTok{12}\NormalTok{) }\SpecialCharTok{\%\textgreater{}\%} \FunctionTok{group\_by}\NormalTok{(response) }\SpecialCharTok{\%\textgreater{}\%}
\NormalTok{  dplyr}\SpecialCharTok{::}\FunctionTok{summarise}\NormalTok{( }\AttributeTok{count =} \FunctionTok{n}\NormalTok{(),}
                    \AttributeTok{nice =} \FunctionTok{unique}\NormalTok{(score\_niceABS),}
                    \AttributeTok{triangular =} \FunctionTok{unique}\NormalTok{(score\_TRI),}
                    \AttributeTok{orthogonal =}  \FunctionTok{unique}\NormalTok{(score\_ORTH),}
                    \AttributeTok{satisficing =}  \FunctionTok{unique}\NormalTok{(score\_SATISFICE),}
                    \AttributeTok{tversky =} \FunctionTok{unique}\NormalTok{(score\_TVERSKY),}
                    \AttributeTok{interpretation =} \FunctionTok{unique}\NormalTok{(int2),}
                    \AttributeTok{scaled =} \FunctionTok{unique}\NormalTok{(score\_SCALED)) }\SpecialCharTok{\%\textgreater{}\%}
  \FunctionTok{arrange}\NormalTok{(interpretation, }\FunctionTok{desc}\NormalTok{(count)) }\SpecialCharTok{\%\textgreater{}\%}
  \FunctionTok{select}\NormalTok{(response, count, interpretation, nice,}
\NormalTok{         triangular, tversky, satisficing, orthogonal, scaled) }\SpecialCharTok{\%\textgreater{}\%}
  \FunctionTok{kbl}\NormalTok{(}\AttributeTok{caption =}\NormalTok{ title, }\AttributeTok{col.names =}\NormalTok{ names) }\SpecialCharTok{\%\textgreater{}\%}  \FunctionTok{kable\_classic}\NormalTok{() }\SpecialCharTok{\%\textgreater{}\%}
  \FunctionTok{add\_header\_above}\NormalTok{(}\FunctionTok{c}\NormalTok{(}\StringTok{" "} \OtherTok{=} \DecValTok{3}\NormalTok{, }\StringTok{"Strict Score"} \OtherTok{=} \DecValTok{1}\NormalTok{, }\StringTok{"Interpretation Scores"}\OtherTok{=}\DecValTok{4}\NormalTok{, }\StringTok{"Discriminant"}\OtherTok{=}\DecValTok{1}\NormalTok{)) }\SpecialCharTok{\%\textgreater{}\%}
  \FunctionTok{pack\_rows}\NormalTok{(}\StringTok{"Triangular"}\NormalTok{, }\DecValTok{1}\NormalTok{, }\DecValTok{3}\NormalTok{) }\SpecialCharTok{\%\textgreater{}\%}
  \FunctionTok{pack\_rows}\NormalTok{(}\StringTok{"Lines{-}Connect"}\NormalTok{, }\DecValTok{4}\NormalTok{, }\DecValTok{6}\NormalTok{) }\SpecialCharTok{\%\textgreater{}\%}
  \FunctionTok{pack\_rows}\NormalTok{(}\StringTok{"Orthogonal"}\NormalTok{, }\DecValTok{7}\NormalTok{, }\DecValTok{8}\NormalTok{) }\SpecialCharTok{\%\textgreater{}\%}
  \FunctionTok{pack\_rows}\NormalTok{(}\StringTok{"Other"}\NormalTok{, }\DecValTok{9}\NormalTok{, }\DecValTok{10}\NormalTok{) }\SpecialCharTok{\%\textgreater{}\%}
  \FunctionTok{pack\_rows}\NormalTok{(}\StringTok{"Unknown"}\NormalTok{, }\DecValTok{11}\NormalTok{, }\DecValTok{14}\NormalTok{)}
\end{Highlighting}
\end{Shaded}

\textbackslash begin\{table\}

\textbackslash caption\{\label{tab:Q12-RESPONSES}Frequency of Selected
Response Options for Question \#12\} \centering

\begin{tabular}[t]{l|r|l|r|r|r|r|r|r}
\hline
\multicolumn{3}{c|}{ } & \multicolumn{1}{c|}{Strict Score} & \multicolumn{4}{c|}{Interpretation Scores} & \multicolumn{1}{c}{Discriminant} \\
\cline{4-4} \cline{5-8} \cline{9-9}
response & n & interpretation & absolute & tri & tversky & satisfice & orthogonal & scaled score\\
\hline
\multicolumn{9}{l}{\textbf{Triangular}}\\
\hline
\hspace{1em}G & 98 & Triangular & 1 & 1.000 & 1.000 & NA & -0.062 & 1.0\\
\hline
\hspace{1em}FG & 3 & Triangular & 1 & 1.000 & 1.000 & NA & -0.062 & 1.0\\
\hline
\hspace{1em}GP & 1 & Triangular & 0 & 0.938 & 0.938 & NA & -0.125 & 1.0\\
\hline
\multicolumn{9}{l}{\textbf{Lines-Connect}}\\
\hline
\hspace{1em}Z & 4 & Tversky & 0 & -0.062 & 1.000 & NA & -0.062 & 0.5\\
\hline
\hspace{1em}BZ & 1 & Tversky & 0 & -0.125 & 0.938 & NA & 0.938 & 0.5\\
\hline
\hspace{1em}B & 206 & Orthogonal & 0 & -0.062 & -0.062 & NA & 1.000 & -1.0\\
\hline
\multicolumn{9}{l}{\textbf{Orthogonal}}\\
\hline
\hspace{1em}BF & 5 & Orthogonal & 0 & -0.062 & -0.062 & NA & 1.000 & -1.0\\
\hline
\hspace{1em} & 3 & blank & 0 & 0.000 & 0.000 & NA & 0.000 & 0.0\\
\hline
\multicolumn{9}{l}{\textbf{Other}}\\
\hline
\hspace{1em}CEGIO & 1 & frenzy & 0 & 0.750 & 0.750 & NA & -0.312 & -0.5\\
\hline
\hspace{1em}J & 3 & ? & 0 & -0.062 & -0.062 & NA & -0.062 & -0.5\\
\hline
\multicolumn{9}{l}{\textbf{Unknown}}\\
\hline
\hspace{1em}E & 2 & ? & 0 & -0.062 & -0.062 & NA & -0.062 & -0.5\\
\hline
\hspace{1em}FM & 1 & ? & 0 & -0.062 & -0.062 & NA & -0.062 & -0.5\\
\hline
\hspace{1em}N & 1 & ? & 0 & -0.062 & -0.062 & NA & -0.062 & -0.5\\
\hline
\hspace{1em}X & 1 & ? & 0 & -0.062 & -0.062 & NA & -0.062 & -0.5\\
\hline
\end{tabular}

\textbackslash end\{table\}

\begin{Shaded}
\begin{Highlighting}[]
\FunctionTok{gf\_dhistogram}\NormalTok{(}\SpecialCharTok{\textasciitilde{}}\NormalTok{ score\_niceABS, }\AttributeTok{fill =} \SpecialCharTok{\textasciitilde{}}\NormalTok{condition, }\AttributeTok{data =}\NormalTok{ df\_items }\SpecialCharTok{\%\textgreater{}\%} \FunctionTok{filter}\NormalTok{(q }\SpecialCharTok{==} \DecValTok{12}\NormalTok{)) }\SpecialCharTok{\%\textgreater{}\%} 
  \FunctionTok{gf\_facet\_grid}\NormalTok{( condition }\SpecialCharTok{\textasciitilde{}}\NormalTok{ ., }\AttributeTok{labeller =}\NormalTok{ label\_both) }\SpecialCharTok{+} 
  \FunctionTok{labs}\NormalTok{( }\AttributeTok{x =} \StringTok{"Scaled Item Score"}\NormalTok{, }\AttributeTok{title =} \StringTok{"Distribution of Scaled Scores | Q12 "}\NormalTok{) }\SpecialCharTok{+} 
  \FunctionTok{theme\_minimal}\NormalTok{() }\SpecialCharTok{+} \FunctionTok{theme}\NormalTok{(}\AttributeTok{legend.position =} \StringTok{"blank"}\NormalTok{)}
\end{Highlighting}
\end{Shaded}

\begin{figure}[H]

{\centering \includegraphics{analysis/SGC3A/2_sgc3A_scoring_files/figure-pdf/Q12-distribution-1.pdf}

}

\end{figure}

\begin{Shaded}
\begin{Highlighting}[]
\FunctionTok{gf\_props}\NormalTok{(}\SpecialCharTok{\textasciitilde{}}\NormalTok{interpretation, }\AttributeTok{fill =} \SpecialCharTok{\textasciitilde{}}\NormalTok{condition, }\AttributeTok{data =}\NormalTok{ df\_items }\SpecialCharTok{\%\textgreater{}\%} \FunctionTok{filter}\NormalTok{(q }\SpecialCharTok{==} \DecValTok{12}\NormalTok{)) }\SpecialCharTok{\%\textgreater{}\%} 
  \FunctionTok{gf\_facet\_grid}\NormalTok{( condition }\SpecialCharTok{\textasciitilde{}}\NormalTok{ ., }\AttributeTok{labeller =}\NormalTok{ label\_both) }\SpecialCharTok{+} 
  \FunctionTok{labs}\NormalTok{( }\AttributeTok{x =} \StringTok{"Interpretation"}\NormalTok{, }\AttributeTok{title =} \StringTok{"Distribution of Interpretations | Q12 "}\NormalTok{) }\SpecialCharTok{+} 
  \FunctionTok{theme\_minimal}\NormalTok{() }\SpecialCharTok{+} \FunctionTok{theme}\NormalTok{(}\AttributeTok{legend.position =} \StringTok{"blank"}\NormalTok{)}
\end{Highlighting}
\end{Shaded}

\begin{figure}[H]

{\centering \includegraphics{analysis/SGC3A/2_sgc3A_scoring_files/figure-pdf/Q12-distribution-2.pdf}

}

\end{figure}

\hypertarget{question-13}{%
\subsubsection{Question \#13}\label{question-13}}

\begin{figure}

{\centering \includegraphics{analysis/SGC3A/static/questions/Q13.png}

}

\caption{\label{fig-Q13}Q13-Question}

\end{figure}

\begin{Shaded}
\begin{Highlighting}[]
\NormalTok{q }\OtherTok{\textless{}{-}}\NormalTok{ keys\_raw }\SpecialCharTok{\%\textgreater{}\%} \FunctionTok{filter}\NormalTok{(Q}\SpecialCharTok{==}\DecValTok{13}\NormalTok{)}
\NormalTok{ignore }\OtherTok{\textless{}{-}}\NormalTok{ q }\SpecialCharTok{\%\textgreater{}\%} \FunctionTok{select}\NormalTok{(}\StringTok{"REF\_POINT"}\NormalTok{)}
\NormalTok{answers }\OtherTok{\textless{}{-}}\NormalTok{ q }\SpecialCharTok{\%\textgreater{}\%} \FunctionTok{select}\NormalTok{(}\StringTok{"TRIANGULAR"}\NormalTok{, }\StringTok{"ORTHOGONAL"}\NormalTok{, }\StringTok{"SATISFICE\_left"}\NormalTok{, }\StringTok{"SATISFICE\_right"}\NormalTok{,}\StringTok{"TV\_max"}\NormalTok{,}\StringTok{"TV\_start"}\NormalTok{, }\StringTok{"TV\_end"}\NormalTok{, }\StringTok{"TV\_dur"}\NormalTok{) }\SpecialCharTok{\%\textgreater{}\%} \FunctionTok{unlist}\NormalTok{()}
\NormalTok{ves }\OtherTok{\textless{}{-}}\NormalTok{ q }\SpecialCharTok{\%\textgreater{}\%} \FunctionTok{mutate}\NormalTok{(}
  \AttributeTok{SATISFICE\_left\_allow =} \StringTok{""}\NormalTok{,}
  \AttributeTok{SATISFICE\_right\_allow =} \StringTok{""}
\NormalTok{) }\SpecialCharTok{\%\textgreater{}\%} \FunctionTok{select}\NormalTok{(}\StringTok{"TRI\_allow"}\NormalTok{, }\StringTok{"ORTH\_allow"}\NormalTok{, }\StringTok{"SATISFICE\_left\_allow"}\NormalTok{,}\StringTok{"SATISFICE\_right\_allow"}\NormalTok{, }\StringTok{"TV\_max\_allow"}\NormalTok{,}\StringTok{"TV\_start\_allow"}\NormalTok{,}\StringTok{"TV\_end\_allow"}\NormalTok{, }\StringTok{"TV\_dur\_allow"}\NormalTok{)}\SpecialCharTok{\%\textgreater{}\%} \FunctionTok{unlist}\NormalTok{()}
\NormalTok{options }\OtherTok{\textless{}{-}}\NormalTok{ q }\SpecialCharTok{\%\textgreater{}\%} \FunctionTok{select}\NormalTok{(}\StringTok{"OPTIONS"}\NormalTok{)}
\NormalTok{question }\OtherTok{=}\NormalTok{ q }\SpecialCharTok{\%\textgreater{}\%}  \FunctionTok{select}\NormalTok{(}\StringTok{"TEXT"}\NormalTok{)}
\NormalTok{scores }\OtherTok{\textless{}{-}} \FunctionTok{c}\NormalTok{(}\StringTok{"Triangular"}\NormalTok{, }\StringTok{"Orthgonal"}\NormalTok{, }\StringTok{"Satisficing [left]"}\NormalTok{, }\StringTok{"Satisficing [right]"}\NormalTok{, }\StringTok{"Tversky [maximal]"}\NormalTok{, }\StringTok{"Tversky [start diagonal]"}\NormalTok{,}
            \StringTok{"Tversky [end diagonal]"}\NormalTok{, }\StringTok{"Tversky [duration line]"}\NormalTok{)}
\NormalTok{d }\OtherTok{=} \FunctionTok{tibble}\NormalTok{(}\AttributeTok{interpretation =}\NormalTok{ scores, }\AttributeTok{answer =}\NormalTok{ answers, }\AttributeTok{allowed=}\NormalTok{ves)}
\NormalTok{d}\SpecialCharTok{$}\NormalTok{answer }\OtherTok{\textless{}{-}} \FunctionTok{replace\_na}\NormalTok{(d}\SpecialCharTok{$}\NormalTok{answer, }\StringTok{""}\NormalTok{)}
\NormalTok{d}\SpecialCharTok{$}\NormalTok{allowed }\OtherTok{\textless{}{-}} \FunctionTok{replace\_na}\NormalTok{(d}\SpecialCharTok{$}\NormalTok{allowed, }\StringTok{""}\NormalTok{)}

\NormalTok{title }\OtherTok{=} \FunctionTok{paste}\NormalTok{(}\StringTok{"Answer Key | Q : "}\NormalTok{, question)}
\NormalTok{cols }\OtherTok{=} \FunctionTok{c}\NormalTok{(}\StringTok{"interpretation"}\NormalTok{, }\StringTok{"answer"}\NormalTok{,}\StringTok{"not penalized"}\NormalTok{)}

\NormalTok{d }\SpecialCharTok{\%\textgreater{}\%} \FunctionTok{kbl}\NormalTok{(}\AttributeTok{caption =}\NormalTok{ title, }\AttributeTok{col.names =}\NormalTok{ cols) }\SpecialCharTok{\%\textgreater{}\%} \FunctionTok{kable\_classic}\NormalTok{() }\SpecialCharTok{\%\textgreater{}\%}
  \FunctionTok{footnote}\NormalTok{(}\AttributeTok{general =} \FunctionTok{paste}\NormalTok{(}\StringTok{"15 response options: "}\NormalTok{, options), }\AttributeTok{general\_title =} \StringTok{"Note: "}\NormalTok{,}\AttributeTok{footnote\_as\_chunk =}\NormalTok{ T)}
\end{Highlighting}
\end{Shaded}

\begin{table}

\caption{Answer Key | Q :  Which 2 shifts end when Z begins?}
\centering
\begin{tabular}[t]{l|l|l}
\hline
interpretation & answer & not penalized\\
\hline
Triangular & EF & \\
\hline
Orthgonal & FX & \\
\hline
Satisficing [left] &  & \\
\hline
Satisficing [right] &  & \\
\hline
Tversky [maximal] &  & \\
\hline
Tversky [start diagonal] &  & \\
\hline
Tversky [end diagonal] &  & \\
\hline
Tversky [duration line] &  & \\
\hline
\multicolumn{3}{l}{\rule{0pt}{1em}\textit{Note: } 15 response options:  ABCDEFGHIJKLMNOPZX}\\
\end{tabular}
\end{table}

\begin{Shaded}
\begin{Highlighting}[]
\NormalTok{title }\OtherTok{\textless{}{-}} \StringTok{"Frequency of Selected Response Options for Question \#13"}
\NormalTok{names }\OtherTok{=} \FunctionTok{c}\NormalTok{(}\StringTok{"response"}\NormalTok{,}\StringTok{"n"}\NormalTok{,}\StringTok{"interpretation"}\NormalTok{,}\StringTok{"absolute"}\NormalTok{,}\StringTok{"tri"}\NormalTok{,}\StringTok{"tversky"}\NormalTok{,}\StringTok{"satisfice"}\NormalTok{,}\StringTok{"orthogonal"}\NormalTok{, }\StringTok{"scaled score"}\NormalTok{)}

\NormalTok{df\_items }\SpecialCharTok{\%\textgreater{}\%} \FunctionTok{filter}\NormalTok{(q }\SpecialCharTok{==} \DecValTok{13}\NormalTok{) }\SpecialCharTok{\%\textgreater{}\%} \FunctionTok{group\_by}\NormalTok{(response) }\SpecialCharTok{\%\textgreater{}\%}
\NormalTok{  dplyr}\SpecialCharTok{::}\FunctionTok{summarise}\NormalTok{( }\AttributeTok{count =} \FunctionTok{n}\NormalTok{(),}
                    \AttributeTok{nice =} \FunctionTok{unique}\NormalTok{(score\_niceABS),}
                    \AttributeTok{triangular =} \FunctionTok{unique}\NormalTok{(score\_TRI),}
                    \AttributeTok{orthogonal =}  \FunctionTok{unique}\NormalTok{(score\_ORTH),}
                    \AttributeTok{satisficing =}  \FunctionTok{unique}\NormalTok{(score\_SATISFICE),}
                    \AttributeTok{tversky =} \FunctionTok{unique}\NormalTok{(score\_TVERSKY),}
                    \AttributeTok{interpretation =} \FunctionTok{unique}\NormalTok{(int2),}
                    \AttributeTok{scaled =} \FunctionTok{unique}\NormalTok{(score\_SCALED)) }\SpecialCharTok{\%\textgreater{}\%}
  \FunctionTok{arrange}\NormalTok{(interpretation, }\FunctionTok{desc}\NormalTok{(count)) }\SpecialCharTok{\%\textgreater{}\%}
  \FunctionTok{select}\NormalTok{(response, count, interpretation, nice,}
\NormalTok{         triangular, tversky, satisficing, orthogonal, scaled) }\SpecialCharTok{\%\textgreater{}\%}
  \FunctionTok{kbl}\NormalTok{(}\AttributeTok{caption =}\NormalTok{ title, }\AttributeTok{col.names =}\NormalTok{ names) }\SpecialCharTok{\%\textgreater{}\%}  \FunctionTok{kable\_classic}\NormalTok{() }\SpecialCharTok{\%\textgreater{}\%}
  \FunctionTok{add\_header\_above}\NormalTok{(}\FunctionTok{c}\NormalTok{(}\StringTok{" "} \OtherTok{=} \DecValTok{3}\NormalTok{, }\StringTok{"Strict Score"} \OtherTok{=} \DecValTok{1}\NormalTok{, }\StringTok{"Interpretation Scores"}\OtherTok{=}\DecValTok{4}\NormalTok{, }\StringTok{"Discriminant"}\OtherTok{=}\DecValTok{1}\NormalTok{)) }\SpecialCharTok{\%\textgreater{}\%}
  \FunctionTok{pack\_rows}\NormalTok{(}\StringTok{"Triangular"}\NormalTok{, }\DecValTok{1}\NormalTok{, }\DecValTok{3}\NormalTok{) }\SpecialCharTok{\%\textgreater{}\%}
  \FunctionTok{pack\_rows}\NormalTok{(}\StringTok{"Orthogonal"}\NormalTok{, }\DecValTok{4}\NormalTok{, }\DecValTok{13}\NormalTok{) }\SpecialCharTok{\%\textgreater{}\%}
  \FunctionTok{pack\_rows}\NormalTok{(}\StringTok{"Other"}\NormalTok{, }\DecValTok{14}\NormalTok{, }\DecValTok{14}\NormalTok{) }\SpecialCharTok{\%\textgreater{}\%}
  \FunctionTok{pack\_rows}\NormalTok{(}\StringTok{"Unknown"}\NormalTok{, }\DecValTok{15}\NormalTok{, }\DecValTok{36}\NormalTok{)}
\end{Highlighting}
\end{Shaded}

\textbackslash begin\{table\}

\textbackslash caption\{\label{tab:Q13-RESPONSES}Frequency of Selected
Response Options for Question \#13\} \centering

\begin{tabular}[t]{l|r|l|r|r|r|r|r|r}
\hline
\multicolumn{3}{c|}{ } & \multicolumn{1}{c|}{Strict Score} & \multicolumn{4}{c|}{Interpretation Scores} & \multicolumn{1}{c}{Discriminant} \\
\cline{4-4} \cline{5-8} \cline{9-9}
response & n & interpretation & absolute & tri & tversky & satisfice & orthogonal & scaled score\\
\hline
\multicolumn{9}{l}{\textbf{Triangular}}\\
\hline
\hspace{1em}EF & 91 & Triangular & 1 & 1.000 & NA & NA & 0.433 & 1.0\\
\hline
\hspace{1em}CE & 1 & Triangular & 0 & 0.433 & NA & NA & -0.133 & 1.0\\
\hline
\hspace{1em}E & 1 & Triangular & 0 & 0.500 & NA & NA & -0.067 & 1.0\\
\hline
\multicolumn{9}{l}{\textbf{Orthogonal}}\\
\hline
\hspace{1em}FX & 141 & Orthogonal & 0 & 0.433 & NA & NA & 1.000 & -1.0\\
\hline
\hspace{1em}X & 9 & Orthogonal & 0 & -0.067 & NA & NA & 0.500 & -1.0\\
\hline
\hspace{1em}OX & 4 & Orthogonal & 0 & -0.133 & NA & NA & 0.433 & -1.0\\
\hline
\hspace{1em}KX & 3 & Orthogonal & 0 & -0.133 & NA & NA & 0.433 & -1.0\\
\hline
\hspace{1em}ACX & 1 & Orthogonal & 0 & -0.200 & NA & NA & 0.367 & -1.0\\
\hline
\hspace{1em}BX & 1 & Orthogonal & 0 & -0.133 & NA & NA & 0.433 & -1.0\\
\hline
\hspace{1em}CX & 1 & Orthogonal & 0 & -0.133 & NA & NA & 0.433 & -1.0\\
\hline
\hspace{1em}DJNX & 1 & Orthogonal & 0 & -0.267 & NA & NA & 0.300 & -1.0\\
\hline
\hspace{1em}GX & 1 & Orthogonal & 0 & -0.133 & NA & NA & 0.433 & -1.0\\
\hline
\hspace{1em}JX & 1 & Orthogonal & 0 & -0.133 & NA & NA & 0.433 & -1.0\\
\hline
\multicolumn{9}{l}{\textbf{Other}}\\
\hline
\hspace{1em} & 5 & blank & 0 & 0.000 & NA & NA & 0.000 & 0.0\\
\hline
\multicolumn{9}{l}{\textbf{Unknown}}\\
\hline
\hspace{1em}HN & 13 & ? & 0 & -0.133 & NA & NA & -0.133 & -0.5\\
\hline
\hspace{1em}BF & 11 & ? & 0 & 0.433 & NA & NA & 0.433 & -0.5\\
\hline
\hspace{1em}F & 10 & ? & 0 & 0.500 & NA & NA & 0.500 & -0.5\\
\hline
\hspace{1em}EX & 6 & ? & 0 & 0.433 & NA & NA & 0.433 & -0.5\\
\hline
\hspace{1em}HL & 5 & ? & 0 & -0.133 & NA & NA & -0.133 & -0.5\\
\hline
\hspace{1em}HLP & 5 & ? & 0 & -0.200 & NA & NA & -0.200 & -0.5\\
\hline
\hspace{1em}BM & 2 & ? & 0 & -0.133 & NA & NA & -0.133 & -0.5\\
\hline
\hspace{1em}CO & 2 & ? & 0 & -0.133 & NA & NA & -0.133 & -0.5\\
\hline
\hspace{1em}DN & 2 & ? & 0 & -0.133 & NA & NA & -0.133 & -0.5\\
\hline
\hspace{1em}AG & 1 & ? & 0 & -0.133 & NA & NA & -0.133 & -0.5\\
\hline
\hspace{1em}BO & 1 & ? & 0 & -0.133 & NA & NA & -0.133 & -0.5\\
\hline
\hspace{1em}C & 1 & ? & 0 & -0.067 & NA & NA & -0.067 & -0.5\\
\hline
\hspace{1em}CG & 1 & ? & 0 & -0.133 & NA & NA & -0.133 & -0.5\\
\hline
\hspace{1em}CGO & 1 & ? & 0 & -0.200 & NA & NA & -0.200 & -0.5\\
\hline
\hspace{1em}CH & 1 & ? & 0 & -0.133 & NA & NA & -0.133 & -0.5\\
\hline
\hspace{1em}D & 1 & ? & 0 & -0.067 & NA & NA & -0.067 & -0.5\\
\hline
\hspace{1em}DKM & 1 & ? & 0 & -0.200 & NA & NA & -0.200 & -0.5\\
\hline
\hspace{1em}H & 1 & ? & 0 & -0.067 & NA & NA & -0.067 & -0.5\\
\hline
\hspace{1em}HZ & 1 & ? & 0 & -0.067 & NA & NA & -0.067 & -0.5\\
\hline
\hspace{1em}LP & 1 & ? & 0 & -0.133 & NA & NA & -0.133 & -0.5\\
\hline
\hspace{1em}NO & 1 & ? & 0 & -0.133 & NA & NA & -0.133 & -0.5\\
\hline
\hspace{1em}NZ & 1 & ? & 0 & -0.067 & NA & NA & -0.067 & -0.5\\
\hline
\end{tabular}

\textbackslash end\{table\}

\begin{Shaded}
\begin{Highlighting}[]
\FunctionTok{gf\_dhistogram}\NormalTok{(}\SpecialCharTok{\textasciitilde{}}\NormalTok{ score\_niceABS, }\AttributeTok{fill =} \SpecialCharTok{\textasciitilde{}}\NormalTok{condition, }\AttributeTok{data =}\NormalTok{ df\_items }\SpecialCharTok{\%\textgreater{}\%} \FunctionTok{filter}\NormalTok{(q }\SpecialCharTok{==} \DecValTok{13}\NormalTok{)) }\SpecialCharTok{\%\textgreater{}\%} 
  \FunctionTok{gf\_facet\_grid}\NormalTok{( condition }\SpecialCharTok{\textasciitilde{}}\NormalTok{ ., }\AttributeTok{labeller =}\NormalTok{ label\_both) }\SpecialCharTok{+} 
  \FunctionTok{labs}\NormalTok{( }\AttributeTok{x =} \StringTok{"Scaled Item Score"}\NormalTok{, }\AttributeTok{title =} \StringTok{"Distribution of Scaled Scores | Q13 "}\NormalTok{) }\SpecialCharTok{+} 
  \FunctionTok{theme\_minimal}\NormalTok{() }\SpecialCharTok{+} \FunctionTok{theme}\NormalTok{(}\AttributeTok{legend.position =} \StringTok{"blank"}\NormalTok{)}
\end{Highlighting}
\end{Shaded}

\begin{figure}[H]

{\centering \includegraphics{analysis/SGC3A/2_sgc3A_scoring_files/figure-pdf/Q13-distribution-1.pdf}

}

\end{figure}

\begin{Shaded}
\begin{Highlighting}[]
\FunctionTok{gf\_props}\NormalTok{(}\SpecialCharTok{\textasciitilde{}}\NormalTok{interpretation, }\AttributeTok{fill =} \SpecialCharTok{\textasciitilde{}}\NormalTok{condition, }\AttributeTok{data =}\NormalTok{ df\_items }\SpecialCharTok{\%\textgreater{}\%} \FunctionTok{filter}\NormalTok{(q }\SpecialCharTok{==} \DecValTok{13}\NormalTok{)) }\SpecialCharTok{\%\textgreater{}\%}
  \FunctionTok{gf\_facet\_grid}\NormalTok{( condition }\SpecialCharTok{\textasciitilde{}}\NormalTok{ ., }\AttributeTok{labeller =}\NormalTok{ label\_both) }\SpecialCharTok{+} 
  \FunctionTok{labs}\NormalTok{( }\AttributeTok{x =} \StringTok{"Interpretation"}\NormalTok{, }\AttributeTok{title =} \StringTok{"Distribution of Interpretations | Q13 "}\NormalTok{) }\SpecialCharTok{+} 
  \FunctionTok{theme\_minimal}\NormalTok{() }\SpecialCharTok{+} \FunctionTok{theme}\NormalTok{(}\AttributeTok{legend.position =} \StringTok{"blank"}\NormalTok{)}
\end{Highlighting}
\end{Shaded}

\begin{figure}[H]

{\centering \includegraphics{analysis/SGC3A/2_sgc3A_scoring_files/figure-pdf/Q13-distribution-2.pdf}

}

\end{figure}

\hypertarget{question-14}{%
\subsubsection{Question \#14}\label{question-14}}

\begin{figure}

{\centering \includegraphics{analysis/SGC3A/static/questions/Q14.png}

}

\caption{\label{fig-Q14}Q14-Question}

\end{figure}

\begin{Shaded}
\begin{Highlighting}[]
\NormalTok{q }\OtherTok{\textless{}{-}}\NormalTok{ keys\_raw }\SpecialCharTok{\%\textgreater{}\%} \FunctionTok{filter}\NormalTok{(Q}\SpecialCharTok{==}\DecValTok{14}\NormalTok{)}
\NormalTok{ignore }\OtherTok{\textless{}{-}}\NormalTok{ q }\SpecialCharTok{\%\textgreater{}\%} \FunctionTok{select}\NormalTok{(}\StringTok{"REF\_POINT"}\NormalTok{)}
\NormalTok{answers }\OtherTok{\textless{}{-}}\NormalTok{ q }\SpecialCharTok{\%\textgreater{}\%} \FunctionTok{select}\NormalTok{(}\StringTok{"TRIANGULAR"}\NormalTok{, }\StringTok{"ORTHOGONAL"}\NormalTok{, }\StringTok{"SATISFICE\_left"}\NormalTok{, }\StringTok{"SATISFICE\_right"}\NormalTok{,}\StringTok{"TV\_max"}\NormalTok{,}\StringTok{"TV\_start"}\NormalTok{, }\StringTok{"TV\_end"}\NormalTok{, }\StringTok{"TV\_dur"}\NormalTok{) }\SpecialCharTok{\%\textgreater{}\%} \FunctionTok{unlist}\NormalTok{()}
\NormalTok{ves }\OtherTok{\textless{}{-}}\NormalTok{ q }\SpecialCharTok{\%\textgreater{}\%} \FunctionTok{mutate}\NormalTok{(}
  \AttributeTok{SATISFICE\_left\_allow =} \StringTok{""}\NormalTok{,}
  \AttributeTok{SATISFICE\_right\_allow =} \StringTok{""}
\NormalTok{) }\SpecialCharTok{\%\textgreater{}\%} \FunctionTok{select}\NormalTok{(}\StringTok{"TRI\_allow"}\NormalTok{, }\StringTok{"ORTH\_allow"}\NormalTok{, }\StringTok{"SATISFICE\_left\_allow"}\NormalTok{,}\StringTok{"SATISFICE\_right\_allow"}\NormalTok{, }\StringTok{"TV\_max\_allow"}\NormalTok{,}\StringTok{"TV\_start\_allow"}\NormalTok{,}\StringTok{"TV\_end\_allow"}\NormalTok{, }\StringTok{"TV\_dur\_allow"}\NormalTok{)}\SpecialCharTok{\%\textgreater{}\%} \FunctionTok{unlist}\NormalTok{()}
\NormalTok{options }\OtherTok{\textless{}{-}}\NormalTok{ q }\SpecialCharTok{\%\textgreater{}\%} \FunctionTok{select}\NormalTok{(}\StringTok{"OPTIONS"}\NormalTok{)}
\NormalTok{question }\OtherTok{=}\NormalTok{ q }\SpecialCharTok{\%\textgreater{}\%}  \FunctionTok{select}\NormalTok{(}\StringTok{"TEXT"}\NormalTok{)}
\NormalTok{scores }\OtherTok{\textless{}{-}} \FunctionTok{c}\NormalTok{(}\StringTok{"Triangular"}\NormalTok{, }\StringTok{"Orthgonal"}\NormalTok{, }\StringTok{"Satisficing [left]"}\NormalTok{, }\StringTok{"Satisficing [right]"}\NormalTok{, }\StringTok{"Tversky [maximal]"}\NormalTok{, }\StringTok{"Tversky [start diagonal]"}\NormalTok{,}
            \StringTok{"Tversky [end diagonal]"}\NormalTok{, }\StringTok{"Tversky [duration line]"}\NormalTok{)}
\NormalTok{d }\OtherTok{=} \FunctionTok{tibble}\NormalTok{(}\AttributeTok{interpretation =}\NormalTok{ scores, }\AttributeTok{answer =}\NormalTok{ answers, }\AttributeTok{allowed=}\NormalTok{ves)}
\NormalTok{d}\SpecialCharTok{$}\NormalTok{answer }\OtherTok{\textless{}{-}} \FunctionTok{replace\_na}\NormalTok{(d}\SpecialCharTok{$}\NormalTok{answer, }\StringTok{""}\NormalTok{)}
\NormalTok{d}\SpecialCharTok{$}\NormalTok{allowed }\OtherTok{\textless{}{-}} \FunctionTok{replace\_na}\NormalTok{(d}\SpecialCharTok{$}\NormalTok{allowed, }\StringTok{""}\NormalTok{)}

\NormalTok{title }\OtherTok{=} \FunctionTok{paste}\NormalTok{(}\StringTok{"Answer Key | Q : "}\NormalTok{, question)}
\NormalTok{cols }\OtherTok{=} \FunctionTok{c}\NormalTok{(}\StringTok{"interpretation"}\NormalTok{, }\StringTok{"answer"}\NormalTok{,}\StringTok{"not penalized"}\NormalTok{)}

\NormalTok{d }\SpecialCharTok{\%\textgreater{}\%} \FunctionTok{kbl}\NormalTok{(}\AttributeTok{caption =}\NormalTok{ title, }\AttributeTok{col.names =}\NormalTok{ cols) }\SpecialCharTok{\%\textgreater{}\%} \FunctionTok{kable\_classic}\NormalTok{() }\SpecialCharTok{\%\textgreater{}\%}
  \FunctionTok{footnote}\NormalTok{(}\AttributeTok{general =} \FunctionTok{paste}\NormalTok{(}\StringTok{"15 response options: "}\NormalTok{, options), }\AttributeTok{general\_title =} \StringTok{"Note: "}\NormalTok{,}\AttributeTok{footnote\_as\_chunk =}\NormalTok{ T)}
\end{Highlighting}
\end{Shaded}

\begin{table}

\caption{Answer Key | Q :  Which shift(s) end at 3pm?}
\centering
\begin{tabular}[t]{l|l|l}
\hline
interpretation & answer & not penalized\\
\hline
Triangular & X & \\
\hline
Orthgonal & B & \\
\hline
Satisficing [left] &  & \\
\hline
Satisficing [right] &  & \\
\hline
Tversky [maximal] & XJND & \\
\hline
Tversky [start diagonal] &  & \\
\hline
Tversky [end diagonal] & X & \\
\hline
Tversky [duration line] & JND & \\
\hline
\multicolumn{3}{l}{\rule{0pt}{1em}\textit{Note: } 15 response options:  ABCDEFGHIJKLMNOPZX}\\
\end{tabular}
\end{table}

\begin{Shaded}
\begin{Highlighting}[]
\NormalTok{title }\OtherTok{\textless{}{-}} \StringTok{"Frequency of Selected Response Options for Question \#14"}
\NormalTok{names }\OtherTok{=} \FunctionTok{c}\NormalTok{(}\StringTok{"response"}\NormalTok{,}\StringTok{"n"}\NormalTok{,}\StringTok{"interpretation"}\NormalTok{,}\StringTok{"absolute"}\NormalTok{,}\StringTok{"tri"}\NormalTok{,}\StringTok{"tversky"}\NormalTok{,}\StringTok{"satisfice"}\NormalTok{,}\StringTok{"orthogonal"}\NormalTok{, }\StringTok{"scaled score"}\NormalTok{)}

\NormalTok{df\_items }\SpecialCharTok{\%\textgreater{}\%} \FunctionTok{filter}\NormalTok{(q }\SpecialCharTok{==} \DecValTok{14}\NormalTok{) }\SpecialCharTok{\%\textgreater{}\%} \FunctionTok{group\_by}\NormalTok{(response) }\SpecialCharTok{\%\textgreater{}\%}
\NormalTok{  dplyr}\SpecialCharTok{::}\FunctionTok{summarise}\NormalTok{( }\AttributeTok{count =} \FunctionTok{n}\NormalTok{(),}
                    \AttributeTok{nice =} \FunctionTok{unique}\NormalTok{(score\_niceABS),}
                    \AttributeTok{triangular =} \FunctionTok{unique}\NormalTok{(score\_TRI),}
                    \AttributeTok{orthogonal =}  \FunctionTok{unique}\NormalTok{(score\_ORTH),}
                    \AttributeTok{satisficing =}  \FunctionTok{unique}\NormalTok{(score\_SATISFICE),}
                    \AttributeTok{tversky =} \FunctionTok{unique}\NormalTok{(score\_TVERSKY),}
                    \AttributeTok{interpretation =} \FunctionTok{unique}\NormalTok{(int2),}
                    \AttributeTok{scaled =} \FunctionTok{unique}\NormalTok{(score\_SCALED)) }\SpecialCharTok{\%\textgreater{}\%}
  \FunctionTok{arrange}\NormalTok{(interpretation, }\FunctionTok{desc}\NormalTok{(count)) }\SpecialCharTok{\%\textgreater{}\%}
  \FunctionTok{select}\NormalTok{(response, count, interpretation, nice,}
\NormalTok{         triangular, tversky, satisficing, orthogonal, scaled) }\SpecialCharTok{\%\textgreater{}\%}
  \FunctionTok{kbl}\NormalTok{(}\AttributeTok{caption =}\NormalTok{ title, }\AttributeTok{col.names =}\NormalTok{ names) }\SpecialCharTok{\%\textgreater{}\%}  \FunctionTok{kable\_classic}\NormalTok{() }\SpecialCharTok{\%\textgreater{}\%}
  \FunctionTok{add\_header\_above}\NormalTok{(}\FunctionTok{c}\NormalTok{(}\StringTok{" "} \OtherTok{=} \DecValTok{3}\NormalTok{, }\StringTok{"Strict Score"} \OtherTok{=} \DecValTok{1}\NormalTok{, }\StringTok{"Interpretation Scores"}\OtherTok{=}\DecValTok{4}\NormalTok{, }\StringTok{"Discriminant"}\OtherTok{=}\DecValTok{1}\NormalTok{)) }\SpecialCharTok{\%\textgreater{}\%}
  \FunctionTok{pack\_rows}\NormalTok{(}\StringTok{"Triangular"}\NormalTok{, }\DecValTok{1}\NormalTok{, }\DecValTok{4}\NormalTok{) }\SpecialCharTok{\%\textgreater{}\%}
  \FunctionTok{pack\_rows}\NormalTok{(}\StringTok{"Orthogonal"}\NormalTok{, }\DecValTok{5}\NormalTok{, }\DecValTok{7}\NormalTok{) }\SpecialCharTok{\%\textgreater{}\%}
  \FunctionTok{pack\_rows}\NormalTok{(}\StringTok{"Other"}\NormalTok{, }\DecValTok{8}\NormalTok{, }\DecValTok{9}\NormalTok{) }\SpecialCharTok{\%\textgreater{}\%}
  \FunctionTok{pack\_rows}\NormalTok{(}\StringTok{"Unknown"}\NormalTok{, }\DecValTok{10}\NormalTok{, }\DecValTok{22}\NormalTok{)}
\end{Highlighting}
\end{Shaded}

\textbackslash begin\{table\}

\textbackslash caption\{\label{tab:Q14-RESPONSES}Frequency of Selected
Response Options for Question \#14\} \centering

\begin{tabular}[t]{l|r|l|r|r|r|r|r|r}
\hline
\multicolumn{3}{c|}{ } & \multicolumn{1}{c|}{Strict Score} & \multicolumn{4}{c|}{Interpretation Scores} & \multicolumn{1}{c}{Discriminant} \\
\cline{4-4} \cline{5-8} \cline{9-9}
response & n & interpretation & absolute & tri & tversky & satisfice & orthogonal & scaled score\\
\hline
\multicolumn{9}{l}{\textbf{Triangular}}\\
\hline
\hspace{1em}X & 107 & Triangular & 1 & 1.000 & 1.000 & NA & -0.059 & 1.0\\
\hline
\hspace{1em}FX & 2 & Triangular & 0 & 0.941 & 0.941 & NA & -0.118 & 1.0\\
\hline
\hspace{1em}EX & 1 & Triangular & 0 & 0.941 & 0.941 & NA & -0.118 & 1.0\\
\hline
\hspace{1em}OX & 1 & Triangular & 0 & 0.941 & 0.941 & NA & -0.118 & 1.0\\
\hline
\multicolumn{9}{l}{\textbf{Orthogonal}}\\
\hline
\hspace{1em}B & 150 & Orthogonal & 0 & -0.059 & -0.059 & NA & 1.000 & -1.0\\
\hline
\hspace{1em}BF & 12 & Orthogonal & 0 & -0.118 & -0.118 & NA & 0.941 & -1.0\\
\hline
\hspace{1em}BIO & 2 & Orthogonal & 0 & -0.176 & -0.176 & NA & 0.882 & -1.0\\
\hline
\multicolumn{9}{l}{\textbf{Other}}\\
\hline
\hspace{1em} & 29 & blank & 0 & 0.000 & 0.000 & NA & 0.000 & 0.0\\
\hline
\hspace{1em}O & 5 & ? & 0 & -0.059 & -0.059 & NA & -0.059 & -0.5\\
\hline
\multicolumn{9}{l}{\textbf{Unknown}}\\
\hline
\hspace{1em}F & 3 & ? & 0 & -0.059 & -0.059 & NA & -0.059 & -0.5\\
\hline
\hspace{1em}G & 3 & ? & 0 & -0.059 & -0.059 & NA & -0.059 & -0.5\\
\hline
\hspace{1em}A & 2 & ? & 0 & -0.059 & -0.059 & NA & -0.059 & -0.5\\
\hline
\hspace{1em}BX & 2 & ? & 0 & 0.941 & 0.941 & NA & 0.941 & -0.5\\
\hline
\hspace{1em}HLP & 2 & ? & 0 & -0.176 & -0.176 & NA & -0.176 & -0.5\\
\hline
\hspace{1em}K & 2 & ? & 0 & -0.059 & -0.059 & NA & -0.059 & -0.5\\
\hline
\hspace{1em}AH & 1 & ? & 0 & -0.118 & -0.118 & NA & -0.118 & -0.5\\
\hline
\hspace{1em}DHO & 1 & ? & 0 & -0.176 & 0.200 & NA & -0.176 & -0.5\\
\hline
\hspace{1em}FG & 1 & ? & 0 & -0.118 & -0.118 & NA & -0.118 & -0.5\\
\hline
\hspace{1em}HL & 1 & ? & 0 & -0.118 & -0.118 & NA & -0.118 & -0.5\\
\hline
\hspace{1em}IJ & 1 & ? & 0 & -0.118 & 0.267 & NA & -0.118 & -0.5\\
\hline
\hspace{1em}M & 1 & ? & 0 & -0.059 & -0.059 & NA & -0.059 & -0.5\\
\hline
\hspace{1em}P & 1 & ? & 0 & -0.059 & -0.059 & NA & -0.059 & -0.5\\
\hline
\end{tabular}

\textbackslash end\{table\}

\begin{Shaded}
\begin{Highlighting}[]
\FunctionTok{gf\_dhistogram}\NormalTok{(}\SpecialCharTok{\textasciitilde{}}\NormalTok{ score\_niceABS, }\AttributeTok{fill =} \SpecialCharTok{\textasciitilde{}}\NormalTok{condition, }\AttributeTok{data =}\NormalTok{ df\_items }\SpecialCharTok{\%\textgreater{}\%} \FunctionTok{filter}\NormalTok{(q }\SpecialCharTok{==} \DecValTok{14}\NormalTok{)) }\SpecialCharTok{\%\textgreater{}\%} 
  \FunctionTok{gf\_facet\_grid}\NormalTok{( condition }\SpecialCharTok{\textasciitilde{}}\NormalTok{ ., }\AttributeTok{labeller =}\NormalTok{ label\_both) }\SpecialCharTok{+} 
  \FunctionTok{labs}\NormalTok{( }\AttributeTok{x =} \StringTok{"Scaled Item Score"}\NormalTok{, }\AttributeTok{title =} \StringTok{"Distribution of Scaled Scores | Q14 "}\NormalTok{) }\SpecialCharTok{+} 
  \FunctionTok{theme\_minimal}\NormalTok{() }\SpecialCharTok{+} \FunctionTok{theme}\NormalTok{(}\AttributeTok{legend.position =} \StringTok{"blank"}\NormalTok{)}
\end{Highlighting}
\end{Shaded}

\begin{figure}[H]

{\centering \includegraphics{analysis/SGC3A/2_sgc3A_scoring_files/figure-pdf/Q14-distribution-1.pdf}

}

\end{figure}

\begin{Shaded}
\begin{Highlighting}[]
\FunctionTok{gf\_props}\NormalTok{(}\SpecialCharTok{\textasciitilde{}}\NormalTok{interpretation, }\AttributeTok{fill =} \SpecialCharTok{\textasciitilde{}}\NormalTok{condition, }\AttributeTok{data =}\NormalTok{ df\_items }\SpecialCharTok{\%\textgreater{}\%} \FunctionTok{filter}\NormalTok{(q }\SpecialCharTok{==} \DecValTok{14}\NormalTok{)) }\SpecialCharTok{\%\textgreater{}\%}
  \FunctionTok{gf\_facet\_grid}\NormalTok{( condition }\SpecialCharTok{\textasciitilde{}}\NormalTok{ ., }\AttributeTok{labeller =}\NormalTok{ label\_both) }\SpecialCharTok{+} 
  \FunctionTok{labs}\NormalTok{( }\AttributeTok{x =} \StringTok{"Interpretation"}\NormalTok{, }\AttributeTok{title =} \StringTok{"Distribution of Interpretations | Q14 "}\NormalTok{) }\SpecialCharTok{+} 
  \FunctionTok{theme\_minimal}\NormalTok{() }\SpecialCharTok{+} \FunctionTok{theme}\NormalTok{(}\AttributeTok{legend.position =} \StringTok{"blank"}\NormalTok{)}
\end{Highlighting}
\end{Shaded}

\begin{figure}[H]

{\centering \includegraphics{analysis/SGC3A/2_sgc3A_scoring_files/figure-pdf/Q14-distribution-2.pdf}

}

\end{figure}

\hypertarget{question-15}{%
\subsubsection{Question \#15}\label{question-15}}

\begin{figure}

{\centering \includegraphics{analysis/SGC3A/static/questions/Q15.png}

}

\caption{\label{fig-Q15}Q15-Question}

\end{figure}

\begin{Shaded}
\begin{Highlighting}[]
\NormalTok{q }\OtherTok{\textless{}{-}}\NormalTok{ keys\_raw }\SpecialCharTok{\%\textgreater{}\%} \FunctionTok{filter}\NormalTok{(Q}\SpecialCharTok{==}\DecValTok{15}\NormalTok{)}
\NormalTok{ignore }\OtherTok{\textless{}{-}}\NormalTok{ q }\SpecialCharTok{\%\textgreater{}\%} \FunctionTok{select}\NormalTok{(}\StringTok{"REF\_POINT"}\NormalTok{)}
\NormalTok{answers }\OtherTok{\textless{}{-}}\NormalTok{ q }\SpecialCharTok{\%\textgreater{}\%} \FunctionTok{select}\NormalTok{(}\StringTok{"TRIANGULAR"}\NormalTok{, }\StringTok{"ORTHOGONAL"}\NormalTok{, }\StringTok{"SATISFICE\_left"}\NormalTok{, }\StringTok{"SATISFICE\_right"}\NormalTok{,}\StringTok{"TV\_max"}\NormalTok{,}\StringTok{"TV\_start"}\NormalTok{, }\StringTok{"TV\_end"}\NormalTok{, }\StringTok{"TV\_dur"}\NormalTok{) }\SpecialCharTok{\%\textgreater{}\%} \FunctionTok{unlist}\NormalTok{()}
\NormalTok{ves }\OtherTok{\textless{}{-}}\NormalTok{ q }\SpecialCharTok{\%\textgreater{}\%} \FunctionTok{mutate}\NormalTok{(}
  \AttributeTok{SATISFICE\_left\_allow =} \StringTok{""}\NormalTok{,}
  \AttributeTok{SATISFICE\_right\_allow =} \StringTok{""}
\NormalTok{) }\SpecialCharTok{\%\textgreater{}\%} \FunctionTok{select}\NormalTok{(}\StringTok{"TRI\_allow"}\NormalTok{, }\StringTok{"ORTH\_allow"}\NormalTok{, }\StringTok{"SATISFICE\_left\_allow"}\NormalTok{,}\StringTok{"SATISFICE\_right\_allow"}\NormalTok{, }\StringTok{"TV\_max\_allow"}\NormalTok{,}\StringTok{"TV\_start\_allow"}\NormalTok{,}\StringTok{"TV\_end\_allow"}\NormalTok{, }\StringTok{"TV\_dur\_allow"}\NormalTok{)}\SpecialCharTok{\%\textgreater{}\%} \FunctionTok{unlist}\NormalTok{()}
\NormalTok{options }\OtherTok{\textless{}{-}}\NormalTok{ q }\SpecialCharTok{\%\textgreater{}\%} \FunctionTok{select}\NormalTok{(}\StringTok{"OPTIONS"}\NormalTok{)}
\NormalTok{question }\OtherTok{=}\NormalTok{ q }\SpecialCharTok{\%\textgreater{}\%}  \FunctionTok{select}\NormalTok{(}\StringTok{"TEXT"}\NormalTok{)}
\NormalTok{scores }\OtherTok{\textless{}{-}} \FunctionTok{c}\NormalTok{(}\StringTok{"Triangular"}\NormalTok{, }\StringTok{"Orthgonal"}\NormalTok{, }\StringTok{"Satisficing [left]"}\NormalTok{, }\StringTok{"Satisficing [right]"}\NormalTok{, }\StringTok{"Tversky [maximal]"}\NormalTok{, }\StringTok{"Tversky [start diagonal]"}\NormalTok{,}
            \StringTok{"Tversky [end diagonal]"}\NormalTok{, }\StringTok{"Tversky [duration line]"}\NormalTok{)}
\NormalTok{d }\OtherTok{=} \FunctionTok{tibble}\NormalTok{(}\AttributeTok{interpretation =}\NormalTok{ scores, }\AttributeTok{answer =}\NormalTok{ answers, }\AttributeTok{allowed=}\NormalTok{ves)}
\NormalTok{d}\SpecialCharTok{$}\NormalTok{answer }\OtherTok{\textless{}{-}} \FunctionTok{replace\_na}\NormalTok{(d}\SpecialCharTok{$}\NormalTok{answer, }\StringTok{""}\NormalTok{)}
\NormalTok{d}\SpecialCharTok{$}\NormalTok{allowed }\OtherTok{\textless{}{-}} \FunctionTok{replace\_na}\NormalTok{(d}\SpecialCharTok{$}\NormalTok{allowed, }\StringTok{""}\NormalTok{)}

\NormalTok{title }\OtherTok{=} \FunctionTok{paste}\NormalTok{(}\StringTok{"Answer Key | Q : "}\NormalTok{, question)}
\NormalTok{cols }\OtherTok{=} \FunctionTok{c}\NormalTok{(}\StringTok{"interpretation"}\NormalTok{, }\StringTok{"answer"}\NormalTok{,}\StringTok{"not penalized"}\NormalTok{)}

\NormalTok{d }\SpecialCharTok{\%\textgreater{}\%} \FunctionTok{kbl}\NormalTok{(}\AttributeTok{caption =}\NormalTok{ title, }\AttributeTok{col.names =}\NormalTok{ cols) }\SpecialCharTok{\%\textgreater{}\%} \FunctionTok{kable\_classic}\NormalTok{() }\SpecialCharTok{\%\textgreater{}\%}
  \FunctionTok{footnote}\NormalTok{(}\AttributeTok{general =} \FunctionTok{paste}\NormalTok{(}\StringTok{"15 response options: "}\NormalTok{, options), }\AttributeTok{general\_title =} \StringTok{"Note: "}\NormalTok{,}\AttributeTok{footnote\_as\_chunk =}\NormalTok{ T)}
\end{Highlighting}
\end{Shaded}

\begin{table}

\caption{Answer Key | Q :  Coffee breaks happen halfway through a shift. </br>Which shifts share a break at 2pm?}
\centering
\begin{tabular}[t]{l|l|l}
\hline
interpretation & answer & not penalized\\
\hline
Triangular & XK & \\
\hline
Orthgonal & EF & B\\
\hline
Satisficing [left] &  & \\
\hline
Satisficing [right] &  & \\
\hline
Tversky [maximal] & XKZ & \\
\hline
Tversky [start diagonal] & Z & \\
\hline
Tversky [end diagonal] &  & \\
\hline
Tversky [duration line] &  & \\
\hline
\multicolumn{3}{l}{\rule{0pt}{1em}\textit{Note: } 15 response options:  ABCDEFGHIJKLMNOPZX}\\
\end{tabular}
\end{table}

\begin{Shaded}
\begin{Highlighting}[]
\NormalTok{title }\OtherTok{\textless{}{-}} \StringTok{"Frequency of Selected Response Options for Question \#15"}
\NormalTok{names }\OtherTok{=} \FunctionTok{c}\NormalTok{(}\StringTok{"response"}\NormalTok{,}\StringTok{"n"}\NormalTok{,}\StringTok{"interpretation"}\NormalTok{,}\StringTok{"absolute"}\NormalTok{,}\StringTok{"tri"}\NormalTok{,}\StringTok{"tversky"}\NormalTok{,}\StringTok{"satisfice"}\NormalTok{,}\StringTok{"orthogonal"}\NormalTok{, }\StringTok{"scaled score"}\NormalTok{)}

\NormalTok{df\_items }\SpecialCharTok{\%\textgreater{}\%} \FunctionTok{filter}\NormalTok{(q }\SpecialCharTok{==} \DecValTok{15}\NormalTok{) }\SpecialCharTok{\%\textgreater{}\%} \FunctionTok{group\_by}\NormalTok{(response) }\SpecialCharTok{\%\textgreater{}\%}
\NormalTok{  dplyr}\SpecialCharTok{::}\FunctionTok{summarise}\NormalTok{( }\AttributeTok{count =} \FunctionTok{n}\NormalTok{(),}
                    \AttributeTok{nice =} \FunctionTok{unique}\NormalTok{(score\_niceABS),}
                    \AttributeTok{triangular =} \FunctionTok{unique}\NormalTok{(score\_TRI),}
                    \AttributeTok{orthogonal =}  \FunctionTok{unique}\NormalTok{(score\_ORTH),}
                    \AttributeTok{satisficing =}  \FunctionTok{unique}\NormalTok{(score\_SATISFICE),}
                    \AttributeTok{tversky =} \FunctionTok{unique}\NormalTok{(score\_TVERSKY),}
                    \AttributeTok{interpretation =} \FunctionTok{unique}\NormalTok{(int2),}
                    \AttributeTok{scaled =} \FunctionTok{unique}\NormalTok{(score\_SCALED)) }\SpecialCharTok{\%\textgreater{}\%}
  \FunctionTok{arrange}\NormalTok{(interpretation, }\FunctionTok{desc}\NormalTok{(count)) }\SpecialCharTok{\%\textgreater{}\%}
  \FunctionTok{select}\NormalTok{(response, count, interpretation, nice,}
\NormalTok{         triangular, tversky, satisficing, orthogonal, scaled) }\SpecialCharTok{\%\textgreater{}\%}
  \FunctionTok{kbl}\NormalTok{(}\AttributeTok{caption =}\NormalTok{ title, }\AttributeTok{col.names =}\NormalTok{ names) }\SpecialCharTok{\%\textgreater{}\%}  \FunctionTok{kable\_classic}\NormalTok{() }\SpecialCharTok{\%\textgreater{}\%}
  \FunctionTok{add\_header\_above}\NormalTok{(}\FunctionTok{c}\NormalTok{(}\StringTok{" "} \OtherTok{=} \DecValTok{3}\NormalTok{, }\StringTok{"Strict Score"} \OtherTok{=} \DecValTok{1}\NormalTok{, }\StringTok{"Interpretation Scores"}\OtherTok{=}\DecValTok{4}\NormalTok{, }\StringTok{"Discriminant"}\OtherTok{=}\DecValTok{1}\NormalTok{)) }\SpecialCharTok{\%\textgreater{}\%}
  \FunctionTok{pack\_rows}\NormalTok{(}\StringTok{"Triangular"}\NormalTok{, }\DecValTok{1}\NormalTok{, }\DecValTok{10}\NormalTok{) }\SpecialCharTok{\%\textgreater{}\%}
  \FunctionTok{pack\_rows}\NormalTok{(}\StringTok{"Lines{-}Connect"}\NormalTok{, }\DecValTok{11}\NormalTok{, }\DecValTok{13}\NormalTok{) }\SpecialCharTok{\%\textgreater{}\%}
  \FunctionTok{pack\_rows}\NormalTok{(}\StringTok{"Orthogonal"}\NormalTok{, }\DecValTok{14}\NormalTok{, }\DecValTok{22}\NormalTok{) }\SpecialCharTok{\%\textgreater{}\%}
  \FunctionTok{pack\_rows}\NormalTok{(}\StringTok{"Other"}\NormalTok{, }\DecValTok{23}\NormalTok{, }\DecValTok{23}\NormalTok{) }\SpecialCharTok{\%\textgreater{}\%}
  \FunctionTok{pack\_rows}\NormalTok{(}\StringTok{"Unknown"}\NormalTok{, }\DecValTok{24}\NormalTok{, }\DecValTok{44}\NormalTok{)}
\end{Highlighting}
\end{Shaded}

\textbackslash begin\{table\}

\textbackslash caption\{\label{tab:Q15-RESPONSES}Frequency of Selected
Response Options for Question \#15\} \centering

\begin{tabular}[t]{l|r|l|r|r|r|r|r|r}
\hline
\multicolumn{3}{c|}{ } & \multicolumn{1}{c|}{Strict Score} & \multicolumn{4}{c|}{Interpretation Scores} & \multicolumn{1}{c}{Discriminant} \\
\cline{4-4} \cline{5-8} \cline{9-9}
response & n & interpretation & absolute & tri & tversky & satisfice & orthogonal & scaled score\\
\hline
\multicolumn{9}{l}{\textbf{Triangular}}\\
\hline
\hspace{1em}KX & 100 & Triangular & 1 & 1.000 & 0.667 & NA & -0.133 & 1.0\\
\hline
\hspace{1em}X & 6 & Triangular & 0 & 0.500 & 0.333 & NA & -0.067 & 1.0\\
\hline
\hspace{1em}CX & 2 & Triangular & 0 & 0.438 & 0.267 & NA & -0.133 & 1.0\\
\hline
\hspace{1em}DJNX & 2 & Triangular & 0 & 0.312 & 0.133 & NA & -0.267 & 1.0\\
\hline
\hspace{1em}AKPX & 1 & Triangular & 0 & 0.875 & 0.533 & NA & -0.267 & 1.0\\
\hline
\hspace{1em}CK & 1 & Triangular & 0 & 0.438 & 0.267 & NA & -0.133 & 1.0\\
\hline
\hspace{1em}GK & 1 & Triangular & 0 & 0.438 & 0.267 & NA & -0.133 & 1.0\\
\hline
\hspace{1em}JX & 1 & Triangular & 0 & 0.438 & 0.267 & NA & -0.133 & 1.0\\
\hline
\hspace{1em}K & 1 & Triangular & 0 & 0.500 & 0.333 & NA & -0.067 & 1.0\\
\hline
\hspace{1em}LX & 1 & Triangular & 0 & 0.438 & 0.267 & NA & -0.133 & 1.0\\
\hline
\multicolumn{9}{l}{\textbf{Lines-Connect}}\\
\hline
\hspace{1em}FZ & 3 & Tversky & 0 & -0.125 & 0.941 & NA & 0.433 & 0.5\\
\hline
\hspace{1em}OZ & 1 & Tversky & 0 & -0.125 & 0.941 & NA & -0.133 & 0.5\\
\hline
\hspace{1em}Z & 1 & Tversky & 0 & -0.062 & 1.000 & NA & -0.067 & 0.5\\
\hline
\multicolumn{9}{l}{\textbf{Orthogonal}}\\
\hline
\hspace{1em}EF & 118 & Orthogonal & 0 & -0.125 & -0.118 & NA & 1.000 & -1.0\\
\hline
\hspace{1em}BF & 17 & Orthogonal & 0 & -0.125 & -0.118 & NA & 0.500 & -1.0\\
\hline
\hspace{1em}F & 13 & Orthogonal & 0 & -0.062 & -0.059 & NA & 0.500 & -1.0\\
\hline
\hspace{1em}E & 8 & Orthogonal & 0 & -0.062 & -0.059 & NA & 0.500 & -1.0\\
\hline
\hspace{1em}BE & 4 & Orthogonal & 0 & -0.125 & -0.118 & NA & 0.500 & -1.0\\
\hline
\hspace{1em}BEF & 1 & Orthogonal & 0 & -0.188 & -0.176 & NA & 1.000 & -1.0\\
\hline
\hspace{1em}EFZ & 1 & Orthogonal & 0 & -0.188 & 0.882 & NA & 0.933 & -1.0\\
\hline
\hspace{1em}EI & 1 & Orthogonal & 0 & -0.125 & -0.118 & NA & 0.433 & -1.0\\
\hline
\hspace{1em}FI & 1 & Orthogonal & 0 & -0.125 & -0.118 & NA & 0.433 & -1.0\\
\hline
\multicolumn{9}{l}{\textbf{Other}}\\
\hline
\hspace{1em} & 11 & blank & 0 & 0.000 & 0.000 & NA & 0.000 & 0.0\\
\hline
\multicolumn{9}{l}{\textbf{Unknown}}\\
\hline
\hspace{1em}G & 4 & ? & 0 & -0.062 & -0.059 & NA & -0.067 & -0.5\\
\hline
\hspace{1em}B & 3 & ? & 0 & -0.062 & -0.059 & NA & 0.000 & -0.5\\
\hline
\hspace{1em}C & 3 & ? & 0 & -0.062 & -0.059 & NA & -0.067 & -0.5\\
\hline
\hspace{1em}O & 3 & ? & 0 & -0.062 & -0.059 & NA & -0.067 & -0.5\\
\hline
\hspace{1em}AG & 2 & ? & 0 & -0.125 & -0.118 & NA & -0.133 & -0.5\\
\hline
\hspace{1em}BM & 2 & ? & 0 & -0.125 & -0.118 & NA & -0.067 & -0.5\\
\hline
\hspace{1em}CG & 2 & ? & 0 & -0.125 & -0.118 & NA & -0.133 & -0.5\\
\hline
\hspace{1em}M & 2 & ? & 0 & -0.062 & -0.059 & NA & -0.067 & -0.5\\
\hline
\hspace{1em}A & 1 & ? & 0 & -0.062 & -0.059 & NA & -0.067 & -0.5\\
\hline
\hspace{1em}BG & 1 & ? & 0 & -0.125 & -0.118 & NA & -0.067 & -0.5\\
\hline
\hspace{1em}DN & 1 & ? & 0 & -0.125 & -0.118 & NA & -0.133 & -0.5\\
\hline
\hspace{1em}FK & 1 & ? & 0 & 0.438 & 0.267 & NA & 0.433 & -0.5\\
\hline
\hspace{1em}FX & 1 & ? & 0 & 0.438 & 0.267 & NA & 0.433 & -0.5\\
\hline
\hspace{1em}H & 1 & ? & 0 & -0.062 & -0.059 & NA & -0.067 & -0.5\\
\hline
\hspace{1em}HN & 1 & ? & 0 & -0.125 & -0.118 & NA & -0.133 & -0.5\\
\hline
\hspace{1em}HO & 1 & ? & 0 & -0.125 & -0.118 & NA & -0.133 & -0.5\\
\hline
\hspace{1em}I & 1 & ? & 0 & -0.062 & -0.059 & NA & -0.067 & -0.5\\
\hline
\hspace{1em}IJ & 1 & ? & 0 & -0.125 & -0.118 & NA & -0.133 & -0.5\\
\hline
\hspace{1em}J & 1 & ? & 0 & -0.062 & -0.059 & NA & -0.067 & -0.5\\
\hline
\hspace{1em}L & 1 & ? & 0 & -0.062 & -0.059 & NA & -0.067 & -0.5\\
\hline
\hspace{1em}N & 1 & ? & 0 & -0.062 & -0.059 & NA & -0.067 & -0.5\\
\hline
\end{tabular}

\textbackslash end\{table\}

\begin{Shaded}
\begin{Highlighting}[]
\FunctionTok{gf\_dhistogram}\NormalTok{(}\SpecialCharTok{\textasciitilde{}}\NormalTok{ score\_niceABS, }\AttributeTok{fill =} \SpecialCharTok{\textasciitilde{}}\NormalTok{condition, }\AttributeTok{data =}\NormalTok{ df\_items }\SpecialCharTok{\%\textgreater{}\%} \FunctionTok{filter}\NormalTok{(q }\SpecialCharTok{==} \DecValTok{15}\NormalTok{)) }\SpecialCharTok{\%\textgreater{}\%} 
  \FunctionTok{gf\_facet\_grid}\NormalTok{( condition }\SpecialCharTok{\textasciitilde{}}\NormalTok{ ., }\AttributeTok{labeller =}\NormalTok{ label\_both) }\SpecialCharTok{+} 
  \FunctionTok{labs}\NormalTok{( }\AttributeTok{x =} \StringTok{"Scaled Item Score"}\NormalTok{, }\AttributeTok{title =} \StringTok{"Distribution of Scaled Scores | Q15 "}\NormalTok{) }\SpecialCharTok{+} 
  \FunctionTok{theme\_minimal}\NormalTok{() }\SpecialCharTok{+} \FunctionTok{theme}\NormalTok{(}\AttributeTok{legend.position =} \StringTok{"blank"}\NormalTok{)}
\end{Highlighting}
\end{Shaded}

\begin{figure}[H]

{\centering \includegraphics{analysis/SGC3A/2_sgc3A_scoring_files/figure-pdf/Q15-distribution-1.pdf}

}

\end{figure}

\begin{Shaded}
\begin{Highlighting}[]
\FunctionTok{gf\_props}\NormalTok{(}\SpecialCharTok{\textasciitilde{}}\NormalTok{interpretation, }\AttributeTok{fill =} \SpecialCharTok{\textasciitilde{}}\NormalTok{condition, }\AttributeTok{data =}\NormalTok{ df\_items }\SpecialCharTok{\%\textgreater{}\%} \FunctionTok{filter}\NormalTok{(q }\SpecialCharTok{==} \DecValTok{15}\NormalTok{)) }\SpecialCharTok{\%\textgreater{}\%}
  \FunctionTok{gf\_facet\_grid}\NormalTok{( condition }\SpecialCharTok{\textasciitilde{}}\NormalTok{ ., }\AttributeTok{labeller =}\NormalTok{ label\_both) }\SpecialCharTok{+} 
  \FunctionTok{labs}\NormalTok{( }\AttributeTok{x =} \StringTok{"Interpretation"}\NormalTok{, }\AttributeTok{title =} \StringTok{"Distribution of Interpretations | Q15 "}\NormalTok{) }\SpecialCharTok{+} 
  \FunctionTok{theme\_minimal}\NormalTok{() }\SpecialCharTok{+} \FunctionTok{theme}\NormalTok{(}\AttributeTok{legend.position =} \StringTok{"blank"}\NormalTok{)}
\end{Highlighting}
\end{Shaded}

\begin{figure}[H]

{\centering \includegraphics{analysis/SGC3A/2_sgc3A_scoring_files/figure-pdf/Q15-distribution-2.pdf}

}

\end{figure}

\hypertarget{export-1}{%
\section{EXPORT}\label{export-1}}

Finally, we export the scores for each item (\texttt{df\_items}) and
summarized over subjects (\texttt{df\_subjects}), as well as cumulative
progress dataframes (\texttt{df\_absolute\_progress},
\texttt{df\_scaled\_progress})

\begin{Shaded}
\begin{Highlighting}[]
\NormalTok{imac }\OtherTok{=} \StringTok{"/Users/amyraefox/Code/SGC{-}Scaffolding\_Graph\_Comprehension/SGC{-}X/ANALYSIS/MAIN"}
\FunctionTok{setwd}\NormalTok{(imac)}

\CommentTok{\#SAVE FILES}
\FunctionTok{write.csv}\NormalTok{(df\_subjects,}\StringTok{"analysis/SGC3A/data/2{-}scored{-}data/sgc3a\_scored\_participants.csv"}\NormalTok{, }\AttributeTok{row.names =} \ConstantTok{FALSE}\NormalTok{)}
\FunctionTok{write.csv}\NormalTok{(df\_items,}\StringTok{"analysis/SGC3A/data/2{-}scored{-}data/sgc3a\_scored\_items.csv"}\NormalTok{, }\AttributeTok{row.names =} \ConstantTok{FALSE}\NormalTok{)}
\FunctionTok{write.csv}\NormalTok{(df\_absolute\_progress,}\StringTok{"analysis/SGC3A/data/2{-}scored{-}data/sgc3a\_absolute\_progress.csv"}\NormalTok{, }\AttributeTok{row.names =} \ConstantTok{FALSE}\NormalTok{)}
\FunctionTok{write.csv}\NormalTok{(df\_scaled\_progress,}\StringTok{"analysis/SGC3A/data/2{-}scored{-}data/sgc3a\_scaled\_progress.csv"}\NormalTok{, }\AttributeTok{row.names =} \ConstantTok{FALSE}\NormalTok{)}

\CommentTok{\#SAVE R Data Structures}
\CommentTok{\#export R DATA STRUCTURES (include codebook metadata)}
\NormalTok{rio}\SpecialCharTok{::}\FunctionTok{export}\NormalTok{(df\_subjects, }\StringTok{"analysis/SGC3A/data/2{-}scored{-}data/sgc3a\_scored\_participants.rds"}\NormalTok{) }\CommentTok{\# to R data structure file}
\NormalTok{rio}\SpecialCharTok{::}\FunctionTok{export}\NormalTok{(df\_items, }\StringTok{"analysis/SGC3A/data/2{-}scored{-}data/sgc3a\_scored\_items.rds"}\NormalTok{) }\CommentTok{\# to R data structure file}
\end{Highlighting}
\end{Shaded}

\hypertarget{resources-1}{%
\section{RESOURCES}\label{resources-1}}

\emph{set operations}

\url{https://stat.ethz.ch/R-manual/R-devel/library/base/html/sets.html}

\emph{kableExtra tables}

\url{https://haozhu233.github.io/kableExtra/awesome_table_in_html.html\#grouped_columns__rows}

\begin{Shaded}
\begin{Highlighting}[]
\FunctionTok{sessionInfo}\NormalTok{()}
\end{Highlighting}
\end{Shaded}

\begin{verbatim}
R version 4.2.1 (2022-06-23)
Platform: x86_64-apple-darwin17.0 (64-bit)
Running under: macOS Big Sur ... 10.16

Matrix products: default
BLAS:   /Library/Frameworks/R.framework/Versions/4.2/Resources/lib/libRblas.0.dylib
LAPACK: /Library/Frameworks/R.framework/Versions/4.2/Resources/lib/libRlapack.dylib

locale:
[1] en_US.UTF-8/en_US.UTF-8/en_US.UTF-8/C/en_US.UTF-8/en_US.UTF-8

attached base packages:
[1] stats     graphics  grDevices utils     datasets  methods   base     

other attached packages:
 [1] forcats_0.5.1    stringr_1.4.0    dplyr_1.0.9      purrr_0.3.4     
 [5] readr_2.1.2      tidyr_1.2.0      tibble_3.1.7     tidyverse_1.3.1 
 [9] Hmisc_4.7-0      Formula_1.2-4    survival_3.3-1   lattice_0.20-45 
[13] pbapply_1.5-0    ggformula_0.10.1 ggridges_0.5.3   scales_1.2.0    
[17] ggstance_0.3.5   ggplot2_3.3.6    kableExtra_1.3.4

loaded via a namespace (and not attached):
 [1] fs_1.5.2            bit64_4.0.5         lubridate_1.8.0    
 [4] webshot_0.5.3       RColorBrewer_1.1-3  httr_1.4.3         
 [7] tools_4.2.1         backports_1.4.1     utf8_1.2.2         
[10] R6_2.5.1            rpart_4.1.16        DBI_1.1.3          
[13] colorspace_2.0-3    nnet_7.3-17         withr_2.5.0        
[16] tidyselect_1.1.2    gridExtra_2.3       curl_4.3.2         
[19] bit_4.0.4           compiler_4.2.1      cli_3.3.0          
[22] rvest_1.0.2         htmlTable_2.4.0     xml2_1.3.3         
[25] labeling_0.4.2      mosaicCore_0.9.0    checkmate_2.1.0    
[28] systemfonts_1.0.4   digest_0.6.29       foreign_0.8-82     
[31] rmarkdown_2.14      svglite_2.1.0       rio_0.5.29         
[34] base64enc_0.1-3     jpeg_0.1-9          pkgconfig_2.0.3    
[37] htmltools_0.5.2     labelled_2.9.1      dbplyr_2.2.1       
[40] fastmap_1.1.0       readxl_1.4.0        htmlwidgets_1.5.4  
[43] rlang_1.0.3         rstudioapi_0.13     farver_2.1.0       
[46] generics_0.1.2      jsonlite_1.8.0      vroom_1.5.7        
[49] zip_2.2.0           magrittr_2.0.3      Matrix_1.4-1       
[52] Rcpp_1.0.8.3        munsell_0.5.0       fansi_1.0.3        
[55] lifecycle_1.0.1     stringi_1.7.6       yaml_2.3.5         
[58] MASS_7.3-57         plyr_1.8.7          grid_4.2.1         
[61] parallel_4.2.1      crayon_1.5.1        haven_2.5.0        
[64] splines_4.2.1       hms_1.1.1           knitr_1.39         
[67] pillar_1.7.0        codetools_0.2-18    reprex_2.0.1       
[70] glue_1.6.2          evaluate_0.15       latticeExtra_0.6-29
[73] data.table_1.14.2   modelr_0.1.8        tzdb_0.3.0         
[76] png_0.1-7           vctrs_0.4.1         tweenr_1.0.2       
[79] cellranger_1.1.0    gtable_0.3.0        polyclip_1.10-0    
[82] assertthat_0.2.1    openxlsx_4.2.5      xfun_0.31          
[85] ggforce_0.3.3       broom_0.8.0         viridisLite_0.4.0  
[88] cluster_2.1.3       ellipsis_0.3.2     
\end{verbatim}

\hypertarget{archive}{%
\section{ARCHIVE}\label{archive}}

\emph{Prior versions of functions for for-loop version of scoring, not
optimized to use mapply}

\begin{Shaded}
\begin{Highlighting}[]
\CommentTok{\# \#CALCULATE THE TRIANGULAR, ORTHOGONAL OR TVERSKIAN SUBSCORES FROM KEYFRAME}
\CommentTok{\# calc\_sub\_score \textless{}{-} function(question, cond, response,keyframe)\{}
\CommentTok{\# }
\CommentTok{\#   \#STEP 1 GET KEY}
\CommentTok{\#   if (question \textless{} 6) \#for q1 {-} q5 find key for question by condition}
\CommentTok{\#   \{}
\CommentTok{\#     \# print(keyframe)}
\CommentTok{\#     \#GET KEY FOR THIS SCORE TYPE, QUESTION AND CONDITION}
\CommentTok{\#     p =  keyframe \%\textgreater{}\% filter(Q == question) \%\textgreater{}\% filter(condition == cond) \%\textgreater{}\% select(set\_p) \%\textgreater{}\% pull(set\_p) \%\textgreater{}\% str\_split("") \%\textgreater{}\% unlist()}
\CommentTok{\#     q =  keyframe \%\textgreater{}\% filter(Q == question) \%\textgreater{}\% filter(condition == cond) \%\textgreater{}\% select(set\_q) \%\textgreater{}\% pull(set\_q) \%\textgreater{}\% str\_split("") \%\textgreater{}\% unlist()}
\CommentTok{\#     pn = keyframe \%\textgreater{}\% filter(Q == question) \%\textgreater{}\% filter(condition == cond) \%\textgreater{}\% select(n\_p)}
\CommentTok{\#     qn = keyframe \%\textgreater{}\% filter(Q == question) \%\textgreater{}\% filter(condition == cond) \%\textgreater{}\% select(n\_q)}
\CommentTok{\# }
\CommentTok{\#   \} else \{}
\CommentTok{\#     \#GET KEY FOR THIS SCORE TYPE, QUESTION}
\CommentTok{\#     p =  keyframe \%\textgreater{}\% filter(Q == question) \%\textgreater{}\% select(set\_p) \%\textgreater{}\% pull(set\_p) \%\textgreater{}\% str\_split("") \%\textgreater{}\% unlist()}
\CommentTok{\#     q =  keyframe \%\textgreater{}\% filter(Q == question) \%\textgreater{}\% select(set\_q) \%\textgreater{}\% pull(set\_q) \%\textgreater{}\% str\_split("") \%\textgreater{}\% unlist()}
\CommentTok{\#     pn = keyframe \%\textgreater{}\% filter(Q == question) \%\textgreater{}\% select(n\_p)}
\CommentTok{\#     qn = keyframe \%\textgreater{}\% filter(Q == question) \%\textgreater{}\% select(n\_q)}
\CommentTok{\#   \}}
\CommentTok{\# }
\CommentTok{\#   \#STEP 2 CALC INTERSECTIONS BETWEEN RESPONSE AND KEY}
\CommentTok{\#   }
\CommentTok{\#   \#if response is not empty, split apart response for set comparison}
\CommentTok{\#     if(response != "")}
\CommentTok{\#     \{ response = response \%\textgreater{}\% str\_split("") \%\textgreater{}\% unlist()\}}
\CommentTok{\#     }
\CommentTok{\#   ps = length(intersect(response,p))}
\CommentTok{\#   qs = length(intersect(response,q))}
\CommentTok{\#   \# df\_items[x,\textquotesingle{}tri\_ps\textquotesingle{}] = tri\_ps}
\CommentTok{\#   \# df\_items[x,\textquotesingle{}tri\_qs\textquotesingle{}] = tri\_qs}
\CommentTok{\# }
\CommentTok{\#   \#STEP 3 CALC f\_partialP schema SCORE FOR THIS INTERSECTION}
\CommentTok{\#   x = f\_partialP(ps,pn,qs,qn) \%\textgreater{}\% unlist() \%\textgreater{}\% as.numeric()}
\CommentTok{\#   }
\CommentTok{\#   \#cleanup}
\CommentTok{\#   rm(p,q,pn,qn,ps,qs)}
\CommentTok{\#   return(x)}
\CommentTok{\# }
\CommentTok{\# \}}
\CommentTok{\# }
\CommentTok{\# \#CALCULATE THE REFERENCE SCORES}
\CommentTok{\# calc\_ref\_score \textless{}{-} function(question, cond, response)\{}
\CommentTok{\#   }
\CommentTok{\#     \#1. GET reference point from REF\_POINT column in raw keys}
\CommentTok{\#     ref\_p = keys\_raw \%\textgreater{}\% filter(Q == question) \%\textgreater{}\% filter(condition == cond) \%\textgreater{}\% select(REF\_POINT) \%\textgreater{}\% pull(REF\_POINT) \%\textgreater{}\% str\_split("") \%\textgreater{}\% unlist()}
\CommentTok{\#      }
\CommentTok{\#     \#2. if response has more than one character, it can\textquotesingle{}t be correct}
\CommentTok{\#     \#there is only ever 1 reference character}
\CommentTok{\#     n = nchar(response)}
\CommentTok{\#     if (n == 0) \{x = 0\}}
\CommentTok{\#     else if(n\textgreater{}1) \{x = 0\}}
\CommentTok{\#     else \{}
\CommentTok{\#       \#3 is the response PRECISELY the REFERENCE POINT?}
\CommentTok{\#       x = ref\_p == response}
\CommentTok{\#       x = as.numeric(x)  }
\CommentTok{\#     \}}
\CommentTok{\#     }
\CommentTok{\#     \#cleanup}
\CommentTok{\#     rm(ref\_p, response, question, cond)   }
\CommentTok{\#     return(x) \#1 = match, 0 = not match}
\CommentTok{\# \}}
\CommentTok{\# }
\CommentTok{\# }
\CommentTok{\# \#CALCULATE SCORE BASED ON UNION OF ORTH \& TRI (SUBJECT SELECTS BOTH ANSWERS )}
\CommentTok{\# calc\_both\_score \textless{}{-} function(question, cond, response)\{}
\CommentTok{\#   }
\CommentTok{\#TRAPDOOR }
\CommentTok{\#   \#since no orth responses exist for impasse condition q1 {-} q5, set to 0}
\CommentTok{\#   if (question \textless{} 6 \& cond == 121) \{x = NA\}}
\CommentTok{\#   }
\CommentTok{\#   \#ELSE }
\CommentTok{\#   \#calculate union of ORTH and TRI}
\CommentTok{\#   else \{}
\CommentTok{\#     if (question \textless{} 6 \& cond == 111) \#for q1 {-} q5 find key for question by condition}
\CommentTok{\#   \{}
\CommentTok{\#      \#grab the tri and orth keys for this question as well as N option set}
\CommentTok{\#      tri\_p =  keys\_tri \%\textgreater{}\%  filter(Q == question) \%\textgreater{}\% filter(condition == cond) \%\textgreater{}\% select(set\_p) \%\textgreater{}\% pull(set\_p) \%\textgreater{}\% str\_split("") \%\textgreater{}\% unlist()}
\CommentTok{\#      orth\_p = keys\_orth \%\textgreater{}\% filter(Q == question) \%\textgreater{}\% filter(condition == cond) \%\textgreater{}\% select(set\_p) \%\textgreater{}\% pull(set\_p) \%\textgreater{}\% str\_split("") \%\textgreater{}\% unlist()}
\CommentTok{\#      set\_n =  keys\_tri \%\textgreater{}\%  filter(Q == question) \%\textgreater{}\% filter(condition == cond) \%\textgreater{}\% select(set\_n) \%\textgreater{}\% pull(set\_n) \%\textgreater{}\% str\_split("") \%\textgreater{}\% unlist() }
\CommentTok{\#      \#1. calc answer that is both tri and orth and only these {-}{-}\textgreater{} union of tri\_p and orth\_p}
\CommentTok{\#      both\_p = union(tri\_p, orth\_p) \#the selection of tri and p}
\CommentTok{\#      \#2. calc answers that should\textquotesingle{}t be selected as diffrence between N [same for all keys] and both\_p}
\CommentTok{\#      both\_q = setdiff(set\_n,both\_p)}
\CommentTok{\#      both\_pn = length(both\_p)}
\CommentTok{\#      both\_qn = length(both\_q)}
\CommentTok{\#   \} else\{}
\CommentTok{\#     }
\CommentTok{\#      \#grab the tri and orth keys for this question as well as N option set}
\CommentTok{\#      tri\_p =  keys\_tri \%\textgreater{}\%  filter(Q == question) \%\textgreater{}\% select(set\_p) \%\textgreater{}\% pull(set\_p) \%\textgreater{}\% str\_split("") \%\textgreater{}\% unlist()}
\CommentTok{\#      orth\_p = keys\_orth \%\textgreater{}\% filter(Q == question) \%\textgreater{}\% select(set\_p) \%\textgreater{}\% pull(set\_p) \%\textgreater{}\% str\_split("") \%\textgreater{}\% unlist()}
\CommentTok{\#      set\_n =  keys\_tri \%\textgreater{}\%  filter(Q == question) \%\textgreater{}\% select(set\_n) \%\textgreater{}\% pull(set\_n) \%\textgreater{}\% str\_split("") \%\textgreater{}\% unlist() }
\CommentTok{\#      \#1. calc answer that is both tri and orth and only these {-}{-}\textgreater{} union of tri\_p and orth\_p}
\CommentTok{\#      both\_p = union(tri\_p, orth\_p) \#the selection of tri and p}
\CommentTok{\#      \#2. calc answers that shouldn\textquotesingle{}t be selected as difference between N [same for all keys] and both\_p}
\CommentTok{\#      both\_q = setdiff(set\_n,both\_p)}
\CommentTok{\#      both\_pn = length(both\_p)}
\CommentTok{\#      both\_qn = length(both\_q)}
\CommentTok{\#   \}}
\CommentTok{\#     }
\CommentTok{\#   \#STEP 2 CALC INTERSECTIONS BETWEEN RESPONSE AND KEY}
\CommentTok{\#   }
\CommentTok{\#   \#if response is not empty, split apart response for set comparison}
\CommentTok{\#     if(response != "")}
\CommentTok{\#     \{ response = response \%\textgreater{}\% str\_split("") \%\textgreater{}\% unlist()\}}
\CommentTok{\#   }
\CommentTok{\#     both\_ps = length(intersect(response,both\_p))}
\CommentTok{\#     both\_qs = length(intersect(response,both\_q))}
\CommentTok{\#   }
\CommentTok{\#  }
\CommentTok{\#   \#STEP 3 CALC f\_partialP schema SCORE FOR THIS INTERSECTION }
\CommentTok{\#   x = f\_partialP(both\_ps,both\_pn,both\_qs,both\_qn)\%\textgreater{}\% unlist() \%\textgreater{}\% as.numeric()}
\CommentTok{\#   }
\CommentTok{\#   \#cleanup}
\CommentTok{\#   rm(both\_p,both\_q,both\_pn,both\_qn,both\_ps,both\_qs, question, cond, response )   }
\CommentTok{\#   \}}
\CommentTok{\#   }
\CommentTok{\#   return(x) \#true correct, trues, false correct, falses}
\CommentTok{\# \}}
\end{Highlighting}
\end{Shaded}

\emph{Looping to do the scoring (not using MAPPLY)}

\begin{Shaded}
\begin{Highlighting}[]
\CommentTok{\#RUN THIS OR THE CALCULATE{-}SCORES{-}MAPPLY}
\CommentTok{\# df\_items = trad }
\CommentTok{\# }
\CommentTok{\# pb \textless{}{-} timerProgressBar() }
\CommentTok{\# on.exit(close(pb)) }
\CommentTok{\#  }
\CommentTok{\# \#CALCULATE SUBSCORES (in loop)}
\CommentTok{\# }
\CommentTok{\# for (x in 1:nrow(df\_items)) \{}
\CommentTok{\#   }
\CommentTok{\#   \#show progress bar }
\CommentTok{\#   setTimerProgressBar(pb, x) }
\CommentTok{\#   }
\CommentTok{\#   \#PREPARE ITEMS FOR SCORING}
\CommentTok{\#   \#sort response vectors alphabetically}
\CommentTok{\#   \#doesn\textquotesingle{}t impact scoring, but does impact response display tables}
\CommentTok{\#    df\_items[x,\textquotesingle{}response\textquotesingle{}] \textless{}{-}  df\_items[x,\textquotesingle{}response\textquotesingle{}] \%\textgreater{}\% str\_split("") \%\textgreater{}\% unlist() \%\textgreater{}\% sort() \%\textgreater{}\% str\_c(collapse="")}
\CommentTok{\# }
\CommentTok{\#   \#get properties of the RESPONSE ITEM}
\CommentTok{\#   qu = df\_items[x,\textquotesingle{}q\textquotesingle{}] \%\textgreater{}\% as.numeric()}
\CommentTok{\#   cond = as.character(df\_items[x,\textquotesingle{}condition\textquotesingle{}]) \%\textgreater{}\% as.numeric()}
\CommentTok{\#   r = df\_items[x,\textquotesingle{}response\textquotesingle{}] }
\CommentTok{\# }
\CommentTok{\#   \#calculate the main subscores}
\CommentTok{\#   df\_items[x,\textquotesingle{}score\_TRI\textquotesingle{}] = calc\_sub\_score(qu, cond, r,keys\_tri)}
\CommentTok{\#   df\_items[x,\textquotesingle{}score\_ORTH\textquotesingle{}] = calc\_sub\_score(qu, cond, r,keys\_orth)}
\CommentTok{\#   df\_items[x,\textquotesingle{}score\_SATISFICE\textquotesingle{}] = calc\_sub\_score(qu, cond, r,keys\_satisfice)}
\CommentTok{\#   df\_items[x,\textquotesingle{}score\_TV\_max\textquotesingle{}] = calc\_sub\_score(qu, cond, r,keys\_tversky\_max)}
\CommentTok{\#   df\_items[x,\textquotesingle{}score\_TV\_start\textquotesingle{}] = calc\_sub\_score(qu, cond, r,keys\_tversky\_start)}
\CommentTok{\#   df\_items[x,\textquotesingle{}score\_TV\_end\textquotesingle{}] = calc\_sub\_score(qu, cond, r,keys\_tversky\_end)}
\CommentTok{\#   df\_items[x,\textquotesingle{}score\_TV\_duration\textquotesingle{}] = calc\_sub\_score(qu, cond, r, keys\_tversky\_duration)}
\CommentTok{\#   }
\CommentTok{\#   \#calculate special subscores}
\CommentTok{\#   df\_items[x,\textquotesingle{}score\_REF\textquotesingle{}] = calc\_ref\_score(qu, cond, r)}
\CommentTok{\#   df\_items[x,\textquotesingle{}score\_BOTH\textquotesingle{}] = calc\_both\_score(qu, cond, r)}
\CommentTok{\# \}}
\CommentTok{\# }
\CommentTok{\# \#CALCULATE ABSOLUTE SCORES}
\CommentTok{\# \#calculate absolute scores dichotomous}
\CommentTok{\# df\_items$score\_ABS = as.integer(df\_items$correct)}
\CommentTok{\# \#niceABS indicates if the response is correct without penalizing the allowable triangular options(ie. the ref point)}
\CommentTok{\# df\_items$score\_niceABS  \textless{}{-} as.integer((df\_items$score\_TRI == 1))}
\CommentTok{\#  }
\CommentTok{\# \#cleanup}
\CommentTok{\# rm(qu,cond,r, x)}

\CommentTok{\# trad\_scored = df\_items}
\end{Highlighting}
\end{Shaded}

\emph{sanity check equivalence of for-loop and mapply scoring}

\begin{Shaded}
\begin{Highlighting}[]
\CommentTok{\#CHECK EQUIVALENCE OF LOOP AND MAPPLY SCORING }
\CommentTok{\# tests = data.frame (}
\CommentTok{\#   alt\_tri = alt\_scored$score\_TRI,}
\CommentTok{\#   trad\_tri = trad\_scored$score\_TRI,}
\CommentTok{\#   alt\_orth = alt\_scored$score\_ORTH,}
\CommentTok{\#   trad\_orth = trad\_scored$score\_ORTH,}
\CommentTok{\#   alt\_ref = alt\_scored$score\_REF,}
\CommentTok{\#   trad\_ref = trad\_scored$score\_REF,}
\CommentTok{\#   alt\_tv\_max = alt\_scored$score\_TV\_max,}
\CommentTok{\#   trad\_tv\_max = trad\_scored$score\_TV\_max,}
\CommentTok{\#   alt\_tv\_dur = alt\_scored$score\_TV\_duration,}
\CommentTok{\#   trad\_tv\_dur = trad\_scored$score\_TV\_duration,}
\CommentTok{\#   alt\_tv\_start = alt\_scored$score\_TV\_start,}
\CommentTok{\#   trad\_tv\_start = trad\_scored$score\_TV\_start,}
\CommentTok{\#   alt\_tv\_end = alt\_scored$score\_TV\_end,}
\CommentTok{\#   trad\_tv\_end = trad\_scored$score\_TV\_end,}
\CommentTok{\#   alt\_both = alt\_scored$score\_BOTH,}
\CommentTok{\#   trad\_both = trad\_scored$score\_BOTH,}
\CommentTok{\#   trad\_response = trad\_scored$response,}
\CommentTok{\#   alt\_response = alt\_scored$response,}
\CommentTok{\#   q\_match = trad\_scored$q == alt\_scored$q,}
\CommentTok{\#   q = trad\_scored$q,}
\CommentTok{\#   c\_match = trad\_scored$condition == alt\_scored$condition,}
\CommentTok{\#   condition = trad\_scored$condition}
\CommentTok{\# )}
\CommentTok{\# }
\CommentTok{\# tests$tri = tests$alt\_tri == tests$trad\_tri}
\CommentTok{\# tests$orth = tests$alt\_orth == tests$trad\_orth}
\CommentTok{\# tests$ref = tests$alt\_ref == tests$trad\_ref}
\CommentTok{\# tests$tvdur = tests$alt\_tv\_dur == tests$trad\_tv\_dur}
\CommentTok{\# tests$tvstart = tests$alt\_tv\_start == tests$trad\_tv\_start}
\CommentTok{\# tests$tvend = tests$alt\_tv\_end == tests$trad\_tv\_end}
\CommentTok{\# tests$both = tests$alt\_both == tests$trad\_both}
\CommentTok{\# }
\CommentTok{\# \#CHECKS }
\CommentTok{\# unique(tests$tri)}
\CommentTok{\# unique(tests$orth)}
\CommentTok{\# unique(tests$ref)}
\CommentTok{\# unique(tests$tvdur)}
\CommentTok{\# unique(tests$tvstart)}
\CommentTok{\# unique(tests$tvend)}
\CommentTok{\# unique(tests$both)}
\CommentTok{\# }
\CommentTok{\# unique(alt\_scored$score\_ABS == trad\_scored$score\_ABS)}
\CommentTok{\# unique(alt\_scored$score\_niceABS == trad\_scored$score\_niceABS)}
\end{Highlighting}
\end{Shaded}

\textbf{Prior inline version of derive interpretation, before
externalizing to a function in the scoring script}. ::: \{.cell
hash=`2\_sgc3A\_scoring\_cache/pdf/unnamed-chunk-76\_e34f9be0861c7d4068ca13acd418b312'\}

\begin{Shaded}
\begin{Highlighting}[]
\CommentTok{\# threshold\_range = 0.5 \#set required variance in subscores to be discriminant}
\CommentTok{\# threshold\_frenzy = 4}
\CommentTok{\# }
\CommentTok{\# for (x in 1:nrow(df\_items)) \{}
\CommentTok{\#   }
\CommentTok{\#   \#CALCULATE MAX TVERSKY SUBSCORE}
\CommentTok{\#   t = df\_items[x,] \%\textgreater{}\% select(score\_TV\_max, score\_TV\_start, score\_TV\_end, score\_TV\_duration) \#reshape}
\CommentTok{\#   t.long = gather(t,score, value, 1:4)}
\CommentTok{\#   t.long[t.long == ""] = NA \#replace empty scores with NA so we can ignore them}
\CommentTok{\#   if(length(unique(t.long$value)) == 1 )\{}
\CommentTok{\#     if(is.na(unique(t.long$value)))\{}
\CommentTok{\#       df\_items[x,\textquotesingle{}score\_TVERSKY\textquotesingle{}] = NA}
\CommentTok{\#       df\_items[x,\textquotesingle{}tv\_type\textquotesingle{}] = NA   }
\CommentTok{\#     \}}
\CommentTok{\#   \} else \{}
\CommentTok{\#     df\_items[x,\textquotesingle{}score\_TVERSKY\textquotesingle{}] = as.numeric(max(t.long$value,na.rm = TRUE))}
\CommentTok{\#     df\_items[x,\textquotesingle{}tv\_type\textquotesingle{}] = t.long[which.max(t.long$value),\textquotesingle{}score\textquotesingle{}]}
\CommentTok{\#   \}}
\CommentTok{\#   }
\CommentTok{\#   \#CALCULATE MAX SATISFICING SUBSCORE}
\CommentTok{\#   t = df\_items[x,] \%\textgreater{}\% select(score\_SAT\_left, score\_SAT\_right)}
\CommentTok{\#   t.long = gather(t,score, value, 1:2)}
\CommentTok{\#   t.long[t.long == ""] = NA \#replace empty scores}
\CommentTok{\#   if(length(unique(t.long$value)) == 1 )\{}
\CommentTok{\#     if(is.na(unique(t.long$value)))\{}
\CommentTok{\#       df\_items[x,\textquotesingle{}score\_SATISFICE\textquotesingle{}] = NA}
\CommentTok{\#       df\_items[x,\textquotesingle{}sat\_type\textquotesingle{}] = NA   }
\CommentTok{\#     \}}
\CommentTok{\#   \} else \{}
\CommentTok{\#     df\_items[x,\textquotesingle{}score\_SATISFICE\textquotesingle{}] = as.numeric(max(t.long$value,na.rm = TRUE))}
\CommentTok{\#     df\_items[x,\textquotesingle{}sat\_type\textquotesingle{}] = t.long[which.max(t.long$value),\textquotesingle{}score\textquotesingle{}]  }
\CommentTok{\#   \}}
\CommentTok{\#   }
\CommentTok{\#   \#NOW CALCULATE RANGE AMONG SUBSCORES}
\CommentTok{\#   \#order of this selection matters in breaking ties! }
\CommentTok{\#   t = df\_items[x,] \%\textgreater{}\% select(score\_TRI, score\_TVERSKY, score\_SATISFICE, score\_ORTH)}
\CommentTok{\#   t.long = gather(t,score, value, 1:4)}
\CommentTok{\#   t.long[t.long == ""] = NA}
\CommentTok{\#   }
\CommentTok{\#   df\_items[x,\textquotesingle{}top\_score\textquotesingle{}] = as.numeric(max(t.long$value,na.rm = TRUE))}
\CommentTok{\#   df\_items[x,\textquotesingle{}top\_type\textquotesingle{}] = t.long[which.max(t.long$value),\textquotesingle{}score\textquotesingle{}]}
\CommentTok{\#   }
\CommentTok{\#   \#calculate the range between highest and lowest scores }
\CommentTok{\#   r = as.numeric(range(t.long$value,na.rm = TRUE))}
\CommentTok{\#   r = diff(r)}
\CommentTok{\#   df\_items[x,\textquotesingle{}range\textquotesingle{}] = r}
\CommentTok{\#   }
\CommentTok{\#   \#DISCRIMINANT BETWEEN SUBSCORES TO PREDICT BEST FIT INTERPRETATION}
\CommentTok{\#   }
\CommentTok{\#   if (r \textless{} threshold\_range) \{}
\CommentTok{\#       \#then we can\textquotesingle{}t predict the interpretation, leave it as "?"}
\CommentTok{\#     df\_items[x,\textquotesingle{}best\textquotesingle{}] = "?"}
\CommentTok{\#   \} else \{}
\CommentTok{\#       p =  df\_items[x,\textquotesingle{}top\_type\textquotesingle{}]}
\CommentTok{\#       if (p == "score\_TRI") \{df\_items[x,\textquotesingle{}best\textquotesingle{}] = "Triangular"}
\CommentTok{\#       \} else if(p == "score\_ORTH") \{df\_items[x,\textquotesingle{}best\textquotesingle{}] = "Orthogonal"}
\CommentTok{\#       \} else if(p == "score\_TVERSKY") \{df\_items[x,\textquotesingle{}best\textquotesingle{}] = "Tversky"}
\CommentTok{\#       \} else if(p == "score\_SATISFICE") \{df\_items[x,\textquotesingle{}best\textquotesingle{}] = "Satisfice"\}}
\CommentTok{\#   \}}
\CommentTok{\#   }
\CommentTok{\#   \#CHECK SPECIAL SITUATIONS}
\CommentTok{\# }
\CommentTok{\#   \#BOTH TRI AND ORTH?  }
\CommentTok{\#   if (!is.na(df\_items[x,\textquotesingle{}score\_BOTH\textquotesingle{}])) \{ \#only check if both is not null}
\CommentTok{\#       if( df\_items[x,\textquotesingle{}score\_BOTH\textquotesingle{}] == 1) \{}
\CommentTok{\#         df\_items[x,\textquotesingle{}best\textquotesingle{}] = "both tri + orth"\}}
\CommentTok{\#   \}}
\CommentTok{\#   }
\CommentTok{\#   \#IS BLANK?}
\CommentTok{\#   if( df\_items[x,\textquotesingle{}num\_o\textquotesingle{}] == 0) \{  }
\CommentTok{\#     df\_items[x,\textquotesingle{}best\textquotesingle{}] = "blank"}
\CommentTok{\#   \}}
\CommentTok{\#   }
\CommentTok{\#   \#IS FRENZY?}
\CommentTok{\#   if( df\_items[x,\textquotesingle{}num\_o\textquotesingle{}] \textgreater{} threshold\_frenzy) \{ }
\CommentTok{\#       df\_items[x,\textquotesingle{}best\textquotesingle{}] = "frenzy"}
\CommentTok{\#   \}}
\CommentTok{\# }
\CommentTok{\#   \#IS REF POINT?}
\CommentTok{\#   if (!is.na(df\_items[x,\textquotesingle{}score\_REF\textquotesingle{}])) \{ \#only check if the score is NOT null}
\CommentTok{\#       if( df\_items[x,\textquotesingle{}score\_REF\textquotesingle{}] == 1) \{}
\CommentTok{\#           df\_items[x,\textquotesingle{}best\textquotesingle{}] = "reference"}
\CommentTok{\#       \}}
\CommentTok{\#   \}}
\CommentTok{\# }
\CommentTok{\# \}\#end loop}
\CommentTok{\# }
\CommentTok{\# \#cleanup }
\CommentTok{\# rm(t, t.long, x, r,p)}
\CommentTok{\# rm(threshold\_frenzy, threshold\_range)}
\CommentTok{\# }
\CommentTok{\# \#set order of levels for response exploration table}
\CommentTok{\# df\_items$int2 \textless{}{-} factor(df\_items$best,}
\CommentTok{\#                                   levels = c("Triangular", "Tversky",}
\CommentTok{\#                                              "Satisfice", "Orthogonal", "reference", "both tri + orth", "blank","frenzy","?"))}
\CommentTok{\# }
\CommentTok{\# \#set order of levels}
\CommentTok{\# df\_items$interpretation \textless{}{-} factor(df\_items$best,}
\CommentTok{\#                                   levels = c("Orthogonal","Satisfice", "frenzy","?","reference","blank",}
\CommentTok{\#                                                "both tri + orth", "Tversky","Triangular"))}
\CommentTok{\# }
\CommentTok{\# \#collapsed representation of scale of interpretations}
\CommentTok{\# df\_items$high\_interpretation \textless{}{-} fct\_collapse(df\_items$interpretation,}
\CommentTok{\#   orthogonal = c("Satisfice", "Orthogonal"),}
\CommentTok{\#   neg.trans = c("frenzy","?"),}
\CommentTok{\#   neutral = c("reference","blank"),}
\CommentTok{\#   pos.trans = c("Tversky","both tri + orth"),}
\CommentTok{\#   triangular = "Triangular"}
\CommentTok{\# ) }
\CommentTok{\# }
\CommentTok{\# \#reorder levels}
\CommentTok{\# df\_items$high\_interpretation = factor(df\_items$high\_interpretation, levels= c("orthogonal", "neg.trans","neutral","pos.trans","triangular"))}
\CommentTok{\# }
\CommentTok{\# \#cleanup }
\CommentTok{\# df\_items \textless{}{-} df\_items \%\textgreater{}\% dplyr::select({-}best)}
\CommentTok{\# }
\CommentTok{\# \#recode as numeric inase they are char }
\CommentTok{\# \# df\_items$score\_TV\_duration \textless{}{-} df\_items$score\_TV\_duration \%\textgreater{}\% as.numeric()}
\CommentTok{\# \# df\_items$score\_SATISFICE \textless{}{-} df\_items$score\_SATISFICE \%\textgreater{}\% as.numeric()}
\end{Highlighting}
\end{Shaded}

:::

\textbf{Old inline calculation of score\_SCALED before externalizing as
function} ::: \{.cell
hash=`2\_sgc3A\_scoring\_cache/pdf/unnamed-chunk-77\_e54aef2ed3372708fd0c453c9a35d340'\}

\begin{Shaded}
\begin{Highlighting}[]
\CommentTok{\# df\_items$score\_SCALED \textless{}{-} recode(df\_items$interpretation,}
\CommentTok{\#                           "Orthogonal" = {-}1,}
\CommentTok{\#                           "Satisfice" = {-}1,}
\CommentTok{\#                           "frenzy" = {-}0.5,}
\CommentTok{\#                           "?" = {-}0.5,}
\CommentTok{\#                           "reference" = 0,}
\CommentTok{\#                           "blank" = 0, }
\CommentTok{\#                           "both tri + orth" = 0.5,}
\CommentTok{\#                           "Tversky" = 0.5,}
\CommentTok{\#                           "Triangular" = 1)}
\end{Highlighting}
\end{Shaded}

:::

\textbf{Original summary by subject before externalizing as function}
::: \{.cell
hash=`2\_sgc3A\_scoring\_cache/pdf/unnamed-chunk-78\_ef54a1f4f64fd28b89f16f91e77290a1'\}

\begin{Shaded}
\begin{Highlighting}[]
\CommentTok{\# \#prep items}
\CommentTok{\# df\_items \textless{}{-} df\_items \%\textgreater{}\% mutate(}
\CommentTok{\#   tv\_type = as.factor(tv\_type),}
\CommentTok{\#   top\_type = as.factor(top\_type)}
\CommentTok{\# )}
\CommentTok{\# }
\CommentTok{\# \#summarize SCORES and TIME by subject}
\CommentTok{\# subjects\_summary \textless{}{-} df\_items \%\textgreater{}\% filter(q \%nin\% c(6,9)) \%\textgreater{}\% group\_by(subject) \%\textgreater{}\% dplyr::summarise (}
\CommentTok{\#   subject = as.character(subject),}
\CommentTok{\#   pretty\_condition = recode\_factor(condition, "111" = "control", "121" =  "impasse"),}
\CommentTok{\#   s\_TRI = sum(score\_TRI,na.rm=TRUE),}
\CommentTok{\#   s\_ORTH = sum(score\_ORTH,na.rm=TRUE),}
\CommentTok{\#   s\_TVERSKY = sum(score\_TVERSKY,na.rm=TRUE),}
\CommentTok{\#   s\_SATISFICE = sum(score\_SATISFICE, na.rm=TRUE),}
\CommentTok{\#   s\_REF = sum(score\_REF,na.rm=TRUE),}
\CommentTok{\#   s\_ABS = sum(score\_ABS,na.rm=TRUE),}
\CommentTok{\#   s\_NABS = sum(score\_niceABS,na.rm=TRUE),}
\CommentTok{\#   s\_SCALED = sum(score\_SCALED,na.rm=TRUE),}
\CommentTok{\#   DV\_percent\_NABS = s\_NABS/13,}
\CommentTok{\#   rt\_m = sum(rt\_s)/60,}
\CommentTok{\#   item\_avg\_rt = mean(rt\_s),}
\CommentTok{\#   item\_min\_rt = min(rt\_s),}
\CommentTok{\#   item\_max\_rt = max(rt\_s),}
\CommentTok{\#   item\_n\_TRI = sum(interpretation == "Triangular"),}
\CommentTok{\#   item\_n\_ORTH = sum(interpretation == "Orthogonal"),}
\CommentTok{\#   item\_n\_TV = sum(interpretation == "Tversky"),}
\CommentTok{\#   item\_n\_SAT = sum(interpretation == "Satisfice"),}
\CommentTok{\#   item\_n\_OTHER = sum(interpretation \%nin\% c("Triangular","Orthogonal","Tversky","Satisfice")),}
\CommentTok{\#   item\_n\_POS = sum(high\_interpretation == "pos.trans"),}
\CommentTok{\#   item\_n\_NEG = sum(high\_interpretation == "neg.trans"),}
\CommentTok{\#   item\_n\_NEUTRAL = sum(high\_interpretation == "neutral")}
\CommentTok{\# ) \%\textgreater{}\% arrange(subject) \%\textgreater{}\% slice(1L)}
\CommentTok{\# }
\CommentTok{\# \#summarize first scaffold item of interest by subject}
\CommentTok{\# subjects\_q1 \textless{}{-} df\_items \%\textgreater{}\% filter(q == 1) \%\textgreater{}\% mutate(}
\CommentTok{\#   item\_q1\_NABS = score\_niceABS,}
\CommentTok{\#   item\_q1\_SCALED = score\_SCALED,}
\CommentTok{\#   item\_q1\_interpretation = interpretation,}
\CommentTok{\#   item\_q1\_rt = rt\_s,}
\CommentTok{\# ) \%\textgreater{}\% dplyr::select(subject, item\_q1\_NABS, item\_q1\_SCALED, item\_q1\_interpretation,item\_q1\_rt) \%\textgreater{}\% arrange(subject)}
\CommentTok{\# }
\CommentTok{\# \#summarize last test item of interest by subject}
\CommentTok{\# subjects\_q5 \textless{}{-} df\_items \%\textgreater{}\% filter(q == 5) \%\textgreater{}\% mutate(}
\CommentTok{\#   item\_q5\_NABS = score\_niceABS,}
\CommentTok{\#   item\_q5\_SCALED = score\_SCALED,}
\CommentTok{\#   item\_q5\_interpretation = interpretation,}
\CommentTok{\#   item\_q5\_rt = rt\_s,}
\CommentTok{\# ) \%\textgreater{}\% dplyr::select(subject, item\_q5\_NABS, item\_q5\_SCALED, item\_q5\_interpretation,item\_q5\_rt) \%\textgreater{}\% arrange(subject)}
\CommentTok{\# }
\CommentTok{\# \#summarize first test item of interest by subject}
\CommentTok{\# subjects\_q7 \textless{}{-} df\_items \%\textgreater{}\% filter(q == 7) \%\textgreater{}\% mutate(}
\CommentTok{\#   item\_q7\_NABS = score\_niceABS,}
\CommentTok{\#   item\_q7\_interpretation = interpretation,}
\CommentTok{\#   item\_q7\_rt = rt\_s,}
\CommentTok{\# ) \%\textgreater{}\% dplyr::select(subject, item\_q7\_NABS, item\_q7\_interpretation,item\_q7\_rt) \%\textgreater{}\% arrange(subject)}
\CommentTok{\# }
\CommentTok{\# \#summarize last test item of interest by subject}
\CommentTok{\# subjects\_q15 \textless{}{-} df\_items \%\textgreater{}\% filter(q == 15) \%\textgreater{}\% mutate(}
\CommentTok{\#   item\_q15\_NABS = score\_niceABS,}
\CommentTok{\#   item\_q15\_interpretation = interpretation,}
\CommentTok{\#   item\_q15\_rt = rt\_s,}
\CommentTok{\# ) \%\textgreater{}\% dplyr::select(subject, item\_q15\_NABS, item\_q15\_interpretation,item\_q15\_rt) \%\textgreater{}\% arrange(subject)}
\CommentTok{\# }
\CommentTok{\# \#summarize scaffold phase performance}
\CommentTok{\# subjects\_scaffold \textless{}{-} df\_items \%\textgreater{}\% filter(q\textless{}6)  \%\textgreater{}\% group\_by(subject) \%\textgreater{}\% dplyr::summarise (}
\CommentTok{\#   item\_scaffold\_NABS = sum(score\_niceABS),}
\CommentTok{\#   item\_scaffold\_SCALED = sum(score\_SCALED),}
\CommentTok{\#   item\_scaffold\_rt = sum(rt\_s)}
\CommentTok{\# )\%\textgreater{}\% dplyr::select(subject, item\_scaffold\_NABS, item\_scaffold\_SCALED, item\_scaffold\_rt) \%\textgreater{}\% arrange(subject)}
\CommentTok{\# }
\CommentTok{\# \#summarize test phase performance}
\CommentTok{\# subjects\_test \textless{}{-} df\_items \%\textgreater{}\% filter(q \%nin\% c(1,2,3,4,5,6,9)) \%\textgreater{}\% group\_by(subject) \%\textgreater{}\% dplyr::summarise (}
\CommentTok{\#   item\_test\_NABS = sum(score\_niceABS),}
\CommentTok{\#   item\_test\_SCALED = sum(score\_SCALED),}
\CommentTok{\#   item\_test\_rt = sum(rt\_s)}
\CommentTok{\# )\%\textgreater{}\% dplyr::select(subject, item\_test\_NABS, item\_test\_SCALED, item\_test\_rt) \%\textgreater{}\% arrange(subject)}
\CommentTok{\# }
\CommentTok{\# \#import subjects}
\CommentTok{\# df\_subjects \textless{}{-} read\_rds(\textquotesingle{}analysis/SGC3A/data/1{-}study{-}level/sgc3a\_participants.rds\textquotesingle{}) \%\textgreater{}\% mutate(subject = as.character(subject)) \%\textgreater{}\% arrange(subject)}
\CommentTok{\# }
\CommentTok{\# \#SANITY CHECK SUBJECT ORDER BEFORE MERGE; BOTH SHOULD BE TRUE}
\CommentTok{\# unique(subjects\_summary$subject == df\_subjects$subject)}
\CommentTok{\# unique(subjects\_summary$subject == subjects\_q1$subject)}
\CommentTok{\# unique(subjects\_summary$subject == subjects\_q5$subject)}
\CommentTok{\# unique(subjects\_summary$subject == subjects\_q7$subject)}
\CommentTok{\# unique(subjects\_summary$subject == subjects\_q15$subject)}
\CommentTok{\# unique(subjects\_summary$subject == subjects\_scaffold$subject)}
\CommentTok{\# unique(subjects\_summary$subject == subjects\_test$subject)}
\CommentTok{\# }
\CommentTok{\# \#CAREFULLY CHECK THIS — RELIES ON }
\CommentTok{\# x = merge(df\_subjects, subjects\_summary)}
\CommentTok{\# x = merge(x, subjects\_q1)}
\CommentTok{\# x = merge(x, subjects\_q5)}
\CommentTok{\# x = merge(x, subjects\_q7)}
\CommentTok{\# x = merge(x, subjects\_q15)}
\CommentTok{\# x = merge(x, subjects\_scaffold)}
\CommentTok{\# x = merge(x, subjects\_test)}
\CommentTok{\# df\_subjects \textless{}{-} x \%\textgreater{}\% dplyr::select({-}absolute\_score) \#drop absolute score from webapp that includes Q6 and Q9}
\CommentTok{\# }
\CommentTok{\# \#cleanup}
\CommentTok{\# rm(subjects\_q1, subjects\_q5, subjects\_q7, subjects\_q15, subjects\_scaffold, subjects\_test, subjects\_summary, x)}
\end{Highlighting}
\end{Shaded}

:::

\textbf{Summarize Cummulative Progress versions before functionize}

\begin{Shaded}
\begin{Highlighting}[]
\CommentTok{\# \#SUMMARIZE{-}CUMULATIVE ABSOLUTE PROGRESS}
\CommentTok{\# }
\CommentTok{\# }
\CommentTok{\# \#filter for valid items}
\CommentTok{\# x \textless{}{-} df\_items \%\textgreater{}\% filter(q \%nin\% c(6,9)) \%\textgreater{}\% dplyr::select(subject,mode, pretty\_condition, q,score\_niceABS) }
\CommentTok{\# }
\CommentTok{\# \#pivot wider}
\CommentTok{\# wide \textless{}{-} x \%\textgreater{}\% pivot\_wider(names\_from=q, names\_glue = "q\_\{q\}", values\_from = score\_niceABS)}
\CommentTok{\# }
\CommentTok{\# \#calc stepwise cumulative score}
\CommentTok{\# wide$c1 = wide$q\_1}
\CommentTok{\# wide$c2 = wide$c1 + wide$q\_2}
\CommentTok{\# wide$c3 = wide$c2 + wide$q\_3}
\CommentTok{\# wide$c4 = wide$c3 + wide$q\_4}
\CommentTok{\# wide$c5 = wide$c4 + wide$q\_5}
\CommentTok{\# wide$c6 = wide$c5 + wide$q\_7}
\CommentTok{\# wide$c7 = wide$c6 + wide$q\_8}
\CommentTok{\# wide$c8 = wide$c7 + wide$q\_10}
\CommentTok{\# wide$c9 = wide$c8 + wide$q\_11}
\CommentTok{\# wide$c10 = wide$c9 + wide$q\_12}
\CommentTok{\# wide$c11 = wide$c10 + wide$q\_13}
\CommentTok{\# wide$c12 = wide$c11 + wide$q\_14}
\CommentTok{\# wide$c13 = wide$c12 + wide$q\_15}
\CommentTok{\# wide \textless{}{-} wide \%\textgreater{}\% dplyr::select(subject,mode, pretty\_condition,c1,c2,c3,c4,c5,c6, c7,c8,c9, c10,c11,c12,c13)}
\CommentTok{\# }
\CommentTok{\# \#lengthen }
\CommentTok{\# df\_absolute\_progress \textless{}{-} wide \%\textgreater{}\% pivot\_longer(cols= c1:c13, names\_to = "question", names\_pattern = "c(.*)", values\_to = "score")}
\CommentTok{\# df\_absolute\_progress$question \textless{}{-} as.integer(df\_absolute\_progress$question)}
\CommentTok{\# }
\CommentTok{\# }
\CommentTok{\# \#cleanup }
\CommentTok{\# rm(x,wide)}
\CommentTok{\#   }
\CommentTok{\# \# SUMMARIZE{-}CUMULATIVE SCALED PROGRESS}
\CommentTok{\# }
\CommentTok{\# \#filter for valid items}
\CommentTok{\# x \textless{}{-} df\_items \%\textgreater{}\% filter(q \%nin\% c(6,9)) \%\textgreater{}\% select(subject,mode, pretty\_condition, q,score\_SCALED)}
\CommentTok{\# }
\CommentTok{\# \#pivot wider}
\CommentTok{\# wide \textless{}{-} x \%\textgreater{}\% pivot\_wider(names\_from=q, names\_glue = "q\_\{q\}", values\_from = score\_SCALED)}
\CommentTok{\# }
\CommentTok{\# \#calc stepwise cumulative score}
\CommentTok{\# wide$c1 = wide$q\_1}
\CommentTok{\# wide$c2 = wide$c1 + wide$q\_2}
\CommentTok{\# wide$c3 = wide$c2 + wide$q\_3}
\CommentTok{\# wide$c4 = wide$c3 + wide$q\_4}
\CommentTok{\# wide$c5 = wide$c4 + wide$q\_5}
\CommentTok{\# wide$c6 = wide$c5 + wide$q\_7}
\CommentTok{\# wide$c7 = wide$c6 + wide$q\_8}
\CommentTok{\# wide$c8 = wide$c7 + wide$q\_10}
\CommentTok{\# wide$c9 = wide$c8 + wide$q\_11}
\CommentTok{\# wide$c10 = wide$c9 + wide$q\_12}
\CommentTok{\# wide$c11 = wide$c10 + wide$q\_13}
\CommentTok{\# wide$c12 = wide$c11 + wide$q\_14}
\CommentTok{\# wide$c13 = wide$c12 + wide$q\_15}
\CommentTok{\# wide \textless{}{-} wide \%\textgreater{}\% select(subject,mode, pretty\_condition,c1,c2,c3,c4,c5,c6, c7,c8,c9, c10,c11,c12,c13)}
\CommentTok{\# }
\CommentTok{\# \#lengthen }
\CommentTok{\# df\_scaled\_progress \textless{}{-} wide \%\textgreater{}\% pivot\_longer(cols= c1:c13, names\_to = "question", names\_pattern = "c(.*)", values\_to = "score")}
\CommentTok{\# df\_scaled\_progress$question \textless{}{-} as.integer(df\_scaled\_progress$question)}
\CommentTok{\# }
\CommentTok{\# \#cleanup }
\CommentTok{\# rm(x,wide)}
\end{Highlighting}
\end{Shaded}

\newpage

\hypertarget{sec-SGC3A-description}{%
\chapter{Description}\label{sec-SGC3A-description}}

\emph{The purpose of this notebook is describe the distributions of
dependent variables for Study SGC3A.}

\begin{longtable}[]{@{}
  >{\raggedright\arraybackslash}p{(\columnwidth - 0\tabcolsep) * \real{0.3472}}@{}}
\toprule()
\begin{minipage}[b]{\linewidth}\raggedright
Pre-Requisite
\end{minipage} \\
\midrule()
\endhead
2\_sgc3A\_scoring.qmd \\
\bottomrule()
\end{longtable}

\begin{Shaded}
\begin{Highlighting}[]
\FunctionTok{library}\NormalTok{(Hmisc) }\CommentTok{\# \%nin\% operator}
\FunctionTok{library}\NormalTok{(mosaic) }\CommentTok{\#simple descriptives [favstats]}

\FunctionTok{library}\NormalTok{(kableExtra) }\CommentTok{\#printing tables }
\FunctionTok{library}\NormalTok{(vcd) }\CommentTok{\#mosaicplots}
\FunctionTok{library}\NormalTok{(ggpubr) }\CommentTok{\#arrange plots}
\FunctionTok{library}\NormalTok{(ggformula) }\CommentTok{\#quick easy plots}
\FunctionTok{library}\NormalTok{(ggdist) }\CommentTok{\# uncertainty viz}

\FunctionTok{library}\NormalTok{(multimode) }\CommentTok{\#test for multimodality}
\FunctionTok{library}\NormalTok{(fitdistrplus) }\CommentTok{\#fitting distributions}
\FunctionTok{library}\NormalTok{(performance) }\CommentTok{\#multimodality}

\FunctionTok{library}\NormalTok{(tidyverse) }\CommentTok{\#ALL THE THINGS}

\CommentTok{\#OUTPUT OPTIONS}
\FunctionTok{library}\NormalTok{(dplyr, }\AttributeTok{warn.conflicts =} \ConstantTok{FALSE}\NormalTok{)}
\FunctionTok{options}\NormalTok{(}\AttributeTok{dplyr.summarise.inform =} \ConstantTok{FALSE}\NormalTok{)}
\FunctionTok{options}\NormalTok{(}\AttributeTok{ggplot2.summarise.inform =} \ConstantTok{FALSE}\NormalTok{)}
\FunctionTok{options}\NormalTok{(}\AttributeTok{scipen=}\DecValTok{1}\NormalTok{, }\AttributeTok{digits=}\DecValTok{3}\NormalTok{)}
\end{Highlighting}
\end{Shaded}

\begin{Shaded}
\begin{Highlighting}[]
\CommentTok{\# }\AlertTok{HACK}\CommentTok{ WD FOR LOCAL RUNNING?}
\NormalTok{imac }\OtherTok{=} \StringTok{"/Users/amyraefox/Code/SGC{-}Scaffolding\_Graph\_Comprehension/SGC{-}X/ANALYSIS/MAIN"}
\CommentTok{\# \# mbp = "/Users/amyfox/Sites/RESEARCH/SGC—Scaffolding Graph Comprehension/SGC{-}X/ANALYSIS/MAIN"}
\FunctionTok{setwd}\NormalTok{(imac)}

\CommentTok{\#IMPORT DATA }
\NormalTok{df\_items }\OtherTok{\textless{}{-}} \FunctionTok{read\_rds}\NormalTok{(}\StringTok{\textquotesingle{}analysis/SGC3A/data/2{-}scored{-}data/sgc3a\_scored\_items.rds\textquotesingle{}}\NormalTok{)}
\NormalTok{df\_subjects }\OtherTok{\textless{}{-}} \FunctionTok{read\_rds}\NormalTok{(}\StringTok{\textquotesingle{}analysis/SGC3A/data/2{-}scored{-}data/sgc3a\_scored\_participants.rds\textquotesingle{}}\NormalTok{) }
\NormalTok{df\_absolute\_progress }\OtherTok{\textless{}{-}} \FunctionTok{read\_csv}\NormalTok{(}\StringTok{\textquotesingle{}analysis/SGC3A/data/2{-}scored{-}data/sgc3a\_absolute\_progress.csv\textquotesingle{}}\NormalTok{)}
\NormalTok{df\_scaled\_progress }\OtherTok{\textless{}{-}} \FunctionTok{read\_csv}\NormalTok{(}\StringTok{\textquotesingle{}analysis/SGC3A/data/2{-}scored{-}data/sgc3a\_scaled\_progress.csv\textquotesingle{}}\NormalTok{)}

\CommentTok{\#SEPARATE ITEM DATA BY QUESTION TYPE}
\CommentTok{\# df\_scaffold \textless{}{-} df\_items \%\textgreater{}\% filter(q \textless{} 6)}
\CommentTok{\# df\_test \textless{}{-} df\_items \%\textgreater{}\% filter(q \textgreater{} 6) \%\textgreater{}\% filter (q \%nin\% c(6,9))}
\CommentTok{\# df\_nondiscrim \textless{}{-} df\_items \%\textgreater{}\% filter (q \%in\% c(6,9))}

\CommentTok{\#SEPARATE ITEM AND SUBJECTS BY MODALITY}
\NormalTok{df\_lab }\OtherTok{\textless{}{-}}\NormalTok{ df\_subjects }\SpecialCharTok{\%\textgreater{}\%} \FunctionTok{filter}\NormalTok{(mode }\SpecialCharTok{==} \StringTok{"lab{-}synch"}\NormalTok{)}
\NormalTok{df\_online }\OtherTok{\textless{}{-}}\NormalTok{ df\_subjects }\SpecialCharTok{\%\textgreater{}\%} \FunctionTok{filter}\NormalTok{(mode }\SpecialCharTok{==} \StringTok{"asynch"}\NormalTok{)}
\end{Highlighting}
\end{Shaded}

\hypertarget{sample-1}{%
\section{SAMPLE}\label{sample-1}}

\hypertarget{data-collection}{%
\subsection{Data Collection}\label{data-collection}}

Data was initially collected (in person, SONA groups in computer lab) in
Fall 2017. In Spring 2018, additional data were collected after small
modifications were made to the experimental platform to increase the
size of multiple-choice input buttons, and to add an additional
free-response question following the main task block. In Fall 2021, the
study was replicated using asynchronous, online SONA pool, with
additional participants collected in Winter 2022.

\begin{Shaded}
\begin{Highlighting}[]
\NormalTok{title }\OtherTok{=} \StringTok{"Participants by Condition and Data Collection Period"}
\NormalTok{cols }\OtherTok{=} \FunctionTok{c}\NormalTok{(}\StringTok{"Control Condition"}\NormalTok{,}\StringTok{"Impasse Condition"}\NormalTok{,}\StringTok{"Total for Period"}\NormalTok{)}
\NormalTok{cont }\OtherTok{\textless{}{-}} \FunctionTok{table}\NormalTok{(df\_subjects}\SpecialCharTok{$}\NormalTok{pretty\_mode, df\_subjects}\SpecialCharTok{$}\NormalTok{condition)}
\NormalTok{cont }\SpecialCharTok{\%\textgreater{}\%} \FunctionTok{addmargins}\NormalTok{() }\SpecialCharTok{\%\textgreater{}\%} \FunctionTok{kbl}\NormalTok{(}\AttributeTok{caption =}\NormalTok{ title, }\AttributeTok{col.names =}\NormalTok{ cols) }\SpecialCharTok{\%\textgreater{}\%}  \FunctionTok{kable\_classic}\NormalTok{()}
\end{Highlighting}
\end{Shaded}

\begin{table}

\caption{Participants by Condition and Data Collection Period}
\centering
\begin{tabular}[t]{l|r|r|r}
\hline
  & Control Condition & Impasse Condition & Total for Period\\
\hline
laboratory & 62 & 64 & 126\\
\hline
online-replication & 96 & 108 & 204\\
\hline
Sum & 158 & 172 & 330\\
\hline
\end{tabular}
\end{table}

\hypertarget{participants-1}{%
\subsection{Participants}\label{participants-1}}

\begin{Shaded}
\begin{Highlighting}[]
\CommentTok{\#Describe participants}
\NormalTok{subject.stats }\OtherTok{\textless{}{-}} \FunctionTok{rbind}\NormalTok{(}
  \StringTok{"lab"}\OtherTok{=}\NormalTok{ df\_lab }\SpecialCharTok{\%\textgreater{}\%}\NormalTok{ dplyr}\SpecialCharTok{::}\FunctionTok{select}\NormalTok{(age) }\SpecialCharTok{\%\textgreater{}\%} \FunctionTok{unlist}\NormalTok{() }\SpecialCharTok{\%\textgreater{}\%} \FunctionTok{favstats}\NormalTok{(),}
  \StringTok{"online"} \OtherTok{=}\NormalTok{ df\_online }\SpecialCharTok{\%\textgreater{}\%} \FunctionTok{filter}\NormalTok{(mode }\SpecialCharTok{==} \StringTok{"asynch"}\NormalTok{) }\SpecialCharTok{\%\textgreater{}\%}\NormalTok{ dplyr}\SpecialCharTok{::}\FunctionTok{select}\NormalTok{(age) }\SpecialCharTok{\%\textgreater{}\%} \FunctionTok{unlist}\NormalTok{() }\SpecialCharTok{\%\textgreater{}\%} \FunctionTok{favstats}\NormalTok{(),}
  \StringTok{"combined"} \OtherTok{=}\NormalTok{ df\_subjects }\SpecialCharTok{\%\textgreater{}\%}\NormalTok{ dplyr}\SpecialCharTok{::}\FunctionTok{select}\NormalTok{(age) }\SpecialCharTok{\%\textgreater{}\%} \FunctionTok{unlist}\NormalTok{() }\SpecialCharTok{\%\textgreater{}\%} \FunctionTok{favstats}\NormalTok{()}
\NormalTok{) }
\NormalTok{subject.stats}\SpecialCharTok{$}\NormalTok{percent.female }\OtherTok{\textless{}{-}} \FunctionTok{c}\NormalTok{(}
\NormalTok{  (df\_lab }\SpecialCharTok{\%\textgreater{}\%}  \FunctionTok{filter}\NormalTok{(gender}\SpecialCharTok{==}\StringTok{"Female"}\NormalTok{) }\SpecialCharTok{\%\textgreater{}\%} \FunctionTok{count}\NormalTok{())}\SpecialCharTok{$}\NormalTok{n}\SpecialCharTok{/}\FunctionTok{count}\NormalTok{(df\_lab) }\SpecialCharTok{\%\textgreater{}\%} \FunctionTok{unlist}\NormalTok{(),}
\NormalTok{  (df\_online }\SpecialCharTok{\%\textgreater{}\%} \FunctionTok{filter}\NormalTok{(gender}\SpecialCharTok{==}\StringTok{"Female"}\NormalTok{) }\SpecialCharTok{\%\textgreater{}\%} \FunctionTok{count}\NormalTok{())}\SpecialCharTok{$}\NormalTok{n}\SpecialCharTok{/}\FunctionTok{count}\NormalTok{(df\_online) }\SpecialCharTok{\%\textgreater{}\%} \FunctionTok{unlist}\NormalTok{(),}
\NormalTok{  (df\_subjects }\SpecialCharTok{\%\textgreater{}\%} \FunctionTok{filter}\NormalTok{(gender}\SpecialCharTok{==}\StringTok{"Female"}\NormalTok{) }\SpecialCharTok{\%\textgreater{}\%} \FunctionTok{count}\NormalTok{())}\SpecialCharTok{$}\NormalTok{n}\SpecialCharTok{/}\FunctionTok{count}\NormalTok{(df\_subjects) }\SpecialCharTok{\%\textgreater{}\%} \FunctionTok{unlist}\NormalTok{()}
\NormalTok{)}

\NormalTok{title }\OtherTok{=} \StringTok{"Descriptive Statistics of Participant Age and Gender"}
\NormalTok{subject.stats }\SpecialCharTok{\%\textgreater{}\%} \FunctionTok{kbl}\NormalTok{ (}\AttributeTok{caption =}\NormalTok{ title) }\SpecialCharTok{\%\textgreater{}\%} \FunctionTok{kable\_classic}\NormalTok{()}\SpecialCharTok{\%\textgreater{}\%} 
  \FunctionTok{footnote}\NormalTok{(}\AttributeTok{general =} \StringTok{"Age in Years"}\NormalTok{, }
           \AttributeTok{general\_title =} \StringTok{"Note: "}\NormalTok{,}\AttributeTok{footnote\_as\_chunk =}\NormalTok{ T) }
\end{Highlighting}
\end{Shaded}

\begin{table}

\caption{Descriptive Statistics of Participant Age and Gender}
\centering
\begin{tabular}[t]{l|r|r|r|r|r|r|r|r|r|r}
\hline
  & min & Q1 & median & Q3 & max & mean & sd & n & missing & percent.female\\
\hline
lab & 18 & 19 & 20 & 21 & 33 & 20.4 & 2.12 & 126 & 0 & 0.619\\
\hline
online & 18 & 20 & 20 & 21 & 31 & 20.6 & 2.00 & 204 & 0 & 0.672\\
\hline
combined & 18 & 19 & 20 & 21 & 33 & 20.5 & 2.05 & 330 & 0 & 0.652\\
\hline
\multicolumn{11}{l}{\rule{0pt}{1em}\textit{Note: } Age in Years}\\
\end{tabular}
\end{table}

For \textbf{in-person} collection, 126 participants (62 \% female )
undergraduate STEM majors at a public American University participated
\emph{in person} in exchange for course credit (age: 18 - 33 years).
Participants were randomly assigned to one of two experimental groups.

For \textbf{online replication} 204 participants (67 \% female )
undergraduate STEM majors at a public American University participated
\emph{online, asynchronously} in exchange for course credit (age: 18 -
31 years). Participants were randomly assigned to one of two
experimental groups.

Combined \textbf{overall} 330 participants (65 \% female ) undergraduate
STEM majors at a public American University participated in exchange for
course credit (age: 18 - 33 years).

\hypertarget{response-accuracy}{%
\section{RESPONSE ACCURACY}\label{response-accuracy}}

\hypertarget{subject-level-scores}{%
\subsection{Subject Level Scores}\label{subject-level-scores}}

Subject level scores summarize the the response accuracy by a particular
participant across all discriminant items in the graph comprehension
task.

\hypertarget{test-phase-absolute-score}{%
\subsubsection{Test Phase Absolute
Score}\label{test-phase-absolute-score}}

Recall from \textbf{?@sec-absolute-scoring} that the absolute score
(following the dichotomous scoring approach) \texttt{s\_NABS} indicates
if the subject's response for a particular item was \emph{perfectly}
correct: whether they selected all correct answer options and no others
(excluding certain allowed exceptions, such as also selecting the data
point referenced in the question). The absolute score for an individual
item is either 0 or 1. When summarized across the entire set of
discriminant items, the total absolute score for an individual subject
ranges from {[}0,13{]}. When summarized across just the test phase
(final items following scaffolding phase) scores for an individual
subject range from {[}0,8{]}. First we examine performance on the test
phase (final 8 questions, appears after scaffolding phase). This tells
us how the participants perform \emph{after} exposure to the 5
scaffolding questions (in the impasse condition).

\begin{Shaded}
\begin{Highlighting}[]
\NormalTok{title }\OtherTok{=} \StringTok{"Descriptive Statistics of TEST PHASE Response Accuracy (Total Absolute Score)"}
\NormalTok{abs.stats }\OtherTok{\textless{}{-}} \FunctionTok{rbind}\NormalTok{(}
  \StringTok{"lab"}\OtherTok{=}\NormalTok{ df\_lab }\SpecialCharTok{\%\textgreater{}\%}\NormalTok{ dplyr}\SpecialCharTok{::}\FunctionTok{select}\NormalTok{(item\_test\_NABS) }\SpecialCharTok{\%\textgreater{}\%} \FunctionTok{unlist}\NormalTok{() }\SpecialCharTok{\%\textgreater{}\%} \FunctionTok{favstats}\NormalTok{(),}
  \StringTok{"online"} \OtherTok{=}\NormalTok{ df\_online }\SpecialCharTok{\%\textgreater{}\%}\NormalTok{ dplyr}\SpecialCharTok{::}\FunctionTok{select}\NormalTok{(item\_test\_NABS) }\SpecialCharTok{\%\textgreater{}\%} \FunctionTok{unlist}\NormalTok{() }\SpecialCharTok{\%\textgreater{}\%} \FunctionTok{favstats}\NormalTok{(),}
  \StringTok{"combined"} \OtherTok{=}\NormalTok{ df\_subjects }\SpecialCharTok{\%\textgreater{}\%}\NormalTok{ dplyr}\SpecialCharTok{::}\FunctionTok{select}\NormalTok{(item\_test\_NABS) }\SpecialCharTok{\%\textgreater{}\%} \FunctionTok{unlist}\NormalTok{() }\SpecialCharTok{\%\textgreater{}\%} \FunctionTok{favstats}\NormalTok{()}
\NormalTok{) }
\NormalTok{abs.stats }\SpecialCharTok{\%\textgreater{}\%} \FunctionTok{kbl}\NormalTok{ (}\AttributeTok{caption =}\NormalTok{ title) }\SpecialCharTok{\%\textgreater{}\%} \FunctionTok{kable\_classic}\NormalTok{() }\SpecialCharTok{\%\textgreater{}\%} 
  \FunctionTok{footnote}\NormalTok{(}\AttributeTok{general =} \StringTok{"\# questions correct [0,8]"}\NormalTok{, }
           \AttributeTok{general\_title =} \StringTok{"Note: "}\NormalTok{,}\AttributeTok{footnote\_as\_chunk =}\NormalTok{ T) }
\end{Highlighting}
\end{Shaded}

\begin{table}

\caption{Descriptive Statistics of TEST PHASE Response Accuracy (Total Absolute Score)}
\centering
\begin{tabular}[t]{l|r|r|r|r|r|r|r|r|r}
\hline
  & min & Q1 & median & Q3 & max & mean & sd & n & missing\\
\hline
lab & 0 & 0 & 0 & 6 & 8 & 2.53 & 3.32 & 126 & 0\\
\hline
online & 0 & 0 & 0 & 6 & 8 & 2.16 & 3.19 & 204 & 0\\
\hline
combined & 0 & 0 & 0 & 6 & 8 & 2.30 & 3.24 & 330 & 0\\
\hline
\multicolumn{10}{l}{\rule{0pt}{1em}\textit{Note: } \# questions correct [0,8]}\\
\end{tabular}
\end{table}

For \emph{in person} collection, total absolute scores in the TEST phase
(n = 126) range from 0 to 8 with a mean score of (M = 2.53, SD = 3.32).

For \emph{online replication}, (online) total absolute scores in the
TEST phase (n = 204) range from 0 to 8 with a slightly lower mean score
of (M = 2.16, SD = 3.19).

When combined \emph{overall}, total absolute accuracy scores in the TEST
phase (n = 330) range from 0 to 8 with a slightly lower mean score of (M
= 2.3, SD = 3.24).

\begin{Shaded}
\begin{Highlighting}[]
\CommentTok{\#GGFORMULA | DENSITY HISTOGRAM SUBJECT TOTAL ABSOLUTE}
  \FunctionTok{gf\_props}\NormalTok{(}\SpecialCharTok{\textasciitilde{}}\NormalTok{item\_test\_NABS, }\AttributeTok{data =}\NormalTok{ df\_subjects) }\SpecialCharTok{+} 
  \FunctionTok{labs}\NormalTok{(}\AttributeTok{x =} \StringTok{"number of correct responses (test phase)"}\NormalTok{,}
       \AttributeTok{y =} \StringTok{"\% of subjects"}\NormalTok{,}
       \AttributeTok{title =} \StringTok{"Distribution of TEST Absolute Score "}\NormalTok{,}
       \AttributeTok{subtitle =} \StringTok{"Modes at high and low ends of scale suggest concentration of high (vs) low understanding"}\NormalTok{) }\SpecialCharTok{+} 
  \FunctionTok{theme\_minimal}\NormalTok{()}
\end{Highlighting}
\end{Shaded}

\begin{figure}[H]

{\centering \includegraphics{analysis/SGC3A/3_sgc3A_description_files/figure-pdf/VIS-SUBJ-ABS-TEST-1.pdf}

}

\end{figure}

\begin{Shaded}
\begin{Highlighting}[]
\DocumentationTok{\#\#GGPUBR | HIST+DENSITY SCORE BY CONDITION/MODE}
\NormalTok{p }\OtherTok{\textless{}{-}} \FunctionTok{gghistogram}\NormalTok{(df\_subjects, }\AttributeTok{x =} \StringTok{"item\_test\_NABS"}\NormalTok{, }\AttributeTok{binwidth =} \FloatTok{0.5}\NormalTok{,}
   \AttributeTok{add =} \StringTok{"mean"}\NormalTok{, }\AttributeTok{rug =} \ConstantTok{TRUE}\NormalTok{,}
   \AttributeTok{fill =} \StringTok{"pretty\_condition"}\NormalTok{, }\CommentTok{\#, palette = c("\#00AFBB", "\#E7B800"),}
   \AttributeTok{add\_density =} \ConstantTok{TRUE}\NormalTok{)}
\FunctionTok{facet}\NormalTok{(p, }\AttributeTok{facet.by=}\FunctionTok{c}\NormalTok{(}\StringTok{"pretty\_condition"}\NormalTok{,}\StringTok{"pretty\_mode"}\NormalTok{)) }\SpecialCharTok{+}
  \FunctionTok{labs}\NormalTok{( }\AttributeTok{title =} \StringTok{"Distribution of TEST Absolute Score"}\NormalTok{,}
        \AttributeTok{subtitle =}\StringTok{"Pattern of response is similar across data collection modes but differs by condition"}\NormalTok{,}
        \AttributeTok{x =} \StringTok{"Total Absolute Score (Test Phase)"}\NormalTok{, }\AttributeTok{y =} \StringTok{"number of subjects"}\NormalTok{) }\SpecialCharTok{+}
  \FunctionTok{theme\_minimal}\NormalTok{() }\SpecialCharTok{+} \FunctionTok{theme}\NormalTok{(}\AttributeTok{legend.position =} \StringTok{"blank"}\NormalTok{)}
\end{Highlighting}
\end{Shaded}

\begin{figure}[H]

{\centering \includegraphics{analysis/SGC3A/3_sgc3A_description_files/figure-pdf/VIS-SUBJ-ABS-TEST-2.pdf}

}

\end{figure}

\begin{Shaded}
\begin{Highlighting}[]
\DocumentationTok{\#\#RAINCLOUD USING GGDISTR}
\FunctionTok{ggplot}\NormalTok{(df\_subjects, }\FunctionTok{aes}\NormalTok{(}\AttributeTok{x =}\NormalTok{ pretty\_condition, }\AttributeTok{y =}\NormalTok{ item\_test\_NABS, }\AttributeTok{fill =}\NormalTok{ pretty\_condition)) }\SpecialCharTok{+} 
\NormalTok{  ggdist}\SpecialCharTok{::}\FunctionTok{stat\_halfeye}\NormalTok{(}
    \AttributeTok{adjust =}\NormalTok{ .}\DecValTok{5}\NormalTok{, }
    \AttributeTok{width =}\NormalTok{ .}\DecValTok{6}\NormalTok{, }
    \AttributeTok{.width =} \DecValTok{0}\NormalTok{, }
    \AttributeTok{justification =} \SpecialCharTok{{-}}\NormalTok{.}\DecValTok{3}\NormalTok{, }
    \AttributeTok{point\_colour =} \ConstantTok{NA}\NormalTok{) }\SpecialCharTok{+} 
  \FunctionTok{geom\_boxplot}\NormalTok{(}
    \AttributeTok{width =}\NormalTok{ .}\DecValTok{15}\NormalTok{, }
    \AttributeTok{outlier.shape =} \ConstantTok{NA}
\NormalTok{  ) }\SpecialCharTok{+}
  \FunctionTok{geom\_point}\NormalTok{(}
    \AttributeTok{size =} \FloatTok{1.3}\NormalTok{,}
    \AttributeTok{alpha =}\NormalTok{ .}\DecValTok{3}\NormalTok{,}
    \AttributeTok{position =} \FunctionTok{position\_jitter}\NormalTok{(}
      \AttributeTok{seed =} \DecValTok{1}\NormalTok{, }\AttributeTok{width =}\NormalTok{ .}\DecValTok{1}
\NormalTok{    )}
\NormalTok{  ) }\SpecialCharTok{+} \FunctionTok{labs}\NormalTok{(}
    \AttributeTok{title =} \StringTok{"Distribution of TEST Absolute Score "}\NormalTok{,}
    \AttributeTok{x =} \StringTok{"Condition"}\NormalTok{, }\AttributeTok{y =} \StringTok{"Total Absolute Score (Test Phase)"}
\NormalTok{  ) }\SpecialCharTok{+} \FunctionTok{theme\_ggdist}\NormalTok{() }\SpecialCharTok{+} \FunctionTok{theme}\NormalTok{(}\AttributeTok{legend.position =} \StringTok{"blank"}\NormalTok{)}
\end{Highlighting}
\end{Shaded}

\begin{figure}[H]

{\centering \includegraphics{analysis/SGC3A/3_sgc3A_description_files/figure-pdf/VIS-SUBJ-ABS-TEST-3.pdf}

}

\end{figure}

\begin{Shaded}
\begin{Highlighting}[]
\CommentTok{\# + coord\_cartesian(xlim = c(1.2, NA), clip = "off")}

\CommentTok{\#PLOT EMPIRICIAL CUMULATIVE DISTRIBUTION FUNCTION}
\FunctionTok{ggplot}\NormalTok{(}\AttributeTok{data =}\NormalTok{ df\_subjects, }\FunctionTok{aes}\NormalTok{(item\_test\_NABS)) }\SpecialCharTok{+} 
  \FunctionTok{stat\_ecdf}\NormalTok{(}\AttributeTok{geom =} \StringTok{"step"}\NormalTok{) }\SpecialCharTok{+} 
  \FunctionTok{facet\_grid}\NormalTok{(pretty\_condition}\SpecialCharTok{\textasciitilde{}}\NormalTok{pretty\_mode) }\SpecialCharTok{+} 
  \FunctionTok{labs}\NormalTok{( }\AttributeTok{title =} \StringTok{"Empirical Cumulative Density Function — TEST Absolute Score "}\NormalTok{,}
        \AttributeTok{x =} \StringTok{"Total Absolute Score (Test Phase) [0,8]"}\NormalTok{, }
        \AttributeTok{y =} \StringTok{"Cumulative Probability"}\NormalTok{) }\SpecialCharTok{+} \FunctionTok{theme\_minimal}\NormalTok{()}
\end{Highlighting}
\end{Shaded}

\begin{verbatim}
Warning in grid.Call(C_textBounds, as.graphicsAnnot(x$label), x$x, x$y, :
conversion failure on 'Empirical Cumulative Density Function — TEST Absolute
Score ' in 'mbcsToSbcs': dot substituted for <e2>
\end{verbatim}

\begin{verbatim}
Warning in grid.Call(C_textBounds, as.graphicsAnnot(x$label), x$x, x$y, :
conversion failure on 'Empirical Cumulative Density Function — TEST Absolute
Score ' in 'mbcsToSbcs': dot substituted for <80>
\end{verbatim}

\begin{verbatim}
Warning in grid.Call(C_textBounds, as.graphicsAnnot(x$label), x$x, x$y, :
conversion failure on 'Empirical Cumulative Density Function — TEST Absolute
Score ' in 'mbcsToSbcs': dot substituted for <94>
\end{verbatim}

\begin{verbatim}
Warning in grid.Call(C_textBounds, as.graphicsAnnot(x$label), x$x, x$y, :
conversion failure on 'Empirical Cumulative Density Function — TEST Absolute
Score ' in 'mbcsToSbcs': dot substituted for <e2>
\end{verbatim}

\begin{verbatim}
Warning in grid.Call(C_textBounds, as.graphicsAnnot(x$label), x$x, x$y, :
conversion failure on 'Empirical Cumulative Density Function — TEST Absolute
Score ' in 'mbcsToSbcs': dot substituted for <80>
\end{verbatim}

\begin{verbatim}
Warning in grid.Call(C_textBounds, as.graphicsAnnot(x$label), x$x, x$y, :
conversion failure on 'Empirical Cumulative Density Function — TEST Absolute
Score ' in 'mbcsToSbcs': dot substituted for <94>
\end{verbatim}

\begin{verbatim}
Warning in grid.Call(C_textBounds, as.graphicsAnnot(x$label), x$x, x$y, :
conversion failure on 'Empirical Cumulative Density Function — TEST Absolute
Score ' in 'mbcsToSbcs': dot substituted for <e2>
\end{verbatim}

\begin{verbatim}
Warning in grid.Call(C_textBounds, as.graphicsAnnot(x$label), x$x, x$y, :
conversion failure on 'Empirical Cumulative Density Function — TEST Absolute
Score ' in 'mbcsToSbcs': dot substituted for <80>
\end{verbatim}

\begin{verbatim}
Warning in grid.Call(C_textBounds, as.graphicsAnnot(x$label), x$x, x$y, :
conversion failure on 'Empirical Cumulative Density Function — TEST Absolute
Score ' in 'mbcsToSbcs': dot substituted for <94>
\end{verbatim}

\begin{verbatim}
Warning in grid.Call(C_textBounds, as.graphicsAnnot(x$label), x$x, x$y, :
conversion failure on 'Empirical Cumulative Density Function — TEST Absolute
Score ' in 'mbcsToSbcs': dot substituted for <e2>
\end{verbatim}

\begin{verbatim}
Warning in grid.Call(C_textBounds, as.graphicsAnnot(x$label), x$x, x$y, :
conversion failure on 'Empirical Cumulative Density Function — TEST Absolute
Score ' in 'mbcsToSbcs': dot substituted for <80>
\end{verbatim}

\begin{verbatim}
Warning in grid.Call(C_textBounds, as.graphicsAnnot(x$label), x$x, x$y, :
conversion failure on 'Empirical Cumulative Density Function — TEST Absolute
Score ' in 'mbcsToSbcs': dot substituted for <94>
\end{verbatim}

\begin{verbatim}
Warning in grid.Call(C_textBounds, as.graphicsAnnot(x$label), x$x, x$y, :
conversion failure on 'Empirical Cumulative Density Function — TEST Absolute
Score ' in 'mbcsToSbcs': dot substituted for <e2>
\end{verbatim}

\begin{verbatim}
Warning in grid.Call(C_textBounds, as.graphicsAnnot(x$label), x$x, x$y, :
conversion failure on 'Empirical Cumulative Density Function — TEST Absolute
Score ' in 'mbcsToSbcs': dot substituted for <80>
\end{verbatim}

\begin{verbatim}
Warning in grid.Call(C_textBounds, as.graphicsAnnot(x$label), x$x, x$y, :
conversion failure on 'Empirical Cumulative Density Function — TEST Absolute
Score ' in 'mbcsToSbcs': dot substituted for <94>
\end{verbatim}

\begin{verbatim}
Warning in grid.Call(C_textBounds, as.graphicsAnnot(x$label), x$x, x$y, :
conversion failure on 'Empirical Cumulative Density Function — TEST Absolute
Score ' in 'mbcsToSbcs': dot substituted for <e2>
\end{verbatim}

\begin{verbatim}
Warning in grid.Call(C_textBounds, as.graphicsAnnot(x$label), x$x, x$y, :
conversion failure on 'Empirical Cumulative Density Function — TEST Absolute
Score ' in 'mbcsToSbcs': dot substituted for <80>
\end{verbatim}

\begin{verbatim}
Warning in grid.Call(C_textBounds, as.graphicsAnnot(x$label), x$x, x$y, :
conversion failure on 'Empirical Cumulative Density Function — TEST Absolute
Score ' in 'mbcsToSbcs': dot substituted for <94>
\end{verbatim}

\begin{verbatim}
Warning in grid.Call(C_textBounds, as.graphicsAnnot(x$label), x$x, x$y, :
conversion failure on 'Empirical Cumulative Density Function — TEST Absolute
Score ' in 'mbcsToSbcs': dot substituted for <e2>
\end{verbatim}

\begin{verbatim}
Warning in grid.Call(C_textBounds, as.graphicsAnnot(x$label), x$x, x$y, :
conversion failure on 'Empirical Cumulative Density Function — TEST Absolute
Score ' in 'mbcsToSbcs': dot substituted for <80>
\end{verbatim}

\begin{verbatim}
Warning in grid.Call(C_textBounds, as.graphicsAnnot(x$label), x$x, x$y, :
conversion failure on 'Empirical Cumulative Density Function — TEST Absolute
Score ' in 'mbcsToSbcs': dot substituted for <94>
\end{verbatim}

\begin{verbatim}
Warning in grid.Call.graphics(C_text, as.graphicsAnnot(x$label), x$x, x$y, :
conversion failure on 'Empirical Cumulative Density Function — TEST Absolute
Score ' in 'mbcsToSbcs': dot substituted for <e2>
\end{verbatim}

\begin{verbatim}
Warning in grid.Call.graphics(C_text, as.graphicsAnnot(x$label), x$x, x$y, :
conversion failure on 'Empirical Cumulative Density Function — TEST Absolute
Score ' in 'mbcsToSbcs': dot substituted for <80>
\end{verbatim}

\begin{verbatim}
Warning in grid.Call.graphics(C_text, as.graphicsAnnot(x$label), x$x, x$y, :
conversion failure on 'Empirical Cumulative Density Function — TEST Absolute
Score ' in 'mbcsToSbcs': dot substituted for <94>
\end{verbatim}

\begin{figure}[H]

{\centering \includegraphics{analysis/SGC3A/3_sgc3A_description_files/figure-pdf/VIS-SUBJ-ABS-TEST-4.pdf}

}

\end{figure}

Visual inspection of this distribution suggests it is not normal, and
likely bimodal. We verify this via an excess mass test
(Ameijeiras-Alsonso et. al 2018). TODO REFERENCE

\begin{Shaded}
\begin{Highlighting}[]
\NormalTok{multimode}\SpecialCharTok{::}\FunctionTok{modetest}\NormalTok{(df\_subjects}\SpecialCharTok{$}\NormalTok{item\_test\_NABS)}
\end{Highlighting}
\end{Shaded}

\begin{verbatim}
Warning in multimode::modetest(df_subjects$item_test_NABS): A modification of
the data was made in order to compute the excess mass or the dip statistic
\end{verbatim}

\begin{verbatim}

    Ameijeiras-Alonso et al. (2019) excess mass test

data:  df_subjects$item_test_NABS
Excess mass = 0.1, p-value <2e-16
alternative hypothesis: true number of modes is greater than 1
\end{verbatim}

\begin{Shaded}
\begin{Highlighting}[]
\NormalTok{n\_modes }\OtherTok{=}\NormalTok{ multimode}\SpecialCharTok{::}\FunctionTok{nmodes}\NormalTok{(df\_subjects}\SpecialCharTok{$}\NormalTok{item\_test\_NABS, }\AttributeTok{bw=}\DecValTok{2}\NormalTok{) }\CommentTok{\#bw = 2questions/15 = 0.15\%}
\NormalTok{l\_modes }\OtherTok{=}\NormalTok{ multimode}\SpecialCharTok{::}\FunctionTok{locmodes}\NormalTok{(df\_subjects}\SpecialCharTok{$}\NormalTok{item\_test\_NABS,}\AttributeTok{mod0 =}\NormalTok{  n\_modes, }\AttributeTok{display =} \ConstantTok{TRUE}\NormalTok{)}
\end{Highlighting}
\end{Shaded}

\begin{verbatim}
Warning in multimode::locmodes(df_subjects$item_test_NABS, mod0 = n_modes, : If
the density function has an unbounded support, artificial modes may have been
created in the tails
\end{verbatim}

\begin{figure}[H]

{\centering \includegraphics{analysis/SGC3A/3_sgc3A_description_files/figure-pdf/CHECK-SUBJ-ABS-TEST-1.pdf}

}

\end{figure}

The excess mass test for multimodality suggests the distribution is
infact multimodal (m = 0.1, p \textless{} 0.001), with two identifiable
modes at 0.013 and 7.894, and an antimode at 2.867.

\begin{tcolorbox}[standard jigsaw,bottomrule=.15mm, opacitybacktitle=0.6, bottomtitle=1mm, toptitle=1mm, titlerule=0mm, title=\textcolor{quarto-callout-note-color}{\faInfo}\hspace{0.5em}{Note}, toprule=.15mm, rightrule=.15mm, colback=white, arc=.35mm, left=2mm, colframe=quarto-callout-note-color-frame, coltitle=black, leftrule=.75mm, opacityback=0, colbacktitle=quarto-callout-note-color!10!white]
\textbf{Condition appears (through visual inspection) to yield a
positive influence on Total Absolute Score in the TEST Phase, across
data collection modalities.}
\end{tcolorbox}

\hypertarget{test-phase-scaled-scores}{%
\subsubsection{Test Phase Scaled
Scores}\label{test-phase-scaled-scores}}

The total scaled score \texttt{s\_SCALED} summarizes the scaled score on
the 13 strategy-discriminant questions, for each subject. This score
ranges from from -13 (all orthogonal) to 13 (all triangular). Recall
that the \texttt{s\_SCALED} score for an item is a numeric
representation of the strategy-consistent response, scaled from -1 to +1
(see Section~\ref{sec-SGC3A-scaledScore})

Most importantly, the Scaled score gives us a way of quantitatively
examining how correctly a participant interpreted the coordinate system
across the entire set of items. It offers a more nuanced look into
performance than absolute score.

\begin{Shaded}
\begin{Highlighting}[]
\NormalTok{title }\OtherTok{=} \StringTok{"Descriptive Statistics of Response Accuracy (Total Scaled Score)"}
\NormalTok{scaled.stats }\OtherTok{\textless{}{-}} \FunctionTok{rbind}\NormalTok{(}
  \StringTok{"lab"}\OtherTok{=}\NormalTok{ df\_lab }\SpecialCharTok{\%\textgreater{}\%}\NormalTok{ dplyr}\SpecialCharTok{::}\FunctionTok{select}\NormalTok{(item\_test\_SCALED) }\SpecialCharTok{\%\textgreater{}\%} \FunctionTok{unlist}\NormalTok{() }\SpecialCharTok{\%\textgreater{}\%} \FunctionTok{favstats}\NormalTok{(),}
  \StringTok{"online"} \OtherTok{=}\NormalTok{ df\_online }\SpecialCharTok{\%\textgreater{}\%}\NormalTok{ dplyr}\SpecialCharTok{::}\FunctionTok{select}\NormalTok{(item\_test\_SCALED) }\SpecialCharTok{\%\textgreater{}\%} \FunctionTok{unlist}\NormalTok{() }\SpecialCharTok{\%\textgreater{}\%} \FunctionTok{favstats}\NormalTok{(),}
  \StringTok{"combined"} \OtherTok{=}\NormalTok{ df\_subjects }\SpecialCharTok{\%\textgreater{}\%}\NormalTok{ dplyr}\SpecialCharTok{::}\FunctionTok{select}\NormalTok{(item\_test\_SCALED) }\SpecialCharTok{\%\textgreater{}\%} \FunctionTok{unlist}\NormalTok{() }\SpecialCharTok{\%\textgreater{}\%} \FunctionTok{favstats}\NormalTok{()}
\NormalTok{) }
\NormalTok{scaled.stats }\SpecialCharTok{\%\textgreater{}\%} \FunctionTok{kbl}\NormalTok{ (}\AttributeTok{caption =}\NormalTok{ title) }\SpecialCharTok{\%\textgreater{}\%} \FunctionTok{kable\_classic}\NormalTok{()}
\end{Highlighting}
\end{Shaded}

\begin{table}

\caption{Descriptive Statistics of Response Accuracy (Total Scaled Score)}
\centering
\begin{tabular}[t]{l|r|r|r|r|r|r|r|r|r}
\hline
  & min & Q1 & median & Q3 & max & mean & sd & n & missing\\
\hline
lab & -8 & -8.0 & -6.00 & 6 & 8 & -2.11 & 6.69 & 126 & 0\\
\hline
online & -8 & -7.5 & -5.75 & 5 & 8 & -2.32 & 6.29 & 204 & 0\\
\hline
combined & -8 & -8.0 & -6.00 & 6 & 8 & -2.24 & 6.43 & 330 & 0\\
\hline
\end{tabular}
\end{table}

For \textbf{in person collection}, TEST phase scaled scores (n = 126)
range from -8 to 8 with a mean score of (M = -2.11, SD = 6.69).

For \textbf{online replication}, TEST phase scaled scores (n = 204)
range from -8 to 8 with a slightly lower mean score of (M = -2.32, SD =
6.29).

When combined \textbf{overall}, TEST phase scaled scores (n = 330) range
from -8 to 8 with a slightly lower mean score of (M = -2.24, SD = 6.44).

\begin{Shaded}
\begin{Highlighting}[]
\CommentTok{\#GGFORMULA | DENSITY HISTOGRAM SUBJECT TOTAL SCALED}
\FunctionTok{gf\_props}\NormalTok{(}\SpecialCharTok{\textasciitilde{}}\NormalTok{item\_test\_SCALED, }\AttributeTok{data =}\NormalTok{ df\_subjects) }\SpecialCharTok{+}
  \FunctionTok{labs}\NormalTok{(}\AttributeTok{x =} \StringTok{"total scaled score (test phase)"}\NormalTok{,}
       \AttributeTok{y =} \StringTok{"\% of subjects"}\NormalTok{,}
       \AttributeTok{title =} \StringTok{"Distribution of TEST Scaled Score "}\NormalTok{,}
       \AttributeTok{subtitle =} \StringTok{"Modes at high and low ends of scale suggest concentration of high (vs) low understanding"}\NormalTok{) }\SpecialCharTok{+} 
  \FunctionTok{theme\_minimal}\NormalTok{()}
\end{Highlighting}
\end{Shaded}

\begin{figure}[H]

{\centering \includegraphics{analysis/SGC3A/3_sgc3A_description_files/figure-pdf/VIS-SUBJ-SCALED-TEST-1.pdf}

}

\end{figure}

\begin{Shaded}
\begin{Highlighting}[]
\DocumentationTok{\#\#GGPUBR | HIST+DENSITY SCORE BY CONDITION/MODE}
\NormalTok{p }\OtherTok{\textless{}{-}} \FunctionTok{gghistogram}\NormalTok{(df\_subjects, }\AttributeTok{x =} \StringTok{"item\_test\_SCALED"}\NormalTok{,}\AttributeTok{binwidth=}\DecValTok{1}\NormalTok{,}
   \AttributeTok{add =} \StringTok{"mean"}\NormalTok{, }\AttributeTok{rug =} \ConstantTok{TRUE}\NormalTok{,}
   \AttributeTok{fill =} \StringTok{"pretty\_condition"}\NormalTok{, }\CommentTok{\#, palette = c("\#00AFBB", "\#E7B800"),}
   \AttributeTok{add\_density =} \ConstantTok{TRUE}\NormalTok{) }
\FunctionTok{facet}\NormalTok{(p, }\AttributeTok{facet.by=}\FunctionTok{c}\NormalTok{(}\StringTok{"pretty\_condition"}\NormalTok{,}\StringTok{"pretty\_mode"}\NormalTok{)) }\SpecialCharTok{+} 
  \FunctionTok{labs}\NormalTok{( }\AttributeTok{title =} \StringTok{"Distribution of TEST Scaled Score"}\NormalTok{,}
        \AttributeTok{subtitle =}\StringTok{"Pattern of response is similar across data collection modes but differs by condition"}\NormalTok{,}
        \AttributeTok{x =} \StringTok{"total scaled score (test phase)"}\NormalTok{, }\AttributeTok{y =} \StringTok{"number of participants"}\NormalTok{) }\SpecialCharTok{+} 
  \FunctionTok{theme\_minimal}\NormalTok{() }\SpecialCharTok{+} \FunctionTok{theme}\NormalTok{(}\AttributeTok{legend.position =} \StringTok{"blank"}\NormalTok{) }
\end{Highlighting}
\end{Shaded}

\begin{figure}[H]

{\centering \includegraphics{analysis/SGC3A/3_sgc3A_description_files/figure-pdf/VIS-SUBJ-SCALED-TEST-2.pdf}

}

\end{figure}

\begin{Shaded}
\begin{Highlighting}[]
\DocumentationTok{\#\#RAINCLOUD USING GGDISTR}
\FunctionTok{ggplot}\NormalTok{(df\_subjects, }\FunctionTok{aes}\NormalTok{(}\AttributeTok{x =}\NormalTok{ pretty\_condition, }\AttributeTok{y =}\NormalTok{ item\_test\_SCALED, }\AttributeTok{fill =}\NormalTok{ pretty\_condition)) }\SpecialCharTok{+} 
\NormalTok{  ggdist}\SpecialCharTok{::}\FunctionTok{stat\_halfeye}\NormalTok{(}
    \AttributeTok{adjust =}\NormalTok{ .}\DecValTok{5}\NormalTok{, }
    \AttributeTok{width =}\NormalTok{ .}\DecValTok{6}\NormalTok{, }
    \AttributeTok{.width =} \DecValTok{0}\NormalTok{, }
    \AttributeTok{justification =} \SpecialCharTok{{-}}\NormalTok{.}\DecValTok{3}\NormalTok{, }
    \AttributeTok{point\_colour =} \ConstantTok{NA}\NormalTok{) }\SpecialCharTok{+} 
  \FunctionTok{geom\_boxplot}\NormalTok{(}
    \AttributeTok{width =}\NormalTok{ .}\DecValTok{15}\NormalTok{, }
    \AttributeTok{outlier.shape =} \ConstantTok{NA}
\NormalTok{  ) }\SpecialCharTok{+}
  \FunctionTok{geom\_point}\NormalTok{(}
    \AttributeTok{size =} \FloatTok{1.3}\NormalTok{,}
    \AttributeTok{alpha =}\NormalTok{ .}\DecValTok{3}\NormalTok{,}
    \AttributeTok{position =} \FunctionTok{position\_jitter}\NormalTok{(}
      \AttributeTok{seed =} \DecValTok{1}\NormalTok{, }\AttributeTok{width =}\NormalTok{ .}\DecValTok{1}
\NormalTok{    )}
\NormalTok{  ) }\SpecialCharTok{+} \FunctionTok{labs}\NormalTok{(}
    \AttributeTok{title =} \StringTok{"Distribution of TEST Scaled Score "}\NormalTok{,}
    \AttributeTok{x =} \StringTok{"Condition"}\NormalTok{, }\AttributeTok{y =} \StringTok{"Total Scaled Score (Test Phase)"}
\NormalTok{  ) }\SpecialCharTok{+} \FunctionTok{theme\_ggdist}\NormalTok{() }\SpecialCharTok{+} \FunctionTok{theme}\NormalTok{(}\AttributeTok{legend.position =} \StringTok{"blank"}\NormalTok{)}
\end{Highlighting}
\end{Shaded}

\begin{figure}[H]

{\centering \includegraphics{analysis/SGC3A/3_sgc3A_description_files/figure-pdf/VIS-SUBJ-SCALED-TEST-3.pdf}

}

\end{figure}

\begin{Shaded}
\begin{Highlighting}[]
\CommentTok{\# + coord\_cartesian(xlim = c(1.2, NA), clip = "off")}


\CommentTok{\#PLOT EMPIRICIAL CUMULATIVE DISTRIBUTION FUNCTION}
\FunctionTok{ggplot}\NormalTok{(}\AttributeTok{data =}\NormalTok{ df\_subjects, }\FunctionTok{aes}\NormalTok{(item\_test\_SCALED)) }\SpecialCharTok{+} 
  \FunctionTok{stat\_ecdf}\NormalTok{(}\AttributeTok{geom =} \StringTok{"step"}\NormalTok{) }\SpecialCharTok{+} 
  \FunctionTok{facet\_grid}\NormalTok{(pretty\_condition }\SpecialCharTok{\textasciitilde{}}\NormalTok{ pretty\_mode) }\SpecialCharTok{+} 
  \FunctionTok{labs}\NormalTok{( }\AttributeTok{title =} \StringTok{"Empirical Cumulative Density Function — Test Phase Scaled Score"}\NormalTok{,}
        \AttributeTok{x =} \StringTok{"Test Phase Scaled Score [{-}8,8]"}\NormalTok{, }
        \AttributeTok{y =} \StringTok{"Cumulative Probability"}\NormalTok{) }\SpecialCharTok{+} \FunctionTok{theme\_minimal}\NormalTok{()}
\end{Highlighting}
\end{Shaded}

\begin{verbatim}
Warning in grid.Call(C_textBounds, as.graphicsAnnot(x$label), x$x, x$y, :
conversion failure on 'Empirical Cumulative Density Function — Test Phase Scaled
Score' in 'mbcsToSbcs': dot substituted for <e2>
\end{verbatim}

\begin{verbatim}
Warning in grid.Call(C_textBounds, as.graphicsAnnot(x$label), x$x, x$y, :
conversion failure on 'Empirical Cumulative Density Function — Test Phase Scaled
Score' in 'mbcsToSbcs': dot substituted for <80>
\end{verbatim}

\begin{verbatim}
Warning in grid.Call(C_textBounds, as.graphicsAnnot(x$label), x$x, x$y, :
conversion failure on 'Empirical Cumulative Density Function — Test Phase Scaled
Score' in 'mbcsToSbcs': dot substituted for <94>
\end{verbatim}

\begin{verbatim}
Warning in grid.Call(C_textBounds, as.graphicsAnnot(x$label), x$x, x$y, :
conversion failure on 'Empirical Cumulative Density Function — Test Phase Scaled
Score' in 'mbcsToSbcs': dot substituted for <e2>
\end{verbatim}

\begin{verbatim}
Warning in grid.Call(C_textBounds, as.graphicsAnnot(x$label), x$x, x$y, :
conversion failure on 'Empirical Cumulative Density Function — Test Phase Scaled
Score' in 'mbcsToSbcs': dot substituted for <80>
\end{verbatim}

\begin{verbatim}
Warning in grid.Call(C_textBounds, as.graphicsAnnot(x$label), x$x, x$y, :
conversion failure on 'Empirical Cumulative Density Function — Test Phase Scaled
Score' in 'mbcsToSbcs': dot substituted for <94>
\end{verbatim}

\begin{verbatim}
Warning in grid.Call(C_textBounds, as.graphicsAnnot(x$label), x$x, x$y, :
conversion failure on 'Empirical Cumulative Density Function — Test Phase Scaled
Score' in 'mbcsToSbcs': dot substituted for <e2>
\end{verbatim}

\begin{verbatim}
Warning in grid.Call(C_textBounds, as.graphicsAnnot(x$label), x$x, x$y, :
conversion failure on 'Empirical Cumulative Density Function — Test Phase Scaled
Score' in 'mbcsToSbcs': dot substituted for <80>
\end{verbatim}

\begin{verbatim}
Warning in grid.Call(C_textBounds, as.graphicsAnnot(x$label), x$x, x$y, :
conversion failure on 'Empirical Cumulative Density Function — Test Phase Scaled
Score' in 'mbcsToSbcs': dot substituted for <94>
\end{verbatim}

\begin{verbatim}
Warning in grid.Call(C_textBounds, as.graphicsAnnot(x$label), x$x, x$y, :
conversion failure on 'Empirical Cumulative Density Function — Test Phase Scaled
Score' in 'mbcsToSbcs': dot substituted for <e2>
\end{verbatim}

\begin{verbatim}
Warning in grid.Call(C_textBounds, as.graphicsAnnot(x$label), x$x, x$y, :
conversion failure on 'Empirical Cumulative Density Function — Test Phase Scaled
Score' in 'mbcsToSbcs': dot substituted for <80>
\end{verbatim}

\begin{verbatim}
Warning in grid.Call(C_textBounds, as.graphicsAnnot(x$label), x$x, x$y, :
conversion failure on 'Empirical Cumulative Density Function — Test Phase Scaled
Score' in 'mbcsToSbcs': dot substituted for <94>
\end{verbatim}

\begin{verbatim}
Warning in grid.Call(C_textBounds, as.graphicsAnnot(x$label), x$x, x$y, :
conversion failure on 'Empirical Cumulative Density Function — Test Phase Scaled
Score' in 'mbcsToSbcs': dot substituted for <e2>
\end{verbatim}

\begin{verbatim}
Warning in grid.Call(C_textBounds, as.graphicsAnnot(x$label), x$x, x$y, :
conversion failure on 'Empirical Cumulative Density Function — Test Phase Scaled
Score' in 'mbcsToSbcs': dot substituted for <80>
\end{verbatim}

\begin{verbatim}
Warning in grid.Call(C_textBounds, as.graphicsAnnot(x$label), x$x, x$y, :
conversion failure on 'Empirical Cumulative Density Function — Test Phase Scaled
Score' in 'mbcsToSbcs': dot substituted for <94>
\end{verbatim}

\begin{verbatim}
Warning in grid.Call(C_textBounds, as.graphicsAnnot(x$label), x$x, x$y, :
conversion failure on 'Empirical Cumulative Density Function — Test Phase Scaled
Score' in 'mbcsToSbcs': dot substituted for <e2>
\end{verbatim}

\begin{verbatim}
Warning in grid.Call(C_textBounds, as.graphicsAnnot(x$label), x$x, x$y, :
conversion failure on 'Empirical Cumulative Density Function — Test Phase Scaled
Score' in 'mbcsToSbcs': dot substituted for <80>
\end{verbatim}

\begin{verbatim}
Warning in grid.Call(C_textBounds, as.graphicsAnnot(x$label), x$x, x$y, :
conversion failure on 'Empirical Cumulative Density Function — Test Phase Scaled
Score' in 'mbcsToSbcs': dot substituted for <94>
\end{verbatim}

\begin{verbatim}
Warning in grid.Call(C_textBounds, as.graphicsAnnot(x$label), x$x, x$y, :
conversion failure on 'Empirical Cumulative Density Function — Test Phase Scaled
Score' in 'mbcsToSbcs': dot substituted for <e2>
\end{verbatim}

\begin{verbatim}
Warning in grid.Call(C_textBounds, as.graphicsAnnot(x$label), x$x, x$y, :
conversion failure on 'Empirical Cumulative Density Function — Test Phase Scaled
Score' in 'mbcsToSbcs': dot substituted for <80>
\end{verbatim}

\begin{verbatim}
Warning in grid.Call(C_textBounds, as.graphicsAnnot(x$label), x$x, x$y, :
conversion failure on 'Empirical Cumulative Density Function — Test Phase Scaled
Score' in 'mbcsToSbcs': dot substituted for <94>
\end{verbatim}

\begin{verbatim}
Warning in grid.Call.graphics(C_text, as.graphicsAnnot(x$label), x$x, x$y, :
conversion failure on 'Empirical Cumulative Density Function — Test Phase Scaled
Score' in 'mbcsToSbcs': dot substituted for <e2>
\end{verbatim}

\begin{verbatim}
Warning in grid.Call.graphics(C_text, as.graphicsAnnot(x$label), x$x, x$y, :
conversion failure on 'Empirical Cumulative Density Function — Test Phase Scaled
Score' in 'mbcsToSbcs': dot substituted for <80>
\end{verbatim}

\begin{verbatim}
Warning in grid.Call.graphics(C_text, as.graphicsAnnot(x$label), x$x, x$y, :
conversion failure on 'Empirical Cumulative Density Function — Test Phase Scaled
Score' in 'mbcsToSbcs': dot substituted for <94>
\end{verbatim}

\begin{figure}[H]

{\centering \includegraphics{analysis/SGC3A/3_sgc3A_description_files/figure-pdf/VIS-SUBJ-SCALED-TEST-4.pdf}

}

\end{figure}

Visual inspection of this distribution suggests it is not normal, and
perhaps perhaps bimodal. We verify this via an excess mass test
(Ameijeiras-Alsonso et. al 2018).

\begin{Shaded}
\begin{Highlighting}[]
\NormalTok{multimode}\SpecialCharTok{::}\FunctionTok{modetest}\NormalTok{(df\_subjects}\SpecialCharTok{$}\NormalTok{item\_test\_SCALED)}
\end{Highlighting}
\end{Shaded}

\begin{verbatim}
Warning in multimode::modetest(df_subjects$item_test_SCALED): A modification of
the data was made in order to compute the excess mass or the dip statistic
\end{verbatim}

\begin{verbatim}

    Ameijeiras-Alonso et al. (2019) excess mass test

data:  df_subjects$item_test_SCALED
Excess mass = 0.2, p-value <2e-16
alternative hypothesis: true number of modes is greater than 1
\end{verbatim}

\begin{Shaded}
\begin{Highlighting}[]
\NormalTok{n\_modes }\OtherTok{=}\NormalTok{ multimode}\SpecialCharTok{::}\FunctionTok{nmodes}\NormalTok{(df\_subjects}\SpecialCharTok{$}\NormalTok{item\_test\_SCALED, }\AttributeTok{bw=}\DecValTok{2}\NormalTok{) }\CommentTok{\#bw = 2questions/15 = 0.15\%}
\NormalTok{l\_modes }\OtherTok{=}\NormalTok{ multimode}\SpecialCharTok{::}\FunctionTok{locmodes}\NormalTok{(df\_subjects}\SpecialCharTok{$}\NormalTok{item\_test\_SCALED,}\AttributeTok{mod0 =}\NormalTok{  n\_modes, }\AttributeTok{display =} \ConstantTok{TRUE}\NormalTok{)}
\end{Highlighting}
\end{Shaded}

\begin{verbatim}
Warning in multimode::locmodes(df_subjects$item_test_SCALED, mod0 = n_modes, :
If the density function has an unbounded support, artificial modes may have been
created in the tails
\end{verbatim}

\begin{figure}[H]

{\centering \includegraphics{analysis/SGC3A/3_sgc3A_description_files/figure-pdf/CHECK-SUBJ-SCALED-TEST-1.pdf}

}

\end{figure}

The excess mass test for multimodality suggests the distribution is in
fact multimodal (m = 0.1, p \textless{} 0.001), with two identifiable
modes at -7.721 and 7.822, and an antimode at 1.93.

\hypertarget{first-item-scores}{%
\subsection{First Item Scores}\label{first-item-scores}}

Next we consider the response accuracy on \emph{just} the first question
of the graph comprehension task: a subject's first exposure to the TM
graph.

\hypertarget{first-item-absolute-score}{%
\subsubsection{First Item Absolute
Score}\label{first-item-absolute-score}}

\begin{Shaded}
\begin{Highlighting}[]
\NormalTok{title }\OtherTok{=} \StringTok{"Proportion of Correct Response on First Item (Lab)"}
\NormalTok{item.contingency }\OtherTok{\textless{}{-}}\NormalTok{ df\_lab }\SpecialCharTok{\%\textgreater{}\%}\NormalTok{ dplyr}\SpecialCharTok{::}\FunctionTok{select}\NormalTok{(item\_q1\_NABS, pretty\_condition) }\SpecialCharTok{\%\textgreater{}\%} \FunctionTok{table}\NormalTok{() }\SpecialCharTok{\%\textgreater{}\%} \FunctionTok{prop.table}\NormalTok{() }\SpecialCharTok{\%\textgreater{}\%} \FunctionTok{addmargins}\NormalTok{()}
\NormalTok{item.contingency }\SpecialCharTok{\%\textgreater{}\%} \FunctionTok{kbl}\NormalTok{ (}\AttributeTok{caption =}\NormalTok{ title) }\SpecialCharTok{\%\textgreater{}\%} \FunctionTok{kable\_classic}\NormalTok{()}
\end{Highlighting}
\end{Shaded}

\begin{table}

\caption{Proportion of Correct Response on First Item (Lab)}
\centering
\begin{tabular}[t]{l|r|r|r}
\hline
  & control & impasse & Sum\\
\hline
0 & 0.413 & 0.357 & 0.77\\
\hline
1 & 0.079 & 0.151 & 0.23\\
\hline
Sum & 0.492 & 0.508 & 1.00\\
\hline
\end{tabular}
\end{table}

\begin{Shaded}
\begin{Highlighting}[]
\NormalTok{title }\OtherTok{=} \StringTok{"Proportion of Correct Response on First Item (Online)"}
\NormalTok{item.contingency }\OtherTok{\textless{}{-}}\NormalTok{ df\_online  }\SpecialCharTok{\%\textgreater{}\%}\NormalTok{ dplyr}\SpecialCharTok{::}\FunctionTok{select}\NormalTok{(item\_q1\_NABS, pretty\_condition) }\SpecialCharTok{\%\textgreater{}\%} \FunctionTok{table}\NormalTok{() }\SpecialCharTok{\%\textgreater{}\%} \FunctionTok{prop.table}\NormalTok{() }\SpecialCharTok{\%\textgreater{}\%} \FunctionTok{addmargins}\NormalTok{()}
\NormalTok{item.contingency }\SpecialCharTok{\%\textgreater{}\%} \FunctionTok{kbl}\NormalTok{ (}\AttributeTok{caption =}\NormalTok{ title) }\SpecialCharTok{\%\textgreater{}\%} \FunctionTok{kable\_classic}\NormalTok{()}
\end{Highlighting}
\end{Shaded}

\begin{table}

\caption{Proportion of Correct Response on First Item (Online)}
\centering
\begin{tabular}[t]{l|r|r|r}
\hline
  & control & impasse & Sum\\
\hline
0 & 0.412 & 0.382 & 0.794\\
\hline
1 & 0.059 & 0.147 & 0.206\\
\hline
Sum & 0.471 & 0.529 & 1.000\\
\hline
\end{tabular}
\end{table}

\begin{Shaded}
\begin{Highlighting}[]
\NormalTok{title }\OtherTok{=} \StringTok{"Proportion of Correct Response on First Item (Combined)"}
\NormalTok{item.contingency }\OtherTok{\textless{}{-}}\NormalTok{ df\_subjects }\SpecialCharTok{\%\textgreater{}\%}\NormalTok{  dplyr}\SpecialCharTok{::}\FunctionTok{select}\NormalTok{(item\_q1\_NABS, pretty\_condition) }\SpecialCharTok{\%\textgreater{}\%} \FunctionTok{table}\NormalTok{() }\SpecialCharTok{\%\textgreater{}\%} \FunctionTok{prop.table}\NormalTok{() }\SpecialCharTok{\%\textgreater{}\%} \FunctionTok{addmargins}\NormalTok{()}
\NormalTok{item.contingency }\SpecialCharTok{\%\textgreater{}\%} \FunctionTok{kbl}\NormalTok{ (}\AttributeTok{caption =}\NormalTok{ title) }\SpecialCharTok{\%\textgreater{}\%} \FunctionTok{kable\_classic}\NormalTok{()}
\end{Highlighting}
\end{Shaded}

\begin{table}

\caption{Proportion of Correct Response on First Item (Combined)}
\centering
\begin{tabular}[t]{l|r|r|r}
\hline
  & control & impasse & Sum\\
\hline
0 & 0.412 & 0.373 & 0.785\\
\hline
1 & 0.067 & 0.148 & 0.215\\
\hline
Sum & 0.479 & 0.521 & 1.000\\
\hline
\end{tabular}
\end{table}

Across data collection sessions, first-item accuracy is consistent
across experimental conditions. Incorrect answers are far more frequent
(78\%) than correct answers (22\%). Accuracy is somewhat improved in the
IMPASSE condition, with roughly 15\% of all IMPASSE-condition questions
answered correctly, compared to only 7\% in the CONTROL condition.

\begin{Shaded}
\begin{Highlighting}[]
\CommentTok{\#PROPORTIONAL BAR CHART}
\FunctionTok{gf\_props}\NormalTok{(}\SpecialCharTok{\textasciitilde{}}\NormalTok{item\_q1\_NABS, }\AttributeTok{data =}\NormalTok{ df\_subjects) }\SpecialCharTok{+}
  \FunctionTok{labs}\NormalTok{(}\AttributeTok{x =} \StringTok{"response accuracy"}\NormalTok{,}
       \AttributeTok{y =} \StringTok{"\% subjects"}\NormalTok{,}
       \AttributeTok{title =} \StringTok{"Proportion of Correct Responses on First Item"}\NormalTok{,}
       \AttributeTok{subtitle=}\StringTok{""}\NormalTok{)}\SpecialCharTok{+}
  \FunctionTok{theme\_minimal}\NormalTok{()}\SpecialCharTok{+} \FunctionTok{theme}\NormalTok{(}\AttributeTok{legend.position =} \StringTok{"none"}\NormalTok{)}\SpecialCharTok{+}\FunctionTok{theme\_ggdist}\NormalTok{()}
\end{Highlighting}
\end{Shaded}

\begin{figure}[H]

{\centering \includegraphics{analysis/SGC3A/3_sgc3A_description_files/figure-pdf/VIS-FIRST-ABSOLUTE-1.pdf}

}

\end{figure}

\begin{Shaded}
\begin{Highlighting}[]
\CommentTok{\#PROPORTIONAL BAR CHART}
\FunctionTok{gf\_props}\NormalTok{(}\SpecialCharTok{\textasciitilde{}}\NormalTok{item\_q1\_NABS, }\AttributeTok{data =}\NormalTok{ df\_subjects, }\AttributeTok{fill =} \SpecialCharTok{\textasciitilde{}}\NormalTok{pretty\_condition) }\SpecialCharTok{\%\textgreater{}\%} 
  \FunctionTok{gf\_facet\_grid}\NormalTok{(pretty\_condition}\SpecialCharTok{\textasciitilde{}}\NormalTok{pretty\_mode) }\SpecialCharTok{+}
  \FunctionTok{labs}\NormalTok{(}\AttributeTok{x =} \StringTok{"response accuracy"}\NormalTok{,}
       \AttributeTok{title =} \StringTok{"Proportion of Correct Responses on First Item (by Modality and Condition)"}\NormalTok{,}
       \AttributeTok{subtitle=}\StringTok{""}\NormalTok{)}\SpecialCharTok{+}
  \FunctionTok{theme\_minimal}\NormalTok{()}\SpecialCharTok{+} \FunctionTok{theme}\NormalTok{(}\AttributeTok{legend.position =} \StringTok{"none"}\NormalTok{) }
\end{Highlighting}
\end{Shaded}

\begin{figure}[H]

{\centering \includegraphics{analysis/SGC3A/3_sgc3A_description_files/figure-pdf/VIS-FIRST-ABSOLUTE-2.pdf}

}

\end{figure}

\begin{Shaded}
\begin{Highlighting}[]
\CommentTok{\#MOSAIC PLOT}
\NormalTok{vcd}\SpecialCharTok{::}\FunctionTok{mosaic}\NormalTok{(}\AttributeTok{main=}\StringTok{"Proportion of Correct Responses on First Item"}\NormalTok{,}
            \AttributeTok{data =}\NormalTok{ df\_subjects, pretty\_condition }\SpecialCharTok{\textasciitilde{}}\NormalTok{ item\_q1\_NABS, }\AttributeTok{rot\_labels=}\FunctionTok{c}\NormalTok{(}\DecValTok{0}\NormalTok{,}\DecValTok{90}\NormalTok{,}\DecValTok{0}\NormalTok{,}\DecValTok{0}\NormalTok{), }
            \AttributeTok{offset\_varnames =} \FunctionTok{c}\NormalTok{(}\AttributeTok{left =} \FloatTok{4.5}\NormalTok{), }\AttributeTok{offset\_labels =} \FunctionTok{c}\NormalTok{(}\AttributeTok{left =} \SpecialCharTok{{-}}\FloatTok{0.5}\NormalTok{),}\AttributeTok{just\_labels =} \StringTok{"right"}\NormalTok{,}
            \AttributeTok{spacing =} \FunctionTok{spacing\_dimequal}\NormalTok{(}\FunctionTok{unit}\NormalTok{(}\DecValTok{1}\SpecialCharTok{:}\DecValTok{2}\NormalTok{, }\StringTok{"lines"}\NormalTok{))) }
\end{Highlighting}
\end{Shaded}

\begin{figure}[H]

{\centering \includegraphics{analysis/SGC3A/3_sgc3A_description_files/figure-pdf/VIS-FIRST-ABSOLUTE-3.pdf}

}

\end{figure}

\hypertarget{first-item-scaled-score}{%
\subsubsection{First Item Scaled Score}\label{first-item-scaled-score}}

At the item level, the scaled score gives us a numeric measure of
correctness of interpretation, ranging from -1 to 1. (note: we evaluate
scaled\_score on the first item rather than interpretation, because no
orthogonal interpretation is available in the impasse condition)

\begin{Shaded}
\begin{Highlighting}[]
\NormalTok{title }\OtherTok{=} \StringTok{"Descriptive Statistics of Response Accuracy (First Item Scaled Score)"}
\NormalTok{firstscaled.stats }\OtherTok{\textless{}{-}} \FunctionTok{rbind}\NormalTok{(}
  \StringTok{"lab"}\OtherTok{=}\NormalTok{ df\_lab }\SpecialCharTok{\%\textgreater{}\%}\NormalTok{ dplyr}\SpecialCharTok{::}\FunctionTok{select}\NormalTok{(item\_q1\_SCALED) }\SpecialCharTok{\%\textgreater{}\%} \FunctionTok{unlist}\NormalTok{() }\SpecialCharTok{\%\textgreater{}\%} \FunctionTok{favstats}\NormalTok{(),}
  \StringTok{"online"} \OtherTok{=}\NormalTok{ df\_online }\SpecialCharTok{\%\textgreater{}\%}\NormalTok{ dplyr}\SpecialCharTok{::}\FunctionTok{select}\NormalTok{(item\_q1\_SCALED) }\SpecialCharTok{\%\textgreater{}\%} \FunctionTok{unlist}\NormalTok{() }\SpecialCharTok{\%\textgreater{}\%} \FunctionTok{favstats}\NormalTok{(),}
  \StringTok{"combined"} \OtherTok{=}\NormalTok{ df\_subjects }\SpecialCharTok{\%\textgreater{}\%}\NormalTok{ dplyr}\SpecialCharTok{::}\FunctionTok{select}\NormalTok{(item\_q1\_SCALED) }\SpecialCharTok{\%\textgreater{}\%} \FunctionTok{unlist}\NormalTok{() }\SpecialCharTok{\%\textgreater{}\%} \FunctionTok{favstats}\NormalTok{()}
\NormalTok{) }
\NormalTok{firstscaled.stats }\SpecialCharTok{\%\textgreater{}\%} \FunctionTok{kbl}\NormalTok{ (}\AttributeTok{caption =}\NormalTok{ title) }\SpecialCharTok{\%\textgreater{}\%} \FunctionTok{kable\_classic}\NormalTok{()}
\end{Highlighting}
\end{Shaded}

\begin{table}

\caption{Descriptive Statistics of Response Accuracy (First Item Scaled Score)}
\centering
\begin{tabular}[t]{l|r|r|r|r|r|r|r|r|r}
\hline
  & min & Q1 & median & Q3 & max & mean & sd & n & missing\\
\hline
lab & -1 & -1 & -1 & 0.5 & 1 & -0.298 & 0.849 & 126 & 0\\
\hline
online & -1 & -1 & -1 & 0.5 & 1 & -0.287 & 0.812 & 204 & 0\\
\hline
combined & -1 & -1 & -1 & 0.5 & 1 & -0.291 & 0.825 & 330 & 0\\
\hline
\end{tabular}
\end{table}

For \textbf{in person} collection, first item scaled scores (n = 126)
range from -1 to 1 with a mean score of (M = -0.3, SD = 0.85).

For \textbf{online replication}, (online) first item scaled scores (n =
204) range from -1 to 1 with a slightly lower mean score of (M = -0.29,
SD = 0.81).

When combined \textbf{overall}, first item scaled scores (n = 330) range
from -1 to 1 with a slightly lower mean score of (M = -0.29, SD = 0.83).

\begin{Shaded}
\begin{Highlighting}[]
\CommentTok{\#GGFORMULA | PROPORTIONAL HISTOGRAM SUBJECT FIRST SCALED}
\FunctionTok{gf\_props}\NormalTok{(}\SpecialCharTok{\textasciitilde{}}\NormalTok{item\_q1\_SCALED, }\AttributeTok{data =}\NormalTok{ df\_subjects) }\SpecialCharTok{+}
  \FunctionTok{labs}\NormalTok{(}\AttributeTok{x =} \StringTok{"scaled score (first item)"}\NormalTok{,}
       \AttributeTok{y =} \StringTok{"\% of subjects"}\NormalTok{,}
       \AttributeTok{title =} \StringTok{"Distribution of First Item Scaled Score"}\NormalTok{,}
       \AttributeTok{subtitle =} \StringTok{""}\NormalTok{) }\SpecialCharTok{+} 
  \FunctionTok{theme\_minimal}\NormalTok{()}
\end{Highlighting}
\end{Shaded}

\begin{figure}[H]

{\centering \includegraphics{analysis/SGC3A/3_sgc3A_description_files/figure-pdf/VIS-FIRST-SCALED-1.pdf}

}

\end{figure}

\begin{Shaded}
\begin{Highlighting}[]
\DocumentationTok{\#\#GGPUBR | HIST+DENSITY SCORE BY CONDITION/MODE}
\NormalTok{p }\OtherTok{\textless{}{-}} \FunctionTok{gghistogram}\NormalTok{(df\_subjects, }\AttributeTok{x =} \StringTok{"item\_q1\_SCALED"}\NormalTok{, }\AttributeTok{binwidth =} \FloatTok{0.5}\NormalTok{,}
   \AttributeTok{add =} \StringTok{"mean"}\NormalTok{, }\AttributeTok{rug =} \ConstantTok{TRUE}\NormalTok{,}
   \AttributeTok{fill =} \StringTok{"pretty\_condition"}\NormalTok{, }\CommentTok{\#, palette = c("\#00AFBB", "\#E7B800"),}
   \AttributeTok{add\_density =} \ConstantTok{TRUE}\NormalTok{) }
\FunctionTok{facet}\NormalTok{(p, }\AttributeTok{facet.by=}\FunctionTok{c}\NormalTok{(}\StringTok{"pretty\_condition"}\NormalTok{,}\StringTok{"pretty\_mode"}\NormalTok{)) }\SpecialCharTok{+} 
  \FunctionTok{labs}\NormalTok{( }\AttributeTok{title =} \StringTok{"Distribution of First Item Scaled Score (by Mode and Condition)"}\NormalTok{,}
        \AttributeTok{subtitle =}\StringTok{"Impasse condition yields more intermediate scores (indicating uncertainty)"}\NormalTok{,}
        \AttributeTok{x =} \StringTok{"scaled score (firt item) "}\NormalTok{, }\AttributeTok{y =} \StringTok{"number of participants"}\NormalTok{) }\SpecialCharTok{+} 
  \FunctionTok{theme\_minimal}\NormalTok{() }\SpecialCharTok{+} \FunctionTok{theme}\NormalTok{(}\AttributeTok{legend.position =} \StringTok{"blank"}\NormalTok{) }
\end{Highlighting}
\end{Shaded}

\begin{figure}[H]

{\centering \includegraphics{analysis/SGC3A/3_sgc3A_description_files/figure-pdf/VIS-FIRST-SCALED-2.pdf}

}

\end{figure}

\begin{Shaded}
\begin{Highlighting}[]
\DocumentationTok{\#\#GGFORMULA | HIST+DENSITY SCORE BY CONDITION/MODE}
\CommentTok{\# stats = df\_subjects \%\textgreater{}\% group\_by(pretty\_condition, mode) \%\textgreater{}\% dplyr::summarise(mean = mean(item\_q1\_SCALED))}
\CommentTok{\# gf\_density(\textasciitilde{}item\_q1\_SCALED, data = df\_subjects) \%\textgreater{}\%}
\CommentTok{\#   gf\_facet\_grid(pretty\_condition\textasciitilde{}mode, labeller = label\_both) \%\textgreater{}\%}
\CommentTok{\#   gf\_lims(x = c({-}1, 1)) \%\textgreater{}\%}
\CommentTok{\#   gf\_vline(data = stats, xintercept = \textasciitilde{}mean, color = "red") +}
\CommentTok{\# labs( title = "Distribution of First Item Scaled Score (by Mode and Condition)",}
\CommentTok{\#         subtitle ="Pattern of response is the same across data collection modes but differs by condition",}
\CommentTok{\#         x = "scaled score (firt item) ", y = "number of participants") + }
\CommentTok{\#   theme\_minimal()}
\end{Highlighting}
\end{Shaded}

\hypertarget{interpretation-scores}{%
\subsection{Interpretation Scores}\label{interpretation-scores}}

Next we consider the the interpretations assigned to each response. For
each response given by a participant to a question, we assign an
interpretation label based on the interpretation the response most
closely matches (see \textbf{?@sec-scoring-interpretation}).

\begin{Shaded}
\begin{Highlighting}[]
\NormalTok{title }\OtherTok{=} \StringTok{"Proportion of Interpretations Across Items Items By Condition (Lab)"}
\NormalTok{item.contingency }\OtherTok{\textless{}{-}}\NormalTok{ df\_items }\SpecialCharTok{\%\textgreater{}\%} \FunctionTok{filter}\NormalTok{(mode }\SpecialCharTok{==} \StringTok{"lab{-}synch"}\NormalTok{) }\SpecialCharTok{\%\textgreater{}\%}\NormalTok{ dplyr}\SpecialCharTok{::}\FunctionTok{select}\NormalTok{(interpretation, pretty\_condition) }\SpecialCharTok{\%\textgreater{}\%} \FunctionTok{table}\NormalTok{() }\SpecialCharTok{\%\textgreater{}\%} \FunctionTok{prop.table}\NormalTok{() }\SpecialCharTok{\%\textgreater{}\%} \FunctionTok{addmargins}\NormalTok{()}
\NormalTok{item.contingency }\SpecialCharTok{\%\textgreater{}\%} \FunctionTok{kbl}\NormalTok{ (}\AttributeTok{caption =}\NormalTok{ title) }\SpecialCharTok{\%\textgreater{}\%} \FunctionTok{kable\_classic}\NormalTok{()}
\end{Highlighting}
\end{Shaded}

\begin{table}

\caption{Proportion of Interpretations Across Items Items By Condition (Lab)}
\centering
\begin{tabular}[t]{l|r|r|r}
\hline
  & control & impasse & Sum\\
\hline
Orthogonal & 0.297 & 0.116 & 0.414\\
\hline
Satisfice & 0.000 & 0.028 & 0.028\\
\hline
frenzy & 0.002 & 0.005 & 0.007\\
\hline
? & 0.026 & 0.053 & 0.079\\
\hline
reference & 0.001 & 0.004 & 0.005\\
\hline
blank & 0.008 & 0.034 & 0.042\\
\hline
both tri + orth & 0.060 & 0.056 & 0.116\\
\hline
Tversky & 0.004 & 0.017 & 0.021\\
\hline
Triangular & 0.094 & 0.195 & 0.288\\
\hline
Sum & 0.492 & 0.508 & 1.000\\
\hline
\end{tabular}
\end{table}

\begin{Shaded}
\begin{Highlighting}[]
\NormalTok{title }\OtherTok{=} \StringTok{"Proportion of Interpretations Across Items Items By Condition (Online)"}
\NormalTok{item.contingency }\OtherTok{\textless{}{-}}\NormalTok{ df\_items }\SpecialCharTok{\%\textgreater{}\%} \FunctionTok{filter}\NormalTok{(mode }\SpecialCharTok{==} \StringTok{"asynch"}\NormalTok{) }\SpecialCharTok{\%\textgreater{}\%}\NormalTok{ dplyr}\SpecialCharTok{::}\FunctionTok{select}\NormalTok{(interpretation, pretty\_condition) }\SpecialCharTok{\%\textgreater{}\%} \FunctionTok{table}\NormalTok{() }\SpecialCharTok{\%\textgreater{}\%} \FunctionTok{prop.table}\NormalTok{() }\SpecialCharTok{\%\textgreater{}\%} \FunctionTok{addmargins}\NormalTok{()}
\NormalTok{item.contingency }\SpecialCharTok{\%\textgreater{}\%} \FunctionTok{kbl}\NormalTok{ (}\AttributeTok{caption =}\NormalTok{ title) }\SpecialCharTok{\%\textgreater{}\%} \FunctionTok{kable\_classic}\NormalTok{()}
\end{Highlighting}
\end{Shaded}

\begin{table}

\caption{Proportion of Interpretations Across Items Items By Condition (Online)}
\centering
\begin{tabular}[t]{l|r|r|r}
\hline
  & control & impasse & Sum\\
\hline
Orthogonal & 0.260 & 0.122 & 0.382\\
\hline
Satisfice & 0.000 & 0.024 & 0.024\\
\hline
frenzy & 0.002 & 0.001 & 0.003\\
\hline
? & 0.050 & 0.066 & 0.116\\
\hline
reference & 0.000 & 0.002 & 0.002\\
\hline
blank & 0.013 & 0.055 & 0.068\\
\hline
both tri + orth & 0.056 & 0.061 & 0.117\\
\hline
Tversky & 0.011 & 0.023 & 0.035\\
\hline
Triangular & 0.078 & 0.175 & 0.253\\
\hline
Sum & 0.471 & 0.529 & 1.000\\
\hline
\end{tabular}
\end{table}

\begin{Shaded}
\begin{Highlighting}[]
\NormalTok{title }\OtherTok{=} \StringTok{"Proportion of Interpretations Across Items Items By Condition (Combined)"}
\NormalTok{item.contingency }\OtherTok{\textless{}{-}}\NormalTok{ df\_items }\SpecialCharTok{\%\textgreater{}\%}\NormalTok{  dplyr}\SpecialCharTok{::}\FunctionTok{select}\NormalTok{(interpretation, pretty\_condition) }\SpecialCharTok{\%\textgreater{}\%} \FunctionTok{table}\NormalTok{() }\SpecialCharTok{\%\textgreater{}\%} \FunctionTok{prop.table}\NormalTok{() }\SpecialCharTok{\%\textgreater{}\%} \FunctionTok{addmargins}\NormalTok{()}
\NormalTok{item.contingency }\SpecialCharTok{\%\textgreater{}\%} \FunctionTok{kbl}\NormalTok{ (}\AttributeTok{caption =}\NormalTok{ title) }\SpecialCharTok{\%\textgreater{}\%} \FunctionTok{kable\_classic}\NormalTok{()}
\end{Highlighting}
\end{Shaded}

\begin{table}

\caption{Proportion of Interpretations Across Items Items By Condition (Combined)}
\centering
\begin{tabular}[t]{l|r|r|r}
\hline
  & control & impasse & Sum\\
\hline
Orthogonal & 0.274 & 0.120 & 0.394\\
\hline
Satisfice & 0.000 & 0.025 & 0.025\\
\hline
frenzy & 0.002 & 0.003 & 0.004\\
\hline
? & 0.041 & 0.061 & 0.102\\
\hline
reference & 0.001 & 0.002 & 0.003\\
\hline
blank & 0.011 & 0.047 & 0.058\\
\hline
both tri + orth & 0.058 & 0.059 & 0.117\\
\hline
Tversky & 0.009 & 0.021 & 0.029\\
\hline
Triangular & 0.084 & 0.183 & 0.267\\
\hline
Sum & 0.479 & 0.521 & 1.000\\
\hline
\end{tabular}
\end{table}

\begin{Shaded}
\begin{Highlighting}[]
\CommentTok{\#PROPORTIONAL BAR CHART}
\FunctionTok{gf\_propsh}\NormalTok{(}\SpecialCharTok{\textasciitilde{}}\NormalTok{interpretation, }\AttributeTok{data =}\NormalTok{ df\_items, }\AttributeTok{fill =} \SpecialCharTok{\textasciitilde{}}\NormalTok{pretty\_condition) }\SpecialCharTok{\%\textgreater{}\%} 
  \FunctionTok{gf\_facet\_grid}\NormalTok{(pretty\_condition}\SpecialCharTok{\textasciitilde{}}\NormalTok{pretty\_mode) }\SpecialCharTok{+}
  \FunctionTok{labs}\NormalTok{(}\AttributeTok{x =} \StringTok{"\% of items"}\NormalTok{,}
       \AttributeTok{title =} \StringTok{"Proportion of Interpretations Across Items"}\NormalTok{,}
       \AttributeTok{subtitle=}\StringTok{"Impasse Condition yields shift from Orthogonal to alternative interpretations"}\NormalTok{)}\SpecialCharTok{+}
  \FunctionTok{theme\_minimal}\NormalTok{()}\SpecialCharTok{+} \FunctionTok{theme}\NormalTok{(}\AttributeTok{legend.position =} \StringTok{"none"}\NormalTok{)}
\end{Highlighting}
\end{Shaded}

\begin{figure}[H]

{\centering \includegraphics{analysis/SGC3A/3_sgc3A_description_files/figure-pdf/VIS-ITEM-INTERPRETATION-1.pdf}

}

\end{figure}

\begin{Shaded}
\begin{Highlighting}[]
\CommentTok{\#MOSAIC PLOT}
\CommentTok{\# vcd::mosaic(main="Proportion of Interpretations across Conditions",}
\CommentTok{\#             data = df\_items, pretty\_condition \textasciitilde{} interpretation, rot\_labels=c(0,90,0,0), }
\CommentTok{\#             offset\_varnames = c(left = 4.5), offset\_labels = c(left = {-}0.5),just\_labels = "right",}
\CommentTok{\#             spacing = spacing\_dimequal(unit(1:2, "lines"))) }
\end{Highlighting}
\end{Shaded}

\hypertarget{cumulative-task-performance}{%
\subsection{Cumulative Task
Performance}\label{cumulative-task-performance}}

\begin{Shaded}
\begin{Highlighting}[]
\CommentTok{\#VISUALIZE progress over time ABSOLUTE score }
\FunctionTok{ggplot}\NormalTok{(}\AttributeTok{data =}\NormalTok{ df\_absolute\_progress, }\FunctionTok{aes}\NormalTok{(}\AttributeTok{x =}\NormalTok{ question, }\AttributeTok{y =}\NormalTok{ score, }\AttributeTok{group =}\NormalTok{ subject, }\AttributeTok{alpha =} \FloatTok{0.01}\NormalTok{, }\AttributeTok{color =}\NormalTok{ pretty\_condition)) }\SpecialCharTok{+} 
 \FunctionTok{geom\_line}\NormalTok{(}\AttributeTok{position=}\FunctionTok{position\_jitter}\NormalTok{(}\AttributeTok{w=}\FloatTok{0.15}\NormalTok{, }\AttributeTok{h=}\FloatTok{0.15}\NormalTok{), }\AttributeTok{size=}\FloatTok{0.1}\NormalTok{) }\SpecialCharTok{+}
 \FunctionTok{facet\_wrap}\NormalTok{(}\SpecialCharTok{\textasciitilde{}}\NormalTok{pretty\_condition) }\SpecialCharTok{+} 
 \FunctionTok{labs}\NormalTok{ (}\AttributeTok{title =} \StringTok{"Cumulative Absolute Score over sequence of task"}\NormalTok{, }\AttributeTok{x =} \StringTok{"Question"}\NormalTok{ , }\AttributeTok{y =} \StringTok{"Cumulative Absolute Score"}\NormalTok{) }\SpecialCharTok{+} 
 \FunctionTok{scale\_x\_continuous}\NormalTok{(}\AttributeTok{breaks =} \FunctionTok{c}\NormalTok{(}\DecValTok{1}\NormalTok{,}\DecValTok{2}\NormalTok{,}\DecValTok{3}\NormalTok{,}\DecValTok{4}\NormalTok{,}\DecValTok{5}\NormalTok{,}\DecValTok{6}\NormalTok{,}\DecValTok{7}\NormalTok{,}\DecValTok{8}\NormalTok{,}\DecValTok{9}\NormalTok{,}\DecValTok{10}\NormalTok{,}\DecValTok{11}\NormalTok{,}\DecValTok{12}\NormalTok{,}\DecValTok{13}\NormalTok{)) }\SpecialCharTok{+}
 \FunctionTok{theme\_minimal}\NormalTok{() }\SpecialCharTok{+} \FunctionTok{theme}\NormalTok{(}\AttributeTok{legend.position =} \StringTok{"blank"}\NormalTok{)}
\end{Highlighting}
\end{Shaded}

\begin{figure}[H]

{\centering \includegraphics{analysis/SGC3A/3_sgc3A_description_files/figure-pdf/VIZ-PROGRESS-1.pdf}

}

\end{figure}

\begin{Shaded}
\begin{Highlighting}[]
\CommentTok{\#VISUALIZE progress over time SCALED score }
\FunctionTok{ggplot}\NormalTok{(}\AttributeTok{data =}\NormalTok{ df\_scaled\_progress, }\FunctionTok{aes}\NormalTok{(}\AttributeTok{x =}\NormalTok{ question, }\AttributeTok{y =}\NormalTok{ score, }\AttributeTok{group =}\NormalTok{ subject, }\AttributeTok{alpha =} \FloatTok{0.01}\NormalTok{, }\AttributeTok{color =}\NormalTok{ pretty\_condition)) }\SpecialCharTok{+} 
 \FunctionTok{geom\_line}\NormalTok{(}\AttributeTok{position=}\FunctionTok{position\_jitter}\NormalTok{(}\AttributeTok{w=}\FloatTok{0.15}\NormalTok{, }\AttributeTok{h=}\FloatTok{0.15}\NormalTok{), }\AttributeTok{size=}\FloatTok{0.1}\NormalTok{) }\SpecialCharTok{+}
 \FunctionTok{facet\_wrap}\NormalTok{(}\SpecialCharTok{\textasciitilde{}}\NormalTok{pretty\_condition) }\SpecialCharTok{+} 
 \FunctionTok{labs}\NormalTok{ (}\AttributeTok{title =} \StringTok{"Cumulative Scaled Score over sequence of task"}\NormalTok{, }\AttributeTok{x =} \StringTok{"Question"}\NormalTok{ , }\AttributeTok{y =} \StringTok{"Cumulative Scaled Score"}\NormalTok{) }\SpecialCharTok{+} 
 \FunctionTok{scale\_x\_continuous}\NormalTok{(}\AttributeTok{breaks =} \FunctionTok{c}\NormalTok{(}\DecValTok{1}\NormalTok{,}\DecValTok{2}\NormalTok{,}\DecValTok{3}\NormalTok{,}\DecValTok{4}\NormalTok{,}\DecValTok{5}\NormalTok{,}\DecValTok{6}\NormalTok{,}\DecValTok{7}\NormalTok{,}\DecValTok{8}\NormalTok{,}\DecValTok{9}\NormalTok{,}\DecValTok{10}\NormalTok{,}\DecValTok{11}\NormalTok{,}\DecValTok{12}\NormalTok{,}\DecValTok{13}\NormalTok{)) }\SpecialCharTok{+}
 \FunctionTok{theme\_minimal}\NormalTok{() }\SpecialCharTok{+} \FunctionTok{theme}\NormalTok{(}\AttributeTok{legend.position =} \StringTok{"blank"}\NormalTok{)}
\end{Highlighting}
\end{Shaded}

\begin{figure}[H]

{\centering \includegraphics{analysis/SGC3A/3_sgc3A_description_files/figure-pdf/VIZ-PROGRESS-2.pdf}

}

\end{figure}

\hypertarget{response-latency}{%
\section{RESPONSE LATENCY}\label{response-latency}}

\begin{itemize}
\tightlist
\item
  {TODO: Investigate super high and super low response times.}.
\item
  {TODO: Investigate appropriate models for response time data. (see:
  https://lindeloev.github.io/shiny-rt/)}.
\item
  Especially see https://lindeloev.github.io/shiny-rt/ for ideas on
  modelling reaction time data
\end{itemize}

\hypertarget{time-on-first-item}{%
\subsection{Time on First Item}\label{time-on-first-item}}

Here we consider the time spent on just the first individual item (first
exposure to graph).

\begin{Shaded}
\begin{Highlighting}[]
\CommentTok{\#DESCRIBE distribution of response time}
\NormalTok{time.stats }\OtherTok{\textless{}{-}} \FunctionTok{rbind}\NormalTok{(}
  \StringTok{"lab"}\OtherTok{=}\NormalTok{ df\_lab}\SpecialCharTok{\%\textgreater{}\%}\NormalTok{ dplyr}\SpecialCharTok{::}\FunctionTok{select}\NormalTok{(item\_q1\_rt) }\SpecialCharTok{\%\textgreater{}\%} \FunctionTok{unlist}\NormalTok{()  }\SpecialCharTok{\%\textgreater{}\%}  \FunctionTok{favstats}\NormalTok{(),}
  \StringTok{"online"}\OtherTok{=}\NormalTok{ df\_online }\SpecialCharTok{\%\textgreater{}\%}\NormalTok{ dplyr}\SpecialCharTok{::}\FunctionTok{select}\NormalTok{(item\_q1\_rt) }\SpecialCharTok{\%\textgreater{}\%} \FunctionTok{unlist}\NormalTok{() }\SpecialCharTok{\%\textgreater{}\%} \FunctionTok{favstats}\NormalTok{(),}
  \StringTok{"combined"}\OtherTok{=}\NormalTok{ df\_subjects }\SpecialCharTok{\%\textgreater{}\%}\NormalTok{ dplyr}\SpecialCharTok{::}\FunctionTok{select}\NormalTok{(item\_q1\_rt) }\SpecialCharTok{\%\textgreater{}\%} \FunctionTok{unlist}\NormalTok{() }\SpecialCharTok{\%\textgreater{}\%} \FunctionTok{favstats}\NormalTok{()}
\NormalTok{)}

\NormalTok{title }\OtherTok{=} \StringTok{"Descriptive Statistics of First Response Time (seconds)"}
\NormalTok{time.stats }\SpecialCharTok{\%\textgreater{}\%} \FunctionTok{kbl}\NormalTok{(}\AttributeTok{caption =}\NormalTok{ title) }\SpecialCharTok{\%\textgreater{}\%} \FunctionTok{kable\_classic}\NormalTok{()}
\end{Highlighting}
\end{Shaded}

\begin{table}

\caption{Descriptive Statistics of First Response Time (seconds)}
\centering
\begin{tabular}[t]{l|r|r|r|r|r|r|r|r|r}
\hline
  & min & Q1 & median & Q3 & max & mean & sd & n & missing\\
\hline
lab & 7.22 & 26.6 & 39.3 & 52.2 & 161 & 44.5 & 26.2 & 126 & 0\\
\hline
online & 4.84 & 19.9 & 31.0 & 48.9 & 306 & 43.3 & 41.3 & 204 & 0\\
\hline
combined & 4.84 & 22.3 & 34.0 & 50.7 & 306 & 43.8 & 36.2 & 330 & 0\\
\hline
\end{tabular}
\end{table}

Response time \emph{on the first item} for \emph{in person} subjects (n
= 126) ranged from 7.22 to 161.36 minutes with a mean duration of (M =
44.53, SD = 26.22).

Response time \emph{on the first item} for \emph{online replication}
subjects (n = 204) ranged from 4.84 to 305.94 minutes with a mean
duration of (M = 43.32, SD = 41.27).

Response time \emph{on the first item} for \emph{combined} subjects (n =
330) ranged from 4.84 to 305.94 minutes with a mean duration of (M =
43.78, SD = 36.23).

\begin{Shaded}
\begin{Highlighting}[]
\CommentTok{\#HISTOGRAM}
\FunctionTok{gf\_dhistogram}\NormalTok{(}\SpecialCharTok{\textasciitilde{}}\NormalTok{item\_q1\_rt, }\AttributeTok{data =}\NormalTok{ df\_subjects) }\SpecialCharTok{\%\textgreater{}\%}
  \FunctionTok{gf\_vline}\NormalTok{(}\AttributeTok{xintercept =} \SpecialCharTok{\textasciitilde{}}\NormalTok{time.stats[}\StringTok{"lab"}\NormalTok{,]}\SpecialCharTok{$}\NormalTok{mean, }\AttributeTok{color =} \StringTok{"black"}\NormalTok{) }\SpecialCharTok{\%\textgreater{}\%} 
  \FunctionTok{gf\_fitdistr}\NormalTok{(}\AttributeTok{dist=}\StringTok{"gamma"}\NormalTok{, }\AttributeTok{color=}\StringTok{"red"}\NormalTok{)}\SpecialCharTok{+}
  \FunctionTok{labs}\NormalTok{(}\AttributeTok{title=}\StringTok{"Distribution of First Item Response Time (seconds)"}\NormalTok{, }\AttributeTok{subtitle =} \StringTok{"fit by gamma distribution"}\NormalTok{, }\AttributeTok{x =} \StringTok{"First Item Response Time (seconds)"}\NormalTok{, }\AttributeTok{y =} \StringTok{"\% items"}\NormalTok{) }\SpecialCharTok{+}  \FunctionTok{theme\_minimal}\NormalTok{()}
\end{Highlighting}
\end{Shaded}

\begin{verbatim}
Warning in densfun(x, parm[1], parm[2], ...): NaNs produced

Warning in densfun(x, parm[1], parm[2], ...): NaNs produced
\end{verbatim}

\begin{figure}[H]

{\centering \includegraphics{analysis/SGC3A/3_sgc3A_description_files/figure-pdf/label - VIS-FIRSTTIMEmessage - falsewarning - false-1.pdf}

}

\end{figure}

\begin{Shaded}
\begin{Highlighting}[]
\DocumentationTok{\#\#GGPUBR | HIST+DENSITY SCORE BY CONDITION/MODE}
\NormalTok{p }\OtherTok{\textless{}{-}} \FunctionTok{gghistogram}\NormalTok{(df\_subjects, }\AttributeTok{x =} \StringTok{"item\_q1\_rt"}\NormalTok{, }\AttributeTok{binwidth =} \FloatTok{0.5}\NormalTok{,}
   \AttributeTok{add =} \StringTok{"mean"}\NormalTok{, }\AttributeTok{rug =} \ConstantTok{TRUE}\NormalTok{,}
   \AttributeTok{fill =} \StringTok{"pretty\_condition"}\NormalTok{, }\CommentTok{\#, palette = c("\#00AFBB", "\#E7B800"),}
   \AttributeTok{add\_density =} \ConstantTok{TRUE}\NormalTok{)}
\FunctionTok{facet}\NormalTok{(p, }\AttributeTok{facet.by=}\FunctionTok{c}\NormalTok{(}\StringTok{"pretty\_condition"}\NormalTok{,}\StringTok{"pretty\_mode"}\NormalTok{)) }\SpecialCharTok{+}
  \FunctionTok{labs}\NormalTok{( }\AttributeTok{title =} \StringTok{"Distribution of First Item Response Time (seconds)"}\NormalTok{,}
        \AttributeTok{subtitle =}\StringTok{""}\NormalTok{,}
        \AttributeTok{x =} \StringTok{"First Item Response Time (seconds)"}\NormalTok{, }\AttributeTok{y =} \StringTok{"number of items"}\NormalTok{) }\SpecialCharTok{+}
  \FunctionTok{theme\_minimal}\NormalTok{() }\SpecialCharTok{+} \FunctionTok{theme}\NormalTok{(}\AttributeTok{legend.position =} \StringTok{"blank"}\NormalTok{)}
\end{Highlighting}
\end{Shaded}

\begin{figure}[H]

{\centering \includegraphics{analysis/SGC3A/3_sgc3A_description_files/figure-pdf/label - VIS-FIRSTTIMEmessage - falsewarning - false-2.pdf}

}

\end{figure}

\begin{Shaded}
\begin{Highlighting}[]
\CommentTok{\#recode as boolean correct}
\NormalTok{df\_subjects }\OtherTok{\textless{}{-}}\NormalTok{ df\_subjects }\SpecialCharTok{\%\textgreater{}\%} \FunctionTok{mutate}\NormalTok{(}
  \AttributeTok{item\_q1\_NABS =} \FunctionTok{as.logical}\NormalTok{(item\_q1\_NABS)}
\NormalTok{)}

\DocumentationTok{\#\#RAINCLOUD USING GGDISTR}
\FunctionTok{ggplot}\NormalTok{(df\_subjects, }\FunctionTok{aes}\NormalTok{(}\AttributeTok{x =}\NormalTok{ pretty\_condition, }\AttributeTok{y =}\NormalTok{ item\_q1\_rt, }\AttributeTok{color =}\NormalTok{ item\_q1\_NABS) ) }\SpecialCharTok{+} 
\NormalTok{  ggdist}\SpecialCharTok{::}\FunctionTok{stat\_halfeye}\NormalTok{(}
    \AttributeTok{side =} \StringTok{"left"}\NormalTok{,}
    \AttributeTok{justification =} \FloatTok{1.2}\NormalTok{, }
    \AttributeTok{adjust =}\NormalTok{ .}\DecValTok{5}\NormalTok{, }
    \AttributeTok{width =}\NormalTok{ .}\DecValTok{6}\NormalTok{, }
    \AttributeTok{.width =} \DecValTok{0}\NormalTok{, }
    \AttributeTok{point\_colour =} \ConstantTok{NA}\NormalTok{) }\SpecialCharTok{+} 
  \FunctionTok{geom\_boxplot}\NormalTok{(}
    \AttributeTok{width =}\NormalTok{ .}\DecValTok{15}\NormalTok{, }
    \AttributeTok{outlier.shape =} \ConstantTok{NA}
\NormalTok{  ) }\SpecialCharTok{+}
  \FunctionTok{geom\_point}\NormalTok{(}
    \AttributeTok{size =} \FloatTok{1.3}\NormalTok{,}
    \AttributeTok{alpha =}\NormalTok{ .}\DecValTok{3}\NormalTok{,}
    \AttributeTok{position =} \FunctionTok{position\_jitter}\NormalTok{( }
      \AttributeTok{seed =} \DecValTok{1}\NormalTok{, }\AttributeTok{width =}\NormalTok{ .}\DecValTok{1}
\NormalTok{  )) }\SpecialCharTok{+} 
  \FunctionTok{labs}\NormalTok{( }\AttributeTok{title =} \StringTok{"Distribution of First Item Response Time (seconds)"}\NormalTok{,}
        \AttributeTok{subtitle =}\StringTok{""}\NormalTok{,}
        \AttributeTok{y =} \StringTok{"First Item Response Time (s)"}\NormalTok{, }\AttributeTok{x =} \StringTok{"Condition"}\NormalTok{) }\SpecialCharTok{+}
  \FunctionTok{theme\_ggdist}\NormalTok{() }
\end{Highlighting}
\end{Shaded}

\begin{figure}[H]

{\centering \includegraphics{analysis/SGC3A/3_sgc3A_description_files/figure-pdf/label - VIS-FIRSTTIMEmessage - falsewarning - false-3.pdf}

}

\end{figure}

\begin{Shaded}
\begin{Highlighting}[]
\CommentTok{\# + theme(legend.position = "blank")}
\CommentTok{\# + coord\_cartesian(xlim = c(1.2, NA), clip = "off")}
\end{Highlighting}
\end{Shaded}

\hypertarget{time-on-item}{%
\subsection{Time on Item}\label{time-on-item}}

Here we consider the time spent on an individual item (across all
items).

\begin{Shaded}
\begin{Highlighting}[]
\CommentTok{\#DESCRIBE distribution of response time}
\NormalTok{time.stats }\OtherTok{\textless{}{-}} \FunctionTok{rbind}\NormalTok{(}
  \StringTok{"lab"}\OtherTok{=}\NormalTok{ df\_items }\SpecialCharTok{\%\textgreater{}\%} \FunctionTok{filter}\NormalTok{(mode }\SpecialCharTok{==} \StringTok{"lab{-}synch"}\NormalTok{) }\SpecialCharTok{\%\textgreater{}\%}\NormalTok{ dplyr}\SpecialCharTok{::}\FunctionTok{select}\NormalTok{(rt\_s) }\SpecialCharTok{\%\textgreater{}\%} \FunctionTok{unlist}\NormalTok{()  }\SpecialCharTok{\%\textgreater{}\%}  \FunctionTok{favstats}\NormalTok{(),}
  \StringTok{"online"}\OtherTok{=}\NormalTok{ df\_items }\SpecialCharTok{\%\textgreater{}\%} \FunctionTok{filter}\NormalTok{(mode }\SpecialCharTok{==} \StringTok{"lab{-}synch"}\NormalTok{) }\SpecialCharTok{\%\textgreater{}\%}\NormalTok{ dplyr}\SpecialCharTok{::}\FunctionTok{select}\NormalTok{(rt\_s) }\SpecialCharTok{\%\textgreater{}\%} \FunctionTok{unlist}\NormalTok{() }\SpecialCharTok{\%\textgreater{}\%} \FunctionTok{favstats}\NormalTok{(),}
  \StringTok{"combined"}\OtherTok{=}\NormalTok{ df\_items }\SpecialCharTok{\%\textgreater{}\%}\NormalTok{   dplyr}\SpecialCharTok{::}\FunctionTok{select}\NormalTok{(rt\_s) }\SpecialCharTok{\%\textgreater{}\%} \FunctionTok{unlist}\NormalTok{() }\SpecialCharTok{\%\textgreater{}\%} \FunctionTok{favstats}\NormalTok{()}
\NormalTok{)}

\NormalTok{title }\OtherTok{=} \StringTok{"Descriptive Statistics of Item Response Latency (seconds)"}
\NormalTok{time.stats }\SpecialCharTok{\%\textgreater{}\%} \FunctionTok{kbl}\NormalTok{(}\AttributeTok{caption =}\NormalTok{ title) }\SpecialCharTok{\%\textgreater{}\%} \FunctionTok{kable\_classic}\NormalTok{()}
\end{Highlighting}
\end{Shaded}

\begin{table}

\caption{Descriptive Statistics of Item Response Latency (seconds)}
\centering
\begin{tabular}[t]{l|r|r|r|r|r|r|r|r|r}
\hline
  & min & Q1 & median & Q3 & max & mean & sd & n & missing\\
\hline
lab & 1.264 & 13.8 & 24.9 & 46.1 & 336 & 35.5 & 33.1 & 1890 & 0\\
\hline
online & 1.264 & 13.8 & 24.9 & 46.1 & 336 & 35.5 & 33.1 & 1890 & 0\\
\hline
combined & 0.003 & 12.5 & 23.7 & 43.9 & 532 & 35.2 & 37.2 & 4950 & 0\\
\hline
\end{tabular}
\end{table}

Time on an individual item for \emph{in person} subjects (n = 1890)
ranged from 1.26 to 336.03 minutes with a mean duration of (M = 35.47,
SD = 33.12).

Time on an individual item for \emph{online replication} subjects (n =
1890) ranged from 1.26 to 336.03 minutes with a mean duration of (M =
35.47, SD = 33.12).

Time on an individual item for \emph{combined} subjects (n = 4950)
ranged from 0 to 336.03 minutes with a mean duration of (M = 35.24, SD =
37.21).

\begin{Shaded}
\begin{Highlighting}[]
\CommentTok{\#HISTOGRAM}
\FunctionTok{gf\_dhistogram}\NormalTok{(}\SpecialCharTok{\textasciitilde{}}\NormalTok{rt\_s, }\AttributeTok{data =}\NormalTok{ df\_items) }\SpecialCharTok{\%\textgreater{}\%}
  \FunctionTok{gf\_vline}\NormalTok{(}\AttributeTok{xintercept =} \SpecialCharTok{\textasciitilde{}}\NormalTok{time.stats[}\StringTok{"lab"}\NormalTok{,]}\SpecialCharTok{$}\NormalTok{mean, }\AttributeTok{color =} \StringTok{"black"}\NormalTok{) }\SpecialCharTok{\%\textgreater{}\%} 
  \FunctionTok{gf\_fitdistr}\NormalTok{(}\AttributeTok{dist=}\StringTok{"gamma"}\NormalTok{, }\AttributeTok{color=}\StringTok{"red"}\NormalTok{)}\SpecialCharTok{+}
  \FunctionTok{labs}\NormalTok{(}\AttributeTok{title=}\StringTok{"Distribution of Item Response Time (seconds)"}\NormalTok{, }\AttributeTok{subtitle =} \StringTok{"fit by gamma distribution"}\NormalTok{, }\AttributeTok{x =} \StringTok{"Item Response Time (seconds)"}\NormalTok{, }\AttributeTok{y =} \StringTok{"\% items"}\NormalTok{) }\SpecialCharTok{+}  \FunctionTok{theme\_minimal}\NormalTok{()}
\end{Highlighting}
\end{Shaded}

\begin{figure}[H]

{\centering \includegraphics{analysis/SGC3A/3_sgc3A_description_files/figure-pdf/label - VIS-ITEMTIMEmessage - falsewarning - false-1.pdf}

}

\end{figure}

\begin{Shaded}
\begin{Highlighting}[]
\DocumentationTok{\#\#GGPUBR | HIST+DENSITY SCORE BY CONDITION/MODE}
\NormalTok{p }\OtherTok{\textless{}{-}} \FunctionTok{gghistogram}\NormalTok{(df\_items, }\AttributeTok{x =} \StringTok{"rt\_s"}\NormalTok{, }\AttributeTok{binwidth =} \FloatTok{0.5}\NormalTok{,}
   \AttributeTok{add =} \StringTok{"mean"}\NormalTok{, }\AttributeTok{rug =} \ConstantTok{TRUE}\NormalTok{,}
   \AttributeTok{fill =} \StringTok{"pretty\_condition"}\NormalTok{, }\CommentTok{\#, palette = c("\#00AFBB", "\#E7B800"),}
   \AttributeTok{add\_density =} \ConstantTok{TRUE}\NormalTok{)}
\FunctionTok{facet}\NormalTok{(p, }\AttributeTok{facet.by=}\FunctionTok{c}\NormalTok{(}\StringTok{"pretty\_condition"}\NormalTok{,}\StringTok{"pretty\_mode"}\NormalTok{)) }\SpecialCharTok{+}
  \FunctionTok{labs}\NormalTok{( }\AttributeTok{title =} \StringTok{"Distribution of Item Response Time (seconds)"}\NormalTok{,}
        \AttributeTok{subtitle =}\StringTok{""}\NormalTok{,}
        \AttributeTok{x =} \StringTok{"Item Response Time (seconds)"}\NormalTok{, }\AttributeTok{y =} \StringTok{"number of items"}\NormalTok{) }\SpecialCharTok{+}
  \FunctionTok{theme\_minimal}\NormalTok{() }\SpecialCharTok{+} \FunctionTok{theme}\NormalTok{(}\AttributeTok{legend.position =} \StringTok{"blank"}\NormalTok{)}
\end{Highlighting}
\end{Shaded}

\begin{figure}[H]

{\centering \includegraphics{analysis/SGC3A/3_sgc3A_description_files/figure-pdf/label - VIS-ITEMTIMEmessage - falsewarning - false-2.pdf}

}

\end{figure}

\begin{Shaded}
\begin{Highlighting}[]
\CommentTok{\#recode as boolean correct}
\NormalTok{df\_items }\OtherTok{\textless{}{-}}\NormalTok{ df\_items }\SpecialCharTok{\%\textgreater{}\%} \FunctionTok{mutate}\NormalTok{(}
  \AttributeTok{score\_niceABS =} \FunctionTok{as.logical}\NormalTok{(score\_niceABS)}
\NormalTok{)}

\DocumentationTok{\#\#RAINCLOUD USING GGDISTR}
\FunctionTok{ggplot}\NormalTok{(df\_items, }\FunctionTok{aes}\NormalTok{(}\AttributeTok{x =}\NormalTok{ pretty\_condition, }\AttributeTok{y =}\NormalTok{ rt\_s, }\AttributeTok{color =}\NormalTok{ score\_niceABS) ) }\SpecialCharTok{+} 
\NormalTok{  ggdist}\SpecialCharTok{::}\FunctionTok{stat\_halfeye}\NormalTok{(}
    \AttributeTok{side =} \StringTok{"left"}\NormalTok{,}
    \AttributeTok{justification =} \FloatTok{1.2}\NormalTok{, }
    \AttributeTok{adjust =}\NormalTok{ .}\DecValTok{5}\NormalTok{, }
    \AttributeTok{width =}\NormalTok{ .}\DecValTok{6}\NormalTok{, }
    \AttributeTok{.width =} \DecValTok{0}\NormalTok{, }
    \AttributeTok{point\_colour =} \ConstantTok{NA}\NormalTok{) }\SpecialCharTok{+} 
  \FunctionTok{geom\_boxplot}\NormalTok{(}
    \AttributeTok{width =}\NormalTok{ .}\DecValTok{15}\NormalTok{, }
    \AttributeTok{outlier.shape =} \ConstantTok{NA}
\NormalTok{  ) }\SpecialCharTok{+}
  \FunctionTok{geom\_point}\NormalTok{(}
    \AttributeTok{size =} \FloatTok{1.3}\NormalTok{,}
    \AttributeTok{alpha =}\NormalTok{ .}\DecValTok{3}\NormalTok{,}
    \AttributeTok{position =} \FunctionTok{position\_jitter}\NormalTok{( }
      \AttributeTok{seed =} \DecValTok{1}\NormalTok{, }\AttributeTok{width =}\NormalTok{ .}\DecValTok{1}
\NormalTok{  )) }\SpecialCharTok{+} 
  \FunctionTok{labs}\NormalTok{( }\AttributeTok{title =} \StringTok{"Distribution of Item Response Time (seconds)"}\NormalTok{,}
        \AttributeTok{subtitle =}\StringTok{""}\NormalTok{,}
        \AttributeTok{y =} \StringTok{"Item Response Time (s)"}\NormalTok{, }\AttributeTok{x =} \StringTok{"Condition"}\NormalTok{) }\SpecialCharTok{+}
  \FunctionTok{theme\_ggdist}\NormalTok{() }
\end{Highlighting}
\end{Shaded}

\begin{figure}[H]

{\centering \includegraphics{analysis/SGC3A/3_sgc3A_description_files/figure-pdf/label - VIS-ITEMTIMEmessage - falsewarning - false-3.pdf}

}

\end{figure}

\begin{Shaded}
\begin{Highlighting}[]
\CommentTok{\# + theme(legend.position = "blank")}
\CommentTok{\# + coord\_cartesian(xlim = c(1.2, NA), clip = "off")}
\end{Highlighting}
\end{Shaded}

\textbf{TODO consider log transform of response latency data} \emph{see}
archive sgc3A\_participants.Rmd

\hypertarget{time-on-scaffold-phase}{%
\subsection{Time on SCAFFOLD Phase}\label{time-on-scaffold-phase}}

Here we consider \emph{just} the time spent on the first five items of
the task (the scaffold phase).

\begin{Shaded}
\begin{Highlighting}[]
\CommentTok{\#DESCRIBE distribution of response time}
\NormalTok{time.stats }\OtherTok{\textless{}{-}} \FunctionTok{rbind}\NormalTok{(}
  \StringTok{"lab"}\OtherTok{=}\NormalTok{ df\_lab }\SpecialCharTok{\%\textgreater{}\%}\NormalTok{ dplyr}\SpecialCharTok{::}\FunctionTok{select}\NormalTok{(item\_scaffold\_rt) }\SpecialCharTok{\%\textgreater{}\%} \FunctionTok{unlist}\NormalTok{()  }\SpecialCharTok{\%\textgreater{}\%}  \FunctionTok{favstats}\NormalTok{(),}
  \StringTok{"online"}\OtherTok{=}\NormalTok{ df\_online }\SpecialCharTok{\%\textgreater{}\%}\NormalTok{ dplyr}\SpecialCharTok{::}\FunctionTok{select}\NormalTok{(item\_scaffold\_rt) }\SpecialCharTok{\%\textgreater{}\%} \FunctionTok{unlist}\NormalTok{() }\SpecialCharTok{\%\textgreater{}\%} \FunctionTok{favstats}\NormalTok{(),}
  \StringTok{"combined"}\OtherTok{=}\NormalTok{ df\_subjects }\SpecialCharTok{\%\textgreater{}\%}\NormalTok{ dplyr}\SpecialCharTok{::}\FunctionTok{select}\NormalTok{(item\_scaffold\_rt) }\SpecialCharTok{\%\textgreater{}\%} \FunctionTok{unlist}\NormalTok{() }\SpecialCharTok{\%\textgreater{}\%} \FunctionTok{favstats}\NormalTok{()}
\NormalTok{)}

\NormalTok{title }\OtherTok{=} \StringTok{"Descriptive Statistics of SCAFFOLD Phase Response Latency (minutes)"}
\NormalTok{time.stats }\SpecialCharTok{\%\textgreater{}\%} \FunctionTok{kbl}\NormalTok{(}\AttributeTok{caption =}\NormalTok{ title) }\SpecialCharTok{\%\textgreater{}\%} \FunctionTok{kable\_classic}\NormalTok{()}
\end{Highlighting}
\end{Shaded}

\begin{table}

\caption{Descriptive Statistics of SCAFFOLD Phase Response Latency (minutes)}
\centering
\begin{tabular}[t]{l|r|r|r|r|r|r|r|r|r}
\hline
  & min & Q1 & median & Q3 & max & mean & sd & n & missing\\
\hline
lab & 1.235 & 2.66 & 3.71 & 4.92 & 11.1 & 4.03 & 1.88 & 126 & 0\\
\hline
online & 0.614 & 2.10 & 2.92 & 4.31 & 15.4 & 3.52 & 2.26 & 204 & 0\\
\hline
combined & 0.614 & 2.29 & 3.25 & 4.58 & 15.4 & 3.72 & 2.13 & 330 & 0\\
\hline
\end{tabular}
\end{table}

Total time on SCAFFOLD phase for \emph{in person} subjects (n = 126)
ranged from 1.24 to 11.1 minutes with a mean duration of (M = 4.03, SD =
1.88).

Total time on SCAFFOLD phase for \emph{online replication} subjects (n =
204) ranged from 0.61 to 15.39 minutes with a mean duration of (M =
3.52, SD = 2.26).

Total time on SCAFFOLD phase for \emph{combined} subjects (n = 330)
ranged from 0.61 to 15.39 minutes with a mean duration of (M = 3.72, SD
= 2.13).

\begin{Shaded}
\begin{Highlighting}[]
\CommentTok{\#HISTOGRAM}
\FunctionTok{gf\_dhistogram}\NormalTok{(}\SpecialCharTok{\textasciitilde{}}\NormalTok{item\_scaffold\_rt, }\AttributeTok{data =}\NormalTok{ df\_subjects) }\SpecialCharTok{\%\textgreater{}\%}
  \FunctionTok{gf\_vline}\NormalTok{(}\AttributeTok{xintercept =} \SpecialCharTok{\textasciitilde{}}\NormalTok{time.stats[}\StringTok{"lab"}\NormalTok{,]}\SpecialCharTok{$}\NormalTok{mean, }\AttributeTok{color =} \StringTok{"black"}\NormalTok{) }\SpecialCharTok{\%\textgreater{}\%} 
  \FunctionTok{gf\_fitdistr}\NormalTok{(}\AttributeTok{dist=}\StringTok{"gamma"}\NormalTok{, }\AttributeTok{color=}\StringTok{"red"}\NormalTok{)}\SpecialCharTok{+}
  \FunctionTok{labs}\NormalTok{(}\AttributeTok{title=}\StringTok{"Distribution of SCAFFOLD Phase Response Time (minutes)"}\NormalTok{, }\AttributeTok{subtitle =} \StringTok{"fit by gamma distribution"}\NormalTok{, }\AttributeTok{x =} \StringTok{"Scaffold Phase Time (minutes)"}\NormalTok{, }\AttributeTok{y =} \StringTok{"\% subjects"}\NormalTok{) }\SpecialCharTok{+}  \FunctionTok{theme\_minimal}\NormalTok{()}
\end{Highlighting}
\end{Shaded}

\begin{figure}[H]

{\centering \includegraphics{analysis/SGC3A/3_sgc3A_description_files/figure-pdf/label - VIS-SCAFFOLDTIMEmessage - falsewarning - false-1.pdf}

}

\end{figure}

\begin{Shaded}
\begin{Highlighting}[]
\DocumentationTok{\#\#GGPUBR | HIST+DENSITY SCORE BY CONDITION/MODE}
\NormalTok{p }\OtherTok{\textless{}{-}} \FunctionTok{gghistogram}\NormalTok{(df\_subjects, }\AttributeTok{x =} \StringTok{"item\_scaffold\_rt"}\NormalTok{, }\AttributeTok{binwidth =} \FloatTok{0.5}\NormalTok{,}
   \AttributeTok{add =} \StringTok{"mean"}\NormalTok{, }\AttributeTok{rug =} \ConstantTok{TRUE}\NormalTok{,}
   \AttributeTok{fill =} \StringTok{"pretty\_condition"}\NormalTok{, }\CommentTok{\#, palette = c("\#00AFBB", "\#E7B800"),}
   \AttributeTok{add\_density =} \ConstantTok{TRUE}\NormalTok{)}
\FunctionTok{facet}\NormalTok{(p, }\AttributeTok{facet.by=}\FunctionTok{c}\NormalTok{(}\StringTok{"pretty\_condition"}\NormalTok{,}\StringTok{"pretty\_mode"}\NormalTok{)) }\SpecialCharTok{+}
  \FunctionTok{labs}\NormalTok{( }\AttributeTok{title =} \StringTok{"Distribution of SCAFFOLD Phase Response Time (minutes)"}\NormalTok{,}
        \AttributeTok{subtitle =}\StringTok{""}\NormalTok{,}
        \AttributeTok{x =} \StringTok{"Scaffold Phase Time (minutes)"}\NormalTok{, }\AttributeTok{y =} \StringTok{"number of subjects"}\NormalTok{) }\SpecialCharTok{+}
  \FunctionTok{theme\_minimal}\NormalTok{() }\SpecialCharTok{+} \FunctionTok{theme}\NormalTok{(}\AttributeTok{legend.position =} \StringTok{"blank"}\NormalTok{)}
\end{Highlighting}
\end{Shaded}

\begin{figure}[H]

{\centering \includegraphics{analysis/SGC3A/3_sgc3A_description_files/figure-pdf/label - VIS-SCAFFOLDTIMEmessage - falsewarning - false-2.pdf}

}

\end{figure}

\begin{Shaded}
\begin{Highlighting}[]
\DocumentationTok{\#\#RAINCLOUD USING GGDISTR}
\FunctionTok{ggplot}\NormalTok{(df\_subjects, }\FunctionTok{aes}\NormalTok{(}\AttributeTok{x =}\NormalTok{ pretty\_condition, }\AttributeTok{y =}\NormalTok{ item\_scaffold\_rt, }\AttributeTok{fill =}\NormalTok{ pretty\_condition)) }\SpecialCharTok{+} 
\NormalTok{  ggdist}\SpecialCharTok{::}\FunctionTok{stat\_halfeye}\NormalTok{(}
    \AttributeTok{side =} \StringTok{"left"}\NormalTok{,}
    \AttributeTok{justification =} \FloatTok{1.2}\NormalTok{, }
    \AttributeTok{adjust =}\NormalTok{ .}\DecValTok{5}\NormalTok{, }
    \AttributeTok{width =}\NormalTok{ .}\DecValTok{6}\NormalTok{, }
    \AttributeTok{.width =} \DecValTok{0}\NormalTok{, }
    \AttributeTok{point\_colour =} \ConstantTok{NA}\NormalTok{) }\SpecialCharTok{+} 
  \FunctionTok{geom\_boxplot}\NormalTok{(}
    \AttributeTok{width =}\NormalTok{ .}\DecValTok{15}\NormalTok{, }
    \AttributeTok{outlier.shape =} \ConstantTok{NA}
\NormalTok{  ) }\SpecialCharTok{+}
  \FunctionTok{geom\_point}\NormalTok{(}
    \AttributeTok{size =} \FloatTok{1.3}\NormalTok{,}
    \AttributeTok{alpha =}\NormalTok{ .}\DecValTok{3}\NormalTok{,}
    \AttributeTok{position =} \FunctionTok{position\_jitter}\NormalTok{(}
      \AttributeTok{seed =} \DecValTok{1}\NormalTok{, }\AttributeTok{width =}\NormalTok{ .}\DecValTok{1}
\NormalTok{  )) }\SpecialCharTok{+} 
  \FunctionTok{labs}\NormalTok{( }\AttributeTok{title =} \StringTok{"Distribution of SCAFFOLD Phase Response Time (minutes)"}\NormalTok{,}
        \AttributeTok{subtitle =}\StringTok{""}\NormalTok{,}
        \AttributeTok{y =} \StringTok{"Total Study Time (minutes)"}\NormalTok{, }\AttributeTok{x =} \StringTok{"Condition"}\NormalTok{) }\SpecialCharTok{+}
  \FunctionTok{theme\_ggdist}\NormalTok{() }\SpecialCharTok{+} \FunctionTok{theme}\NormalTok{(}\AttributeTok{legend.position =} \StringTok{"blank"}\NormalTok{)}
\end{Highlighting}
\end{Shaded}

\begin{figure}[H]

{\centering \includegraphics{analysis/SGC3A/3_sgc3A_description_files/figure-pdf/label - VIS-SCAFFOLDTIMEmessage - falsewarning - false-3.pdf}

}

\end{figure}

\begin{Shaded}
\begin{Highlighting}[]
\CommentTok{\# + coord\_cartesian(xlim = c(1.2, NA), clip = "off")}
\end{Highlighting}
\end{Shaded}

\textbf{TODO consider log transform of response latency data} \emph{see}
archive sgc3A\_participants.Rmd

\hypertarget{time-on-test-phase}{%
\subsection{Time on TEST Phase}\label{time-on-test-phase}}

Here we consider \emph{just} the time spent on the remaining eight
discriminant items of the task (the test phase).

\begin{Shaded}
\begin{Highlighting}[]
\CommentTok{\#DESCRIBE distribution of response time}
\NormalTok{time.stats }\OtherTok{\textless{}{-}} \FunctionTok{rbind}\NormalTok{(}
  \StringTok{"lab"}\OtherTok{=}\NormalTok{ df\_lab }\SpecialCharTok{\%\textgreater{}\%}\NormalTok{ dplyr}\SpecialCharTok{::}\FunctionTok{select}\NormalTok{(item\_test\_rt) }\SpecialCharTok{\%\textgreater{}\%} \FunctionTok{unlist}\NormalTok{()  }\SpecialCharTok{\%\textgreater{}\%}  \FunctionTok{favstats}\NormalTok{(),}
  \StringTok{"online"}\OtherTok{=}\NormalTok{ df\_online }\SpecialCharTok{\%\textgreater{}\%}\NormalTok{ dplyr}\SpecialCharTok{::}\FunctionTok{select}\NormalTok{(item\_test\_rt) }\SpecialCharTok{\%\textgreater{}\%} \FunctionTok{unlist}\NormalTok{() }\SpecialCharTok{\%\textgreater{}\%} \FunctionTok{favstats}\NormalTok{(),}
  \StringTok{"combined"}\OtherTok{=}\NormalTok{ df\_subjects }\SpecialCharTok{\%\textgreater{}\%}\NormalTok{ dplyr}\SpecialCharTok{::}\FunctionTok{select}\NormalTok{(item\_test\_rt) }\SpecialCharTok{\%\textgreater{}\%} \FunctionTok{unlist}\NormalTok{() }\SpecialCharTok{\%\textgreater{}\%} \FunctionTok{favstats}\NormalTok{()}
\NormalTok{)}

\NormalTok{title }\OtherTok{=} \StringTok{"Descriptive Statistics of TEST Phase Response Latency (minutes)"}
\NormalTok{time.stats }\SpecialCharTok{\%\textgreater{}\%} \FunctionTok{kbl}\NormalTok{(}\AttributeTok{caption =}\NormalTok{ title) }\SpecialCharTok{\%\textgreater{}\%} \FunctionTok{kable\_classic}\NormalTok{()}
\end{Highlighting}
\end{Shaded}

\begin{table}

\caption{Descriptive Statistics of TEST Phase Response Latency (minutes)}
\centering
\begin{tabular}[t]{l|r|r|r|r|r|r|r|r|r}
\hline
  & min & Q1 & median & Q3 & max & mean & sd & n & missing\\
\hline
lab & 1.022 & 2.97 & 3.75 & 4.76 & 10.8 & 4.00 & 1.37 & 126 & 0\\
\hline
online & 0.703 & 3.10 & 3.89 & 5.17 & 13.5 & 4.41 & 2.24 & 204 & 0\\
\hline
combined & 0.703 & 3.03 & 3.80 & 4.99 & 13.5 & 4.26 & 1.96 & 330 & 0\\
\hline
\end{tabular}
\end{table}

Total time on TEST phase for \emph{in person} subjects (n = 126) ranged
from 1.02 to 10.85 minutes with a mean duration of (M = 4, SD = 1.37).

Total time on TEST phase for \emph{online replication} subjects (n =
204) ranged from 0.7 to 13.49 minutes with a mean duration of (M = 4.41,
SD = 2.24).

Total time on TEST phase for \emph{combined} subjects (n = 330) ranged
from 0.7 to 13.49 minutes with a mean duration of (M = 4.26, SD = 1.96).

\begin{Shaded}
\begin{Highlighting}[]
\CommentTok{\#HISTOGRAM}
\FunctionTok{gf\_dhistogram}\NormalTok{(}\SpecialCharTok{\textasciitilde{}}\NormalTok{item\_test\_rt, }\AttributeTok{data =}\NormalTok{ df\_subjects) }\SpecialCharTok{\%\textgreater{}\%}
  \FunctionTok{gf\_vline}\NormalTok{(}\AttributeTok{xintercept =} \SpecialCharTok{\textasciitilde{}}\NormalTok{time.stats[}\StringTok{"lab"}\NormalTok{,]}\SpecialCharTok{$}\NormalTok{mean, }\AttributeTok{color =} \StringTok{"black"}\NormalTok{) }\SpecialCharTok{\%\textgreater{}\%} 
  \FunctionTok{gf\_fitdistr}\NormalTok{(}\AttributeTok{dist=}\StringTok{"gamma"}\NormalTok{, }\AttributeTok{color=}\StringTok{"red"}\NormalTok{)}\SpecialCharTok{+}
  \FunctionTok{labs}\NormalTok{(}\AttributeTok{title=}\StringTok{"Distribution of TEST Phase Response Time (minutes)"}\NormalTok{, }\AttributeTok{subtitle =} \StringTok{"fit by gamma distribution"}\NormalTok{, }\AttributeTok{x =} \StringTok{"Test Phase Time (minutes)"}\NormalTok{, }\AttributeTok{y =} \StringTok{"\% subjects"}\NormalTok{) }\SpecialCharTok{+}  \FunctionTok{theme\_minimal}\NormalTok{()}
\end{Highlighting}
\end{Shaded}

\begin{figure}[H]

{\centering \includegraphics{analysis/SGC3A/3_sgc3A_description_files/figure-pdf/label - VIS-TESTTIMEmessage - falsewarning - false-1.pdf}

}

\end{figure}

\begin{Shaded}
\begin{Highlighting}[]
\DocumentationTok{\#\#GGPUBR | HIST+DENSITY SCORE BY CONDITION/MODE}
\NormalTok{p }\OtherTok{\textless{}{-}} \FunctionTok{gghistogram}\NormalTok{(df\_subjects, }\AttributeTok{x =} \StringTok{"item\_test\_rt"}\NormalTok{, }\AttributeTok{binwidth =} \FloatTok{0.5}\NormalTok{,}
   \AttributeTok{add =} \StringTok{"mean"}\NormalTok{, }\AttributeTok{rug =} \ConstantTok{TRUE}\NormalTok{,}
   \AttributeTok{fill =} \StringTok{"pretty\_condition"}\NormalTok{, }\CommentTok{\#, palette = c("\#00AFBB", "\#E7B800"),}
   \AttributeTok{add\_density =} \ConstantTok{TRUE}\NormalTok{)}
\FunctionTok{facet}\NormalTok{(p, }\AttributeTok{facet.by=}\FunctionTok{c}\NormalTok{(}\StringTok{"pretty\_condition"}\NormalTok{,}\StringTok{"pretty\_mode"}\NormalTok{)) }\SpecialCharTok{+}
  \FunctionTok{labs}\NormalTok{( }\AttributeTok{title =} \StringTok{"Distribution of TEST Phase Response Time (minutes)"}\NormalTok{,}
        \AttributeTok{subtitle =}\StringTok{""}\NormalTok{,}
        \AttributeTok{x =} \StringTok{"Test Phase Time (minutes)"}\NormalTok{, }\AttributeTok{y =} \StringTok{"number of subjects"}\NormalTok{) }\SpecialCharTok{+}
  \FunctionTok{theme\_minimal}\NormalTok{() }\SpecialCharTok{+} \FunctionTok{theme}\NormalTok{(}\AttributeTok{legend.position =} \StringTok{"blank"}\NormalTok{)}
\end{Highlighting}
\end{Shaded}

\begin{figure}[H]

{\centering \includegraphics{analysis/SGC3A/3_sgc3A_description_files/figure-pdf/label - VIS-TESTTIMEmessage - falsewarning - false-2.pdf}

}

\end{figure}

\begin{Shaded}
\begin{Highlighting}[]
\DocumentationTok{\#\#RAINCLOUD USING GGDISTR}
\FunctionTok{ggplot}\NormalTok{(df\_subjects, }\FunctionTok{aes}\NormalTok{(}\AttributeTok{x =}\NormalTok{ pretty\_condition, }\AttributeTok{y =}\NormalTok{ item\_test\_rt, }\AttributeTok{fill =}\NormalTok{ pretty\_condition)) }\SpecialCharTok{+} 
\NormalTok{  ggdist}\SpecialCharTok{::}\FunctionTok{stat\_halfeye}\NormalTok{(}
    \AttributeTok{side =} \StringTok{"left"}\NormalTok{,}
    \AttributeTok{justification =} \FloatTok{1.2}\NormalTok{, }
    \AttributeTok{adjust =}\NormalTok{ .}\DecValTok{5}\NormalTok{, }
    \AttributeTok{width =}\NormalTok{ .}\DecValTok{6}\NormalTok{, }
    \AttributeTok{.width =} \DecValTok{0}\NormalTok{, }
    \AttributeTok{point\_colour =} \ConstantTok{NA}\NormalTok{) }\SpecialCharTok{+} 
  \FunctionTok{geom\_boxplot}\NormalTok{(}
    \AttributeTok{width =}\NormalTok{ .}\DecValTok{15}\NormalTok{, }
    \AttributeTok{outlier.shape =} \ConstantTok{NA}
\NormalTok{  ) }\SpecialCharTok{+}
  \FunctionTok{geom\_point}\NormalTok{(}
    \AttributeTok{size =} \FloatTok{1.3}\NormalTok{,}
    \AttributeTok{alpha =}\NormalTok{ .}\DecValTok{3}\NormalTok{,}
    \AttributeTok{position =} \FunctionTok{position\_jitter}\NormalTok{(}
      \AttributeTok{seed =} \DecValTok{1}\NormalTok{, }\AttributeTok{width =}\NormalTok{ .}\DecValTok{1}
\NormalTok{  )) }\SpecialCharTok{+} 
  \FunctionTok{labs}\NormalTok{( }\AttributeTok{title =} \StringTok{"Distribution of TEST Phase Response Time (minutes)"}\NormalTok{,}
        \AttributeTok{subtitle =}\StringTok{""}\NormalTok{,}
        \AttributeTok{y =} \StringTok{"Test Phase Time (minutes)"}\NormalTok{, }\AttributeTok{x =} \StringTok{"Condition"}\NormalTok{) }\SpecialCharTok{+}
  \FunctionTok{theme\_ggdist}\NormalTok{() }\SpecialCharTok{+} \FunctionTok{theme}\NormalTok{(}\AttributeTok{legend.position =} \StringTok{"blank"}\NormalTok{)}
\end{Highlighting}
\end{Shaded}

\begin{figure}[H]

{\centering \includegraphics{analysis/SGC3A/3_sgc3A_description_files/figure-pdf/label - VIS-TESTTIMEmessage - falsewarning - false-3.pdf}

}

\end{figure}

\begin{Shaded}
\begin{Highlighting}[]
\CommentTok{\# + coord\_cartesian(xlim = c(1.2, NA), clip = "off")}
\end{Highlighting}
\end{Shaded}

\textbf{TODO consider log transform of response latency data} \emph{see}
archive sgc3A\_participants.Rmd

\hypertarget{time-on-study}{%
\subsection{Time on Study}\label{time-on-study}}

\begin{Shaded}
\begin{Highlighting}[]
\CommentTok{\#DESCRIBE distribution of response time}
\NormalTok{time.stats }\OtherTok{\textless{}{-}} \FunctionTok{rbind}\NormalTok{(}
  \StringTok{"lab"}\OtherTok{=}\NormalTok{ df\_lab }\SpecialCharTok{\%\textgreater{}\%}\NormalTok{ dplyr}\SpecialCharTok{::}\FunctionTok{select}\NormalTok{(totaltime\_m) }\SpecialCharTok{\%\textgreater{}\%} \FunctionTok{unlist}\NormalTok{() }\SpecialCharTok{\%\textgreater{}\%} \FunctionTok{favstats}\NormalTok{(),}
  \StringTok{"online"}\OtherTok{=}\NormalTok{ df\_online }\SpecialCharTok{\%\textgreater{}\%}\NormalTok{ dplyr}\SpecialCharTok{::}\FunctionTok{select}\NormalTok{(totaltime\_m) }\SpecialCharTok{\%\textgreater{}\%} \FunctionTok{unlist}\NormalTok{() }\SpecialCharTok{\%\textgreater{}\%} \FunctionTok{favstats}\NormalTok{(),}
  \StringTok{"combined"}\OtherTok{=}\NormalTok{ df\_subjects }\SpecialCharTok{\%\textgreater{}\%}\NormalTok{ dplyr}\SpecialCharTok{::}\FunctionTok{select}\NormalTok{(totaltime\_m) }\SpecialCharTok{\%\textgreater{}\%} \FunctionTok{unlist}\NormalTok{() }\SpecialCharTok{\%\textgreater{}\%} \FunctionTok{favstats}\NormalTok{()}
\NormalTok{)}

\NormalTok{title }\OtherTok{=} \StringTok{"Descriptive Statistics of Response Latency (Total Minutes on Study)"}
\NormalTok{time.stats }\SpecialCharTok{\%\textgreater{}\%} \FunctionTok{kbl}\NormalTok{(}\AttributeTok{caption =}\NormalTok{ title) }\SpecialCharTok{\%\textgreater{}\%} \FunctionTok{kable\_classic}\NormalTok{()}
\end{Highlighting}
\end{Shaded}

\begin{table}

\caption{Descriptive Statistics of Response Latency (Total Minutes on Study)}
\centering
\begin{tabular}[t]{l|r|r|r|r|r|r|r|r|r}
\hline
  & min & Q1 & median & Q3 & max & mean & sd & n & missing\\
\hline
lab & 6.01 & 10.50 & 12.2 & 14.4 & 23.9 & 12.8 & 3.37 & 126 & 0\\
\hline
online & 2.91 & 9.18 & 11.5 & 15.0 & 111.0 & 13.4 & 9.21 & 204 & 0\\
\hline
combined & 2.91 & 9.55 & 12.0 & 14.7 & 111.0 & 13.2 & 7.53 & 330 & 0\\
\hline
\end{tabular}
\end{table}

Total time on study for \emph{in person} subjects (n = 126) ranged from
6.01 to 23.86 minutes with a mean duration of (M = 12.8, SD = 3.37).

Total time on study for \emph{online replication} subjects (n = 204)
ranged from 2.91 to 111.02 minutes with a mean duration of (M = 13.37,
SD = 9.21).

Total time on study for \emph{combined} subjects (n = 330) ranged from
2.91 to 111.02 minutes with a mean duration of (M = 13.16, SD = 7.53).

\begin{Shaded}
\begin{Highlighting}[]
\CommentTok{\#HISTOGRAM}
\FunctionTok{gf\_dhistogram}\NormalTok{(}\SpecialCharTok{\textasciitilde{}}\NormalTok{totaltime\_m, }\AttributeTok{data =}\NormalTok{ df\_subjects) }\SpecialCharTok{\%\textgreater{}\%}
  \FunctionTok{gf\_vline}\NormalTok{(}\AttributeTok{xintercept =} \SpecialCharTok{\textasciitilde{}}\NormalTok{time.stats[}\StringTok{"lab"}\NormalTok{,]}\SpecialCharTok{$}\NormalTok{mean, }\AttributeTok{color =} \StringTok{"black"}\NormalTok{) }\SpecialCharTok{\%\textgreater{}\%} 
  \FunctionTok{gf\_fitdistr}\NormalTok{(}\AttributeTok{dist=}\StringTok{"gamma"}\NormalTok{, }\AttributeTok{color=}\StringTok{"red"}\NormalTok{)}\SpecialCharTok{+}
  \FunctionTok{labs}\NormalTok{(}\AttributeTok{title=}\StringTok{"Distribution of Total Time on Study (in minutes)"}\NormalTok{, }\AttributeTok{subtitle =} \StringTok{"fit by gamma distribution"}\NormalTok{, }\AttributeTok{x =} \StringTok{"Total Time (mins)"}\NormalTok{, }\AttributeTok{y =} \StringTok{"\% subjects"}\NormalTok{) }\SpecialCharTok{+}  \FunctionTok{theme\_minimal}\NormalTok{()}
\end{Highlighting}
\end{Shaded}

\begin{figure}[H]

{\centering \includegraphics{analysis/SGC3A/3_sgc3A_description_files/figure-pdf/label - VIS-TOTALTIMEmessage - falsewarning - false-1.pdf}

}

\end{figure}

\begin{Shaded}
\begin{Highlighting}[]
\CommentTok{\#HISTOGRAM by CONDITION and MODE}
\FunctionTok{gf\_dhistogram}\NormalTok{(}\SpecialCharTok{\textasciitilde{}}\NormalTok{totaltime\_m, }\AttributeTok{data =}\NormalTok{ df\_subjects) }\SpecialCharTok{\%\textgreater{}\%}
  \FunctionTok{gf\_facet\_grid}\NormalTok{(pretty\_condition }\SpecialCharTok{\textasciitilde{}}\NormalTok{ pretty\_mode) }\SpecialCharTok{+} 
  \FunctionTok{labs}\NormalTok{(}\AttributeTok{title=}\StringTok{"Distribution of Total Time on Study ( minutes)"}\NormalTok{, }\AttributeTok{x =} \StringTok{"Total Time (mins)"}\NormalTok{, }\AttributeTok{y =} \StringTok{"\% subjects"}\NormalTok{) }\SpecialCharTok{+}  \FunctionTok{theme\_minimal}\NormalTok{()}
\end{Highlighting}
\end{Shaded}

\begin{figure}[H]

{\centering \includegraphics{analysis/SGC3A/3_sgc3A_description_files/figure-pdf/label - VIS-TOTALTIMEmessage - falsewarning - false-2.pdf}

}

\end{figure}

\begin{Shaded}
\begin{Highlighting}[]
\DocumentationTok{\#\#GGPUBR | HIST+DENSITY SCORE BY CONDITION/MODE}
\NormalTok{p }\OtherTok{\textless{}{-}} \FunctionTok{gghistogram}\NormalTok{(df\_subjects, }\AttributeTok{x =} \StringTok{"totaltime\_m"}\NormalTok{, }\AttributeTok{binwidth =} \FloatTok{0.5}\NormalTok{,}
   \AttributeTok{add =} \StringTok{"mean"}\NormalTok{, }\AttributeTok{rug =} \ConstantTok{TRUE}\NormalTok{,}
   \AttributeTok{fill =} \StringTok{"pretty\_condition"}\NormalTok{, }\CommentTok{\#, palette = c("\#00AFBB", "\#E7B800"),}
   \AttributeTok{add\_density =} \ConstantTok{TRUE}\NormalTok{)}
\FunctionTok{facet}\NormalTok{(p, }\AttributeTok{facet.by=}\FunctionTok{c}\NormalTok{(}\StringTok{"pretty\_condition"}\NormalTok{,}\StringTok{"pretty\_mode"}\NormalTok{)) }\SpecialCharTok{+}
  \FunctionTok{labs}\NormalTok{( }\AttributeTok{title =} \StringTok{"Distribution of Total Time on Study (minutes)"}\NormalTok{,}
        \AttributeTok{subtitle =}\StringTok{""}\NormalTok{,}
        \AttributeTok{x =} \StringTok{"Total Study Time (minutes)"}\NormalTok{, }\AttributeTok{y =} \StringTok{"number of subjects"}\NormalTok{) }\SpecialCharTok{+}
  \FunctionTok{theme\_minimal}\NormalTok{() }\SpecialCharTok{+} \FunctionTok{theme}\NormalTok{(}\AttributeTok{legend.position =} \StringTok{"blank"}\NormalTok{)}
\end{Highlighting}
\end{Shaded}

\begin{figure}[H]

{\centering \includegraphics{analysis/SGC3A/3_sgc3A_description_files/figure-pdf/label - VIS-TOTALTIMEmessage - falsewarning - false-3.pdf}

}

\end{figure}

\begin{Shaded}
\begin{Highlighting}[]
\CommentTok{\# GGPUBR ARRANGE EXAMPLE WITH CUSTOM DIST FITTING}
\CommentTok{\# plab \textless{}{-} gf\_dhistogram(\textasciitilde{}totaltime\_m, data = df\_subjects) \%\textgreater{}\%}
\CommentTok{\#   gf\_vline(xintercept = \textasciitilde{}time.stats["lab",]$mean, color = "black") \%\textgreater{}\% }
\CommentTok{\#   gf\_facet\_grid(pretty\_condition \textasciitilde{} .) + }
\CommentTok{\#   labs(title="Lab", x = "Total Time (mins)", y = "\% subjects") +  theme\_minimal()}
\CommentTok{\# }
\CommentTok{\# ponline \textless{}{-} gf\_dhistogram(\textasciitilde{}totaltime\_m, data = df\_subjects) \%\textgreater{}\%}
\CommentTok{\#   gf\_vline(xintercept = \textasciitilde{}time.stats["online",]$mean, color = "black") \%\textgreater{}\% }
\CommentTok{\#   gf\_facet\_grid(pretty\_condition \textasciitilde{} .) + }
\CommentTok{\#   labs(title="Online", x = "Total Time (mins)", y = "\% subjects") +  theme\_minimal()}
\CommentTok{\# }
\CommentTok{\# plot \textless{}{-}ggarrange(plab, ponline, common.legend = TRUE, nrow = 1, ncol =2)}
\CommentTok{\# }
\CommentTok{\# annotate\_figure(plot, }
\CommentTok{\#                 top = text\_grob("Total Time by Study Mode",color = "black", face = "bold", size = 14),}
\CommentTok{\#                 bottom = text\_grob("fit by Gamma distribution", face = "italic", size = 10))}


\DocumentationTok{\#\#RAINCLOUD USING GGDISTR}
\FunctionTok{ggplot}\NormalTok{(df\_subjects, }\FunctionTok{aes}\NormalTok{(}\AttributeTok{x =}\NormalTok{ pretty\_condition, }\AttributeTok{y =}\NormalTok{ totaltime\_m, }\AttributeTok{fill =}\NormalTok{ pretty\_condition)) }\SpecialCharTok{+} 
\NormalTok{  ggdist}\SpecialCharTok{::}\FunctionTok{stat\_halfeye}\NormalTok{(}
    \AttributeTok{side =} \StringTok{"left"}\NormalTok{,}
    \AttributeTok{justification =} \FloatTok{1.2}\NormalTok{, }
    \AttributeTok{adjust =}\NormalTok{ .}\DecValTok{5}\NormalTok{, }
    \AttributeTok{width =}\NormalTok{ .}\DecValTok{6}\NormalTok{, }
    \AttributeTok{.width =} \DecValTok{0}\NormalTok{, }
    \AttributeTok{point\_colour =} \ConstantTok{NA}\NormalTok{) }\SpecialCharTok{+} 
  \FunctionTok{geom\_boxplot}\NormalTok{(}
    \AttributeTok{width =}\NormalTok{ .}\DecValTok{15}\NormalTok{, }
    \AttributeTok{outlier.shape =} \ConstantTok{NA}
\NormalTok{  ) }\SpecialCharTok{+}
  \FunctionTok{geom\_point}\NormalTok{(}
    \AttributeTok{size =} \FloatTok{1.3}\NormalTok{,}
    \AttributeTok{alpha =}\NormalTok{ .}\DecValTok{3}\NormalTok{,}
    \AttributeTok{position =} \FunctionTok{position\_jitter}\NormalTok{(}
      \AttributeTok{seed =} \DecValTok{1}\NormalTok{, }\AttributeTok{width =}\NormalTok{ .}\DecValTok{1}
\NormalTok{  )) }\SpecialCharTok{+} 
  \FunctionTok{labs}\NormalTok{( }\AttributeTok{title =} \StringTok{"Distribution of Total Time on Study (minutes)"}\NormalTok{,}
        \AttributeTok{subtitle =}\StringTok{""}\NormalTok{,}
        \AttributeTok{y =} \StringTok{"Total Study Time (minutes)"}\NormalTok{, }\AttributeTok{x =} \StringTok{"Condition"}\NormalTok{) }\SpecialCharTok{+}
  \FunctionTok{theme\_ggdist}\NormalTok{() }\SpecialCharTok{+} \FunctionTok{theme}\NormalTok{(}\AttributeTok{legend.position =} \StringTok{"blank"}\NormalTok{)}
\end{Highlighting}
\end{Shaded}

\begin{figure}[H]

{\centering \includegraphics{analysis/SGC3A/3_sgc3A_description_files/figure-pdf/label - VIS-TOTALTIMEmessage - falsewarning - false-4.pdf}

}

\end{figure}

\begin{Shaded}
\begin{Highlighting}[]
\CommentTok{\# + coord\_cartesian(xlim = c(1.2, NA), clip = "off")}
\end{Highlighting}
\end{Shaded}

\textbf{TODO consider log transform of response latency data} \emph{see}
archive sgc3A\_participants.Rmd

\hypertarget{wip-exploring}{%
\chapter{WIP EXPLORING}\label{wip-exploring}}

\textbf{TODO} Cumulative speed?

\begin{Shaded}
\begin{Highlighting}[]
\FunctionTok{gf\_line}\NormalTok{(rt\_s }\SpecialCharTok{\textasciitilde{}}\NormalTok{ q, }\AttributeTok{group =} \SpecialCharTok{\textasciitilde{}}\NormalTok{subject, }\AttributeTok{data =}\NormalTok{ df\_items }\SpecialCharTok{\%\textgreater{}\%} \FunctionTok{filter}\NormalTok{(q }\SpecialCharTok{\%nin\%} \FunctionTok{c}\NormalTok{(}\DecValTok{6}\NormalTok{,}\DecValTok{9}\NormalTok{)) }\SpecialCharTok{\%\textgreater{}\%} \FunctionTok{filter}\NormalTok{(mode}\SpecialCharTok{==}\StringTok{"asynch"}\NormalTok{), }\AttributeTok{color =} \SpecialCharTok{\textasciitilde{}}\NormalTok{score\_niceABS, }\AttributeTok{alpha =} \FloatTok{0.2}\NormalTok{) }\SpecialCharTok{\%\textgreater{}\%} 
  \FunctionTok{gf\_facet\_grid}\NormalTok{( pretty\_condition }\SpecialCharTok{\textasciitilde{}}\NormalTok{ score\_niceABS) }\SpecialCharTok{+}
  \FunctionTok{labs}\NormalTok{(}\AttributeTok{title =} \StringTok{" ONLINE response time BY item and accuracy"}\NormalTok{)}
\end{Highlighting}
\end{Shaded}

\begin{figure}[H]

{\centering \includegraphics{analysis/SGC3A/3_sgc3A_description_files/figure-pdf/unnamed-chunk-23-1.pdf}

}

\end{figure}

\begin{Shaded}
\begin{Highlighting}[]
\FunctionTok{gf\_boxplot}\NormalTok{( item\_avg\_rt }\SpecialCharTok{\textasciitilde{}}\NormalTok{ pretty\_condition,  }\AttributeTok{data =}\NormalTok{ df\_subjects) }\SpecialCharTok{\%\textgreater{}\%} 
  \FunctionTok{gf\_facet\_wrap}\NormalTok{(}\SpecialCharTok{\textasciitilde{}}\NormalTok{pretty\_mode) }\SpecialCharTok{+}
  \FunctionTok{labs}\NormalTok{(}\AttributeTok{title =} \StringTok{"Average item response time by mode and condition"}\NormalTok{)}
\end{Highlighting}
\end{Shaded}

\begin{figure}[H]

{\centering \includegraphics{analysis/SGC3A/3_sgc3A_description_files/figure-pdf/unnamed-chunk-23-2.pdf}

}

\end{figure}

\begin{Shaded}
\begin{Highlighting}[]
\FunctionTok{gf\_boxplot}\NormalTok{( item\_scaffold\_rt }\SpecialCharTok{\textasciitilde{}}\NormalTok{ pretty\_condition,  }\AttributeTok{data =}\NormalTok{ df\_subjects) }\SpecialCharTok{\%\textgreater{}\%} 
  \FunctionTok{gf\_facet\_wrap}\NormalTok{(}\SpecialCharTok{\textasciitilde{}}\NormalTok{pretty\_mode) }\SpecialCharTok{+}
  \FunctionTok{labs}\NormalTok{(}\AttributeTok{title =} \StringTok{"Average SCAFFOLD response time by mode and condition"}\NormalTok{)}
\end{Highlighting}
\end{Shaded}

\begin{figure}[H]

{\centering \includegraphics{analysis/SGC3A/3_sgc3A_description_files/figure-pdf/unnamed-chunk-23-3.pdf}

}

\end{figure}

\begin{Shaded}
\begin{Highlighting}[]
\FunctionTok{gf\_boxplot}\NormalTok{( item\_test\_rt }\SpecialCharTok{\textasciitilde{}}\NormalTok{ pretty\_condition,  }\AttributeTok{data =}\NormalTok{ df\_subjects) }\SpecialCharTok{\%\textgreater{}\%} 
  \FunctionTok{gf\_facet\_wrap}\NormalTok{(}\SpecialCharTok{\textasciitilde{}}\NormalTok{pretty\_mode) }\SpecialCharTok{+}
  \FunctionTok{labs}\NormalTok{(}\AttributeTok{title =} \StringTok{"Average TEST response time by mode and condition"}\NormalTok{)}
\end{Highlighting}
\end{Shaded}

\begin{figure}[H]

{\centering \includegraphics{analysis/SGC3A/3_sgc3A_description_files/figure-pdf/unnamed-chunk-23-4.pdf}

}

\end{figure}

\begin{Shaded}
\begin{Highlighting}[]
\FunctionTok{gf\_boxplot}\NormalTok{( totaltime\_m }\SpecialCharTok{\textasciitilde{}}\NormalTok{ pretty\_condition,  }\AttributeTok{data =}\NormalTok{ df\_subjects) }\SpecialCharTok{\%\textgreater{}\%} 
  \FunctionTok{gf\_facet\_wrap}\NormalTok{(}\SpecialCharTok{\textasciitilde{}}\NormalTok{pretty\_mode) }\SpecialCharTok{+}
  \FunctionTok{labs}\NormalTok{(}\AttributeTok{title =} \StringTok{"Average TOTAL response time by mode and condition"}\NormalTok{)}
\end{Highlighting}
\end{Shaded}

\begin{figure}[H]

{\centering \includegraphics{analysis/SGC3A/3_sgc3A_description_files/figure-pdf/unnamed-chunk-23-5.pdf}

}

\end{figure}

\begin{Shaded}
\begin{Highlighting}[]
\FunctionTok{gf\_boxplot}\NormalTok{( item\_q1\_rt }\SpecialCharTok{\textasciitilde{}}\NormalTok{ pretty\_condition,  }\AttributeTok{data =}\NormalTok{ df\_subjects) }\SpecialCharTok{\%\textgreater{}\%} 
  \FunctionTok{gf\_jitter}\NormalTok{(}\AttributeTok{width=}\FloatTok{0.2}\NormalTok{, }\AttributeTok{alpha =} \FloatTok{0.5}\NormalTok{, }\AttributeTok{size =} \FloatTok{0.75}\NormalTok{, }\AttributeTok{color =} \SpecialCharTok{\textasciitilde{}}\NormalTok{item\_q1\_NABS) }\SpecialCharTok{\%\textgreater{}\%} 
  \FunctionTok{gf\_facet\_wrap}\NormalTok{(}\SpecialCharTok{\textasciitilde{}}\NormalTok{pretty\_mode) }\SpecialCharTok{+}
  \FunctionTok{labs}\NormalTok{(}\AttributeTok{title =} \StringTok{"Average FIRST ITEM response time by mode and condition"}\NormalTok{)}
\end{Highlighting}
\end{Shaded}

\begin{figure}[H]

{\centering \includegraphics{analysis/SGC3A/3_sgc3A_description_files/figure-pdf/unnamed-chunk-23-6.pdf}

}

\end{figure}

\hypertarget{replication-check}{%
\section{REPLICATION CHECK}\label{replication-check}}

\hypertarget{data-collection-mode-on-absolute-score}{%
\subsection{Data Collection Mode on Absolute
Score}\label{data-collection-mode-on-absolute-score}}

\textbf{Does Mode Change Effect of Condition on Score?}

To verify that the data collected in the lab and remotely online are
comparable, we perform a t-test on group means of ABSOLUTE SCORE for
each condition, and examine whether data collection modality is a
significant predictor of variance in absolute score

\begin{Shaded}
\begin{Highlighting}[]
\FunctionTok{paste}\NormalTok{(}\StringTok{"Two Sample T{-}Test for S\_ABS LAB vs. ONLINE control condition"}\NormalTok{)}
\end{Highlighting}
\end{Shaded}

\begin{verbatim}
[1] "Two Sample T-Test for S_ABS LAB vs. ONLINE control condition"
\end{verbatim}

\begin{Shaded}
\begin{Highlighting}[]
\FunctionTok{t.test}\NormalTok{(}\AttributeTok{data =}\NormalTok{ df\_subjects }\SpecialCharTok{\%\textgreater{}\%} \FunctionTok{filter}\NormalTok{(condition }\SpecialCharTok{==} \DecValTok{111}\NormalTok{), s\_ABS }\SpecialCharTok{\textasciitilde{}}\NormalTok{ mode )}
\end{Highlighting}
\end{Shaded}

\begin{verbatim}

    Welch Two Sample t-test

data:  s_ABS by mode
t = 0.5, df = 120, p-value = 0.6
alternative hypothesis: true difference in means between group lab-synch and group asynch is not equal to 0
95 percent confidence interval:
 -1.09  1.84
sample estimates:
mean in group lab-synch    mean in group asynch 
                   2.68                    2.30 
\end{verbatim}

\begin{Shaded}
\begin{Highlighting}[]
\FunctionTok{paste}\NormalTok{(}\StringTok{"Two Sample T{-}Test for S\_ABS LAB vs. ONLINE impasse condition"}\NormalTok{)}
\end{Highlighting}
\end{Shaded}

\begin{verbatim}
[1] "Two Sample T-Test for S_ABS LAB vs. ONLINE impasse condition"
\end{verbatim}

\begin{Shaded}
\begin{Highlighting}[]
\FunctionTok{t.test}\NormalTok{(}\AttributeTok{data =}\NormalTok{ df\_subjects }\SpecialCharTok{\%\textgreater{}\%} \FunctionTok{filter}\NormalTok{(condition }\SpecialCharTok{==} \DecValTok{121}\NormalTok{), s\_ABS }\SpecialCharTok{\textasciitilde{}}\NormalTok{ mode )}
\end{Highlighting}
\end{Shaded}

\begin{verbatim}

    Welch Two Sample t-test

data:  s_ABS by mode
t = 1, df = 135, p-value = 0.3
alternative hypothesis: true difference in means between group lab-synch and group asynch is not equal to 0
95 percent confidence interval:
 -0.727  2.435
sample estimates:
mean in group lab-synch    mean in group asynch 
                   5.44                    4.58 
\end{verbatim}

\begin{Shaded}
\begin{Highlighting}[]
\FunctionTok{paste}\NormalTok{(}\StringTok{"OLS Linear Regression Predicting Absolute Score by Data Collection Mode"}\NormalTok{)}
\end{Highlighting}
\end{Shaded}

\begin{verbatim}
[1] "OLS Linear Regression Predicting Absolute Score by Data Collection Mode"
\end{verbatim}

\begin{Shaded}
\begin{Highlighting}[]
\FunctionTok{summary}\NormalTok{(}\FunctionTok{lm}\NormalTok{(}\AttributeTok{data =}\NormalTok{ df\_subjects, }\AttributeTok{formula =}\NormalTok{ s\_ABS }\SpecialCharTok{\textasciitilde{}}\NormalTok{ mode ))}
\end{Highlighting}
\end{Shaded}

\begin{verbatim}

Call:
lm(formula = s_ABS ~ mode, data = df_subjects)

Residuals:
   Min     1Q Median     3Q    Max 
 -4.08  -3.51  -2.51   4.49   9.49 

Coefficients:
            Estimate Std. Error t value Pr(>|t|)    
(Intercept)     4.08       0.44    9.27   <2e-16 ***
modeasynch     -0.57       0.56   -1.02     0.31    
---
Signif. codes:  0 '***' 0.001 '**' 0.01 '*' 0.05 '.' 0.1 ' ' 1

Residual standard error: 4.94 on 328 degrees of freedom
Multiple R-squared:  0.00314,   Adjusted R-squared:  0.000105 
F-statistic: 1.03 on 1 and 328 DF,  p-value: 0.31
\end{verbatim}

Both t-tests are non-significant with 95\% confidence intervals
including 0. Further, an OLS linear regression model predicting
cumulative absolute score indicates that data collection mode is not a
significant predictor, explaining only 0.01\% of variance in absolute
score, F(1,328) = 1.03, p \textgreater{} 0.05.

\begin{tcolorbox}[standard jigsaw,bottomrule=.15mm, opacitybacktitle=0.6, bottomtitle=1mm, toptitle=1mm, titlerule=0mm, title={Decision}, toprule=.15mm, rightrule=.15mm, colback=white, arc=.35mm, left=2mm, colframe=quarto-callout-color-frame, coltitle=black, leftrule=.75mm, opacityback=0, colbacktitle=quarto-callout-color!10!white]
\textbf{It is reasonable to infer that data from both in-person and
remote studies arise from the same data generating process.}
\end{tcolorbox}

\hypertarget{data-collection-mode-on-cumulative-score}{%
\subsection{Data Collection Mode on Cumulative
Score}\label{data-collection-mode-on-cumulative-score}}

\textbf{Are the by-condition group means significantly different by data
collection modality?}

To verify that the data collected in the lab and remotely online are
comparable, we perform a t-test on group means of SCALED SCORE for each
condition.

\begin{Shaded}
\begin{Highlighting}[]
\FunctionTok{paste}\NormalTok{(}\StringTok{"Two Sample T{-}Test for s\_SCALED LAB vs. ONLINE control condition"}\NormalTok{)}
\end{Highlighting}
\end{Shaded}

\begin{verbatim}
[1] "Two Sample T-Test for s_SCALED LAB vs. ONLINE control condition"
\end{verbatim}

\begin{Shaded}
\begin{Highlighting}[]
\FunctionTok{t.test}\NormalTok{(}\AttributeTok{data =}\NormalTok{ df\_subjects }\SpecialCharTok{\%\textgreater{}\%} \FunctionTok{filter}\NormalTok{(condition }\SpecialCharTok{==} \DecValTok{111}\NormalTok{), s\_SCALED }\SpecialCharTok{\textasciitilde{}}\NormalTok{ mode )}
\end{Highlighting}
\end{Shaded}

\begin{verbatim}

    Welch Two Sample t-test

data:  s_SCALED by mode
t = -0.1, df = 117, p-value = 0.9
alternative hypothesis: true difference in means between group lab-synch and group asynch is not equal to 0
95 percent confidence interval:
 -3.15  2.83
sample estimates:
mean in group lab-synch    mean in group asynch 
                  -6.52                   -6.36 
\end{verbatim}

\begin{Shaded}
\begin{Highlighting}[]
\FunctionTok{paste}\NormalTok{(}\StringTok{"Two Sample T{-}Test for s\_SCALED LAB vs. ONLINE impasse condition"}\NormalTok{)}
\end{Highlighting}
\end{Shaded}

\begin{verbatim}
[1] "Two Sample T-Test for s_SCALED LAB vs. ONLINE impasse condition"
\end{verbatim}

\begin{Shaded}
\begin{Highlighting}[]
\FunctionTok{t.test}\NormalTok{(}\AttributeTok{data =}\NormalTok{ df\_subjects }\SpecialCharTok{\%\textgreater{}\%} \FunctionTok{filter}\NormalTok{(condition }\SpecialCharTok{==} \DecValTok{121}\NormalTok{), s\_SCALED }\SpecialCharTok{\textasciitilde{}}\NormalTok{ mode )}
\end{Highlighting}
\end{Shaded}

\begin{verbatim}

    Welch Two Sample t-test

data:  s_SCALED by mode
t = 0.5, df = 130, p-value = 0.6
alternative hypothesis: true difference in means between group lab-synch and group asynch is not equal to 0
95 percent confidence interval:
 -2.08  3.49
sample estimates:
mean in group lab-synch    mean in group asynch 
                  1.008                   0.306 
\end{verbatim}

\begin{Shaded}
\begin{Highlighting}[]
\FunctionTok{paste}\NormalTok{(}\StringTok{"OLS Linear Regression Predicting Scaled Score by Data Collection Mode"}\NormalTok{)}
\end{Highlighting}
\end{Shaded}

\begin{verbatim}
[1] "OLS Linear Regression Predicting Scaled Score by Data Collection Mode"
\end{verbatim}

\begin{Shaded}
\begin{Highlighting}[]
\FunctionTok{summary}\NormalTok{(}\FunctionTok{lm}\NormalTok{(}\AttributeTok{data =}\NormalTok{ df\_subjects, }\AttributeTok{formula =}\NormalTok{ s\_SCALED }\SpecialCharTok{\textasciitilde{}}\NormalTok{ mode ))}
\end{Highlighting}
\end{Shaded}

\begin{verbatim}

Call:
lm(formula = s_SCALED ~ mode, data = df_subjects)

Residuals:
   Min     1Q Median     3Q    Max 
-10.30  -7.80  -4.42  10.33  15.83 

Coefficients:
            Estimate Std. Error t value Pr(>|t|)   
(Intercept)   -2.698      0.853   -3.16   0.0017 **
modeasynch    -0.135      1.085   -0.12   0.9011   
---
Signif. codes:  0 '***' 0.001 '**' 0.01 '*' 0.05 '.' 0.1 ' ' 1

Residual standard error: 9.58 on 328 degrees of freedom
Multiple R-squared:  4.71e-05,  Adjusted R-squared:  -0.003 
F-statistic: 0.0155 on 1 and 328 DF,  p-value: 0.901
\end{verbatim}

Both t-tests are non-significant with 95\% confidence intervals
including 0. Further, an OLS linear regression model predicting
cumulative scaled score indicates that data collection mode is not a
significant predictor, explaining less than 0.001\% of variance in
absolute score, F(1,328) = 0.0078, p \textgreater{} 0.05.

\begin{tcolorbox}[standard jigsaw,bottomrule=.15mm, opacitybacktitle=0.6, bottomtitle=1mm, toptitle=1mm, titlerule=0mm, title={Decision}, toprule=.15mm, rightrule=.15mm, colback=white, arc=.35mm, left=2mm, colframe=quarto-callout-color-frame, coltitle=black, leftrule=.75mm, opacityback=0, colbacktitle=quarto-callout-color!10!white]
\textbf{It is reasonable to infer that data from both in-person and
remote studies arise from the same data generating process.}
\end{tcolorbox}

\hypertarget{resources-2}{%
\section{RESOURCES}\label{resources-2}}

\begin{itemize}
\tightlist
\item
  https://rpkgs.datanovia.com/ggpubr/reference/index.html
\end{itemize}

\begin{Shaded}
\begin{Highlighting}[]
\FunctionTok{sessionInfo}\NormalTok{()}
\end{Highlighting}
\end{Shaded}

\begin{verbatim}
R version 4.2.1 (2022-06-23)
Platform: x86_64-apple-darwin17.0 (64-bit)
Running under: macOS Big Sur ... 10.16

Matrix products: default
BLAS:   /Library/Frameworks/R.framework/Versions/4.2/Resources/lib/libRblas.0.dylib
LAPACK: /Library/Frameworks/R.framework/Versions/4.2/Resources/lib/libRlapack.dylib

locale:
[1] en_US.UTF-8/en_US.UTF-8/en_US.UTF-8/C/en_US.UTF-8/en_US.UTF-8

attached base packages:
[1] grid      stats     graphics  grDevices utils     datasets  methods  
[8] base     

other attached packages:
 [1] forcats_0.5.1      stringr_1.4.0      purrr_0.3.4        readr_2.1.2       
 [5] tidyr_1.2.0        tibble_3.1.7       tidyverse_1.3.1    performance_0.9.1 
 [9] fitdistrplus_1.1-8 MASS_7.3-57        multimode_1.5      ggdist_3.1.1      
[13] ggpubr_0.4.0       vcd_1.4-10         kableExtra_1.3.4   mosaic_1.8.3      
[17] ggridges_0.5.3     mosaicData_0.20.2  ggformula_0.10.1   ggstance_0.3.5    
[21] dplyr_1.0.9        Matrix_1.4-1       Hmisc_4.7-0        ggplot2_3.3.6     
[25] Formula_1.2-4      survival_3.3-1     lattice_0.20-45   

loaded via a namespace (and not attached):
  [1] colorspace_2.0-3     ggsignif_0.6.3       ellipsis_0.3.2      
  [4] mclust_5.4.10        leaflet_2.1.1        htmlTable_2.4.0     
  [7] fs_1.5.2             base64enc_0.1-3      ggdendro_0.1.23     
 [10] rstudioapi_0.13      farver_2.1.0         bit64_4.0.5         
 [13] ggrepel_0.9.1        lubridate_1.8.0      mvtnorm_1.1-3       
 [16] fansi_1.0.3          xml2_1.3.3           codetools_0.2-18    
 [19] splines_4.2.1        rootSolve_1.8.2.3    knitr_1.39          
 [22] polyclip_1.10-0      jsonlite_1.8.0       broom_0.8.0         
 [25] dbplyr_2.2.1         cluster_2.1.3        png_0.1-7           
 [28] ggforce_0.3.3        compiler_4.2.1       httr_1.4.3          
 [31] backports_1.4.1      assertthat_0.2.1     fastmap_1.1.0       
 [34] cli_3.3.0            tweenr_1.0.2         htmltools_0.5.2     
 [37] tools_4.2.1          gtable_0.3.0         glue_1.6.2          
 [40] Rcpp_1.0.8.3         carData_3.0-5        cellranger_1.1.0    
 [43] vctrs_0.4.1          svglite_2.1.0        crosstalk_1.2.0     
 [46] insight_0.17.1       lmtest_0.9-40        xfun_0.31           
 [49] rvest_1.0.2          lifecycle_1.0.1      mosaicCore_0.9.0    
 [52] rstatix_0.7.0        zoo_1.8-10           scales_1.2.0        
 [55] vroom_1.5.7          hms_1.1.1            parallel_4.2.1      
 [58] RColorBrewer_1.1-3   yaml_2.3.5           gridExtra_2.3       
 [61] labelled_2.9.1       rpart_4.1.16         latticeExtra_0.6-29 
 [64] stringi_1.7.6        checkmate_2.1.0      rlang_1.0.3         
 [67] pkgconfig_2.0.3      systemfonts_1.0.4    distributional_0.3.0
 [70] pracma_2.3.8         evaluate_0.15        labeling_0.4.2      
 [73] ks_1.13.5            htmlwidgets_1.5.4    bit_4.0.4           
 [76] tidyselect_1.1.2     plyr_1.8.7           magrittr_2.0.3      
 [79] R6_2.5.1             generics_0.1.2       DBI_1.1.3           
 [82] pillar_1.7.0         haven_2.5.0          foreign_0.8-82      
 [85] withr_2.5.0          abind_1.4-5          nnet_7.3-17         
 [88] modelr_0.1.8         crayon_1.5.1         car_3.1-0           
 [91] KernSmooth_2.23-20   utf8_1.2.2           tzdb_0.3.0          
 [94] rmarkdown_2.14       jpeg_0.1-9           readxl_1.4.0        
 [97] data.table_1.14.2    reprex_2.0.1         diptest_0.76-0      
[100] digest_0.6.29        webshot_0.5.3        munsell_0.5.0       
[103] viridisLite_0.4.0   
\end{verbatim}

\hypertarget{archive-1}{%
\section{ARCHIVE}\label{archive-1}}

Sample ridgeplot code

\begin{Shaded}
\begin{Highlighting}[]
\CommentTok{\#RIDGEPLOT}
\CommentTok{\# ggplot(data = df\_subjects, aes(x = s\_NABS, y = mode)) +}
\CommentTok{\#   geom\_density\_ridges() + xlim(0,13)+}
\CommentTok{\#   facet\_wrap(\textasciitilde{}condition, labeller = label\_both) +}
\CommentTok{\# labs(x = "total number correct ",}
\CommentTok{\# y = "proportion of subjects",}
\CommentTok{\#        title = "Subject Cumulative Score (Absolute)",}
\CommentTok{\#        subtitle = "Score distributions are comparable across modalities and different across conditions") +}
\CommentTok{\#   theme\_minimal()}
\end{Highlighting}
\end{Shaded}

\hypertarget{what-kind-of-distribution-is-total-score}{%
\subsection{What Kind of Distribution is Total
Score?}\label{what-kind-of-distribution-is-total-score}}

What kind of distribution is the Total Absolute Score (TEST Phase) data?
We use the \texttt{fitdistrplus} package to compare the distribution of
this variable to a variety of probability distribution families. First,
we transform the \# correct items to \% correct items by dividing it by
the total number of items (n = 8).

\begin{Shaded}
\begin{Highlighting}[]
\CommentTok{\#describe the distribution}
\FunctionTok{descdist}\NormalTok{(}\AttributeTok{data =}\NormalTok{ df\_subjects}\SpecialCharTok{$}\NormalTok{item\_test\_NABS}\SpecialCharTok{/}\DecValTok{8}\NormalTok{, }\AttributeTok{discrete =} \ConstantTok{FALSE}\NormalTok{, }\AttributeTok{boot =} \DecValTok{1000}\NormalTok{)}
\end{Highlighting}
\end{Shaded}

\begin{figure}[H]

{\centering \includegraphics{analysis/SGC3A/3_sgc3A_description_files/figure-pdf/FIT-DIST-TOTAL-ABS-TEST-1.pdf}

}

\end{figure}

\begin{verbatim}
summary statistics
------
min:  0   max:  1 
median:  0 
mean:  0.287 
estimated sd:  0.405 
estimated skewness:  0.876 
estimated kurtosis:  1.93 
\end{verbatim}

\begin{Shaded}
\begin{Highlighting}[]
\FunctionTok{print}\NormalTok{(}\StringTok{"FIT A NORMAL DISTRIBUTION"}\NormalTok{)}
\end{Highlighting}
\end{Shaded}

\begin{verbatim}
[1] "FIT A NORMAL DISTRIBUTION"
\end{verbatim}

\begin{Shaded}
\begin{Highlighting}[]
\NormalTok{normal\_ }\OtherTok{=} \FunctionTok{fitdist}\NormalTok{(df\_subjects}\SpecialCharTok{$}\NormalTok{item\_test\_NABS}\SpecialCharTok{/}\DecValTok{8}\NormalTok{,}\StringTok{"norm"}\NormalTok{)}
\FunctionTok{plot}\NormalTok{(normal\_)}
\end{Highlighting}
\end{Shaded}

\begin{figure}[H]

{\centering \includegraphics{analysis/SGC3A/3_sgc3A_description_files/figure-pdf/FIT-DIST-TOTAL-ABS-TEST-2.pdf}

}

\end{figure}

\begin{Shaded}
\begin{Highlighting}[]
\FunctionTok{print}\NormalTok{(}\StringTok{"FIT A BETA DISTRIBUTION"}\NormalTok{)}
\end{Highlighting}
\end{Shaded}

\begin{verbatim}
[1] "FIT A BETA DISTRIBUTION"
\end{verbatim}

\begin{Shaded}
\begin{Highlighting}[]
\NormalTok{beta\_ }\OtherTok{=} \FunctionTok{fitdist}\NormalTok{(df\_subjects}\SpecialCharTok{$}\NormalTok{item\_test\_NABS}\SpecialCharTok{/}\DecValTok{8}\NormalTok{,}\StringTok{"beta"}\NormalTok{, }\AttributeTok{method=}\StringTok{"mme"}\NormalTok{ )}
\FunctionTok{plot}\NormalTok{(beta\_)}
\end{Highlighting}
\end{Shaded}

\begin{figure}[H]

{\centering \includegraphics{analysis/SGC3A/3_sgc3A_description_files/figure-pdf/FIT-DIST-TOTAL-ABS-TEST-3.pdf}

}

\end{figure}

\begin{Shaded}
\begin{Highlighting}[]
\FunctionTok{summary}\NormalTok{(beta\_)}
\end{Highlighting}
\end{Shaded}

\begin{verbatim}
Fitting of the distribution ' beta ' by matching moments 
Parameters : 
       estimate
shape1   0.0721
shape2   0.1786
Loglikelihood:  Inf   AIC:  -Inf   BIC:  -Inf 
\end{verbatim}

Interpreting the Cullen and Frey graph, it \emph{appears} that number
percentage of correct responses per subject may follow a beta
distribution (u-shape tpe). If we fit this variable using both a normal
and beta distribution (using method of moments), it appears that the
beta distribution provides a much better fit. The parameter estimates
for the beta distribution are: shape1 = 0.072, shape2 = 0.179. The beta
distribution is a flexible distribution insofar as it can model a wide
range of shapes with its two parameters. TODO: HOW might this be applied
to the total score data?

\textbf{Analysis Notes} - This distribution is very bimodal, so OLS
linear regression estimating means may not be informative, as the mean
actually falls near the location of the anitmode (least common value) -
Should investigate log transform to see if residuals of LM will be
normal (no) - Should investigate beta regression

\hypertarget{whole-task-scores}{%
\subsection{Whole Task Scores}\label{whole-task-scores}}

\hypertarget{absolute-score-1}{%
\paragraph{Absolute Score}\label{absolute-score-1}}

Total Scores that include \emph{both} Scaffolding Phase as well as Test
Phase performance.

\begin{Shaded}
\begin{Highlighting}[]
\NormalTok{title }\OtherTok{=} \StringTok{"Descriptive Statistics of Response Accuracy (Total Absolute Score)"}
\NormalTok{abs.stats }\OtherTok{\textless{}{-}} \FunctionTok{rbind}\NormalTok{(}
  \StringTok{"lab"}\OtherTok{=}\NormalTok{ df\_lab }\SpecialCharTok{\%\textgreater{}\%}\NormalTok{ dplyr}\SpecialCharTok{::}\FunctionTok{select}\NormalTok{(s\_NABS) }\SpecialCharTok{\%\textgreater{}\%} \FunctionTok{unlist}\NormalTok{() }\SpecialCharTok{\%\textgreater{}\%} \FunctionTok{favstats}\NormalTok{(),}
  \StringTok{"online"} \OtherTok{=}\NormalTok{ df\_online }\SpecialCharTok{\%\textgreater{}\%}\NormalTok{ dplyr}\SpecialCharTok{::}\FunctionTok{select}\NormalTok{(s\_NABS) }\SpecialCharTok{\%\textgreater{}\%} \FunctionTok{unlist}\NormalTok{() }\SpecialCharTok{\%\textgreater{}\%} \FunctionTok{favstats}\NormalTok{(),}
  \StringTok{"combined"} \OtherTok{=}\NormalTok{ df\_subjects }\SpecialCharTok{\%\textgreater{}\%}\NormalTok{ dplyr}\SpecialCharTok{::}\FunctionTok{select}\NormalTok{(s\_NABS) }\SpecialCharTok{\%\textgreater{}\%} \FunctionTok{unlist}\NormalTok{() }\SpecialCharTok{\%\textgreater{}\%} \FunctionTok{favstats}\NormalTok{()}
\NormalTok{) }
\NormalTok{abs.stats }\SpecialCharTok{\%\textgreater{}\%} \FunctionTok{kbl}\NormalTok{ (}\AttributeTok{caption =}\NormalTok{ title) }\SpecialCharTok{\%\textgreater{}\%} \FunctionTok{kable\_classic}\NormalTok{()}
\end{Highlighting}
\end{Shaded}

\begin{table}

\caption{Descriptive Statistics of Response Accuracy (Total Absolute Score)}
\centering
\begin{tabular}[t]{l|r|r|r|r|r|r|r|r|r}
\hline
  & min & Q1 & median & Q3 & max & mean & sd & n & missing\\
\hline
lab & 0 & 0 & 1 & 9.00 & 13 & 4.11 & 5.09 & 126 & 0\\
\hline
online & 0 & 0 & 1 & 8.00 & 13 & 3.52 & 4.89 & 204 & 0\\
\hline
combined & 0 & 0 & 1 & 8.75 & 13 & 3.75 & 4.97 & 330 & 0\\
\hline
\end{tabular}
\end{table}

For \emph{in person} collection, total absolute scores (n = 126) range
from 0 to 13 with a mean score of (M = 4.11, SD = 5.09).

For \emph{online replication}, (online) total absolute accuracy scores
(n = 204) range from 0 to 13 with a slightly lower mean score of (M =
3.52, SD = 4.89).

When combined \emph{overall}, total absolute accuracy scores (n = 330)
range from 0 to 13 with a slightly lower mean score of (M = 3.75, SD =
4.97).

\begin{Shaded}
\begin{Highlighting}[]
\CommentTok{\#GGFORMULA | DENSITY HISTOGRAM SUBJECT TOTAL ABSOLUTE}
  \FunctionTok{gf\_props}\NormalTok{(}\SpecialCharTok{\textasciitilde{}}\NormalTok{s\_NABS, }\AttributeTok{data =}\NormalTok{ df\_subjects) }\SpecialCharTok{+} 
  \FunctionTok{labs}\NormalTok{(}\AttributeTok{x =} \StringTok{"number of correct responses"}\NormalTok{,}
       \AttributeTok{y =} \StringTok{"\% of subjects"}\NormalTok{,}
       \AttributeTok{title =} \StringTok{"Distribution of Task Absolute Score"}\NormalTok{,}
       \AttributeTok{subtitle =} \StringTok{"Modes at high and low ends of scale suggest concentration of high (vs) low understanding"}\NormalTok{) }\SpecialCharTok{+} 
  \FunctionTok{theme\_minimal}\NormalTok{()}
\end{Highlighting}
\end{Shaded}

\begin{figure}[H]

{\centering \includegraphics{analysis/SGC3A/3_sgc3A_description_files/figure-pdf/VIS-SUBJ-ABS-1.pdf}

}

\end{figure}

\begin{Shaded}
\begin{Highlighting}[]
\DocumentationTok{\#\#GGPUBR | HIST+DENSITY SCORE BY CONDITION/MODE}
\NormalTok{p }\OtherTok{\textless{}{-}} \FunctionTok{gghistogram}\NormalTok{(df\_subjects, }\AttributeTok{x =} \StringTok{"s\_NABS"}\NormalTok{, }\AttributeTok{binwidth =} \DecValTok{1}\NormalTok{,}
   \AttributeTok{add =} \StringTok{"mean"}\NormalTok{, }\AttributeTok{rug =} \ConstantTok{TRUE}\NormalTok{,}
   \AttributeTok{fill =} \StringTok{"pretty\_condition"}\NormalTok{, }\CommentTok{\#, palette = c("\#00AFBB", "\#E7B800"),}
   \AttributeTok{add\_density =} \ConstantTok{TRUE}\NormalTok{)}
\FunctionTok{facet}\NormalTok{(p, }\AttributeTok{facet.by=}\FunctionTok{c}\NormalTok{(}\StringTok{"pretty\_condition"}\NormalTok{,}\StringTok{"pretty\_mode"}\NormalTok{)) }\SpecialCharTok{+}
  \FunctionTok{labs}\NormalTok{( }\AttributeTok{title =} \StringTok{"Distribution of Task Absolute Score (by Mode and Condition)"}\NormalTok{,}
        \AttributeTok{subtitle =}\StringTok{"Pattern of response is the same across data collection modes but differs by condition"}\NormalTok{,}
        \AttributeTok{x =} \StringTok{"Total Absolute Score"}\NormalTok{, }\AttributeTok{y =} \StringTok{"number of subjects"}\NormalTok{) }\SpecialCharTok{+}
  \FunctionTok{theme\_minimal}\NormalTok{() }\SpecialCharTok{+} \FunctionTok{theme}\NormalTok{(}\AttributeTok{legend.position =} \StringTok{"blank"}\NormalTok{)}
\end{Highlighting}
\end{Shaded}

\begin{figure}[H]

{\centering \includegraphics{analysis/SGC3A/3_sgc3A_description_files/figure-pdf/VIS-SUBJ-ABS-2.pdf}

}

\end{figure}

\begin{Shaded}
\begin{Highlighting}[]
\DocumentationTok{\#\#RAINCLOUD USING GGDISTR}
\FunctionTok{ggplot}\NormalTok{(df\_subjects, }\FunctionTok{aes}\NormalTok{(}\AttributeTok{x =}\NormalTok{ pretty\_condition, }\AttributeTok{y =}\NormalTok{ s\_NABS, }\AttributeTok{fill =}\NormalTok{ pretty\_condition)) }\SpecialCharTok{+} 
\NormalTok{  ggdist}\SpecialCharTok{::}\FunctionTok{stat\_halfeye}\NormalTok{(}
    \AttributeTok{adjust =}\NormalTok{ .}\DecValTok{5}\NormalTok{, }
    \AttributeTok{width =}\NormalTok{ .}\DecValTok{6}\NormalTok{, }
    \AttributeTok{.width =} \DecValTok{0}\NormalTok{, }
    \AttributeTok{justification =} \SpecialCharTok{{-}}\NormalTok{.}\DecValTok{3}\NormalTok{, }
    \AttributeTok{point\_colour =} \ConstantTok{NA}\NormalTok{) }\SpecialCharTok{+} 
  \FunctionTok{geom\_boxplot}\NormalTok{(}
    \AttributeTok{width =}\NormalTok{ .}\DecValTok{15}\NormalTok{, }
    \AttributeTok{outlier.shape =} \ConstantTok{NA}
\NormalTok{  ) }\SpecialCharTok{+}
  \FunctionTok{geom\_point}\NormalTok{(}
    \AttributeTok{size =} \FloatTok{1.3}\NormalTok{,}
    \AttributeTok{alpha =}\NormalTok{ .}\DecValTok{3}\NormalTok{,}
    \AttributeTok{position =} \FunctionTok{position\_jitter}\NormalTok{(}
      \AttributeTok{seed =} \DecValTok{1}\NormalTok{, }\AttributeTok{width =}\NormalTok{ .}\DecValTok{1}
\NormalTok{    )}
\NormalTok{  ) }\SpecialCharTok{+} \FunctionTok{labs}\NormalTok{(}
    \AttributeTok{title =} \StringTok{"Distribution of Task Absolute Score"}\NormalTok{,}
    \AttributeTok{x =} \StringTok{"Condition"}\NormalTok{, }\AttributeTok{y =} \StringTok{"Total Absolute Score"}
\NormalTok{  ) }\SpecialCharTok{+} \FunctionTok{theme\_ggdist}\NormalTok{() }\SpecialCharTok{+} \FunctionTok{theme}\NormalTok{(}\AttributeTok{legend.position =} \StringTok{"blank"}\NormalTok{)}
\end{Highlighting}
\end{Shaded}

\begin{figure}[H]

{\centering \includegraphics{analysis/SGC3A/3_sgc3A_description_files/figure-pdf/VIS-SUBJ-ABS-3.pdf}

}

\end{figure}

\begin{Shaded}
\begin{Highlighting}[]
\CommentTok{\# + coord\_cartesian(xlim = c(1.2, NA), clip = "off")}


\CommentTok{\#PLOT EMPIRICIAL CUMULATIVE DISTRIBUTION FUNCTION}
\FunctionTok{ggplot}\NormalTok{(}\AttributeTok{data =}\NormalTok{ df\_subjects, }\FunctionTok{aes}\NormalTok{(s\_NABS)) }\SpecialCharTok{+} 
  \FunctionTok{stat\_ecdf}\NormalTok{(}\AttributeTok{geom =} \StringTok{"step"}\NormalTok{) }\SpecialCharTok{+} 
  \FunctionTok{facet\_grid}\NormalTok{(pretty\_condition }\SpecialCharTok{\textasciitilde{}}\NormalTok{ pretty\_mode) }\SpecialCharTok{+} 
  \FunctionTok{labs}\NormalTok{( }\AttributeTok{title =} \StringTok{"Empirical Cumulative Density Function — Task Absolute Score"}\NormalTok{,}
        \AttributeTok{x =} \StringTok{"Task Absolute Score [0,13]"}\NormalTok{, }
        \AttributeTok{y =} \StringTok{"Cumulative Probability"}\NormalTok{) }\SpecialCharTok{+} \FunctionTok{theme\_minimal}\NormalTok{()}
\end{Highlighting}
\end{Shaded}

\begin{verbatim}
Warning in grid.Call(C_textBounds, as.graphicsAnnot(x$label), x$x, x$y, :
conversion failure on 'Empirical Cumulative Density Function — Task Absolute
Score' in 'mbcsToSbcs': dot substituted for <e2>
\end{verbatim}

\begin{verbatim}
Warning in grid.Call(C_textBounds, as.graphicsAnnot(x$label), x$x, x$y, :
conversion failure on 'Empirical Cumulative Density Function — Task Absolute
Score' in 'mbcsToSbcs': dot substituted for <80>
\end{verbatim}

\begin{verbatim}
Warning in grid.Call(C_textBounds, as.graphicsAnnot(x$label), x$x, x$y, :
conversion failure on 'Empirical Cumulative Density Function — Task Absolute
Score' in 'mbcsToSbcs': dot substituted for <94>
\end{verbatim}

\begin{verbatim}
Warning in grid.Call(C_textBounds, as.graphicsAnnot(x$label), x$x, x$y, :
conversion failure on 'Empirical Cumulative Density Function — Task Absolute
Score' in 'mbcsToSbcs': dot substituted for <e2>
\end{verbatim}

\begin{verbatim}
Warning in grid.Call(C_textBounds, as.graphicsAnnot(x$label), x$x, x$y, :
conversion failure on 'Empirical Cumulative Density Function — Task Absolute
Score' in 'mbcsToSbcs': dot substituted for <80>
\end{verbatim}

\begin{verbatim}
Warning in grid.Call(C_textBounds, as.graphicsAnnot(x$label), x$x, x$y, :
conversion failure on 'Empirical Cumulative Density Function — Task Absolute
Score' in 'mbcsToSbcs': dot substituted for <94>
\end{verbatim}

\begin{verbatim}
Warning in grid.Call(C_textBounds, as.graphicsAnnot(x$label), x$x, x$y, :
conversion failure on 'Empirical Cumulative Density Function — Task Absolute
Score' in 'mbcsToSbcs': dot substituted for <e2>
\end{verbatim}

\begin{verbatim}
Warning in grid.Call(C_textBounds, as.graphicsAnnot(x$label), x$x, x$y, :
conversion failure on 'Empirical Cumulative Density Function — Task Absolute
Score' in 'mbcsToSbcs': dot substituted for <80>
\end{verbatim}

\begin{verbatim}
Warning in grid.Call(C_textBounds, as.graphicsAnnot(x$label), x$x, x$y, :
conversion failure on 'Empirical Cumulative Density Function — Task Absolute
Score' in 'mbcsToSbcs': dot substituted for <94>
\end{verbatim}

\begin{verbatim}
Warning in grid.Call(C_textBounds, as.graphicsAnnot(x$label), x$x, x$y, :
conversion failure on 'Empirical Cumulative Density Function — Task Absolute
Score' in 'mbcsToSbcs': dot substituted for <e2>
\end{verbatim}

\begin{verbatim}
Warning in grid.Call(C_textBounds, as.graphicsAnnot(x$label), x$x, x$y, :
conversion failure on 'Empirical Cumulative Density Function — Task Absolute
Score' in 'mbcsToSbcs': dot substituted for <80>
\end{verbatim}

\begin{verbatim}
Warning in grid.Call(C_textBounds, as.graphicsAnnot(x$label), x$x, x$y, :
conversion failure on 'Empirical Cumulative Density Function — Task Absolute
Score' in 'mbcsToSbcs': dot substituted for <94>
\end{verbatim}

\begin{verbatim}
Warning in grid.Call(C_textBounds, as.graphicsAnnot(x$label), x$x, x$y, :
conversion failure on 'Empirical Cumulative Density Function — Task Absolute
Score' in 'mbcsToSbcs': dot substituted for <e2>
\end{verbatim}

\begin{verbatim}
Warning in grid.Call(C_textBounds, as.graphicsAnnot(x$label), x$x, x$y, :
conversion failure on 'Empirical Cumulative Density Function — Task Absolute
Score' in 'mbcsToSbcs': dot substituted for <80>
\end{verbatim}

\begin{verbatim}
Warning in grid.Call(C_textBounds, as.graphicsAnnot(x$label), x$x, x$y, :
conversion failure on 'Empirical Cumulative Density Function — Task Absolute
Score' in 'mbcsToSbcs': dot substituted for <94>
\end{verbatim}

\begin{verbatim}
Warning in grid.Call(C_textBounds, as.graphicsAnnot(x$label), x$x, x$y, :
conversion failure on 'Empirical Cumulative Density Function — Task Absolute
Score' in 'mbcsToSbcs': dot substituted for <e2>
\end{verbatim}

\begin{verbatim}
Warning in grid.Call(C_textBounds, as.graphicsAnnot(x$label), x$x, x$y, :
conversion failure on 'Empirical Cumulative Density Function — Task Absolute
Score' in 'mbcsToSbcs': dot substituted for <80>
\end{verbatim}

\begin{verbatim}
Warning in grid.Call(C_textBounds, as.graphicsAnnot(x$label), x$x, x$y, :
conversion failure on 'Empirical Cumulative Density Function — Task Absolute
Score' in 'mbcsToSbcs': dot substituted for <94>
\end{verbatim}

\begin{verbatim}
Warning in grid.Call(C_textBounds, as.graphicsAnnot(x$label), x$x, x$y, :
conversion failure on 'Empirical Cumulative Density Function — Task Absolute
Score' in 'mbcsToSbcs': dot substituted for <e2>
\end{verbatim}

\begin{verbatim}
Warning in grid.Call(C_textBounds, as.graphicsAnnot(x$label), x$x, x$y, :
conversion failure on 'Empirical Cumulative Density Function — Task Absolute
Score' in 'mbcsToSbcs': dot substituted for <80>
\end{verbatim}

\begin{verbatim}
Warning in grid.Call(C_textBounds, as.graphicsAnnot(x$label), x$x, x$y, :
conversion failure on 'Empirical Cumulative Density Function — Task Absolute
Score' in 'mbcsToSbcs': dot substituted for <94>
\end{verbatim}

\begin{verbatim}
Warning in grid.Call.graphics(C_text, as.graphicsAnnot(x$label), x$x, x$y, :
conversion failure on 'Empirical Cumulative Density Function — Task Absolute
Score' in 'mbcsToSbcs': dot substituted for <e2>
\end{verbatim}

\begin{verbatim}
Warning in grid.Call.graphics(C_text, as.graphicsAnnot(x$label), x$x, x$y, :
conversion failure on 'Empirical Cumulative Density Function — Task Absolute
Score' in 'mbcsToSbcs': dot substituted for <80>
\end{verbatim}

\begin{verbatim}
Warning in grid.Call.graphics(C_text, as.graphicsAnnot(x$label), x$x, x$y, :
conversion failure on 'Empirical Cumulative Density Function — Task Absolute
Score' in 'mbcsToSbcs': dot substituted for <94>
\end{verbatim}

\begin{figure}[H]

{\centering \includegraphics{analysis/SGC3A/3_sgc3A_description_files/figure-pdf/VIS-SUBJ-ABS-4.pdf}

}

\end{figure}

Visual inspection of this distribution suggests it is not normal, and
likely bimodal. We verify this via an excess mass test
(Ameijeiras-Alsonso et. al 2018). TODO REFERENCE

\begin{Shaded}
\begin{Highlighting}[]
\NormalTok{multimode}\SpecialCharTok{::}\FunctionTok{modetest}\NormalTok{(df\_subjects}\SpecialCharTok{$}\NormalTok{s\_NABS)}
\end{Highlighting}
\end{Shaded}

\begin{verbatim}
Warning in multimode::modetest(df_subjects$s_NABS): A modification of the data
was made in order to compute the excess mass or the dip statistic
\end{verbatim}

\begin{verbatim}

    Ameijeiras-Alonso et al. (2019) excess mass test

data:  df_subjects$s_NABS
Excess mass = 0.1, p-value <2e-16
alternative hypothesis: true number of modes is greater than 1
\end{verbatim}

\begin{Shaded}
\begin{Highlighting}[]
\NormalTok{n\_modes }\OtherTok{=}\NormalTok{ multimode}\SpecialCharTok{::}\FunctionTok{nmodes}\NormalTok{(df\_subjects}\SpecialCharTok{$}\NormalTok{s\_NABS, }\AttributeTok{bw=}\DecValTok{2}\NormalTok{) }\CommentTok{\#bw = 2questions/15 = 0.15\%}
\NormalTok{l\_modes }\OtherTok{=}\NormalTok{ multimode}\SpecialCharTok{::}\FunctionTok{locmodes}\NormalTok{(df\_subjects}\SpecialCharTok{$}\NormalTok{s\_NABS,}\AttributeTok{mod0 =}\NormalTok{  n\_modes, }\AttributeTok{display =} \ConstantTok{TRUE}\NormalTok{)}
\end{Highlighting}
\end{Shaded}

\begin{verbatim}
Warning in multimode::locmodes(df_subjects$s_NABS, mod0 = n_modes, display =
TRUE): If the density function has an unbounded support, artificial modes may
have been created in the tails
\end{verbatim}

\begin{figure}[H]

{\centering \includegraphics{analysis/SGC3A/3_sgc3A_description_files/figure-pdf/CHECK-SUBJ-ABS-1.pdf}

}

\end{figure}

The excess mass test for multimodality suggests the distribution is
infact multimodal (m = 0.1, p \textless{} 0.001), with two identifiable
modes at 0.26 and 12.261, and an antimode at 6.985.

\begin{tcolorbox}[standard jigsaw,bottomrule=.15mm, opacitybacktitle=0.6, bottomtitle=1mm, toptitle=1mm, titlerule=0mm, title=\textcolor{quarto-callout-note-color}{\faInfo}\hspace{0.5em}{Note}, toprule=.15mm, rightrule=.15mm, colback=white, arc=.35mm, left=2mm, colframe=quarto-callout-note-color-frame, coltitle=black, leftrule=.75mm, opacityback=0, colbacktitle=quarto-callout-note-color!10!white]
\textbf{Condition appears (through visual inspection) to yield a
positive influence on Total Absolute Score (across the entire task),
across data collection modalities.}
\end{tcolorbox}

\hypertarget{scaled-score-1}{%
\paragraph{Scaled Score}\label{scaled-score-1}}

\begin{Shaded}
\begin{Highlighting}[]
\NormalTok{title }\OtherTok{=} \StringTok{"Descriptive Statistics of Response Accuracy (Total Scaled Score)"}
\NormalTok{scaled.stats }\OtherTok{\textless{}{-}} \FunctionTok{rbind}\NormalTok{(}
  \StringTok{"lab"}\OtherTok{=}\NormalTok{ df\_lab }\SpecialCharTok{\%\textgreater{}\%}\NormalTok{ dplyr}\SpecialCharTok{::}\FunctionTok{select}\NormalTok{(s\_SCALED) }\SpecialCharTok{\%\textgreater{}\%} \FunctionTok{unlist}\NormalTok{() }\SpecialCharTok{\%\textgreater{}\%} \FunctionTok{favstats}\NormalTok{(),}
  \StringTok{"online"} \OtherTok{=}\NormalTok{ df\_online }\SpecialCharTok{\%\textgreater{}\%}\NormalTok{ dplyr}\SpecialCharTok{::}\FunctionTok{select}\NormalTok{(s\_SCALED) }\SpecialCharTok{\%\textgreater{}\%} \FunctionTok{unlist}\NormalTok{() }\SpecialCharTok{\%\textgreater{}\%} \FunctionTok{favstats}\NormalTok{(),}
  \StringTok{"combined"} \OtherTok{=}\NormalTok{ df\_subjects }\SpecialCharTok{\%\textgreater{}\%}\NormalTok{ dplyr}\SpecialCharTok{::}\FunctionTok{select}\NormalTok{(s\_SCALED) }\SpecialCharTok{\%\textgreater{}\%} \FunctionTok{unlist}\NormalTok{() }\SpecialCharTok{\%\textgreater{}\%} \FunctionTok{favstats}\NormalTok{()}
\NormalTok{) }
\NormalTok{scaled.stats }\SpecialCharTok{\%\textgreater{}\%} \FunctionTok{kbl}\NormalTok{ (}\AttributeTok{caption =}\NormalTok{ title) }\SpecialCharTok{\%\textgreater{}\%} \FunctionTok{kable\_classic}\NormalTok{()}
\end{Highlighting}
\end{Shaded}

\begin{table}

\caption{Descriptive Statistics of Response Accuracy (Total Scaled Score)}
\centering
\begin{tabular}[t]{l|r|r|r|r|r|r|r|r|r}
\hline
  & min & Q1 & median & Q3 & max & mean & sd & n & missing\\
\hline
lab & -13 & -12.0 & -7.50 & 8.75 & 13 & -2.70 & 10.08 & 126 & 0\\
\hline
online & -13 & -10.0 & -7.00 & 6.62 & 13 & -2.83 & 9.26 & 204 & 0\\
\hline
combined & -13 & -10.5 & -7.25 & 7.50 & 13 & -2.78 & 9.56 & 330 & 0\\
\hline
\end{tabular}
\end{table}

For \textbf{in person collection}, total scaled scores (n = 126) range
from -13 to 13 with a mean score of (M = -2.7, SD = 10.08).

For \textbf{online replication}, total scaled scores (n = 204) range
from -13 to 13 with a slightly lower mean score of (M = -2.83, SD =
9.26).

When combined \textbf{overall}, total scaled scores (n = 330) range from
-13 to 13 with a slightly lower mean score of (M = -2.78, SD = 9.56).

\begin{Shaded}
\begin{Highlighting}[]
\CommentTok{\#GGFORMULA | DENSITY HISTOGRAM SUBJECT TOTAL SCALED}
\FunctionTok{gf\_props}\NormalTok{(}\SpecialCharTok{\textasciitilde{}}\NormalTok{s\_SCALED, }\AttributeTok{data =}\NormalTok{ df\_subjects) }\SpecialCharTok{+}
  \FunctionTok{labs}\NormalTok{(}\AttributeTok{x =} \StringTok{"total scaled score"}\NormalTok{,}
       \AttributeTok{y =} \StringTok{"\% of subjects"}\NormalTok{,}
       \AttributeTok{title =} \StringTok{"Distribution of Total Scaled Score"}\NormalTok{,}
       \AttributeTok{subtitle =} \StringTok{"Modes at high and low ends of scale suggest concentration of high (vs) low understanding"}\NormalTok{) }\SpecialCharTok{+} 
  \FunctionTok{theme\_minimal}\NormalTok{()}
\end{Highlighting}
\end{Shaded}

\begin{figure}[H]

{\centering \includegraphics{analysis/SGC3A/3_sgc3A_description_files/figure-pdf/VIS-SUBJ-SCALED-1.pdf}

}

\end{figure}

\begin{Shaded}
\begin{Highlighting}[]
\DocumentationTok{\#\#GGPUBR | HIST+DENSITY SCORE BY CONDITION/MODE}
\NormalTok{p }\OtherTok{\textless{}{-}} \FunctionTok{gghistogram}\NormalTok{(df\_subjects, }\AttributeTok{x =} \StringTok{"s\_SCALED"}\NormalTok{,}\AttributeTok{binwidth=}\DecValTok{1}\NormalTok{,}
   \AttributeTok{add =} \StringTok{"mean"}\NormalTok{, }\AttributeTok{rug =} \ConstantTok{TRUE}\NormalTok{,}
   \AttributeTok{fill =} \StringTok{"pretty\_condition"}\NormalTok{, }\CommentTok{\#, palette = c("\#00AFBB", "\#E7B800"),}
   \AttributeTok{add\_density =} \ConstantTok{TRUE}\NormalTok{) }
\FunctionTok{facet}\NormalTok{(p, }\AttributeTok{facet.by=}\FunctionTok{c}\NormalTok{(}\StringTok{"pretty\_condition"}\NormalTok{,}\StringTok{"pretty\_mode"}\NormalTok{)) }\SpecialCharTok{+} 
  \FunctionTok{labs}\NormalTok{( }\AttributeTok{title =} \StringTok{"Distribution of Total Scaled Score (by Condition and Mode)"}\NormalTok{,}
        \AttributeTok{subtitle =}\StringTok{"Pattern of response is similar across data collection modes but differs by condition"}\NormalTok{,}
        \AttributeTok{x =} \StringTok{"total scaled score"}\NormalTok{, }\AttributeTok{y =} \StringTok{"number of participants"}\NormalTok{) }\SpecialCharTok{+} 
  \FunctionTok{theme\_minimal}\NormalTok{() }\SpecialCharTok{+} \FunctionTok{theme}\NormalTok{(}\AttributeTok{legend.position =} \StringTok{"blank"}\NormalTok{) }
\end{Highlighting}
\end{Shaded}

\begin{figure}[H]

{\centering \includegraphics{analysis/SGC3A/3_sgc3A_description_files/figure-pdf/VIS-SUBJ-SCALED-2.pdf}

}

\end{figure}

\begin{Shaded}
\begin{Highlighting}[]
\DocumentationTok{\#\#RAINCLOUD USING GGDISTR}
\FunctionTok{ggplot}\NormalTok{(df\_subjects, }\FunctionTok{aes}\NormalTok{(}\AttributeTok{x =}\NormalTok{ pretty\_condition, }\AttributeTok{y =}\NormalTok{ s\_SCALED, }\AttributeTok{fill =}\NormalTok{ pretty\_condition)) }\SpecialCharTok{+} 
\NormalTok{  ggdist}\SpecialCharTok{::}\FunctionTok{stat\_halfeye}\NormalTok{(}
    \AttributeTok{adjust =}\NormalTok{ .}\DecValTok{5}\NormalTok{, }
    \AttributeTok{width =}\NormalTok{ .}\DecValTok{6}\NormalTok{, }
    \AttributeTok{.width =} \DecValTok{0}\NormalTok{, }
    \AttributeTok{justification =} \SpecialCharTok{{-}}\NormalTok{.}\DecValTok{3}\NormalTok{, }
    \AttributeTok{point\_colour =} \ConstantTok{NA}\NormalTok{) }\SpecialCharTok{+} 
  \FunctionTok{geom\_boxplot}\NormalTok{(}
    \AttributeTok{width =}\NormalTok{ .}\DecValTok{15}\NormalTok{, }
    \AttributeTok{outlier.shape =} \ConstantTok{NA}
\NormalTok{  ) }\SpecialCharTok{+}
  \FunctionTok{geom\_point}\NormalTok{(}
    \AttributeTok{size =} \FloatTok{1.3}\NormalTok{,}
    \AttributeTok{alpha =}\NormalTok{ .}\DecValTok{3}\NormalTok{,}
    \AttributeTok{position =} \FunctionTok{position\_jitter}\NormalTok{(}
      \AttributeTok{seed =} \DecValTok{1}\NormalTok{, }\AttributeTok{width =}\NormalTok{ .}\DecValTok{1}
\NormalTok{    )}
\NormalTok{  ) }\SpecialCharTok{+} \FunctionTok{labs}\NormalTok{(}
    \AttributeTok{title =} \StringTok{"Distribution of Task Scaled Score "}\NormalTok{,}
    \AttributeTok{x =} \StringTok{"Condition"}\NormalTok{, }\AttributeTok{y =} \StringTok{"Total Scaled Score"}
\NormalTok{  ) }\SpecialCharTok{+} \FunctionTok{theme\_ggdist}\NormalTok{() }\SpecialCharTok{+} \FunctionTok{theme}\NormalTok{(}\AttributeTok{legend.position =} \StringTok{"blank"}\NormalTok{)}
\end{Highlighting}
\end{Shaded}

\begin{figure}[H]

{\centering \includegraphics{analysis/SGC3A/3_sgc3A_description_files/figure-pdf/VIS-SUBJ-SCALED-3.pdf}

}

\end{figure}

\begin{Shaded}
\begin{Highlighting}[]
\CommentTok{\# + coord\_cartesian(xlim = c(1.2, NA), clip = "off")}

\CommentTok{\#PLOT EMPIRICIAL CUMULATIVE DISTRIBUTION FUNCTION}
\FunctionTok{ggplot}\NormalTok{(}\AttributeTok{data =}\NormalTok{ df\_subjects, }\FunctionTok{aes}\NormalTok{(s\_SCALED)) }\SpecialCharTok{+} 
  \FunctionTok{stat\_ecdf}\NormalTok{(}\AttributeTok{geom =} \StringTok{"step"}\NormalTok{) }\SpecialCharTok{+} 
  \FunctionTok{facet\_grid}\NormalTok{(pretty\_condition }\SpecialCharTok{\textasciitilde{}}\NormalTok{ pretty\_mode) }\SpecialCharTok{+} 
  \FunctionTok{labs}\NormalTok{( }\AttributeTok{title =} \StringTok{"Empirical Cumulative Density Function — Task Scaled Score"}\NormalTok{,}
        \AttributeTok{x =} \StringTok{"Task Scaled Score [{-}13, 13]"}\NormalTok{, }
        \AttributeTok{y =} \StringTok{"Cumulative Probability"}\NormalTok{) }\SpecialCharTok{+} \FunctionTok{theme\_minimal}\NormalTok{()}
\end{Highlighting}
\end{Shaded}

\begin{verbatim}
Warning in grid.Call(C_textBounds, as.graphicsAnnot(x$label), x$x, x$y, :
conversion failure on 'Empirical Cumulative Density Function — Task Scaled
Score' in 'mbcsToSbcs': dot substituted for <e2>
\end{verbatim}

\begin{verbatim}
Warning in grid.Call(C_textBounds, as.graphicsAnnot(x$label), x$x, x$y, :
conversion failure on 'Empirical Cumulative Density Function — Task Scaled
Score' in 'mbcsToSbcs': dot substituted for <80>
\end{verbatim}

\begin{verbatim}
Warning in grid.Call(C_textBounds, as.graphicsAnnot(x$label), x$x, x$y, :
conversion failure on 'Empirical Cumulative Density Function — Task Scaled
Score' in 'mbcsToSbcs': dot substituted for <94>
\end{verbatim}

\begin{verbatim}
Warning in grid.Call(C_textBounds, as.graphicsAnnot(x$label), x$x, x$y, :
conversion failure on 'Empirical Cumulative Density Function — Task Scaled
Score' in 'mbcsToSbcs': dot substituted for <e2>
\end{verbatim}

\begin{verbatim}
Warning in grid.Call(C_textBounds, as.graphicsAnnot(x$label), x$x, x$y, :
conversion failure on 'Empirical Cumulative Density Function — Task Scaled
Score' in 'mbcsToSbcs': dot substituted for <80>
\end{verbatim}

\begin{verbatim}
Warning in grid.Call(C_textBounds, as.graphicsAnnot(x$label), x$x, x$y, :
conversion failure on 'Empirical Cumulative Density Function — Task Scaled
Score' in 'mbcsToSbcs': dot substituted for <94>
\end{verbatim}

\begin{verbatim}
Warning in grid.Call(C_textBounds, as.graphicsAnnot(x$label), x$x, x$y, :
conversion failure on 'Empirical Cumulative Density Function — Task Scaled
Score' in 'mbcsToSbcs': dot substituted for <e2>
\end{verbatim}

\begin{verbatim}
Warning in grid.Call(C_textBounds, as.graphicsAnnot(x$label), x$x, x$y, :
conversion failure on 'Empirical Cumulative Density Function — Task Scaled
Score' in 'mbcsToSbcs': dot substituted for <80>
\end{verbatim}

\begin{verbatim}
Warning in grid.Call(C_textBounds, as.graphicsAnnot(x$label), x$x, x$y, :
conversion failure on 'Empirical Cumulative Density Function — Task Scaled
Score' in 'mbcsToSbcs': dot substituted for <94>
\end{verbatim}

\begin{verbatim}
Warning in grid.Call(C_textBounds, as.graphicsAnnot(x$label), x$x, x$y, :
conversion failure on 'Empirical Cumulative Density Function — Task Scaled
Score' in 'mbcsToSbcs': dot substituted for <e2>
\end{verbatim}

\begin{verbatim}
Warning in grid.Call(C_textBounds, as.graphicsAnnot(x$label), x$x, x$y, :
conversion failure on 'Empirical Cumulative Density Function — Task Scaled
Score' in 'mbcsToSbcs': dot substituted for <80>
\end{verbatim}

\begin{verbatim}
Warning in grid.Call(C_textBounds, as.graphicsAnnot(x$label), x$x, x$y, :
conversion failure on 'Empirical Cumulative Density Function — Task Scaled
Score' in 'mbcsToSbcs': dot substituted for <94>
\end{verbatim}

\begin{verbatim}
Warning in grid.Call(C_textBounds, as.graphicsAnnot(x$label), x$x, x$y, :
conversion failure on 'Empirical Cumulative Density Function — Task Scaled
Score' in 'mbcsToSbcs': dot substituted for <e2>
\end{verbatim}

\begin{verbatim}
Warning in grid.Call(C_textBounds, as.graphicsAnnot(x$label), x$x, x$y, :
conversion failure on 'Empirical Cumulative Density Function — Task Scaled
Score' in 'mbcsToSbcs': dot substituted for <80>
\end{verbatim}

\begin{verbatim}
Warning in grid.Call(C_textBounds, as.graphicsAnnot(x$label), x$x, x$y, :
conversion failure on 'Empirical Cumulative Density Function — Task Scaled
Score' in 'mbcsToSbcs': dot substituted for <94>
\end{verbatim}

\begin{verbatim}
Warning in grid.Call(C_textBounds, as.graphicsAnnot(x$label), x$x, x$y, :
conversion failure on 'Empirical Cumulative Density Function — Task Scaled
Score' in 'mbcsToSbcs': dot substituted for <e2>
\end{verbatim}

\begin{verbatim}
Warning in grid.Call(C_textBounds, as.graphicsAnnot(x$label), x$x, x$y, :
conversion failure on 'Empirical Cumulative Density Function — Task Scaled
Score' in 'mbcsToSbcs': dot substituted for <80>
\end{verbatim}

\begin{verbatim}
Warning in grid.Call(C_textBounds, as.graphicsAnnot(x$label), x$x, x$y, :
conversion failure on 'Empirical Cumulative Density Function — Task Scaled
Score' in 'mbcsToSbcs': dot substituted for <94>
\end{verbatim}

\begin{verbatim}
Warning in grid.Call(C_textBounds, as.graphicsAnnot(x$label), x$x, x$y, :
conversion failure on 'Empirical Cumulative Density Function — Task Scaled
Score' in 'mbcsToSbcs': dot substituted for <e2>
\end{verbatim}

\begin{verbatim}
Warning in grid.Call(C_textBounds, as.graphicsAnnot(x$label), x$x, x$y, :
conversion failure on 'Empirical Cumulative Density Function — Task Scaled
Score' in 'mbcsToSbcs': dot substituted for <80>
\end{verbatim}

\begin{verbatim}
Warning in grid.Call(C_textBounds, as.graphicsAnnot(x$label), x$x, x$y, :
conversion failure on 'Empirical Cumulative Density Function — Task Scaled
Score' in 'mbcsToSbcs': dot substituted for <94>
\end{verbatim}

\begin{verbatim}
Warning in grid.Call.graphics(C_text, as.graphicsAnnot(x$label), x$x, x$y, :
conversion failure on 'Empirical Cumulative Density Function — Task Scaled
Score' in 'mbcsToSbcs': dot substituted for <e2>
\end{verbatim}

\begin{verbatim}
Warning in grid.Call.graphics(C_text, as.graphicsAnnot(x$label), x$x, x$y, :
conversion failure on 'Empirical Cumulative Density Function — Task Scaled
Score' in 'mbcsToSbcs': dot substituted for <80>
\end{verbatim}

\begin{verbatim}
Warning in grid.Call.graphics(C_text, as.graphicsAnnot(x$label), x$x, x$y, :
conversion failure on 'Empirical Cumulative Density Function — Task Scaled
Score' in 'mbcsToSbcs': dot substituted for <94>
\end{verbatim}

\begin{figure}[H]

{\centering \includegraphics{analysis/SGC3A/3_sgc3A_description_files/figure-pdf/VIS-SUBJ-SCALED-4.pdf}

}

\end{figure}

Visual inspection of this distribution suggests it is not normal, and
perhaps perhaps bimodal. We verify this via an excess mass test
(Ameijeiras-Alsonso et. al 2018).

\begin{Shaded}
\begin{Highlighting}[]
\NormalTok{multimode}\SpecialCharTok{::}\FunctionTok{modetest}\NormalTok{(df\_subjects}\SpecialCharTok{$}\NormalTok{s\_SCALED)}
\end{Highlighting}
\end{Shaded}

\begin{verbatim}
Warning in multimode::modetest(df_subjects$s_SCALED): A modification of the data
was made in order to compute the excess mass or the dip statistic
\end{verbatim}

\begin{verbatim}

    Ameijeiras-Alonso et al. (2019) excess mass test

data:  df_subjects$s_SCALED
Excess mass = 0.1, p-value <2e-16
alternative hypothesis: true number of modes is greater than 1
\end{verbatim}

\begin{Shaded}
\begin{Highlighting}[]
\NormalTok{n\_modes }\OtherTok{=}\NormalTok{ multimode}\SpecialCharTok{::}\FunctionTok{nmodes}\NormalTok{(df\_subjects}\SpecialCharTok{$}\NormalTok{s\_SCALED, }\AttributeTok{bw=}\DecValTok{2}\NormalTok{) }\CommentTok{\#bw = 2questions/15 = 0.15\%}
\NormalTok{l\_modes }\OtherTok{=}\NormalTok{ multimode}\SpecialCharTok{::}\FunctionTok{locmodes}\NormalTok{(df\_subjects}\SpecialCharTok{$}\NormalTok{s\_SCALED,}\AttributeTok{mod0 =}\NormalTok{  n\_modes, }\AttributeTok{display =} \ConstantTok{TRUE}\NormalTok{)}
\end{Highlighting}
\end{Shaded}

\begin{verbatim}
Warning in multimode::locmodes(df_subjects$s_SCALED, mod0 = n_modes, display =
TRUE): If the density function has an unbounded support, artificial modes may
have been created in the tails
\end{verbatim}

\begin{figure}[H]

{\centering \includegraphics{analysis/SGC3A/3_sgc3A_description_files/figure-pdf/CHECK-SUBJ-SCALED-1.pdf}

}

\end{figure}

The excess mass test for multimodality suggests the distribution is in
fact multimodal (m = 0.1, p \textless{} 0.001), with two identifiable
modes at -11.195 and 12.103, and an antimode at 2.942.

\textbf{Analysis Notes} - As with absolute score, the distribution of
scaled score is very bimodal - Same need to investigate transformations
and alternative distributions for regression

\begin{tcolorbox}[standard jigsaw,bottomrule=.15mm, opacitybacktitle=0.6, bottomtitle=1mm, toptitle=1mm, titlerule=0mm, title=\textcolor{quarto-callout-note-color}{\faInfo}\hspace{0.5em}{Note}, toprule=.15mm, rightrule=.15mm, colback=white, arc=.35mm, left=2mm, colframe=quarto-callout-note-color-frame, coltitle=black, leftrule=.75mm, opacityback=0, colbacktitle=quarto-callout-note-color!10!white]
\textbf{Condition appears (through visual inspection) to yield a
positive influence on Total Scaled Score across data collection
modalities.}
\end{tcolorbox}

\hypertarget{item-level-scores}{%
\subsection{Item Level Scores}\label{item-level-scores}}

\hypertarget{item-absolute-score}{%
\subsubsection{Item Absolute Score}\label{item-absolute-score}}

Whote Task Accuracy summarized over items rather than subjects

\begin{Shaded}
\begin{Highlighting}[]
\NormalTok{x }\OtherTok{\textless{}{-}}\NormalTok{ df\_items }\SpecialCharTok{\%\textgreater{}\%} \FunctionTok{mutate}\NormalTok{(}\AttributeTok{score =} \FunctionTok{as.logical}\NormalTok{(score\_ABS))}

\NormalTok{title }\OtherTok{=} \StringTok{"Proportion of Correct Items By Condition (Lab)"}

\NormalTok{item.contingency }\OtherTok{\textless{}{-}}\NormalTok{ df\_items }\SpecialCharTok{\%\textgreater{}\%} \FunctionTok{filter}\NormalTok{(mode }\SpecialCharTok{==} \StringTok{"lab{-}synch"}\NormalTok{) }\SpecialCharTok{\%\textgreater{}\%}\NormalTok{ dplyr}\SpecialCharTok{::}\FunctionTok{select}\NormalTok{(score\_ABS, condition) }\SpecialCharTok{\%\textgreater{}\%} \FunctionTok{table}\NormalTok{() }\SpecialCharTok{\%\textgreater{}\%} \FunctionTok{prop.table}\NormalTok{() }\SpecialCharTok{\%\textgreater{}\%} \FunctionTok{addmargins}\NormalTok{()}
\NormalTok{item.contingency }\SpecialCharTok{\%\textgreater{}\%} \FunctionTok{kbl}\NormalTok{ (}\AttributeTok{caption =}\NormalTok{ title) }\SpecialCharTok{\%\textgreater{}\%} \FunctionTok{kable\_classic}\NormalTok{()}
\end{Highlighting}
\end{Shaded}

\begin{table}

\caption{Proportion of Correct Items By Condition (Lab)}
\centering
\begin{tabular}[t]{l|r|r|r}
\hline
  & 111 & 121 & Sum\\
\hline
0 & 0.344 & 0.268 & 0.613\\
\hline
1 & 0.148 & 0.240 & 0.387\\
\hline
Sum & 0.492 & 0.508 & 1.000\\
\hline
\end{tabular}
\end{table}

\begin{Shaded}
\begin{Highlighting}[]
\NormalTok{title }\OtherTok{=} \StringTok{"Proportion of Correct Items By Condition (Online)"}
\NormalTok{item.contingency }\OtherTok{\textless{}{-}}\NormalTok{ df\_items }\SpecialCharTok{\%\textgreater{}\%} \FunctionTok{filter}\NormalTok{(mode }\SpecialCharTok{==} \StringTok{"asynch"}\NormalTok{) }\SpecialCharTok{\%\textgreater{}\%}\NormalTok{ dplyr}\SpecialCharTok{::}\FunctionTok{select}\NormalTok{(score\_ABS, condition) }\SpecialCharTok{\%\textgreater{}\%} \FunctionTok{table}\NormalTok{() }\SpecialCharTok{\%\textgreater{}\%} \FunctionTok{prop.table}\NormalTok{() }\SpecialCharTok{\%\textgreater{}\%} \FunctionTok{addmargins}\NormalTok{()}
\NormalTok{item.contingency }\SpecialCharTok{\%\textgreater{}\%} \FunctionTok{kbl}\NormalTok{ (}\AttributeTok{caption =}\NormalTok{ title) }\SpecialCharTok{\%\textgreater{}\%} \FunctionTok{kable\_classic}\NormalTok{()}
\end{Highlighting}
\end{Shaded}

\begin{table}

\caption{Proportion of Correct Items By Condition (Online)}
\centering
\begin{tabular}[t]{l|r|r|r}
\hline
  & 111 & 121 & Sum\\
\hline
0 & 0.342 & 0.307 & 0.649\\
\hline
1 & 0.128 & 0.223 & 0.351\\
\hline
Sum & 0.471 & 0.529 & 1.000\\
\hline
\end{tabular}
\end{table}

\begin{Shaded}
\begin{Highlighting}[]
\CommentTok{\#VISUALIZE distribution of response accuracy across ITEMS}

\CommentTok{\#HISTOGRAM}
\NormalTok{stats }\OtherTok{=}\NormalTok{ df\_items }\SpecialCharTok{\%\textgreater{}\%} \FunctionTok{group\_by}\NormalTok{(condition, mode) }\SpecialCharTok{\%\textgreater{}\%}\NormalTok{ dplyr}\SpecialCharTok{::}\FunctionTok{summarise}\NormalTok{(}\AttributeTok{mean =} \FunctionTok{mean}\NormalTok{(score\_niceABS))}
\FunctionTok{gf\_props}\NormalTok{(}\SpecialCharTok{\textasciitilde{}}\NormalTok{score\_niceABS, }\AttributeTok{data =}\NormalTok{ df\_items) }\SpecialCharTok{\%\textgreater{}\%} 
  \FunctionTok{gf\_facet\_grid}\NormalTok{(condition}\SpecialCharTok{\textasciitilde{}}\NormalTok{mode, }\AttributeTok{labeller =}\NormalTok{ label\_both) }\SpecialCharTok{+}
  \FunctionTok{labs}\NormalTok{(}\AttributeTok{x =} \StringTok{"Item Absolute Score"}\NormalTok{,}
       \AttributeTok{title =} \StringTok{"Item Absolute Score"}\NormalTok{,}
       \AttributeTok{subtitle=}\StringTok{"Across modalities, the impasse condition yielded more correct responses"}\NormalTok{)}\SpecialCharTok{+}
  \FunctionTok{theme\_minimal}\NormalTok{()}
\end{Highlighting}
\end{Shaded}

\begin{figure}[H]

{\centering \includegraphics{analysis/SGC3A/3_sgc3A_description_files/figure-pdf/unnamed-chunk-36-1.pdf}

}

\end{figure}

\hypertarget{item-scaled-score}{%
\subsubsection{Item Scaled Score}\label{item-scaled-score}}

At the item level, the scaled score gives us a numeric measure of
correctness of interpretation, ranging from -1 to 1.

\begin{Shaded}
\begin{Highlighting}[]
\NormalTok{title }\OtherTok{=} \StringTok{"Descriptive Statistics of Item Response Accuracy (Scaled Score)"}
\NormalTok{scaled.stats.items }\OtherTok{\textless{}{-}} \FunctionTok{rbind}\NormalTok{(}
  \StringTok{"lab"}\OtherTok{=}\NormalTok{ df\_items }\SpecialCharTok{\%\textgreater{}\%} \FunctionTok{filter}\NormalTok{(mode }\SpecialCharTok{==} \StringTok{\textquotesingle{}lab{-}synch\textquotesingle{}}\NormalTok{) }\SpecialCharTok{\%\textgreater{}\%}\NormalTok{ dplyr}\SpecialCharTok{::}\FunctionTok{select}\NormalTok{(score\_SCALED) }\SpecialCharTok{\%\textgreater{}\%} \FunctionTok{unlist}\NormalTok{() }\SpecialCharTok{\%\textgreater{}\%} \FunctionTok{favstats}\NormalTok{(),}
  \StringTok{"online"} \OtherTok{=}\NormalTok{ df\_items }\SpecialCharTok{\%\textgreater{}\%} \FunctionTok{filter}\NormalTok{(mode }\SpecialCharTok{==} \StringTok{"asynch"}\NormalTok{) }\SpecialCharTok{\%\textgreater{}\%}\NormalTok{ dplyr}\SpecialCharTok{::}\FunctionTok{select}\NormalTok{(score\_SCALED) }\SpecialCharTok{\%\textgreater{}\%} \FunctionTok{unlist}\NormalTok{() }\SpecialCharTok{\%\textgreater{}\%} \FunctionTok{favstats}\NormalTok{()}
\NormalTok{) }
\NormalTok{scaled.stats.items }\SpecialCharTok{\%\textgreater{}\%} \FunctionTok{kbl}\NormalTok{ (}\AttributeTok{caption =}\NormalTok{ title) }\SpecialCharTok{\%\textgreater{}\%} \FunctionTok{kable\_classic}\NormalTok{()}
\end{Highlighting}
\end{Shaded}

\begin{table}

\caption{Descriptive Statistics of Item Response Accuracy (Scaled Score)}
\centering
\begin{tabular}[t]{l|r|r|r|r|r|r|r|r|r}
\hline
  & min & Q1 & median & Q3 & max & mean & sd & n & missing\\
\hline
lab & -1 & -1 & -0.5 & 1 & 1 & -0.128 & 0.877 & 1890 & 0\\
\hline
online & -1 & -1 & -0.5 & 1 & 1 & -0.136 & 0.842 & 3060 & 0\\
\hline
\end{tabular}
\end{table}

\begin{Shaded}
\begin{Highlighting}[]
\CommentTok{\#VISUALIZE distribution of response accuracy across ITEMS}

\CommentTok{\#HISTOGRAM}
\NormalTok{stats }\OtherTok{=}\NormalTok{ df\_items }\SpecialCharTok{\%\textgreater{}\%} \FunctionTok{group\_by}\NormalTok{(condition, mode) }\SpecialCharTok{\%\textgreater{}\%}\NormalTok{ dplyr}\SpecialCharTok{::}\FunctionTok{summarise}\NormalTok{(}\AttributeTok{mean =} \FunctionTok{mean}\NormalTok{(score\_SCALED))}
\FunctionTok{gf\_props}\NormalTok{(}\SpecialCharTok{\textasciitilde{}}\NormalTok{score\_SCALED, }\AttributeTok{data =}\NormalTok{ df\_items) }\SpecialCharTok{\%\textgreater{}\%} 
  \FunctionTok{gf\_facet\_grid}\NormalTok{(condition}\SpecialCharTok{\textasciitilde{}}\NormalTok{mode, }\AttributeTok{labeller =}\NormalTok{ label\_both) }\SpecialCharTok{\%\textgreater{}\%} 
  \FunctionTok{gf\_vline}\NormalTok{(}\AttributeTok{data =}\NormalTok{ stats, }\AttributeTok{xintercept =} \SpecialCharTok{\textasciitilde{}}\NormalTok{mean, }\AttributeTok{color =} \StringTok{"red"}\NormalTok{) }\SpecialCharTok{+}
  \FunctionTok{labs}\NormalTok{(}\AttributeTok{x =} \StringTok{"Scaled Score for Item"}\NormalTok{,}
       \AttributeTok{y =} \StringTok{"Proportion of Items"}\NormalTok{,}
       \AttributeTok{title =} \StringTok{"Distribution of Accuracy per Item (Scale Score)"}\NormalTok{,}
       \AttributeTok{subtitle=}\StringTok{"The impasse condition shifts density toward the positive score"}\NormalTok{)}\SpecialCharTok{+}
  \FunctionTok{theme\_minimal}\NormalTok{()}
\end{Highlighting}
\end{Shaded}

\begin{figure}[H]

{\centering \includegraphics{analysis/SGC3A/3_sgc3A_description_files/figure-pdf/unnamed-chunk-38-1.pdf}

}

\end{figure}

\newpage

\hypertarget{sec-SGC3A-hypotesting}{%
\chapter{Hypothesis Testing}\label{sec-SGC3A-hypotesting}}

\textbf{TODO}

\begin{itemize}
\tightlist
\item
  HURDLE MODEL? (mixture model w/ 0 + count)
\item
  consider zero-inflated (poisson or neg binom) with \_rt as predictor
  of count process and condition as predictor of excess zeros
\item
  review models already created in ARCHIVE?
\item
  explore response consistency
\end{itemize}

\emph{The purpose of this notebook is test the hypotheses that
determined the design of the SGC3A study.}

\begin{longtable}[]{@{}
  >{\raggedright\arraybackslash}p{(\columnwidth - 0\tabcolsep) * \real{0.3056}}@{}}
\toprule()
\begin{minipage}[b]{\linewidth}\raggedright
Pre-Requisite
\end{minipage} \\
\midrule()
\endhead
2\_sgc3A\_scoring.qmd \\
\bottomrule()
\end{longtable}

\begin{Shaded}
\begin{Highlighting}[]
\FunctionTok{library}\NormalTok{(Hmisc) }\CommentTok{\# \%nin\% operator}

\FunctionTok{library}\NormalTok{(ggpubr) }\CommentTok{\#arrange plots}
\FunctionTok{library}\NormalTok{(ggformula) }\CommentTok{\#easy graphs}
\FunctionTok{library}\NormalTok{(vcd) }\CommentTok{\#mosaic plots}
\FunctionTok{library}\NormalTok{(vcdExtra) }\CommentTok{\#mosaic plots}
\FunctionTok{library}\NormalTok{(kableExtra) }\CommentTok{\#printing tables }

\FunctionTok{library}\NormalTok{(report) }\CommentTok{\#easystats reporting}
\FunctionTok{library}\NormalTok{(see) }\CommentTok{\#easystats visualization}
\FunctionTok{library}\NormalTok{(performance) }\CommentTok{\#easystats model diagnostics}
\FunctionTok{library}\NormalTok{(qqplotr) }\CommentTok{\#confint on qq plot}
\FunctionTok{library}\NormalTok{(gmodels) }\CommentTok{\#contingency table and CHISQR}
\FunctionTok{library}\NormalTok{(equatiomatic) }\CommentTok{\#extract model equation}
\FunctionTok{library}\NormalTok{(pscl) }\CommentTok{\#zeroinfl / hurdle models }

\FunctionTok{library}\NormalTok{(tidyverse) }\CommentTok{\#ALL THE THINGS}

\CommentTok{\#OUTPUT OPTIONS}
\FunctionTok{library}\NormalTok{(dplyr, }\AttributeTok{warn.conflicts =} \ConstantTok{FALSE}\NormalTok{)}
\FunctionTok{options}\NormalTok{(}\AttributeTok{dplyr.summarise.inform =} \ConstantTok{FALSE}\NormalTok{)}
\FunctionTok{options}\NormalTok{(}\AttributeTok{ggplot2.summarise.inform =} \ConstantTok{FALSE}\NormalTok{)}
\FunctionTok{options}\NormalTok{(}\AttributeTok{scipen=}\DecValTok{1}\NormalTok{, }\AttributeTok{digits=}\DecValTok{3}\NormalTok{)}

\CommentTok{\# Custom ggplot theme to make pretty plots}
\CommentTok{\# Get the font at https://fonts.google.com/specimen/Barlow+Semi+Condensed}
\CommentTok{\# theme\_clean \textless{}{-} function() \{}
\CommentTok{\#   theme\_minimal(base\_family = "Barlow Semi Condensed") +}
\CommentTok{\#     theme(panel.grid.minor = element\_blank(),}
\CommentTok{\#           plot.title = element\_text(family = "BarlowSemiCondensed{-}Bold"),}
\CommentTok{\#           axis.title = element\_text(family = "BarlowSemiCondensed{-}Medium"),}
\CommentTok{\#           strip.text = element\_text(family = "BarlowSemiCondensed{-}Bold",}
\CommentTok{\#                                     size = rel(1), hjust = 0),}
\CommentTok{\#           strip.background = element\_rect(fill = "grey80", color = NA))}
\CommentTok{\# \}}

\CommentTok{\# Make labels use Barlow by default}
\CommentTok{\# update\_geom\_defaults("label\_repel", list(family = "Barlow Semi Condensed"))}
\end{Highlighting}
\end{Shaded}

\textbf{Research Questions}

In SGC3A we set out to answer the following question: Does posing a
mental impasse improve performance on the graph comprehension task?

\textbf{Experimental Hypothesis}\\
\emph{Learners posed with scenario designed to evoke a mental impasse
will be more likely to correct interpret the graph.}

\begin{itemize}
\tightlist
\item
  H1A \textbar{} Learners in the IMPASSE condition will score higher on
  the TEST Phase than learners in CONTROL.
\item
  H1B \textbar{} Learners in the IMPASSE condition will be more likely
  to correctly answer the first question than learners in CONTROL.
\item
  H1C \textbar{} Learners in the IMPASSE condition will spend more time
  on the first question than learners in CONTROL.
\end{itemize}

\textbf{Null Hypothesis}\\
\emph{No significant differences in performance will exist between
learners in the IMPASSE and CONTROL conditions.}

\begin{Shaded}
\begin{Highlighting}[]
\CommentTok{\# }\AlertTok{HACK}\CommentTok{ WD FOR LOCAL RUNNING?}
\CommentTok{\# imac = "/Users/amyraefox/Code/SGC{-}Scaffolding\_Graph\_Comprehension/SGC{-}X/ANALYSIS/MAIN"}
\CommentTok{\# mbp = "/Users/amyfox/Sites/RESEARCH/SGC—Scaffolding Graph Comprehension/SGC{-}X/ANALYSIS/MAIN"}
\CommentTok{\# setwd(mbp)}

\CommentTok{\#IMPORT DATA }
\NormalTok{df\_items }\OtherTok{\textless{}{-}} \FunctionTok{read\_rds}\NormalTok{(}\StringTok{\textquotesingle{}analysis/SGC3A/data/2{-}scored{-}data/sgc3a\_scored\_items.rds\textquotesingle{}}\NormalTok{)}
\NormalTok{df\_subjects }\OtherTok{\textless{}{-}} \FunctionTok{read\_rds}\NormalTok{(}\StringTok{\textquotesingle{}analysis/SGC3A/data/2{-}scored{-}data/sgc3a\_scored\_participants.rds\textquotesingle{}}\NormalTok{)}

\CommentTok{\#TRANSFORMATIONS }
\CommentTok{\#1. test phase absolute score as percentage}
\NormalTok{df\_subjects }\OtherTok{\textless{}{-}}\NormalTok{ df\_subjects }\SpecialCharTok{\%\textgreater{}\%} \FunctionTok{mutate}\NormalTok{(}
  \AttributeTok{DV\_percent\_test\_NABS =}\NormalTok{ item\_test\_NABS}\SpecialCharTok{/}\DecValTok{8} \CommentTok{\#for 8 Qs in test phase}
\NormalTok{)}

\CommentTok{\#SEPARATE ITEM DATA BY QUESTION TYPE}
\NormalTok{df\_scaffold }\OtherTok{\textless{}{-}}\NormalTok{ df\_items }\SpecialCharTok{\%\textgreater{}\%} \FunctionTok{filter}\NormalTok{(q }\SpecialCharTok{\textless{}} \DecValTok{6}\NormalTok{)}
\NormalTok{df\_test }\OtherTok{\textless{}{-}}\NormalTok{ df\_items }\SpecialCharTok{\%\textgreater{}\%} \FunctionTok{filter}\NormalTok{(q }\SpecialCharTok{\textgreater{}} \DecValTok{6}\NormalTok{) }\SpecialCharTok{\%\textgreater{}\%} \FunctionTok{filter}\NormalTok{ (q }\SpecialCharTok{\%nin\%} \FunctionTok{c}\NormalTok{(}\DecValTok{6}\NormalTok{,}\DecValTok{9}\NormalTok{))}
\NormalTok{df\_nondiscrim }\OtherTok{\textless{}{-}}\NormalTok{ df\_items }\SpecialCharTok{\%\textgreater{}\%} \FunctionTok{filter}\NormalTok{ (q }\SpecialCharTok{\%in\%} \FunctionTok{c}\NormalTok{(}\DecValTok{6}\NormalTok{,}\DecValTok{9}\NormalTok{))}

\NormalTok{df\_lab }\OtherTok{\textless{}{-}}\NormalTok{ df\_subjects }\SpecialCharTok{\%\textgreater{}\%} \FunctionTok{filter}\NormalTok{(pretty\_mode }\SpecialCharTok{==} \StringTok{"laboratory"}\NormalTok{)}
\NormalTok{df\_online }\OtherTok{\textless{}{-}}\NormalTok{ df\_subjects }\SpecialCharTok{\%\textgreater{}\%} \FunctionTok{filter}\NormalTok{(pretty\_mode }\SpecialCharTok{==} \StringTok{"online{-}replication"}\NormalTok{)}
\end{Highlighting}
\end{Shaded}

\hypertarget{h1a-test-phase-accuracy}{%
\section{H1A \textbar{} TEST PHASE
ACCURACY}\label{h1a-test-phase-accuracy}}

On the TEST Phase of the graph comprehension task (the final 8
questions, encountered after the 5 scaffolded questions) does the
impasse condition affect performance on the graph comprehension task?

\begin{longtable}[]{@{}
  >{\raggedright\arraybackslash}p{(\columnwidth - 2\tabcolsep) * \real{0.1491}}
  >{\raggedright\arraybackslash}p{(\columnwidth - 2\tabcolsep) * \real{0.8509}}@{}}
\toprule()
\begin{minipage}[b]{\linewidth}\raggedright
Research Question
\end{minipage} & \begin{minipage}[b]{\linewidth}\raggedright
Does posing a mental impasse improve performance?
\end{minipage} \\
\midrule()
\endhead
\textbf{Hypothesis} & (H1A) Participants in the IMPASSE condition will
have significantly higher TEST PHASE performance than those in the
CONTROL condition. \\
\textbf{Analysis Strategy} & \begin{minipage}[t]{\linewidth}\raggedright
OLS Linear Regression \texttt{DV\_percent\_test\_NABS} \textasciitilde{}
\texttt{condition} (absolute scoring)\\
OLS Linear Regression \texttt{item\_test\_SCALED} \textasciitilde{}
\texttt{condition} (scaled scoring)\strut
\end{minipage} \\
\textbf{Alternatives} & \begin{minipage}[t]{\linewidth}\raggedright
\textbf{Exploring alternatives.}\\
\emph{Simple linear regression models do a poor job of fitting the
(bimodal) outcome distributions (both absolute and scaled scores)}

\begin{itemize}
\tightlist
\item
  Hurdle model (mixture model w/ binomial + count)
\item
  Negative Binomial / Zero Inflated Negative Binom for overdispersed
  count?
\item
  Beta regression?
\item
  Other way to account for the severe bimodality?
\item
  ``shift function'' way to characterize difference in bimodal
  distributions
\end{itemize}\strut
\end{minipage} \\
\textbf{Inference} & \textbf{\emph{TODO}} \textbf{when done} \\
\bottomrule()
\end{longtable}

\hypertarget{test-phase-absolute-score-1}{%
\subsection{Test Phase Absolute
Score}\label{test-phase-absolute-score-1}}

\hypertarget{linear-regression}{%
\subsubsection{Linear Regression}\label{linear-regression}}

\hypertarget{in-person}{%
\paragraph{(In Person)}\label{in-person}}

\hypertarget{visualization}{%
\subparagraph{Visualization}\label{visualization}}

\begin{Shaded}
\begin{Highlighting}[]
\CommentTok{\#HISTOGRAM}
\NormalTok{stats }\OtherTok{=}\NormalTok{ df\_lab }\SpecialCharTok{\%\textgreater{}\%} \FunctionTok{group\_by}\NormalTok{(pretty\_condition) }\SpecialCharTok{\%\textgreater{}\%}\NormalTok{ dplyr}\SpecialCharTok{::}\FunctionTok{summarise}\NormalTok{(}\AttributeTok{mean =} \FunctionTok{mean}\NormalTok{(DV\_percent\_test\_NABS)}\SpecialCharTok{*}\DecValTok{100}\NormalTok{)}
\NormalTok{gmean }\OtherTok{=}\NormalTok{ df\_lab }\SpecialCharTok{\%\textgreater{}\%}\NormalTok{ dplyr}\SpecialCharTok{::}\FunctionTok{summarise}\NormalTok{(}\AttributeTok{mean =} \FunctionTok{mean}\NormalTok{(DV\_percent\_test\_NABS)}\SpecialCharTok{*}\DecValTok{100}\NormalTok{)}
\FunctionTok{gf\_props}\NormalTok{(}\SpecialCharTok{\textasciitilde{}}\NormalTok{DV\_percent\_test\_NABS}\SpecialCharTok{*}\DecValTok{100}\NormalTok{, }\AttributeTok{fill =} \SpecialCharTok{\textasciitilde{}}\NormalTok{pretty\_condition, }\AttributeTok{data =}\NormalTok{ df\_lab) }\SpecialCharTok{\%\textgreater{}\%} \FunctionTok{gf\_facet\_grid}\NormalTok{(}\SpecialCharTok{\textasciitilde{}}\NormalTok{pretty\_condition) }\SpecialCharTok{\%\textgreater{}\%} 
  \FunctionTok{gf\_vline}\NormalTok{(}\AttributeTok{data =}\NormalTok{ stats, }\AttributeTok{xintercept =} \SpecialCharTok{\textasciitilde{}}\NormalTok{mean, }\AttributeTok{color =} \StringTok{"red"}\NormalTok{) }\SpecialCharTok{+}
  \FunctionTok{labs}\NormalTok{(}\AttributeTok{x =} \StringTok{"\% Correct"}\NormalTok{,}
       \AttributeTok{y =} \StringTok{"proportion of subjects"}\NormalTok{,}
       \AttributeTok{title =} \StringTok{"(LAB) TEST Phase Absolute Score (\% Correct)"}\NormalTok{,}
       \AttributeTok{subtitle =} \StringTok{""}\NormalTok{) }\SpecialCharTok{+} 
  \FunctionTok{theme\_minimal}\NormalTok{()}
\end{Highlighting}
\end{Shaded}

\begin{figure}[H]

{\centering \includegraphics{analysis/SGC3A/4_sgc3A_hypotesting_files/figure-pdf/VIS-TEST-ABS-LAB-1.pdf}

}

\end{figure}

\hypertarget{model}{%
\subparagraph{Model}\label{model}}

\begin{Shaded}
\begin{Highlighting}[]
\CommentTok{\#SCORE predicted by CONDITION}
\NormalTok{lab.testabs.lm1 }\OtherTok{\textless{}{-}} \FunctionTok{lm}\NormalTok{(DV\_percent\_test\_NABS }\SpecialCharTok{\textasciitilde{}}\NormalTok{ pretty\_condition, }\AttributeTok{data =}\NormalTok{ df\_lab)}
\FunctionTok{paste}\NormalTok{(}\StringTok{"Model"}\NormalTok{)}
\end{Highlighting}
\end{Shaded}

\begin{verbatim}
[1] "Model"
\end{verbatim}

\begin{Shaded}
\begin{Highlighting}[]
\FunctionTok{summary}\NormalTok{(lab.testabs.lm1)}
\end{Highlighting}
\end{Shaded}

\begin{verbatim}

Call:
lm(formula = DV_percent_test_NABS ~ pretty_condition, data = df_lab)

Residuals:
   Min     1Q Median     3Q    Max 
-0.416 -0.214 -0.214  0.459  0.786 

Coefficients:
                        Estimate Std. Error t value Pr(>|t|)    
(Intercept)               0.2137     0.0513    4.16 0.000058 ***
pretty_conditionimpasse   0.2023     0.0720    2.81   0.0058 ** 
---
Signif. codes:  0 '***' 0.001 '**' 0.01 '*' 0.05 '.' 0.1 ' ' 1

Residual standard error: 0.404 on 124 degrees of freedom
Multiple R-squared:  0.0598,    Adjusted R-squared:  0.0522 
F-statistic: 7.89 on 1 and 124 DF,  p-value: 0.00579
\end{verbatim}

\begin{Shaded}
\begin{Highlighting}[]
\FunctionTok{paste}\NormalTok{(}\StringTok{"Partition Variance"}\NormalTok{)}
\end{Highlighting}
\end{Shaded}

\begin{verbatim}
[1] "Partition Variance"
\end{verbatim}

\begin{Shaded}
\begin{Highlighting}[]
\FunctionTok{anova}\NormalTok{(lab.testabs.lm1)}
\end{Highlighting}
\end{Shaded}

\begin{verbatim}
Analysis of Variance Table

Response: DV_percent_test_NABS
                  Df Sum Sq Mean Sq F value Pr(>F)   
pretty_condition   1   1.29   1.289    7.89 0.0058 **
Residuals        124  20.26   0.163                  
---
Signif. codes:  0 '***' 0.001 '**' 0.01 '*' 0.05 '.' 0.1 ' ' 1
\end{verbatim}

\begin{Shaded}
\begin{Highlighting}[]
\FunctionTok{paste}\NormalTok{(}\StringTok{"Confidence Interval on Parameter Estimates"}\NormalTok{)}
\end{Highlighting}
\end{Shaded}

\begin{verbatim}
[1] "Confidence Interval on Parameter Estimates"
\end{verbatim}

\begin{Shaded}
\begin{Highlighting}[]
\FunctionTok{confint}\NormalTok{(lab.testabs.lm1)}
\end{Highlighting}
\end{Shaded}

\begin{verbatim}
                         2.5 % 97.5 %
(Intercept)             0.1121  0.315
pretty_conditionimpasse 0.0597  0.345
\end{verbatim}

\begin{Shaded}
\begin{Highlighting}[]
\CommentTok{\# report(m1) \#sanity check}
\CommentTok{\#print model equation}
\NormalTok{eq }\OtherTok{\textless{}{-}} \FunctionTok{extract\_eq}\NormalTok{(lab.testabs.lm1, }\AttributeTok{use\_coefs =} \ConstantTok{TRUE}\NormalTok{)}
\end{Highlighting}
\end{Shaded}

\textbf{Model equation} ,

\begin{equation}
, \operatorname{\widehat{DV\_percent\_test\_NABS}} = 0.21 + 0.2(\operatorname{pretty\_condition}_{\operatorname{impasse}}), 
\end{equation}

\textbf{For (In Person)} an OLS linear regression predicting test-phase
(\% correct) by experimental condition explains a statistically
significant though small 6\% variance in accuracy (F(1,124) = 7.89, p
\textless{} 0.01). The estimated beta coefficient (\(\beta\) = 0.20,
95\% CI {[}0.05, 0.35{]}) predicts that participants in the impasse
condition will on average score 20\% higher than those in the control
condition.

\begin{Shaded}
\begin{Highlighting}[]
\CommentTok{\#MODEL ESTIMATES WITH UNCERTAINTY}

\CommentTok{\#setup references }
\NormalTok{m }\OtherTok{\textless{}{-}}\NormalTok{ lab.testabs.lm1}
\NormalTok{df }\OtherTok{\textless{}{-}}\NormalTok{ df\_lab }
\NormalTok{x }\OtherTok{\textless{}{-}}\NormalTok{ df\_lab}\SpecialCharTok{$}\NormalTok{DV\_percent\_test\_NABS}


\FunctionTok{library}\NormalTok{(ggdist)}
\end{Highlighting}
\end{Shaded}

\begin{verbatim}

Attaching package: 'ggdist'
\end{verbatim}

\begin{verbatim}
The following objects are masked from 'package:ggridges':

    scale_point_color_continuous, scale_point_color_discrete,
    scale_point_colour_continuous, scale_point_colour_discrete,
    scale_point_fill_continuous, scale_point_fill_discrete,
    scale_point_size_continuous
\end{verbatim}

\begin{Shaded}
\begin{Highlighting}[]
\FunctionTok{library}\NormalTok{(broom)}
\FunctionTok{library}\NormalTok{(modelr)}
\end{Highlighting}
\end{Shaded}

\begin{verbatim}

Attaching package: 'modelr'
\end{verbatim}

\begin{verbatim}
The following object is masked from 'package:broom':

    bootstrap
\end{verbatim}

\begin{verbatim}
The following objects are masked from 'package:performance':

    mae, mse, rmse
\end{verbatim}

\begin{verbatim}
The following object is masked from 'package:ggformula':

    na.warn
\end{verbatim}

\begin{Shaded}
\begin{Highlighting}[]
\FunctionTok{library}\NormalTok{(distributional)}
\end{Highlighting}
\end{Shaded}

\begin{verbatim}

Attaching package: 'distributional'
\end{verbatim}

\begin{verbatim}
The following object is masked from 'package:gnm':

    parameters
\end{verbatim}

\begin{Shaded}
\begin{Highlighting}[]
\CommentTok{\#uncertainty model visualization}
\NormalTok{df }\SpecialCharTok{\%\textgreater{}\%}
  \FunctionTok{data\_grid}\NormalTok{(pretty\_condition) }\SpecialCharTok{\%\textgreater{}\%}
  \FunctionTok{augment}\NormalTok{(m, }\AttributeTok{newdata =}\NormalTok{ ., }\AttributeTok{se\_fit =} \ConstantTok{TRUE}\NormalTok{) }\SpecialCharTok{\%\textgreater{}\%}
  \FunctionTok{ggplot}\NormalTok{(}\FunctionTok{aes}\NormalTok{(}\AttributeTok{y =}\NormalTok{ pretty\_condition)) }\SpecialCharTok{+}
  \FunctionTok{stat\_halfeye}\NormalTok{(}
    \FunctionTok{aes}\NormalTok{(}\AttributeTok{xdist =} \FunctionTok{dist\_student\_t}\NormalTok{(}\AttributeTok{df =} \FunctionTok{df.residual}\NormalTok{(m), }
        \AttributeTok{mu =}\NormalTok{ .fitted, }\AttributeTok{sigma =}\NormalTok{ .se.fit)), }\AttributeTok{scale =}\NormalTok{ .}\DecValTok{5}\NormalTok{) }\SpecialCharTok{+}
  \CommentTok{\# add raw data in too (scale = .5 above adjusts the halfeye height so}
  \CommentTok{\# that the data fit in as well)}
  \FunctionTok{geom\_jitter}\NormalTok{(}\FunctionTok{aes}\NormalTok{(}\AttributeTok{x =}\NormalTok{ x), }\AttributeTok{data =}\NormalTok{ df, }\AttributeTok{pch =} \StringTok{"|"}\NormalTok{, }\AttributeTok{size =} \DecValTok{2}\NormalTok{, }
              \AttributeTok{position =}   \FunctionTok{position\_nudge}\NormalTok{(}\AttributeTok{y =} \SpecialCharTok{{-}}\NormalTok{.}\DecValTok{15}\NormalTok{), }\AttributeTok{alpha =} \FloatTok{0.5}\NormalTok{) }\SpecialCharTok{+}  
  \FunctionTok{labs}\NormalTok{ (}\AttributeTok{title =} \StringTok{"Model Estimates with Uncertainty"}\NormalTok{, }\AttributeTok{x =} \StringTok{"model coefficient"}\NormalTok{) }\SpecialCharTok{+} 
  \FunctionTok{theme\_minimal}\NormalTok{()}
\end{Highlighting}
\end{Shaded}

\begin{figure}[H]

{\centering \includegraphics{analysis/SGC3A/4_sgc3A_hypotesting_files/figure-pdf/unnamed-chunk-5-1.pdf}

}

\end{figure}

\hypertarget{diagnostics}{%
\subparagraph{Diagnostics}\label{diagnostics}}

\begin{Shaded}
\begin{Highlighting}[]
\CommentTok{\#model diagnostics}
\FunctionTok{check\_model}\NormalTok{(lab.testabs.lm1, }\AttributeTok{panel =} \ConstantTok{TRUE}\NormalTok{)}
\end{Highlighting}
\end{Shaded}

\begin{figure}[H]

{\centering \includegraphics{analysis/SGC3A/4_sgc3A_hypotesting_files/figure-pdf/DIAG-TEST-ABS-LAB-1.pdf}

}

\end{figure}

(1) RESIDUAL DISTRIBUTION: \ensuremath{3.44\times 10^{-11}} (2)
HOMOGENEITY: 0.387 (3) HETERSCEDASTICITY: 0.382 (4) AUTOCORRELATION:
0.438 (5) OUTLIERS: FALSE, FALSE, FALSE, FALSE, FALSE, FALSE, FALSE,
FALSE, FALSE, FALSE, FALSE, FALSE, FALSE, FALSE, FALSE, FALSE, FALSE,
FALSE, FALSE, FALSE, FALSE, FALSE, FALSE, FALSE, FALSE, FALSE, FALSE,
FALSE, FALSE, FALSE, FALSE, FALSE, FALSE, FALSE, FALSE, FALSE, FALSE,
FALSE, FALSE, FALSE, FALSE, FALSE, FALSE, FALSE, FALSE, FALSE, FALSE,
FALSE, FALSE, FALSE, FALSE, FALSE, FALSE, FALSE, FALSE, FALSE, FALSE,
FALSE, FALSE, FALSE, FALSE, FALSE, FALSE, FALSE, FALSE, FALSE, FALSE,
FALSE, FALSE, FALSE, FALSE, FALSE, FALSE, FALSE, FALSE, FALSE, FALSE,
FALSE, FALSE, FALSE, FALSE, FALSE, FALSE, FALSE, FALSE, FALSE, FALSE,
FALSE, FALSE, FALSE, FALSE, FALSE, FALSE, FALSE, FALSE, FALSE, FALSE,
FALSE, FALSE, FALSE, FALSE, FALSE, FALSE, FALSE, FALSE, FALSE, FALSE,
FALSE, FALSE, FALSE, FALSE, FALSE, FALSE, FALSE, FALSE, FALSE, FALSE,
FALSE, FALSE, FALSE, FALSE, FALSE, FALSE, FALSE, FALSE, FALSE

\hypertarget{inference}{%
\subparagraph{Inference}\label{inference}}

OLS Linear Regression on \% correct in the TEST PHASE shows that
condition explains a small but statistically significant amount of
variance (impasse \textgreater{} control). However, the model is a poor
fit to the data: (1) the model predictions for each group are closer to
the anitimode of each of distribution than the group modes, and (2) the
distribution of residuals is not normal.

\hypertarget{online-replication}{%
\paragraph{(Online Replication)}\label{online-replication}}

\hypertarget{visualization-1}{%
\subparagraph{Visualization}\label{visualization-1}}

\begin{Shaded}
\begin{Highlighting}[]
\CommentTok{\#HISTOGRAM}
\NormalTok{stats }\OtherTok{=}\NormalTok{ df\_online }\SpecialCharTok{\%\textgreater{}\%} \FunctionTok{group\_by}\NormalTok{(pretty\_condition) }\SpecialCharTok{\%\textgreater{}\%}\NormalTok{ dplyr}\SpecialCharTok{::}\FunctionTok{summarise}\NormalTok{(}\AttributeTok{mean =} \FunctionTok{mean}\NormalTok{(DV\_percent\_test\_NABS)}\SpecialCharTok{*}\DecValTok{100}\NormalTok{)}
\NormalTok{gmean }\OtherTok{=}\NormalTok{ df\_online }\SpecialCharTok{\%\textgreater{}\%}\NormalTok{ dplyr}\SpecialCharTok{::}\FunctionTok{summarise}\NormalTok{(}\AttributeTok{mean =} \FunctionTok{mean}\NormalTok{(DV\_percent\_test\_NABS)}\SpecialCharTok{*}\DecValTok{100}\NormalTok{)}
\FunctionTok{gf\_props}\NormalTok{(}\SpecialCharTok{\textasciitilde{}}\NormalTok{DV\_percent\_test\_NABS}\SpecialCharTok{*}\DecValTok{100}\NormalTok{, }\AttributeTok{fill =} \SpecialCharTok{\textasciitilde{}}\NormalTok{pretty\_condition, }\AttributeTok{data =}\NormalTok{ df\_online) }\SpecialCharTok{\%\textgreater{}\%} 
  \FunctionTok{gf\_facet\_grid}\NormalTok{(}\SpecialCharTok{\textasciitilde{}}\NormalTok{pretty\_condition) }\SpecialCharTok{\%\textgreater{}\%} 
  \FunctionTok{gf\_vline}\NormalTok{(}\AttributeTok{data =}\NormalTok{ stats, }\AttributeTok{xintercept =} \SpecialCharTok{\textasciitilde{}}\NormalTok{mean, }\AttributeTok{color =} \StringTok{"red"}\NormalTok{) }\SpecialCharTok{+}
  \FunctionTok{labs}\NormalTok{(}\AttributeTok{x =} \StringTok{"\% Correct"}\NormalTok{,}
       \AttributeTok{y =} \StringTok{"proportion of subjects"}\NormalTok{,}
       \AttributeTok{title =} \StringTok{"(ONLINE) TEST Phase Absolute Score (\% Correct)"}\NormalTok{,}
       \AttributeTok{subtitle =} \StringTok{""}\NormalTok{) }\SpecialCharTok{+} \FunctionTok{theme\_minimal}\NormalTok{()}
\end{Highlighting}
\end{Shaded}

\begin{figure}[H]

{\centering \includegraphics{analysis/SGC3A/4_sgc3A_hypotesting_files/figure-pdf/VIS-TEST-ABS-ONLINE-1.pdf}

}

\end{figure}

\hypertarget{model-1}{%
\subparagraph{Model}\label{model-1}}

\begin{Shaded}
\begin{Highlighting}[]
\CommentTok{\#SCORE predicted by CONDITION}
\NormalTok{rep.testabs.lm1 }\OtherTok{\textless{}{-}} \FunctionTok{lm}\NormalTok{(DV\_percent\_test\_NABS }\SpecialCharTok{\textasciitilde{}}\NormalTok{ pretty\_condition, }\AttributeTok{data =}\NormalTok{ df\_online)}
\FunctionTok{paste}\NormalTok{(}\StringTok{"Model"}\NormalTok{)}
\end{Highlighting}
\end{Shaded}

\begin{verbatim}
[1] "Model"
\end{verbatim}

\begin{Shaded}
\begin{Highlighting}[]
\FunctionTok{summary}\NormalTok{(rep.testabs.lm1)}
\end{Highlighting}
\end{Shaded}

\begin{verbatim}

Call:
lm(formula = DV_percent_test_NABS ~ pretty_condition, data = df_online)

Residuals:
   Min     1Q Median     3Q    Max 
-0.354 -0.354 -0.174  0.396  0.826 

Coefficients:
                        Estimate Std. Error t value Pr(>|t|)    
(Intercept)               0.1745     0.0398    4.39 0.000019 ***
pretty_conditionimpasse   0.1797     0.0547    3.29   0.0012 ** 
---
Signif. codes:  0 '***' 0.001 '**' 0.01 '*' 0.05 '.' 0.1 ' ' 1

Residual standard error: 0.39 on 202 degrees of freedom
Multiple R-squared:  0.0508,    Adjusted R-squared:  0.0461 
F-statistic: 10.8 on 1 and 202 DF,  p-value: 0.0012
\end{verbatim}

\begin{Shaded}
\begin{Highlighting}[]
\FunctionTok{paste}\NormalTok{(}\StringTok{"Partition Variance"}\NormalTok{)}
\end{Highlighting}
\end{Shaded}

\begin{verbatim}
[1] "Partition Variance"
\end{verbatim}

\begin{Shaded}
\begin{Highlighting}[]
\FunctionTok{anova}\NormalTok{(rep.testabs.lm1)}
\end{Highlighting}
\end{Shaded}

\begin{verbatim}
Analysis of Variance Table

Response: DV_percent_test_NABS
                  Df Sum Sq Mean Sq F value Pr(>F)   
pretty_condition   1   1.64   1.641    10.8 0.0012 **
Residuals        202  30.69   0.152                  
---
Signif. codes:  0 '***' 0.001 '**' 0.01 '*' 0.05 '.' 0.1 ' ' 1
\end{verbatim}

\begin{Shaded}
\begin{Highlighting}[]
\FunctionTok{paste}\NormalTok{(}\StringTok{"Confidence Interval on Parameter Estimates"}\NormalTok{)}
\end{Highlighting}
\end{Shaded}

\begin{verbatim}
[1] "Confidence Interval on Parameter Estimates"
\end{verbatim}

\begin{Shaded}
\begin{Highlighting}[]
\FunctionTok{confint}\NormalTok{(rep.testabs.lm1)}
\end{Highlighting}
\end{Shaded}

\begin{verbatim}
                         2.5 % 97.5 %
(Intercept)             0.0960  0.253
pretty_conditionimpasse 0.0719  0.287
\end{verbatim}

\begin{Shaded}
\begin{Highlighting}[]
\CommentTok{\# report(m1) \#sanity check}
\CommentTok{\#print model equation}
\NormalTok{eq }\OtherTok{\textless{}{-}} \FunctionTok{extract\_eq}\NormalTok{(rep.testabs.lm1)}
\end{Highlighting}
\end{Shaded}

\textbf{Model equation} ,

\begin{equation}
, \operatorname{DV\_percent\_test\_NABS} = \alpha + \beta_{1}(\operatorname{pretty\_condition}_{\operatorname{impasse}}) + \epsilon, 
\end{equation}

\textbf{For online replication} an OLS linear regression predicting
test-phase (\% correct) by experimental condition explains a
statistically significant though small 5\% variance in accuracy
(F(1,202) = 10.8, p \textless{} 0.01). The estimated beta coefficient
(\(\beta\) = 0.18, 95\% CI {[}0.07, 0.29{]}) predicts that participants
in the impasse condition will on average score 18\% higher than those in
the control condition.

\begin{Shaded}
\begin{Highlighting}[]
\CommentTok{\#MODEL ESTIMATES WITH UNCERTAINTY}

\CommentTok{\#setup references }
\NormalTok{m }\OtherTok{\textless{}{-}}\NormalTok{ rep.testabs.lm1}
\NormalTok{df }\OtherTok{\textless{}{-}}\NormalTok{ df\_online }
\NormalTok{x }\OtherTok{\textless{}{-}}\NormalTok{ df\_online}\SpecialCharTok{$}\NormalTok{DV\_percent\_test\_NABS}

\CommentTok{\#uncertainty model visualization}
\NormalTok{df }\SpecialCharTok{\%\textgreater{}\%}
  \FunctionTok{data\_grid}\NormalTok{(pretty\_condition) }\SpecialCharTok{\%\textgreater{}\%}
  \FunctionTok{augment}\NormalTok{(m, }\AttributeTok{newdata =}\NormalTok{ ., }\AttributeTok{se\_fit =} \ConstantTok{TRUE}\NormalTok{) }\SpecialCharTok{\%\textgreater{}\%}
  \FunctionTok{ggplot}\NormalTok{(}\FunctionTok{aes}\NormalTok{(}\AttributeTok{y =}\NormalTok{ pretty\_condition)) }\SpecialCharTok{+}
  \FunctionTok{stat\_halfeye}\NormalTok{(}
    \FunctionTok{aes}\NormalTok{(}\AttributeTok{xdist =} \FunctionTok{dist\_student\_t}\NormalTok{(}\AttributeTok{df =} \FunctionTok{df.residual}\NormalTok{(m), }
        \AttributeTok{mu =}\NormalTok{ .fitted, }\AttributeTok{sigma =}\NormalTok{ .se.fit)), }\AttributeTok{scale =}\NormalTok{ .}\DecValTok{5}\NormalTok{) }\SpecialCharTok{+}
  \CommentTok{\# add raw data in too (scale = .5 above adjusts the halfeye height so}
  \CommentTok{\# that the data fit in as well)}
  \FunctionTok{geom\_jitter}\NormalTok{(}\FunctionTok{aes}\NormalTok{(}\AttributeTok{x =}\NormalTok{ x), }\AttributeTok{data =}\NormalTok{ df, }\AttributeTok{pch =} \StringTok{"|"}\NormalTok{, }\AttributeTok{size =} \DecValTok{2}\NormalTok{, }
              \AttributeTok{position =}   \FunctionTok{position\_nudge}\NormalTok{(}\AttributeTok{y =} \SpecialCharTok{{-}}\NormalTok{.}\DecValTok{15}\NormalTok{), }\AttributeTok{alpha =} \FloatTok{0.5}\NormalTok{) }\SpecialCharTok{+}  
  \FunctionTok{labs}\NormalTok{ (}\AttributeTok{title =} \StringTok{"Model Estimates with Uncertainty"}\NormalTok{, }\AttributeTok{x =} \StringTok{"model coefficient"}\NormalTok{) }\SpecialCharTok{+} 
  \FunctionTok{theme\_minimal}\NormalTok{()}
\end{Highlighting}
\end{Shaded}

\begin{figure}[H]

{\centering \includegraphics{analysis/SGC3A/4_sgc3A_hypotesting_files/figure-pdf/unnamed-chunk-9-1.pdf}

}

\end{figure}

\hypertarget{diagnostics-1}{%
\subparagraph{Diagnostics}\label{diagnostics-1}}

\begin{Shaded}
\begin{Highlighting}[]
\CommentTok{\#model diagnostics}
\FunctionTok{check\_model}\NormalTok{(rep.testabs.lm1, }\AttributeTok{panel =} \ConstantTok{TRUE}\NormalTok{)}
\end{Highlighting}
\end{Shaded}

\begin{figure}[H]

{\centering \includegraphics{analysis/SGC3A/4_sgc3A_hypotesting_files/figure-pdf/DIAG-TEST-ABS-ONLINE-1.pdf}

}

\end{figure}

(1) RESIDUAL DISTRIBUTION: \ensuremath{2.834\times 10^{-16}} (2)
HOMOGENEITY: 0.029 (3) HETERSCEDASTICITY: 0.031 (4) AUTOCORRELATION:
0.686 (5) OUTLIERS: FALSE, FALSE, FALSE, FALSE, FALSE, FALSE, FALSE,
FALSE, FALSE, FALSE, FALSE, FALSE, FALSE, FALSE, FALSE, FALSE, FALSE,
FALSE, FALSE, FALSE, FALSE, FALSE, FALSE, FALSE, FALSE, FALSE, FALSE,
FALSE, FALSE, FALSE, FALSE, FALSE, FALSE, FALSE, FALSE, FALSE, FALSE,
FALSE, FALSE, FALSE, FALSE, FALSE, FALSE, FALSE, FALSE, FALSE, FALSE,
FALSE, FALSE, FALSE, FALSE, FALSE, FALSE, FALSE, FALSE, FALSE, FALSE,
FALSE, FALSE, FALSE, FALSE, FALSE, FALSE, FALSE, FALSE, FALSE, FALSE,
FALSE, FALSE, FALSE, FALSE, FALSE, FALSE, FALSE, FALSE, FALSE, FALSE,
FALSE, FALSE, FALSE, FALSE, FALSE, FALSE, FALSE, FALSE, FALSE, FALSE,
FALSE, FALSE, FALSE, FALSE, FALSE, FALSE, FALSE, FALSE, FALSE, FALSE,
FALSE, FALSE, FALSE, FALSE, FALSE, FALSE, FALSE, FALSE, FALSE, FALSE,
FALSE, FALSE, FALSE, FALSE, FALSE, FALSE, FALSE, FALSE, FALSE, FALSE,
FALSE, FALSE, FALSE, FALSE, FALSE, FALSE, FALSE, FALSE, FALSE, FALSE,
FALSE, FALSE, FALSE, FALSE, FALSE, FALSE, FALSE, FALSE, FALSE, FALSE,
FALSE, FALSE, FALSE, FALSE, FALSE, FALSE, FALSE, FALSE, FALSE, FALSE,
FALSE, FALSE, FALSE, FALSE, FALSE, FALSE, FALSE, FALSE, FALSE, FALSE,
FALSE, FALSE, FALSE, FALSE, FALSE, FALSE, FALSE, FALSE, FALSE, FALSE,
FALSE, FALSE, FALSE, FALSE, FALSE, FALSE, FALSE, FALSE, FALSE, FALSE,
FALSE, FALSE, FALSE, FALSE, FALSE, FALSE, FALSE, FALSE, FALSE, FALSE,
FALSE, FALSE, FALSE, FALSE, FALSE, FALSE, FALSE, FALSE, FALSE, FALSE,
FALSE, FALSE, FALSE, FALSE, FALSE, FALSE, FALSE

\hypertarget{inference-1}{%
\subparagraph{Inference}\label{inference-1}}

\textbf{For in person collection} OLS Linear Regression on \% correct in
the TEST PHASE shows that condition explains a small but statistically
significant amount of variance (impasse \textgreater{} control).
However, the model is a poor fit to the data: (1) the model predictions
for each group are closer to the anitimode of each of distribution than
the group modes, and (2) the distribution of residuals is not normal,
and the LM assumptions of homogeneity of variance (between groups) and
homogeneity of error variance appears to be violated.

\hypertarget{test-phase-scaled-score}{%
\subsection{Test Phase Scaled Score}\label{test-phase-scaled-score}}

While Absolute Score (as \# or \% correct) gives an indication of
accuracy, it does not differentiate between different kinds of incorrect
answers. The Scaled score includes this extra information see
\textbf{?@sec-scoring-scaledScore}

\hypertarget{linear-regression-1}{%
\subsubsection{Linear Regression}\label{linear-regression-1}}

\hypertarget{in-person-1}{%
\paragraph{(In Person)}\label{in-person-1}}

\hypertarget{visualization-2}{%
\subparagraph{Visualization}\label{visualization-2}}

\begin{Shaded}
\begin{Highlighting}[]
\CommentTok{\#HISTOGRAM}
\NormalTok{stats }\OtherTok{=}\NormalTok{ df\_lab }\SpecialCharTok{\%\textgreater{}\%} \FunctionTok{group\_by}\NormalTok{(pretty\_condition) }\SpecialCharTok{\%\textgreater{}\%}\NormalTok{ dplyr}\SpecialCharTok{::}\FunctionTok{summarise}\NormalTok{(}\AttributeTok{mean =} \FunctionTok{mean}\NormalTok{(item\_test\_SCALED))}
\FunctionTok{gf\_props}\NormalTok{(}\SpecialCharTok{\textasciitilde{}}\NormalTok{item\_test\_SCALED, }\AttributeTok{fill =} \SpecialCharTok{\textasciitilde{}}\NormalTok{pretty\_condition, }\AttributeTok{data =}\NormalTok{ df\_lab) }\SpecialCharTok{\%\textgreater{}\%} \FunctionTok{gf\_facet\_grid}\NormalTok{(}\SpecialCharTok{\textasciitilde{}}\NormalTok{pretty\_condition) }\SpecialCharTok{\%\textgreater{}\%} 
  \FunctionTok{gf\_vline}\NormalTok{(}\AttributeTok{data =}\NormalTok{ stats, }\AttributeTok{xintercept =} \SpecialCharTok{\textasciitilde{}}\NormalTok{mean, }\AttributeTok{color =} \StringTok{"red"}\NormalTok{) }\SpecialCharTok{+}
  \FunctionTok{labs}\NormalTok{(}\AttributeTok{x =} \StringTok{"Test Phase Scaled Score [{-}8, +8]"}\NormalTok{,}
       \AttributeTok{y =} \StringTok{"proportion of subjects"}\NormalTok{,}
       \AttributeTok{title =} \StringTok{"(LAB) TEST Phase Scaled Score "}\NormalTok{,}
       \AttributeTok{subtitle =} \StringTok{""}\NormalTok{) }\SpecialCharTok{+} 
  \FunctionTok{theme\_minimal}\NormalTok{()}
\end{Highlighting}
\end{Shaded}

\begin{figure}[H]

{\centering \includegraphics{analysis/SGC3A/4_sgc3A_hypotesting_files/figure-pdf/VIS-TEST-SCALED-LAB-1.pdf}

}

\end{figure}

\hypertarget{model-2}{%
\subparagraph{Model}\label{model-2}}

\begin{Shaded}
\begin{Highlighting}[]
\CommentTok{\#SCORE predicted by CONDITION}
\NormalTok{lab.test\_scaled.lm1 }\OtherTok{\textless{}{-}} \FunctionTok{lm}\NormalTok{(item\_test\_SCALED }\SpecialCharTok{\textasciitilde{}}\NormalTok{ pretty\_condition, }\AttributeTok{data =}\NormalTok{ df\_lab)}
\FunctionTok{paste}\NormalTok{(}\StringTok{"Model"}\NormalTok{)}
\end{Highlighting}
\end{Shaded}

\begin{verbatim}
[1] "Model"
\end{verbatim}

\begin{Shaded}
\begin{Highlighting}[]
\FunctionTok{summary}\NormalTok{(lab.test\_scaled.lm1)}
\end{Highlighting}
\end{Shaded}

\begin{verbatim}

Call:
lm(formula = item_test_SCALED ~ pretty_condition, data = df_lab)

Residuals:
   Min     1Q Median     3Q    Max 
 -7.74  -3.98  -3.23   6.76  12.02 

Coefficients:
                        Estimate Std. Error t value Pr(>|t|)    
(Intercept)               -4.024      0.818   -4.92  2.7e-06 ***
pretty_conditionimpasse    3.766      1.148    3.28   0.0013 ** 
---
Signif. codes:  0 '***' 0.001 '**' 0.01 '*' 0.05 '.' 0.1 ' ' 1

Residual standard error: 6.44 on 124 degrees of freedom
Multiple R-squared:  0.0798,    Adjusted R-squared:  0.0724 
F-statistic: 10.8 on 1 and 124 DF,  p-value: 0.00135
\end{verbatim}

\begin{Shaded}
\begin{Highlighting}[]
\FunctionTok{paste}\NormalTok{(}\StringTok{"Partition Variance"}\NormalTok{)}
\end{Highlighting}
\end{Shaded}

\begin{verbatim}
[1] "Partition Variance"
\end{verbatim}

\begin{Shaded}
\begin{Highlighting}[]
\FunctionTok{anova}\NormalTok{(lab.test\_scaled.lm1)}
\end{Highlighting}
\end{Shaded}

\begin{verbatim}
Analysis of Variance Table

Response: item_test_SCALED
                  Df Sum Sq Mean Sq F value Pr(>F)   
pretty_condition   1    447     447    10.8 0.0013 **
Residuals        124   5150      42                  
---
Signif. codes:  0 '***' 0.001 '**' 0.01 '*' 0.05 '.' 0.1 ' ' 1
\end{verbatim}

\begin{Shaded}
\begin{Highlighting}[]
\FunctionTok{paste}\NormalTok{(}\StringTok{"Confidence Interval on Parameter Estimates"}\NormalTok{)}
\end{Highlighting}
\end{Shaded}

\begin{verbatim}
[1] "Confidence Interval on Parameter Estimates"
\end{verbatim}

\begin{Shaded}
\begin{Highlighting}[]
\FunctionTok{confint}\NormalTok{(lab.test\_scaled.lm1)}
\end{Highlighting}
\end{Shaded}

\begin{verbatim}
                        2.5 % 97.5 %
(Intercept)             -5.64  -2.40
pretty_conditionimpasse  1.49   6.04
\end{verbatim}

\begin{Shaded}
\begin{Highlighting}[]
\CommentTok{\# report(m1) \#sanity check}
\CommentTok{\#print model equation}
\NormalTok{eq }\OtherTok{\textless{}{-}} \FunctionTok{extract\_eq}\NormalTok{(lab.test\_scaled.lm1, }\AttributeTok{use\_coefs =} \ConstantTok{TRUE}\NormalTok{)}
\end{Highlighting}
\end{Shaded}

\textbf{Model equation} ,

\begin{equation}
, \operatorname{\widehat{item\_test\_SCALED}} = -4.02 + 3.77(\operatorname{pretty\_condition}_{\operatorname{impasse}}), 
\end{equation}

\textbf{For (In Person)} an OLS linear regression predicting test-phase
(\% correct) by experimental condition explains a statistically
significant though small 8\% variance in accuracy (F(1,124) = 10.8, p
\textless{} 0.005). The estimated beta coefficient (\(\beta\) = 3.77,
95\% CI {[}1.49, 6.04{]}) predicts that participants in the impasse
condition will on average 4 points higher than those in the control
condition.

\begin{Shaded}
\begin{Highlighting}[]
\CommentTok{\#MODEL ESTIMATES WITH UNCERTAINTY}

\CommentTok{\#setup references }
\NormalTok{m }\OtherTok{\textless{}{-}}\NormalTok{ lab.test\_scaled.lm1}
\NormalTok{df }\OtherTok{\textless{}{-}}\NormalTok{ df\_lab }
\NormalTok{x }\OtherTok{\textless{}{-}}\NormalTok{ df\_lab}\SpecialCharTok{$}\NormalTok{item\_test\_SCALED}

\CommentTok{\#uncertainty model visualization}
\NormalTok{df }\SpecialCharTok{\%\textgreater{}\%}
  \FunctionTok{data\_grid}\NormalTok{(pretty\_condition) }\SpecialCharTok{\%\textgreater{}\%}
  \FunctionTok{augment}\NormalTok{(m, }\AttributeTok{newdata =}\NormalTok{ ., }\AttributeTok{se\_fit =} \ConstantTok{TRUE}\NormalTok{) }\SpecialCharTok{\%\textgreater{}\%}
  \FunctionTok{ggplot}\NormalTok{(}\FunctionTok{aes}\NormalTok{(}\AttributeTok{y =}\NormalTok{ pretty\_condition)) }\SpecialCharTok{+}
  \FunctionTok{stat\_halfeye}\NormalTok{(}
    \FunctionTok{aes}\NormalTok{(}\AttributeTok{xdist =} \FunctionTok{dist\_student\_t}\NormalTok{(}\AttributeTok{df =} \FunctionTok{df.residual}\NormalTok{(m), }
        \AttributeTok{mu =}\NormalTok{ .fitted, }\AttributeTok{sigma =}\NormalTok{ .se.fit)), }\AttributeTok{scale =}\NormalTok{ .}\DecValTok{5}\NormalTok{) }\SpecialCharTok{+}
  \CommentTok{\# add raw data in too (scale = .5 above adjusts the halfeye height so}
  \CommentTok{\# that the data fit in as well)}
  \FunctionTok{geom\_jitter}\NormalTok{(}\FunctionTok{aes}\NormalTok{(}\AttributeTok{x =}\NormalTok{ x), }\AttributeTok{data =}\NormalTok{ df, }\AttributeTok{pch =} \StringTok{"|"}\NormalTok{, }\AttributeTok{size =} \DecValTok{2}\NormalTok{, }
              \AttributeTok{position =}   \FunctionTok{position\_nudge}\NormalTok{(}\AttributeTok{y =} \SpecialCharTok{{-}}\NormalTok{.}\DecValTok{15}\NormalTok{), }\AttributeTok{alpha =} \FloatTok{0.5}\NormalTok{) }\SpecialCharTok{+}  
  \FunctionTok{labs}\NormalTok{ (}\AttributeTok{title =} \StringTok{"Model Estimates with Uncertainty"}\NormalTok{, }\AttributeTok{x =} \StringTok{"model coefficient"}\NormalTok{) }\SpecialCharTok{+} 
  \FunctionTok{theme\_minimal}\NormalTok{()}
\end{Highlighting}
\end{Shaded}

\begin{figure}[H]

{\centering \includegraphics{analysis/SGC3A/4_sgc3A_hypotesting_files/figure-pdf/unnamed-chunk-13-1.pdf}

}

\end{figure}

\hypertarget{diagnostics-2}{%
\subparagraph{Diagnostics}\label{diagnostics-2}}

\begin{Shaded}
\begin{Highlighting}[]
\CommentTok{\#model diagnostics}
\FunctionTok{check\_model}\NormalTok{(lab.test\_scaled.lm1, }\AttributeTok{panel =} \ConstantTok{TRUE}\NormalTok{)}
\end{Highlighting}
\end{Shaded}

\begin{figure}[H]

{\centering \includegraphics{analysis/SGC3A/4_sgc3A_hypotesting_files/figure-pdf/DIAG-TEST-SCALED-LAB-1.pdf}

}

\end{figure}

(1) RESIDUAL DISTRIBUTION: \ensuremath{9.53\times 10^{-10}}\\
(2) HOMOGENEITY: 0.547\\
(3) HETERSCEDASTICITY: 0.542\\
(4) AUTOCORRELATION: 0.206 (5) OUTLIERS: FALSE, FALSE, FALSE, FALSE,
FALSE, FALSE, FALSE, FALSE, FALSE, FALSE, FALSE, FALSE, FALSE, FALSE,
FALSE, FALSE, FALSE, FALSE, FALSE, FALSE, FALSE, FALSE, FALSE, FALSE,
FALSE, FALSE, FALSE, FALSE, FALSE, FALSE, FALSE, FALSE, FALSE, FALSE,
FALSE, FALSE, FALSE, FALSE, FALSE, FALSE, FALSE, FALSE, FALSE, FALSE,
FALSE, FALSE, FALSE, FALSE, FALSE, FALSE, FALSE, FALSE, FALSE, FALSE,
FALSE, FALSE, FALSE, FALSE, FALSE, FALSE, FALSE, FALSE, FALSE, FALSE,
FALSE, FALSE, FALSE, FALSE, FALSE, FALSE, FALSE, FALSE, FALSE, FALSE,
FALSE, FALSE, FALSE, FALSE, FALSE, FALSE, FALSE, FALSE, FALSE, FALSE,
FALSE, FALSE, FALSE, FALSE, FALSE, FALSE, FALSE, FALSE, FALSE, FALSE,
FALSE, FALSE, FALSE, FALSE, FALSE, FALSE, FALSE, FALSE, FALSE, FALSE,
FALSE, FALSE, FALSE, FALSE, FALSE, FALSE, FALSE, FALSE, FALSE, FALSE,
FALSE, FALSE, FALSE, FALSE, FALSE, FALSE, FALSE, FALSE, FALSE, FALSE,
FALSE, FALSE

\hypertarget{inference-2}{%
\subparagraph{Inference}\label{inference-2}}

OLS Linear Regression on SCALED SCORE in the TEST PHASE shows that
condition explains a small but statistically significant amount of
variance (impasse \textgreater{} control). However, the model is a poor
fit to the data: (1) the model predictions for each group are closer to
the anitimode of each of distribution than the group modes, and (2) the
distribution of residuals is not normal. (Assumptions of homogenity of
variance across groups, and homogeneity of variance in residuals are
met)

\hypertarget{online-replication-1}{%
\paragraph{(Online Replication)}\label{online-replication-1}}

\hypertarget{visualization-3}{%
\subparagraph{Visualization}\label{visualization-3}}

\begin{Shaded}
\begin{Highlighting}[]
\CommentTok{\#HISTOGRAM}
\NormalTok{stats }\OtherTok{=}\NormalTok{ df\_online }\SpecialCharTok{\%\textgreater{}\%} \FunctionTok{group\_by}\NormalTok{(pretty\_condition) }\SpecialCharTok{\%\textgreater{}\%}\NormalTok{ dplyr}\SpecialCharTok{::}\FunctionTok{summarise}\NormalTok{(}\AttributeTok{mean =} \FunctionTok{mean}\NormalTok{(item\_test\_SCALED))}
\FunctionTok{gf\_props}\NormalTok{(}\SpecialCharTok{\textasciitilde{}}\NormalTok{item\_test\_SCALED, }\AttributeTok{fill =} \SpecialCharTok{\textasciitilde{}}\NormalTok{pretty\_condition, }\AttributeTok{data =}\NormalTok{ df\_lab) }\SpecialCharTok{\%\textgreater{}\%} \FunctionTok{gf\_facet\_grid}\NormalTok{(}\SpecialCharTok{\textasciitilde{}}\NormalTok{pretty\_condition) }\SpecialCharTok{\%\textgreater{}\%} 
  \FunctionTok{gf\_vline}\NormalTok{(}\AttributeTok{data =}\NormalTok{ stats, }\AttributeTok{xintercept =} \SpecialCharTok{\textasciitilde{}}\NormalTok{mean, }\AttributeTok{color =} \StringTok{"red"}\NormalTok{) }\SpecialCharTok{+}
  \FunctionTok{labs}\NormalTok{(}\AttributeTok{x =} \StringTok{"Test Phase Scaled Score [{-}8, +8]"}\NormalTok{,}
       \AttributeTok{y =} \StringTok{"proportion of subjects"}\NormalTok{,}
       \AttributeTok{title =} \StringTok{"(ONLINE) TEST Phase Scaled Score "}\NormalTok{,}
       \AttributeTok{subtitle =} \StringTok{""}\NormalTok{) }\SpecialCharTok{+} 
  \FunctionTok{theme\_minimal}\NormalTok{()}
\end{Highlighting}
\end{Shaded}

\begin{figure}[H]

{\centering \includegraphics{analysis/SGC3A/4_sgc3A_hypotesting_files/figure-pdf/VIS-TEST-SCALED-ONLINE-1.pdf}

}

\end{figure}

\hypertarget{model-3}{%
\subparagraph{Model}\label{model-3}}

\begin{Shaded}
\begin{Highlighting}[]
\CommentTok{\#SCORE predicted by CONDITION}
\NormalTok{rep.test\_scaled.lm1 }\OtherTok{\textless{}{-}} \FunctionTok{lm}\NormalTok{(item\_test\_SCALED }\SpecialCharTok{\textasciitilde{}}\NormalTok{ pretty\_condition, }\AttributeTok{data =}\NormalTok{ df\_online)}
\FunctionTok{paste}\NormalTok{(}\StringTok{"Model"}\NormalTok{)}
\end{Highlighting}
\end{Shaded}

\begin{verbatim}
[1] "Model"
\end{verbatim}

\begin{Shaded}
\begin{Highlighting}[]
\FunctionTok{summary}\NormalTok{(rep.test\_scaled.lm1)}
\end{Highlighting}
\end{Shaded}

\begin{verbatim}

Call:
lm(formula = item_test_SCALED ~ pretty_condition, data = df_online)

Residuals:
   Min     1Q Median     3Q    Max 
 -7.20  -3.96  -2.46   6.80  12.04 

Coefficients:
                        Estimate Std. Error t value Pr(>|t|)    
(Intercept)               -4.036      0.622   -6.49  6.4e-10 ***
pretty_conditionimpasse    3.236      0.854    3.79   0.0002 ***
---
Signif. codes:  0 '***' 0.001 '**' 0.01 '*' 0.05 '.' 0.1 ' ' 1

Residual standard error: 6.09 on 202 degrees of freedom
Multiple R-squared:  0.0663,    Adjusted R-squared:  0.0617 
F-statistic: 14.3 on 1 and 202 DF,  p-value: 0.000201
\end{verbatim}

\begin{Shaded}
\begin{Highlighting}[]
\FunctionTok{paste}\NormalTok{(}\StringTok{"Partition Variance"}\NormalTok{)}
\end{Highlighting}
\end{Shaded}

\begin{verbatim}
[1] "Partition Variance"
\end{verbatim}

\begin{Shaded}
\begin{Highlighting}[]
\FunctionTok{anova}\NormalTok{(rep.test\_scaled.lm1)}
\end{Highlighting}
\end{Shaded}

\begin{verbatim}
Analysis of Variance Table

Response: item_test_SCALED
                  Df Sum Sq Mean Sq F value Pr(>F)    
pretty_condition   1    532     532    14.3 0.0002 ***
Residuals        202   7493      37                   
---
Signif. codes:  0 '***' 0.001 '**' 0.01 '*' 0.05 '.' 0.1 ' ' 1
\end{verbatim}

\begin{Shaded}
\begin{Highlighting}[]
\FunctionTok{paste}\NormalTok{(}\StringTok{"Confidence Interval on Parameter Estimates"}\NormalTok{)}
\end{Highlighting}
\end{Shaded}

\begin{verbatim}
[1] "Confidence Interval on Parameter Estimates"
\end{verbatim}

\begin{Shaded}
\begin{Highlighting}[]
\FunctionTok{confint}\NormalTok{(rep.test\_scaled.lm1)}
\end{Highlighting}
\end{Shaded}

\begin{verbatim}
                        2.5 % 97.5 %
(Intercept)             -5.26  -2.81
pretty_conditionimpasse  1.55   4.92
\end{verbatim}

\begin{Shaded}
\begin{Highlighting}[]
\CommentTok{\# report(m1) \#sanity check}
\CommentTok{\#print model equation}
\NormalTok{eq }\OtherTok{\textless{}{-}} \FunctionTok{extract\_eq}\NormalTok{(rep.test\_scaled.lm1, }\AttributeTok{use\_coefs =} \ConstantTok{TRUE}\NormalTok{)}
\end{Highlighting}
\end{Shaded}

\textbf{Model equation} ,

\begin{equation}
, \operatorname{\widehat{item\_test\_SCALED}} = -4.04 + 3.24(\operatorname{pretty\_condition}_{\operatorname{impasse}}), 
\end{equation}

\textbf{For online replication} an OLS linear regression predicting
test-phase (\% correct) by experimental condition explains a
statistically significant though small 7\% variance in accuracy
(F(1,202) = 14.3, p \textless{} 0.001). The estimated beta coefficient
(\(\beta\) = 3.24, 95\% CI {[}1.55, 4.62{]}) predicts that participants
in the impasse condition will on average 3 points higher than those in
the control condition.

\begin{Shaded}
\begin{Highlighting}[]
\CommentTok{\#MODEL ESTIMATES WITH UNCERTAINTY}

\CommentTok{\#setup references }
\NormalTok{m }\OtherTok{\textless{}{-}}\NormalTok{ rep.test\_scaled.lm1}
\NormalTok{df }\OtherTok{\textless{}{-}}\NormalTok{ df\_online }
\NormalTok{x }\OtherTok{\textless{}{-}}\NormalTok{ df\_online}\SpecialCharTok{$}\NormalTok{item\_test\_SCALED}

\CommentTok{\#uncertainty model visualization}
\NormalTok{df }\SpecialCharTok{\%\textgreater{}\%}
  \FunctionTok{data\_grid}\NormalTok{(pretty\_condition) }\SpecialCharTok{\%\textgreater{}\%}
  \FunctionTok{augment}\NormalTok{(m, }\AttributeTok{newdata =}\NormalTok{ ., }\AttributeTok{se\_fit =} \ConstantTok{TRUE}\NormalTok{) }\SpecialCharTok{\%\textgreater{}\%}
  \FunctionTok{ggplot}\NormalTok{(}\FunctionTok{aes}\NormalTok{(}\AttributeTok{y =}\NormalTok{ pretty\_condition)) }\SpecialCharTok{+}
  \FunctionTok{stat\_halfeye}\NormalTok{(}
    \FunctionTok{aes}\NormalTok{(}\AttributeTok{xdist =} \FunctionTok{dist\_student\_t}\NormalTok{(}\AttributeTok{df =} \FunctionTok{df.residual}\NormalTok{(m), }
        \AttributeTok{mu =}\NormalTok{ .fitted, }\AttributeTok{sigma =}\NormalTok{ .se.fit)), }\AttributeTok{scale =}\NormalTok{ .}\DecValTok{5}\NormalTok{) }\SpecialCharTok{+}
  \CommentTok{\# add raw data in too (scale = .5 above adjusts the halfeye height so}
  \CommentTok{\# that the data fit in as well)}
  \FunctionTok{geom\_jitter}\NormalTok{(}\FunctionTok{aes}\NormalTok{(}\AttributeTok{x =}\NormalTok{ x), }\AttributeTok{data =}\NormalTok{ df, }\AttributeTok{pch =} \StringTok{"|"}\NormalTok{, }\AttributeTok{size =} \DecValTok{2}\NormalTok{, }
              \AttributeTok{position =}   \FunctionTok{position\_nudge}\NormalTok{(}\AttributeTok{y =} \SpecialCharTok{{-}}\NormalTok{.}\DecValTok{15}\NormalTok{), }\AttributeTok{alpha =} \FloatTok{0.5}\NormalTok{) }\SpecialCharTok{+}  
  \FunctionTok{labs}\NormalTok{ (}\AttributeTok{title =} \StringTok{"Model Estimates with Uncertainty"}\NormalTok{, }\AttributeTok{x =} \StringTok{"model coefficient"}\NormalTok{) }\SpecialCharTok{+} 
  \FunctionTok{theme\_minimal}\NormalTok{()}
\end{Highlighting}
\end{Shaded}

\begin{figure}[H]

{\centering \includegraphics{analysis/SGC3A/4_sgc3A_hypotesting_files/figure-pdf/unnamed-chunk-17-1.pdf}

}

\end{figure}

\hypertarget{diagnostics-3}{%
\subparagraph{Diagnostics}\label{diagnostics-3}}

\begin{Shaded}
\begin{Highlighting}[]
\CommentTok{\#model diagnostics}
\FunctionTok{check\_model}\NormalTok{(rep.test\_scaled.lm1, }\AttributeTok{panel =} \ConstantTok{TRUE}\NormalTok{)}
\end{Highlighting}
\end{Shaded}

\begin{figure}[H]

{\centering \includegraphics{analysis/SGC3A/4_sgc3A_hypotesting_files/figure-pdf/DIAG-TEST-SCALED-ONLINE-1.pdf}

}

\end{figure}

(1) RESIDUAL DISTRIBUTION: \ensuremath{1.891\times 10^{-12}}\\
(2) HOMOGENEITY: 0.097\\
(3) HETERSCEDASTICITY: 0.098\\
(4) AUTOCORRELATION: 0.81 (5) OUTLIERS: FALSE, FALSE, FALSE, FALSE,
FALSE, FALSE, FALSE, FALSE, FALSE, FALSE, FALSE, FALSE, FALSE, FALSE,
FALSE, FALSE, FALSE, FALSE, FALSE, FALSE, FALSE, FALSE, FALSE, FALSE,
FALSE, FALSE, FALSE, FALSE, FALSE, FALSE, FALSE, FALSE, FALSE, FALSE,
FALSE, FALSE, FALSE, FALSE, FALSE, FALSE, FALSE, FALSE, FALSE, FALSE,
FALSE, FALSE, FALSE, FALSE, FALSE, FALSE, FALSE, FALSE, FALSE, FALSE,
FALSE, FALSE, FALSE, FALSE, FALSE, FALSE, FALSE, FALSE, FALSE, FALSE,
FALSE, FALSE, FALSE, FALSE, FALSE, FALSE, FALSE, FALSE, FALSE, FALSE,
FALSE, FALSE, FALSE, FALSE, FALSE, FALSE, FALSE, FALSE, FALSE, FALSE,
FALSE, FALSE, FALSE, FALSE, FALSE, FALSE, FALSE, FALSE, FALSE, FALSE,
FALSE, FALSE, FALSE, FALSE, FALSE, FALSE, FALSE, FALSE, FALSE, FALSE,
FALSE, FALSE, FALSE, FALSE, FALSE, FALSE, FALSE, FALSE, FALSE, FALSE,
FALSE, FALSE, FALSE, FALSE, FALSE, FALSE, FALSE, FALSE, FALSE, FALSE,
FALSE, FALSE, FALSE, FALSE, FALSE, FALSE, FALSE, FALSE, FALSE, FALSE,
FALSE, FALSE, FALSE, FALSE, FALSE, FALSE, FALSE, FALSE, FALSE, FALSE,
FALSE, FALSE, FALSE, FALSE, FALSE, FALSE, FALSE, FALSE, FALSE, FALSE,
FALSE, FALSE, FALSE, FALSE, FALSE, FALSE, FALSE, FALSE, FALSE, FALSE,
FALSE, FALSE, FALSE, FALSE, FALSE, FALSE, FALSE, FALSE, FALSE, FALSE,
FALSE, FALSE, FALSE, FALSE, FALSE, FALSE, FALSE, FALSE, FALSE, FALSE,
FALSE, FALSE, FALSE, FALSE, FALSE, FALSE, FALSE, FALSE, FALSE, FALSE,
FALSE, FALSE, FALSE, FALSE, FALSE, FALSE, FALSE, FALSE, FALSE, FALSE

\hypertarget{inference-3}{%
\subparagraph{Inference}\label{inference-3}}

\textbf{For online replication} an OLS Linear Regression on SCALED SCORE
in the TEST PHASE shows that condition explains a small but
statistically significant amount of variance (impasse \textgreater{}
control). However, the model is a poor fit to the data: (1) the model
predictions for each group are closer to the anitimode of each of
distribution than the group modes, and (2) the distribution of residuals
is not normal. (Assumptions of homogenity of variance across groups, and
homogeneity of variance in residuals are met)

\hypertarget{todo-robust-alternative}{%
\section{TODO ROBUST ALTERNATIVE}\label{todo-robust-alternative}}

\hypertarget{wip-exploring-1}{%
\subsection{WIP EXPLORING}\label{wip-exploring-1}}

\hypertarget{beta-regression-correct}{%
\subsubsection{Beta Regression (\%
Correct)}\label{beta-regression-correct}}

Beta regression on \% correct (with standard transformation for
including {[}0,1{]})

\begin{Shaded}
\begin{Highlighting}[]
\FunctionTok{library}\NormalTok{(betareg)}

\NormalTok{sub }\OtherTok{\textless{}{-}}\NormalTok{ df\_subjects }\SpecialCharTok{\%\textgreater{}\%}\NormalTok{ dplyr}\SpecialCharTok{::}\FunctionTok{select}\NormalTok{(condition, DV\_percent\_NABS)}
\NormalTok{n }\OtherTok{=} \FunctionTok{nrow}\NormalTok{(sub) }\SpecialCharTok{\%\textgreater{}\%} \FunctionTok{unlist}\NormalTok{()}
\NormalTok{sub}\SpecialCharTok{$}\NormalTok{dv\_transformed }\OtherTok{=}\NormalTok{ (sub}\SpecialCharTok{$}\NormalTok{DV\_percent\_NABS }\SpecialCharTok{*}\NormalTok{ (n}\DecValTok{{-}1}\NormalTok{) }\SpecialCharTok{+} \FloatTok{0.5}\NormalTok{)}\SpecialCharTok{/}\NormalTok{n}

\FunctionTok{histogram}\NormalTok{(sub}\SpecialCharTok{$}\NormalTok{dv\_transformed)}
\end{Highlighting}
\end{Shaded}

\begin{figure}[H]

{\centering \includegraphics{analysis/SGC3A/4_sgc3A_hypotesting_files/figure-pdf/unnamed-chunk-19-1.pdf}

}

\end{figure}

\begin{Shaded}
\begin{Highlighting}[]
\FunctionTok{histogram}\NormalTok{(df\_subjects}\SpecialCharTok{$}\NormalTok{DV\_percent\_NABS)}
\end{Highlighting}
\end{Shaded}

\begin{figure}[H]

{\centering \includegraphics{analysis/SGC3A/4_sgc3A_hypotesting_files/figure-pdf/unnamed-chunk-19-2.pdf}

}

\end{figure}

\begin{Shaded}
\begin{Highlighting}[]
\NormalTok{mb }\OtherTok{\textless{}{-}} \FunctionTok{betareg}\NormalTok{(dv\_transformed }\SpecialCharTok{\textasciitilde{}}\NormalTok{ condition, }\AttributeTok{data =}\NormalTok{ sub)}

\FunctionTok{summary}\NormalTok{(mb)}
\end{Highlighting}
\end{Shaded}

\begin{verbatim}

Call:
betareg(formula = dv_transformed ~ condition, data = sub)

Standardized weighted residuals 2:
   Min     1Q Median     3Q    Max 
-1.057 -0.453 -0.216  0.541  1.690 

Coefficients (mean model with logit link):
             Estimate Std. Error z value Pr(>|z|)    
(Intercept)    -0.969      0.108   -8.97   <2e-16 ***
condition121    0.556      0.143    3.89   0.0001 ***

Phi coefficients (precision model with identity link):
      Estimate Std. Error z value Pr(>|z|)    
(phi)   0.6604     0.0425    15.5   <2e-16 ***
---
Signif. codes:  0 '***' 0.001 '**' 0.01 '*' 0.05 '.' 0.1 ' ' 1 

Type of estimator: ML (maximum likelihood)
Log-likelihood:  506 on 3 Df
Pseudo R-squared: 0.0725
Number of iterations: 12 (BFGS) + 1 (Fisher scoring) 
\end{verbatim}

\begin{Shaded}
\begin{Highlighting}[]
\FunctionTok{plot}\NormalTok{(mb)}
\end{Highlighting}
\end{Shaded}

\begin{figure}[H]

{\centering \includegraphics{analysis/SGC3A/4_sgc3A_hypotesting_files/figure-pdf/unnamed-chunk-19-3.pdf}

}

\end{figure}

\begin{figure}[H]

{\centering \includegraphics{analysis/SGC3A/4_sgc3A_hypotesting_files/figure-pdf/unnamed-chunk-19-4.pdf}

}

\end{figure}

\begin{figure}[H]

{\centering \includegraphics{analysis/SGC3A/4_sgc3A_hypotesting_files/figure-pdf/unnamed-chunk-19-5.pdf}

}

\end{figure}

\begin{figure}[H]

{\centering \includegraphics{analysis/SGC3A/4_sgc3A_hypotesting_files/figure-pdf/unnamed-chunk-19-6.pdf}

}

\end{figure}

\hypertarget{linear-regression-transformed-correct}{%
\subsubsection{Linear Regression (Transformed \# Correct)
??}\label{linear-regression-transformed-correct}}

\hypertarget{h1b-q1-accuracy}{%
\section{H1B \textbar{} Q1 ACCURACY}\label{h1b-q1-accuracy}}

The graph comprehension tasks includes 15 questions completed in
sequence. But the first question the reader encounters (Q1) is the most
important, as it is their \emph{first exposure} to the unconventional
triangular coordinate system.

TODO: - does impasse yield different exploration behavior? (characterize
mouse) - does impasse yield more time on task? (characterize response
time ? number of answers then de-selected?)

TODO: Think about characterizing how variable the interpretations are
across a participant. Do they form an interpretation and hold it
constant? Or do they change question to question.

\hypertarget{response-accuracy-of-first-question-by-condition}{%
\subsection{Response Accuracy of First Question by
Condition}\label{response-accuracy-of-first-question-by-condition}}

\hypertarget{chi-square-accuracy-condition}{%
\subsubsection{Chi Square \textbar{} Accuracy \textasciitilde{}
Condition}\label{chi-square-accuracy-condition}}

\begin{longtable}[]{@{}
  >{\raggedright\arraybackslash}p{(\columnwidth - 2\tabcolsep) * \real{0.0440}}
  >{\raggedright\arraybackslash}p{(\columnwidth - 2\tabcolsep) * \real{0.9560}}@{}}
\toprule()
\begin{minipage}[b]{\linewidth}\raggedright
Research Question
\end{minipage} & \begin{minipage}[b]{\linewidth}\raggedright
Does the frequency of correct (vs) incorrect responses on the first
question differ by condition? {[}Is response accuracy independent of
condition?{]}
\end{minipage} \\
\midrule()
\endhead
\textbf{Analysis Strategy} & Chi-Square test of independence on outcome
\texttt{score\_niceABS} by \texttt{condition} for \texttt{df\_items}
where \texttt{q\ ==\ 1} \\
\textbf{Justification} & (0) simplest method to examine independence of
two categorical factors; logistic regression is recommended for binomial
\textasciitilde{} continuous

(1) independence assumption : as we only consider responses on the first
question, each observation corresponds to an individual subject, and are
thus independent

(2) frequency size assumption : expected frequency in each cell of the
contingency table is greater than 5 (more than 5 correct , more than 5
incorrect responses) \\
\textbf{Steps} & (1) Express raw data as contingency table \& visualize

(2) Calculate Chi-Squared Statistic and p-value

(3) Interpret Odds-Ratio as effect size \\
\textbf{Inference} & \textbf{Lab} For the (In Person) (n=126) the
Pearson's Chi-squared test (of independence) indicates a relationship
between response accuracy on the first question and experimental
condition approaching statistical significance, \(\chi^2\) (1) = 10.3, p
= 0.07. In this particular data sample, the odds ratio (2.18, p = 0.055,
95\% CI {[}0.982, +Inf{]}) indicates that the odds of producing a
correct response on the first question were 2.18 times greater if a
subject was in the impasse condition, than in the control condition.

\textbf{Online} For online data collection (n=204), a Pearson's
Chi-squared test (of independence) indicates a statistically significant
relationship between response accuracy on the first question and
experimental condition, \(\chi^2\) (1) = 7.26, p = 0.009. The odds ratio
(2.68, p = 0.005, 95\% CI {[}1.37, +Inf{]}) indicates that the odds of
producing a correct response on the first question were 2.68 times
greater if a subject was in the impasse condition, than in the control
condition. \\
\bottomrule()
\end{longtable}

\begin{Shaded}
\begin{Highlighting}[]
\CommentTok{\#FITER THE DATASET}
\NormalTok{df }\OtherTok{=}\NormalTok{ df\_items }\SpecialCharTok{\%\textgreater{}\%} \FunctionTok{filter}\NormalTok{(q}\SpecialCharTok{==}\DecValTok{1}\NormalTok{) }

\CommentTok{\#PROPORTIONAL BAR CHART}
\FunctionTok{gf\_props}\NormalTok{(}\SpecialCharTok{\textasciitilde{}}\NormalTok{score\_niceABS, }\AttributeTok{data =}\NormalTok{ df, }\AttributeTok{fill =} \SpecialCharTok{\textasciitilde{}}\NormalTok{mode) }\SpecialCharTok{\%\textgreater{}\%} 
  \FunctionTok{gf\_facet\_grid}\NormalTok{(mode}\SpecialCharTok{\textasciitilde{}}\NormalTok{condition, }\AttributeTok{labeller =}\NormalTok{ label\_both) }\SpecialCharTok{+}
  \FunctionTok{labs}\NormalTok{(}\AttributeTok{x =} \StringTok{"Correct Response on Q 1"}\NormalTok{,}
       \AttributeTok{title =} \StringTok{"Accuracy on First Question by Condition (Both Modalities)"}\NormalTok{,}
       \AttributeTok{subtitle=}\StringTok{"Impasse Condition yields a greater proportion of correct responses than control "}\NormalTok{)}\SpecialCharTok{+}
  \FunctionTok{theme\_minimal}\NormalTok{()}\SpecialCharTok{+} \FunctionTok{theme}\NormalTok{(}\AttributeTok{legend.position =} \StringTok{"none"}\NormalTok{)}
\end{Highlighting}
\end{Shaded}

\begin{figure}[H]

{\centering \includegraphics{analysis/SGC3A/4_sgc3A_hypotesting_files/figure-pdf/VIS-Q1ACC.by.COND-bar-1.pdf}

}

\end{figure}

A proportional bar chart visualizing the proportion of incorrect (x =0)
vs correct (x = 1) responses in each condition (right/left facet) for
each data collection modality (top/bottom) reveal that the pattern of
responses appear the same regardless of the data collection modality. In
both data collection sessions, the proportion of incorrect responses is
much greater than the proportion of correct responses, regardless of
condition. In the impasse condition, the difference in proportions is
smaller than the control condition (i.e.~There are more correct
responses in the impasse condition than the control condition).

\begin{Shaded}
\begin{Highlighting}[]
\CommentTok{\#MOSAIC PLOT}
\NormalTok{vcd}\SpecialCharTok{::}\FunctionTok{mosaic}\NormalTok{(}\AttributeTok{main=}\StringTok{"Accuracy on First Question by Condition (Both Modalities)"}\NormalTok{,}
            \AttributeTok{data =}\NormalTok{ df, score\_niceABS }\SpecialCharTok{\textasciitilde{}}\NormalTok{ condition, }\AttributeTok{rot\_labels=}\FunctionTok{c}\NormalTok{(}\DecValTok{0}\NormalTok{,}\DecValTok{90}\NormalTok{,}\DecValTok{0}\NormalTok{,}\DecValTok{0}\NormalTok{),}
            \AttributeTok{offset\_varnames =} \FunctionTok{c}\NormalTok{(}\AttributeTok{left =} \FloatTok{4.5}\NormalTok{), }\AttributeTok{offset\_labels =} \FunctionTok{c}\NormalTok{(}\AttributeTok{left =} \SpecialCharTok{{-}}\FloatTok{0.5}\NormalTok{),}\AttributeTok{just\_labels =} \StringTok{"right"}\NormalTok{,}
            \AttributeTok{spacing =} \FunctionTok{spacing\_dimequal}\NormalTok{(}\FunctionTok{unit}\NormalTok{(}\DecValTok{1}\SpecialCharTok{:}\DecValTok{2}\NormalTok{, }\StringTok{"lines"}\NormalTok{)))}
\end{Highlighting}
\end{Shaded}

\begin{figure}[H]

{\centering \includegraphics{analysis/SGC3A/4_sgc3A_hypotesting_files/figure-pdf/VIS-Q1ACC.by.COND-mosaic-1.pdf}

}

\end{figure}

\begin{Shaded}
\begin{Highlighting}[]
\CommentTok{\#PRINT CONTINGENCY TABLE}
\NormalTok{title }\OtherTok{=} \StringTok{"Proportion of Correct Responses On First Item (Both Modalities)"}
\NormalTok{item.contingency }\OtherTok{\textless{}{-}}\NormalTok{  df }\SpecialCharTok{\%\textgreater{}\%}\NormalTok{ dplyr}\SpecialCharTok{::}\FunctionTok{select}\NormalTok{(condition, score\_niceABS) }\SpecialCharTok{\%\textgreater{}\%} \FunctionTok{table}\NormalTok{() }\SpecialCharTok{\%\textgreater{}\%} \FunctionTok{prop.table}\NormalTok{() }\SpecialCharTok{\%\textgreater{}\%} \FunctionTok{addmargins}\NormalTok{()}
\NormalTok{item.contingency }\SpecialCharTok{\%\textgreater{}\%} \FunctionTok{kbl}\NormalTok{ (}\AttributeTok{caption =}\NormalTok{ title) }\SpecialCharTok{\%\textgreater{}\%} \FunctionTok{kable\_classic}\NormalTok{()}
\end{Highlighting}
\end{Shaded}

\begin{table}

\caption{Proportion of Correct Responses On First Item (Both Modalities)}
\centering
\begin{tabular}[t]{l|r|r|r}
\hline
  & 0 & 1 & Sum\\
\hline
111 & 0.412 & 0.067 & 0.479\\
\hline
121 & 0.373 & 0.148 & 0.521\\
\hline
Sum & 0.785 & 0.215 & 1.000\\
\hline
\end{tabular}
\end{table}

A mosaic plot condition by response accuracy on the first question
(across both data collection modalities) reveals the same pattern (the
mosaic plot is an alternative visualization technique to the
proportional bar chart). The relative size of condition boxes (111 vs
121) reflects that the sample is roughly evenly split across
experimental conditions. The difference in size between 0 (incorrect)
and 1 (correct) reflects that the proportion of correct responses (1) is
greater in the impasse condition (121).

Next, we compute a contingency table and Pearson's Chi-Squared test for
each data collection modality.

\begin{Shaded}
\begin{Highlighting}[]
\NormalTok{df }\OtherTok{=}\NormalTok{ df\_items }\SpecialCharTok{\%\textgreater{}\%} \FunctionTok{filter}\NormalTok{(q}\SpecialCharTok{==}\DecValTok{1}\NormalTok{) }\SpecialCharTok{\%\textgreater{}\%} \FunctionTok{filter}\NormalTok{(mode }\SpecialCharTok{==} \StringTok{"lab{-}synch"}\NormalTok{)}
\FunctionTok{CrossTable}\NormalTok{( }\AttributeTok{x =}\NormalTok{ df}\SpecialCharTok{$}\NormalTok{condition, }\AttributeTok{y =}\NormalTok{ df}\SpecialCharTok{$}\NormalTok{score\_niceABS, }\AttributeTok{fisher =} \ConstantTok{TRUE}\NormalTok{, }\AttributeTok{chisq=}\ConstantTok{TRUE}\NormalTok{, }\AttributeTok{expected =} \ConstantTok{TRUE}\NormalTok{, }\AttributeTok{sresid =} \ConstantTok{TRUE}\NormalTok{)}
\end{Highlighting}
\end{Shaded}

\begin{verbatim}

 
   Cell Contents
|-------------------------|
|                       N |
|              Expected N |
| Chi-square contribution |
|           N / Row Total |
|           N / Col Total |
|         N / Table Total |
|-------------------------|

 
Total Observations in Table:  126 

 
             | df$score_niceABS 
df$condition |         0 |         1 | Row Total | 
-------------|-----------|-----------|-----------|
         111 |        52 |        10 |        62 | 
             |    47.730 |    14.270 |           | 
             |     0.382 |     1.278 |           | 
             |     0.839 |     0.161 |     0.492 | 
             |     0.536 |     0.345 |           | 
             |     0.413 |     0.079 |           | 
-------------|-----------|-----------|-----------|
         121 |        45 |        19 |        64 | 
             |    49.270 |    14.730 |           | 
             |     0.370 |     1.238 |           | 
             |     0.703 |     0.297 |     0.508 | 
             |     0.464 |     0.655 |           | 
             |     0.357 |     0.151 |           | 
-------------|-----------|-----------|-----------|
Column Total |        97 |        29 |       126 | 
             |     0.770 |     0.230 |           | 
-------------|-----------|-----------|-----------|

 
Statistics for All Table Factors


Pearson's Chi-squared test 
------------------------------------------------------------
Chi^2 =  3.27     d.f. =  1     p =  0.0707 

Pearson's Chi-squared test with Yates' continuity correction 
------------------------------------------------------------
Chi^2 =  2.55     d.f. =  1     p =  0.111 

 
Fisher's Exact Test for Count Data
------------------------------------------------------------
Sample estimate odds ratio:  2.18 

Alternative hypothesis: true odds ratio is not equal to 1
p =  0.0909 
95% confidence interval:  0.86 5.84 

Alternative hypothesis: true odds ratio is less than 1
p =  0.979 
95% confidence interval:  0 5.03 

Alternative hypothesis: true odds ratio is greater than 1
p =  0.0547 
95% confidence interval:  0.982 Inf 


 
\end{verbatim}

\textbf{For the (In Person)} (n=126) the Pearson's Chi-squared test (of
independence) indicates a relationship between response accuracy on the
first question and experimental condition approaching statistical
significance, \(\chi^2\) (1) = 10.3, p = 0.07. Thus we have insufficient
evidence to reject the null hypothesis that the odds ratio is equal to
1. In this particular data sample, the odds ratio (Odds Ratio = 2.18, p
= 0.055, 95\% CI {[}0.982, +Inf{]}) indicates that the odds of producing
a correct response on the first question were 2.18 times greater if a
subject was in the impasse condition, than in the control condition .

\begin{Shaded}
\begin{Highlighting}[]
\NormalTok{df }\OtherTok{=}\NormalTok{ df\_items }\SpecialCharTok{\%\textgreater{}\%} \FunctionTok{filter}\NormalTok{(q}\SpecialCharTok{==}\DecValTok{1}\NormalTok{) }\SpecialCharTok{\%\textgreater{}\%} \FunctionTok{filter}\NormalTok{(mode }\SpecialCharTok{==} \StringTok{"asynch"}\NormalTok{)}
\FunctionTok{CrossTable}\NormalTok{( }\AttributeTok{x =}\NormalTok{ df}\SpecialCharTok{$}\NormalTok{condition, }\AttributeTok{y =}\NormalTok{ df}\SpecialCharTok{$}\NormalTok{score\_niceABS, }\AttributeTok{fisher =} \ConstantTok{TRUE}\NormalTok{, }\AttributeTok{chisq=}\ConstantTok{TRUE}\NormalTok{, }\AttributeTok{expected =} \ConstantTok{TRUE}\NormalTok{, }\AttributeTok{sresid =} \ConstantTok{TRUE}\NormalTok{)}
\end{Highlighting}
\end{Shaded}

\begin{verbatim}

 
   Cell Contents
|-------------------------|
|                       N |
|              Expected N |
| Chi-square contribution |
|           N / Row Total |
|           N / Col Total |
|         N / Table Total |
|-------------------------|

 
Total Observations in Table:  204 

 
             | df$score_niceABS 
df$condition |         0 |         1 | Row Total | 
-------------|-----------|-----------|-----------|
         111 |        84 |        12 |        96 | 
             |    76.235 |    19.765 |           | 
             |     0.791 |     3.050 |           | 
             |     0.875 |     0.125 |     0.471 | 
             |     0.519 |     0.286 |           | 
             |     0.412 |     0.059 |           | 
-------------|-----------|-----------|-----------|
         121 |        78 |        30 |       108 | 
             |    85.765 |    22.235 |           | 
             |     0.703 |     2.711 |           | 
             |     0.722 |     0.278 |     0.529 | 
             |     0.481 |     0.714 |           | 
             |     0.382 |     0.147 |           | 
-------------|-----------|-----------|-----------|
Column Total |       162 |        42 |       204 | 
             |     0.794 |     0.206 |           | 
-------------|-----------|-----------|-----------|

 
Statistics for All Table Factors


Pearson's Chi-squared test 
------------------------------------------------------------
Chi^2 =  7.26     d.f. =  1     p =  0.00707 

Pearson's Chi-squared test with Yates' continuity correction 
------------------------------------------------------------
Chi^2 =  6.35     d.f. =  1     p =  0.0117 

 
Fisher's Exact Test for Count Data
------------------------------------------------------------
Sample estimate odds ratio:  2.68 

Alternative hypothesis: true odds ratio is not equal to 1
p =  0.00894 
95% confidence interval:  1.23 6.17 

Alternative hypothesis: true odds ratio is less than 1
p =  0.998 
95% confidence interval:  0 5.42 

Alternative hypothesis: true odds ratio is greater than 1
p =  0.00539 
95% confidence interval:  1.37 Inf 


 
\end{verbatim}

\textbf{For online data collection} (n=204), a Pearson's Chi-squared
test (of independence) indicates a statistically significant
relationship between response accuracy on the first question and
experimental condition, \(\chi^2\) (1) = 7.26, p = 0.009. Thus we have
sufficient evidence to reject the null hypothesis that the odds ratio is
equal to 1. The odds ratio (Odds Ratio = 2.68, p = 0.005, 95\% CI
{[}1.37, +Inf{]}) indicates that the odds of producing a correct
response on the first question were 2.68 times greater if a subject was
in the impasse condition, than in the control condition .

\begin{Shaded}
\begin{Highlighting}[]
\NormalTok{df }\OtherTok{=}\NormalTok{ df\_items }\SpecialCharTok{\%\textgreater{}\%} \FunctionTok{filter}\NormalTok{(q}\SpecialCharTok{==}\DecValTok{1}\NormalTok{) }
\FunctionTok{CrossTable}\NormalTok{( }\AttributeTok{x =}\NormalTok{ df}\SpecialCharTok{$}\NormalTok{condition, }\AttributeTok{y =}\NormalTok{ df}\SpecialCharTok{$}\NormalTok{score\_niceABS, }\AttributeTok{fisher =} \ConstantTok{TRUE}\NormalTok{, }\AttributeTok{chisq=}\ConstantTok{TRUE}\NormalTok{, }\AttributeTok{expected =} \ConstantTok{TRUE}\NormalTok{, }\AttributeTok{sresid =} \ConstantTok{TRUE}\NormalTok{)}
\end{Highlighting}
\end{Shaded}

\begin{verbatim}

 
   Cell Contents
|-------------------------|
|                       N |
|              Expected N |
| Chi-square contribution |
|           N / Row Total |
|           N / Col Total |
|         N / Table Total |
|-------------------------|

 
Total Observations in Table:  330 

 
             | df$score_niceABS 
df$condition |         0 |         1 | Row Total | 
-------------|-----------|-----------|-----------|
         111 |       136 |        22 |       158 | 
             |   124.006 |    33.994 |           | 
             |     1.160 |     4.232 |           | 
             |     0.861 |     0.139 |     0.479 | 
             |     0.525 |     0.310 |           | 
             |     0.412 |     0.067 |           | 
-------------|-----------|-----------|-----------|
         121 |       123 |        49 |       172 | 
             |   134.994 |    37.006 |           | 
             |     1.066 |     3.887 |           | 
             |     0.715 |     0.285 |     0.521 | 
             |     0.475 |     0.690 |           | 
             |     0.373 |     0.148 |           | 
-------------|-----------|-----------|-----------|
Column Total |       259 |        71 |       330 | 
             |     0.785 |     0.215 |           | 
-------------|-----------|-----------|-----------|

 
Statistics for All Table Factors


Pearson's Chi-squared test 
------------------------------------------------------------
Chi^2 =  10.3     d.f. =  1     p =  0.0013 

Pearson's Chi-squared test with Yates' continuity correction 
------------------------------------------------------------
Chi^2 =  9.5     d.f. =  1     p =  0.00205 

 
Fisher's Exact Test for Count Data
------------------------------------------------------------
Sample estimate odds ratio:  2.46 

Alternative hypothesis: true odds ratio is not equal to 1
p =  0.00131 
95% confidence interval:  1.37 4.53 

Alternative hypothesis: true odds ratio is less than 1
p =  1 
95% confidence interval:  0 4.12 

Alternative hypothesis: true odds ratio is greater than 1
p =  0.000928 
95% confidence interval:  1.49 Inf 


 
\end{verbatim}

\textbf{Combining data across both sessions} (n=330), a Pearson's
Chi-squared test suggests a statistically significant relationship
between response accuracy on the first question and experimental
condition, \(\chi_2\) (1) = 10.3, p = 0.001. The sample odds ratio
(2.46, p = 0.001, 95\% CI {[}1.37, 4.53{]}) indicates that the odds of
providing a correct response to the first question are 2.46 higher for
subjects in the impasse condition than those in the control condition.

\hypertarget{h1c-q1-latency}{%
\section{H1C \textbar{} Q1 LATENCY}\label{h1c-q1-latency}}

\begin{longtable}[]{@{}
  >{\raggedright\arraybackslash}p{(\columnwidth - 2\tabcolsep) * \real{0.3333}}
  >{\raggedright\arraybackslash}p{(\columnwidth - 2\tabcolsep) * \real{0.6389}}@{}}
\toprule()
\begin{minipage}[b]{\linewidth}\raggedright
Research Question
\end{minipage} & \begin{minipage}[b]{\linewidth}\raggedright
\end{minipage} \\
\midrule()
\endhead
\textbf{Hypothesis} & \\
\textbf{Analysis Strategy} & \\
\textbf{Alternatives} & \\
\textbf{Inference} & \\
\bottomrule()
\end{longtable}

\hypertarget{q1-response-latency}{%
\subsection{Q1 Response Latency}\label{q1-response-latency}}

\hypertarget{linear-regression-2}{%
\subsubsection{Linear Regression}\label{linear-regression-2}}

\hypertarget{in-person-2}{%
\paragraph{(In Person)}\label{in-person-2}}

\hypertarget{visualization-4}{%
\subparagraph{Visualization}\label{visualization-4}}

\begin{Shaded}
\begin{Highlighting}[]
\CommentTok{\#HISTOGRAM}
\NormalTok{stats }\OtherTok{=}\NormalTok{ df\_lab }\SpecialCharTok{\%\textgreater{}\%} \FunctionTok{group\_by}\NormalTok{(pretty\_condition) }\SpecialCharTok{\%\textgreater{}\%}\NormalTok{ dplyr}\SpecialCharTok{::}\FunctionTok{summarise}\NormalTok{(}\AttributeTok{mean =} \FunctionTok{mean}\NormalTok{(item\_q1\_rt))}
\FunctionTok{gf\_dhistogram}\NormalTok{(}\SpecialCharTok{\textasciitilde{}}\NormalTok{item\_q1\_rt, }\AttributeTok{fill =} \SpecialCharTok{\textasciitilde{}}\NormalTok{pretty\_condition, }\AttributeTok{data =}\NormalTok{ df\_lab) }\SpecialCharTok{\%\textgreater{}\%} \FunctionTok{gf\_facet\_grid}\NormalTok{(}\SpecialCharTok{\textasciitilde{}}\NormalTok{pretty\_condition) }\SpecialCharTok{\%\textgreater{}\%} 
  \FunctionTok{gf\_vline}\NormalTok{(}\AttributeTok{data =}\NormalTok{ stats, }\AttributeTok{xintercept =} \SpecialCharTok{\textasciitilde{}}\NormalTok{mean, }\AttributeTok{color =} \StringTok{"red"}\NormalTok{) }\SpecialCharTok{+}
  \FunctionTok{labs}\NormalTok{(}\AttributeTok{x =} \StringTok{"Response Time (seconds)"}\NormalTok{,}
       \CommentTok{\# y = "proportion of participants",}
       \AttributeTok{title =} \StringTok{"(LAB) First Question Response Time"}\NormalTok{,}
       \AttributeTok{subtitle =} \StringTok{""}\NormalTok{) }\SpecialCharTok{+} 
  \FunctionTok{theme\_minimal}\NormalTok{()}
\end{Highlighting}
\end{Shaded}

\begin{figure}[H]

{\centering \includegraphics{analysis/SGC3A/4_sgc3A_hypotesting_files/figure-pdf/VIS-TEST-Q1TIME-1.pdf}

}

\end{figure}

\hypertarget{model-4}{%
\subparagraph{Model}\label{model-4}}

\begin{Shaded}
\begin{Highlighting}[]
\CommentTok{\#SCORE predicted by CONDITION}
\NormalTok{lab.q1t.lm1 }\OtherTok{\textless{}{-}} \FunctionTok{lm}\NormalTok{(item\_q1\_rt }\SpecialCharTok{\textasciitilde{}}\NormalTok{ condition, }\AttributeTok{data =}\NormalTok{ df\_lab)}
\FunctionTok{paste}\NormalTok{(}\StringTok{"Model"}\NormalTok{)}
\end{Highlighting}
\end{Shaded}

\begin{verbatim}
[1] "Model"
\end{verbatim}

\begin{Shaded}
\begin{Highlighting}[]
\FunctionTok{summary}\NormalTok{(lab.q1t.lm1)}
\end{Highlighting}
\end{Shaded}

\begin{verbatim}

Call:
lm(formula = item_q1_rt ~ condition, data = df_lab)

Residuals:
   Min     1Q Median     3Q    Max 
-44.40 -17.46  -4.72   9.20 109.73 

Coefficients:
             Estimate Std. Error t value Pr(>|t|)    
(Intercept)     37.20       3.21    11.6   <2e-16 ***
condition121    14.43       4.51     3.2   0.0017 ** 
---
Signif. codes:  0 '***' 0.001 '**' 0.01 '*' 0.05 '.' 0.1 ' ' 1

Residual standard error: 25.3 on 124 degrees of freedom
Multiple R-squared:  0.0763,    Adjusted R-squared:  0.0688 
F-statistic: 10.2 on 1 and 124 DF,  p-value: 0.00175
\end{verbatim}

\begin{Shaded}
\begin{Highlighting}[]
\FunctionTok{paste}\NormalTok{(}\StringTok{"Partition Variance"}\NormalTok{)}
\end{Highlighting}
\end{Shaded}

\begin{verbatim}
[1] "Partition Variance"
\end{verbatim}

\begin{Shaded}
\begin{Highlighting}[]
\FunctionTok{anova}\NormalTok{(lab.q1t.lm1)}
\end{Highlighting}
\end{Shaded}

\begin{verbatim}
Analysis of Variance Table

Response: item_q1_rt
           Df Sum Sq Mean Sq F value Pr(>F)   
condition   1   6556    6556    10.2 0.0017 **
Residuals 124  79400     640                  
---
Signif. codes:  0 '***' 0.001 '**' 0.01 '*' 0.05 '.' 0.1 ' ' 1
\end{verbatim}

\begin{Shaded}
\begin{Highlighting}[]
\FunctionTok{paste}\NormalTok{(}\StringTok{"Confidence Interval on Parameter Estimates"}\NormalTok{)}
\end{Highlighting}
\end{Shaded}

\begin{verbatim}
[1] "Confidence Interval on Parameter Estimates"
\end{verbatim}

\begin{Shaded}
\begin{Highlighting}[]
\FunctionTok{confint}\NormalTok{(lab.q1t.lm1)}
\end{Highlighting}
\end{Shaded}

\begin{verbatim}
             2.5 % 97.5 %
(Intercept)   30.8   43.6
condition121   5.5   23.4
\end{verbatim}

\begin{Shaded}
\begin{Highlighting}[]
\CommentTok{\# report(lab.q1t.lm1) \#sanity check}
\CommentTok{\#print model equation}
\NormalTok{eq }\OtherTok{\textless{}{-}} \FunctionTok{extract\_eq}\NormalTok{(lab.q1t.lm1, }\AttributeTok{use\_coefs =} \ConstantTok{TRUE}\NormalTok{)}
\end{Highlighting}
\end{Shaded}

\textbf{Model equation} ,

\begin{equation}
, \operatorname{\widehat{item\_q1\_rt}} = 37.2 + 14.43(\operatorname{condition}_{\operatorname{121}}), 
\end{equation}

\textbf{For (In Person)} an OLS linear regression predicting response
time on the first question by experimental condition explains a
statistically significant though small 8\% variance in accuracy
(F(1,124) = 10.2, p \textless{} 0.01). The estimated beta coefficient
(\(\beta\) = 14.43, 95\% CI {[}5.5, 23.4{]}) predicts that participants
in the impasse condition will spend on average 14 seconds more on the
first question than those in the control condition.

\hypertarget{diagnostics-4}{%
\subparagraph{Diagnostics}\label{diagnostics-4}}

\begin{Shaded}
\begin{Highlighting}[]
\CommentTok{\#model diagnostics}
\FunctionTok{check\_model}\NormalTok{(lab.q1t.lm1, }\AttributeTok{panel =} \ConstantTok{TRUE}\NormalTok{)}
\end{Highlighting}
\end{Shaded}

\begin{figure}[H]

{\centering \includegraphics{analysis/SGC3A/4_sgc3A_hypotesting_files/figure-pdf/DIAG-Q1TIME-lab-1.pdf}

}

\end{figure}

\begin{enumerate}
\def\labelenumi{(\arabic{enumi})}
\tightlist
\item
  RESIDUAL DISTRIBUTION: \ensuremath{3.44\times 10^{-11}}
\item
  HOMOGENEITY: 0.387
\item
  HETERSCEDASTICITY: 0.382
\item
  AUTOCORRELATION: 0.468
\item
  OUTLIERS: FALSE, FALSE, FALSE, FALSE, FALSE, FALSE, FALSE, FALSE,
  FALSE, FALSE, FALSE, FALSE, FALSE, FALSE, FALSE, FALSE, FALSE, FALSE,
  FALSE, FALSE, FALSE, FALSE, FALSE, FALSE, FALSE, FALSE, FALSE, FALSE,
  FALSE, FALSE, FALSE, FALSE, FALSE, FALSE, FALSE, FALSE, FALSE, FALSE,
  FALSE, FALSE, FALSE, FALSE, FALSE, FALSE, FALSE, FALSE, FALSE, FALSE,
  FALSE, FALSE, FALSE, FALSE, FALSE, FALSE, FALSE, FALSE, FALSE, FALSE,
  FALSE, FALSE, FALSE, FALSE, FALSE, FALSE, FALSE, FALSE, FALSE, FALSE,
  FALSE, FALSE, FALSE, FALSE, FALSE, FALSE, FALSE, FALSE, FALSE, FALSE,
  FALSE, FALSE, FALSE, FALSE, FALSE, FALSE, FALSE, FALSE, FALSE, FALSE,
  FALSE, FALSE, FALSE, FALSE, FALSE, FALSE, FALSE, FALSE, FALSE, FALSE,
  FALSE, FALSE, FALSE, FALSE, FALSE, FALSE, FALSE, FALSE, FALSE, FALSE,
  FALSE, FALSE, FALSE, FALSE, FALSE, FALSE, FALSE, FALSE, FALSE, FALSE,
  FALSE, FALSE, FALSE, FALSE, FALSE, FALSE, FALSE, FALSE
\end{enumerate}

\hypertarget{inference-4}{%
\subparagraph{Inference}\label{inference-4}}

OLS Linear Regression on Q1 response time shows that condition explains
a small but statistically significant amount of variance (impasse
\textgreater{} control). However, the model violates the assumption of
normally distributed residuals.

\hypertarget{online-replication-2}{%
\paragraph{(Online Replication)}\label{online-replication-2}}

\hypertarget{visualization-5}{%
\subparagraph{Visualization}\label{visualization-5}}

\begin{Shaded}
\begin{Highlighting}[]
\CommentTok{\#HISTOGRAM}
\NormalTok{stats }\OtherTok{=}\NormalTok{ df\_online }\SpecialCharTok{\%\textgreater{}\%} \FunctionTok{group\_by}\NormalTok{(pretty\_condition) }\SpecialCharTok{\%\textgreater{}\%}\NormalTok{ dplyr}\SpecialCharTok{::}\FunctionTok{summarise}\NormalTok{(}\AttributeTok{mean =} \FunctionTok{mean}\NormalTok{(item\_q1\_rt))}
\FunctionTok{gf\_dhistogram}\NormalTok{(}\SpecialCharTok{\textasciitilde{}}\NormalTok{item\_q1\_rt, }\AttributeTok{fill =} \SpecialCharTok{\textasciitilde{}}\NormalTok{pretty\_condition, }\AttributeTok{data =}\NormalTok{ df\_lab) }\SpecialCharTok{\%\textgreater{}\%} \FunctionTok{gf\_facet\_grid}\NormalTok{(}\SpecialCharTok{\textasciitilde{}}\NormalTok{pretty\_condition) }\SpecialCharTok{\%\textgreater{}\%} 
  \FunctionTok{gf\_vline}\NormalTok{(}\AttributeTok{data =}\NormalTok{ stats, }\AttributeTok{xintercept =} \SpecialCharTok{\textasciitilde{}}\NormalTok{mean, }\AttributeTok{color =} \StringTok{"red"}\NormalTok{) }\SpecialCharTok{+}
  \FunctionTok{labs}\NormalTok{(}\AttributeTok{x =} \StringTok{"Response Time (seconds)"}\NormalTok{,}
       \CommentTok{\# y = "proportion of participants",}
       \AttributeTok{title =} \StringTok{"(ONLINE) First Question Response Time"}\NormalTok{,}
       \AttributeTok{subtitle =} \StringTok{""}\NormalTok{) }\SpecialCharTok{+} 
  \FunctionTok{theme\_minimal}\NormalTok{()}
\end{Highlighting}
\end{Shaded}

\begin{figure}[H]

{\centering \includegraphics{analysis/SGC3A/4_sgc3A_hypotesting_files/figure-pdf/VIS-TEST-Q1TIME-online-1.pdf}

}

\end{figure}

\hypertarget{model-5}{%
\subparagraph{Model}\label{model-5}}

\begin{Shaded}
\begin{Highlighting}[]
\CommentTok{\#SCORE predicted by CONDITION}
\NormalTok{rep.q1t.lm1 }\OtherTok{\textless{}{-}} \FunctionTok{lm}\NormalTok{(item\_q1\_rt }\SpecialCharTok{\textasciitilde{}}\NormalTok{ condition, }\AttributeTok{data =}\NormalTok{ df\_online)}
\FunctionTok{paste}\NormalTok{(}\StringTok{"Model"}\NormalTok{)}
\end{Highlighting}
\end{Shaded}

\begin{verbatim}
[1] "Model"
\end{verbatim}

\begin{Shaded}
\begin{Highlighting}[]
\FunctionTok{summary}\NormalTok{(rep.q1t.lm1)}
\end{Highlighting}
\end{Shaded}

\begin{verbatim}

Call:
lm(formula = item_q1_rt ~ condition, data = df_online)

Residuals:
   Min     1Q Median     3Q    Max 
-46.91 -20.33 -11.02   5.04 253.76 

Coefficients:
             Estimate Std. Error t value Pr(>|t|)    
(Intercept)     33.36       4.11    8.11  4.7e-14 ***
condition121    18.83       5.65    3.33    0.001 ** 
---
Signif. codes:  0 '***' 0.001 '**' 0.01 '*' 0.05 '.' 0.1 ' ' 1

Residual standard error: 40.3 on 202 degrees of freedom
Multiple R-squared:  0.0521,    Adjusted R-squared:  0.0474 
F-statistic: 11.1 on 1 and 202 DF,  p-value: 0.00103
\end{verbatim}

\begin{Shaded}
\begin{Highlighting}[]
\FunctionTok{paste}\NormalTok{(}\StringTok{"Partition Variance"}\NormalTok{)}
\end{Highlighting}
\end{Shaded}

\begin{verbatim}
[1] "Partition Variance"
\end{verbatim}

\begin{Shaded}
\begin{Highlighting}[]
\FunctionTok{anova}\NormalTok{(rep.q1t.lm1)}
\end{Highlighting}
\end{Shaded}

\begin{verbatim}
Analysis of Variance Table

Response: item_q1_rt
           Df Sum Sq Mean Sq F value Pr(>F)   
condition   1  18011   18011    11.1  0.001 **
Residuals 202 327799    1623                  
---
Signif. codes:  0 '***' 0.001 '**' 0.01 '*' 0.05 '.' 0.1 ' ' 1
\end{verbatim}

\begin{Shaded}
\begin{Highlighting}[]
\FunctionTok{paste}\NormalTok{(}\StringTok{"Confidence Interval on Parameter Estimates"}\NormalTok{)}
\end{Highlighting}
\end{Shaded}

\begin{verbatim}
[1] "Confidence Interval on Parameter Estimates"
\end{verbatim}

\begin{Shaded}
\begin{Highlighting}[]
\FunctionTok{confint}\NormalTok{(rep.q1t.lm1)}
\end{Highlighting}
\end{Shaded}

\begin{verbatim}
             2.5 % 97.5 %
(Intercept)  25.25   41.5
condition121  7.68   30.0
\end{verbatim}

\begin{Shaded}
\begin{Highlighting}[]
\CommentTok{\# report(rep.q1t.lm1) \#sanity check}
\CommentTok{\#print model equation}
\NormalTok{eq }\OtherTok{\textless{}{-}} \FunctionTok{extract\_eq}\NormalTok{(rep.q1t.lm1, }\AttributeTok{use\_coefs =} \ConstantTok{TRUE}\NormalTok{)}
\end{Highlighting}
\end{Shaded}

\textbf{Model equation} ,

\begin{equation}
, \operatorname{\widehat{item\_q1\_rt}} = 33.36 + 18.83(\operatorname{condition}_{\operatorname{121}}), 
\end{equation}

\textbf{For (In Person)} an OLS linear regression predicting response
time on the first question by experimental condition explains a
statistically significant though small 5\% variance in accuracy
(F(1,202) = 11.1, p \textless{} 0.001). The estimated beta coefficient
(\(\beta\) = 18.83, 95\% CI {[}7.68, 30{]}) predicts that participants
in the impasse condition will spend on average 19 seconds more on the
first question than those in the control condition.

\hypertarget{diagnostics-5}{%
\subparagraph{Diagnostics}\label{diagnostics-5}}

\begin{Shaded}
\begin{Highlighting}[]
\CommentTok{\#model diagnostics}
\FunctionTok{check\_model}\NormalTok{(rep.q1t.lm1, }\AttributeTok{panel =} \ConstantTok{TRUE}\NormalTok{)}
\end{Highlighting}
\end{Shaded}

\begin{figure}[H]

{\centering \includegraphics{analysis/SGC3A/4_sgc3A_hypotesting_files/figure-pdf/DIAG-Q1TIME-online-1.pdf}

}

\end{figure}

\begin{enumerate}
\def\labelenumi{(\arabic{enumi})}
\tightlist
\item
  RESIDUAL DISTRIBUTION: \ensuremath{7.231\times 10^{-19}}
\item
  HOMOGENEITY: \ensuremath{2.047\times 10^{-5}}
\item
  HETERSCEDASTICITY: \ensuremath{5.293\times 10^{-5}}
\item
  AUTOCORRELATION: 0.11
\item
  OUTLIERS: FALSE, FALSE, FALSE, FALSE, FALSE, FALSE, FALSE, FALSE,
  FALSE, FALSE, FALSE, FALSE, FALSE, FALSE, FALSE, FALSE, FALSE, FALSE,
  FALSE, FALSE, FALSE, FALSE, FALSE, FALSE, FALSE, FALSE, FALSE, FALSE,
  FALSE, FALSE, FALSE, FALSE, FALSE, FALSE, FALSE, FALSE, FALSE, FALSE,
  FALSE, FALSE, FALSE, FALSE, FALSE, FALSE, FALSE, FALSE, FALSE, FALSE,
  FALSE, FALSE, FALSE, FALSE, FALSE, FALSE, FALSE, FALSE, FALSE, FALSE,
  FALSE, FALSE, FALSE, FALSE, FALSE, FALSE, FALSE, FALSE, FALSE, FALSE,
  FALSE, FALSE, FALSE, FALSE, FALSE, FALSE, FALSE, FALSE, FALSE, FALSE,
  FALSE, FALSE, FALSE, FALSE, FALSE, FALSE, FALSE, FALSE, FALSE, FALSE,
  FALSE, FALSE, FALSE, FALSE, FALSE, FALSE, FALSE, FALSE, FALSE, FALSE,
  FALSE, FALSE, FALSE, FALSE, FALSE, FALSE, FALSE, FALSE, FALSE, FALSE,
  FALSE, FALSE, FALSE, FALSE, FALSE, FALSE, FALSE, FALSE, FALSE, FALSE,
  FALSE, FALSE, FALSE, FALSE, FALSE, FALSE, FALSE, FALSE, FALSE, FALSE,
  FALSE, FALSE, FALSE, FALSE, FALSE, FALSE, FALSE, FALSE, FALSE, FALSE,
  FALSE, FALSE, FALSE, FALSE, FALSE, FALSE, FALSE, FALSE, FALSE, FALSE,
  FALSE, FALSE, FALSE, FALSE, FALSE, FALSE, FALSE, FALSE, FALSE, FALSE,
  FALSE, FALSE, FALSE, FALSE, FALSE, FALSE, FALSE, FALSE, FALSE, FALSE,
  FALSE, FALSE, FALSE, FALSE, FALSE, FALSE, FALSE, FALSE, FALSE, FALSE,
  FALSE, FALSE, FALSE, FALSE, FALSE, FALSE, FALSE, FALSE, FALSE, FALSE,
  FALSE, FALSE, FALSE, FALSE, FALSE, FALSE, FALSE, FALSE, FALSE, FALSE,
  FALSE, FALSE, FALSE, FALSE, FALSE, FALSE
\end{enumerate}

\hypertarget{inference-5}{%
\subparagraph{Inference}\label{inference-5}}

OLS Linear Regression on Q1 response time shows that condition explains
a small but statistically significant amount of variance (impasse
\textgreater{} control). However, the model violates the assumption of
normally distributed residuals.

\hypertarget{resources-3}{%
\section{RESOURCES}\label{resources-3}}

\begin{itemize}
\tightlist
\item
  https://rpkgs.datanovia.com/ggpubr/reference/index.html
\end{itemize}

\hypertarget{archive-2}{%
\section{ARCHIVE}\label{archive-2}}

\hypertarget{test-phase-absolute-score-questions}{%
\subsection{Test Phase Absolute Score (\#
questions)}\label{test-phase-absolute-score-questions}}

\hypertarget{linear-regression-3}{%
\subsubsection{Linear Regression}\label{linear-regression-3}}

\emph{LM on Test Phase absolute score \textbf{as number of questions},
rather than \% correct.}

\begin{Shaded}
\begin{Highlighting}[]
\CommentTok{\#SCORE predicted by CONDITION}
\NormalTok{lm}\FloatTok{.1} \OtherTok{\textless{}{-}} \FunctionTok{lm}\NormalTok{(item\_test\_NABS }\SpecialCharTok{\textasciitilde{}}\NormalTok{ condition, }\AttributeTok{data =}\NormalTok{ df\_subjects)}
\FunctionTok{paste}\NormalTok{(}\StringTok{"Model"}\NormalTok{)}
\end{Highlighting}
\end{Shaded}

\begin{verbatim}
[1] "Model"
\end{verbatim}

\begin{Shaded}
\begin{Highlighting}[]
\FunctionTok{summary}\NormalTok{(lm}\FloatTok{.1}\NormalTok{)}
\end{Highlighting}
\end{Shaded}

\begin{verbatim}

Call:
lm(formula = item_test_NABS ~ condition, data = df_subjects)

Residuals:
   Min     1Q Median     3Q    Max 
 -3.02  -2.77  -1.52   2.98   6.48 

Coefficients:
             Estimate Std. Error t value Pr(>|t|)    
(Intercept)     1.519      0.251    6.04  4.1e-09 ***
condition121    1.498      0.348    4.30  2.2e-05 ***
---
Signif. codes:  0 '***' 0.001 '**' 0.01 '*' 0.05 '.' 0.1 ' ' 1

Residual standard error: 3.16 on 328 degrees of freedom
Multiple R-squared:  0.0535,    Adjusted R-squared:  0.0506 
F-statistic: 18.5 on 1 and 328 DF,  p-value: 0.0000222
\end{verbatim}

\begin{Shaded}
\begin{Highlighting}[]
\FunctionTok{paste}\NormalTok{(}\StringTok{"Partition Variance"}\NormalTok{)}
\end{Highlighting}
\end{Shaded}

\begin{verbatim}
[1] "Partition Variance"
\end{verbatim}

\begin{Shaded}
\begin{Highlighting}[]
\FunctionTok{anova}\NormalTok{(lm}\FloatTok{.1}\NormalTok{)}
\end{Highlighting}
\end{Shaded}

\begin{verbatim}
Analysis of Variance Table

Response: item_test_NABS
           Df Sum Sq Mean Sq F value   Pr(>F)    
condition   1    185     185    18.5 0.000022 ***
Residuals 328   3274      10                     
---
Signif. codes:  0 '***' 0.001 '**' 0.01 '*' 0.05 '.' 0.1 ' ' 1
\end{verbatim}

\begin{Shaded}
\begin{Highlighting}[]
\FunctionTok{paste}\NormalTok{(}\StringTok{"Confidence Interval on Parameter Estimates"}\NormalTok{)}
\end{Highlighting}
\end{Shaded}

\begin{verbatim}
[1] "Confidence Interval on Parameter Estimates"
\end{verbatim}

\begin{Shaded}
\begin{Highlighting}[]
\FunctionTok{confint}\NormalTok{(lm}\FloatTok{.1}\NormalTok{)}
\end{Highlighting}
\end{Shaded}

\begin{verbatim}
             2.5 % 97.5 %
(Intercept)  1.025   2.01
condition121 0.814   2.18
\end{verbatim}

\begin{Shaded}
\begin{Highlighting}[]
\FunctionTok{report}\NormalTok{(lm}\FloatTok{.1}\NormalTok{) }\CommentTok{\#sanity check}
\end{Highlighting}
\end{Shaded}

\begin{verbatim}
Warning: 'data_findcols()' is deprecated and will be removed in a future update.
  Its usage is discouraged. Please use 'data_find()' instead.

Warning: 'data_findcols()' is deprecated and will be removed in a future update.
  Its usage is discouraged. Please use 'data_find()' instead.

Warning: 'data_findcols()' is deprecated and will be removed in a future update.
  Its usage is discouraged. Please use 'data_find()' instead.
\end{verbatim}

\begin{verbatim}
We fitted a linear model (estimated using OLS) to predict item_test_NABS with condition (formula: item_test_NABS ~ condition). The model explains a statistically significant and weak proportion of variance (R2 = 0.05, F(1, 328) = 18.52, p < .001, adj. R2 = 0.05). The model's intercept, corresponding to condition = 111, is at 1.52 (95% CI [1.02, 2.01], t(328) = 6.04, p < .001). Within this model:

  - The effect of condition [121] is statistically significant and positive (beta = 1.50, 95% CI [0.81, 2.18], t(328) = 4.30, p < .001; Std. beta = 0.46, 95% CI [0.25, 0.67])

Standardized parameters were obtained by fitting the model on a standardized version of the dataset. 95% Confidence Intervals (CIs) and p-values were computed using the Wald approximation.
\end{verbatim}

\begin{Shaded}
\begin{Highlighting}[]
\FunctionTok{check\_model}\NormalTok{(lm}\FloatTok{.1}\NormalTok{)}
\end{Highlighting}
\end{Shaded}

\begin{figure}[H]

{\centering \includegraphics{analysis/SGC3A/4_sgc3A_hypotesting_files/figure-pdf/unnamed-chunk-31-1.pdf}

}

\end{figure}

\hypertarget{poisson-regression-todo}{%
\subsubsection{Poisson Regression TODO}\label{poisson-regression-todo}}

https://stats.oarc.ucla.edu/r/dae/poisson-regression/

The outcome variable absolute score is clearly not normal. As it
represents the cumulative number of items a participant has answered
correctly, we can consider it a type of \emph{count}, (ie. count of the
number of questions the participant got correct) and attempt to model it
using a General Linear Model with the Poisson distribution (and the
default log-link function).

\begin{Shaded}
\begin{Highlighting}[]
\CommentTok{\#POISSON}

\CommentTok{\#SCORE predicted by CONDITION {-}{-}\textgreater{} POISSON DISTRIBUTION}
\NormalTok{p}\FloatTok{.1} \OtherTok{\textless{}{-}} \FunctionTok{glm}\NormalTok{(item\_test\_NABS }\SpecialCharTok{\textasciitilde{}}\NormalTok{ condition, }\AttributeTok{data =}\NormalTok{ df\_subjects, }\AttributeTok{family =} \StringTok{"poisson"}\NormalTok{)}
\FunctionTok{paste}\NormalTok{(}\StringTok{"Model"}\NormalTok{)}
\end{Highlighting}
\end{Shaded}

\begin{verbatim}
[1] "Model"
\end{verbatim}

\begin{Shaded}
\begin{Highlighting}[]
\FunctionTok{summary}\NormalTok{(p}\FloatTok{.1}\NormalTok{)}
\end{Highlighting}
\end{Shaded}

\begin{verbatim}

Call:
glm(formula = item_test_NABS ~ condition, family = "poisson", 
    data = df_subjects)

Deviance Residuals: 
   Min      1Q  Median      3Q     Max  
 -2.46   -2.28   -1.74    1.51    3.69  

Coefficients:
             Estimate Std. Error z value Pr(>|z|)    
(Intercept)    0.4180     0.0645    6.48  9.4e-11 ***
condition121   0.6864     0.0781    8.79  < 2e-16 ***
---
Signif. codes:  0 '***' 0.001 '**' 0.01 '*' 0.05 '.' 0.1 ' ' 1

(Dispersion parameter for poisson family taken to be 1)

    Null deviance: 1579.3  on 329  degrees of freedom
Residual deviance: 1496.7  on 328  degrees of freedom
AIC: 1956

Number of Fisher Scoring iterations: 6
\end{verbatim}

\begin{Shaded}
\begin{Highlighting}[]
\FunctionTok{paste}\NormalTok{(}\StringTok{"Partition Variance"}\NormalTok{)}
\end{Highlighting}
\end{Shaded}

\begin{verbatim}
[1] "Partition Variance"
\end{verbatim}

\begin{Shaded}
\begin{Highlighting}[]
\FunctionTok{anova}\NormalTok{(p}\FloatTok{.1}\NormalTok{)}
\end{Highlighting}
\end{Shaded}

\begin{verbatim}
Analysis of Deviance Table

Model: poisson, link: log

Response: item_test_NABS

Terms added sequentially (first to last)

          Df Deviance Resid. Df Resid. Dev
NULL                        329       1579
condition  1     82.7       328       1497
\end{verbatim}

\begin{Shaded}
\begin{Highlighting}[]
\FunctionTok{paste}\NormalTok{(}\StringTok{"Confidence Interval on Parameter Estimates"}\NormalTok{)}
\end{Highlighting}
\end{Shaded}

\begin{verbatim}
[1] "Confidence Interval on Parameter Estimates"
\end{verbatim}

\begin{Shaded}
\begin{Highlighting}[]
\FunctionTok{confint}\NormalTok{(p}\FloatTok{.1}\NormalTok{)}
\end{Highlighting}
\end{Shaded}

\begin{verbatim}
Waiting for profiling to be done...
\end{verbatim}

\begin{verbatim}
             2.5 % 97.5 %
(Intercept)  0.289  0.542
condition121 0.535  0.841
\end{verbatim}

\begin{Shaded}
\begin{Highlighting}[]
\FunctionTok{report}\NormalTok{(p}\FloatTok{.1}\NormalTok{) }\CommentTok{\#sanity check}
\end{Highlighting}
\end{Shaded}

\begin{verbatim}
Warning: 'data_findcols()' is deprecated and will be removed in a future update.
  Its usage is discouraged. Please use 'data_find()' instead.

Warning: 'data_findcols()' is deprecated and will be removed in a future update.
  Its usage is discouraged. Please use 'data_find()' instead.

Warning: 'data_findcols()' is deprecated and will be removed in a future update.
  Its usage is discouraged. Please use 'data_find()' instead.

Warning: 'data_findcols()' is deprecated and will be removed in a future update.
  Its usage is discouraged. Please use 'data_find()' instead.

Warning: 'data_findcols()' is deprecated and will be removed in a future update.
  Its usage is discouraged. Please use 'data_find()' instead.
\end{verbatim}

\begin{verbatim}
We fitted a poisson model (estimated using ML) to predict item_test_NABS with condition (formula: item_test_NABS ~ condition). The model's explanatory power is moderate (Nagelkerke's R2 = 0.22). The model's intercept, corresponding to condition = 111, is at 0.42 (95% CI [0.29, 0.54], p < .001). Within this model:

  - The effect of condition [121] is statistically significant and positive (beta = 0.69, 95% CI [0.53, 0.84], p < .001; Std. beta = 0.69, 95% CI [0.53, 0.84])

Standardized parameters were obtained by fitting the model on a standardized version of the dataset. 95% Confidence Intervals (CIs) and p-values were computed using 
\end{verbatim}

\begin{Shaded}
\begin{Highlighting}[]
\FunctionTok{check\_model}\NormalTok{(p}\FloatTok{.1}\NormalTok{)}
\end{Highlighting}
\end{Shaded}

\begin{figure}[H]

{\centering \includegraphics{analysis/SGC3A/4_sgc3A_hypotesting_files/figure-pdf/unnamed-chunk-32-1.pdf}

}

\end{figure}

\hypertarget{zero-inflated-poisson}{%
\subsubsection{Zero Inflated Poisson}\label{zero-inflated-poisson}}

https://stats.oarc.ucla.edu/r/dae/zip/\\
Poisson count process with excess zeros

\begin{Shaded}
\begin{Highlighting}[]
\CommentTok{\#ZERO INFLATED POISSON}

\NormalTok{zinfp}\FloatTok{.1} \OtherTok{\textless{}{-}} \FunctionTok{zeroinfl}\NormalTok{(item\_test\_NABS }\SpecialCharTok{\textasciitilde{}}\NormalTok{  item\_q1\_rt}\SpecialCharTok{|}\NormalTok{ condition , }\AttributeTok{data =}\NormalTok{ df\_subjects)}
\FunctionTok{summary}\NormalTok{(zinfp}\FloatTok{.1}\NormalTok{)}
\end{Highlighting}
\end{Shaded}

\begin{verbatim}

Call:
zeroinfl(formula = item_test_NABS ~ item_q1_rt | condition, data = df_subjects)

Pearson residuals:
   Min     1Q Median     3Q    Max 
-0.934 -0.821 -0.548  0.965  2.421 

Count model coefficients (poisson with log link):
            Estimate Std. Error z value Pr(>|z|)    
(Intercept) 1.654243   0.059975   27.58   <2e-16 ***
item_q1_rt  0.001690   0.000849    1.99    0.047 *  

Zero-inflation model coefficients (binomial with logit link):
             Estimate Std. Error z value Pr(>|z|)    
(Intercept)     0.978      0.179    5.46  4.7e-08 ***
condition121   -1.055      0.236   -4.48  7.5e-06 ***
---
Signif. codes:  0 '***' 0.001 '**' 0.01 '*' 0.05 '.' 0.1 ' ' 1 

Number of iterations in BFGS optimization: 7 
Log-likelihood: -531 on 4 Df
\end{verbatim}

\begin{Shaded}
\begin{Highlighting}[]
\FunctionTok{report}\NormalTok{(zinfp}\FloatTok{.1}\NormalTok{)}
\end{Highlighting}
\end{Shaded}

\begin{verbatim}
Warning: 'data_findcols()' is deprecated and will be removed in a future update.
  Its usage is discouraged. Please use 'data_find()' instead.

Warning: 'data_findcols()' is deprecated and will be removed in a future update.
  Its usage is discouraged. Please use 'data_find()' instead.

Warning: 'data_findcols()' is deprecated and will be removed in a future update.
  Its usage is discouraged. Please use 'data_find()' instead.

Warning: 'data_findcols()' is deprecated and will be removed in a future update.
  Its usage is discouraged. Please use 'data_find()' instead.

Warning: 'data_findcols()' is deprecated and will be removed in a future update.
  Its usage is discouraged. Please use 'data_find()' instead.
\end{verbatim}

\begin{verbatim}
We fitted a zero-inflated poisson model to predict item_test_NABS with item_q1_rt and condition (formula: item_test_NABS ~ item_q1_rt). The model's explanatory power is substantial (R2 = 0.35, adj. R2 = 0.35). The model's intercept, corresponding to item_q1_rt = 0, is at 1.65 (95% CI [1.54, 1.77], p < .001). Within this model:

  - The effect of item q1 rt is statistically significant and positive (beta = 1.69e-03, 95% CI [2.52e-05, 3.35e-03], p = 0.047; Std. beta = 0.06, 95% CI [7.11e-04, 0.12])
  - The effect of condition [121] is statistically significant and negative (beta = -1.06, 95% CI [-1.52, -0.59], p < .001; Std. beta = -1.06, 95% CI [-1.52, -0.59])

Standardized parameters were obtained by fitting the model on a standardized version of the dataset.
\end{verbatim}

\begin{Shaded}
\begin{Highlighting}[]
\FunctionTok{performance}\NormalTok{(zinfp}\FloatTok{.1}\NormalTok{)}
\end{Highlighting}
\end{Shaded}

\begin{verbatim}
# Indices of model performance

AIC      |      BIC |    R2 | R2 (adj.) |  RMSE | Sigma | Score_log | Score_spherical
-------------------------------------------------------------------------------------
1070.173 | 1085.370 | 0.354 |     0.350 | 3.131 | 3.150 |    -1.609 |           0.044
\end{verbatim}

\begin{Shaded}
\begin{Highlighting}[]
\CommentTok{\# check\_model(zinfp.1)}
\end{Highlighting}
\end{Shaded}

\hypertarget{negative-binomial-regression}{%
\subsubsection{Negative Binomial
Regression}\label{negative-binomial-regression}}

https://stats.oarc.ucla.edu/r/dae/negative-binomial-regression/ -
overdispersed count data (variance much greater than mean)

\begin{Shaded}
\begin{Highlighting}[]
\CommentTok{\#NEGATIVE BIONOMIAL REGRESSION}
\CommentTok{\# {-} https://stats.oarc.ucla.edu/r/dae/negative{-}binomial{-}regression/}
\CommentTok{\# {-} Overdispersed Count variables}

\FunctionTok{library}\NormalTok{(MASS)}
\end{Highlighting}
\end{Shaded}

\begin{verbatim}

Attaching package: 'MASS'
\end{verbatim}

\begin{verbatim}
The following object is masked from 'package:dplyr':

    select
\end{verbatim}

\begin{Shaded}
\begin{Highlighting}[]
\NormalTok{nb}\FloatTok{.1} \OtherTok{\textless{}{-}} \FunctionTok{glm.nb}\NormalTok{(item\_test\_NABS }\SpecialCharTok{\textasciitilde{}}\NormalTok{ condition, }\AttributeTok{data =}\NormalTok{ df\_subjects)}
\FunctionTok{summary}\NormalTok{(nb}\FloatTok{.1}\NormalTok{)}
\end{Highlighting}
\end{Shaded}

\begin{verbatim}

Call:
glm.nb(formula = item_test_NABS ~ condition, data = df_subjects, 
    init.theta = 0.253501538, link = log)

Deviance Residuals: 
   Min      1Q  Median      3Q     Max  
-1.139  -1.102  -0.993   0.378   1.091  

Coefficients:
             Estimate Std. Error z value Pr(>|z|)   
(Intercept)     0.418      0.171    2.45   0.0143 * 
condition121    0.686      0.232    2.95   0.0031 **
---
Signif. codes:  0 '***' 0.001 '**' 0.01 '*' 0.05 '.' 0.1 ' ' 1

(Dispersion parameter for Negative Binomial(0.254) family taken to be 1)

    Null deviance: 279.52  on 329  degrees of freedom
Residual deviance: 270.97  on 328  degrees of freedom
AIC: 1194

Number of Fisher Scoring iterations: 1

              Theta:  0.2535 
          Std. Err.:  0.0315 

 2 x log-likelihood:  -1188.1290 
\end{verbatim}

\begin{Shaded}
\begin{Highlighting}[]
\FunctionTok{report}\NormalTok{(nb}\FloatTok{.1}\NormalTok{)}
\end{Highlighting}
\end{Shaded}

\begin{verbatim}
Warning: 'data_findcols()' is deprecated and will be removed in a future update.
  Its usage is discouraged. Please use 'data_find()' instead.

Warning: 'data_findcols()' is deprecated and will be removed in a future update.
  Its usage is discouraged. Please use 'data_find()' instead.

Warning: 'data_findcols()' is deprecated and will be removed in a future update.
  Its usage is discouraged. Please use 'data_find()' instead.

Warning: 'data_findcols()' is deprecated and will be removed in a future update.
  Its usage is discouraged. Please use 'data_find()' instead.

Warning: 'data_findcols()' is deprecated and will be removed in a future update.
  Its usage is discouraged. Please use 'data_find()' instead.
\end{verbatim}

\begin{verbatim}
We fitted a negative-binomial model (estimated using ML) to predict item_test_NABS with condition (formula: item_test_NABS ~ condition). The model's explanatory power is weak (Nagelkerke's R2 = 0.04). The model's intercept, corresponding to condition = 111, is at 0.42 (95% CI [0.10, 0.77], p = 0.014). Within this model:

  - The effect of condition [121] is statistically significant and positive (beta = 0.69, 95% CI [0.23, 1.14], p = 0.003; Std. beta = 0.69, 95% CI [0.23, 1.14])

Standardized parameters were obtained by fitting the model on a standardized version of the dataset. 95% Confidence Intervals (CIs) and p-values were computed using 
\end{verbatim}

\begin{Shaded}
\begin{Highlighting}[]
\FunctionTok{check\_model}\NormalTok{(nb}\FloatTok{.1}\NormalTok{)}
\end{Highlighting}
\end{Shaded}

\begin{figure}[H]

{\centering \includegraphics{analysis/SGC3A/4_sgc3A_hypotesting_files/figure-pdf/unnamed-chunk-34-1.pdf}

}

\end{figure}

\begin{Shaded}
\begin{Highlighting}[]
\CommentTok{\#check model assumption}
\CommentTok{\#assumes conditional means are not equal to conditional variances}
\CommentTok{\#conduct likelihood ration test to compare and test [need poisson]}
\NormalTok{m3 }\OtherTok{\textless{}{-}} \FunctionTok{glm}\NormalTok{(item\_test\_NABS }\SpecialCharTok{\textasciitilde{}}\NormalTok{ condition, }\AttributeTok{family =} \StringTok{"poisson"}\NormalTok{, }\AttributeTok{data =}\NormalTok{ df\_subjects)}
\FunctionTok{pchisq}\NormalTok{(}\DecValTok{2} \SpecialCharTok{*}\NormalTok{ (}\FunctionTok{logLik}\NormalTok{(nb}\FloatTok{.1}\NormalTok{) }\SpecialCharTok{{-}} \FunctionTok{logLik}\NormalTok{(m3)), }\AttributeTok{df =} \DecValTok{1}\NormalTok{, }\AttributeTok{lower.tail =} \ConstantTok{FALSE}\NormalTok{)}
\end{Highlighting}
\end{Shaded}

\begin{verbatim}
'log Lik.' 4.3e-168 (df=3)
\end{verbatim}

\begin{Shaded}
\begin{Highlighting}[]
\CommentTok{\#A large (+) log likelihood suggests that the negative binomial is more appropriate than the Poisson model}


\CommentTok{\#EXPONENTIATE PARAMETER ESTIMATES}
\NormalTok{est }\OtherTok{\textless{}{-}} \FunctionTok{cbind}\NormalTok{(}\AttributeTok{Estimate =} \FunctionTok{coef}\NormalTok{(nb}\FloatTok{.1}\NormalTok{), }\FunctionTok{confint}\NormalTok{(nb}\FloatTok{.1}\NormalTok{))}
\end{Highlighting}
\end{Shaded}

\begin{verbatim}
Waiting for profiling to be done...
\end{verbatim}

\begin{Shaded}
\begin{Highlighting}[]
\CommentTok{\#exponentiate parameter estimates}
\FunctionTok{print}\NormalTok{(}\StringTok{"Exponentiated Estimates"}\NormalTok{)}
\end{Highlighting}
\end{Shaded}

\begin{verbatim}
[1] "Exponentiated Estimates"
\end{verbatim}

\begin{Shaded}
\begin{Highlighting}[]
\FunctionTok{exp}\NormalTok{(est)}
\end{Highlighting}
\end{Shaded}

\begin{verbatim}
             Estimate 2.5 % 97.5 %
(Intercept)      1.52  1.10   2.15
condition121     1.99  1.26   3.13
\end{verbatim}

The variable condition has a coefficient of 0.67, (p \textless{} 0.005).
This means that for the impasse condition, the expected log count \# of
questions increases by 0.67. By exponentiating the estimate we see that
\# question correct rate for the impasse condition is nearly 2x that of
the control condition.

\textbf{Diagnostics} ??

\hypertarget{zero-inflated-negative-binomial-regression}{%
\subsubsection{Zero Inflated Negative Binomial
Regression}\label{zero-inflated-negative-binomial-regression}}

https://stats.oarc.ucla.edu/r/dae/zinb/ count data that are
overdispersed and have excess zeros

Zero-inflated negative binomial regression is for modelling count
variables with excessive zeros, and especially when the count data are
overdispersed (mean is much larger than variance). It can help account
for situations where theory suggests that excess zeros are generated by
2 separate processes, one that includes the other count values, and the
other that is just the zeros, and thus that the \emph{excess} zeros can
be modelled independently.

Total Absolute Score (\# items correct) may fit this situation, as the
data are overdispersed (variance much greater than the mean) and there
are are very large number of zeros. It is theoretically plausible that
these excess zeros (no answers correct) are the result of a different
`process' \ldots{} (i.e) little understanding and/or resistance to
restructuring understanding of the coordinate system. However, I am not
certain if it is plausible to suggest that the zeros themselves are the
result of two different processes: (ie. perhaps trying to understand,
and not trying to understand?) \textless- this could maybe be
disentangled by first question latency?

The model includes: - A logistic model to model which of the two
processes the zero outcome is associated with - A negative binomial
model to model the count process

\begin{Shaded}
\begin{Highlighting}[]
\FunctionTok{library}\NormalTok{(pscl) }\CommentTok{\#  for zeroinfl negbinomial}

\CommentTok{\#ZERO INFLATED NEGATIVE BINOMIAL}
\NormalTok{zinb}\FloatTok{.1} \OtherTok{\textless{}{-}} \FunctionTok{zeroinfl}\NormalTok{(item\_test\_NABS }\SpecialCharTok{\textasciitilde{}}\NormalTok{ condition }\SpecialCharTok{|}\NormalTok{ condition , }\AttributeTok{data =}\NormalTok{ df\_subjects, }\AttributeTok{dist =} \StringTok{"negbin"}\NormalTok{)}
\CommentTok{\#before the | is the count part, after the | is the logit model}
\FunctionTok{paste}\NormalTok{(}\StringTok{"Model"}\NormalTok{)}
\end{Highlighting}
\end{Shaded}

\begin{verbatim}
[1] "Model"
\end{verbatim}

\begin{Shaded}
\begin{Highlighting}[]
\FunctionTok{summary}\NormalTok{(zinb}\FloatTok{.1}\NormalTok{)}
\end{Highlighting}
\end{Shaded}

\begin{verbatim}

Call:
zeroinfl(formula = item_test_NABS ~ condition | condition, data = df_subjects, 
    dist = "negbin")

Pearson residuals:
   Min     1Q Median     3Q    Max 
-0.866 -0.794 -0.538  0.856  2.294 

Count model coefficients (negbin with log link):
             Estimate Std. Error z value Pr(>|z|)    
(Intercept)    1.7126     0.0728   23.54  < 2e-16 ***
condition121   0.0451     0.0880    0.51  0.60810    
Log(theta)     3.1851     0.8732    3.65  0.00026 ***

Zero-inflation model coefficients (binomial with logit link):
             Estimate Std. Error z value Pr(>|z|)    
(Intercept)     0.974      0.179    5.43  5.5e-08 ***
condition121   -1.056      0.236   -4.47  7.7e-06 ***
---
Signif. codes:  0 '***' 0.001 '**' 0.01 '*' 0.05 '.' 0.1 ' ' 1 

Theta = 24.169 
Number of iterations in BFGS optimization: 7 
Log-likelihood: -532 on 5 Df
\end{verbatim}

\begin{Shaded}
\begin{Highlighting}[]
\FunctionTok{report}\NormalTok{(zinb}\FloatTok{.1}\NormalTok{)}
\end{Highlighting}
\end{Shaded}

\begin{verbatim}
Warning: 'data_findcols()' is deprecated and will be removed in a future update.
  Its usage is discouraged. Please use 'data_find()' instead.

Warning: 'data_findcols()' is deprecated and will be removed in a future update.
  Its usage is discouraged. Please use 'data_find()' instead.

Warning: 'data_findcols()' is deprecated and will be removed in a future update.
  Its usage is discouraged. Please use 'data_find()' instead.

Warning: 'data_findcols()' is deprecated and will be removed in a future update.
  Its usage is discouraged. Please use 'data_find()' instead.

Warning: 'data_findcols()' is deprecated and will be removed in a future update.
  Its usage is discouraged. Please use 'data_find()' instead.
\end{verbatim}

\begin{verbatim}
We fitted a zero-inflated negative-binomial model to predict item_test_NABS with condition (formula: item_test_NABS ~ condition). The model's explanatory power is substantial (R2 = 0.36, adj. R2 = 0.36). The model's intercept, corresponding to condition = 111, is at 1.71 (95% CI [1.57, 1.86], p < .001). Within this model:

  - The effect of condition [121] is statistically non-significant and positive (beta = 0.05, 95% CI [-0.13, 0.22], p = 0.608; Std. beta = 0.05, 95% CI [-0.13, 0.22])
  - The effect of condition [121] is statistically significant and negative (beta = -1.06, 95% CI [-1.52, -0.59], p < .001; Std. beta = -1.06, 95% CI [-1.52, -0.59])

Standardized parameters were obtained by fitting the model on a standardized version of the dataset.
\end{verbatim}

\begin{Shaded}
\begin{Highlighting}[]
\FunctionTok{performance}\NormalTok{(zinb}\FloatTok{.1}\NormalTok{)}
\end{Highlighting}
\end{Shaded}

\begin{verbatim}
# Indices of model performance

AIC      |      BIC |    R2 | R2 (adj.) |  RMSE | Sigma | Score_log | Score_spherical
-------------------------------------------------------------------------------------
1073.880 | 1092.876 | 0.363 |     0.359 | 3.150 | 3.174 |    -1.649 |           0.043
\end{verbatim}

\begin{Shaded}
\begin{Highlighting}[]
\CommentTok{\# \#EXPONENTIATE PARAMETER ESTIMATES}
\CommentTok{\# est \textless{}{-} cbind(Estimate = coef(zinb.1), confint(zinb.1))}
\CommentTok{\# \#exponentiate parameter estimates}
\CommentTok{\# print("Exponentiated Estimates")}
\CommentTok{\# exp(est)}
\end{Highlighting}
\end{Shaded}

In the count model, the coefficient for the condition is very small, and
not significant (suggesting it does not contribute to the count yielding
process?).

In the zero-inflation model, the coefficient for the condition variable
is -1.056 and statistically significant. This suggests that the log odds
of being an excessive zero decrease by 1.06 if you are in the impasse
condition (exponentiate it?)

\textbf{TODO come back to this and discuss further}\\

\hypertarget{hurdle-model}{%
\subsubsection{HURDLE MODEL}\label{hurdle-model}}

https://data.library.virginia.edu/getting-started-with-hurdle-models/
https://en.wikipedia.org/wiki/Hurdle\_model\#:\textasciitilde:text=A\%20hurdle\%20model\%20is\%20a,of\%20the\%20non\%2Dzero\%20values.

class of models for count data with both overdispersion and excess
zeros;\\
different from zero-inflated models where the excess zeros are theorized
to arise from two different processes; in the hurdle model, there is a
model for P(x=0) and a separate model for P(x!=0)

The model includes: - A binary logit model to model whether the
observation takes a positive count or not. - a truncated Poisson or
Negative binomial model that only fits positive counts

This allows us to model: (1) Does the student get \emph{any} questions
right? (2) How many questions does the student get right?

\begin{Shaded}
\begin{Highlighting}[]
\FunctionTok{library}\NormalTok{(pscl) }\CommentTok{\#zero{-}inf and hurdle models }
\FunctionTok{library}\NormalTok{(countreg) }\CommentTok{\#rootogram}
\end{Highlighting}
\end{Shaded}

\begin{verbatim}
Registered S3 methods overwritten by 'countreg':
  method                 from
  print.zeroinfl         pscl
  print.summary.zeroinfl pscl
  summary.zeroinfl       pscl
  coef.zeroinfl          pscl
  vcov.zeroinfl          pscl
  logLik.zeroinfl        pscl
  predict.zeroinfl       pscl
  residuals.zeroinfl     pscl
  fitted.zeroinfl        pscl
  terms.zeroinfl         pscl
  model.matrix.zeroinfl  pscl
  extractAIC.zeroinfl    pscl
  print.hurdle           pscl
  print.summary.hurdle   pscl
  summary.hurdle         pscl
  coef.hurdle            pscl
  vcov.hurdle            pscl
  logLik.hurdle          pscl
  predict.hurdle         pscl
  residuals.hurdle       pscl
  fitted.hurdle          pscl
  terms.hurdle           pscl
  model.matrix.hurdle    pscl
  extractAIC.hurdle      pscl
\end{verbatim}

\begin{verbatim}

Attaching package: 'countreg'
\end{verbatim}

\begin{verbatim}
The following objects are masked from 'package:pscl':

    hurdle, hurdle.control, hurdletest, zeroinfl, zeroinfl.control
\end{verbatim}

\begin{verbatim}
The following object is masked from 'package:vcd':

    rootogram
\end{verbatim}

\begin{Shaded}
\begin{Highlighting}[]
\NormalTok{h}\FloatTok{.1} \OtherTok{\textless{}{-}} \FunctionTok{hurdle}\NormalTok{(s\_NABS }\SpecialCharTok{\textasciitilde{}}\NormalTok{ condition }\SpecialCharTok{|}\NormalTok{ condition , }\AttributeTok{data =}\NormalTok{ df\_subjects,}
              \AttributeTok{zero.dist =} \StringTok{"binomial"}\NormalTok{, }\AttributeTok{dist =} \StringTok{"poisson"}\NormalTok{)}

\FunctionTok{summary}\NormalTok{(h}\FloatTok{.1}\NormalTok{)}
\end{Highlighting}
\end{Shaded}

\begin{verbatim}

Call:
hurdle(formula = s_NABS ~ condition | condition, data = df_subjects, 
    dist = "poisson", zero.dist = "binomial")

Pearson residuals:
   Min     1Q Median     3Q    Max 
-1.255 -0.936 -0.693  1.038  2.959 

Count model coefficients (truncated poisson with log link):
             Estimate Std. Error z value Pr(>|z|)    
(Intercept)    1.8872     0.0509   37.10   <2e-16 ***
condition121   0.0661     0.0614    1.08     0.28    
Zero hurdle model coefficients (binomial with logit link):
             Estimate Std. Error z value Pr(>|z|)    
(Intercept)    -0.518      0.164   -3.15   0.0016 ** 
condition121    1.354      0.234    5.79  6.9e-09 ***
---
Signif. codes:  0 '***' 0.001 '**' 0.01 '*' 0.05 '.' 0.1 ' ' 1 

Number of iterations in BFGS optimization: 10 
Log-likelihood: -873 on 4 Df
\end{verbatim}

\begin{Shaded}
\begin{Highlighting}[]
\CommentTok{\# h.2 \textless{}{-} hurdle(item\_test\_NABS \textasciitilde{} condition, data = df\_subjects,}
\CommentTok{\#               zero.dist = "binomial", dist = "negbin")}
\CommentTok{\# }
\CommentTok{\# summary(h.2)}



\FunctionTok{rootogram}\NormalTok{(h}\FloatTok{.1}\NormalTok{)}
\end{Highlighting}
\end{Shaded}

\begin{figure}[H]

{\centering \includegraphics{analysis/SGC3A/4_sgc3A_hypotesting_files/figure-pdf/unnamed-chunk-36-1.pdf}

}

\end{figure}

\begin{Shaded}
\begin{Highlighting}[]
\FunctionTok{performance}\NormalTok{(h}\FloatTok{.1}\NormalTok{)}
\end{Highlighting}
\end{Shaded}

\begin{verbatim}
# Indices of model performance

AIC      |      BIC |    R2 | R2 (adj.) |  RMSE | Sigma | Score_log | Score_spherical
-------------------------------------------------------------------------------------
1753.435 | 1768.631 | 0.480 |     0.476 | 4.808 | 4.838 |    -3.434 |           0.038
\end{verbatim}

\hypertarget{model-comparison}{%
\subsubsection{Model Comparison}\label{model-comparison}}

\begin{Shaded}
\begin{Highlighting}[]
\FunctionTok{compare\_performance}\NormalTok{(lm}\FloatTok{.1}\NormalTok{, p}\FloatTok{.1}\NormalTok{, nb}\FloatTok{.1}\NormalTok{, zinb}\FloatTok{.1}\NormalTok{)}
\end{Highlighting}
\end{Shaded}

\begin{verbatim}
# Comparison of Model Performance Indices

Name   |    Model |      AIC | AIC weights |      BIC | BIC weights |  RMSE | Sigma | Score_log | Score_spherical |    R2 | R2 (adj.) | Nagelkerke's R2
-------------------------------------------------------------------------------------------------------------------------------------------------------
lm.1   |       lm | 1699.782 |     < 0.001 | 1711.179 |     < 0.001 | 3.150 | 3.160 |           |                 | 0.053 |     0.051 |                
p.1    |      glm | 1955.788 |     < 0.001 | 1963.386 |     < 0.001 | 3.150 | 2.136 |    -2.957 |           0.042 |       |           |           0.223
nb.1   |   negbin | 1194.129 |     < 0.001 | 1205.526 |     < 0.001 | 3.150 | 0.909 |    -2.137 |           0.046 |       |           |           0.045
zinb.1 | zeroinfl | 1073.880 |        1.00 | 1092.876 |        1.00 | 3.150 | 3.174 |    -1.649 |           0.043 | 0.363 |     0.359 |                
\end{verbatim}

For modelling test phase absolute score (\# items correct) it seems that
the zero inflated negative binomial model is the best fit according to
R2 and AIC, however, I am not clear on the implications of the
interpretation (non significant in count process, significant on logit
process), and also not clear if \# items correct is truly a count
process.

\newpage

\hypertarget{sec-SGC3A-exploration}{%
\chapter{Exploratory Analyses}\label{sec-SGC3A-exploration}}

\textbf{TODO}\\
- response consistency - clarify core questions being asked\\
- review models already created in ARCHIVE?\\
- explore response consistency - fix references

\emph{The purpose of this notebook is exploratory analyses of data
collected for study SGC3A.}

\begin{longtable}[]{@{}
  >{\raggedright\arraybackslash}p{(\columnwidth - 0\tabcolsep) * \real{0.3472}}@{}}
\toprule()
\begin{minipage}[b]{\linewidth}\raggedright
Pre-Requisite
\end{minipage} \\
\midrule()
\endhead
\begin{minipage}[t]{\linewidth}\raggedright
1\_sgc3A\_harmonize.qmd\\
2\_sgc3A\_scoring.qmd\strut
\end{minipage} \\
\bottomrule()
\end{longtable}

\begin{Shaded}
\begin{Highlighting}[]
\FunctionTok{library}\NormalTok{(multimode) }\CommentTok{\#mode mass tests}
\FunctionTok{library}\NormalTok{(Hmisc) }\CommentTok{\# \%nin\% operator}
\FunctionTok{library}\NormalTok{(ggpubr) }\CommentTok{\#arrange plots}
\FunctionTok{library}\NormalTok{(ggformula) }\CommentTok{\#easy graphs}
\FunctionTok{library}\NormalTok{(report) }\CommentTok{\#easystats reporting}
\FunctionTok{library}\NormalTok{(see) }\CommentTok{\#easystats visualization}
\FunctionTok{library}\NormalTok{(performance) }\CommentTok{\#easystats model diagnostics}
\FunctionTok{library}\NormalTok{(gmodels) }\CommentTok{\#contingency table and CHISQR}
\FunctionTok{library}\NormalTok{(vcd) }\CommentTok{\#mosaic plots}
\FunctionTok{library}\NormalTok{(vcdExtra) }\CommentTok{\#mosaic plots}
\FunctionTok{library}\NormalTok{(kableExtra) }\CommentTok{\#printing tables }
\FunctionTok{library}\NormalTok{(tidyverse) }\CommentTok{\#ALL THE THINGS}


\CommentTok{\#OUTPUT OPTIONS}
\FunctionTok{library}\NormalTok{(dplyr, }\AttributeTok{warn.conflicts =} \ConstantTok{FALSE}\NormalTok{)}
\FunctionTok{options}\NormalTok{(}\AttributeTok{dplyr.summarise.inform =} \ConstantTok{FALSE}\NormalTok{)}
\FunctionTok{options}\NormalTok{(}\AttributeTok{ggplot2.summarise.inform =} \ConstantTok{FALSE}\NormalTok{)}
\FunctionTok{options}\NormalTok{(}\AttributeTok{scipen=}\DecValTok{1}\NormalTok{, }\AttributeTok{digits=}\DecValTok{3}\NormalTok{)}
\end{Highlighting}
\end{Shaded}

\begin{Shaded}
\begin{Highlighting}[]
\CommentTok{\#IMPORT DATA }
\NormalTok{df\_items }\OtherTok{\textless{}{-}} \FunctionTok{read\_rds}\NormalTok{(}\StringTok{\textquotesingle{}analysis/SGC3A/data/2{-}scored{-}data/sgc3a\_scored\_items.rds\textquotesingle{}}\NormalTok{)}
\NormalTok{df\_subjects }\OtherTok{\textless{}{-}} \FunctionTok{read\_rds}\NormalTok{(}\StringTok{\textquotesingle{}analysis/SGC3A/data/2{-}scored{-}data/sgc3a\_scored\_participants.rds\textquotesingle{}}\NormalTok{)}


\CommentTok{\#SEPARATE ITEM DATA BY QUESTION TYPE}
\NormalTok{df\_scaffold }\OtherTok{\textless{}{-}}\NormalTok{ df\_items }\SpecialCharTok{\%\textgreater{}\%} \FunctionTok{filter}\NormalTok{(q }\SpecialCharTok{\textless{}} \DecValTok{6}\NormalTok{)}
\NormalTok{df\_test }\OtherTok{\textless{}{-}}\NormalTok{ df\_items }\SpecialCharTok{\%\textgreater{}\%} \FunctionTok{filter}\NormalTok{(q }\SpecialCharTok{\textgreater{}} \DecValTok{6}\NormalTok{) }\SpecialCharTok{\%\textgreater{}\%} \FunctionTok{filter}\NormalTok{ (q }\SpecialCharTok{\%nin\%} \FunctionTok{c}\NormalTok{(}\DecValTok{6}\NormalTok{,}\DecValTok{9}\NormalTok{))}
\NormalTok{df\_nondiscrim }\OtherTok{\textless{}{-}}\NormalTok{ df\_items }\SpecialCharTok{\%\textgreater{}\%} \FunctionTok{filter}\NormalTok{ (q }\SpecialCharTok{\%in\%} \FunctionTok{c}\NormalTok{(}\DecValTok{6}\NormalTok{,}\DecValTok{9}\NormalTok{))}
\end{Highlighting}
\end{Shaded}

Exploratory Questions

Consistency \textbar{} How consistent are learners in their
interpretation of the graph? Do they adopt an interpretation on the
first question and hold constant? Or do they change interpretations from
question to question? Are there any interpretations that serve as
`absorbing states' (i.e.~once encountered, the learner does not exist
this state).

Time Course of Exploration \textbar{} What is the relationship between
response accuracy (and interpretation) and time spent on each item?

Can exploration strategies be derived from mouse cursor activity?

\begin{itemize}
\tightlist
\item
  does response time predict interpretation vs.~non interpretation?
\end{itemize}

\hypertarget{mass-movement}{%
\subsubsection{Mass Movement}\label{mass-movement}}

``movement of mass'' from one mode to another

Considering only families of unimodal distributions, the most probably
distribution (as predicted by package \texttt{performance}) is
negative-binomial.

\begin{Shaded}
\begin{Highlighting}[]
\NormalTok{df }\OtherTok{\textless{}{-}}\NormalTok{ df\_subjects }\SpecialCharTok{\%\textgreater{}\%} \FunctionTok{filter}\NormalTok{(condition}\SpecialCharTok{==}\DecValTok{111}\NormalTok{)}
\NormalTok{multimode}\SpecialCharTok{::}\FunctionTok{modetest}\NormalTok{(df}\SpecialCharTok{$}\NormalTok{s\_NABS)}
\end{Highlighting}
\end{Shaded}

\begin{verbatim}
Warning in multimode::modetest(df$s_NABS): A modification of the data was made
in order to compute the excess mass or the dip statistic
\end{verbatim}

\begin{verbatim}

    Ameijeiras-Alonso et al. (2019) excess mass test

data:  df$s_NABS
Excess mass = 0.09, p-value <2e-16
alternative hypothesis: true number of modes is greater than 1
\end{verbatim}

\begin{Shaded}
\begin{Highlighting}[]
\NormalTok{n\_modes }\OtherTok{=}\NormalTok{ multimode}\SpecialCharTok{::}\FunctionTok{nmodes}\NormalTok{(}\AttributeTok{data =}\NormalTok{ df}\SpecialCharTok{$}\NormalTok{s\_NABS, }\AttributeTok{bw=}\DecValTok{2}\NormalTok{)}
\NormalTok{multimode}\SpecialCharTok{::}\FunctionTok{locmodes}\NormalTok{(df}\SpecialCharTok{$}\NormalTok{s\_NABS,}\AttributeTok{mod0 =}\NormalTok{  n\_modes, }\AttributeTok{display =} \ConstantTok{TRUE}\NormalTok{)}
\end{Highlighting}
\end{Shaded}

\begin{verbatim}
Warning in multimode::locmodes(df$s_NABS, mod0 = n_modes, display = TRUE): If
the density function has an unbounded support, artificial modes may have been
created in the tails
\end{verbatim}

\begin{figure}[H]

{\centering \includegraphics{analysis/SGC3A/5_sgc3A_exploration_files/figure-pdf/MASS-111-1.pdf}

}

\end{figure}

\begin{verbatim}

Estimated location
Modes: 0.136  12.6 
Antimode: 7.63 

Estimated value of the density
Modes: 0.268  0.0432 
Antimode: 0.00648 

Critical bandwidth: 1.05
\end{verbatim}

\begin{Shaded}
\begin{Highlighting}[]
\NormalTok{df }\OtherTok{\textless{}{-}}\NormalTok{ df\_subjects }\SpecialCharTok{\%\textgreater{}\%} \FunctionTok{filter}\NormalTok{(condition}\SpecialCharTok{==}\DecValTok{121}\NormalTok{)}
\NormalTok{multimode}\SpecialCharTok{::}\FunctionTok{modetest}\NormalTok{(df}\SpecialCharTok{$}\NormalTok{s\_NABS)}
\end{Highlighting}
\end{Shaded}

\begin{verbatim}
Warning in multimode::modetest(df$s_NABS): A modification of the data was made
in order to compute the excess mass or the dip statistic
\end{verbatim}

\begin{verbatim}

    Ameijeiras-Alonso et al. (2019) excess mass test

data:  df$s_NABS
Excess mass = 0.1, p-value <2e-16
alternative hypothesis: true number of modes is greater than 1
\end{verbatim}

\begin{Shaded}
\begin{Highlighting}[]
\NormalTok{n\_modes }\OtherTok{=}\NormalTok{ multimode}\SpecialCharTok{::}\FunctionTok{nmodes}\NormalTok{(}\AttributeTok{data =}\NormalTok{ df}\SpecialCharTok{$}\NormalTok{s\_NABS, }\AttributeTok{bw=}\DecValTok{2}\NormalTok{)}
\NormalTok{multimode}\SpecialCharTok{::}\FunctionTok{locmodes}\NormalTok{(df}\SpecialCharTok{$}\NormalTok{s\_NABS,}\AttributeTok{mod0 =}\NormalTok{  n\_modes, }\AttributeTok{display =} \ConstantTok{TRUE}\NormalTok{)}
\end{Highlighting}
\end{Shaded}

\begin{verbatim}
Warning in multimode::locmodes(df$s_NABS, mod0 = n_modes, display = TRUE): If
the density function has an unbounded support, artificial modes may have been
created in the tails
\end{verbatim}

\begin{figure}[H]

{\centering \includegraphics{analysis/SGC3A/5_sgc3A_exploration_files/figure-pdf/MASS-121-1.pdf}

}

\end{figure}

\begin{verbatim}

Estimated location
Modes: 0.456    12 
Antimode: 4.89 

Estimated value of the density
Modes: 0.148  0.0703 
Antimode: 0.025 

Critical bandwidth: 1.27
\end{verbatim}

\hypertarget{response-consistency}{%
\section{RESPONSE CONSISTENCY}\label{response-consistency}}

\textbf{TODO}

\hypertarget{xyz}{%
\section{XYZ}\label{xyz}}

\hypertarget{xyz-1}{%
\subsection{XYZ}\label{xyz-1}}

\hypertarget{xyz-2}{%
\subsubsection{XYZ}\label{xyz-2}}

\begin{longtable}[]{@{}
  >{\raggedright\arraybackslash}p{(\columnwidth - 2\tabcolsep) * \real{0.0375}}
  >{\raggedright\arraybackslash}p{(\columnwidth - 2\tabcolsep) * \real{0.9625}}@{}}
\toprule()
\begin{minipage}[b]{\linewidth}\raggedright
Research Question
\end{minipage} & \begin{minipage}[b]{\linewidth}\raggedright
Does the frequency of correct (vs) incorrect responses on the first
question differ by condition? {[}Is response accuracy independent of
condition?{]}
\end{minipage} \\
\midrule()
\endhead
\textbf{Analysis Strategy} & Chi-Square test of independence on outcome
\texttt{score\_niceABS} by \texttt{condition} for \texttt{df\_items}
where \texttt{q\ ==\ 1} \\
\textbf{Justification} & (0) simplest method to examine independence of
two categorical factors; logistic regression is recommended for binomial
\textasciitilde{} continuous

(1) independence assumption : as we only consider responses on the first
question, each observation corresponds to an individual subject, and are
thus independent

(2) frequency size assumption : expected frequency in each cell of the
contingency table is greater than 5 (more than 5 correct , more than 5
incorrect responses) \\
\textbf{Steps} & (1) Express raw data as contingency table \& visualize

(2) Calculate Chi-Squared Statistic and p-value

(3) Interpret Odds-Ratio as effect size \\
\textbf{Inference} & \textbf{Lab} A Pearson's Chi-squared test (of
independence) indicates a relationship between response accuracy on the
first question and experimental condition approaching statistical
significance, \(\chi^2\) (1) = 10.3, p = 0.07. Thus we have insufficient
evidence to reject the null hypothesis that the odds ratio is equal to
1. In this particular data sample, the sample odds ratio 2.18 indicates
that the odds of producing a correct response on the first question were
2.18 times greater if a subject was in the impasse condition, than if
the control condition (Odds Ratio = 2.18, p = 0.055, 95\% CI {[}0.982,
+Inf{]}).

\textbf{Online} A Pearson's Chi-squared test (of independence) indicates
a statistically significant relationship between response accuracy on
the first question and experimental condition, \(\chi^2\) (1) = 7.26, p
= 0.009. Thus we have sufficient evidence to reject the null hypothesis
that the odds ratio is equal to 1. The sample odds ratio = 2.68
indicates that the odds of producing a correct response on the first
question were 2.68 times greater if a subject was in the impasse
condition, than in the control condition (Odds Ratio = 2.68, p = 0.005,
95\% CI {[}1.37, +Inf{]}). \\
\bottomrule()
\end{longtable}

\begin{Shaded}
\begin{Highlighting}[]
\CommentTok{\#FITER THE DATASET}
\NormalTok{df }\OtherTok{=}\NormalTok{ df\_items }\SpecialCharTok{\%\textgreater{}\%} \FunctionTok{filter}\NormalTok{(q}\SpecialCharTok{==}\DecValTok{1}\NormalTok{) }

\CommentTok{\#PROPORTIONAL BAR CHART}
\FunctionTok{gf\_props}\NormalTok{(}\SpecialCharTok{\textasciitilde{}}\NormalTok{score\_niceABS, }\AttributeTok{data =}\NormalTok{ df, }\AttributeTok{fill =} \SpecialCharTok{\textasciitilde{}}\NormalTok{mode) }\SpecialCharTok{\%\textgreater{}\%} 
  \FunctionTok{gf\_facet\_grid}\NormalTok{(mode}\SpecialCharTok{\textasciitilde{}}\NormalTok{condition, }\AttributeTok{labeller =}\NormalTok{ label\_both) }\SpecialCharTok{+}
  \FunctionTok{labs}\NormalTok{(}\AttributeTok{x =} \StringTok{"Correct Response on Q 1"}\NormalTok{,}
       \AttributeTok{title =} \StringTok{"Accuracy on First Question by Condition (Both Modalities)"}\NormalTok{,}
       \AttributeTok{subtitle=}\StringTok{"Impasse Condition yields a greater proportion of correct responses than control "}\NormalTok{)}\SpecialCharTok{+}
  \FunctionTok{theme\_minimal}\NormalTok{()}\SpecialCharTok{+} \FunctionTok{theme}\NormalTok{(}\AttributeTok{legend.position =} \StringTok{"none"}\NormalTok{)}
\end{Highlighting}
\end{Shaded}

\begin{figure}[H]

{\centering \includegraphics{analysis/SGC3A/5_sgc3A_exploration_files/figure-pdf/VIS-LR-Q1.tri.by.Cond-1.pdf}

}

\end{figure}

\begin{Shaded}
\begin{Highlighting}[]
\CommentTok{\#MOSAIC PLOT}
\NormalTok{vcd}\SpecialCharTok{::}\FunctionTok{mosaic}\NormalTok{(}\AttributeTok{main=}\StringTok{"Accuracy on First Question by Condition (Both Modalities)"}\NormalTok{,}
            \AttributeTok{data =}\NormalTok{ df, score\_niceABS }\SpecialCharTok{\textasciitilde{}}\NormalTok{ condition, }\AttributeTok{rot\_labels=}\FunctionTok{c}\NormalTok{(}\DecValTok{0}\NormalTok{,}\DecValTok{90}\NormalTok{,}\DecValTok{0}\NormalTok{,}\DecValTok{0}\NormalTok{),}
            \AttributeTok{offset\_varnames =} \FunctionTok{c}\NormalTok{(}\AttributeTok{left =} \FloatTok{4.5}\NormalTok{), }\AttributeTok{offset\_labels =} \FunctionTok{c}\NormalTok{(}\AttributeTok{left =} \SpecialCharTok{{-}}\FloatTok{0.5}\NormalTok{),}\AttributeTok{just\_labels =} \StringTok{"right"}\NormalTok{,}
            \AttributeTok{spacing =} \FunctionTok{spacing\_dimequal}\NormalTok{(}\FunctionTok{unit}\NormalTok{(}\DecValTok{1}\SpecialCharTok{:}\DecValTok{2}\NormalTok{, }\StringTok{"lines"}\NormalTok{)))}
\end{Highlighting}
\end{Shaded}

\begin{figure}[H]

{\centering \includegraphics{analysis/SGC3A/5_sgc3A_exploration_files/figure-pdf/VIS-LR-Q1.tri.by.Cond-2.pdf}

}

\end{figure}

\begin{Shaded}
\begin{Highlighting}[]
\CommentTok{\#PRINT CONTINGENCY TABLE}
\NormalTok{title }\OtherTok{=} \StringTok{"Proportion of Correct Responses On First Item (Both Modalities)"}
\NormalTok{item.contingency }\OtherTok{\textless{}{-}}\NormalTok{  df }\SpecialCharTok{\%\textgreater{}\%}\NormalTok{ dplyr}\SpecialCharTok{::}\FunctionTok{select}\NormalTok{(condition, score\_niceABS) }\SpecialCharTok{\%\textgreater{}\%} \FunctionTok{table}\NormalTok{() }\SpecialCharTok{\%\textgreater{}\%} \FunctionTok{prop.table}\NormalTok{() }\SpecialCharTok{\%\textgreater{}\%} \FunctionTok{addmargins}\NormalTok{()}
\NormalTok{item.contingency }\SpecialCharTok{\%\textgreater{}\%} \FunctionTok{kbl}\NormalTok{ (}\AttributeTok{caption =}\NormalTok{ title) }\SpecialCharTok{\%\textgreater{}\%} \FunctionTok{kable\_classic}\NormalTok{()}
\end{Highlighting}
\end{Shaded}

\begin{table}

\caption{Proportion of Correct Responses On First Item (Both Modalities)}
\centering
\begin{tabular}[t]{l|r|r|r}
\hline
  & 0 & 1 & Sum\\
\hline
111 & 0.412 & 0.067 & 0.479\\
\hline
121 & 0.373 & 0.148 & 0.521\\
\hline
Sum & 0.785 & 0.215 & 1.000\\
\hline
\end{tabular}
\end{table}

\begin{Shaded}
\begin{Highlighting}[]
\NormalTok{df }\OtherTok{=}\NormalTok{ df\_items }\SpecialCharTok{\%\textgreater{}\%} \FunctionTok{filter}\NormalTok{(q}\SpecialCharTok{==}\DecValTok{1}\NormalTok{) }\SpecialCharTok{\%\textgreater{}\%} \FunctionTok{filter}\NormalTok{(mode }\SpecialCharTok{==} \StringTok{"lab{-}synch"}\NormalTok{)}
\FunctionTok{CrossTable}\NormalTok{( }\AttributeTok{x =}\NormalTok{ df}\SpecialCharTok{$}\NormalTok{condition, }\AttributeTok{y =}\NormalTok{ df}\SpecialCharTok{$}\NormalTok{score\_niceABS, }\AttributeTok{fisher =} \ConstantTok{TRUE}\NormalTok{, }\AttributeTok{chisq=}\ConstantTok{TRUE}\NormalTok{, }\AttributeTok{expected =} \ConstantTok{TRUE}\NormalTok{, }\AttributeTok{sresid =} \ConstantTok{TRUE}\NormalTok{)}
\end{Highlighting}
\end{Shaded}

\begin{verbatim}

 
   Cell Contents
|-------------------------|
|                       N |
|              Expected N |
| Chi-square contribution |
|           N / Row Total |
|           N / Col Total |
|         N / Table Total |
|-------------------------|

 
Total Observations in Table:  126 

 
             | df$score_niceABS 
df$condition |         0 |         1 | Row Total | 
-------------|-----------|-----------|-----------|
         111 |        52 |        10 |        62 | 
             |    47.730 |    14.270 |           | 
             |     0.382 |     1.278 |           | 
             |     0.839 |     0.161 |     0.492 | 
             |     0.536 |     0.345 |           | 
             |     0.413 |     0.079 |           | 
-------------|-----------|-----------|-----------|
         121 |        45 |        19 |        64 | 
             |    49.270 |    14.730 |           | 
             |     0.370 |     1.238 |           | 
             |     0.703 |     0.297 |     0.508 | 
             |     0.464 |     0.655 |           | 
             |     0.357 |     0.151 |           | 
-------------|-----------|-----------|-----------|
Column Total |        97 |        29 |       126 | 
             |     0.770 |     0.230 |           | 
-------------|-----------|-----------|-----------|

 
Statistics for All Table Factors


Pearson's Chi-squared test 
------------------------------------------------------------
Chi^2 =  3.27     d.f. =  1     p =  0.0707 

Pearson's Chi-squared test with Yates' continuity correction 
------------------------------------------------------------
Chi^2 =  2.55     d.f. =  1     p =  0.111 

 
Fisher's Exact Test for Count Data
------------------------------------------------------------
Sample estimate odds ratio:  2.18 

Alternative hypothesis: true odds ratio is not equal to 1
p =  0.0909 
95% confidence interval:  0.86 5.84 

Alternative hypothesis: true odds ratio is less than 1
p =  0.979 
95% confidence interval:  0 5.03 

Alternative hypothesis: true odds ratio is greater than 1
p =  0.0547 
95% confidence interval:  0.982 Inf 


 
\end{verbatim}

Inspecting the output of the Chi-Squared test, we first see that we meet
the assumption of expected frequency in each cell (indicated by the
second row in each box, `Expected N'. The model predicts more than 5
observations in each cell.) The Pearson's Chi-squared test (of
independence) indicates a relationship between response accuracy on the
first question and experimental condition approaching statistical
significance, \(\chi^2\) (1) = 10.3, p = 0.07. Thus we have insufficient
evidence to reject the null hypothesis that the odds ratio is not equal
to 1. In this particular data sample, the odds ratio 2.18 indicates that
the odds of producing a correct response on the first question were 2.18
times greater if a subject was in the impasse condition, than in the
control condition (Odds Ratio = 2.18, p = 0.055, 95\% CI {[}0.982,
+Inf{]}).

\begin{Shaded}
\begin{Highlighting}[]
\NormalTok{df }\OtherTok{=}\NormalTok{ df\_items }\SpecialCharTok{\%\textgreater{}\%} \FunctionTok{filter}\NormalTok{(q}\SpecialCharTok{==}\DecValTok{1}\NormalTok{) }\SpecialCharTok{\%\textgreater{}\%} \FunctionTok{filter}\NormalTok{(mode }\SpecialCharTok{==} \StringTok{"asynch"}\NormalTok{)}
\FunctionTok{CrossTable}\NormalTok{( }\AttributeTok{x =}\NormalTok{ df}\SpecialCharTok{$}\NormalTok{condition, }\AttributeTok{y =}\NormalTok{ df}\SpecialCharTok{$}\NormalTok{score\_niceABS, }\AttributeTok{fisher =} \ConstantTok{TRUE}\NormalTok{, }\AttributeTok{chisq=}\ConstantTok{TRUE}\NormalTok{, }\AttributeTok{expected =} \ConstantTok{TRUE}\NormalTok{, }\AttributeTok{sresid =} \ConstantTok{TRUE}\NormalTok{)}
\end{Highlighting}
\end{Shaded}

\begin{verbatim}

 
   Cell Contents
|-------------------------|
|                       N |
|              Expected N |
| Chi-square contribution |
|           N / Row Total |
|           N / Col Total |
|         N / Table Total |
|-------------------------|

 
Total Observations in Table:  204 

 
             | df$score_niceABS 
df$condition |         0 |         1 | Row Total | 
-------------|-----------|-----------|-----------|
         111 |        84 |        12 |        96 | 
             |    76.235 |    19.765 |           | 
             |     0.791 |     3.050 |           | 
             |     0.875 |     0.125 |     0.471 | 
             |     0.519 |     0.286 |           | 
             |     0.412 |     0.059 |           | 
-------------|-----------|-----------|-----------|
         121 |        78 |        30 |       108 | 
             |    85.765 |    22.235 |           | 
             |     0.703 |     2.711 |           | 
             |     0.722 |     0.278 |     0.529 | 
             |     0.481 |     0.714 |           | 
             |     0.382 |     0.147 |           | 
-------------|-----------|-----------|-----------|
Column Total |       162 |        42 |       204 | 
             |     0.794 |     0.206 |           | 
-------------|-----------|-----------|-----------|

 
Statistics for All Table Factors


Pearson's Chi-squared test 
------------------------------------------------------------
Chi^2 =  7.26     d.f. =  1     p =  0.00707 

Pearson's Chi-squared test with Yates' continuity correction 
------------------------------------------------------------
Chi^2 =  6.35     d.f. =  1     p =  0.0117 

 
Fisher's Exact Test for Count Data
------------------------------------------------------------
Sample estimate odds ratio:  2.68 

Alternative hypothesis: true odds ratio is not equal to 1
p =  0.00894 
95% confidence interval:  1.23 6.17 

Alternative hypothesis: true odds ratio is less than 1
p =  0.998 
95% confidence interval:  0 5.42 

Alternative hypothesis: true odds ratio is greater than 1
p =  0.00539 
95% confidence interval:  1.37 Inf 


 
\end{verbatim}

Inspecting the output of the Chi-Squared test, we first see that we meet
the assumption of expected frequency in each cell (indicated by the
second row in each box, `Expected N'. The model predicts more than 5
observations in each cell.) The Pearson's Chi-squared test (of
independence) indicates a statistically significant relationship between
response accuracy on the first question and experimental condition,
\(\chi^2\) (1) = 7.26, p = 0.009. Thus we have sufficient evidence to
reject the null hypothesis that the odds ratio is not equal to 1. The
odds ratio = 2.68 indicates that the odds of producing a correct
response on the first question were 2.68 times greater if a subject was
in the impasse condition, than in the control condition (Odds Ratio =
2.68, p = 0.005, 95\% CI {[}1.37, +Inf{]}).

\hypertarget{copied-from-3}{%
\section{COPIED FROM 3}\label{copied-from-3}}

\textbf{Does the IMPASSE condition more accurate interpretation?}

To address this question we assess how much variance in cumulative
(absolute) score is explained by experimental condition.

\begin{Shaded}
\begin{Highlighting}[]
\CommentTok{\#SCORE predicted by CONDITION}
\NormalTok{m1 }\OtherTok{\textless{}{-}} \FunctionTok{lm}\NormalTok{(s\_SCALED }\SpecialCharTok{\textasciitilde{}}\NormalTok{ condition, }\AttributeTok{data =}\NormalTok{ df\_subjects }\SpecialCharTok{\%\textgreater{}\%} \FunctionTok{filter}\NormalTok{(mode}\SpecialCharTok{==}\StringTok{"lab{-}synch"}\NormalTok{))}
\FunctionTok{paste}\NormalTok{(}\StringTok{"Model"}\NormalTok{)}
\end{Highlighting}
\end{Shaded}

\begin{verbatim}
[1] "Model"
\end{verbatim}

\begin{Shaded}
\begin{Highlighting}[]
\FunctionTok{summary}\NormalTok{(m1)}
\end{Highlighting}
\end{Shaded}

\begin{verbatim}

Call:
lm(formula = s_SCALED ~ condition, data = df_subjects %>% filter(mode == 
    "lab-synch"))

Residuals:
   Min     1Q Median     3Q    Max 
-12.51  -6.48  -4.48   9.38  19.52 

Coefficients:
             Estimate Std. Error t value Pr(>|t|)    
(Intercept)     -6.52       1.19   -5.48  2.3e-07 ***
condition121     7.53       1.67    4.51  1.5e-05 ***
---
Signif. codes:  0 '***' 0.001 '**' 0.01 '*' 0.05 '.' 0.1 ' ' 1

Residual standard error: 9.38 on 124 degrees of freedom
Multiple R-squared:  0.141, Adjusted R-squared:  0.134 
F-statistic: 20.3 on 1 and 124 DF,  p-value: 0.0000151
\end{verbatim}

\begin{Shaded}
\begin{Highlighting}[]
\FunctionTok{paste}\NormalTok{(}\StringTok{"Partition Variance"}\NormalTok{)}
\end{Highlighting}
\end{Shaded}

\begin{verbatim}
[1] "Partition Variance"
\end{verbatim}

\begin{Shaded}
\begin{Highlighting}[]
\FunctionTok{anova}\NormalTok{(m1)}
\end{Highlighting}
\end{Shaded}

\begin{verbatim}
Analysis of Variance Table

Response: s_SCALED
           Df Sum Sq Mean Sq F value   Pr(>F)    
condition   1   1787    1787    20.3 0.000015 ***
Residuals 124  10910      88                     
---
Signif. codes:  0 '***' 0.001 '**' 0.01 '*' 0.05 '.' 0.1 ' ' 1
\end{verbatim}

\begin{Shaded}
\begin{Highlighting}[]
\FunctionTok{paste}\NormalTok{(}\StringTok{"Confidence Interval on Parameter Estimates"}\NormalTok{)}
\end{Highlighting}
\end{Shaded}

\begin{verbatim}
[1] "Confidence Interval on Parameter Estimates"
\end{verbatim}

\begin{Shaded}
\begin{Highlighting}[]
\FunctionTok{confint}\NormalTok{(m1)}
\end{Highlighting}
\end{Shaded}

\begin{verbatim}
             2.5 % 97.5 %
(Intercept)  -8.88  -4.17
condition121  4.22  10.84
\end{verbatim}

\begin{Shaded}
\begin{Highlighting}[]
\CommentTok{\# report(m1) \#sanity check}
\end{Highlighting}
\end{Shaded}

\textbf{For in-lab data collection} an OLS linear regression predicting
scaled score by experimental condition explains a statistically
significant and moderate 15\% variance in score (F(1,124) = 22.7, p
\textless{} 0.001). The estimated beta coefficient (\(/beta\) = 7.88,
95\% CI {[}4.61, 11.2{]}) predicts that participants in the impasse
condition will on average score around 8 points (31\%) higher than those
in the control condition.

\begin{Shaded}
\begin{Highlighting}[]
\CommentTok{\#SCORE predicted by CONDITION}
\NormalTok{m1 }\OtherTok{\textless{}{-}} \FunctionTok{lm}\NormalTok{(s\_SCALED }\SpecialCharTok{\textasciitilde{}}\NormalTok{ condition, }\AttributeTok{data =}\NormalTok{ df\_subjects }\SpecialCharTok{\%\textgreater{}\%} \FunctionTok{filter}\NormalTok{(mode}\SpecialCharTok{==}\StringTok{"asynch"}\NormalTok{))}
\FunctionTok{paste}\NormalTok{(}\StringTok{"Model"}\NormalTok{)}
\end{Highlighting}
\end{Shaded}

\begin{verbatim}
[1] "Model"
\end{verbatim}

\begin{Shaded}
\begin{Highlighting}[]
\FunctionTok{summary}\NormalTok{(m1)}
\end{Highlighting}
\end{Shaded}

\begin{verbatim}

Call:
lm(formula = s_SCALED ~ condition, data = df_subjects %>% filter(mode == 
    "asynch"))

Residuals:
   Min     1Q Median     3Q    Max 
-12.31  -6.63  -3.63   7.74  19.36 

Coefficients:
             Estimate Std. Error t value Pr(>|t|)    
(Intercept)    -6.365      0.883   -7.21  1.1e-11 ***
condition121    6.670      1.214    5.49  1.2e-07 ***
---
Signif. codes:  0 '***' 0.001 '**' 0.01 '*' 0.05 '.' 0.1 ' ' 1

Residual standard error: 8.65 on 202 degrees of freedom
Multiple R-squared:  0.13,  Adjusted R-squared:  0.126 
F-statistic: 30.2 on 1 and 202 DF,  p-value: 1.17e-07
\end{verbatim}

\begin{Shaded}
\begin{Highlighting}[]
\FunctionTok{paste}\NormalTok{(}\StringTok{"Partition Variance"}\NormalTok{)}
\end{Highlighting}
\end{Shaded}

\begin{verbatim}
[1] "Partition Variance"
\end{verbatim}

\begin{Shaded}
\begin{Highlighting}[]
\FunctionTok{anova}\NormalTok{(m1)}
\end{Highlighting}
\end{Shaded}

\begin{verbatim}
Analysis of Variance Table

Response: s_SCALED
           Df Sum Sq Mean Sq F value  Pr(>F)    
condition   1   2261    2261    30.2 1.2e-07 ***
Residuals 202  15128      75                    
---
Signif. codes:  0 '***' 0.001 '**' 0.01 '*' 0.05 '.' 0.1 ' ' 1
\end{verbatim}

\begin{Shaded}
\begin{Highlighting}[]
\FunctionTok{paste}\NormalTok{(}\StringTok{"Confidence Interval on Parameter Estimates"}\NormalTok{)}
\end{Highlighting}
\end{Shaded}

\begin{verbatim}
[1] "Confidence Interval on Parameter Estimates"
\end{verbatim}

\begin{Shaded}
\begin{Highlighting}[]
\FunctionTok{confint}\NormalTok{(m1)}
\end{Highlighting}
\end{Shaded}

\begin{verbatim}
             2.5 % 97.5 %
(Intercept)  -8.11  -4.62
condition121  4.28   9.06
\end{verbatim}

\begin{Shaded}
\begin{Highlighting}[]
\CommentTok{\#report(m1) \#sanity check}
\end{Highlighting}
\end{Shaded}

\textbf{For the online replication}, an OLS linear regression model
predicting scaled score by condition explains a statistically
significant and moderate 14\% of variance in absolute score (F(1,202) =
32.4, p \textless{} 0.001). The beta coefficient for condition indicates
that on average, participants in the IMPASSE group scored 6.8 points
higher on the task than those in the control condition (CI{[}4.43,
9.12{]}).

\begin{tcolorbox}[standard jigsaw,bottomrule=.15mm, opacitybacktitle=0.6, bottomtitle=1mm, toptitle=1mm, titlerule=0mm, title=\textcolor{quarto-callout-note-color}{\faInfo}\hspace{0.5em}{Note}, toprule=.15mm, rightrule=.15mm, colback=white, arc=.35mm, left=2mm, colframe=quarto-callout-note-color-frame, coltitle=black, leftrule=.75mm, opacityback=0, colbacktitle=quarto-callout-note-color!10!white]
\textbf{From these models we can reasonably conclude that the impasse
condition yields a reliable, moderate sized effect of improved
interpretation on the graph reading tasks.}
\end{tcolorbox}

\hypertarget{item-level-performance}{%
\section{Item-Level Performance}\label{item-level-performance}}

Individual differences with a mixed model.

\hypertarget{model-peeking}{%
\section{Model Peeking}\label{model-peeking}}

TODO

\begin{itemize}
\tightlist
\item
  multiple regression with condition and response time
\end{itemize}

\begin{Shaded}
\begin{Highlighting}[]
\FunctionTok{library}\NormalTok{(supernova)}
\end{Highlighting}
\end{Shaded}

\begin{verbatim}

Attaching package: 'supernova'
\end{verbatim}

\begin{verbatim}
The following object is masked from 'package:scales':

    number
\end{verbatim}

\begin{Shaded}
\begin{Highlighting}[]
\FunctionTok{library}\NormalTok{(report)}
\FunctionTok{library}\NormalTok{(lmerTest)}
\end{Highlighting}
\end{Shaded}

\begin{verbatim}
Loading required package: lme4
\end{verbatim}

\begin{verbatim}
Loading required package: Matrix
\end{verbatim}

\begin{verbatim}

Attaching package: 'Matrix'
\end{verbatim}

\begin{verbatim}
The following objects are masked from 'package:tidyr':

    expand, pack, unpack
\end{verbatim}

\begin{verbatim}

Attaching package: 'lmerTest'
\end{verbatim}

\begin{verbatim}
The following object is masked from 'package:lme4':

    lmer
\end{verbatim}

\begin{verbatim}
The following object is masked from 'package:stats':

    step
\end{verbatim}

\begin{Shaded}
\begin{Highlighting}[]
\FunctionTok{library}\NormalTok{(lme4)}
\NormalTok{m1 }\OtherTok{\textless{}{-}} \FunctionTok{lm}\NormalTok{( s\_SCALED }\SpecialCharTok{\textasciitilde{}}\NormalTok{ condition, }\AttributeTok{data =}\NormalTok{ df\_subjects }\SpecialCharTok{\%\textgreater{}\%} \FunctionTok{filter}\NormalTok{(mode}\SpecialCharTok{==}\StringTok{\textquotesingle{}asynch\textquotesingle{}}\NormalTok{))}
\NormalTok{m1}
\end{Highlighting}
\end{Shaded}

\begin{verbatim}

Call:
lm(formula = s_SCALED ~ condition, data = df_subjects %>% filter(mode == 
    "asynch"))

Coefficients:
 (Intercept)  condition121  
       -6.36          6.67  
\end{verbatim}

\begin{Shaded}
\begin{Highlighting}[]
\FunctionTok{summary}\NormalTok{(m1)}
\end{Highlighting}
\end{Shaded}

\begin{verbatim}

Call:
lm(formula = s_SCALED ~ condition, data = df_subjects %>% filter(mode == 
    "asynch"))

Residuals:
   Min     1Q Median     3Q    Max 
-12.31  -6.63  -3.63   7.74  19.36 

Coefficients:
             Estimate Std. Error t value Pr(>|t|)    
(Intercept)    -6.365      0.883   -7.21  1.1e-11 ***
condition121    6.670      1.214    5.49  1.2e-07 ***
---
Signif. codes:  0 '***' 0.001 '**' 0.01 '*' 0.05 '.' 0.1 ' ' 1

Residual standard error: 8.65 on 202 degrees of freedom
Multiple R-squared:  0.13,  Adjusted R-squared:  0.126 
F-statistic: 30.2 on 1 and 202 DF,  p-value: 1.17e-07
\end{verbatim}

\begin{Shaded}
\begin{Highlighting}[]
\FunctionTok{anova}\NormalTok{(m1)}
\end{Highlighting}
\end{Shaded}

\begin{verbatim}
Analysis of Variance Table

Response: s_SCALED
           Df Sum Sq Mean Sq F value  Pr(>F)    
condition   1   2261    2261    30.2 1.2e-07 ***
Residuals 202  15128      75                    
---
Signif. codes:  0 '***' 0.001 '**' 0.01 '*' 0.05 '.' 0.1 ' ' 1
\end{verbatim}

\begin{Shaded}
\begin{Highlighting}[]
\FunctionTok{superanova}\NormalTok{(m1)}
\end{Highlighting}
\end{Shaded}

\begin{verbatim}
 Analysis of Variance Table (Type III SS)
 Model: s_SCALED ~ condition

                                SS  df       MS      F    PRE     p
 ----- --------------- | --------- --- -------- ------ ------ -----
 Model (error reduced) |  2261.177   1 2261.177 30.193 0.1300 .0000
 Error (from model)    | 15128.156 202   74.892                    
 ----- --------------- | --------- --- -------- ------ ------ -----
 Total (empty model)   | 17389.333 203   85.662                    
\end{verbatim}

\begin{Shaded}
\begin{Highlighting}[]
\FunctionTok{plot}\NormalTok{(m1)}
\end{Highlighting}
\end{Shaded}

\begin{figure}[H]

{\centering \includegraphics{analysis/SGC3A/5_sgc3A_exploration_files/figure-pdf/unnamed-chunk-10-1.pdf}

}

\end{figure}

\begin{figure}[H]

{\centering \includegraphics{analysis/SGC3A/5_sgc3A_exploration_files/figure-pdf/unnamed-chunk-10-2.pdf}

}

\end{figure}

\begin{figure}[H]

{\centering \includegraphics{analysis/SGC3A/5_sgc3A_exploration_files/figure-pdf/unnamed-chunk-10-3.pdf}

}

\end{figure}

\begin{figure}[H]

{\centering \includegraphics{analysis/SGC3A/5_sgc3A_exploration_files/figure-pdf/unnamed-chunk-10-4.pdf}

}

\end{figure}

\begin{Shaded}
\begin{Highlighting}[]
\FunctionTok{gf\_histogram}\NormalTok{(}\SpecialCharTok{\textasciitilde{}}\NormalTok{s\_SCALED, }\AttributeTok{data =}\NormalTok{ df\_subjects)}
\end{Highlighting}
\end{Shaded}

\begin{figure}[H]

{\centering \includegraphics{analysis/SGC3A/5_sgc3A_exploration_files/figure-pdf/unnamed-chunk-10-5.pdf}

}

\end{figure}

\begin{Shaded}
\begin{Highlighting}[]
\FunctionTok{gf\_histogram}\NormalTok{(}\SpecialCharTok{\textasciitilde{}}\NormalTok{m1}\SpecialCharTok{$}\NormalTok{residuals)}
\end{Highlighting}
\end{Shaded}

\begin{figure}[H]

{\centering \includegraphics{analysis/SGC3A/5_sgc3A_exploration_files/figure-pdf/unnamed-chunk-10-6.pdf}

}

\end{figure}

\begin{Shaded}
\begin{Highlighting}[]
\CommentTok{\#Assess assumption of independence of errors}
\CommentTok{\#DW statistic should be close to 2}
\FunctionTok{library}\NormalTok{(car)}
\end{Highlighting}
\end{Shaded}

\begin{verbatim}
Loading required package: carData
\end{verbatim}

\begin{verbatim}

Attaching package: 'carData'
\end{verbatim}

\begin{verbatim}
The following object is masked from 'package:vcdExtra':

    Burt
\end{verbatim}

\begin{verbatim}

Attaching package: 'car'
\end{verbatim}

\begin{verbatim}
The following object is masked from 'package:dplyr':

    recode
\end{verbatim}

\begin{verbatim}
The following object is masked from 'package:purrr':

    some
\end{verbatim}

\begin{Shaded}
\begin{Highlighting}[]
\FunctionTok{durbinWatsonTest}\NormalTok{(m1)}
\end{Highlighting}
\end{Shaded}

\begin{verbatim}
 lag Autocorrelation D-W Statistic p-value
   1        -0.00262          1.99   0.942
 Alternative hypothesis: rho != 0
\end{verbatim}

\begin{Shaded}
\begin{Highlighting}[]
\CommentTok{\#Test for equality of variance}
\CommentTok{\#H0 is equality; p \textgreater{} 0.05 infer you can\textquotesingle{}t reject null}
\FunctionTok{leveneTest}\NormalTok{(m1)}
\end{Highlighting}
\end{Shaded}

\begin{verbatim}
Levene's Test for Homogeneity of Variance (center = median)
       Df F value Pr(>F)  
group   1     6.3  0.013 *
      202                 
---
Signif. codes:  0 '***' 0.001 '**' 0.01 '*' 0.05 '.' 0.1 ' ' 1
\end{verbatim}

A simple linear regression model predicting cumulative scaled score (at
subject level) by condition explains 13\% of the total variance,
F(1,329) = 47.8, p \textless{} 0.001. The model predicts that
participants in the impasse condition will score on average 6.38 points
higher than those in the control condition, 95\% CI {[}4.56, 8.19{]}.

\begin{Shaded}
\begin{Highlighting}[]
\FunctionTok{t.test}\NormalTok{(s\_SCALED }\SpecialCharTok{\textasciitilde{}}\NormalTok{ condition, }\AttributeTok{data =}\NormalTok{ df\_subjects)}
\end{Highlighting}
\end{Shaded}

\begin{verbatim}

    Welch Two Sample t-test

data:  s_SCALED by condition
t = -7, df = 325, p-value = 7e-12
alternative hypothesis: true difference in means between group 111 and group 121 is not equal to 0
95 percent confidence interval:
 -8.93 -5.06
sample estimates:
mean in group 111 mean in group 121 
           -6.427             0.567 
\end{verbatim}

\begin{Shaded}
\begin{Highlighting}[]
\CommentTok{\#\%\textgreater{}\% report()}
\end{Highlighting}
\end{Shaded}

\begin{Shaded}
\begin{Highlighting}[]
\CommentTok{\# report\_participants(df\_subjects)}
\NormalTok{m1 }\SpecialCharTok{\%\textgreater{}\%} \FunctionTok{report}\NormalTok{()}
\end{Highlighting}
\end{Shaded}

\begin{verbatim}
Warning: 'data_findcols()' is deprecated and will be removed in a future update.
  Its usage is discouraged. Please use 'data_find()' instead.

Warning: 'data_findcols()' is deprecated and will be removed in a future update.
  Its usage is discouraged. Please use 'data_find()' instead.

Warning: 'data_findcols()' is deprecated and will be removed in a future update.
  Its usage is discouraged. Please use 'data_find()' instead.
\end{verbatim}

\begin{verbatim}
We fitted a linear model (estimated using OLS) to predict s_SCALED with condition (formula: s_SCALED ~ condition). The model explains a statistically significant and moderate proportion of variance (R2 = 0.13, F(1, 202) = 30.19, p < .001, adj. R2 = 0.13). The model's intercept, corresponding to condition = 111, is at -6.36 (95% CI [-8.11, -4.62], t(202) = -7.21, p < .001). Within this model:

  - The effect of condition [121] is statistically significant and positive (beta = 6.67, 95% CI [4.28, 9.06], t(202) = 5.49, p < .001; Std. beta = 0.72, 95% CI [0.46, 0.98])

Standardized parameters were obtained by fitting the model on a standardized version of the dataset. 95% Confidence Intervals (CIs) and p-values were computed using the Wald approximation.
\end{verbatim}

\begin{Shaded}
\begin{Highlighting}[]
\FunctionTok{anova}\NormalTok{(m1) }\SpecialCharTok{\%\textgreater{}\%} \FunctionTok{report}\NormalTok{()}
\end{Highlighting}
\end{Shaded}

\begin{verbatim}
For one-way between subjects designs, partial eta squared is equivalent to eta squared.
Returning eta squared.
\end{verbatim}

\begin{verbatim}
Warning: 'data_findcols()' is deprecated and will be removed in a future update.
  Its usage is discouraged. Please use 'data_find()' instead.
\end{verbatim}

\begin{verbatim}
The ANOVA suggests that:

  - The main effect of condition is statistically significant and medium (F(1, 202) = 30.19, p < .001; Eta2 = 0.13, 95% CI [0.07, 1.00])

Effect sizes were labelled following Field's (2013) recommendations.
\end{verbatim}

\begin{Shaded}
\begin{Highlighting}[]
\CommentTok{\#significant intercept means that group is significantly different than zero}
\end{Highlighting}
\end{Shaded}

\begin{Shaded}
\begin{Highlighting}[]
\CommentTok{\#logistic regression on on scaled df\_subjects because residuals not normal in lm?}
\NormalTok{mlog }\OtherTok{\textless{}{-}} \FunctionTok{glm}\NormalTok{(s\_SCALED }\SpecialCharTok{\textasciitilde{}}\NormalTok{ condition , }\AttributeTok{data =}\NormalTok{ df\_subjects, }\AttributeTok{family =}\NormalTok{ gaussian)}
\FunctionTok{summary}\NormalTok{(mlog)}
\end{Highlighting}
\end{Shaded}

\begin{verbatim}

Call:
glm(formula = s_SCALED ~ condition, family = gaussian, data = df_subjects)

Deviance Residuals: 
   Min      1Q  Median      3Q     Max  
-12.57   -6.57   -3.82    8.43   19.43  

Coefficients:
             Estimate Std. Error t value Pr(>|t|)    
(Intercept)    -6.427      0.709   -9.06  < 2e-16 ***
condition121    6.994      0.982    7.12  6.8e-12 ***
---
Signif. codes:  0 '***' 0.001 '**' 0.01 '*' 0.05 '.' 0.1 ' ' 1

(Dispersion parameter for gaussian family taken to be 79.4)

    Null deviance: 30088  on 329  degrees of freedom
Residual deviance: 26059  on 328  degrees of freedom
AIC: 2384

Number of Fisher Scoring iterations: 2
\end{verbatim}

\begin{Shaded}
\begin{Highlighting}[]
\FunctionTok{report}\NormalTok{(mlog)}
\end{Highlighting}
\end{Shaded}

\begin{verbatim}
Warning: 'data_findcols()' is deprecated and will be removed in a future update.
  Its usage is discouraged. Please use 'data_find()' instead.

Warning: 'data_findcols()' is deprecated and will be removed in a future update.
  Its usage is discouraged. Please use 'data_find()' instead.

Warning: 'data_findcols()' is deprecated and will be removed in a future update.
  Its usage is discouraged. Please use 'data_find()' instead.
\end{verbatim}

\begin{verbatim}
We fitted a linear model (estimated using ML) to predict s_SCALED with condition (formula: s_SCALED ~ condition). The model's explanatory power is moderate (R2 = 0.13). The model's intercept, corresponding to condition = 111, is at -6.43 (95% CI [-7.82, -5.04], t(328) = -9.06, p < .001). Within this model:

  - The effect of condition [121] is statistically significant and positive (beta = 6.99, 95% CI [5.07, 8.92], t(328) = 7.12, p < .001; Std. beta = 0.73, 95% CI [0.53, 0.93])

Standardized parameters were obtained by fitting the model on a standardized version of the dataset. 95% Confidence Intervals (CIs) and p-values were computed using 
\end{verbatim}

\begin{Shaded}
\begin{Highlighting}[]
\CommentTok{\#logistic regression on niceABS by condition}
\CommentTok{\#pretends that questions are independent and not from same subjects INVALID}
\NormalTok{mlog }\OtherTok{\textless{}{-}} \FunctionTok{glm}\NormalTok{(score\_niceABS }\SpecialCharTok{\textasciitilde{}}\NormalTok{ condition , }\AttributeTok{data =}\NormalTok{ df\_items }\SpecialCharTok{\%\textgreater{}\%} \FunctionTok{filter}\NormalTok{(q}\SpecialCharTok{\textless{}}\DecValTok{6}\NormalTok{), }\AttributeTok{family =} \FunctionTok{binomial}\NormalTok{())}
\FunctionTok{summary}\NormalTok{(mlog)}
\end{Highlighting}
\end{Shaded}

\begin{verbatim}

Call:
glm(formula = score_niceABS ~ condition, family = binomial(), 
    data = df_items %>% filter(q < 6))

Deviance Residuals: 
   Min      1Q  Median      3Q     Max  
-0.980  -0.980  -0.649   1.389   1.823  

Coefficients:
             Estimate Std. Error z value Pr(>|z|)    
(Intercept)   -1.4508     0.0907  -15.99   <2e-16 ***
condition121   0.9672     0.1147    8.43   <2e-16 ***
---
Signif. codes:  0 '***' 0.001 '**' 0.01 '*' 0.05 '.' 0.1 ' ' 1

(Dispersion parameter for binomial family taken to be 1)

    Null deviance: 1986.2  on 1649  degrees of freedom
Residual deviance: 1911.3  on 1648  degrees of freedom
AIC: 1915

Number of Fisher Scoring iterations: 4
\end{verbatim}

\begin{Shaded}
\begin{Highlighting}[]
\FunctionTok{report}\NormalTok{(mlog)}
\end{Highlighting}
\end{Shaded}

\begin{verbatim}
Warning: 'data_findcols()' is deprecated and will be removed in a future update.
  Its usage is discouraged. Please use 'data_find()' instead.

Warning: 'data_findcols()' is deprecated and will be removed in a future update.
  Its usage is discouraged. Please use 'data_find()' instead.

Warning: 'data_findcols()' is deprecated and will be removed in a future update.
  Its usage is discouraged. Please use 'data_find()' instead.

Warning: 'data_findcols()' is deprecated and will be removed in a future update.
  Its usage is discouraged. Please use 'data_find()' instead.

Warning: 'data_findcols()' is deprecated and will be removed in a future update.
  Its usage is discouraged. Please use 'data_find()' instead.
\end{verbatim}

\begin{verbatim}
We fitted a logistic model (estimated using ML) to predict score_niceABS with condition (formula: score_niceABS ~ condition). The model's explanatory power is weak (Tjur's R2 = 0.04). The model's intercept, corresponding to condition = 111, is at -1.45 (95% CI [-1.63, -1.28], p < .001). Within this model:

  - The effect of condition [121] is statistically significant and positive (beta = 0.97, 95% CI [0.74, 1.19], p < .001; Std. beta = 0.97, 95% CI [0.74, 1.19])

Standardized parameters were obtained by fitting the model on a standardized version of the dataset. 95% Confidence Intervals (CIs) and p-values were computed using 
\end{verbatim}

\begin{Shaded}
\begin{Highlighting}[]
\NormalTok{m2 }\OtherTok{\textless{}{-}} \FunctionTok{lm}\NormalTok{( s\_NABS }\SpecialCharTok{\textasciitilde{}}\NormalTok{ condition, }\AttributeTok{data =}\NormalTok{ df\_subjects)}
\NormalTok{m2}
\end{Highlighting}
\end{Shaded}

\begin{verbatim}

Call:
lm(formula = s_NABS ~ condition, data = df_subjects)

Coefficients:
 (Intercept)  condition121  
        2.47          2.46  
\end{verbatim}

\begin{Shaded}
\begin{Highlighting}[]
\FunctionTok{summary}\NormalTok{(m2)}
\end{Highlighting}
\end{Shaded}

\begin{verbatim}

Call:
lm(formula = s_NABS ~ condition, data = df_subjects)

Residuals:
   Min     1Q Median     3Q    Max 
 -4.92  -3.67  -2.47   4.08  10.53 

Coefficients:
             Estimate Std. Error t value Pr(>|t|)    
(Intercept)     2.468      0.384    6.43  4.4e-10 ***
condition121    2.456      0.531    4.62  5.5e-06 ***
---
Signif. codes:  0 '***' 0.001 '**' 0.01 '*' 0.05 '.' 0.1 ' ' 1

Residual standard error: 4.82 on 328 degrees of freedom
Multiple R-squared:  0.0611,    Adjusted R-squared:  0.0583 
F-statistic: 21.4 on 1 and 328 DF,  p-value: 5.49e-06
\end{verbatim}

\begin{Shaded}
\begin{Highlighting}[]
\FunctionTok{anova}\NormalTok{(m2)}
\end{Highlighting}
\end{Shaded}

\begin{verbatim}
Analysis of Variance Table

Response: s_NABS
           Df Sum Sq Mean Sq F value  Pr(>F)    
condition   1    497     497    21.4 5.5e-06 ***
Residuals 328   7629      23                    
---
Signif. codes:  0 '***' 0.001 '**' 0.01 '*' 0.05 '.' 0.1 ' ' 1
\end{verbatim}

\begin{Shaded}
\begin{Highlighting}[]
\FunctionTok{supernova}\NormalTok{(m2)}
\end{Highlighting}
\end{Shaded}

\begin{verbatim}
 Analysis of Variance Table (Type III SS)
 Model: s_NABS ~ condition

                               SS  df      MS      F    PRE     p
 ----- --------------- | -------- --- ------- ------ ------ -----
 Model (error reduced) |  496.765   1 496.765 21.357 0.0611 .0000
 Error (from model)    | 7629.359 328  23.260                    
 ----- --------------- | -------- --- ------- ------ ------ -----
 Total (empty model)   | 8126.124 329  24.699                    
\end{verbatim}

A simple linear regression model predicting cumulative absolute score by
condition explains 5\% of variance, F(1,328) = 16.36, p \textless{}
0.001. The model predicts that subjects in the impasse condition will
score on average 2 points higher than those in the control condition
(Beta = 2.02, 95\% CI {[}1.04, 3.00{]})

\begin{Shaded}
\begin{Highlighting}[]
\FunctionTok{report}\NormalTok{(m2)}
\end{Highlighting}
\end{Shaded}

\begin{verbatim}
Warning: 'data_findcols()' is deprecated and will be removed in a future update.
  Its usage is discouraged. Please use 'data_find()' instead.

Warning: 'data_findcols()' is deprecated and will be removed in a future update.
  Its usage is discouraged. Please use 'data_find()' instead.

Warning: 'data_findcols()' is deprecated and will be removed in a future update.
  Its usage is discouraged. Please use 'data_find()' instead.
\end{verbatim}

\begin{verbatim}
We fitted a linear model (estimated using OLS) to predict s_NABS with condition (formula: s_NABS ~ condition). The model explains a statistically significant and weak proportion of variance (R2 = 0.06, F(1, 328) = 21.36, p < .001, adj. R2 = 0.06). The model's intercept, corresponding to condition = 111, is at 2.47 (95% CI [1.71, 3.22], t(328) = 6.43, p < .001). Within this model:

  - The effect of condition [121] is statistically significant and positive (beta = 2.46, 95% CI [1.41, 3.50], t(328) = 4.62, p < .001; Std. beta = 0.49, 95% CI [0.28, 0.70])

Standardized parameters were obtained by fitting the model on a standardized version of the dataset. 95% Confidence Intervals (CIs) and p-values were computed using the Wald approximation.
\end{verbatim}

\begin{Shaded}
\begin{Highlighting}[]
\NormalTok{m.m1 }\OtherTok{\textless{}{-}} \FunctionTok{lmer}\NormalTok{( score\_SCALED }\SpecialCharTok{\textasciitilde{}}\NormalTok{ (}\DecValTok{1} \SpecialCharTok{+}\NormalTok{ condition}\SpecialCharTok{|}\NormalTok{subject), }\AttributeTok{data =}\NormalTok{ df\_items)}
\end{Highlighting}
\end{Shaded}

\begin{verbatim}
Warning in checkConv(attr(opt, "derivs"), opt$par, ctrl = control$checkConv, :
Model failed to converge with max|grad| = 0.00344864 (tol = 0.002, component 1)
\end{verbatim}

\begin{Shaded}
\begin{Highlighting}[]
\NormalTok{m.m1}
\end{Highlighting}
\end{Shaded}

\begin{verbatim}
Linear mixed model fit by REML ['lmerModLmerTest']
Formula: score_SCALED ~ (1 + condition | subject)
   Data: df_items
REML criterion at convergence: 9940
Random effects:
 Groups   Name         Std.Dev. Corr 
 subject  (Intercept)  0.620         
          condition121 0.865    -0.72
 Residual              0.601         
Number of obs: 4950, groups:  subject, 330
Fixed Effects:
(Intercept)  
     -0.126  
optimizer (nloptwrap) convergence code: 0 (OK) ; 0 optimizer warnings; 1 lme4 warnings 
\end{verbatim}

\begin{Shaded}
\begin{Highlighting}[]
\FunctionTok{summary}\NormalTok{(m.m1)}
\end{Highlighting}
\end{Shaded}

\begin{verbatim}
Linear mixed model fit by REML. t-tests use Satterthwaite's method [
lmerModLmerTest]
Formula: score_SCALED ~ (1 + condition | subject)
   Data: df_items

REML criterion at convergence: 9940

Scaled residuals: 
   Min     1Q Median     3Q    Max 
-2.902 -0.659 -0.299  0.533  2.718 

Random effects:
 Groups   Name         Variance Std.Dev. Corr 
 subject  (Intercept)  0.384    0.620         
          condition121 0.748    0.865    -0.72
 Residual              0.362    0.601         
Number of obs: 4950, groups:  subject, 330

Fixed effects:
            Estimate Std. Error       df t value Pr(>|t|)    
(Intercept)  -0.1259     0.0346 328.5075   -3.64  0.00032 ***
---
Signif. codes:  0 '***' 0.001 '**' 0.01 '*' 0.05 '.' 0.1 ' ' 1
optimizer (nloptwrap) convergence code: 0 (OK)
Model failed to converge with max|grad| = 0.00344864 (tol = 0.002, component 1)
\end{verbatim}

\begin{Shaded}
\begin{Highlighting}[]
\FunctionTok{report}\NormalTok{(m.m1)}
\end{Highlighting}
\end{Shaded}

\begin{verbatim}
Package 'merDeriv' needs to be installed to compute confidence intervals
  for random effect parameters.
\end{verbatim}

\begin{verbatim}
Warning: 'data_findcols()' is deprecated and will be removed in a future update.
  Its usage is discouraged. Please use 'data_find()' instead.
\end{verbatim}

\begin{verbatim}
Package 'merDeriv' needs to be installed to compute confidence intervals
  for random effect parameters.
\end{verbatim}

\begin{verbatim}
Warning: 'data_findcols()' is deprecated and will be removed in a future update.
  Its usage is discouraged. Please use 'data_find()' instead.
\end{verbatim}

\begin{verbatim}
Warning: 'data_findcols()' is deprecated and will be removed in a future update.
  Its usage is discouraged. Please use 'data_find()' instead.
\end{verbatim}

\begin{verbatim}
We fitted a constant (intercept-only) linear mixed model (estimated using REML and nloptwrap optimizer) to predict score_SCALED (formula: score_SCALED ~ 1). The model included condition and subject as random effects (formula: ~1 + condition | subject). . The model's intercept is at -0.13 (95% CI [-0.19, -0.06], t(4945) = -3.64, p < .001). Within this model:

  -  ()

Standardized parameters were obtained by fitting the model on a standardized version of the dataset. 95% Confidence Intervals (CIs) and p-values were computed using 
\end{verbatim}

\begin{Shaded}
\begin{Highlighting}[]
\NormalTok{m.m2 }\OtherTok{\textless{}{-}} \FunctionTok{lmer}\NormalTok{( score\_SCALED }\SpecialCharTok{\textasciitilde{}}\NormalTok{ (}\DecValTok{1}\SpecialCharTok{+}\NormalTok{ condition}\SpecialCharTok{|}\NormalTok{q), }\AttributeTok{data =}\NormalTok{ df\_items)}
\NormalTok{m.m2}
\end{Highlighting}
\end{Shaded}

\begin{verbatim}
Linear mixed model fit by REML ['lmerModLmerTest']
Formula: score_SCALED ~ (1 + condition | q)
   Data: df_items
REML criterion at convergence: 11669
Random effects:
 Groups   Name         Std.Dev. Corr 
 q        (Intercept)  0.577         
          condition121 0.517    -0.93
 Residual              0.778         
Number of obs: 4950, groups:  q, 15
Fixed Effects:
(Intercept)  
     0.0983  
\end{verbatim}

\begin{Shaded}
\begin{Highlighting}[]
\FunctionTok{summary}\NormalTok{(m.m2)}
\end{Highlighting}
\end{Shaded}

\begin{verbatim}
Linear mixed model fit by REML. t-tests use Satterthwaite's method [
lmerModLmerTest]
Formula: score_SCALED ~ (1 + condition | q)
   Data: df_items

REML criterion at convergence: 11669

Scaled residuals: 
   Min     1Q Median     3Q    Max 
-1.830 -0.732 -0.424  0.910  2.147 

Random effects:
 Groups   Name         Variance Std.Dev. Corr 
 q        (Intercept)  0.333    0.577         
          condition121 0.267    0.517    -0.93
 Residual              0.605    0.778         
Number of obs: 4950, groups:  q, 15

Fixed effects:
            Estimate Std. Error      df t value Pr(>|t|)
(Intercept)   0.0983     0.0570 14.0000    1.73     0.11
\end{verbatim}

\begin{Shaded}
\begin{Highlighting}[]
\FunctionTok{report}\NormalTok{(m.m2)}
\end{Highlighting}
\end{Shaded}

\begin{verbatim}
Package 'merDeriv' needs to be installed to compute confidence intervals
  for random effect parameters.
\end{verbatim}

\begin{verbatim}
Warning: 'data_findcols()' is deprecated and will be removed in a future update.
  Its usage is discouraged. Please use 'data_find()' instead.
\end{verbatim}

\begin{verbatim}
Package 'merDeriv' needs to be installed to compute confidence intervals
  for random effect parameters.
\end{verbatim}

\begin{verbatim}
Warning: 'data_findcols()' is deprecated and will be removed in a future update.
  Its usage is discouraged. Please use 'data_find()' instead.

Warning: 'data_findcols()' is deprecated and will be removed in a future update.
  Its usage is discouraged. Please use 'data_find()' instead.
\end{verbatim}

\begin{verbatim}
We fitted a constant (intercept-only) linear mixed model (estimated using REML and nloptwrap optimizer) to predict score_SCALED (formula: score_SCALED ~ 1). The model included condition and q as random effects (formula: ~1 + condition | q). . The model's intercept is at 0.10 (95% CI [-0.01, 0.21], t(4945) = 1.73, p = 0.084). Within this model:

  -  ()

Standardized parameters were obtained by fitting the model on a standardized version of the dataset. 95% Confidence Intervals (CIs) and p-values were computed using 
\end{verbatim}

\begin{Shaded}
\begin{Highlighting}[]
\NormalTok{m.m3 }\OtherTok{\textless{}{-}} \FunctionTok{lmer}\NormalTok{( score\_SCALED }\SpecialCharTok{\textasciitilde{}}\NormalTok{ (}\DecValTok{1}\SpecialCharTok{+}\NormalTok{ condition}\SpecialCharTok{|}\NormalTok{q) }\SpecialCharTok{+}\NormalTok{ (}\DecValTok{1}\SpecialCharTok{+}\NormalTok{condition}\SpecialCharTok{|}\NormalTok{subject), }\AttributeTok{data =}\NormalTok{ df\_items)}
\end{Highlighting}
\end{Shaded}

\begin{verbatim}
Warning in checkConv(attr(opt, "derivs"), opt$par, ctrl = control$checkConv, :
Model failed to converge with max|grad| = 0.0117282 (tol = 0.002, component 1)
\end{verbatim}

\begin{Shaded}
\begin{Highlighting}[]
\NormalTok{m.m3 }\SpecialCharTok{\%\textgreater{}\%} \FunctionTok{summary}\NormalTok{() }
\end{Highlighting}
\end{Shaded}

\begin{verbatim}
Linear mixed model fit by REML. t-tests use Satterthwaite's method [
lmerModLmerTest]
Formula: score_SCALED ~ (1 + condition | q) + (1 + condition | subject)
   Data: df_items

REML criterion at convergence: 8836

Scaled residuals: 
   Min     1Q Median     3Q    Max 
-3.796 -0.585 -0.083  0.662  3.431 

Random effects:
 Groups   Name         Variance Std.Dev. Corr 
 subject  (Intercept)  0.3334   0.577         
          condition121 0.0951   0.308    -0.30
 q        (Intercept)  0.2445   0.494         
          condition121 0.1848   0.430    -0.90
 Residual              0.2804   0.530         
Number of obs: 4950, groups:  subject, 330; q, 15

Fixed effects:
            Estimate Std. Error      df t value Pr(>|t|)
(Intercept)   0.0434     0.0706 24.8205    0.61     0.54
optimizer (nloptwrap) convergence code: 0 (OK)
Model failed to converge with max|grad| = 0.0117282 (tol = 0.002, component 1)
\end{verbatim}

\begin{Shaded}
\begin{Highlighting}[]
\NormalTok{m.m3 }\SpecialCharTok{\%\textgreater{}\%} \FunctionTok{report}\NormalTok{()}
\end{Highlighting}
\end{Shaded}

\begin{verbatim}
Package 'merDeriv' needs to be installed to compute confidence intervals
  for random effect parameters.
\end{verbatim}

\begin{verbatim}
Warning: 'data_findcols()' is deprecated and will be removed in a future update.
  Its usage is discouraged. Please use 'data_find()' instead.
\end{verbatim}

\begin{verbatim}
Package 'merDeriv' needs to be installed to compute confidence intervals
  for random effect parameters.
\end{verbatim}

\begin{verbatim}
Warning: 'data_findcols()' is deprecated and will be removed in a future update.
  Its usage is discouraged. Please use 'data_find()' instead.
\end{verbatim}

\begin{verbatim}
Warning: 'data_findcols()' is deprecated and will be removed in a future update.
  Its usage is discouraged. Please use 'data_find()' instead.
\end{verbatim}

\begin{verbatim}
We fitted a constant (intercept-only) linear mixed model (estimated using REML and nloptwrap optimizer) to predict score_SCALED (formula: score_SCALED ~ 1). The model included condition, q and subject as random effects (formula: list(~1 + condition | q, ~1 + condition | subject)). . The model's intercept is at 0.04 (95% CI [-0.10, 0.18], t(4942) = 0.61, p = 0.539). Within this model:

  -  ()

Standardized parameters were obtained by fitting the model on a standardized version of the dataset. 95% Confidence Intervals (CIs) and p-values were computed using 
\end{verbatim}

\begin{Shaded}
\begin{Highlighting}[]
\FunctionTok{anova}\NormalTok{(m.m1, m.m2, m.m3)}
\end{Highlighting}
\end{Shaded}

\begin{verbatim}
refitting model(s) with ML (instead of REML)
\end{verbatim}

\begin{verbatim}
Data: df_items
Models:
m.m1: score_SCALED ~ (1 + condition | subject)
m.m2: score_SCALED ~ (1 + condition | q)
m.m3: score_SCALED ~ (1 + condition | q) + (1 + condition | subject)
     npar   AIC   BIC logLik deviance Chisq Df Pr(>Chisq)    
m.m1    5  9945  9978  -4968     9935                        
m.m2    5 11675 11708  -5832    11665     0  0               
m.m3    8  8849  8901  -4416     8833  2832  3     <2e-16 ***
---
Signif. codes:  0 '***' 0.001 '**' 0.01 '*' 0.05 '.' 0.1 ' ' 1
\end{verbatim}

\hypertarget{resources-4}{%
\section{RESOURCES}\label{resources-4}}

\part{SGC3B}

\newpage

\textbf{TODO UPDATE ALL}

\hypertarget{sec-SGC3B-introduction}{%
\chapter{Introduction}\label{sec-SGC3B-introduction}}

\textbf{In Study 3B we compare the efficacy of the explicit
{[}interaction{]} scaffold and the implicit {[}impasse{]} scaffold.}

\begin{longtable}[]{@{}
  >{\raggedright\arraybackslash}p{(\columnwidth - 2\tabcolsep) * \real{0.2518}}
  >{\raggedright\arraybackslash}p{(\columnwidth - 2\tabcolsep) * \real{0.7482}}@{}}
\toprule()
\endhead
\includegraphics{analysis/utils/img/111.png} &
\begin{minipage}[t]{\linewidth}\raggedright
\textbf{Control-Condition}\\
Demo:
\href{https://limitless-plains-85018.herokuapp.com/?study=SGC3B\&condition=111\&session=WEB-DEMO}{111}\strut
\end{minipage} \\
\includegraphics{analysis/utils/img/121.png} &
\begin{minipage}[t]{\linewidth}\raggedright
\textbf{Impasse-Condition}\\
Demo:
\href{https://limitless-plains-85018.herokuapp.com/?study=SGC3B\&condition=121\&session=WEB-DEMO}{121}\strut
\end{minipage} \\
\bottomrule()
\end{longtable}

\begin{Shaded}
\begin{Highlighting}[]
\FunctionTok{library}\NormalTok{(codebook) }\CommentTok{\#data dictionary}
\FunctionTok{library}\NormalTok{(tidyverse) }\CommentTok{\#ALL THE THINGS}
\FunctionTok{library}\NormalTok{(kableExtra) }\CommentTok{\#tables}

\CommentTok{\#set some output options}
\FunctionTok{library}\NormalTok{(dplyr, }\AttributeTok{warn.conflicts =} \ConstantTok{FALSE}\NormalTok{)}
\FunctionTok{options}\NormalTok{(}\AttributeTok{dplyr.summarise.inform =} \ConstantTok{FALSE}\NormalTok{)}
\FunctionTok{options}\NormalTok{(}\AttributeTok{scipen=}\DecValTok{1}\NormalTok{, }\AttributeTok{digits=}\DecValTok{3}\NormalTok{)}
\end{Highlighting}
\end{Shaded}

\begin{Shaded}
\begin{Highlighting}[]
\CommentTok{\# }\AlertTok{HACK}\CommentTok{ WD FOR LOCAL RUNNING?}
\NormalTok{imac }\OtherTok{=} \StringTok{"/Users/amyraefox/Code/SGC{-}Scaffolding\_Graph\_Comprehension/SGC{-}X/ANALYSIS/MAIN"}
\CommentTok{\# \# mbp = "/Users/amyfox/Sites/RESEARCH/SGC—Scaffolding Graph Comprehension/SGC{-}X/ANALYSIS/MAIN"}
\FunctionTok{setwd}\NormalTok{(imac)}

\CommentTok{\#IMPORT DATA }
\NormalTok{df\_subjects }\OtherTok{\textless{}{-}} \FunctionTok{read\_rds}\NormalTok{(}\StringTok{\textquotesingle{}analysis/SGC3B/data/1{-}study{-}level/sgc3b\_participants.rds\textquotesingle{}}\NormalTok{)}
\end{Highlighting}
\end{Shaded}

\begin{Shaded}
\begin{Highlighting}[]
\NormalTok{title }\OtherTok{=} \StringTok{"Participants by Condition and Data Collection Period"}
\NormalTok{cols }\OtherTok{=} \FunctionTok{c}\NormalTok{(}\StringTok{"NONE{-}NONE"}\NormalTok{,}\StringTok{"NONE{-}Impasse"}\NormalTok{,}\StringTok{"IMG{-}NONE"}\NormalTok{,}\StringTok{"IMG{-}Impasse"}\NormalTok{,}\StringTok{"IXV{-}NONE"}\NormalTok{,}\StringTok{"IXV{-}Impasse"}\NormalTok{,}\StringTok{"Total for Period"}\NormalTok{)}
\NormalTok{cont }\OtherTok{\textless{}{-}} \FunctionTok{table}\NormalTok{(df\_subjects}\SpecialCharTok{$}\NormalTok{pretty\_mode, df\_subjects}\SpecialCharTok{$}\NormalTok{condition)}
\NormalTok{cont }\SpecialCharTok{\%\textgreater{}\%} \FunctionTok{addmargins}\NormalTok{() }\SpecialCharTok{\%\textgreater{}\%} \FunctionTok{kbl}\NormalTok{(}\AttributeTok{caption =}\NormalTok{ title, }\AttributeTok{col.names =}\NormalTok{ cols) }\SpecialCharTok{\%\textgreater{}\%}  \FunctionTok{kable\_classic}\NormalTok{()}
\end{Highlighting}
\end{Shaded}

\begin{table}

\caption{Participants by Condition and Data Collection Period}
\centering
\begin{tabular}[t]{l|r|r|r|r|r|r|r}
\hline
  & NONE-NONE & NONE-Impasse & IMG-NONE & IMG-Impasse & IXV-NONE & IXV-Impasse & Total for Period\\
\hline
laboratory & 62 & 64 & 45 & 30 & 59 & 30 & 290\\
\hline
online-replication & 68 & 71 & 5 & 12 & 3 & 11 & 170\\
\hline
Sum & 130 & 135 & 50 & 42 & 62 & 41 & 460\\
\hline
\end{tabular}
\end{table}

\hypertarget{hypotheses-1}{%
\subsection{Hypotheses}\label{hypotheses-1}}

\textbf{Experimental Hypothesis}\\
\emph{Learners posed with scenario designed to evoke a mental impasse
will be more likely to correct interpret the graph.}

\begin{itemize}
\tightlist
\item
  H1A \textbar{} Learners in the IMPASSE condition will score higher on
  the TEST Phase than learners in CONTROL.
\item
  H1B \textbar{} Learners in the IMPASSE condition will be more likely
  to correctly answer the first question than learners in CONTROL.
\item
  H1C \textbar{} Learners in the IMPASSE condition will spend more time
  on the first question than learners in CONTROL.
\end{itemize}

\textbf{Null Hypothesis}\\
\emph{No significant differences in performance will exist between
learners in the IMPASSE and CONTROL conditions.}

\textbf{Exploratory Questions}

\begin{itemize}
\tightlist
\item
  Consistency \textbar{} How consistent are learners in their
  interpretation of the graph? Do they adopt an interpretation on the
  first question and hold constant? Or do they change interpretations
  from question to question? Are there any interpretations that serve as
  `absorbing states' (i.e.~once encountered, the learner does not exist
  this state).
\item
  Time Course of Exploration \textbar{} What is the relationship between
  response accuracy (and interpretation) and time spent on each item?
\item
  Can exploration strategies be derived from mouse cursor activity?
\end{itemize}

\hypertarget{methods-1}{%
\section{METHODS}\label{methods-1}}

\hypertarget{design-1}{%
\subsection{Design}\label{design-1}}

We employed a mixed design with 1 between-subjects factor with 2 levels
(Scaffold: control, impasse) and 15 items (within-subjects factor).

Independent Variables:

\begin{itemize}
\tightlist
\item
  B-S (Scaffold: control,impasse)
\item
  W-S (Item x 15)
\end{itemize}

Dependent Variables:

\begin{itemize}
\tightlist
\item
  Response Accuracy : Is the response triangular-correct?
\item
  Response Interpretation : (derived) With which interpretation of the
  graph is the subject's response on an individual question consistent?
\item
  Response Latency : Time from stimulus onset to clicking `Submit'
  button: time in (s)
\end{itemize}

\hypertarget{materials-1}{%
\subsection{Materials}\label{materials-1}}

Stimuli consisted of a series of 15 graph comprehension questions, each
testing a different combination of time interval relations, to be read
from a Triangular-Model graph. Figure~\ref{fig-sample}. The list of
questions can be found \href{static/stimuli/sgcx_questions.csv}{here}.

\begin{figure}

{\centering \includegraphics{analysis/SGC3B/static/stimuli/sample_task.png}

}

\caption{\label{fig-sample}Sample Question (Q=1) for Graph Comprehension
Task}

\end{figure}

Note that across both control and impasse conditions, both the question,
response options and graph structure were identical. The experimental
manipulation (posing a mental impasse) was accomplished by changing the
position of datapoints in the impasse-condition graph, such that for any
given question, there was no available response option if the reader
were to interpret the graph as cartesian (making an orthogonal rather
than diagonal projection from the x-axis.)

\emph{The green line indicates the ideal-scanpath to the correct
(triangular) answer to the first question, and the red line indicates
the (incorrect) orthogonal interpretation. In the IMPASSE figure (at
right), there are no data points that intersect the red line. We
hypothesize that this presents the reader with an obstacle, at which
point they are forced to confront their interpretation of the coordinate
system and (ideally) develop a new strategy.}

\begin{figure}

{\centering \includegraphics{analysis/SGC3B/static/stimuli/3B_conditions.png}

}

\caption{\label{fig-conditions}Sample Question (Q=1) graphs for each
condition}

\end{figure}

\hypertarget{procedure-1}{%
\subsection{Procedure}\label{procedure-1}}

Participants completed the study via a web-browser.

(1) Upon starting, they submitted informed consent, before reading task
instructions.

(2) Participants were introduced to a scenario in which they were to
play the role of a project manager, scheduling shifts for a group of
employees. The schedule of the employees was presented in a
TriangularModel (TM) graph, and they would be answering question about
the schedule.

(3) Then participants completed an experimental block of 15 items.

(3B) The first five items in the task are defined as the SCAFFOLDING
block. In the IMPASSE condition, the first five questions included an
IMPASSE problem state. For participants in the CONTROL condition, the
dataset was structure such that there was always an available
`orthogonal answer' for the first 5 questions.

(3B) The remaining 10 items are defined as the TESTING block. In both
conditions, these questions were not structured as impasse
(i.e.~contained an available orthogonal answer)

(4) Following the experimental block, participants answered a
free-response question about their strategy for reading the graph,
followed by a demographic questionnaire and debrief.

\hypertarget{sample-2}{%
\subsection{Sample}\label{sample-2}}

Data was collected by convenience sample of a university subject pool.
Initial data (Fall 2017, Spring 2018) were collected in-person, with
large groups of students simultaneously completing the study
(independently) in a computer lab. In Fall 2021 and Winter 2022 we
collected additional data to replicate results in a remote format
(students completing the study asynchronously on their own computers).

\hypertarget{analysis-1}{%
\section{ANALYSIS}\label{analysis-1}}

\hypertarget{sec-SGC3B-harmonize}{%
\subsection{Data Preparation}\label{sec-SGC3B-harmonize}}

Data were collected via a custom web application and stored in a NoSQL
database. The following exclusion criteria were applied during data
cleaning:

\begin{itemize}
\tightlist
\item
  completion status : ``success'' ; subject must have finished all parts
  of the study, including demographic questionnaire
\item
  session ID: {[}in list{]} ; subject must have been assigned to valid
  data collection session (discard testing and piloting data)
\item
  browser interaction violations \textless{} 3; subject must have fewer
  than 3 violations of non-allowed browser interactions (i.e.~resizing
  window, leaving browser tab or leaving fullscreen mode)
\item
  self-rated effort \textgreater{} 2; subjects who reported, ``not
  trying hard/rushing through questions'' or ``started out trying hard
  but giving up at some point'' were excluded from analysis.
\item
  attention check ==TRUE ; subjects who failed to answer a mid-study
  attention check question (Graph Comprehension Task Question \#6) are
  excluded
\end{itemize}

Before analysis, data files from individual data collection periods are
harmonized into a common data format.

\begin{longtable}[]{@{}
  >{\raggedright\arraybackslash}p{(\columnwidth - 2\tabcolsep) * \real{0.8382}}
  >{\raggedright\arraybackslash}p{(\columnwidth - 2\tabcolsep) * \real{0.1618}}@{}}
\toprule()
\begin{minipage}[b]{\linewidth}\raggedright
Pre-Requisite
\end{minipage} & \begin{minipage}[b]{\linewidth}\raggedright
Followed By
\end{minipage} \\
\midrule()
\endhead
spring17\_clean\_data.Rmd spring18\_clean\_data.Rmd
fall21\_clean\_data.Rmd winter2022\_clean\_sgc3b.Rmd &
2\_sgc3B\_scoring.qmd \\
\bottomrule()
\end{longtable}

Data for study SGC\_3B were collected across four time periods,
interrupted by the Covid-19 pandemic.

\begin{longtable}[]{@{}ll@{}}
\toprule()
Period & Modality \\
\midrule()
\endhead
Fall 2017 & in person, SONA groups in computer lab \\
Spring 2018 & in person, SONA groups in computer lab \\
Fall 2021 & asynchronous, online, SONA \\
\bottomrule()
\end{longtable}

Data collected in Fall 2017, Spring 2018 constitute the original SGC\_3B
study, conducted in person. Data collected in Fall 2021 constitute the
web-based replication, conducted online (asynchronously). In all cases,
the experiment was administered via a web application.

The underlying data structure of the stimulus web application changed
across the data collection period, resulting in slightly different data
files (i.e.~columns are not named consistently). In this section, we
combine the files from each data collection period into a single
\emph{harmonized} data file for analysis (one for participants, one for
items).

\hypertarget{participants-2}{%
\subsubsection{Participants}\label{participants-2}}

First we import participant-level data from each data collection period,
selecting only the columns relevant for analysis, and renaming columns
to be consistent across each file. The result is a single data frame
\texttt{df\_subjects} containing one row for each subject (across all
periods). Note that we \emph{are not} discarding any \emph{response}
data. Rather, we discard columns that are automatically recorded by the
stimulus web application and help the application run.

\emph{Note that we discard some columns representing scores calculated
in the stimulus engine. These scores were calculated differently across
collection periods, and so we discard them and recalculate scores in the
next analysis notebook. No raw data (responses and response times) are
discarded, only algorithmically-derived scores for the responses.}

\begin{Shaded}
\begin{Highlighting}[]
\CommentTok{\#IMPORT PARTICIPANT DATA}

\CommentTok{\# }\AlertTok{HACK}\CommentTok{ WD FOR LOCAL RUNNING?}
\NormalTok{imac }\OtherTok{=} \StringTok{"/Users/amyraefox/Code/SGC{-}Scaffolding\_Graph\_Comprehension/SGC{-}X/ANALYSIS/MAIN"}
\CommentTok{\# \# mbp = "/Users/amyfox/Sites/RESEARCH/SGC—Scaffolding Graph Comprehension/SGC{-}X/ANALYSIS/MAIN"}
\FunctionTok{setwd}\NormalTok{(imac)}

\CommentTok{\#set datafiles}
\NormalTok{fall17 }\OtherTok{\textless{}{-}} \StringTok{"analysis/SGC3B/data/0{-}session{-}level/fall17\_sgc3b\_participants.csv"}
\NormalTok{spring18 }\OtherTok{\textless{}{-}} \StringTok{"analysis/SGC3B/data/0{-}session{-}level/spring18\_sgc3b\_participants.csv"}
\NormalTok{fall21 }\OtherTok{\textless{}{-}} \StringTok{"analysis/SGC3B/data/0{-}session{-}level/fall21\_sgc3b\_participants.csv"}
\NormalTok{meta }\OtherTok{\textless{}{-}} \StringTok{"analysis/SGC3A/data/0{-}session{-}level/winter22\_sgc3a\_participants.csv"} \CommentTok{\#FOR SCHEMA ONLY}

\CommentTok{\#read datafiles, set mode and term}
\NormalTok{df\_subjects\_fall17 }\OtherTok{\textless{}{-}} \FunctionTok{read\_csv}\NormalTok{(fall17) }\SpecialCharTok{\%\textgreater{}\%} \FunctionTok{mutate}\NormalTok{(}\AttributeTok{mode =} \StringTok{"lab{-}synch"}\NormalTok{, }\AttributeTok{term =} \StringTok{"fall17"}\NormalTok{)}
\NormalTok{df\_subjects\_spring18 }\OtherTok{\textless{}{-}} \FunctionTok{read\_csv}\NormalTok{(spring18) }\SpecialCharTok{\%\textgreater{}\%} \FunctionTok{mutate}\NormalTok{(}\AttributeTok{mode =} \StringTok{"lab{-}synch"}\NormalTok{, }\AttributeTok{term =} \StringTok{"spring18"}\NormalTok{)}
\NormalTok{df\_subjects\_fall21 }\OtherTok{\textless{}{-}} \FunctionTok{read\_csv}\NormalTok{(fall21) }\SpecialCharTok{\%\textgreater{}\%} \FunctionTok{mutate}\NormalTok{(}\AttributeTok{mode =} \StringTok{"asynch"}\NormalTok{, }\AttributeTok{term =} \StringTok{"fall21"}\NormalTok{)}
\NormalTok{meta }\OtherTok{\textless{}{-}} \FunctionTok{read\_csv}\NormalTok{(meta)}

\CommentTok{\#SAVE METADATA FROM SGC3A, but no rows }
\NormalTok{df\_subjects }\OtherTok{\textless{}{-}}\NormalTok{ meta }\SpecialCharTok{\%\textgreater{}\%} \FunctionTok{filter}\NormalTok{(condition}\SpecialCharTok{==}\StringTok{\textquotesingle{}X\textquotesingle{}}\NormalTok{) }\SpecialCharTok{\%\textgreater{}\%} 
\NormalTok{  dplyr}\SpecialCharTok{::}\FunctionTok{select}\NormalTok{(}
\NormalTok{  subject,condition,term,mode,}
\NormalTok{  gender,age,language, schoolyear, country,}
\NormalTok{  effort,difficulty,confidence,enjoyment,other,}
\NormalTok{  totaltime\_m, }
  \CommentTok{\# absolute\_score, \#drop absolute score as this is re{-}scored [though should be the same]}
  \CommentTok{\#exploratory factors}
\NormalTok{  violations, browser, width, height}
\NormalTok{)}

\CommentTok{\#COMPARE COLS}
\CommentTok{\# janitor::compare\_df\_cols(df\_subjects\_fall17, df\_subjects\_spring18, df\_subjects\_fall21,meta)}

\CommentTok{\#reduce data collected using OLD webapp to useful columns}
\NormalTok{df\_subjects\_before }\OtherTok{\textless{}{-}} \FunctionTok{rbind}\NormalTok{(df\_subjects\_fall17, df\_subjects\_spring18, df\_subjects\_fall21) }\SpecialCharTok{\%\textgreater{}\%} 
  \CommentTok{\#rename and summarize some columns}
  \FunctionTok{mutate}\NormalTok{(}
    \AttributeTok{totaltime\_m =}\NormalTok{ totalTime }\SpecialCharTok{/} \DecValTok{1000} \SpecialCharTok{/} \DecValTok{60}\NormalTok{,  }
    \AttributeTok{absolute\_score =}\NormalTok{ triangular\_score,}
    \AttributeTok{language =}\NormalTok{ native\_language,}
    \AttributeTok{gender =}\NormalTok{ sex,}
    \AttributeTok{schoolyear =}\NormalTok{ year) }\SpecialCharTok{\%\textgreater{}\%} 
  \CommentTok{\#create placeholders for cols not collected until NEW webapp [for later rbind]}
  \FunctionTok{mutate}\NormalTok{(}
    \AttributeTok{effort =} \StringTok{"NULL"}\NormalTok{,}
    \AttributeTok{difficulty =} \StringTok{"NULL"}\NormalTok{,}
    \AttributeTok{confidence =} \StringTok{"NULL"}\NormalTok{,}
    \AttributeTok{enjoyment =} \StringTok{"NULL"}\NormalTok{,}
    \AttributeTok{other =} \StringTok{"NULL"}\NormalTok{,}
    \AttributeTok{disability =} \StringTok{"NULL"}\NormalTok{,}
    \AttributeTok{violations =} \StringTok{"NULL"}\NormalTok{,}
    \AttributeTok{browser =} \StringTok{"NULL"}\NormalTok{,}
    \AttributeTok{width =} \StringTok{"NULL"}\NormalTok{,}
    \AttributeTok{height =} \StringTok{"NULL"}
\NormalTok{  ) }\SpecialCharTok{\%\textgreater{}\%} 
  \CommentTok{\#select only columns we\textquotesingle{}ll be analyzing, discard others}
\NormalTok{  dplyr}\SpecialCharTok{::}\FunctionTok{select}\NormalTok{(subject, condition, term, mode, }
                \CommentTok{\#demographics}
\NormalTok{                gender, age, language, schoolyear, country,}
                \CommentTok{\#placeholder effort survey}
\NormalTok{                effort, difficulty, confidence, enjoyment, }
                \CommentTok{\#placeholder misc }
\NormalTok{                other, disability,}
                \CommentTok{\#response characteristics}
\NormalTok{                totaltime\_m, }
                \CommentTok{\# absolute\_score, \#drop absolute score as this is re{-}scored [though should be the same]}
                \CommentTok{\#exploratory factors}
\NormalTok{                violations, browser, width, height)}

\NormalTok{effort\_labels }\OtherTok{\textless{}{-}} \FunctionTok{c}\NormalTok{(}\StringTok{"I tried my best on each question"}\NormalTok{, }\StringTok{"I tried my best on most questions"}\NormalTok{)}

\CommentTok{\#combine dataframes from old and new webapps}
\NormalTok{df\_subjects }\OtherTok{\textless{}{-}} \FunctionTok{rbind}\NormalTok{(df\_subjects, df\_subjects\_before) }\SpecialCharTok{\%\textgreater{}\%} 
  \CommentTok{\#refactor factors}
  \FunctionTok{mutate}\NormalTok{ (}
    \AttributeTok{subject =} \FunctionTok{factor}\NormalTok{(subject),}
    \AttributeTok{condition =} \FunctionTok{factor}\NormalTok{(condition),}
    \AttributeTok{pretty\_condition =} \FunctionTok{recode\_factor}\NormalTok{(condition, }\StringTok{"111"} \OtherTok{=} \StringTok{"control"}\NormalTok{, }\StringTok{"121"} \OtherTok{=}  \StringTok{"impasse"}\NormalTok{),}
    \AttributeTok{pretty\_mode =} \FunctionTok{recode\_factor}\NormalTok{(mode, }\StringTok{"lab{-}synch"} \OtherTok{=} \StringTok{"laboratory"}\NormalTok{, }\StringTok{"asynch"} \OtherTok{=}  \StringTok{"online{-}replication"}\NormalTok{),}
    \AttributeTok{term =} \FunctionTok{factor}\NormalTok{(term, }\AttributeTok{levels=} \FunctionTok{c}\NormalTok{(}\StringTok{"fall17"}\NormalTok{,}\StringTok{"spring18"}\NormalTok{,}\StringTok{"fall21"}\NormalTok{,}\StringTok{"winter22"}\NormalTok{)),}
    \AttributeTok{mode =} \FunctionTok{factor}\NormalTok{(mode, }\AttributeTok{levels=}\FunctionTok{c}\NormalTok{(}\StringTok{"lab{-}synch"}\NormalTok{,}\StringTok{"asynch"}\NormalTok{)),}
    \AttributeTok{gender =} \FunctionTok{factor}\NormalTok{(gender),}
    \AttributeTok{schoolyear =} \FunctionTok{factor}\NormalTok{(schoolyear, }\AttributeTok{levels=}\FunctionTok{c}\NormalTok{(}\StringTok{"First"}\NormalTok{,}\StringTok{"Second"}\NormalTok{,}\StringTok{"Third"}\NormalTok{,}\StringTok{"Fourth"}\NormalTok{,}\StringTok{"Fifth"}\NormalTok{,}\StringTok{"Other"}\NormalTok{))}
\NormalTok{  )}

\CommentTok{\#FIX METADATA}
\CommentTok{\#Add metadata for columns that lost it [factors, for some reason!]}
\FunctionTok{var\_label}\NormalTok{(df\_subjects}\SpecialCharTok{$}\NormalTok{subject) }\OtherTok{\textless{}{-}} \StringTok{"ID of subject (randomly assigned in stimulus app)."}
\FunctionTok{var\_label}\NormalTok{(df\_subjects}\SpecialCharTok{$}\NormalTok{condition) }\OtherTok{\textless{}{-}} \StringTok{"ID indicates randomly assigned condition (111 {-}\textgreater{} control, 121 {-}\textgreater{} impasse)."}
\FunctionTok{var\_label}\NormalTok{(df\_subjects}\SpecialCharTok{$}\NormalTok{term) }\OtherTok{\textless{}{-}} \StringTok{"indicates if session was run with experimenter present or asynchronously"}
\FunctionTok{var\_label}\NormalTok{(df\_subjects}\SpecialCharTok{$}\NormalTok{mode) }\OtherTok{\textless{}{-}} \StringTok{"indicates mode in which the participant completed the study"}
\FunctionTok{var\_label}\NormalTok{(df\_subjects}\SpecialCharTok{$}\NormalTok{gender) }\OtherTok{\textless{}{-}} \StringTok{"What is your gender identity?"}
\FunctionTok{var\_label}\NormalTok{(df\_subjects}\SpecialCharTok{$}\NormalTok{schoolyear) }\OtherTok{\textless{}{-}} \StringTok{"What is your year in school?"}

\CommentTok{\#CLEANUP}
\FunctionTok{rm}\NormalTok{(df\_subjects\_fall17,df\_subjects\_fall21, df\_subjects\_spring18, df\_subjects\_before)}
\FunctionTok{rm}\NormalTok{(fall17,fall21,spring18,meta)}
\end{Highlighting}
\end{Shaded}

\hypertarget{items-1}{%
\subsubsection{Items}\label{items-1}}

Next we import item-level data from each data collection period,
selecting only the columns relevant for analysis, and renaming columns
to be consistent across each file. The result is a single data frame
\texttt{df\_items} containing one row for each \emph{graph comprehension
task question} (qs=15) (across all periods). A second data frame
\texttt{df\_freeresponse} contains one row for each free response
strategy question (last question posed to participants in Winter2022)
Note that we \emph{do not} discard any \emph{response} data. Rather, we
\emph{do} discard several columns representing accuracy scores for
responses that were calculated in the stimulus engine. These scores were
calculated differently across collection periods, and so we discard them
and recalculate scores in the next analysis notebook. Original response
data are always preserved.

\begin{Shaded}
\begin{Highlighting}[]
\CommentTok{\# }\AlertTok{HACK}\CommentTok{ WD FOR LOCAL RUNNING?}
\NormalTok{imac }\OtherTok{=} \StringTok{"/Users/amyraefox/Code/SGC{-}Scaffolding\_Graph\_Comprehension/SGC{-}X/ANALYSIS/MAIN"}
\CommentTok{\# \#mbp = "/Users/amyfox/Sites/RESEARCH/SGC—Scaffolding Graph Comprehension/SGC{-}X/ANALYSIS/MAIN"}
\FunctionTok{setwd}\NormalTok{(imac)}

\CommentTok{\#set datafiles}
\NormalTok{fall17 }\OtherTok{\textless{}{-}} \StringTok{"analysis/SGC3B/data/0{-}session{-}level/fall17\_sgc3b\_blocks.csv"}
\NormalTok{spring18 }\OtherTok{\textless{}{-}} \StringTok{"analysis/SGC3B/data/0{-}session{-}level/spring18\_sgc3b\_blocks.csv"}
\NormalTok{fall21 }\OtherTok{\textless{}{-}} \StringTok{"analysis/SGC3B/data/0{-}session{-}level/fall21\_sgc3b\_blocks.csv"}
\NormalTok{meta }\OtherTok{\textless{}{-}} \StringTok{"analysis/SGC3A/data/0{-}session{-}level/winter22\_sgc3a\_items.rds"} \CommentTok{\#FOR SCHEMA ONLY}

\CommentTok{\#read datafiles, set mode and term}
\NormalTok{df\_items\_fall17 }\OtherTok{\textless{}{-}} \FunctionTok{read\_csv}\NormalTok{(fall17) }\SpecialCharTok{\%\textgreater{}\%} \FunctionTok{mutate}\NormalTok{(}\AttributeTok{mode =} \StringTok{"lab{-}synch"}\NormalTok{, }\AttributeTok{term =} \StringTok{"fall17"}\NormalTok{)}
\NormalTok{df\_items\_spring18 }\OtherTok{\textless{}{-}} \FunctionTok{read\_csv}\NormalTok{(spring18) }\SpecialCharTok{\%\textgreater{}\%} \FunctionTok{mutate}\NormalTok{(}\AttributeTok{mode =} \StringTok{"lab{-}synch"}\NormalTok{, }\AttributeTok{term =} \StringTok{"spring18"}\NormalTok{)}
\NormalTok{df\_items\_fall21 }\OtherTok{\textless{}{-}} \FunctionTok{read\_csv}\NormalTok{(fall21) }\SpecialCharTok{\%\textgreater{}\%} \FunctionTok{mutate}\NormalTok{(}\AttributeTok{mode =} \StringTok{"asynch"}\NormalTok{, }\AttributeTok{term =} \StringTok{"fall21"}\NormalTok{)}
\NormalTok{meta }\OtherTok{\textless{}{-}} \FunctionTok{read\_rds}\NormalTok{(meta) }\CommentTok{\#use RDS file as it contains metadata}

\CommentTok{\#get mapping being question \# and interval relation the question tests, that is encoded only in the winter22 data files}
\NormalTok{map\_relations }\OtherTok{\textless{}{-}}\NormalTok{ meta }\SpecialCharTok{\%\textgreater{}\%} \FunctionTok{group\_by}\NormalTok{(q) }\SpecialCharTok{\%\textgreater{}\%}\NormalTok{ dplyr}\SpecialCharTok{::}\FunctionTok{select}\NormalTok{(q,relation) }\SpecialCharTok{\%\textgreater{}\%} \FunctionTok{unique}\NormalTok{()}

\CommentTok{\#SAVE METADATA FROM WINTER, but no rows }
\NormalTok{df\_items }\OtherTok{\textless{}{-}}\NormalTok{ meta }\SpecialCharTok{\%\textgreater{}\%} \FunctionTok{filter}\NormalTok{(condition}\SpecialCharTok{==}\StringTok{\textquotesingle{}X\textquotesingle{}}\NormalTok{) }\SpecialCharTok{\%\textgreater{}\%}\NormalTok{ dplyr}\SpecialCharTok{::}\FunctionTok{select}\NormalTok{(}
\NormalTok{  subject,condition,term,mode,}
\NormalTok{  question, q, answer, correct, rt\_s}
\NormalTok{) }
  
\CommentTok{\#reduce data collected using old webapp}
\NormalTok{df\_items\_before }\OtherTok{\textless{}{-}} \FunctionTok{rbind}\NormalTok{(df\_items\_fall17, df\_items\_spring18, df\_items\_fall21) }\SpecialCharTok{\%\textgreater{}\%} 
  \FunctionTok{mutate}\NormalTok{(}\AttributeTok{rt\_s =}\NormalTok{ rt }\SpecialCharTok{/} \DecValTok{1000}\NormalTok{, }\AttributeTok{correct =} \FunctionTok{as.logical}\NormalTok{(correct)) }\SpecialCharTok{\%\textgreater{}\%} 
\NormalTok{  dplyr}\SpecialCharTok{::}\FunctionTok{select}\NormalTok{(subject, condition, term, mode, question, q, answer, correct, rt\_s) }

\CommentTok{\#COMPARE COLS}
\NormalTok{janitor}\SpecialCharTok{::}\FunctionTok{compare\_df\_cols}\NormalTok{(df\_items\_before, df\_items)}
\end{Highlighting}
\end{Shaded}

\begin{verbatim}
  column_name df_items_before  df_items
1      answer       character character
2   condition         numeric    factor
3     correct         logical    factor
4        mode       character    factor
5           q         numeric    factor
6    question       character character
7        rt_s         numeric   numeric
8     subject       character    factor
9        term       character    factor
\end{verbatim}

\begin{Shaded}
\begin{Highlighting}[]
\CommentTok{\#combine dataframes from old and new webapps}
\NormalTok{df\_items }\OtherTok{\textless{}{-}} \FunctionTok{rbind}\NormalTok{(df\_items, df\_items\_before) }\SpecialCharTok{\%\textgreater{}\%} 
  \CommentTok{\#refactorize columns}
  \FunctionTok{mutate}\NormalTok{(}
    \AttributeTok{subject =} \FunctionTok{factor}\NormalTok{(subject),}
    \AttributeTok{condition =} \FunctionTok{factor}\NormalTok{(condition),}
    \AttributeTok{term =} \FunctionTok{factor}\NormalTok{(term, }\AttributeTok{levels=} \FunctionTok{c}\NormalTok{(}\StringTok{"fall17"}\NormalTok{,}\StringTok{"spring18"}\NormalTok{,}\StringTok{"fall21"}\NormalTok{,}\StringTok{"winter22"}\NormalTok{)),}
    \AttributeTok{mode =} \FunctionTok{factor}\NormalTok{(mode, }\AttributeTok{levels=}\FunctionTok{c}\NormalTok{(}\StringTok{"lab{-}synch"}\NormalTok{,}\StringTok{"asynch"}\NormalTok{)),}
    \AttributeTok{q =} \FunctionTok{as.integer}\NormalTok{(q)) }\SpecialCharTok{\%\textgreater{}\%} 
  \CommentTok{\#rename answer column to RESPONSE }
  \FunctionTok{rename}\NormalTok{(}\AttributeTok{response =}\NormalTok{ answer) }\SpecialCharTok{\%\textgreater{}\%} 
  \CommentTok{\#remove all commas and make as character string}
  \FunctionTok{mutate}\NormalTok{(}
    \AttributeTok{response =} \FunctionTok{str\_remove\_all}\NormalTok{(}\FunctionTok{as.character}\NormalTok{(response), }\StringTok{","}\NormalTok{),}
    \AttributeTok{num\_o =} \FunctionTok{str\_length}\NormalTok{(response)}
\NormalTok{  ) }\SpecialCharTok{\%\textgreater{}\%} 
  \CommentTok{\# handle NA values (why are some empty responses blank and others NA?) }
  \FunctionTok{mutate}\NormalTok{(}
    \AttributeTok{response =} \FunctionTok{replace\_na}\NormalTok{(response, }\StringTok{""}\NormalTok{),}
    \AttributeTok{num\_o =} \FunctionTok{replace\_na}\NormalTok{(num\_o, }\DecValTok{0}\NormalTok{)}
\NormalTok{  )}


\CommentTok{\#FIX METADATA}
\CommentTok{\#Add metadata for columns that lost it [factors, for some reason!]}
\FunctionTok{var\_label}\NormalTok{(df\_items}\SpecialCharTok{$}\NormalTok{subject) }\OtherTok{\textless{}{-}} \StringTok{"ID of subject (randomly assigned in stimulus app)."}
\FunctionTok{var\_label}\NormalTok{(df\_items}\SpecialCharTok{$}\NormalTok{condition) }\OtherTok{\textless{}{-}} \StringTok{"ID indicates randomly assigned condition (111 {-}\textgreater{} control, 121 {-}\textgreater{} impasse)."}
\FunctionTok{var\_label}\NormalTok{(df\_items}\SpecialCharTok{$}\NormalTok{term) }\OtherTok{\textless{}{-}} \StringTok{"indicates if session was run with experimenter present or asynchronously"}
\FunctionTok{var\_label}\NormalTok{(df\_items}\SpecialCharTok{$}\NormalTok{mode) }\OtherTok{\textless{}{-}} \StringTok{"indicates mode in which the participant completed the study"}
\FunctionTok{var\_label}\NormalTok{(df\_items}\SpecialCharTok{$}\NormalTok{q) }\OtherTok{\textless{}{-}} \StringTok{"Question Number (in order)"}
\FunctionTok{var\_label}\NormalTok{(df\_items}\SpecialCharTok{$}\NormalTok{correct) }\OtherTok{\textless{}{-}} \StringTok{"Is the response (strictly) correct? [dichotomous scoring]"}
\FunctionTok{var\_label}\NormalTok{(df\_items}\SpecialCharTok{$}\NormalTok{response) }\OtherTok{\textless{}{-}} \StringTok{"options (datapoints) selected by the subject"}
\FunctionTok{var\_label}\NormalTok{(df\_items}\SpecialCharTok{$}\NormalTok{num\_o) }\OtherTok{\textless{}{-}} \StringTok{"number of options selected by the subject"}

\CommentTok{\#HANDLE FREE RESPONSE QUESTION \#16 }
\CommentTok{\#save \textasciigrave{}free response\textasciigrave{} Q\#16 in its own dataframe}
\NormalTok{df\_freeresponse }\OtherTok{\textless{}{-}}\NormalTok{ df\_items }\SpecialCharTok{\%\textgreater{}\%} \FunctionTok{filter}\NormalTok{(q }\SpecialCharTok{==} \DecValTok{16}\NormalTok{) }\SpecialCharTok{\%\textgreater{}\%}\NormalTok{ dplyr}\SpecialCharTok{::}\FunctionTok{select}\NormalTok{(}\SpecialCharTok{{-}}\NormalTok{question,}\SpecialCharTok{{-}}\NormalTok{correct,}\SpecialCharTok{{-}}\NormalTok{rt\_s,}\SpecialCharTok{{-}}\NormalTok{num\_o)}
\CommentTok{\#add question description}
\NormalTok{df\_freeresponse }\OtherTok{\textless{}{-}}\NormalTok{ df\_freeresponse }\SpecialCharTok{\%\textgreater{}\%} \FunctionTok{mutate}\NormalTok{(}
  \AttributeTok{question =} \StringTok{"Please describe how to determine what event(s) start at 12pm?"}\NormalTok{,}
  \AttributeTok{response =} \FunctionTok{as.character}\NormalTok{(response) }\CommentTok{\#doesn\textquotesingle{}t need to be factor}
\NormalTok{)}
\CommentTok{\#remove \textquotesingle{}free response\textquotesingle{} Q\#16 from df\_items}
\NormalTok{df\_items }\OtherTok{\textless{}{-}}\NormalTok{ df\_items }\SpecialCharTok{\%\textgreater{}\%} \FunctionTok{filter}\NormalTok{ (q }\SpecialCharTok{!=} \DecValTok{16}\NormalTok{)}

\CommentTok{\#add back pretty condition }
\NormalTok{df\_items }\OtherTok{\textless{}{-}}\NormalTok{ df\_items }\SpecialCharTok{\%\textgreater{}\%} \FunctionTok{mutate}\NormalTok{(}
  \AttributeTok{pretty\_condition =} \FunctionTok{recode\_factor}\NormalTok{(condition, }
                                   \StringTok{"111"} \OtherTok{=} \StringTok{"none{-}none"}\NormalTok{, }\StringTok{"121"} \OtherTok{=}  \StringTok{"none{-}impasse"}\NormalTok{, }
                                   \StringTok{"211"} \OtherTok{=} \StringTok{"img{-}none"}\NormalTok{, }\StringTok{"221"} \OtherTok{=}  \StringTok{"img{-}impasse"}\NormalTok{, }
                                   \StringTok{"311"} \OtherTok{=} \StringTok{"ixv{-}none"}\NormalTok{, }\StringTok{"321"} \OtherTok{=}  \StringTok{"ixv{-}impasse"}\NormalTok{, }
\NormalTok{                                   ),}
  \AttributeTok{pretty\_mode =} \FunctionTok{recode\_factor}\NormalTok{(mode, }\StringTok{"lab{-}synch"} \OtherTok{=} \StringTok{"laboratory"}\NormalTok{, }\StringTok{"asynch"} \OtherTok{=}  \StringTok{"online{-}replication"}\NormalTok{)}
\NormalTok{) }

\CommentTok{\#CLEANUP}
\FunctionTok{rm}\NormalTok{(df\_items\_fall17,df\_items\_fall21, df\_items\_spring18, df\_items\_before)}
\FunctionTok{rm}\NormalTok{(fall17,fall21,spring18,meta, map\_relations)}
\end{Highlighting}
\end{Shaded}

\hypertarget{validation-1}{%
\subsubsection{Validation}\label{validation-1}}

Next, we validate that we have the complete number of item-level records
based on the number of subject-level records

\begin{Shaded}
\begin{Highlighting}[]
\CommentTok{\#the number of items should be equal to 15 x the number of subjects}
\FunctionTok{nrow}\NormalTok{(df\_items) }\SpecialCharTok{==} \DecValTok{15}\SpecialCharTok{*} \FunctionTok{nrow}\NormalTok{(df\_subjects) }\CommentTok{\#TRUE}
\end{Highlighting}
\end{Shaded}

\begin{verbatim}
[1] TRUE
\end{verbatim}

\begin{Shaded}
\begin{Highlighting}[]
\CommentTok{\#each subject should have 15 items}
\NormalTok{df\_items }\SpecialCharTok{\%\textgreater{}\%} \FunctionTok{group\_by}\NormalTok{(subject) }\SpecialCharTok{\%\textgreater{}\%} \FunctionTok{summarise}\NormalTok{(}\AttributeTok{n =} \FunctionTok{n}\NormalTok{()) }\SpecialCharTok{\%\textgreater{}\%} \FunctionTok{filter}\NormalTok{(n }\SpecialCharTok{!=} \DecValTok{15}\NormalTok{) }\SpecialCharTok{\%\textgreater{}\%} \FunctionTok{nrow}\NormalTok{() }\SpecialCharTok{==} \DecValTok{0}
\end{Highlighting}
\end{Shaded}

\begin{verbatim}
[1] TRUE
\end{verbatim}

\hypertarget{export-2}{%
\subsubsection{Export}\label{export-2}}

Finally, we export the (session-harmonized) data for analysis, as CSVs,
and .RDS (includes metadata)

\begin{Shaded}
\begin{Highlighting}[]
\CommentTok{\# }\AlertTok{HACK}\CommentTok{ WD FOR LOCAL RUNNING?}
\NormalTok{imac }\OtherTok{=} \StringTok{"/Users/amyraefox/Code/SGC{-}Scaffolding\_Graph\_Comprehension/SGC{-}X/ANALYSIS/MAIN"}
\CommentTok{\# \#mbp = "/Users/amyfox/Sites/RESEARCH/SGC—Scaffolding Graph Comprehension/SGC{-}X/ANALYSIS/MAIN"}
\FunctionTok{setwd}\NormalTok{(imac)}

\CommentTok{\#SAVE FILES}
\FunctionTok{write.csv}\NormalTok{(df\_subjects,}\StringTok{"analysis/SGC3B/data/1{-}study{-}level/sgc3b\_participants.csv"}\NormalTok{, }\AttributeTok{row.names =} \ConstantTok{FALSE}\NormalTok{)}
\FunctionTok{write.csv}\NormalTok{(df\_items,}\StringTok{"analysis/SGC3B/data/1{-}study{-}level/sgc3b\_items.csv"}\NormalTok{, }\AttributeTok{row.names =} \ConstantTok{FALSE}\NormalTok{)}
\FunctionTok{write.csv}\NormalTok{(df\_freeresponse,}\StringTok{"analysis/SGC3B/data/1{-}study{-}level/sgc3b\_freeresponse.csv"}\NormalTok{, }\AttributeTok{row.names =} \ConstantTok{FALSE}\NormalTok{)}

\CommentTok{\#SAVE R Data Structures }
\CommentTok{\#export R DATA STRUCTURES (include codebook metadata)}
\NormalTok{rio}\SpecialCharTok{::}\FunctionTok{export}\NormalTok{(df\_subjects, }\StringTok{"analysis/SGC3B/data/1{-}study{-}level/sgc3b\_participants.rds"}\NormalTok{) }\CommentTok{\# to R data structure file}
\NormalTok{rio}\SpecialCharTok{::}\FunctionTok{export}\NormalTok{(df\_items, }\StringTok{"analysis/SGC3B/data/1{-}study{-}level/sgc3b\_items.rds"}\NormalTok{) }\CommentTok{\# to R data structure file}
\end{Highlighting}
\end{Shaded}

\hypertarget{resources-5}{%
\section{RESOURCES}\label{resources-5}}

\begin{Shaded}
\begin{Highlighting}[]
\FunctionTok{sessionInfo}\NormalTok{()}
\end{Highlighting}
\end{Shaded}

\begin{verbatim}
R version 4.2.1 (2022-06-23)
Platform: x86_64-apple-darwin17.0 (64-bit)
Running under: macOS Big Sur ... 10.16

Matrix products: default
BLAS:   /Library/Frameworks/R.framework/Versions/4.2/Resources/lib/libRblas.0.dylib
LAPACK: /Library/Frameworks/R.framework/Versions/4.2/Resources/lib/libRlapack.dylib

locale:
[1] en_US.UTF-8/en_US.UTF-8/en_US.UTF-8/C/en_US.UTF-8/en_US.UTF-8

attached base packages:
[1] stats     graphics  grDevices utils     datasets  methods   base     

other attached packages:
 [1] kableExtra_1.3.4 forcats_0.5.1    stringr_1.4.0    dplyr_1.0.9     
 [5] purrr_0.3.4      readr_2.1.2      tidyr_1.2.0      tibble_3.1.7    
 [9] ggplot2_3.3.6    tidyverse_1.3.1  codebook_0.9.2  

loaded via a namespace (and not attached):
 [1] Rcpp_1.0.8.3      svglite_2.1.0     lubridate_1.8.0   assertthat_0.2.1 
 [5] digest_0.6.29     utf8_1.2.2        R6_2.5.1          cellranger_1.1.0 
 [9] backports_1.4.1   reprex_2.0.1      labelled_2.9.1    evaluate_0.15    
[13] httr_1.4.3        pillar_1.7.0      rlang_1.0.3       curl_4.3.2       
[17] readxl_1.4.0      data.table_1.14.2 rstudioapi_0.13   rmarkdown_2.14   
[21] webshot_0.5.3     foreign_0.8-82    bit_4.0.4         munsell_0.5.0    
[25] broom_0.8.0       janitor_2.1.0     compiler_4.2.1    modelr_0.1.8     
[29] xfun_0.31         pkgconfig_2.0.3   systemfonts_1.0.4 htmltools_0.5.2  
[33] tidyselect_1.1.2  rio_0.5.29        fansi_1.0.3       viridisLite_0.4.0
[37] crayon_1.5.1      tzdb_0.3.0        dbplyr_2.2.1      withr_2.5.0      
[41] grid_4.2.1        jsonlite_1.8.0    gtable_0.3.0      lifecycle_1.0.1  
[45] DBI_1.1.3         magrittr_2.0.3    scales_1.2.0      zip_2.2.0        
[49] cli_3.3.0         stringi_1.7.6     vroom_1.5.7       fs_1.5.2         
[53] snakecase_0.11.0  xml2_1.3.3        ellipsis_0.3.2    generics_0.1.2   
[57] vctrs_0.4.1       openxlsx_4.2.5    tools_4.2.1       bit64_4.0.5      
[61] glue_1.6.2        hms_1.1.1         parallel_4.2.1    fastmap_1.1.0    
[65] yaml_2.3.5        colorspace_2.0-3  rvest_1.0.2       knitr_1.39       
[69] haven_2.5.0      
\end{verbatim}

\newpage

\hypertarget{sec-SGC3B-scoring}{%
\chapter{Response Scoring}\label{sec-SGC3B-scoring}}

\emph{The purpose of this notebook is to score (assign a measure of
accuracy) to response data for the SGC3B study. This is required because
the question type on the graph comprehension task used a `Multiple
Response' (MR) question design. Here, we evaluate different approaches
to scoring multiple response questions, and transform raw item responses
(e.g.~boxes ABC are checked) to a measure of response accuracy.
(Warning: this notebook takes several minutes to execute.)} To review
the strategy behind Multiple Response scoring for the SGC project, refer
to section \textbf{?@sec-scoring}.

\begin{Shaded}
\begin{Highlighting}[]
\FunctionTok{options}\NormalTok{(}\AttributeTok{scipen=}\DecValTok{1}\NormalTok{, }\AttributeTok{digits=}\DecValTok{3}\NormalTok{)}

\FunctionTok{library}\NormalTok{(kableExtra) }\CommentTok{\#printing tables }
\FunctionTok{library}\NormalTok{(ggformula) }\CommentTok{\#quick graphs}
\FunctionTok{library}\NormalTok{(pbapply) }\CommentTok{\#progress bar and time estimate for *apply fns}
\FunctionTok{library}\NormalTok{(Hmisc) }\CommentTok{\# \%nin\% operator}
\FunctionTok{library}\NormalTok{(tidyverse) }\CommentTok{\#ALL THE THINGS}
\end{Highlighting}
\end{Shaded}

\hypertarget{score-sgc-data-1}{%
\section{SCORE SGC DATA}\label{score-sgc-data-1}}

To review the strategy behind Multiple Response scoring for the SGC
project, refer to section \textbf{?@sec-scoring}.

In SGC we are fundamentally interested in understanding how a
participant interprets the presented graph (stimulus). The \textbf{graph
comprehension task} asks them to select the data points in the graph
that meet the criteria posed in the question. To assess a participant's
performance, for each question (q=15) we will calculate the following
scores:

\emph{An overall, strict score:}\\
1. \textbf{Absolute Score} : using dichotomous scoring referencing true
(Triangular) answer. (see 1.2)

\emph{Sub-scores, for each alternative graph interpretation}\\
2. \textbf{Triangular Score} : using partial scoring {[}-1/q, +1/p{]}
referencing true (Triangular) answer key.

3. \textbf{Orthogonal Score} : using partial scoring {[}-1/q, +1/p{]}
referencing (incorrect Orthogonal) answer key.

Based on prior observational studies where we observed emergence of
other alternative interpretations (i.e.~transitional interpretations) we
also calculate subscores for these alternatives.

4. \textbf{Tversky Score} : using partial scoring {[}-1/q, +1/p{]}
referencing (incorrect connecting-lines strategy) answer key. 5.
\textbf{Satisficing Score} : using partial scoring {[}-1/q, +1/p{]}
referencing (incorrect satisficing strategy) answer key.

\hypertarget{sec-SGC3B-keys}{%
\subsection{Prepare Answer Keys}\label{sec-SGC3B-keys}}

We start by importing three answer keys: (1) Q1 - Q5 {[}control
condition{]}, (2) Q1-Q5 {[}impasse condition{]}, (3) Q6-15. Separate
answer keys by condition are required for Q1-Q5 because the stimuli for
each condition visualize a different underlying dataset (i.e.~the graphs
show datapoints in different positions). Q6-Q15 are identical across
conditions. Each answer key includes a row for each question, and a
column defining the subset of response options that correspond to
different graph interpretations.

\begin{Shaded}
\begin{Highlighting}[]
\CommentTok{\# \#HACK WD FOR LOCAL RUNNING?}
\NormalTok{imac }\OtherTok{=} \StringTok{"/Users/amyraefox/Code/SGC{-}Scaffolding\_Graph\_Comprehension/SGC{-}X/ANALYSIS/MAIN"}
\FunctionTok{setwd}\NormalTok{(imac)}

\CommentTok{\#SAVE KEYS FOR FUTURE USE}
\NormalTok{keys\_raw }\OtherTok{\textless{}{-}}  \FunctionTok{read\_csv}\NormalTok{(}\StringTok{"analysis/utils/keys/parsed\_keys/keys\_raw"}\NormalTok{)}
\NormalTok{keys\_orth }\OtherTok{\textless{}{-}}  \FunctionTok{read\_csv}\NormalTok{(}\StringTok{"analysis/utils/keys/parsed\_keys/keys\_orth"}\NormalTok{)}
\NormalTok{keys\_tri }\OtherTok{\textless{}{-}}  \FunctionTok{read\_csv}\NormalTok{(}\StringTok{"analysis/utils/keys/parsed\_keys/keys\_tri"}\NormalTok{)}
\NormalTok{keys\_satisfice\_left }\OtherTok{\textless{}{-}}  \FunctionTok{read\_csv}\NormalTok{(}\StringTok{"analysis/utils/keys/parsed\_keys/keys\_satisfice\_left"}\NormalTok{)}
\NormalTok{keys\_satisfice\_right }\OtherTok{\textless{}{-}}  \FunctionTok{read\_csv}\NormalTok{(}\StringTok{"analysis/utils/keys/parsed\_keys/keys\_satisfice\_right"}\NormalTok{)}
\NormalTok{keys\_tversky\_duration }\OtherTok{\textless{}{-}}  \FunctionTok{read\_csv}\NormalTok{(}\StringTok{"analysis/utils/keys/parsed\_keys/keys\_tversky\_duration"}\NormalTok{)}
\NormalTok{keys\_tversky\_end }\OtherTok{\textless{}{-}}  \FunctionTok{read\_csv}\NormalTok{(}\StringTok{"analysis/utils/keys/parsed\_keys/keys\_tversky\_end"}\NormalTok{)}
\NormalTok{keys\_tversky\_max }\OtherTok{\textless{}{-}}  \FunctionTok{read\_csv}\NormalTok{(}\StringTok{"analysis/utils/keys/parsed\_keys/keys\_tversky\_max"}\NormalTok{)}
\NormalTok{keys\_tversky\_start }\OtherTok{\textless{}{-}}  \FunctionTok{read\_csv}\NormalTok{(}\StringTok{"analysis/utils/keys/parsed\_keys/keys\_tversky\_start"}\NormalTok{)}
\end{Highlighting}
\end{Shaded}

\hypertarget{sec-SGC3B-subscores}{%
\subsection{Calculate Subscores}\label{sec-SGC3B-subscores}}

Next, we import the item-level response data. For each row in the item
level dataset (indicating the response to a single question-item for a
single subject), we compare the raw response
\texttt{df\_items\$response} with the answer keys in each interpretation
(e.g.~\texttt{keys\_orth}, \texttt{keys\_tri}, etc\ldots), then using
those sets, determine the number of correctly selected items(p) and
incorrectly selected items (q), which in turn are used to calculate
partial{[}-1/q, +1/p{]} scores for each interpretation. The resulting
scores are then stored on each item in \texttt{df\_items}, and can be
used to determine which graph interpretation the subject held.

Specifically, the following scores are calculated for each item:

\textbf{Interpretation Subscores}

\begin{itemize}
\tightlist
\item
  \texttt{score\_TRI} How consistent is the response with the
  \textbf{triangular}interpretation?
\item
  \texttt{score\_ORTH} How consistent is the response with the
  \textbf{orthogonal}interpretation?
\item
  \texttt{score\_SATISFICE} is calculated by taking the maximum value of
  :

  \begin{itemize}
  \tightlist
  \item
    \texttt{score\_SAT\_left} How consistent is the response with the
    \textbf{(left side) Satisficing} interpretation?
  \item
    \texttt{score\_SAT\_right} How consistent is the response with the
    \textbf{(right side) Satisficing} interpretation
  \end{itemize}
\item
  \texttt{score\_TVERSKY} is calculated by taking the maximum value of:

  \begin{itemize}
  \tightlist
  \item
    \texttt{score\_TV\_max} How consistent is the response with the
    \textbf{(maximal) Tversky} interpretation?
  \item
    \texttt{score\_TV\_start} How consistent is the response with the
    \textbf{(start-time) Tversky} interpretation?
  \item
    \texttt{score\_TV\_end} How consistent is the response with the
    \textbf{(end-time) Tversky} interpretation?
  \item
    \texttt{score\_TV\_duration} How consistent is the response with the
    \textbf{(duration) Tversky} interpretation?
  \end{itemize}
\item
  \texttt{score\_REF} Did the response select only the \textbf{reference
  point}?
\item
  \texttt{score\_BOTH} How consistent is the response with \textbf{both}
  the orthogonal and triangular interpretations?
\end{itemize}

\textbf{Absolute Scores}

\begin{itemize}
\tightlist
\item
  \texttt{score\_ABS} Is the response strictly correct? (triangular
  interpretation)
\item
  \texttt{score\_niceABS} Is the response strictly correct? (triangular
  interpretation, not penalizing ref points). This is a more generous
  version of the Absolute score that does not penalize the participant
  if in addition to the correct answer \emph{in addition to} they also
  select the data point referenced in the question.
\end{itemize}

\begin{Shaded}
\begin{Highlighting}[]
\CommentTok{\#HACK WD FOR LOCAL RUNNING?}
\NormalTok{imac }\OtherTok{=} \StringTok{"/Users/amyraefox/Code/SGC{-}Scaffolding\_Graph\_Comprehension/SGC{-}X/ANALYSIS/MAIN"}
\FunctionTok{setwd}\NormalTok{(imac)}

\CommentTok{\# backup \textless{}{-} read\_rds(\textquotesingle{}analysis/SGC3B/data/1{-}study{-}level/sgc3b\_items.rds\textquotesingle{}) \#for troubleshooting only}
\NormalTok{df\_items }\OtherTok{\textless{}{-}} \FunctionTok{read\_rds}\NormalTok{(}\StringTok{\textquotesingle{}analysis/SGC3B/data/1{-}study{-}level/sgc3b\_items.rds\textquotesingle{}}\NormalTok{)}
\end{Highlighting}
\end{Shaded}

\begin{Shaded}
\begin{Highlighting}[]
\CommentTok{\# \#HACK WD FOR LOCAL RUNNING?}
\NormalTok{imac }\OtherTok{=} \StringTok{"/Users/amyraefox/Code/SGC{-}Scaffolding\_Graph\_Comprehension/SGC{-}X/ANALYSIS/MAIN"}
\FunctionTok{setwd}\NormalTok{(imac)}

\FunctionTok{source}\NormalTok{(}\StringTok{"analysis/utils/scoring.R"}\NormalTok{)}
\end{Highlighting}
\end{Shaded}

\emph{note: this cell takes approximately 30 minutes to run on the full
df\_items dataframe with 4950 records}

\begin{Shaded}
\begin{Highlighting}[]
\CommentTok{\#RUN THIS \textless{}OR\textgreater{} THE CALCULATE{-}SCORES{-}FORLOOP [not both]}

\CommentTok{\#ALPHEBETIZE RESPONSE}
\NormalTok{df\_items}\SpecialCharTok{$}\NormalTok{response }\OtherTok{=} \FunctionTok{pbmapply}\NormalTok{(reorder\_inplace, df\_items}\SpecialCharTok{$}\NormalTok{response)}

\CommentTok{\#STRATEGY PARTIAL{-}SUBSCORES}
\NormalTok{df\_items}\SpecialCharTok{$}\NormalTok{score\_TRI }\OtherTok{=} \FunctionTok{pbmapply}\NormalTok{(calc\_subscore, df\_items}\SpecialCharTok{$}\NormalTok{q, df\_items}\SpecialCharTok{$}\NormalTok{condition, df\_items}\SpecialCharTok{$}\NormalTok{response, }\AttributeTok{MoreArgs =} \FunctionTok{list}\NormalTok{(}\AttributeTok{keyframe =}\NormalTok{ keys\_tri))}
\NormalTok{df\_items}\SpecialCharTok{$}\NormalTok{score\_ORTH }\OtherTok{=} \FunctionTok{pbmapply}\NormalTok{(calc\_subscore, df\_items}\SpecialCharTok{$}\NormalTok{q, df\_items}\SpecialCharTok{$}\NormalTok{condition, df\_items}\SpecialCharTok{$}\NormalTok{response, }\AttributeTok{MoreArgs =} \FunctionTok{list}\NormalTok{(}\AttributeTok{keyframe =}\NormalTok{ keys\_orth))}
\NormalTok{df\_items}\SpecialCharTok{$}\NormalTok{score\_SAT\_left }\OtherTok{=} \FunctionTok{pbmapply}\NormalTok{(calc\_subscore, df\_items}\SpecialCharTok{$}\NormalTok{q, df\_items}\SpecialCharTok{$}\NormalTok{condition, df\_items}\SpecialCharTok{$}\NormalTok{response, }\AttributeTok{MoreArgs =} \FunctionTok{list}\NormalTok{(}\AttributeTok{keyframe =}\NormalTok{ keys\_satisfice\_left))}
\NormalTok{df\_items}\SpecialCharTok{$}\NormalTok{score\_SAT\_right }\OtherTok{=} \FunctionTok{pbmapply}\NormalTok{(calc\_subscore, df\_items}\SpecialCharTok{$}\NormalTok{q, df\_items}\SpecialCharTok{$}\NormalTok{condition, df\_items}\SpecialCharTok{$}\NormalTok{response, }\AttributeTok{MoreArgs =} \FunctionTok{list}\NormalTok{(}\AttributeTok{keyframe =}\NormalTok{ keys\_satisfice\_right))}
\NormalTok{df\_items}\SpecialCharTok{$}\NormalTok{score\_TV\_max }\OtherTok{=} \FunctionTok{pbmapply}\NormalTok{(calc\_subscore, df\_items}\SpecialCharTok{$}\NormalTok{q, df\_items}\SpecialCharTok{$}\NormalTok{condition, df\_items}\SpecialCharTok{$}\NormalTok{response, }\AttributeTok{MoreArgs =} \FunctionTok{list}\NormalTok{(}\AttributeTok{keyframe =}\NormalTok{ keys\_tversky\_max))}
\NormalTok{df\_items}\SpecialCharTok{$}\NormalTok{score\_TV\_start }\OtherTok{=} \FunctionTok{pbmapply}\NormalTok{(calc\_subscore, df\_items}\SpecialCharTok{$}\NormalTok{q, df\_items}\SpecialCharTok{$}\NormalTok{condition, df\_items}\SpecialCharTok{$}\NormalTok{response, }\AttributeTok{MoreArgs =} \FunctionTok{list}\NormalTok{(}\AttributeTok{keyframe =}\NormalTok{ keys\_tversky\_start))}
\NormalTok{df\_items}\SpecialCharTok{$}\NormalTok{score\_TV\_end }\OtherTok{=} \FunctionTok{pbmapply}\NormalTok{(calc\_subscore, df\_items}\SpecialCharTok{$}\NormalTok{q, df\_items}\SpecialCharTok{$}\NormalTok{condition, df\_items}\SpecialCharTok{$}\NormalTok{response, }\AttributeTok{MoreArgs =} \FunctionTok{list}\NormalTok{(}\AttributeTok{keyframe =}\NormalTok{ keys\_tversky\_end))}
\NormalTok{df\_items}\SpecialCharTok{$}\NormalTok{score\_TV\_duration }\OtherTok{=} \FunctionTok{pbmapply}\NormalTok{(calc\_subscore, df\_items}\SpecialCharTok{$}\NormalTok{q, df\_items}\SpecialCharTok{$}\NormalTok{condition, df\_items}\SpecialCharTok{$}\NormalTok{response, }\AttributeTok{MoreArgs =} \FunctionTok{list}\NormalTok{(}\AttributeTok{keyframe =}\NormalTok{ keys\_tversky\_duration))}

\CommentTok{\#SPECIAL ABSOLUTE SUBSCORES}
\NormalTok{df\_items}\SpecialCharTok{$}\NormalTok{score\_REF }\OtherTok{=} \FunctionTok{pbmapply}\NormalTok{(calc\_refscore, df\_items}\SpecialCharTok{$}\NormalTok{q, df\_items}\SpecialCharTok{$}\NormalTok{response)}
\NormalTok{df\_items}\SpecialCharTok{$}\NormalTok{score\_BOTH }\OtherTok{=} \FunctionTok{as.integer}\NormalTok{((df\_items}\SpecialCharTok{$}\NormalTok{score\_TRI }\SpecialCharTok{==} \DecValTok{1}\NormalTok{) }\SpecialCharTok{\&}\NormalTok{ (df\_items}\SpecialCharTok{$}\NormalTok{score\_ORTH }\SpecialCharTok{==}\DecValTok{1}\NormalTok{))}

\CommentTok{\#ABSOLUTE SCORES}
\NormalTok{df\_items}\SpecialCharTok{$}\NormalTok{score\_ABS }\OtherTok{=} \FunctionTok{as.integer}\NormalTok{(df\_items}\SpecialCharTok{$}\NormalTok{correct) }
\NormalTok{df\_items}\SpecialCharTok{$}\NormalTok{score\_niceABS  }\OtherTok{\textless{}{-}} \FunctionTok{as.integer}\NormalTok{((df\_items}\SpecialCharTok{$}\NormalTok{score\_TRI }\SpecialCharTok{==} \DecValTok{1}\NormalTok{)) }\CommentTok{\#tri doesn\textquotesingle{}t penalize ref or ve{-}area}
\end{Highlighting}
\end{Shaded}

\hypertarget{sec-SGC3B-interpretation}{%
\subsection{Derive Interpretation}\label{sec-SGC3B-interpretation}}

Finally, we use the interpretation subscores to classify the response as
a particular interpretation. This classification algorithm : (1) First
decides if the response matches one or more `special' situations (blank
response, reference point response, both ORTH and TRI) (2) If response
doesn't match a special situation, it compares the individual subscores,
and subscores and decides if they are \emph{discriminant} (i.e.~are the
scores different enough to make a prediction). A discriminant threshold
of 0.5pts (on a scale from -1 to +1 is used) (2) If the variance in
subscores surpasses the threshold, the interpretation is classified
based on the highest subscore (TRIANGULAR, ORTHOGONAL, TVERSKY,
SATISFICE) (3) If the variance does not surpass the threshold, the
interpretation is labelled as ``?'', indicating it cannot be classified,
and is of an unknown interpretation.

The final output is called \texttt{interpretation}.

\begin{Shaded}
\begin{Highlighting}[]
\CommentTok{\#stoopid extra copying for troubleshooting safety}
\NormalTok{temp }\OtherTok{\textless{}{-}}\NormalTok{ df\_items }
\NormalTok{temp }\OtherTok{\textless{}{-}} \FunctionTok{derive\_interpretation}\NormalTok{(temp)}
\NormalTok{df\_items }\OtherTok{\textless{}{-}}\NormalTok{ temp }
\end{Highlighting}
\end{Shaded}

\hypertarget{sec-SGC3B-scaledScore}{%
\subsection{Derive Scaled Score}\label{sec-SGC3B-scaledScore}}

The \texttt{interpretation} response variable gives us the finest grain
indication of the reader's understanding of the graph for a particular
question. However, it is a categorical variable, which poses a challenge
for analyzing cumulative performance at the subject level. To address
this challenge, we derive a \emph{scaled\_score} that converts each
possible interpretation to a numeric value on a scale from -1 to +1.
This scaling takes advantage of the observation that each interpretation
can be positioned along a spectrum of understanding from completely
incorrect (orthogonal) to completely correct (triangular). Alternative
interpretations lay somewhere between.

Specifically, we assign the following values to each interpretation:

\begin{itemize}
\tightlist
\item
  (-1) : ORTHOGONAL, SATISFICE (satisfice represents an attempt at an
  orthogonal answer when none is available)
\item
  (-0.5): ? (some alternative that cannot be identified; but meaningful
  that it is not orthogonal)
\item
  (0): REFERENCE POINT, BLANK (indicates the individual thinks there is
  no answer, recognizes that ORTHOGONAL cannot be correct, but does not
  conceive of triangular)
\item
  (+0.5) TVERSKY, BOTH TRI + ORTH (indicates that they ``see'' a
  triangular response, but lack certainty and also select the ORTHOGONAL
  response)
\item
  (+1) TRIANGULAR +1
\end{itemize}

\begin{Shaded}
\begin{Highlighting}[]
\NormalTok{df\_items}\SpecialCharTok{$}\NormalTok{score\_SCALED }\OtherTok{\textless{}{-}} \FunctionTok{calc\_scaled}\NormalTok{(df\_items}\SpecialCharTok{$}\NormalTok{interpretation)}
\end{Highlighting}
\end{Shaded}

\hypertarget{summarize-by-subject-1}{%
\section{SUMMARIZE BY SUBJECT}\label{summarize-by-subject-1}}

Next, we summarize the item level scores by subject in order to
calculate cummulative subscores to be stored on the subject record.

\begin{Shaded}
\begin{Highlighting}[]
\CommentTok{\# \#HACK WD FOR LOCAL RUNNING?}
\NormalTok{imac }\OtherTok{=} \StringTok{"/Users/amyraefox/Code/SGC{-}Scaffolding\_Graph\_Comprehension/SGC{-}X/ANALYSIS/MAIN"}
\FunctionTok{setwd}\NormalTok{(imac)}

\CommentTok{\#import subjects}
\NormalTok{df\_subjects }\OtherTok{\textless{}{-}} \FunctionTok{read\_rds}\NormalTok{(}\StringTok{\textquotesingle{}analysis/SGC3B/data/1{-}study{-}level/sgc3b\_participants.rds\textquotesingle{}}\NormalTok{) }\SpecialCharTok{\%\textgreater{}\%} \FunctionTok{mutate}\NormalTok{(}\AttributeTok{subject =} \FunctionTok{as.character}\NormalTok{(subject)) }\SpecialCharTok{\%\textgreater{}\%} \FunctionTok{arrange}\NormalTok{(subject)}

\CommentTok{\#make temporary copies for testing safety}
\NormalTok{s }\OtherTok{=}\NormalTok{ df\_subjects}
\NormalTok{i }\OtherTok{=}\NormalTok{ df\_items }

\CommentTok{\#summarize}
\NormalTok{test\_subs }\OtherTok{\textless{}{-}} \FunctionTok{summarise\_bySubject}\NormalTok{(s,i)}
\end{Highlighting}
\end{Shaded}

\begin{verbatim}
`summarise()` has grouped output by 'subject'. You can override using the
`.groups` argument.
\end{verbatim}

\begin{verbatim}
[1] TRUE
[1] TRUE
[1] TRUE
[1] TRUE
[1] TRUE
[1] TRUE
[1] TRUE
\end{verbatim}

\begin{Shaded}
\begin{Highlighting}[]
\NormalTok{df\_subjects }\OtherTok{\textless{}{-}}\NormalTok{ test\_subs}
\end{Highlighting}
\end{Shaded}

We also summarize absolute and scaled score progress at each question in
the task, to explore cumulative performance over the task.

\begin{Shaded}
\begin{Highlighting}[]
\CommentTok{\#GET ABSOLUTE PROGRESS }
\NormalTok{df\_absolute\_progress }\OtherTok{\textless{}{-}} \FunctionTok{progress\_Absolute}\NormalTok{(df\_items)}

\CommentTok{\#GET SCALED PROGRESS}
\NormalTok{df\_scaled\_progress }\OtherTok{\textless{}{-}} \FunctionTok{progress\_Scaled}\NormalTok{(df\_items)}
\end{Highlighting}
\end{Shaded}

\hypertarget{explore-distributions-1}{%
\section{EXPLORE DISTRIBUTIONS}\label{explore-distributions-1}}

\begin{Shaded}
\begin{Highlighting}[]
\FunctionTok{options}\NormalTok{(}\AttributeTok{repr.plot.width =}\DecValTok{9}\NormalTok{, }\AttributeTok{repr.plot.height =}\DecValTok{12}\NormalTok{)}

\CommentTok{\#create temp data frame for visualizations}
\NormalTok{df }\OtherTok{=}\NormalTok{ df\_items }\SpecialCharTok{\%\textgreater{}\%} \FunctionTok{filter}\NormalTok{ (q }\SpecialCharTok{\%nin\%} \FunctionTok{c}\NormalTok{(}\DecValTok{6}\NormalTok{,}\DecValTok{9}\NormalTok{)) }\SpecialCharTok{\%\textgreater{}\%} \FunctionTok{mutate}\NormalTok{(}
  \AttributeTok{score\_niceABS =} \FunctionTok{as.factor}\NormalTok{(score\_niceABS),}
  \AttributeTok{pretty\_interpretation =} \FunctionTok{factor}\NormalTok{(interpretation,}
    \AttributeTok{levels =} \FunctionTok{c}\NormalTok{(}\StringTok{"Orthogonal"}\NormalTok{, }\StringTok{"Satisfice"}\NormalTok{, }
               \StringTok{"frenzy"}\NormalTok{,}\StringTok{"?"}\NormalTok{,}
                \StringTok{"reference"}\NormalTok{,}\StringTok{"blank"}\NormalTok{,}
                \StringTok{"Tversky"}\NormalTok{, }\StringTok{"both tri + orth"}\NormalTok{,}
               \StringTok{"Triangular"}\NormalTok{ ))}
\NormalTok{  )}
\end{Highlighting}
\end{Shaded}

\hypertarget{absolute-score-2}{%
\subsection{Absolute Score}\label{absolute-score-2}}

\begin{Shaded}
\begin{Highlighting}[]
\CommentTok{\#DISTRIBUTION ABSOLUTE SCORE FULL }\AlertTok{TASK}
\FunctionTok{gf\_props}\NormalTok{(}\SpecialCharTok{\textasciitilde{}}\NormalTok{score\_niceABS, }\AttributeTok{fill =} \SpecialCharTok{\textasciitilde{}}\NormalTok{pretty\_condition, }\AttributeTok{position =} \FunctionTok{position\_dodge}\NormalTok{(), }\AttributeTok{data =}\NormalTok{ df) }\SpecialCharTok{+}
  \FunctionTok{labs}\NormalTok{( }\AttributeTok{x =} \StringTok{"Absolute Score"}\NormalTok{, }
        \AttributeTok{title =} \StringTok{"Distribution of Absolute Score (all Items)"}\NormalTok{,}
        \AttributeTok{subtitle =} \FunctionTok{paste}\NormalTok{(}\StringTok{"Impasse Condition (blue) yields more correct responses across the entire task"}\NormalTok{),}
        \AttributeTok{y =} \StringTok{"Proportion of Items"}\NormalTok{) }\SpecialCharTok{+}
  \FunctionTok{scale\_fill\_discrete}\NormalTok{(}\AttributeTok{name =} \StringTok{"Condition"}\NormalTok{) }\SpecialCharTok{+}  
  \FunctionTok{theme\_minimal}\NormalTok{()}
\end{Highlighting}
\end{Shaded}

\begin{figure}[H]

{\centering \includegraphics{analysis/SGC3B/2_sgc3B_scoring_files/figure-pdf/DISTR-ABSCORE-1.pdf}

}

\end{figure}

\begin{Shaded}
\begin{Highlighting}[]
\CommentTok{\#DISTRIBUTION ABSOLUTE SCORE BY ITEM}
\FunctionTok{gf\_props}\NormalTok{(}\SpecialCharTok{\textasciitilde{}}\NormalTok{score\_niceABS, }\AttributeTok{fill =} \SpecialCharTok{\textasciitilde{}}\NormalTok{pretty\_condition, }\AttributeTok{position =} \FunctionTok{position\_dodge}\NormalTok{(), }\AttributeTok{data =}\NormalTok{ df)  }\SpecialCharTok{\%\textgreater{}\%} 
  \FunctionTok{gf\_facet\_grid}\NormalTok{(pretty\_condition}\SpecialCharTok{\textasciitilde{}}\NormalTok{q) }\SpecialCharTok{+} 
  \FunctionTok{labs}\NormalTok{( }\AttributeTok{x =} \StringTok{"Absolute Score"}\NormalTok{, }
        \AttributeTok{title =} \StringTok{"Distribution of Absolute Score (by Item)"}\NormalTok{,}
        \AttributeTok{subtitle =} \StringTok{"Impasse Condition (blue) yields more correct responses on each item"}\NormalTok{,}
        \AttributeTok{y =} \StringTok{"Proprition of Subjects"}\NormalTok{) }\SpecialCharTok{+}
  \FunctionTok{scale\_fill\_discrete}\NormalTok{(}\AttributeTok{name =} \StringTok{"Condition"}\NormalTok{) }\SpecialCharTok{+}  
  \FunctionTok{theme\_minimal}\NormalTok{()}
\end{Highlighting}
\end{Shaded}

\begin{figure}[H]

{\centering \includegraphics{analysis/SGC3B/2_sgc3B_scoring_files/figure-pdf/DISTR-ABSCORE-2.pdf}

}

\end{figure}

\begin{Shaded}
\begin{Highlighting}[]
\CommentTok{\#DISTRIBUTION ABSOLUTE SCORE BY SUBJECT}
\FunctionTok{gf\_props}\NormalTok{(}\SpecialCharTok{\textasciitilde{}}\NormalTok{s\_NABS, }\AttributeTok{fill =} \SpecialCharTok{\textasciitilde{}}\NormalTok{pretty\_condition, }\AttributeTok{position =} \FunctionTok{position\_dodge}\NormalTok{(), }\AttributeTok{data =}\NormalTok{ df\_subjects) }\SpecialCharTok{\%\textgreater{}\%} 
\FunctionTok{gf\_facet\_grid}\NormalTok{(pretty\_condition }\SpecialCharTok{\textasciitilde{}}\NormalTok{. )}\SpecialCharTok{+}
  \FunctionTok{labs}\NormalTok{( }\AttributeTok{x =} \StringTok{"Total Absolute Score"}\NormalTok{, }
        \AttributeTok{title =} \StringTok{"Distribution of Total Absolute Score (by Subject)"}\NormalTok{,}
        \AttributeTok{subtitle =} \StringTok{"Impasse Condition (blue) yields higher total absolute scores"}\NormalTok{,}
        \AttributeTok{y =} \StringTok{"Proportion of Subjects"}\NormalTok{) }\SpecialCharTok{+}
  \FunctionTok{scale\_fill\_discrete}\NormalTok{(}\AttributeTok{name =} \StringTok{"Condition"}\NormalTok{) }\SpecialCharTok{+}  
  \FunctionTok{theme\_minimal}\NormalTok{() }\SpecialCharTok{+} \FunctionTok{theme}\NormalTok{(}\AttributeTok{legend.position =} \StringTok{"blank"}\NormalTok{)}
\end{Highlighting}
\end{Shaded}

\begin{figure}[H]

{\centering \includegraphics{analysis/SGC3B/2_sgc3B_scoring_files/figure-pdf/DISTR-ABSCORE-3.pdf}

}

\end{figure}

\begin{Shaded}
\begin{Highlighting}[]
\CommentTok{\#DISTRIBUTION ABSOLUTE SCORE }\AlertTok{TEST}\CommentTok{ PHASE}
\FunctionTok{gf\_histogram}\NormalTok{(}\SpecialCharTok{\textasciitilde{}}\NormalTok{item\_test\_NABS, }\AttributeTok{fill =} \SpecialCharTok{\textasciitilde{}}\NormalTok{pretty\_condition, }\AttributeTok{data =}\NormalTok{ df\_subjects) }\SpecialCharTok{\%\textgreater{}\%} 
  \FunctionTok{gf\_facet\_wrap}\NormalTok{(}\SpecialCharTok{\textasciitilde{}}\NormalTok{pretty\_condition) }\SpecialCharTok{+} 
  \FunctionTok{labs}\NormalTok{( }\AttributeTok{x =} \StringTok{"Absolute Score in TEST Phase"}\NormalTok{, }
        \AttributeTok{title =} \StringTok{"Distribution of TEST PHASE Absolute Score (all Items)"}\NormalTok{,}
        \AttributeTok{subtitle =} \FunctionTok{paste}\NormalTok{(}\StringTok{""}\NormalTok{),}
        \AttributeTok{y =} \StringTok{"Proportion of Items"}\NormalTok{) }\SpecialCharTok{+}
  \FunctionTok{scale\_fill\_discrete}\NormalTok{(}\AttributeTok{name =} \StringTok{"Condition"}\NormalTok{) }\SpecialCharTok{+}  
  \FunctionTok{theme\_minimal}\NormalTok{()}
\end{Highlighting}
\end{Shaded}

\begin{figure}[H]

{\centering \includegraphics{analysis/SGC3B/2_sgc3B_scoring_files/figure-pdf/DISTR-ABSCORE-4.pdf}

}

\end{figure}

\hypertarget{scaled-score-2}{%
\subsection{Scaled Score}\label{scaled-score-2}}

\begin{Shaded}
\begin{Highlighting}[]
\FunctionTok{options}\NormalTok{(}\AttributeTok{repr.plot.width =}\DecValTok{9}\NormalTok{, }\AttributeTok{repr.plot.height =}\DecValTok{12}\NormalTok{)}

\CommentTok{\#DISTRIBUTION SCALED SCORE FULL }\AlertTok{TASK}
\FunctionTok{gf\_props}\NormalTok{(}\SpecialCharTok{\textasciitilde{}}\NormalTok{score\_SCALED, }\AttributeTok{fill =} \SpecialCharTok{\textasciitilde{}}\NormalTok{pretty\_condition, }\AttributeTok{position =} \FunctionTok{position\_dodge}\NormalTok{(), }\AttributeTok{data =}\NormalTok{ df) }\SpecialCharTok{+}
  \FunctionTok{labs}\NormalTok{( }\AttributeTok{x =} \StringTok{"Scaled Score"}\NormalTok{, }
        \AttributeTok{title =} \StringTok{"Distribution of Scaled Score (all Items)"}\NormalTok{,}
        \AttributeTok{subtitle =} \StringTok{"Impasse Condition (blue) yields higher scaled scores across the entire task"}\NormalTok{,}
        \AttributeTok{y =} \StringTok{"Proportion of Items"}\NormalTok{) }\SpecialCharTok{+}
  \FunctionTok{scale\_fill\_discrete}\NormalTok{(}\AttributeTok{name =} \StringTok{"Condition"}\NormalTok{) }\SpecialCharTok{+}  
  \FunctionTok{theme\_minimal}\NormalTok{()}
\end{Highlighting}
\end{Shaded}

\begin{figure}[H]

{\centering \includegraphics{analysis/SGC3B/2_sgc3B_scoring_files/figure-pdf/DISTR-SCALEDSCORE-1.pdf}

}

\end{figure}

\begin{Shaded}
\begin{Highlighting}[]
\CommentTok{\#DISTRIBUTION SCALED SCORE BY ITEM}
\FunctionTok{gf\_props}\NormalTok{(}\SpecialCharTok{\textasciitilde{}}\NormalTok{score\_SCALED, }\AttributeTok{fill =} \SpecialCharTok{\textasciitilde{}}\NormalTok{pretty\_condition, }\AttributeTok{position =} \FunctionTok{position\_dodge}\NormalTok{(), }\AttributeTok{data =}\NormalTok{ df)  }\SpecialCharTok{\%\textgreater{}\%} 
  \FunctionTok{gf\_facet\_grid}\NormalTok{(q}\SpecialCharTok{\textasciitilde{}}\NormalTok{pretty\_condition) }\SpecialCharTok{+} 
  \FunctionTok{labs}\NormalTok{( }\AttributeTok{x =} \StringTok{"Scaled Score"}\NormalTok{, }
        \AttributeTok{title =} \StringTok{"Distribution of Scaled Score (by Item)"}\NormalTok{,}
        \AttributeTok{subtitle =} \StringTok{"Impasse Condition (blue) yields higher scaled scores on each item"}\NormalTok{,}
        \AttributeTok{y =} \StringTok{"Proportion of Subjects"}\NormalTok{) }\SpecialCharTok{+}
  \FunctionTok{scale\_fill\_discrete}\NormalTok{(}\AttributeTok{name =} \StringTok{"Condition"}\NormalTok{) }\SpecialCharTok{+}  \FunctionTok{scale\_y\_continuous}\NormalTok{(}\AttributeTok{breaks=}\FunctionTok{c}\NormalTok{(}\DecValTok{0}\NormalTok{,}\FloatTok{0.5}\NormalTok{)) }\SpecialCharTok{+} 
  \FunctionTok{theme\_minimal}\NormalTok{() }\SpecialCharTok{+} \FunctionTok{theme}\NormalTok{(}\AttributeTok{legend.position=}\StringTok{"blank"}\NormalTok{)}
\end{Highlighting}
\end{Shaded}

\begin{figure}[H]

{\centering \includegraphics{analysis/SGC3B/2_sgc3B_scoring_files/figure-pdf/DISTR-SCALEDSCORE-2.pdf}

}

\end{figure}

\begin{Shaded}
\begin{Highlighting}[]
\CommentTok{\#DISTRIBUTION SCALED SCORE BY SUBJECT}
\FunctionTok{gf\_props}\NormalTok{(}\SpecialCharTok{\textasciitilde{}}\NormalTok{s\_SCALED, }\AttributeTok{fill =} \SpecialCharTok{\textasciitilde{}}\NormalTok{pretty\_condition, }\AttributeTok{data =}\NormalTok{ df\_subjects)  }\SpecialCharTok{\%\textgreater{}\%} 
  \FunctionTok{gf\_facet\_grid}\NormalTok{(pretty\_condition }\SpecialCharTok{\textasciitilde{}}\NormalTok{. )}\SpecialCharTok{+}
  \FunctionTok{labs}\NormalTok{( }\AttributeTok{x =} \StringTok{"Total Scaled Score"}\NormalTok{, }
        \AttributeTok{title =} \StringTok{"Distribution of Total Scaled Score (by Subject)"}\NormalTok{,}
        \AttributeTok{subtitle =} \StringTok{"Impasse Condition (blue) yields higher cumulative scaled scores"}\NormalTok{,}
        \AttributeTok{y =} \StringTok{"Number of Subjects"}\NormalTok{) }\SpecialCharTok{+}
  \FunctionTok{scale\_fill\_discrete}\NormalTok{(}\AttributeTok{name =} \StringTok{"Condition"}\NormalTok{) }\SpecialCharTok{+}  
  \FunctionTok{theme\_minimal}\NormalTok{()}
\end{Highlighting}
\end{Shaded}

\begin{figure}[H]

{\centering \includegraphics{analysis/SGC3B/2_sgc3B_scoring_files/figure-pdf/DISTR-SCALEDSCORE-3.pdf}

}

\end{figure}

\begin{Shaded}
\begin{Highlighting}[]
\CommentTok{\#DISTRIBUTION SCALED SCORE }\AlertTok{TEST}\CommentTok{ PHASE}
\FunctionTok{gf\_histogram}\NormalTok{(}\SpecialCharTok{\textasciitilde{}}\NormalTok{item\_test\_SCALED, }\AttributeTok{fill =} \SpecialCharTok{\textasciitilde{}}\NormalTok{pretty\_condition, }\AttributeTok{data =}\NormalTok{ df\_subjects) }\SpecialCharTok{\%\textgreater{}\%} 
  \FunctionTok{gf\_facet\_wrap}\NormalTok{(}\SpecialCharTok{\textasciitilde{}}\NormalTok{pretty\_condition) }\SpecialCharTok{+} 
  \FunctionTok{labs}\NormalTok{( }\AttributeTok{x =} \StringTok{"Scaled Score in TEST Phase"}\NormalTok{, }
        \AttributeTok{title =} \StringTok{"Distribution of TEST PHASE Scaled Score (all Items)"}\NormalTok{,}
        \AttributeTok{subtitle =} \FunctionTok{paste}\NormalTok{(}\StringTok{""}\NormalTok{),}
        \AttributeTok{y =} \StringTok{"Proportion of Items"}\NormalTok{) }\SpecialCharTok{+}
  \FunctionTok{scale\_fill\_discrete}\NormalTok{(}\AttributeTok{name =} \StringTok{"Condition"}\NormalTok{) }\SpecialCharTok{+}  
  \FunctionTok{theme\_minimal}\NormalTok{()}
\end{Highlighting}
\end{Shaded}

\begin{figure}[H]

{\centering \includegraphics{analysis/SGC3B/2_sgc3B_scoring_files/figure-pdf/DISTR-SCALEDSCORE-4.pdf}

}

\end{figure}

\begin{itemize}
\tightlist
\item
  TODO: INVESTIGATE if some of the scores assigned to 0 should be
  assigned to -0.5 to balance
\item
  TODO: INVESTIGATE DISTRIBUTIONS of each subscore type
\end{itemize}

\hypertarget{interpretations-1}{%
\subsection{Interpretations}\label{interpretations-1}}

\begin{Shaded}
\begin{Highlighting}[]
\CommentTok{\#DISTRIBUTION OF INTERPRETATION}
\FunctionTok{gf\_props}\NormalTok{(}\SpecialCharTok{\textasciitilde{}}\NormalTok{pretty\_interpretation, }\AttributeTok{fill =} \SpecialCharTok{\textasciitilde{}}\NormalTok{pretty\_condition, }\AttributeTok{data =}\NormalTok{ df) }\SpecialCharTok{\%\textgreater{}\%} 
  \FunctionTok{gf\_facet\_grid}\NormalTok{( pretty\_condition }\SpecialCharTok{\textasciitilde{}}\NormalTok{ ., }\AttributeTok{labeller =}\NormalTok{ label\_both) }\SpecialCharTok{+} 
  \FunctionTok{labs}\NormalTok{( }\AttributeTok{title =} \StringTok{"Distribution of Interpretations (across Task)"}\NormalTok{,}
        \AttributeTok{x =} \StringTok{"Graph Interpretation"}\NormalTok{,}
        \AttributeTok{y =} \StringTok{"Proportion of Responses"}\NormalTok{,}
        \AttributeTok{subtitle =} \StringTok{"Impasse condition (blue) yields fewer Orthogonal and more Triangular responses"}\NormalTok{) }\SpecialCharTok{+} 
  \FunctionTok{theme\_minimal}\NormalTok{() }\SpecialCharTok{+} \FunctionTok{theme}\NormalTok{(}\AttributeTok{legend.position =} \StringTok{"blank"}\NormalTok{)}
\end{Highlighting}
\end{Shaded}

\begin{figure}[H]

{\centering \includegraphics{analysis/SGC3B/2_sgc3B_scoring_files/figure-pdf/DISTR-INTERPRETATIONS-1.pdf}

}

\end{figure}

\begin{Shaded}
\begin{Highlighting}[]
\CommentTok{\#DISTRIBUTION OF INTERPRETATION ACROSS ITEMS}
\FunctionTok{gf\_propsh}\NormalTok{(}\SpecialCharTok{\textasciitilde{}}\NormalTok{ pretty\_interpretation, }\AttributeTok{fill =} \SpecialCharTok{\textasciitilde{}}\NormalTok{pretty\_condition, }\AttributeTok{data =}\NormalTok{ df) }\SpecialCharTok{\%\textgreater{}\%} 
  \FunctionTok{gf\_facet\_grid}\NormalTok{( pretty\_condition}\SpecialCharTok{\textasciitilde{}}\NormalTok{q) }\SpecialCharTok{+} 
  \FunctionTok{labs}\NormalTok{( }\AttributeTok{title =} \StringTok{"Distribution of Interpretations (by Item)"}\NormalTok{,}
        \AttributeTok{subtitle =} \StringTok{"Impasse condition (blue) yields more Triangular responses on each question"}\NormalTok{,}
        \AttributeTok{y =} \StringTok{"Interpretation"}\NormalTok{, }\AttributeTok{x =} \StringTok{"Proportion of Subjects"}\NormalTok{) }\SpecialCharTok{+} \FunctionTok{theme\_minimal}\NormalTok{() }\SpecialCharTok{+} \FunctionTok{theme}\NormalTok{(}\AttributeTok{legend.position =} \StringTok{"blank"}\NormalTok{)}
\end{Highlighting}
\end{Shaded}

\begin{figure}[H]

{\centering \includegraphics{analysis/SGC3B/2_sgc3B_scoring_files/figure-pdf/DISTR-INTERPRETATIONS-2.pdf}

}

\end{figure}

\begin{Shaded}
\begin{Highlighting}[]
\CommentTok{\#DISTRIBUTION OF INTERPRETATION TYPE ACROSS ITEMS}
\FunctionTok{gf\_propsh}\NormalTok{(}\SpecialCharTok{\textasciitilde{}}\NormalTok{ high\_interpretation, }\AttributeTok{fill =} \SpecialCharTok{\textasciitilde{}}\NormalTok{pretty\_condition, }\AttributeTok{data =}\NormalTok{ df) }\SpecialCharTok{\%\textgreater{}\%} 
  \FunctionTok{gf\_facet\_grid}\NormalTok{( pretty\_condition}\SpecialCharTok{\textasciitilde{}}\NormalTok{q) }\SpecialCharTok{+} 
  \FunctionTok{labs}\NormalTok{( }\AttributeTok{title =} \StringTok{"Distribution of Interpretations (by Item)"}\NormalTok{,}
        \AttributeTok{subtitle =} \StringTok{"Impasse condition (blue) yields more positive trending responses on each question"}\NormalTok{,}
        \AttributeTok{y =} \StringTok{"Interpretation"}\NormalTok{, }\AttributeTok{x =} \StringTok{"Proportion of Subjects"}\NormalTok{) }\SpecialCharTok{+} \FunctionTok{theme\_minimal}\NormalTok{() }\SpecialCharTok{+} \FunctionTok{theme}\NormalTok{(}\AttributeTok{legend.position =} \StringTok{"blank"}\NormalTok{)}
\end{Highlighting}
\end{Shaded}

\begin{figure}[H]

{\centering \includegraphics{analysis/SGC3B/2_sgc3B_scoring_files/figure-pdf/DISTR-INTERPRETATIONS-3.pdf}

}

\end{figure}

\hypertarget{progress-over-task-1}{%
\subsection{Progress over Task}\label{progress-over-task-1}}

\begin{Shaded}
\begin{Highlighting}[]
\CommentTok{\#VISUALIZE progress over time ABSOLUTE score }
\FunctionTok{ggplot}\NormalTok{(}\AttributeTok{data =}\NormalTok{ df\_absolute\_progress, }\FunctionTok{aes}\NormalTok{(}\AttributeTok{x =}\NormalTok{ question, }\AttributeTok{y =}\NormalTok{ score, }\AttributeTok{group =}\NormalTok{ subject, }\AttributeTok{alpha =} \FloatTok{0.01}\NormalTok{, }\AttributeTok{color =}\NormalTok{ pretty\_condition)) }\SpecialCharTok{+} 
 \FunctionTok{geom\_line}\NormalTok{(}\AttributeTok{position=}\FunctionTok{position\_jitter}\NormalTok{(}\AttributeTok{w=}\FloatTok{0.15}\NormalTok{, }\AttributeTok{h=}\FloatTok{0.15}\NormalTok{), }\AttributeTok{size=}\FloatTok{0.1}\NormalTok{) }\SpecialCharTok{+}
 \FunctionTok{facet\_wrap}\NormalTok{(}\SpecialCharTok{\textasciitilde{}}\NormalTok{pretty\_condition) }\SpecialCharTok{+} 
 \FunctionTok{labs}\NormalTok{ (}\AttributeTok{title =} \StringTok{"Cumulative Absolute Score over sequence of task"}\NormalTok{, }\AttributeTok{x =} \StringTok{"Question"}\NormalTok{ , }\AttributeTok{y =} \StringTok{"Cumulative Absolute Score"}\NormalTok{) }\SpecialCharTok{+} 
 \FunctionTok{scale\_x\_continuous}\NormalTok{(}\AttributeTok{breaks =} \FunctionTok{c}\NormalTok{(}\DecValTok{1}\NormalTok{,}\DecValTok{2}\NormalTok{,}\DecValTok{3}\NormalTok{,}\DecValTok{4}\NormalTok{,}\DecValTok{5}\NormalTok{,}\DecValTok{6}\NormalTok{,}\DecValTok{7}\NormalTok{,}\DecValTok{8}\NormalTok{,}\DecValTok{9}\NormalTok{,}\DecValTok{10}\NormalTok{,}\DecValTok{11}\NormalTok{,}\DecValTok{12}\NormalTok{,}\DecValTok{13}\NormalTok{)) }\SpecialCharTok{+}
 \FunctionTok{theme\_minimal}\NormalTok{() }\SpecialCharTok{+} \FunctionTok{theme}\NormalTok{(}\AttributeTok{legend.position =} \StringTok{"blank"}\NormalTok{)}
\end{Highlighting}
\end{Shaded}

\begin{figure}[H]

{\centering \includegraphics{analysis/SGC3B/2_sgc3B_scoring_files/figure-pdf/VIZ-PROGRESS-1.pdf}

}

\end{figure}

\begin{Shaded}
\begin{Highlighting}[]
\CommentTok{\#VISUALIZE progress over time SCALED score }
\FunctionTok{ggplot}\NormalTok{(}\AttributeTok{data =}\NormalTok{ df\_scaled\_progress, }\FunctionTok{aes}\NormalTok{(}\AttributeTok{x =}\NormalTok{ question, }\AttributeTok{y =}\NormalTok{ score, }\AttributeTok{group =}\NormalTok{ subject, }\AttributeTok{alpha =} \FloatTok{0.01}\NormalTok{, }\AttributeTok{color =}\NormalTok{ pretty\_condition)) }\SpecialCharTok{+} 
 \FunctionTok{geom\_line}\NormalTok{(}\AttributeTok{position=}\FunctionTok{position\_jitter}\NormalTok{(}\AttributeTok{w=}\FloatTok{0.15}\NormalTok{, }\AttributeTok{h=}\FloatTok{0.15}\NormalTok{), }\AttributeTok{size=}\FloatTok{0.1}\NormalTok{) }\SpecialCharTok{+}
 \FunctionTok{facet\_wrap}\NormalTok{(}\SpecialCharTok{\textasciitilde{}}\NormalTok{pretty\_condition) }\SpecialCharTok{+} 
 \FunctionTok{labs}\NormalTok{ (}\AttributeTok{title =} \StringTok{"Cumulative Scaled Score over sequence of task"}\NormalTok{, }\AttributeTok{x =} \StringTok{"Question"}\NormalTok{ , }\AttributeTok{y =} \StringTok{"Cumulative Scaled Score"}\NormalTok{) }\SpecialCharTok{+} 
 \FunctionTok{scale\_x\_continuous}\NormalTok{(}\AttributeTok{breaks =} \FunctionTok{c}\NormalTok{(}\DecValTok{1}\NormalTok{,}\DecValTok{2}\NormalTok{,}\DecValTok{3}\NormalTok{,}\DecValTok{4}\NormalTok{,}\DecValTok{5}\NormalTok{,}\DecValTok{6}\NormalTok{,}\DecValTok{7}\NormalTok{,}\DecValTok{8}\NormalTok{,}\DecValTok{9}\NormalTok{,}\DecValTok{10}\NormalTok{,}\DecValTok{11}\NormalTok{,}\DecValTok{12}\NormalTok{,}\DecValTok{13}\NormalTok{)) }\SpecialCharTok{+}
 \FunctionTok{theme\_minimal}\NormalTok{() }\SpecialCharTok{+} \FunctionTok{theme}\NormalTok{(}\AttributeTok{legend.position =} \StringTok{"blank"}\NormalTok{)}
\end{Highlighting}
\end{Shaded}

\begin{figure}[H]

{\centering \includegraphics{analysis/SGC3B/2_sgc3B_scoring_files/figure-pdf/VIZ-PROGRESS-2.pdf}

}

\end{figure}

\hypertarget{interpretation-subscores-1}{%
\subsection{Interpretation Subscores}\label{interpretation-subscores-1}}

\begin{Shaded}
\begin{Highlighting}[]
\FunctionTok{gf\_density}\NormalTok{(}\SpecialCharTok{\textasciitilde{}}\NormalTok{ s\_TRI, }\AttributeTok{fill =} \SpecialCharTok{\textasciitilde{}}\NormalTok{pretty\_condition, }\AttributeTok{data =}\NormalTok{ df\_subjects) }\SpecialCharTok{\%\textgreater{}\%} 
  \FunctionTok{gf\_facet\_wrap}\NormalTok{( }\SpecialCharTok{\textasciitilde{}}\NormalTok{ pretty\_condition) }\SpecialCharTok{+} 
  \FunctionTok{labs}\NormalTok{( }\AttributeTok{title =} \StringTok{"Distribution of Total Triangular Score"}\NormalTok{,}
        \AttributeTok{subtitle =} \StringTok{"Impasse shifts density toward higher Triagular scores"}\NormalTok{,}
        \AttributeTok{x =} \StringTok{"Item Triangular Score"}\NormalTok{, }\AttributeTok{y =} \StringTok{"Proportion of Subjects"}\NormalTok{) }\SpecialCharTok{+} 
        \FunctionTok{theme\_minimal}\NormalTok{() }\SpecialCharTok{+} \FunctionTok{theme}\NormalTok{(}\AttributeTok{legend.position =} \StringTok{"blank"}\NormalTok{)}
\end{Highlighting}
\end{Shaded}

\begin{figure}[H]

{\centering \includegraphics{analysis/SGC3B/2_sgc3B_scoring_files/figure-pdf/DIST-SUBSCORES-1.pdf}

}

\end{figure}

\begin{Shaded}
\begin{Highlighting}[]
\FunctionTok{gf\_density}\NormalTok{(}\SpecialCharTok{\textasciitilde{}}\NormalTok{ s\_ORTH, }\AttributeTok{fill =} \SpecialCharTok{\textasciitilde{}}\NormalTok{pretty\_condition, }\AttributeTok{data =}\NormalTok{ df\_subjects) }\SpecialCharTok{\%\textgreater{}\%} 
  \FunctionTok{gf\_facet\_wrap}\NormalTok{( }\SpecialCharTok{\textasciitilde{}}\NormalTok{ pretty\_condition) }\SpecialCharTok{+} 
  \FunctionTok{labs}\NormalTok{( }\AttributeTok{title =} \StringTok{"Distribution of Total Orthogonal Score"}\NormalTok{,}
        \AttributeTok{subtitle =} \StringTok{"Impasse shifts density toward lower Orthogonal scores"}\NormalTok{,}
        \AttributeTok{x =} \StringTok{"Item Orthogonal Score"}\NormalTok{, }\AttributeTok{y =} \StringTok{"Proportion of Subjects"}\NormalTok{) }\SpecialCharTok{+} 
        \FunctionTok{theme\_minimal}\NormalTok{() }\SpecialCharTok{+} \FunctionTok{theme}\NormalTok{(}\AttributeTok{legend.position =} \StringTok{"blank"}\NormalTok{)}
\end{Highlighting}
\end{Shaded}

\begin{figure}[H]

{\centering \includegraphics{analysis/SGC3B/2_sgc3B_scoring_files/figure-pdf/DIST-SUBSCORES-2.pdf}

}

\end{figure}

\begin{Shaded}
\begin{Highlighting}[]
\FunctionTok{gf\_density}\NormalTok{(}\SpecialCharTok{\textasciitilde{}}\NormalTok{ s\_TVERSKY, }\AttributeTok{fill =} \SpecialCharTok{\textasciitilde{}}\NormalTok{pretty\_condition, }\AttributeTok{data =}\NormalTok{ df\_subjects) }\SpecialCharTok{\%\textgreater{}\%} 
  \FunctionTok{gf\_facet\_wrap}\NormalTok{( }\SpecialCharTok{\textasciitilde{}}\NormalTok{ pretty\_condition) }\SpecialCharTok{+} 
  \FunctionTok{labs}\NormalTok{( }\AttributeTok{title =} \StringTok{"Distribution of Total Tversky Score"}\NormalTok{,}
        \AttributeTok{subtitle =} \StringTok{"Impasse shifts density toward higher Tversky scores"}\NormalTok{,}
        \AttributeTok{x =} \StringTok{"Item Orthogonal Score"}\NormalTok{, }\AttributeTok{y =} \StringTok{"Proportion of Subjects"}\NormalTok{) }\SpecialCharTok{+} 
        \FunctionTok{theme\_minimal}\NormalTok{() }\SpecialCharTok{+} \FunctionTok{theme}\NormalTok{(}\AttributeTok{legend.position =} \StringTok{"blank"}\NormalTok{)}
\end{Highlighting}
\end{Shaded}

\begin{figure}[H]

{\centering \includegraphics{analysis/SGC3B/2_sgc3B_scoring_files/figure-pdf/DIST-SUBSCORES-3.pdf}

}

\end{figure}

\begin{Shaded}
\begin{Highlighting}[]
\FunctionTok{gf\_histogram}\NormalTok{(}\SpecialCharTok{\textasciitilde{}}\NormalTok{ s\_SATISFICE, }\AttributeTok{fill =} \SpecialCharTok{\textasciitilde{}}\NormalTok{pretty\_condition, }\AttributeTok{data =}\NormalTok{ df\_subjects) }\SpecialCharTok{\%\textgreater{}\%} 
  \FunctionTok{gf\_facet\_wrap}\NormalTok{( }\SpecialCharTok{\textasciitilde{}}\NormalTok{ pretty\_condition) }\SpecialCharTok{+} 
  \FunctionTok{labs}\NormalTok{( }\AttributeTok{title =} \StringTok{"Distribution of Total Satisfice Score"}\NormalTok{,}
        \AttributeTok{subtitle =} \StringTok{"Satisficing only occurs in impasse, when no orthogonal response is available"}\NormalTok{,}
        \AttributeTok{x =} \StringTok{"Item Orthogonal Score"}\NormalTok{, }\AttributeTok{y =} \StringTok{"Proportion of Subjects"}\NormalTok{) }\SpecialCharTok{+} 
        \FunctionTok{theme\_minimal}\NormalTok{() }\SpecialCharTok{+} \FunctionTok{theme}\NormalTok{(}\AttributeTok{legend.position =} \StringTok{"blank"}\NormalTok{)}
\end{Highlighting}
\end{Shaded}

\begin{figure}[H]

{\centering \includegraphics{analysis/SGC3B/2_sgc3B_scoring_files/figure-pdf/DIST-SUBSCORES-4.pdf}

}

\end{figure}

\hypertarget{export-3}{%
\section{EXPORT}\label{export-3}}

Finally, we export the scores for each item (\texttt{df\_items}) and
summarized over subjects (\texttt{df\_subjects}), as well as cumulative
progress dataframes (\texttt{df\_absolute\_progress},
\texttt{df\_scaled\_progress})

\begin{Shaded}
\begin{Highlighting}[]
\NormalTok{imac }\OtherTok{=} \StringTok{"/Users/amyraefox/Code/SGC{-}Scaffolding\_Graph\_Comprehension/SGC{-}X/ANALYSIS/MAIN"}
\FunctionTok{setwd}\NormalTok{(imac)}

\CommentTok{\#SAVE FILES}
\FunctionTok{write.csv}\NormalTok{(df\_subjects,}\StringTok{"analysis/SGC3B/data/2{-}scored{-}data/sgc3b\_scored\_participants.csv"}\NormalTok{, }\AttributeTok{row.names =} \ConstantTok{FALSE}\NormalTok{)}
\FunctionTok{write.csv}\NormalTok{(df\_items,}\StringTok{"analysis/SGC3B/data/2{-}scored{-}data/sgc3b\_scored\_items.csv"}\NormalTok{, }\AttributeTok{row.names =} \ConstantTok{FALSE}\NormalTok{)}
\FunctionTok{write.csv}\NormalTok{(df\_absolute\_progress,}\StringTok{"analysis/SGC3B/data/2{-}scored{-}data/sgc3b\_absolute\_progress.csv"}\NormalTok{, }\AttributeTok{row.names =} \ConstantTok{FALSE}\NormalTok{)}
\FunctionTok{write.csv}\NormalTok{(df\_scaled\_progress,}\StringTok{"analysis/SGC3B/data/2{-}scored{-}data/sgc3b\_scaled\_progress.csv"}\NormalTok{, }\AttributeTok{row.names =} \ConstantTok{FALSE}\NormalTok{)}

\CommentTok{\#SAVE R Data Structures}
\CommentTok{\#export R DATA STRUCTURES (include codebook metadata)}
\NormalTok{rio}\SpecialCharTok{::}\FunctionTok{export}\NormalTok{(df\_subjects, }\StringTok{"analysis/SGC3B/data/2{-}scored{-}data/sgc3b\_scored\_participants.rds"}\NormalTok{) }\CommentTok{\# to R data structure file}
\NormalTok{rio}\SpecialCharTok{::}\FunctionTok{export}\NormalTok{(df\_items, }\StringTok{"analysis/SGC3B/data/2{-}scored{-}data/sgc3b\_scored\_items.rds"}\NormalTok{) }\CommentTok{\# to R data structure file}
\end{Highlighting}
\end{Shaded}

\hypertarget{resources-6}{%
\section{RESOURCES}\label{resources-6}}

\begin{Shaded}
\begin{Highlighting}[]
\FunctionTok{sessionInfo}\NormalTok{()}
\end{Highlighting}
\end{Shaded}

\begin{verbatim}
R version 4.2.1 (2022-06-23)
Platform: x86_64-apple-darwin17.0 (64-bit)
Running under: macOS Big Sur ... 10.16

Matrix products: default
BLAS:   /Library/Frameworks/R.framework/Versions/4.2/Resources/lib/libRblas.0.dylib
LAPACK: /Library/Frameworks/R.framework/Versions/4.2/Resources/lib/libRlapack.dylib

locale:
[1] en_US.UTF-8/en_US.UTF-8/en_US.UTF-8/C/en_US.UTF-8/en_US.UTF-8

attached base packages:
[1] stats     graphics  grDevices utils     datasets  methods   base     

other attached packages:
 [1] forcats_0.5.1    stringr_1.4.0    dplyr_1.0.9      purrr_0.3.4     
 [5] readr_2.1.2      tidyr_1.2.0      tibble_3.1.7     tidyverse_1.3.1 
 [9] Hmisc_4.7-0      Formula_1.2-4    survival_3.3-1   lattice_0.20-45 
[13] pbapply_1.5-0    ggformula_0.10.1 ggridges_0.5.3   scales_1.2.0    
[17] ggstance_0.3.5   ggplot2_3.3.6    kableExtra_1.3.4

loaded via a namespace (and not attached):
 [1] fs_1.5.2            bit64_4.0.5         lubridate_1.8.0    
 [4] webshot_0.5.3       RColorBrewer_1.1-3  httr_1.4.3         
 [7] tools_4.2.1         backports_1.4.1     utf8_1.2.2         
[10] R6_2.5.1            rpart_4.1.16        DBI_1.1.3          
[13] colorspace_2.0-3    nnet_7.3-17         withr_2.5.0        
[16] tidyselect_1.1.2    gridExtra_2.3       curl_4.3.2         
[19] bit_4.0.4           compiler_4.2.1      cli_3.3.0          
[22] rvest_1.0.2         htmlTable_2.4.0     xml2_1.3.3         
[25] labeling_0.4.2      mosaicCore_0.9.0    checkmate_2.1.0    
[28] systemfonts_1.0.4   digest_0.6.29       foreign_0.8-82     
[31] rmarkdown_2.14      svglite_2.1.0       rio_0.5.29         
[34] base64enc_0.1-3     jpeg_0.1-9          pkgconfig_2.0.3    
[37] htmltools_0.5.2     labelled_2.9.1      dbplyr_2.2.1       
[40] fastmap_1.1.0       readxl_1.4.0        htmlwidgets_1.5.4  
[43] rlang_1.0.3         rstudioapi_0.13     farver_2.1.0       
[46] generics_0.1.2      jsonlite_1.8.0      vroom_1.5.7        
[49] zip_2.2.0           magrittr_2.0.3      Matrix_1.4-1       
[52] Rcpp_1.0.8.3        munsell_0.5.0       fansi_1.0.3        
[55] lifecycle_1.0.1     stringi_1.7.6       yaml_2.3.5         
[58] MASS_7.3-57         plyr_1.8.7          grid_4.2.1         
[61] parallel_4.2.1      crayon_1.5.1        haven_2.5.0        
[64] splines_4.2.1       hms_1.1.1           knitr_1.39         
[67] pillar_1.7.0        codetools_0.2-18    reprex_2.0.1       
[70] glue_1.6.2          evaluate_0.15       latticeExtra_0.6-29
[73] data.table_1.14.2   modelr_0.1.8        tzdb_0.3.0         
[76] png_0.1-7           vctrs_0.4.1         tweenr_1.0.2       
[79] cellranger_1.1.0    gtable_0.3.0        polyclip_1.10-0    
[82] assertthat_0.2.1    openxlsx_4.2.5      xfun_0.31          
[85] ggforce_0.3.3       broom_0.8.0         viridisLite_0.4.0  
[88] cluster_2.1.3       ellipsis_0.3.2     
\end{verbatim}

\part{SGC4A}

\newpage

\hypertarget{sec-SGC4A-introduction}{%
\chapter{Introduction}\label{sec-SGC4A-introduction}}

\textbf{In Study 4A we explore the extent to which the design of the
axes and gridlines of the graph influence how a reader interprets its
underlying coordinate system.}

\begin{longtable}[]{@{}
  >{\raggedright\arraybackslash}p{(\columnwidth - 2\tabcolsep) * \real{0.4000}}
  >{\raggedright\arraybackslash}p{(\columnwidth - 2\tabcolsep) * \real{0.6000}}@{}}
\caption{\textbf{SGC4A Study Conditions}}\tabularnewline
\toprule()
\endhead
\includegraphics{analysis/utils/img/111.png} &
\begin{minipage}[t]{\linewidth}\raggedright
\textbf{Orthogonal-Full}\\
Demo:
\href{https://limitless-plains-85018.herokuapp.com/?study=SGC4A\&condition=111\&session=WEB-DEMO}{111}\strut
\end{minipage} \\
\includegraphics{analysis/utils/img/114.png} &
\begin{minipage}[t]{\linewidth}\raggedright
\textbf{Orthogonal-Sparse}\\
Demo:
\href{https://limitless-plains-85018.herokuapp.com/?study=SGC4A\&condition=114\&session=WEB-DEMO}{114}\strut
\end{minipage} \\
\includegraphics{analysis/utils/img/115.png} &
\begin{minipage}[t]{\linewidth}\raggedright
\textbf{Orthogonal-Grid}\\
Demo:
\href{https://limitless-plains-85018.herokuapp.com/?study=SGC4A\&condition=115\&session=WEB-DEMO}{115}\strut
\end{minipage} \\
\includegraphics{analysis/utils/img/113.png} &
\begin{minipage}[t]{\linewidth}\raggedright
\textbf{Triangular-Sparse}\\
Demo:
\href{https://limitless-plains-85018.herokuapp.com/?study=SGC4A\&condition=113\&session=WEB-DEMO}{113}\\
\strut
\end{minipage} \\
\bottomrule()
\end{longtable}

\begin{Shaded}
\begin{Highlighting}[]
\FunctionTok{library}\NormalTok{(codebook) }\CommentTok{\#data dictionary}
\FunctionTok{library}\NormalTok{(tidyverse) }\CommentTok{\#ALL THE THINGS}
\FunctionTok{library}\NormalTok{(kableExtra) }\CommentTok{\#tables}

\CommentTok{\#set some output options}
\FunctionTok{library}\NormalTok{(dplyr, }\AttributeTok{warn.conflicts =} \ConstantTok{FALSE}\NormalTok{)}
\FunctionTok{options}\NormalTok{(}\AttributeTok{dplyr.summarise.inform =} \ConstantTok{FALSE}\NormalTok{)}
\FunctionTok{options}\NormalTok{(}\AttributeTok{scipen=}\DecValTok{1}\NormalTok{, }\AttributeTok{digits=}\DecValTok{3}\NormalTok{)}
\end{Highlighting}
\end{Shaded}

\begin{Shaded}
\begin{Highlighting}[]
\CommentTok{\# }\AlertTok{HACK}\CommentTok{ WD FOR LOCAL RUNNING?}
\CommentTok{\# imac = "/Users/amyraefox/Code/SGC{-}Scaffolding\_Graph\_Comprehension/SGC{-}X/ANALYSIS/MAIN"}
\CommentTok{\# mbp = "/Users/amyfox/Sites/RESEARCH/SGC—Scaffolding Graph Comprehension/SGC{-}X/ANALYSIS/MAIN"}
\CommentTok{\# setwd(mbp)}

\CommentTok{\#IMPORT DATA }
\NormalTok{df\_subjects }\OtherTok{\textless{}{-}} \FunctionTok{read\_rds}\NormalTok{(}\StringTok{\textquotesingle{}analysis/SGC4A/data/0{-}study{-}level/sgc4a\_participants.rds\textquotesingle{}}\NormalTok{)}
\end{Highlighting}
\end{Shaded}

\begin{Shaded}
\begin{Highlighting}[]
\NormalTok{title }\OtherTok{=} \StringTok{"Participants by Condition"}
\NormalTok{cols }\OtherTok{=} \FunctionTok{c}\NormalTok{(}\StringTok{"Condition"}\NormalTok{,}\StringTok{"n"}\NormalTok{)}
\NormalTok{cont }\OtherTok{\textless{}{-}} \FunctionTok{table}\NormalTok{(df\_subjects}\SpecialCharTok{$}\NormalTok{pretty\_condition)}
\NormalTok{cont }\SpecialCharTok{\%\textgreater{}\%}  \FunctionTok{addmargins}\NormalTok{() }\SpecialCharTok{\%\textgreater{}\%} \FunctionTok{kbl}\NormalTok{(}\AttributeTok{caption =}\NormalTok{ title, }\AttributeTok{col.names =}\NormalTok{ cols) }\SpecialCharTok{\%\textgreater{}\%} \FunctionTok{kable\_classic}\NormalTok{()}
\end{Highlighting}
\end{Shaded}

\begin{table}

\caption{Participants by Condition}
\centering
\begin{tabular}[t]{l|r}
\hline
Condition & n\\
\hline
Orth-Full & 88\\
\hline
Orth-Sparse & 88\\
\hline
Orth-Grid & 98\\
\hline
Tri-Sparse & 86\\
\hline
Sum & 360\\
\hline
\end{tabular}
\end{table}

\textbf{Experimental Hypothesis:}\\
\emph{We hypothesize that the design of the major axes (specifically
orthogonal) axes establish for the learner the basis of the coordinate
system. Differently oriented axes should lead the reader to be more open
to alternative coordinate systems.}

\begin{itemize}
\tightlist
\item
  H1 The Triangular axes condition should yield significantly better
  performance

  \begin{itemize}
  \tightlist
  \item
    H1A \textbar{} Triangular axis condition should yield higher total
    absolute score
  \item
    H1B \textbar{} Triangular axis condition should yield higher first
    question score
  \item
    H1C \textbar{} Triangular axis condition should yield lower response
    times on the first question (both overall, and specifically among
    participants who answer correctly)
  \end{itemize}
\item
  H2 \textbar{} Addition of gridlines inside the orthogonal axes does
  not change performance

  \begin{itemize}
  \tightlist
  \item
    H2A \textbar{} No difference in overall performance between
    Orth-Full, Orth-Sparse and Orth-Grid
  \item
    H2B \textbar{} No difference in first question score between
    Orth-Full, Orth-Sparse and Orth-Grid
  \item
    H2C \textbar{} No difference in response latency between Orth-Full,
    Orth-Sparse and Orth-Grid
  \end{itemize}
\end{itemize}

\textbf{Exploratory Questions}

\begin{itemize}
\tightlist
\item
  mouse path traversal: Does axes design yield different exploration
  pattern with the mouse?
\end{itemize}

\hypertarget{methods-2}{%
\section{METHODS}\label{methods-2}}

\hypertarget{design-2}{%
\subsection{Design}\label{design-2}}

We employed a mixed design with 1 between-subjects factor with 4 levels
(Graphical Framework: ORTH-FULL, ORTH-SPARSE, ORTH-GRID, TRI-SPARSE) and
15 items (within-subjects factor).

Independent Variables:

\begin{itemize}
\tightlist
\item
  B-S (Graphical Framework: ORTH-FULL, ORTH-SPARSE, ORTH-GRID,
  TRI-SPARSE)
\item
  W-S (Item x 15)
\end{itemize}

Dependent Variables:

\begin{itemize}
\tightlist
\item
  Response Accuracy : Is the response triangular-correct?
\item
  Response Interpretation : (derived) With which interpretation of the
  graph is the subject's response on an individual question consistent?
\item
  Response Latency : Time from stimulus onset to clicking `Submit'
  button: time in (s)
\end{itemize}

\hypertarget{materials-2}{%
\subsection{Materials}\label{materials-2}}

Stimuli consisted of a series of 15 graph comprehension questions, each
testing a different combination of time interval relations, to be read
from a Triangular-Model graph. Figure~\ref{fig-sample}. The list of
questions can be found \href{utils/stimuli/sgcx_questions.csv}{here}.

\begin{figure}

{\centering \includegraphics{analysis/utils/img/sample_graphComprehensionTask.png}

}

\caption{\label{fig-sample}Sample Question (Q=1) for Graph Comprehension
Task}

\end{figure}

\hypertarget{procedure-2}{%
\subsection{Procedure}\label{procedure-2}}

Participants completed the study via a web-browser.

(1) Upon starting, they submitted informed consent, before reading task
instructions.

(2) Participants were introduced to a scenario in which they were to
play the role of a project manager, scheduling shifts for a group of
employees. The schedule of the employees was presented in a
TriangularModel (TM) graph, and they would be answering question about
the schedule.

(3) Then participants completed an experimental block of 15 items : the
Graph Comprehension Task

(4) Following the experimental block, participants answered a
free-response question about their strategy for reading the graph,
followed by a demographic questionnaire and debrief.

\hypertarget{sample-3}{%
\subsection{Sample}\label{sample-3}}

Data were collected by convenience sample of a university subject pool
during the winter of 2022. Participants accessed the study via a web
browser (asynchronously). The stimulus application required the
participant stay in full-screen mode for the entirety of the study.

\hypertarget{analysis-2}{%
\section{ANALYSIS}\label{analysis-2}}

\hypertarget{sec-SGC4A-harmonize}{%
\subsection{Data Preparation}\label{sec-SGC4A-harmonize}}

Data were collected via a custom web application and stored in a NoSQL
database. The following exclusion criteria were applied during data
cleaning:

\begin{itemize}
\tightlist
\item
  completion status : ``success'' ; subject must have finished all parts
  of the study, including demographic questionnaire
\item
  session ID: {[}in list{]} ; subject must have been assigned to valid
  data collection session (discard testing and piloting data)
\item
  browser interaction violations \textless{} 3; subject must have fewer
  than 3 violations of non-allowed browser interactions (i.e.~resizing
  window, leaving browser tab or leaving fullscreen mode)
\item
  self-rated effort \textgreater{} 2; subjects who reported, ``not
  trying hard/rushing through questions'' or ``started out trying hard
  but giving up at some point'' were excluded from analysis.
\item
  attention check ==TRUE ; subjects who failed to answer a mid-study
  attention check question (Graph Comprehension Task Question \#6) are
  excluded
\end{itemize}

\begin{longtable}[]{@{}ll@{}}
\toprule()
Pre-Requisite & Followed By \\
\midrule()
\endhead
winter2022\_clean\_sgc4a.Rmd & 2\_sgc4A\_scoring.qmd \\
\bottomrule()
\end{longtable}

The underlying data structure of the stimulus web application changed
across the data collection period, resulting in slightly different data
files (i.e.~columns are not named consistently). In this section, we
combine the files from each data collection period into a single
\emph{harmonized} data file for analysis (one for participants, one for
items).

\hypertarget{participants-3}{%
\subsubsection{Participants}\label{participants-3}}

First we import participant-level data, selecting only the columns
relevant for analysis. The result is a single data frame
\texttt{df\_subjects} containing one row for each subject (across all
periods). Note that we \emph{are not} discarding any \emph{response}
data. Rather, we discard columns that are automatically recorded by the
stimulus web application and help the application run.

\emph{Note that we discard some columns representing scores calculated
in the stimulus engine. These scores were calculated differently across
collection periods, and so we discard them and recalculate scores in the
next analysis notebook. No raw data (responses and response times) are
discarded, only algorithmically-derived scores for the responses.}

\begin{Shaded}
\begin{Highlighting}[]
\CommentTok{\#IMPORT PARTICIPANT DATA}

\CommentTok{\# }\AlertTok{HACK}\CommentTok{ WD FOR LOCAL RUNNING?}
\CommentTok{\# imac = "/Users/amyraefox/Code/SGC{-}Scaffolding\_Graph\_Comprehension/SGC{-}X/ANALYSIS/MAIN"}
\CommentTok{\# mbp = "/Users/amyfox/Sites/RESEARCH/SGC—Scaffolding Graph Comprehension/SGC{-}X/ANALYSIS/MAIN"}
\CommentTok{\# setwd(mbp)}

\CommentTok{\#import file}
\NormalTok{df\_subjects }\OtherTok{\textless{}{-}} \FunctionTok{read\_rds}\NormalTok{(}\StringTok{"analysis/SGC4A/data/0{-}study{-}level/sgc4a\_participants.rds"}\NormalTok{) }\CommentTok{\#use RDS file as it contains metadata}

\CommentTok{\#save \textquotesingle{}explanation\textquotesingle{} columns from winter22, which is actually a response to a free response item (Q16); was recorded with item\_level data in old webapp}
\NormalTok{df\_q16 }\OtherTok{\textless{}{-}}\NormalTok{ df\_subjects }\SpecialCharTok{\%\textgreater{}\%} 
  \FunctionTok{select}\NormalTok{(subject, condition, term , mode, explanation) }\SpecialCharTok{\%\textgreater{}\%} 
  \FunctionTok{mutate}\NormalTok{(}
    \AttributeTok{q =} \DecValTok{16}\NormalTok{,}
    \AttributeTok{response =}\NormalTok{ explanation}
\NormalTok{  ) }\SpecialCharTok{\%\textgreater{}\%} \FunctionTok{select}\NormalTok{(}\SpecialCharTok{{-}}\NormalTok{explanation)}

\CommentTok{\#reduce data collected using NEW webapp to useful columns}
\NormalTok{df\_subjects }\OtherTok{\textless{}{-}}\NormalTok{ df\_subjects }\SpecialCharTok{\%\textgreater{}\%} 
  \FunctionTok{mutate}\NormalTok{(}\AttributeTok{score =}\NormalTok{ absolute\_score) }\SpecialCharTok{\%\textgreater{}\%} 
  \CommentTok{\#select only columns we\textquotesingle{}ll be analyzing, discard others}
\NormalTok{  dplyr}\SpecialCharTok{::}\FunctionTok{select}\NormalTok{( subject, condition, pretty\_condition, term, mode, }
                 \CommentTok{\#demographics}
\NormalTok{                 gender, age, language, schoolyear, country,}
                 \CommentTok{\#effort survey}
\NormalTok{                 effort, difficulty, confidence, enjoyment, }
                 \CommentTok{\#explanations}
\NormalTok{                 other,disability,}
                 \CommentTok{\#response characteristics}
\NormalTok{                 totaltime\_m,}
                 \CommentTok{\#exploratory factors}
\NormalTok{                 violations, browser, width, height}
\NormalTok{                 )}


\NormalTok{effort\_labels }\OtherTok{\textless{}{-}} \FunctionTok{c}\NormalTok{(}\StringTok{"I tried my best on each question"}\NormalTok{, }\StringTok{"I tried my best on most questions"}\NormalTok{)}

\CommentTok{\#set factors}
\NormalTok{df\_subjects }\OtherTok{\textless{}{-}}\NormalTok{ df\_subjects }\SpecialCharTok{\%\textgreater{}\%} 
  \CommentTok{\#refactor factors}
  \FunctionTok{mutate}\NormalTok{ (}
    \AttributeTok{subject =} \FunctionTok{factor}\NormalTok{(subject),}
    \AttributeTok{condition =} \FunctionTok{factor}\NormalTok{(condition),}
    \AttributeTok{term =} \FunctionTok{factor}\NormalTok{(term),}
    \AttributeTok{mode =} \FunctionTok{factor}\NormalTok{(mode),}
    \AttributeTok{gender =} \FunctionTok{factor}\NormalTok{(gender),}
    \AttributeTok{schoolyear =} \FunctionTok{factor}\NormalTok{(schoolyear, }\AttributeTok{levels=}\FunctionTok{c}\NormalTok{(}\StringTok{"First"}\NormalTok{,}\StringTok{"Second"}\NormalTok{,}\StringTok{"Third"}\NormalTok{,}\StringTok{"Fourth"}\NormalTok{,}\StringTok{"Fifth"}\NormalTok{,}\StringTok{"Other"}\NormalTok{))}
\NormalTok{  )}
\end{Highlighting}
\end{Shaded}

\hypertarget{items-2}{%
\subsubsection{Items}\label{items-2}}

Next we import item-level data from each data collection period,
selecting only the columns relevant for analysis. The result is a single
data frame \texttt{df\_items} containing one row for each \emph{graph
comprehension task question} (qs=15) (across all periods). A second data
frame \texttt{df\_freeresponse} contains one row for each free response
strategy question (last question posed to participants in Winter2022)
Note that we \emph{do not} discard any \emph{response} data. Rather, we
\emph{do} discard several columns representing accuracy scores for
responses that were calculated in the stimulus engine. These scores were
calculated differently across collection periods, and so we discard them
and recalculate scores in the next analysis notebook. Original response
data are always preserved.

\begin{Shaded}
\begin{Highlighting}[]
\CommentTok{\# }\AlertTok{HACK}\CommentTok{ WD FOR LOCAL RUNNING?}
\CommentTok{\# imac = "/Users/amyraefox/Code/SGC{-}Scaffolding\_Graph\_Comprehension/SGC{-}X/ANALYSIS/MAIN"}
\CommentTok{\#mbp = "/Users/amyfox/Sites/RESEARCH/SGC—Scaffolding Graph Comprehension/SGC{-}X/ANALYSIS/MAIN"}
\CommentTok{\#setwd(mbp)}

\CommentTok{\#read datafiles}
\NormalTok{df\_items }\OtherTok{\textless{}{-}} \FunctionTok{read\_rds}\NormalTok{(}\StringTok{"analysis/SGC4A/data/0{-}study{-}level/sgc4a\_items.rds"}\NormalTok{) }\CommentTok{\#use RDS file as it contains metadata}

\CommentTok{\#reduce data collected using new webapp}
\NormalTok{df\_items }\OtherTok{\textless{}{-}}\NormalTok{ df\_items }\SpecialCharTok{\%\textgreater{}\%} 
  \FunctionTok{select}\NormalTok{(subject, condition, pretty\_condition, term, mode, question, q, answer, correct, rt\_s) }\SpecialCharTok{\%\textgreater{}\%} \CommentTok{\#unfactor before combine}
  \FunctionTok{mutate}\NormalTok{(}
    \AttributeTok{subject =} \FunctionTok{as.character}\NormalTok{(subject),}
    \AttributeTok{condition =} \FunctionTok{as.character}\NormalTok{(condition),}
    \AttributeTok{term =} \FunctionTok{as.character}\NormalTok{(term),}
    \AttributeTok{mode =} \FunctionTok{as.character}\NormalTok{(mode),}
    \AttributeTok{q =} \FunctionTok{as.integer}\NormalTok{(q),}
    \AttributeTok{correct =} \FunctionTok{as.logical}\NormalTok{(correct)}
\NormalTok{  ) }\SpecialCharTok{\%\textgreater{}\%} 
  \FunctionTok{mutate}\NormalTok{(}
    \AttributeTok{response =} \FunctionTok{str\_remove\_all}\NormalTok{(}\FunctionTok{as.character}\NormalTok{(answer), }\StringTok{","}\NormalTok{),}
    \AttributeTok{num\_o =} \FunctionTok{str\_length}\NormalTok{(response)}
\NormalTok{  ) }\SpecialCharTok{\%\textgreater{}\%} 
  \CommentTok{\# handle NA values (why are some empty responses blank and others NA?) }
  \FunctionTok{mutate}\NormalTok{(}
    \AttributeTok{response =} \FunctionTok{replace\_na}\NormalTok{(response, }\StringTok{""}\NormalTok{),}
    \AttributeTok{num\_o =} \FunctionTok{replace\_na}\NormalTok{(num\_o, }\DecValTok{0}\NormalTok{)}
\NormalTok{  )}
\end{Highlighting}
\end{Shaded}

\hypertarget{validation-2}{%
\subsubsection{Validation}\label{validation-2}}

Next, we validate that we have the complete number of item-level records
based on the number of subject-level records

\begin{Shaded}
\begin{Highlighting}[]
\CommentTok{\#the number of items should be equal to 15 x the number of subjects}
\FunctionTok{nrow}\NormalTok{(df\_items) }\SpecialCharTok{==} \DecValTok{15}\SpecialCharTok{*} \FunctionTok{nrow}\NormalTok{(df\_subjects) }\CommentTok{\#TRUE}
\end{Highlighting}
\end{Shaded}

\begin{verbatim}
[1] TRUE
\end{verbatim}

\begin{Shaded}
\begin{Highlighting}[]
\CommentTok{\#each subject should have 15 items}
\NormalTok{df\_items }\SpecialCharTok{\%\textgreater{}\%} \FunctionTok{group\_by}\NormalTok{(subject) }\SpecialCharTok{\%\textgreater{}\%} \FunctionTok{summarise}\NormalTok{(}\AttributeTok{n =} \FunctionTok{n}\NormalTok{()) }\SpecialCharTok{\%\textgreater{}\%} \FunctionTok{filter}\NormalTok{(n }\SpecialCharTok{!=} \DecValTok{15}\NormalTok{) }\SpecialCharTok{\%\textgreater{}\%} \FunctionTok{nrow}\NormalTok{() }\SpecialCharTok{==} \DecValTok{0}
\end{Highlighting}
\end{Shaded}

\begin{verbatim}
[1] TRUE
\end{verbatim}

\hypertarget{export-4}{%
\subsubsection{Export}\label{export-4}}

Finally, we export the (session-harmonized) data for analysis, as CSVs,
and .RDS (includes metadata)

\begin{Shaded}
\begin{Highlighting}[]
\CommentTok{\# }\AlertTok{HACK}\CommentTok{ WD FOR LOCAL RUNNING?}
\CommentTok{\# imac = "/Users/amyraefox/Code/SGC{-}Scaffolding\_Graph\_Comprehension/SGC{-}X/ANALYSIS/MAIN"}
\CommentTok{\# mbp = "/Users/amyfox/Sites/RESEARCH/SGC—Scaffolding Graph Comprehension/SGC{-}X/ANALYSIS/MAIN"}
\CommentTok{\# setwd(mbp)}

\CommentTok{\#SAVE FILES}
\FunctionTok{write.csv}\NormalTok{(df\_subjects,}\StringTok{"analysis/SGC4A/data/1{-}study{-}level/sgc4a\_participants.csv"}\NormalTok{, }\AttributeTok{row.names =} \ConstantTok{FALSE}\NormalTok{)}
\FunctionTok{write.csv}\NormalTok{(df\_items,}\StringTok{"analysis/SGC4A/data/1{-}study{-}level/sgc4a\_items.csv"}\NormalTok{, }\AttributeTok{row.names =} \ConstantTok{FALSE}\NormalTok{)}
\FunctionTok{write.csv}\NormalTok{(df\_q16,}\StringTok{"analysis/SGC4A/data/1{-}study{-}level/sgc4a\_freeresponse.csv"}\NormalTok{, }\AttributeTok{row.names =} \ConstantTok{FALSE}\NormalTok{)}

\CommentTok{\#SAVE R Data Structures }
\CommentTok{\#export R DATA STRUCTURES (include codebook metadata)}
\NormalTok{rio}\SpecialCharTok{::}\FunctionTok{export}\NormalTok{(df\_subjects, }\StringTok{"analysis/SGC4A/data/1{-}study{-}level/sgc4a\_participants.rds"}\NormalTok{) }\CommentTok{\# to R data structure file}
\NormalTok{rio}\SpecialCharTok{::}\FunctionTok{export}\NormalTok{(df\_items, }\StringTok{"analysis/SGC4A/data/1{-}study{-}level/sgc4a\_items.rds"}\NormalTok{) }\CommentTok{\# to R data structure file}
\end{Highlighting}
\end{Shaded}

\hypertarget{resources-7}{%
\section{RESOURCES}\label{resources-7}}

\begin{Shaded}
\begin{Highlighting}[]
\FunctionTok{sessionInfo}\NormalTok{()}
\end{Highlighting}
\end{Shaded}

\begin{verbatim}
R version 4.2.1 (2022-06-23)
Platform: x86_64-apple-darwin17.0 (64-bit)
Running under: macOS Big Sur ... 10.16

Matrix products: default
BLAS:   /Library/Frameworks/R.framework/Versions/4.2/Resources/lib/libRblas.0.dylib
LAPACK: /Library/Frameworks/R.framework/Versions/4.2/Resources/lib/libRlapack.dylib

locale:
[1] en_US.UTF-8/en_US.UTF-8/en_US.UTF-8/C/en_US.UTF-8/en_US.UTF-8

attached base packages:
[1] stats     graphics  grDevices utils     datasets  methods   base     

other attached packages:
 [1] kableExtra_1.3.4 forcats_0.5.1    stringr_1.4.0    dplyr_1.0.9     
 [5] purrr_0.3.4      readr_2.1.2      tidyr_1.2.0      tibble_3.1.7    
 [9] ggplot2_3.3.6    tidyverse_1.3.1  codebook_0.9.2  

loaded via a namespace (and not attached):
 [1] Rcpp_1.0.8.3      svglite_2.1.0     lubridate_1.8.0   assertthat_0.2.1 
 [5] digest_0.6.29     utf8_1.2.2        R6_2.5.1          cellranger_1.1.0 
 [9] backports_1.4.1   reprex_2.0.1      labelled_2.9.1    evaluate_0.15    
[13] httr_1.4.3        pillar_1.7.0      rlang_1.0.3       curl_4.3.2       
[17] readxl_1.4.0      data.table_1.14.2 rstudioapi_0.13   rmarkdown_2.14   
[21] webshot_0.5.3     foreign_0.8-82    munsell_0.5.0     broom_0.8.0      
[25] compiler_4.2.1    modelr_0.1.8      xfun_0.31         pkgconfig_2.0.3  
[29] systemfonts_1.0.4 htmltools_0.5.2   tidyselect_1.1.2  rio_0.5.29       
[33] codetools_0.2-18  fansi_1.0.3       viridisLite_0.4.0 crayon_1.5.1     
[37] tzdb_0.3.0        dbplyr_2.2.1      withr_2.5.0       grid_4.2.1       
[41] jsonlite_1.8.0    gtable_0.3.0      lifecycle_1.0.1   DBI_1.1.3        
[45] magrittr_2.0.3    scales_1.2.0      zip_2.2.0         cli_3.3.0        
[49] stringi_1.7.6     fs_1.5.2          xml2_1.3.3        ellipsis_0.3.2   
[53] generics_0.1.2    vctrs_0.4.1       openxlsx_4.2.5    tools_4.2.1      
[57] glue_1.6.2        hms_1.1.1         fastmap_1.1.0     yaml_2.3.5       
[61] colorspace_2.0-3  rvest_1.0.2       knitr_1.39        haven_2.5.0      
\end{verbatim}

\newpage

\hypertarget{sec-SGC4A-scoring}{%
\chapter{Response Scoring}\label{sec-SGC4A-scoring}}

\emph{The purpose of this notebook is to score (assign a measure of
accuracy) to response data for the SGC4A study. This is required because
the question type on the graph comprehension task used a `Multiple
Response' (MR) question design. Here, we evaluate different approaches
to scoring multiple response questions, and transform raw item responses
(e.g.~boxes ABC are checked) to a measure of response accuracy.
(Warning: this notebook takes several minutes to execute.)} To review
the strategy behind Multiple Response scoring for the SGC project, refer
to section \textbf{?@sec-scoring}.

\begin{Shaded}
\begin{Highlighting}[]
\FunctionTok{options}\NormalTok{(}\AttributeTok{scipen=}\DecValTok{1}\NormalTok{, }\AttributeTok{digits=}\DecValTok{3}\NormalTok{)}

\FunctionTok{library}\NormalTok{(kableExtra) }\CommentTok{\#printing tables }
\FunctionTok{library}\NormalTok{(ggformula) }\CommentTok{\#quick graphs}
\FunctionTok{library}\NormalTok{(pbapply) }\CommentTok{\#progress bar and time estimate for *apply fns}
\FunctionTok{library}\NormalTok{(Hmisc) }\CommentTok{\# \%nin\% operator}
\FunctionTok{library}\NormalTok{(tidyverse) }\CommentTok{\#ALL THE THINGS}
\end{Highlighting}
\end{Shaded}

\hypertarget{score-sgc-data-2}{%
\section{SCORE SGC DATA}\label{score-sgc-data-2}}

To review the strategy behind Multiple Response scoring for the SGC
project, refer to section \textbf{?@sec-scoring}.

In SGC we are fundamentally interested in understanding how a
participant interprets the presented graph (stimulus). The \textbf{graph
comprehension task} asks them to select the data points in the graph
that meet the criteria posed in the question. To assess a participant's
performance, for each question (q=15) we will calculate the following
scores:

\emph{An overall, strict score:}\\
1. \textbf{Absolute Score} : using dichotomous scoring referencing true
(Triangular) answer. (see 1.2)

\emph{Sub-scores, for each alternative graph interpretation}\\
2. \textbf{Triangular Score} : using partial scoring {[}-1/q, +1/p{]}
referencing true (Triangular) answer key.

3. \textbf{Orthogonal Score} : using partial scoring {[}-1/q, +1/p{]}
referencing (incorrect Orthogonal) answer key.

Based on prior observational studies where we observed emergence of
other alternative interpretations (i.e.~transitional interpretations) we
also calculate subscores for these alternatives.

4. \textbf{Tversky Score} : using partial scoring {[}-1/q, +1/p{]}
referencing (incorrect connecting-lines strategy) answer key. 5.
\textbf{Satisficing Score} : using partial scoring {[}-1/q, +1/p{]}
referencing (incorrect satisficing strategy) answer key.

\hypertarget{sec-SGC4A-keys}{%
\subsection{Prepare Answer Keys}\label{sec-SGC4A-keys}}

We start by importing three answer keys: (1) Q1 - Q5 {[}control
condition{]}, (2) Q1-Q5 {[}impasse condition{]}, (3) Q6-15. Separate
answer keys by condition are required for Q1-Q5 because the stimuli for
each condition visualize a different underlying dataset (i.e.~the graphs
show datapoints in different positions). Q6-Q15 are identical across
conditions. Each answer key includes a row for each question, and a
column defining the subset of response options that correspond to
different graph interpretations.

\begin{Shaded}
\begin{Highlighting}[]
\CommentTok{\# \#HACK WD FOR LOCAL RUNNING?}
\NormalTok{imac }\OtherTok{=} \StringTok{"/Users/amyraefox/Code/SGC{-}Scaffolding\_Graph\_Comprehension/SGC{-}X/ANALYSIS/MAIN"}
\FunctionTok{setwd}\NormalTok{(imac)}

\CommentTok{\#SAVE KEYS FOR FUTURE USE}
\NormalTok{keys\_raw }\OtherTok{\textless{}{-}}  \FunctionTok{read\_csv}\NormalTok{(}\StringTok{"analysis/utils/keys/parsed\_keys/keys\_raw"}\NormalTok{)}
\NormalTok{keys\_orth }\OtherTok{\textless{}{-}}  \FunctionTok{read\_csv}\NormalTok{(}\StringTok{"analysis/utils/keys/parsed\_keys/keys\_orth"}\NormalTok{)}
\NormalTok{keys\_tri }\OtherTok{\textless{}{-}}  \FunctionTok{read\_csv}\NormalTok{(}\StringTok{"analysis/utils/keys/parsed\_keys/keys\_tri"}\NormalTok{)}
\NormalTok{keys\_satisfice\_left }\OtherTok{\textless{}{-}}  \FunctionTok{read\_csv}\NormalTok{(}\StringTok{"analysis/utils/keys/parsed\_keys/keys\_satisfice\_left"}\NormalTok{)}
\NormalTok{keys\_satisfice\_right }\OtherTok{\textless{}{-}}  \FunctionTok{read\_csv}\NormalTok{(}\StringTok{"analysis/utils/keys/parsed\_keys/keys\_satisfice\_right"}\NormalTok{)}
\NormalTok{keys\_tversky\_duration }\OtherTok{\textless{}{-}}  \FunctionTok{read\_csv}\NormalTok{(}\StringTok{"analysis/utils/keys/parsed\_keys/keys\_tversky\_duration"}\NormalTok{)}
\NormalTok{keys\_tversky\_end }\OtherTok{\textless{}{-}}  \FunctionTok{read\_csv}\NormalTok{(}\StringTok{"analysis/utils/keys/parsed\_keys/keys\_tversky\_end"}\NormalTok{)}
\NormalTok{keys\_tversky\_max }\OtherTok{\textless{}{-}}  \FunctionTok{read\_csv}\NormalTok{(}\StringTok{"analysis/utils/keys/parsed\_keys/keys\_tversky\_max"}\NormalTok{)}
\NormalTok{keys\_tversky\_start }\OtherTok{\textless{}{-}}  \FunctionTok{read\_csv}\NormalTok{(}\StringTok{"analysis/utils/keys/parsed\_keys/keys\_tversky\_start"}\NormalTok{)}
\end{Highlighting}
\end{Shaded}

\hypertarget{sec-SGC4A-subscores}{%
\subsection{Calculate Subscores}\label{sec-SGC4A-subscores}}

Next, we import the item-level response data. For each row in the item
level dataset (indicating the response to a single question-item for a
single subject), we compare the raw response
\texttt{df\_items\$response} with the answer keys in each interpretation
(e.g.~\texttt{keys\_orth}, \texttt{keys\_tri}, etc\ldots), then using
those sets, determine the number of correctly selected items(p) and
incorrectly selected items (q), which in turn are used to calculate
partial{[}-1/q, +1/p{]} scores for each interpretation. The resulting
scores are then stored on each item in \texttt{df\_items}, and can be
used to determine which graph interpretation the subject held.

Specifically, the following scores are calculated for each item:

\textbf{Interpretation Subscores}

\begin{itemize}
\tightlist
\item
  \texttt{score\_TRI} How consistent is the response with the
  \textbf{triangular}interpretation?
\item
  \texttt{score\_ORTH} How consistent is the response with the
  \textbf{orthogonal}interpretation?
\item
  \texttt{score\_SATISFICE} is calculated by taking the maximum value of
  :

  \begin{itemize}
  \tightlist
  \item
    \texttt{score\_SAT\_left} How consistent is the response with the
    \textbf{(left side) Satisficing} interpretation?
  \item
    \texttt{score\_SAT\_right} How consistent is the response with the
    \textbf{(right side) Satisficing} interpretation
  \end{itemize}
\item
  \texttt{score\_TVERSKY} is calculated by taking the maximum value of:

  \begin{itemize}
  \tightlist
  \item
    \texttt{score\_TV\_max} How consistent is the response with the
    \textbf{(maximal) Tversky} interpretation?
  \item
    \texttt{score\_TV\_start} How consistent is the response with the
    \textbf{(start-time) Tversky} interpretation?
  \item
    \texttt{score\_TV\_end} How consistent is the response with the
    \textbf{(end-time) Tversky} interpretation?
  \item
    \texttt{score\_TV\_duration} How consistent is the response with the
    \textbf{(duration) Tversky} interpretation?
  \end{itemize}
\item
  \texttt{score\_REF} Did the response select only the \textbf{reference
  point}?
\item
  \texttt{score\_BOTH} How consistent is the response with \textbf{both}
  the orthogonal and triangular interpretations?
\end{itemize}

\textbf{Absolute Scores}

\begin{itemize}
\tightlist
\item
  \texttt{score\_ABS} Is the response strictly correct? (triangular
  interpretation)
\item
  \texttt{score\_niceABS} Is the response strictly correct? (triangular
  interpretation, not penalizing ref points). This is a more generous
  version of the Absolute score that does not penalize the participant
  if in addition to the correct answer \emph{in addition to} they also
  select the data point referenced in the question.
\end{itemize}

\begin{Shaded}
\begin{Highlighting}[]
\CommentTok{\#HACK WD FOR LOCAL RUNNING?}
\NormalTok{imac }\OtherTok{=} \StringTok{"/Users/amyraefox/Code/SGC{-}Scaffolding\_Graph\_Comprehension/SGC{-}X/ANALYSIS/MAIN"}
\FunctionTok{setwd}\NormalTok{(imac)}

\CommentTok{\#backup \textless{}{-} read\_rds(\textquotesingle{}analysis/SGC4A/data/1{-}study{-}level/sgc4a\_items.rds\textquotesingle{}) \#for troubleshooting only}
\NormalTok{df\_items }\OtherTok{\textless{}{-}} \FunctionTok{read\_rds}\NormalTok{(}\StringTok{\textquotesingle{}analysis/SGC4A/data/1{-}study{-}level/sgc4a\_items.rds\textquotesingle{}}\NormalTok{)}
\end{Highlighting}
\end{Shaded}

\begin{Shaded}
\begin{Highlighting}[]
\CommentTok{\# \#HACK WD FOR LOCAL RUNNING?}
\NormalTok{imac }\OtherTok{=} \StringTok{"/Users/amyraefox/Code/SGC{-}Scaffolding\_Graph\_Comprehension/SGC{-}X/ANALYSIS/MAIN"}
\FunctionTok{setwd}\NormalTok{(imac)}

\FunctionTok{source}\NormalTok{(}\StringTok{"analysis/utils/scoring.R"}\NormalTok{)}
\end{Highlighting}
\end{Shaded}

\emph{note: this cell takes approximately 30 minutes to run on the full
df\_items dataframe with 4950 records}

\begin{Shaded}
\begin{Highlighting}[]
\CommentTok{\#RUN THIS \textless{}OR\textgreater{} THE CALCULATE{-}SCORES{-}FORLOOP [not both]}

\CommentTok{\#ALPHEBETIZE RESPONSE}
\NormalTok{df\_items}\SpecialCharTok{$}\NormalTok{response }\OtherTok{=} \FunctionTok{pbmapply}\NormalTok{(reorder\_inplace, df\_items}\SpecialCharTok{$}\NormalTok{response)}

\CommentTok{\#STRATEGY PARTIAL{-}SUBSCORES}
\NormalTok{df\_items}\SpecialCharTok{$}\NormalTok{score\_TRI }\OtherTok{=} \FunctionTok{pbmapply}\NormalTok{(calc\_subscore, df\_items}\SpecialCharTok{$}\NormalTok{q, df\_items}\SpecialCharTok{$}\NormalTok{condition, df\_items}\SpecialCharTok{$}\NormalTok{response, }\AttributeTok{MoreArgs =} \FunctionTok{list}\NormalTok{(}\AttributeTok{keyframe =}\NormalTok{ keys\_tri))}
\NormalTok{df\_items}\SpecialCharTok{$}\NormalTok{score\_ORTH }\OtherTok{=} \FunctionTok{pbmapply}\NormalTok{(calc\_subscore, df\_items}\SpecialCharTok{$}\NormalTok{q, df\_items}\SpecialCharTok{$}\NormalTok{condition, df\_items}\SpecialCharTok{$}\NormalTok{response, }\AttributeTok{MoreArgs =} \FunctionTok{list}\NormalTok{(}\AttributeTok{keyframe =}\NormalTok{ keys\_orth))}
\NormalTok{df\_items}\SpecialCharTok{$}\NormalTok{score\_SAT\_left }\OtherTok{=} \FunctionTok{pbmapply}\NormalTok{(calc\_subscore, df\_items}\SpecialCharTok{$}\NormalTok{q, df\_items}\SpecialCharTok{$}\NormalTok{condition, df\_items}\SpecialCharTok{$}\NormalTok{response, }\AttributeTok{MoreArgs =} \FunctionTok{list}\NormalTok{(}\AttributeTok{keyframe =}\NormalTok{ keys\_satisfice\_left))}
\NormalTok{df\_items}\SpecialCharTok{$}\NormalTok{score\_SAT\_right }\OtherTok{=} \FunctionTok{pbmapply}\NormalTok{(calc\_subscore, df\_items}\SpecialCharTok{$}\NormalTok{q, df\_items}\SpecialCharTok{$}\NormalTok{condition, df\_items}\SpecialCharTok{$}\NormalTok{response, }\AttributeTok{MoreArgs =} \FunctionTok{list}\NormalTok{(}\AttributeTok{keyframe =}\NormalTok{ keys\_satisfice\_right))}
\NormalTok{df\_items}\SpecialCharTok{$}\NormalTok{score\_TV\_max }\OtherTok{=} \FunctionTok{pbmapply}\NormalTok{(calc\_subscore, df\_items}\SpecialCharTok{$}\NormalTok{q, df\_items}\SpecialCharTok{$}\NormalTok{condition, df\_items}\SpecialCharTok{$}\NormalTok{response, }\AttributeTok{MoreArgs =} \FunctionTok{list}\NormalTok{(}\AttributeTok{keyframe =}\NormalTok{ keys\_tversky\_max))}
\NormalTok{df\_items}\SpecialCharTok{$}\NormalTok{score\_TV\_start }\OtherTok{=} \FunctionTok{pbmapply}\NormalTok{(calc\_subscore, df\_items}\SpecialCharTok{$}\NormalTok{q, df\_items}\SpecialCharTok{$}\NormalTok{condition, df\_items}\SpecialCharTok{$}\NormalTok{response, }\AttributeTok{MoreArgs =} \FunctionTok{list}\NormalTok{(}\AttributeTok{keyframe =}\NormalTok{ keys\_tversky\_start))}
\NormalTok{df\_items}\SpecialCharTok{$}\NormalTok{score\_TV\_end }\OtherTok{=} \FunctionTok{pbmapply}\NormalTok{(calc\_subscore, df\_items}\SpecialCharTok{$}\NormalTok{q, df\_items}\SpecialCharTok{$}\NormalTok{condition, df\_items}\SpecialCharTok{$}\NormalTok{response, }\AttributeTok{MoreArgs =} \FunctionTok{list}\NormalTok{(}\AttributeTok{keyframe =}\NormalTok{ keys\_tversky\_end))}
\NormalTok{df\_items}\SpecialCharTok{$}\NormalTok{score\_TV\_duration }\OtherTok{=} \FunctionTok{pbmapply}\NormalTok{(calc\_subscore, df\_items}\SpecialCharTok{$}\NormalTok{q, df\_items}\SpecialCharTok{$}\NormalTok{condition, df\_items}\SpecialCharTok{$}\NormalTok{response, }\AttributeTok{MoreArgs =} \FunctionTok{list}\NormalTok{(}\AttributeTok{keyframe =}\NormalTok{ keys\_tversky\_duration))}

\CommentTok{\#SPECIAL ABSOLUTE SUBSCORES}
\NormalTok{df\_items}\SpecialCharTok{$}\NormalTok{score\_REF }\OtherTok{=} \FunctionTok{pbmapply}\NormalTok{(calc\_refscore, df\_items}\SpecialCharTok{$}\NormalTok{q, df\_items}\SpecialCharTok{$}\NormalTok{response)}
\NormalTok{df\_items}\SpecialCharTok{$}\NormalTok{score\_BOTH }\OtherTok{=} \FunctionTok{as.integer}\NormalTok{((df\_items}\SpecialCharTok{$}\NormalTok{score\_TRI }\SpecialCharTok{==} \DecValTok{1}\NormalTok{) }\SpecialCharTok{\&}\NormalTok{ (df\_items}\SpecialCharTok{$}\NormalTok{score\_ORTH }\SpecialCharTok{==}\DecValTok{1}\NormalTok{))}

\CommentTok{\#ABSOLUTE SCORES}
\NormalTok{df\_items}\SpecialCharTok{$}\NormalTok{score\_ABS }\OtherTok{=} \FunctionTok{as.integer}\NormalTok{(df\_items}\SpecialCharTok{$}\NormalTok{correct) }
\NormalTok{df\_items}\SpecialCharTok{$}\NormalTok{score\_niceABS  }\OtherTok{\textless{}{-}} \FunctionTok{as.integer}\NormalTok{((df\_items}\SpecialCharTok{$}\NormalTok{score\_TRI }\SpecialCharTok{==} \DecValTok{1}\NormalTok{)) }\CommentTok{\#tri doesn\textquotesingle{}t penalize ref or ve{-}area}
\end{Highlighting}
\end{Shaded}

\hypertarget{sec-SGC4A-interpretation}{%
\subsection{Derive Interpretation}\label{sec-SGC4A-interpretation}}

Finally, we use the interpretation subscores to classify the response as
a particular interpretation. This classification algorithm : (1) First
decides if the response matches one or more `special' situations (blank
response, reference point response, both ORTH and TRI) (2) If response
doesn't match a special situation, it compares the individual subscores,
and subscores and decides if they are \emph{discriminant} (i.e.~are the
scores different enough to make a prediction). A discriminant threshold
of 0.5pts (on a scale from -1 to +1 is used) (2) If the variance in
subscores surpasses the threshold, the interpretation is classified
based on the highest subscore (TRIANGULAR, ORTHOGONAL, TVERSKY,
SATISFICE) (3) If the variance does not surpass the threshold, the
interpretation is labelled as ``?'', indicating it cannot be classified,
and is of an unknown interpretation.

The final output is called \texttt{interpretation}.

\begin{Shaded}
\begin{Highlighting}[]
\CommentTok{\#stoopid extra copying for troubleshooting safety}
\NormalTok{temp }\OtherTok{\textless{}{-}}\NormalTok{ df\_items }
\NormalTok{temp }\OtherTok{\textless{}{-}} \FunctionTok{derive\_interpretation}\NormalTok{(temp)}
\NormalTok{df\_items }\OtherTok{\textless{}{-}}\NormalTok{ temp }
\end{Highlighting}
\end{Shaded}

\hypertarget{sec-SGC4A-scaledScore}{%
\subsection{Derive Scaled Score}\label{sec-SGC4A-scaledScore}}

The \texttt{interpretation} response variable gives us the finest grain
indication of the reader's understanding of the graph for a particular
question. However, it is a categorical variable, which poses a challenge
for analyzing cumulative performance at the subject level. To address
this challenge, we derive a \emph{scaled\_score} that converts each
possible interpretation to a numeric value on a scale from -1 to +1.
This scaling takes advantage of the observation that each interpretation
can be positioned along a spectrum of understanding from completely
incorrect (orthogonal) to completely correct (triangular). Alternative
interpretations lay somewhere between.

Specifically, we assign the following values to each interpretation:

\begin{itemize}
\tightlist
\item
  (-1) : ORTHOGONAL, SATISFICE (satisfice represents an attempt at an
  orthogonal answer when none is available)
\item
  (-0.5): ? (some alternative that cannot be identified; but meaningful
  that it is not orthogonal)
\item
  (0): REFERENCE POINT, BLANK (indicates the individual thinks there is
  no answer, recognizes that ORTHOGONAL cannot be correct, but does not
  conceive of triangular)
\item
  (+0.5) TVERSKY, BOTH TRI + ORTH (indicates that they ``see'' a
  triangular response, but lack certainty and also select the ORTHOGONAL
  response)
\item
  (+1) TRIANGULAR +1
\end{itemize}

\begin{Shaded}
\begin{Highlighting}[]
\NormalTok{df\_items}\SpecialCharTok{$}\NormalTok{score\_SCALED }\OtherTok{\textless{}{-}} \FunctionTok{calc\_scaled}\NormalTok{(df\_items}\SpecialCharTok{$}\NormalTok{interpretation)}
\end{Highlighting}
\end{Shaded}

\hypertarget{summarize-by-subject-2}{%
\section{SUMMARIZE BY SUBJECT}\label{summarize-by-subject-2}}

Next, we summarize the item level scores by subject in order to
calculate cummulative subscores to be stored on the subject record.

\begin{Shaded}
\begin{Highlighting}[]
\CommentTok{\# \#HACK WD FOR LOCAL RUNNING?}
\NormalTok{imac }\OtherTok{=} \StringTok{"/Users/amyraefox/Code/SGC{-}Scaffolding\_Graph\_Comprehension/SGC{-}X/ANALYSIS/MAIN"}
\FunctionTok{setwd}\NormalTok{(imac)}

\CommentTok{\#import subjects}
\NormalTok{df\_subjects }\OtherTok{\textless{}{-}} \FunctionTok{read\_rds}\NormalTok{(}\StringTok{\textquotesingle{}analysis/SGC4A/data/1{-}study{-}level/sgc4a\_participants.rds\textquotesingle{}}\NormalTok{) }\SpecialCharTok{\%\textgreater{}\%} \FunctionTok{mutate}\NormalTok{(}\AttributeTok{subject =} \FunctionTok{as.character}\NormalTok{(subject)) }\SpecialCharTok{\%\textgreater{}\%} \FunctionTok{arrange}\NormalTok{(subject)}

\CommentTok{\#make temporary copies for testing safety}
\NormalTok{s }\OtherTok{=}\NormalTok{ df\_subjects}
\NormalTok{i }\OtherTok{=}\NormalTok{ df\_items }

\CommentTok{\#summarize}
\NormalTok{test\_subs }\OtherTok{\textless{}{-}} \FunctionTok{summarise\_bySubject}\NormalTok{(s,i)}
\end{Highlighting}
\end{Shaded}

\begin{verbatim}
`summarise()` has grouped output by 'subject'. You can override using the
`.groups` argument.
\end{verbatim}

\begin{verbatim}
[1] TRUE
[1] TRUE
[1] TRUE
[1] TRUE
[1] TRUE
[1] TRUE
[1] TRUE
\end{verbatim}

\begin{Shaded}
\begin{Highlighting}[]
\NormalTok{df\_subjects }\OtherTok{\textless{}{-}}\NormalTok{ test\_subs}
\end{Highlighting}
\end{Shaded}

We also summarize absolute and scaled score progress at each question in
the task, to explore cumulative performance over the task.

\begin{Shaded}
\begin{Highlighting}[]
\CommentTok{\#GET ABSOLUTE PROGRESS }
\NormalTok{df\_absolute\_progress }\OtherTok{\textless{}{-}} \FunctionTok{progress\_Absolute}\NormalTok{(df\_items)}

\CommentTok{\#GET SCALED PROGRESS}
\NormalTok{df\_scaled\_progress }\OtherTok{\textless{}{-}} \FunctionTok{progress\_Scaled}\NormalTok{(df\_items)}
\end{Highlighting}
\end{Shaded}

\hypertarget{explore-distributions-2}{%
\section{EXPLORE DISTRIBUTIONS}\label{explore-distributions-2}}

\begin{Shaded}
\begin{Highlighting}[]
\FunctionTok{options}\NormalTok{(}\AttributeTok{repr.plot.width =}\DecValTok{9}\NormalTok{, }\AttributeTok{repr.plot.height =}\DecValTok{12}\NormalTok{)}

\CommentTok{\#create temp data frame for visualizations}
\NormalTok{df }\OtherTok{=}\NormalTok{ df\_items }\SpecialCharTok{\%\textgreater{}\%} \FunctionTok{filter}\NormalTok{ (q }\SpecialCharTok{\%nin\%} \FunctionTok{c}\NormalTok{(}\DecValTok{6}\NormalTok{,}\DecValTok{9}\NormalTok{)) }\SpecialCharTok{\%\textgreater{}\%} \FunctionTok{mutate}\NormalTok{(}
  \AttributeTok{score\_niceABS =} \FunctionTok{as.factor}\NormalTok{(score\_niceABS),}
  \AttributeTok{pretty\_condition =} \FunctionTok{as.factor}\NormalTok{(condition),}
  \AttributeTok{pretty\_interpretation =} \FunctionTok{factor}\NormalTok{(interpretation,}
    \AttributeTok{levels =} \FunctionTok{c}\NormalTok{(}\StringTok{"Orthogonal"}\NormalTok{, }\StringTok{"Satisfice"}\NormalTok{, }
               \StringTok{"frenzy"}\NormalTok{,}\StringTok{"?"}\NormalTok{,}
                \StringTok{"reference"}\NormalTok{,}\StringTok{"blank"}\NormalTok{,}
                \StringTok{"Tversky"}\NormalTok{, }\StringTok{"both tri + orth"}\NormalTok{,}
                \StringTok{"Triangular"}\NormalTok{ ))}
\NormalTok{  )}
\end{Highlighting}
\end{Shaded}

\hypertarget{absolute-score-3}{%
\subsection{Absolute Score}\label{absolute-score-3}}

\begin{Shaded}
\begin{Highlighting}[]
\CommentTok{\#DISTRIBUTION ABSOLUTE SCORE FULL }\AlertTok{TASK}
\FunctionTok{gf\_props}\NormalTok{(}\SpecialCharTok{\textasciitilde{}}\NormalTok{score\_niceABS, }\AttributeTok{fill =} \SpecialCharTok{\textasciitilde{}}\NormalTok{pretty\_condition, }\AttributeTok{position =} \FunctionTok{position\_dodge}\NormalTok{(), }\AttributeTok{data =}\NormalTok{ df) }\SpecialCharTok{+}
  \FunctionTok{labs}\NormalTok{( }\AttributeTok{x =} \StringTok{"Absolute Score"}\NormalTok{, }
        \AttributeTok{title =} \StringTok{"Distribution of Absolute Score (all Items)"}\NormalTok{,}
        \AttributeTok{subtitle =} \FunctionTok{paste}\NormalTok{(}\StringTok{""}\NormalTok{),}
        \AttributeTok{y =} \StringTok{"Proportion of Items"}\NormalTok{) }\SpecialCharTok{+}
  \FunctionTok{scale\_fill\_discrete}\NormalTok{(}\AttributeTok{name =} \StringTok{"Condition"}\NormalTok{) }\SpecialCharTok{+}  
  \FunctionTok{theme\_minimal}\NormalTok{()}
\end{Highlighting}
\end{Shaded}

\begin{figure}[H]

{\centering \includegraphics{analysis/SGC4A/2_sgc4A_scoring_files/figure-pdf/DISTR-ABSCORE-1.pdf}

}

\end{figure}

\begin{Shaded}
\begin{Highlighting}[]
\CommentTok{\#DISTRIBUTION ABSOLUTE SCORE BY ITEM}
\FunctionTok{gf\_props}\NormalTok{(}\SpecialCharTok{\textasciitilde{}}\NormalTok{score\_niceABS, }\AttributeTok{fill =} \SpecialCharTok{\textasciitilde{}}\NormalTok{pretty\_condition, }\AttributeTok{position =} \FunctionTok{position\_dodge}\NormalTok{(), }\AttributeTok{data =}\NormalTok{ df)  }\SpecialCharTok{\%\textgreater{}\%}
  \FunctionTok{gf\_facet\_grid}\NormalTok{(pretty\_condition}\SpecialCharTok{\textasciitilde{}}\NormalTok{q) }\SpecialCharTok{+}
  \FunctionTok{labs}\NormalTok{( }\AttributeTok{x =} \StringTok{"Absolute Score"}\NormalTok{,}
        \AttributeTok{title =} \StringTok{"Distribution of Absolute Score (by Item)"}\NormalTok{,}
        \AttributeTok{subtitle =} \StringTok{""}\NormalTok{,}
        \AttributeTok{y =} \StringTok{"Proprition of Subjects"}\NormalTok{) }\SpecialCharTok{+}
  \FunctionTok{scale\_fill\_discrete}\NormalTok{(}\AttributeTok{name =} \StringTok{"Condition"}\NormalTok{) }\SpecialCharTok{+}
  \FunctionTok{theme\_minimal}\NormalTok{()}
\end{Highlighting}
\end{Shaded}

\begin{figure}[H]

{\centering \includegraphics{analysis/SGC4A/2_sgc4A_scoring_files/figure-pdf/DISTR-ABSCORE-2.pdf}

}

\end{figure}

\begin{Shaded}
\begin{Highlighting}[]
\CommentTok{\#DISTRIBUTION ABSOLUTE SCORE BY SUBJECT}
\FunctionTok{gf\_props}\NormalTok{(}\SpecialCharTok{\textasciitilde{}}\NormalTok{s\_NABS, }\AttributeTok{fill =} \SpecialCharTok{\textasciitilde{}}\NormalTok{pretty\_condition, }\AttributeTok{position =} \FunctionTok{position\_dodge}\NormalTok{(), }\AttributeTok{data =}\NormalTok{ df\_subjects) }\SpecialCharTok{\%\textgreater{}\%}
\FunctionTok{gf\_facet\_grid}\NormalTok{(pretty\_condition }\SpecialCharTok{\textasciitilde{}}\NormalTok{. )}\SpecialCharTok{+}
  \FunctionTok{labs}\NormalTok{( }\AttributeTok{x =} \StringTok{"Total Absolute Score"}\NormalTok{,}
        \AttributeTok{title =} \StringTok{"Distribution of Total Absolute Score (by Subject)"}\NormalTok{,}
        \AttributeTok{subtitle =} \StringTok{""}\NormalTok{,}
        \AttributeTok{y =} \StringTok{"Proportion of Subjects"}\NormalTok{) }\SpecialCharTok{+}
  \FunctionTok{scale\_fill\_discrete}\NormalTok{(}\AttributeTok{name =} \StringTok{"Condition"}\NormalTok{) }\SpecialCharTok{+}
  \FunctionTok{theme\_minimal}\NormalTok{() }\SpecialCharTok{+} \FunctionTok{theme}\NormalTok{(}\AttributeTok{legend.position =} \StringTok{"blank"}\NormalTok{)}
\end{Highlighting}
\end{Shaded}

\begin{figure}[H]

{\centering \includegraphics{analysis/SGC4A/2_sgc4A_scoring_files/figure-pdf/DISTR-ABSCORE-3.pdf}

}

\end{figure}

\hypertarget{scaled-score-3}{%
\subsection{Scaled Score}\label{scaled-score-3}}

\begin{Shaded}
\begin{Highlighting}[]
\FunctionTok{options}\NormalTok{(}\AttributeTok{repr.plot.width =}\DecValTok{9}\NormalTok{, }\AttributeTok{repr.plot.height =}\DecValTok{12}\NormalTok{)}

\CommentTok{\#DISTRIBUTION SCALED SCORE FULL }\AlertTok{TASK}
\FunctionTok{gf\_props}\NormalTok{(}\SpecialCharTok{\textasciitilde{}}\NormalTok{score\_SCALED, }\AttributeTok{fill =} \SpecialCharTok{\textasciitilde{}}\NormalTok{pretty\_condition, }\AttributeTok{position =} \FunctionTok{position\_dodge}\NormalTok{(), }\AttributeTok{data =}\NormalTok{ df) }\SpecialCharTok{+}
  \FunctionTok{labs}\NormalTok{( }\AttributeTok{x =} \StringTok{"Scaled Score"}\NormalTok{, }
        \AttributeTok{title =} \StringTok{"Distribution of Scaled Score (all Items)"}\NormalTok{,}
        \AttributeTok{subtitle =} \StringTok{""}\NormalTok{,}
        \AttributeTok{y =} \StringTok{"Proportion of Items"}\NormalTok{) }\SpecialCharTok{+}
  \FunctionTok{scale\_fill\_discrete}\NormalTok{(}\AttributeTok{name =} \StringTok{"Condition"}\NormalTok{) }\SpecialCharTok{+}  
  \FunctionTok{theme\_minimal}\NormalTok{()}
\end{Highlighting}
\end{Shaded}

\begin{figure}[H]

{\centering \includegraphics{analysis/SGC4A/2_sgc4A_scoring_files/figure-pdf/DISTR-SCALEDSCORE-1.pdf}

}

\end{figure}

\begin{Shaded}
\begin{Highlighting}[]
\CommentTok{\#DISTRIBUTION SCALED SCORE BY ITEM}
\FunctionTok{gf\_props}\NormalTok{(}\SpecialCharTok{\textasciitilde{}}\NormalTok{score\_SCALED, }\AttributeTok{fill =} \SpecialCharTok{\textasciitilde{}}\NormalTok{pretty\_condition, }\AttributeTok{position =} \FunctionTok{position\_dodge}\NormalTok{(), }\AttributeTok{data =}\NormalTok{ df)  }\SpecialCharTok{\%\textgreater{}\%}
  \FunctionTok{gf\_facet\_grid}\NormalTok{(q}\SpecialCharTok{\textasciitilde{}}\NormalTok{pretty\_condition) }\SpecialCharTok{+}
  \FunctionTok{labs}\NormalTok{( }\AttributeTok{x =} \StringTok{"Scaled Score"}\NormalTok{,}
        \AttributeTok{title =} \StringTok{"Distribution of Scaled Score (by Item)"}\NormalTok{,}
        \AttributeTok{subtitle =} \StringTok{""}\NormalTok{,}
        \AttributeTok{y =} \StringTok{"Proportion of Subjects"}\NormalTok{) }\SpecialCharTok{+}
  \FunctionTok{scale\_fill\_discrete}\NormalTok{(}\AttributeTok{name =} \StringTok{"Condition"}\NormalTok{) }\SpecialCharTok{+}  \FunctionTok{scale\_y\_continuous}\NormalTok{(}\AttributeTok{breaks=}\FunctionTok{c}\NormalTok{(}\DecValTok{0}\NormalTok{,}\FloatTok{0.5}\NormalTok{)) }\SpecialCharTok{+}
  \FunctionTok{theme\_minimal}\NormalTok{() }\SpecialCharTok{+} \FunctionTok{theme}\NormalTok{(}\AttributeTok{legend.position=}\StringTok{"blank"}\NormalTok{)}
\end{Highlighting}
\end{Shaded}

\begin{figure}[H]

{\centering \includegraphics{analysis/SGC4A/2_sgc4A_scoring_files/figure-pdf/DISTR-SCALEDSCORE-2.pdf}

}

\end{figure}

\begin{Shaded}
\begin{Highlighting}[]
\CommentTok{\#DISTRIBUTION SCALED SCORE BY SUBJECT}
\FunctionTok{gf\_props}\NormalTok{(}\SpecialCharTok{\textasciitilde{}}\NormalTok{s\_SCALED, }\AttributeTok{fill =} \SpecialCharTok{\textasciitilde{}}\NormalTok{pretty\_condition, }\AttributeTok{data =}\NormalTok{ df\_subjects)  }\SpecialCharTok{\%\textgreater{}\%}
  \FunctionTok{gf\_facet\_grid}\NormalTok{(pretty\_condition }\SpecialCharTok{\textasciitilde{}}\NormalTok{. )}\SpecialCharTok{+}
  \FunctionTok{labs}\NormalTok{( }\AttributeTok{x =} \StringTok{"Total Scaled Score"}\NormalTok{,}
        \AttributeTok{title =} \StringTok{"Distribution of Total Scaled Score (by Subject)"}\NormalTok{,}
        \AttributeTok{subtitle =} \StringTok{""}\NormalTok{,}
        \AttributeTok{y =} \StringTok{"Number of Subjects"}\NormalTok{) }\SpecialCharTok{+}
  \FunctionTok{scale\_fill\_discrete}\NormalTok{(}\AttributeTok{name =} \StringTok{"Condition"}\NormalTok{) }\SpecialCharTok{+}
  \FunctionTok{theme\_minimal}\NormalTok{()}
\end{Highlighting}
\end{Shaded}

\begin{figure}[H]

{\centering \includegraphics{analysis/SGC4A/2_sgc4A_scoring_files/figure-pdf/DISTR-SCALEDSCORE-3.pdf}

}

\end{figure}

\hypertarget{interpretations-2}{%
\subsection{Interpretations}\label{interpretations-2}}

\begin{Shaded}
\begin{Highlighting}[]
\CommentTok{\#DISTRIBUTION OF INTERPRETATION}
\FunctionTok{gf\_props}\NormalTok{(}\SpecialCharTok{\textasciitilde{}}\NormalTok{pretty\_interpretation, }\AttributeTok{fill =} \SpecialCharTok{\textasciitilde{}}\NormalTok{pretty\_condition, }\AttributeTok{data =}\NormalTok{ df) }\SpecialCharTok{\%\textgreater{}\%} 
  \FunctionTok{gf\_facet\_grid}\NormalTok{( pretty\_condition }\SpecialCharTok{\textasciitilde{}}\NormalTok{ ., }\AttributeTok{labeller =}\NormalTok{ label\_both) }\SpecialCharTok{+} 
  \FunctionTok{labs}\NormalTok{( }\AttributeTok{title =} \StringTok{"Distribution of Interpretations (across Task)"}\NormalTok{,}
        \AttributeTok{x =} \StringTok{"Graph Interpretation"}\NormalTok{,}
        \AttributeTok{y =} \StringTok{"Proportion of Responses"}\NormalTok{,}
        \AttributeTok{subtitle =} \StringTok{""}\NormalTok{) }\SpecialCharTok{+} 
  \FunctionTok{theme\_minimal}\NormalTok{() }\SpecialCharTok{+} \FunctionTok{theme}\NormalTok{(}\AttributeTok{legend.position =} \StringTok{"blank"}\NormalTok{)}
\end{Highlighting}
\end{Shaded}

\begin{figure}[H]

{\centering \includegraphics{analysis/SGC4A/2_sgc4A_scoring_files/figure-pdf/DISTR-INTERPRETATIONS-1.pdf}

}

\end{figure}

\begin{Shaded}
\begin{Highlighting}[]
\CommentTok{\#DISTRIBUTION OF INTERPRETATION ACROSS ITEMS}
\FunctionTok{gf\_propsh}\NormalTok{(}\SpecialCharTok{\textasciitilde{}}\NormalTok{ pretty\_interpretation, }\AttributeTok{fill =} \SpecialCharTok{\textasciitilde{}}\NormalTok{pretty\_condition, }\AttributeTok{data =}\NormalTok{ df) }\SpecialCharTok{\%\textgreater{}\%} 
  \FunctionTok{gf\_facet\_grid}\NormalTok{( pretty\_condition}\SpecialCharTok{\textasciitilde{}}\NormalTok{q) }\SpecialCharTok{+} 
  \FunctionTok{labs}\NormalTok{( }\AttributeTok{title =} \StringTok{"Distribution of Interpretations (by Item)"}\NormalTok{,}
        \AttributeTok{subtitle =} \StringTok{""}\NormalTok{,}
        \AttributeTok{y =} \StringTok{"Interpretation"}\NormalTok{, }\AttributeTok{x =} \StringTok{"Proportion of Subjects"}\NormalTok{) }\SpecialCharTok{+} \FunctionTok{theme\_minimal}\NormalTok{() }\SpecialCharTok{+} \FunctionTok{theme}\NormalTok{(}\AttributeTok{legend.position =} \StringTok{"blank"}\NormalTok{)}
\end{Highlighting}
\end{Shaded}

\begin{figure}[H]

{\centering \includegraphics{analysis/SGC4A/2_sgc4A_scoring_files/figure-pdf/DISTR-INTERPRETATIONS-2.pdf}

}

\end{figure}

\begin{Shaded}
\begin{Highlighting}[]
\CommentTok{\#DISTRIBUTION OF INTERPRETATION TYPE ACROSS ITEMS}
\FunctionTok{gf\_propsh}\NormalTok{(}\SpecialCharTok{\textasciitilde{}}\NormalTok{ high\_interpretation, }\AttributeTok{fill =} \SpecialCharTok{\textasciitilde{}}\NormalTok{pretty\_condition, }\AttributeTok{data =}\NormalTok{ df) }\SpecialCharTok{\%\textgreater{}\%} 
  \FunctionTok{gf\_facet\_grid}\NormalTok{( pretty\_condition}\SpecialCharTok{\textasciitilde{}}\NormalTok{q) }\SpecialCharTok{+} 
  \FunctionTok{labs}\NormalTok{( }\AttributeTok{title =} \StringTok{"Distribution of Interpretations (by Item)"}\NormalTok{,}
        \AttributeTok{subtitle =} \StringTok{""}\NormalTok{,}
        \AttributeTok{y =} \StringTok{"Interpretation"}\NormalTok{, }\AttributeTok{x =} \StringTok{"Proportion of Subjects"}\NormalTok{) }\SpecialCharTok{+} \FunctionTok{theme\_minimal}\NormalTok{() }\SpecialCharTok{+} \FunctionTok{theme}\NormalTok{(}\AttributeTok{legend.position =} \StringTok{"blank"}\NormalTok{)}
\end{Highlighting}
\end{Shaded}

\begin{figure}[H]

{\centering \includegraphics{analysis/SGC4A/2_sgc4A_scoring_files/figure-pdf/DISTR-INTERPRETATIONS-3.pdf}

}

\end{figure}

\hypertarget{progress-over-task-2}{%
\subsection{Progress over Task}\label{progress-over-task-2}}

\begin{Shaded}
\begin{Highlighting}[]
\CommentTok{\#VISUALIZE progress over time ABSOLUTE score }
\FunctionTok{ggplot}\NormalTok{(}\AttributeTok{data =}\NormalTok{ df\_absolute\_progress, }\FunctionTok{aes}\NormalTok{(}\AttributeTok{x =}\NormalTok{ question, }\AttributeTok{y =}\NormalTok{ score, }\AttributeTok{group =}\NormalTok{ subject, }\AttributeTok{alpha =} \FloatTok{0.01}\NormalTok{, }\AttributeTok{color =}\NormalTok{ pretty\_condition)) }\SpecialCharTok{+} 
 \FunctionTok{geom\_line}\NormalTok{(}\AttributeTok{position=}\FunctionTok{position\_jitter}\NormalTok{(}\AttributeTok{w=}\FloatTok{0.15}\NormalTok{, }\AttributeTok{h=}\FloatTok{0.05}\NormalTok{), }\AttributeTok{size=}\FloatTok{0.1}\NormalTok{) }\SpecialCharTok{+}
 \FunctionTok{facet\_wrap}\NormalTok{(}\SpecialCharTok{\textasciitilde{}}\NormalTok{pretty\_condition) }\SpecialCharTok{+} 
 \FunctionTok{labs}\NormalTok{ (}\AttributeTok{title =} \StringTok{"Cumulative Absolute Score over sequence of task"}\NormalTok{, }\AttributeTok{x =} \StringTok{"Question"}\NormalTok{ , }\AttributeTok{y =} \StringTok{"Cumulative Absolute Score"}\NormalTok{) }\SpecialCharTok{+} 
 \FunctionTok{scale\_x\_continuous}\NormalTok{(}\AttributeTok{breaks =} \FunctionTok{c}\NormalTok{(}\DecValTok{1}\NormalTok{,}\DecValTok{2}\NormalTok{,}\DecValTok{3}\NormalTok{,}\DecValTok{4}\NormalTok{,}\DecValTok{5}\NormalTok{,}\DecValTok{6}\NormalTok{,}\DecValTok{7}\NormalTok{,}\DecValTok{8}\NormalTok{,}\DecValTok{9}\NormalTok{,}\DecValTok{10}\NormalTok{,}\DecValTok{11}\NormalTok{,}\DecValTok{12}\NormalTok{,}\DecValTok{13}\NormalTok{)) }\SpecialCharTok{+}
 \FunctionTok{theme\_minimal}\NormalTok{() }\SpecialCharTok{+} \FunctionTok{theme}\NormalTok{(}\AttributeTok{legend.position =} \StringTok{"blank"}\NormalTok{)}
\end{Highlighting}
\end{Shaded}

\begin{figure}[H]

{\centering \includegraphics{analysis/SGC4A/2_sgc4A_scoring_files/figure-pdf/VIZ-PROGRESS-1.pdf}

}

\end{figure}

\begin{Shaded}
\begin{Highlighting}[]
\CommentTok{\#VISUALIZE progress over time SCALED score }
\FunctionTok{ggplot}\NormalTok{(}\AttributeTok{data =}\NormalTok{ df\_scaled\_progress, }\FunctionTok{aes}\NormalTok{(}\AttributeTok{x =}\NormalTok{ question, }\AttributeTok{y =}\NormalTok{ score, }\AttributeTok{group =}\NormalTok{ subject, }\AttributeTok{alpha =} \FloatTok{0.01}\NormalTok{, }\AttributeTok{color =}\NormalTok{ pretty\_condition)) }\SpecialCharTok{+} 
 \FunctionTok{geom\_line}\NormalTok{(}\AttributeTok{position=}\FunctionTok{position\_jitter}\NormalTok{(}\AttributeTok{w=}\FloatTok{0.15}\NormalTok{, }\AttributeTok{h=}\FloatTok{0.05}\NormalTok{), }\AttributeTok{size=}\FloatTok{0.1}\NormalTok{) }\SpecialCharTok{+}
 \FunctionTok{facet\_wrap}\NormalTok{(}\SpecialCharTok{\textasciitilde{}}\NormalTok{pretty\_condition) }\SpecialCharTok{+} 
 \FunctionTok{labs}\NormalTok{ (}\AttributeTok{title =} \StringTok{"Cumulative Scaled Score over sequence of task"}\NormalTok{, }\AttributeTok{x =} \StringTok{"Question"}\NormalTok{ , }\AttributeTok{y =} \StringTok{"Cumulative Scaled Score"}\NormalTok{) }\SpecialCharTok{+} 
 \FunctionTok{scale\_x\_continuous}\NormalTok{(}\AttributeTok{breaks =} \FunctionTok{c}\NormalTok{(}\DecValTok{1}\NormalTok{,}\DecValTok{2}\NormalTok{,}\DecValTok{3}\NormalTok{,}\DecValTok{4}\NormalTok{,}\DecValTok{5}\NormalTok{,}\DecValTok{6}\NormalTok{,}\DecValTok{7}\NormalTok{,}\DecValTok{8}\NormalTok{,}\DecValTok{9}\NormalTok{,}\DecValTok{10}\NormalTok{,}\DecValTok{11}\NormalTok{,}\DecValTok{12}\NormalTok{,}\DecValTok{13}\NormalTok{)) }\SpecialCharTok{+}
 \FunctionTok{theme\_minimal}\NormalTok{() }\SpecialCharTok{+} \FunctionTok{theme}\NormalTok{(}\AttributeTok{legend.position =} \StringTok{"blank"}\NormalTok{)}
\end{Highlighting}
\end{Shaded}

\begin{figure}[H]

{\centering \includegraphics{analysis/SGC4A/2_sgc4A_scoring_files/figure-pdf/VIZ-PROGRESS-2.pdf}

}

\end{figure}

\hypertarget{interpretation-subscores-2}{%
\subsection{Interpretation Subscores}\label{interpretation-subscores-2}}

\begin{Shaded}
\begin{Highlighting}[]
\FunctionTok{gf\_density}\NormalTok{(}\SpecialCharTok{\textasciitilde{}}\NormalTok{ s\_TRI, }\AttributeTok{fill =} \SpecialCharTok{\textasciitilde{}}\NormalTok{pretty\_condition, }\AttributeTok{data =}\NormalTok{ df\_subjects) }\SpecialCharTok{\%\textgreater{}\%}
  \FunctionTok{gf\_facet\_wrap}\NormalTok{( }\SpecialCharTok{\textasciitilde{}}\NormalTok{ pretty\_condition) }\SpecialCharTok{+}
  \FunctionTok{labs}\NormalTok{( }\AttributeTok{title =} \StringTok{"Distribution of Total Triangular Score"}\NormalTok{,}
        \AttributeTok{subtitle =} \StringTok{"Impasse shifts density toward higher Triagular scores"}\NormalTok{,}
        \AttributeTok{x =} \StringTok{"Item Triangular Score"}\NormalTok{, }\AttributeTok{y =} \StringTok{"Proportion of Subjects"}\NormalTok{) }\SpecialCharTok{+}
        \FunctionTok{theme\_minimal}\NormalTok{() }\SpecialCharTok{+} \FunctionTok{theme}\NormalTok{(}\AttributeTok{legend.position =} \StringTok{"blank"}\NormalTok{)}
\end{Highlighting}
\end{Shaded}

\begin{figure}[H]

{\centering \includegraphics{analysis/SGC4A/2_sgc4A_scoring_files/figure-pdf/DIST-SUBSCORES-1.pdf}

}

\end{figure}

\begin{Shaded}
\begin{Highlighting}[]
\FunctionTok{gf\_density}\NormalTok{(}\SpecialCharTok{\textasciitilde{}}\NormalTok{ s\_ORTH, }\AttributeTok{fill =} \SpecialCharTok{\textasciitilde{}}\NormalTok{pretty\_condition, }\AttributeTok{data =}\NormalTok{ df\_subjects) }\SpecialCharTok{\%\textgreater{}\%}
  \FunctionTok{gf\_facet\_wrap}\NormalTok{( }\SpecialCharTok{\textasciitilde{}}\NormalTok{ pretty\_condition) }\SpecialCharTok{+}
  \FunctionTok{labs}\NormalTok{( }\AttributeTok{title =} \StringTok{"Distribution of Total Orthogonal Score"}\NormalTok{,}
        \AttributeTok{subtitle =} \StringTok{"Impasse shifts density toward lower Orthogonal scores"}\NormalTok{,}
        \AttributeTok{x =} \StringTok{"Item Orthogonal Score"}\NormalTok{, }\AttributeTok{y =} \StringTok{"Proportion of Subjects"}\NormalTok{) }\SpecialCharTok{+}
        \FunctionTok{theme\_minimal}\NormalTok{() }\SpecialCharTok{+} \FunctionTok{theme}\NormalTok{(}\AttributeTok{legend.position =} \StringTok{"blank"}\NormalTok{)}
\end{Highlighting}
\end{Shaded}

\begin{figure}[H]

{\centering \includegraphics{analysis/SGC4A/2_sgc4A_scoring_files/figure-pdf/DIST-SUBSCORES-2.pdf}

}

\end{figure}

\begin{Shaded}
\begin{Highlighting}[]
\FunctionTok{gf\_density}\NormalTok{(}\SpecialCharTok{\textasciitilde{}}\NormalTok{ s\_TVERSKY, }\AttributeTok{fill =} \SpecialCharTok{\textasciitilde{}}\NormalTok{pretty\_condition, }\AttributeTok{data =}\NormalTok{ df\_subjects) }\SpecialCharTok{\%\textgreater{}\%}
  \FunctionTok{gf\_facet\_wrap}\NormalTok{( }\SpecialCharTok{\textasciitilde{}}\NormalTok{ pretty\_condition) }\SpecialCharTok{+}
  \FunctionTok{labs}\NormalTok{( }\AttributeTok{title =} \StringTok{"Distribution of Total Tversky Score"}\NormalTok{,}
        \AttributeTok{subtitle =} \StringTok{"Impasse shifts density toward higher Tversky scores"}\NormalTok{,}
        \AttributeTok{x =} \StringTok{"Item Orthogonal Score"}\NormalTok{, }\AttributeTok{y =} \StringTok{"Proportion of Subjects"}\NormalTok{) }\SpecialCharTok{+}
        \FunctionTok{theme\_minimal}\NormalTok{() }\SpecialCharTok{+} \FunctionTok{theme}\NormalTok{(}\AttributeTok{legend.position =} \StringTok{"blank"}\NormalTok{)}
\end{Highlighting}
\end{Shaded}

\begin{figure}[H]

{\centering \includegraphics{analysis/SGC4A/2_sgc4A_scoring_files/figure-pdf/DIST-SUBSCORES-3.pdf}

}

\end{figure}

\begin{Shaded}
\begin{Highlighting}[]
\FunctionTok{gf\_histogram}\NormalTok{(}\SpecialCharTok{\textasciitilde{}}\NormalTok{ s\_SATISFICE, }\AttributeTok{fill =} \SpecialCharTok{\textasciitilde{}}\NormalTok{pretty\_condition, }\AttributeTok{data =}\NormalTok{ df\_subjects) }\SpecialCharTok{\%\textgreater{}\%}
  \FunctionTok{gf\_facet\_wrap}\NormalTok{( }\SpecialCharTok{\textasciitilde{}}\NormalTok{ pretty\_condition) }\SpecialCharTok{+}
  \FunctionTok{labs}\NormalTok{( }\AttributeTok{title =} \StringTok{"Distribution of Total Satisfice Score"}\NormalTok{,}
        \AttributeTok{subtitle =} \StringTok{"Satisficing only occurs in impasse, when no orthogonal response is available"}\NormalTok{,}
        \AttributeTok{x =} \StringTok{"Item Orthogonal Score"}\NormalTok{, }\AttributeTok{y =} \StringTok{"Proportion of Subjects"}\NormalTok{) }\SpecialCharTok{+}
        \FunctionTok{theme\_minimal}\NormalTok{() }\SpecialCharTok{+} \FunctionTok{theme}\NormalTok{(}\AttributeTok{legend.position =} \StringTok{"blank"}\NormalTok{)}
\end{Highlighting}
\end{Shaded}

\begin{figure}[H]

{\centering \includegraphics{analysis/SGC4A/2_sgc4A_scoring_files/figure-pdf/DIST-SUBSCORES-4.pdf}

}

\end{figure}

\hypertarget{peeking}{%
\section{PEEKING}\label{peeking}}

\begin{Shaded}
\begin{Highlighting}[]
\FunctionTok{library}\NormalTok{(performance)}
\FunctionTok{library}\NormalTok{(report)}
\NormalTok{m1 }\OtherTok{\textless{}{-}} \FunctionTok{lm}\NormalTok{(s\_SCALED }\SpecialCharTok{\textasciitilde{}}\NormalTok{ pretty\_condition, }\AttributeTok{data =}\NormalTok{ df\_subjects)}
\FunctionTok{summary}\NormalTok{(m1)}
\end{Highlighting}
\end{Shaded}

\begin{verbatim}

Call:
lm(formula = s_SCALED ~ pretty_condition, data = df_subjects)

Residuals:
   Min     1Q Median     3Q    Max 
 -8.60  -5.60  -3.13   1.65  21.37 

Coefficients:
                            Estimate Std. Error t value Pr(>|t|)    
(Intercept)                   -7.153      0.867   -8.25  3.1e-15 ***
pretty_conditionOrth-Sparse    1.000      1.226    0.82    0.415    
pretty_conditionOrth-Grid     -1.219      1.195   -1.02    0.308    
pretty_conditionTri-Sparse     2.752      1.233    2.23    0.026 *  
---
Signif. codes:  0 '***' 0.001 '**' 0.01 '*' 0.05 '.' 0.1 ' ' 1

Residual standard error: 8.13 on 356 degrees of freedom
Multiple R-squared:  0.0316,    Adjusted R-squared:  0.0234 
F-statistic: 3.87 on 3 and 356 DF,  p-value: 0.00958
\end{verbatim}

\begin{Shaded}
\begin{Highlighting}[]
\FunctionTok{anova}\NormalTok{(m1)}
\end{Highlighting}
\end{Shaded}

\begin{verbatim}
Analysis of Variance Table

Response: s_SCALED
                  Df Sum Sq Mean Sq F value Pr(>F)   
pretty_condition   3    768   256.0    3.87 0.0096 **
Residuals        356  23558    66.2                  
---
Signif. codes:  0 '***' 0.001 '**' 0.01 '*' 0.05 '.' 0.1 ' ' 1
\end{verbatim}

\begin{Shaded}
\begin{Highlighting}[]
\FunctionTok{report}\NormalTok{(m1)}
\end{Highlighting}
\end{Shaded}

\begin{verbatim}
Warning: 'data_findcols()' is deprecated and will be removed in a future update.
  Its usage is discouraged. Please use 'data_find()' instead.

Warning: 'data_findcols()' is deprecated and will be removed in a future update.
  Its usage is discouraged. Please use 'data_find()' instead.

Warning: 'data_findcols()' is deprecated and will be removed in a future update.
  Its usage is discouraged. Please use 'data_find()' instead.
\end{verbatim}

\begin{verbatim}
We fitted a linear model (estimated using OLS) to predict s_SCALED with pretty_condition (formula: s_SCALED ~ pretty_condition). The model explains a statistically significant and weak proportion of variance (R2 = 0.03, F(3, 356) = 3.87, p = 0.010, adj. R2 = 0.02). The model's intercept, corresponding to pretty_condition = Orth-Full, is at -7.15 (95% CI [-8.86, -5.45], t(356) = -8.25, p < .001). Within this model:

  - The effect of pretty condition [Orth-Sparse] is statistically non-significant and positive (beta = 1.00, 95% CI [-1.41, 3.41], t(356) = 0.82, p = 0.415; Std. beta = 0.12, 95% CI [-0.17, 0.41])
  - The effect of pretty condition [Orth-Grid] is statistically non-significant and negative (beta = -1.22, 95% CI [-3.57, 1.13], t(356) = -1.02, p = 0.308; Std. beta = -0.15, 95% CI [-0.43, 0.14])
  - The effect of pretty condition [Tri-Sparse] is statistically significant and positive (beta = 2.75, 95% CI [0.33, 5.18], t(356) = 2.23, p = 0.026; Std. beta = 0.33, 95% CI [0.04, 0.63])

Standardized parameters were obtained by fitting the model on a standardized version of the dataset. 95% Confidence Intervals (CIs) and p-values were computed using the Wald approximation.
\end{verbatim}

\hypertarget{export-5}{%
\section{EXPORT}\label{export-5}}

Finally, we export the scores for each item (\texttt{df\_items}) and
summarized over subjects (\texttt{df\_subjects}), as well as cumulative
progress dataframes (\texttt{df\_absolute\_progress},
\texttt{df\_scaled\_progress})

\begin{Shaded}
\begin{Highlighting}[]
\CommentTok{\# \#HACK WD FOR LOCAL RUNNING?}
\NormalTok{imac }\OtherTok{=} \StringTok{"/Users/amyraefox/Code/SGC{-}Scaffolding\_Graph\_Comprehension/SGC{-}X/ANALYSIS/MAIN"}
\CommentTok{\# \# mbp = "/Users/amyfox/Sites/RESEARCH/SGC—Scaffolding Graph Comprehension/SGC{-}X/ANALYSIS/MAIN"}
\FunctionTok{setwd}\NormalTok{(imac)}

\CommentTok{\#SAVE FILES}
\FunctionTok{write.csv}\NormalTok{(df\_subjects,}\StringTok{"analysis/SGC4A/data/2{-}scored{-}data/sgc4a\_scored\_participants.csv"}\NormalTok{, }\AttributeTok{row.names =} \ConstantTok{FALSE}\NormalTok{)}
\FunctionTok{write.csv}\NormalTok{(df\_items,}\StringTok{"analysis/SGC4A/data/2{-}scored{-}data/sgc4a\_scored\_items.csv"}\NormalTok{, }\AttributeTok{row.names =} \ConstantTok{FALSE}\NormalTok{)}
\FunctionTok{write.csv}\NormalTok{(df\_absolute\_progress,}\StringTok{"analysis/SGC4A/data/2{-}scored{-}data/sgc4a\_absolute\_progress.csv"}\NormalTok{, }\AttributeTok{row.names =} \ConstantTok{FALSE}\NormalTok{)}
\FunctionTok{write.csv}\NormalTok{(df\_scaled\_progress,}\StringTok{"analysis/SGC4A/data/2{-}scored{-}data/sgc4a\_scaled\_progress.csv"}\NormalTok{, }\AttributeTok{row.names =} \ConstantTok{FALSE}\NormalTok{)}

\CommentTok{\#SAVE R Data Structures }
\CommentTok{\#export R DATA STRUCTURES (include codebook metadata)}
\NormalTok{rio}\SpecialCharTok{::}\FunctionTok{export}\NormalTok{(df\_subjects, }\StringTok{"analysis/SGC4A/data/2{-}scored{-}data/sgc4a\_scored\_participants.rds"}\NormalTok{) }\CommentTok{\# to R data structure file}
\NormalTok{rio}\SpecialCharTok{::}\FunctionTok{export}\NormalTok{(df\_items, }\StringTok{"analysis/SGC4A/data/2{-}scored{-}data/sgc4a\_scored\_items.rds"}\NormalTok{) }\CommentTok{\# to R data structure file}
\end{Highlighting}
\end{Shaded}

\hypertarget{resources-8}{%
\section{RESOURCES}\label{resources-8}}

\begin{Shaded}
\begin{Highlighting}[]
\FunctionTok{sessionInfo}\NormalTok{()}
\end{Highlighting}
\end{Shaded}

\begin{verbatim}
R version 4.2.1 (2022-06-23)
Platform: x86_64-apple-darwin17.0 (64-bit)
Running under: macOS Big Sur ... 10.16

Matrix products: default
BLAS:   /Library/Frameworks/R.framework/Versions/4.2/Resources/lib/libRblas.0.dylib
LAPACK: /Library/Frameworks/R.framework/Versions/4.2/Resources/lib/libRlapack.dylib

locale:
[1] en_US.UTF-8/en_US.UTF-8/en_US.UTF-8/C/en_US.UTF-8/en_US.UTF-8

attached base packages:
[1] stats     graphics  grDevices utils     datasets  methods   base     

other attached packages:
 [1] report_0.5.1      performance_0.9.1 forcats_0.5.1     stringr_1.4.0    
 [5] dplyr_1.0.9       purrr_0.3.4       readr_2.1.2       tidyr_1.2.0      
 [9] tibble_3.1.7      tidyverse_1.3.1   Hmisc_4.7-0       Formula_1.2-4    
[13] survival_3.3-1    lattice_0.20-45   pbapply_1.5-0     ggformula_0.10.1 
[17] ggridges_0.5.3    scales_1.2.0      ggstance_0.3.5    ggplot2_3.3.6    
[21] kableExtra_1.3.4 

loaded via a namespace (and not attached):
 [1] colorspace_2.0-3    ellipsis_0.3.2      rio_0.5.29         
 [4] htmlTable_2.4.0     parameters_0.18.1   base64enc_0.1-3    
 [7] fs_1.5.2            rstudioapi_0.13     farver_2.1.0       
[10] bit64_4.0.5         fansi_1.0.3         lubridate_1.8.0    
[13] xml2_1.3.3          codetools_0.2-18    splines_4.2.1      
[16] knitr_1.39          polyclip_1.10-0     jsonlite_1.8.0     
[19] broom_0.8.0         cluster_2.1.3       dbplyr_2.2.1       
[22] png_0.1-7           ggforce_0.3.3       effectsize_0.7.0   
[25] compiler_4.2.1      httr_1.4.3          backports_1.4.1    
[28] assertthat_0.2.1    Matrix_1.4-1        fastmap_1.1.0      
[31] cli_3.3.0           tweenr_1.0.2        htmltools_0.5.2    
[34] tools_4.2.1         gtable_0.3.0        glue_1.6.2         
[37] Rcpp_1.0.8.3        cellranger_1.1.0    vctrs_0.4.1        
[40] svglite_2.1.0       insight_0.17.1      xfun_0.31          
[43] openxlsx_4.2.5      rvest_1.0.2         lifecycle_1.0.1    
[46] mosaicCore_0.9.0    MASS_7.3-57         vroom_1.5.7        
[49] hms_1.1.1           parallel_4.2.1      RColorBrewer_1.1-3 
[52] yaml_2.3.5          curl_4.3.2          gridExtra_2.3      
[55] labelled_2.9.1      rpart_4.1.16        latticeExtra_0.6-29
[58] stringi_1.7.6       bayestestR_0.12.1   checkmate_2.1.0    
[61] zip_2.2.0           rlang_1.0.3         pkgconfig_2.0.3    
[64] systemfonts_1.0.4   evaluate_0.15       htmlwidgets_1.5.4  
[67] labeling_0.4.2      bit_4.0.4           tidyselect_1.1.2   
[70] plyr_1.8.7          magrittr_2.0.3      R6_2.5.1           
[73] generics_0.1.2      DBI_1.1.3           pillar_1.7.0       
[76] haven_2.5.0         foreign_0.8-82      withr_2.5.0        
[79] datawizard_0.4.1    nnet_7.3-17         modelr_0.1.8       
[82] crayon_1.5.1        utf8_1.2.2          tzdb_0.3.0         
[85] rmarkdown_2.14      jpeg_0.1-9          grid_4.2.1         
[88] readxl_1.4.0        data.table_1.14.2   reprex_2.0.1       
[91] digest_0.6.29       webshot_0.5.3       munsell_0.5.0      
[94] viridisLite_0.4.0  
\end{verbatim}

\part{SGC4B}

\newpage

\hypertarget{sec-SGC4B-introduction}{%
\chapter{Introduction}\label{sec-SGC4B-introduction}}

\textbf{In Study 4B we explore the extent to which the design of the
marks indicating data points influence how a reader interprets its
underlying coordinate system.}

\begin{longtable}[]{@{}
  >{\raggedright\arraybackslash}p{(\columnwidth - 2\tabcolsep) * \real{0.4000}}
  >{\raggedright\arraybackslash}p{(\columnwidth - 2\tabcolsep) * \real{0.6000}}@{}}
\caption{\textbf{SGC4B Study Conditions}}\tabularnewline
\toprule()
\endhead
\includegraphics{analysis/utils/img/111.png} &
\begin{minipage}[t]{\linewidth}\raggedright
\textbf{Point}\\
Demo:
\href{https://limitless-plains-85018.herokuapp.com/?study=SGC4B\&condition=111\&session=WEB-DEMO}{111}\strut
\end{minipage} \\
\includegraphics{analysis/utils/img/1112.png} &
\begin{minipage}[t]{\linewidth}\raggedright
\textbf{Arrow}\\
Demo:
\href{https://limitless-plains-85018.herokuapp.com/?study=SGC4B\&condition=1112\&session=WEB-DEMO}{1112}\strut
\end{minipage} \\
\includegraphics{analysis/utils/img/1113.png} &
\begin{minipage}[t]{\linewidth}\raggedright
\textbf{Cross}\\
Demo:
\href{https://limitless-plains-85018.herokuapp.com/?study=SGC4B\&condition=1113\&session=WEB-DEMO}{1113}\strut
\end{minipage} \\
\bottomrule()
\end{longtable}

\begin{Shaded}
\begin{Highlighting}[]
\FunctionTok{library}\NormalTok{(codebook) }\CommentTok{\#data dictionary}
\FunctionTok{library}\NormalTok{(tidyverse) }\CommentTok{\#ALL THE THINGS}
\FunctionTok{library}\NormalTok{(kableExtra) }\CommentTok{\#tables}

\CommentTok{\#set some output options}
\FunctionTok{library}\NormalTok{(dplyr, }\AttributeTok{warn.conflicts =} \ConstantTok{FALSE}\NormalTok{)}
\FunctionTok{options}\NormalTok{(}\AttributeTok{dplyr.summarise.inform =} \ConstantTok{FALSE}\NormalTok{)}
\FunctionTok{options}\NormalTok{(}\AttributeTok{scipen=}\DecValTok{1}\NormalTok{, }\AttributeTok{digits=}\DecValTok{3}\NormalTok{)}
\end{Highlighting}
\end{Shaded}

\begin{Shaded}
\begin{Highlighting}[]
\CommentTok{\# }\AlertTok{HACK}\CommentTok{ WD FOR LOCAL RUNNING?}
\CommentTok{\#imac = "/Users/amyraefox/Code/SGC{-}Scaffolding\_Graph\_Comprehension/SGC{-}X/ANALYSIS/MAIN"}
\CommentTok{\# mbp = "/Users/amyfox/Sites/RESEARCH/SGC—Scaffolding Graph Comprehension/SGC{-}X/ANALYSIS/MAIN"}
\CommentTok{\#setwd(imac)}

\CommentTok{\#IMPORT DATA }
\NormalTok{df\_subjects }\OtherTok{\textless{}{-}} \FunctionTok{read\_rds}\NormalTok{(}\StringTok{\textquotesingle{}analysis/SGC4B/data/0{-}study{-}level/sgc4b\_participants.rds\textquotesingle{}}\NormalTok{)}
\end{Highlighting}
\end{Shaded}

\begin{Shaded}
\begin{Highlighting}[]
\NormalTok{title }\OtherTok{=} \StringTok{"Participants by Condition"}
\NormalTok{cols }\OtherTok{=} \FunctionTok{c}\NormalTok{(}\StringTok{"Condition"}\NormalTok{,}\StringTok{"n"}\NormalTok{)}
\NormalTok{cont }\OtherTok{\textless{}{-}} \FunctionTok{table}\NormalTok{(df\_subjects}\SpecialCharTok{$}\NormalTok{pretty\_condition)}
\NormalTok{cont }\SpecialCharTok{\%\textgreater{}\%}  \FunctionTok{addmargins}\NormalTok{() }\SpecialCharTok{\%\textgreater{}\%} \FunctionTok{kbl}\NormalTok{(}\AttributeTok{caption =}\NormalTok{ title, }\AttributeTok{col.names =}\NormalTok{ cols) }\SpecialCharTok{\%\textgreater{}\%} \FunctionTok{kable\_classic}\NormalTok{()}
\end{Highlighting}
\end{Shaded}

\begin{table}

\caption{Participants by Condition}
\centering
\begin{tabular}[t]{l|r}
\hline
Condition & n\\
\hline
point & 91\\
\hline
arrow & 98\\
\hline
cross & 83\\
\hline
Sum & 272\\
\hline
\end{tabular}
\end{table}

\hypertarget{hypotheses-2}{%
\subsection{Hypotheses}\label{hypotheses-2}}

\textbf{Experimental Hypothesis:}\\
\emph{We hypothesize that the design of the MARK that represents an
individual data point can also influence the interpretation of the
coordinate system, even though it does not explicitly represent any
aspect of the coordinate system, and that is not the function of the
mark in the graphical framework. Marks can provide visual cues that
direct the reader toward more appropriate interpretations.}

\begin{itemize}
\tightlist
\item
  H1 The ARROW condition should yield significantly better performance

  \begin{itemize}
  \tightlist
  \item
    H1A \textbar{} ARROW condition should yield higher total absolute
    score
  \item
    H1B \textbar{} ARROW condition should yield higher first question
    score
  \item
    H1C \textbar{} ARROW condition should yield lower response times on
    the first question (both overall, and specifically among
    participants who answer correctly)
  \end{itemize}
\item
  H2 \textbar{} The CROSS condition should not yield significantly
  better performance

  \begin{itemize}
  \tightlist
  \item
    H2A \textbar{} No difference in overall performance between POINT
    and CROSS
  \item
    H2B \textbar{} No difference in first question score between POINT
    and CROSS
  \item
    H2C \textbar{} No difference in response latency between POINT and
    CROSS
  \end{itemize}
\end{itemize}

\hypertarget{methods-3}{%
\section{METHODS}\label{methods-3}}

\hypertarget{design-3}{%
\subsection{Design}\label{design-3}}

We employed a mixed design with 1 between-subjects factor with 3 levels
(Mark: POINT, CROSS, ARROW) and 15 items (within-subjects factor).

Independent Variables:

\begin{itemize}
\tightlist
\item
  B-S (Mark Design: Point, Arrow, Cross )
\item
  W-S (Item x 15)
\end{itemize}

Dependent Variables:

\begin{itemize}
\tightlist
\item
  Response Accuracy : Is the response triangular-correct?
\item
  Response Interpretation : (derived) With which interpretation of the
  graph is the subject's response on an individual question consistent?
\item
  Response Latency : Time from stimulus onset to clicking `Submit'
  button: time in (s)
\end{itemize}

\hypertarget{materials-3}{%
\subsection{Materials}\label{materials-3}}

Stimuli consisted of a series of 15 graph comprehension questions, each
testing a different combination of time interval relations, to be read
from a Triangular-Model graph. Figure~\ref{fig-sample}. The list of
questions can be found \href{static/stimuli/sgcx_questions.csv}{here}.

\begin{figure}

{\centering \includegraphics{analysis/utils/img/sample_graphComprehensionTask.png}

}

\caption{\label{fig-sample}Sample Question (Q=1) for Graph Comprehension
Task}

\end{figure}

In each experimental

\hypertarget{procedure-3}{%
\subsection{Procedure}\label{procedure-3}}

Participants completed the study via a web-browser.

(1) Upon starting, they submitted informed consent, before reading task
instructions.

(2) Participants were introduced to a scenario in which they were to
play the role of a project manager, scheduling shifts for a group of
employees. The schedule of the employees was presented in a Triangular
Model (TM) graph, and they would be answering question about the
schedule.

(3) Then participants completed an experimental block of 15 items: The
Graph Comprehension Task.

(4) Following the experimental block, participants answered a
free-response question about their strategy for reading the graph,
followed by a demographic questionnaire and debrief.

\hypertarget{sample-4}{%
\subsection{Sample}\label{sample-4}}

Data were collected by convenience sample of a university subject pool
during the winter of 2022. Participants accessed the study via a web
browser (asynchronously). The stimulus application required the
participant stay in full-screen mode for the entirety of the study.

\hypertarget{analysis-3}{%
\section{ANALYSIS}\label{analysis-3}}

\hypertarget{sec-SGC4B-harmonize}{%
\subsection{Data Preparation}\label{sec-SGC4B-harmonize}}

Data were collected via a custom web application and stored in a NoSQL
database. The following exclusion criteria were applied during data
cleaning:

\begin{itemize}
\tightlist
\item
  completion status : ``success'' ; subject must have finished all parts
  of the study, including demographic questionnaire
\item
  session ID: {[}in list{]} ; subject must have been assigned to valid
  data collection session (discard testing and piloting data)
\item
  browser interaction violations \textless{} 3; subject must have fewer
  than 3 violations of non-allowed browser interactions (i.e.~resizing
  window, leaving browser tab or leaving fullscreen mode)
\item
  self-rated effort \textgreater{} 2; subjects who reported, ``not
  trying hard/rushing through questions'' or ``started out trying hard
  but giving up at some point'' were excluded from analysis.
\item
  attention check ==TRUE ; subjects who failed to answer a mid-study
  attention check question (Graph Comprehension Task Question \#6) are
  excluded
\end{itemize}

\begin{longtable}[]{@{}ll@{}}
\toprule()
Pre-Requisite & Followed By \\
\midrule()
\endhead
winter2022\_clean\_sgc4b.Rmd & 2\_sgc4B\_scoring.qmd \\
\bottomrule()
\end{longtable}

The underlying data structure of the stimulus web application changed
across the data collection period, resulting in slightly different data
files (i.e.~columns are not named consistently). In this section, we
combine the files from each data collection period into a single
\emph{harmonized} data file for analysis (one for participants, one for
items).

\hypertarget{participants-4}{%
\subsubsection{Participants}\label{participants-4}}

First we import participant-level data, selecting only the columns
relevant for analysis. The result is a single data frame
\texttt{df\_subjects} containing one row for each subject (across all
periods). Note that we \emph{are not} discarding any \emph{response}
data. Rather, we discard columns that are automatically recorded by the
stimulus web application and help the application run.

\emph{Note that we discard some columns representing scores calculated
in the stimulus engine. These scores were calculated differently across
collection periods, and so we discard them and recalculate scores in the
next analysis notebook. No raw data (responses and response times) are
discarded, only algorithmically-derived scores for the responses.}

\begin{Shaded}
\begin{Highlighting}[]
\CommentTok{\#IMPORT PARTICIPANT DATA}

\CommentTok{\# }\AlertTok{HACK}\CommentTok{ WD FOR LOCAL RUNNING?}
\CommentTok{\#imac = "/Users/amyraefox/Code/SGC{-}Scaffolding\_Graph\_Comprehension/SGC{-}X/ANALYSIS/MAIN"}
\CommentTok{\# mbp = "/Users/amyfox/Sites/RESEARCH/SGC—Scaffolding Graph Comprehension/SGC{-}X/ANALYSIS/MAIN"}
\CommentTok{\#setwd(imac)}

\CommentTok{\#import file}
\NormalTok{df\_subjects }\OtherTok{\textless{}{-}} \FunctionTok{read\_rds}\NormalTok{(}\StringTok{"analysis/SGC4B/data/0{-}study{-}level/sgc4b\_participants.rds"}\NormalTok{) }\CommentTok{\#use RDS file as it contains metadata}

\CommentTok{\#NO EXPLANATION COLUMN IN SGC4B DATASET; TRIAL NOT COLLECTED }
\CommentTok{\#save \textquotesingle{}explanation\textquotesingle{} columns from winter22, which is actually a response to a free response item (Q16); was recorded with item\_level data in old webapp}
\CommentTok{\# df\_q16 \textless{}{-} df\_subjects \%\textgreater{}\% }
\CommentTok{\#   select(subject, condition, term , mode, explanation) \%\textgreater{}\% }
\CommentTok{\#   mutate(}
\CommentTok{\#     q = 16,}
\CommentTok{\#     response = explanation}
\CommentTok{\#   ) \%\textgreater{}\% select({-}explanation)}

\CommentTok{\#reduce data collected using NEW webapp to useful columns}
\CommentTok{\#drop absolute score because we rescore in 2\_scoring}
\NormalTok{df\_subjects }\OtherTok{\textless{}{-}}\NormalTok{ df\_subjects }\SpecialCharTok{\%\textgreater{}\%} 
  \CommentTok{\#select only columns we\textquotesingle{}ll be analyzing, discard others}
\NormalTok{  dplyr}\SpecialCharTok{::}\FunctionTok{select}\NormalTok{( subject, condition, pretty\_condition, study, term, mode, }
                 \CommentTok{\#demographics}
\NormalTok{                 gender, age, language, schoolyear, country,}
                 \CommentTok{\#effort survey}
\NormalTok{                 effort, difficulty, confidence, enjoyment, }
                 \CommentTok{\#explanations}
\NormalTok{                 other,disability,}
                 \CommentTok{\#response characteristics}
\NormalTok{                 totaltime\_m,}
                 \CommentTok{\#exploratory factors}
\NormalTok{                 violations, browser, width, height}
\NormalTok{                 )}


\NormalTok{effort\_labels }\OtherTok{\textless{}{-}} \FunctionTok{c}\NormalTok{(}\StringTok{"I tried my best on each question"}\NormalTok{, }\StringTok{"I tried my best on most questions"}\NormalTok{)}

\CommentTok{\#set factors}
\NormalTok{df\_subjects }\OtherTok{\textless{}{-}}\NormalTok{ df\_subjects }\SpecialCharTok{\%\textgreater{}\%} 
  \CommentTok{\#refactor factors}
  \FunctionTok{mutate}\NormalTok{ (}
    \AttributeTok{subject =} \FunctionTok{factor}\NormalTok{(subject),}
    \AttributeTok{condition =} \FunctionTok{factor}\NormalTok{(condition),}
    \AttributeTok{term =} \FunctionTok{factor}\NormalTok{(term),}
    \AttributeTok{mode =} \FunctionTok{factor}\NormalTok{(mode),}
    \AttributeTok{gender =} \FunctionTok{factor}\NormalTok{(gender),}
    \AttributeTok{schoolyear =} \FunctionTok{factor}\NormalTok{(schoolyear, }\AttributeTok{levels=}\FunctionTok{c}\NormalTok{(}\StringTok{"First"}\NormalTok{,}\StringTok{"Second"}\NormalTok{,}\StringTok{"Third"}\NormalTok{,}\StringTok{"Fourth"}\NormalTok{,}\StringTok{"Fifth"}\NormalTok{,}\StringTok{"Other"}\NormalTok{))}
\NormalTok{  )}
\end{Highlighting}
\end{Shaded}

\hypertarget{items-3}{%
\subsubsection{Items}\label{items-3}}

Next we import item-level data from each data collection period,
selecting only the columns relevant for analysis. The result is a single
data frame \texttt{df\_items} containing one row for each \emph{graph
comprehension task question} (qs=15) (across all periods). A second data
frame \texttt{df\_freeresponse} contains one row for each free response
strategy question (last question posed to participants in Winter2022)
Note that we \emph{do not} discard any \emph{response} data. Rather, we
\emph{do} discard several columns representing accuracy scores for
responses that were calculated in the stimulus engine. These scores were
calculated differently across collection periods, and so we discard them
and recalculate scores in the next analysis notebook. Original response
data are always preserved.

\begin{Shaded}
\begin{Highlighting}[]
\CommentTok{\# }\AlertTok{HACK}\CommentTok{ WD FOR LOCAL RUNNING?}
\CommentTok{\#imac = "/Users/amyraefox/Code/SGC{-}Scaffolding\_Graph\_Comprehension/SGC{-}X/ANALYSIS/MAIN"}
\CommentTok{\# mbp = "/Users/amyfox/Sites/RESEARCH/SGC—Scaffolding Graph Comprehension/SGC{-}X/ANALYSIS/MAIN"}
\CommentTok{\#setwd(imac)}

\CommentTok{\#read datafiles}
\NormalTok{df\_items }\OtherTok{\textless{}{-}} \FunctionTok{read\_rds}\NormalTok{(}\StringTok{"analysis/SGC4B/data/0{-}study{-}level/sgc4b\_items.rds"}\NormalTok{) }\CommentTok{\#use RDS file as it contains metadata}

\CommentTok{\#reduce data collected using new webapp}
\NormalTok{df\_items }\OtherTok{\textless{}{-}}\NormalTok{ df\_items }\SpecialCharTok{\%\textgreater{}\%} 
  \FunctionTok{select}\NormalTok{(subject, condition, pretty\_condition, study, term, mode, question, q, answer, correct, rt\_s) }\SpecialCharTok{\%\textgreater{}\%} \CommentTok{\#unfactor before combine}
  \FunctionTok{mutate}\NormalTok{(}
    \AttributeTok{subject =} \FunctionTok{as.character}\NormalTok{(subject),}
    \AttributeTok{condition =} \FunctionTok{as.character}\NormalTok{(condition),}
    \AttributeTok{term =} \FunctionTok{as.character}\NormalTok{(term),}
    \AttributeTok{mode =} \FunctionTok{as.character}\NormalTok{(mode),}
    \AttributeTok{q =} \FunctionTok{as.integer}\NormalTok{(q),}
    \AttributeTok{correct =} \FunctionTok{as.logical}\NormalTok{(correct)}
\NormalTok{  ) }\SpecialCharTok{\%\textgreater{}\%} 
  \FunctionTok{mutate}\NormalTok{(}
    \AttributeTok{response =} \FunctionTok{str\_remove\_all}\NormalTok{(}\FunctionTok{as.character}\NormalTok{(answer), }\StringTok{","}\NormalTok{),}
    \AttributeTok{num\_o =} \FunctionTok{str\_length}\NormalTok{(response)}
\NormalTok{  )}\SpecialCharTok{\%\textgreater{}\%} 
  \CommentTok{\# handle NA values (why are some empty responses blank and others NA?) }
  \FunctionTok{mutate}\NormalTok{(}
    \AttributeTok{response =} \FunctionTok{replace\_na}\NormalTok{(response, }\StringTok{""}\NormalTok{),}
    \AttributeTok{num\_o =} \FunctionTok{replace\_na}\NormalTok{(num\_o, }\DecValTok{0}\NormalTok{)}
\NormalTok{  )}
\end{Highlighting}
\end{Shaded}

\hypertarget{validation-3}{%
\subsubsection{Validation}\label{validation-3}}

Next, we validate that we have the complete number of item-level records
based on the number of subject-level records

\begin{Shaded}
\begin{Highlighting}[]
\CommentTok{\#the number of items should be equal to 15 x the number of subjects}
\FunctionTok{nrow}\NormalTok{(df\_items) }\SpecialCharTok{==} \DecValTok{15}\SpecialCharTok{*} \FunctionTok{nrow}\NormalTok{(df\_subjects) }\CommentTok{\#TRUE}
\end{Highlighting}
\end{Shaded}

\begin{verbatim}
[1] TRUE
\end{verbatim}

\begin{Shaded}
\begin{Highlighting}[]
\CommentTok{\#each subject should have 15 items}
\NormalTok{df\_items }\SpecialCharTok{\%\textgreater{}\%} \FunctionTok{group\_by}\NormalTok{(subject) }\SpecialCharTok{\%\textgreater{}\%} \FunctionTok{summarise}\NormalTok{(}\AttributeTok{n =} \FunctionTok{n}\NormalTok{()) }\SpecialCharTok{\%\textgreater{}\%} \FunctionTok{filter}\NormalTok{(n }\SpecialCharTok{!=} \DecValTok{15}\NormalTok{) }\SpecialCharTok{\%\textgreater{}\%} \FunctionTok{nrow}\NormalTok{() }\SpecialCharTok{==} \DecValTok{0}
\end{Highlighting}
\end{Shaded}

\begin{verbatim}
[1] TRUE
\end{verbatim}

\hypertarget{export-6}{%
\subsubsection{Export}\label{export-6}}

Finally, we export the (session-harmonized) data for analysis, as CSVs,
and .RDS (includes metadata)

\begin{Shaded}
\begin{Highlighting}[]
\CommentTok{\# }\AlertTok{HACK}\CommentTok{ WD FOR LOCAL RUNNING?}
\CommentTok{\#imac = "/Users/amyraefox/Code/SGC{-}Scaffolding\_Graph\_Comprehension/SGC{-}X/ANALYSIS/MAIN"}
\CommentTok{\# mbp = "/Users/amyfox/Sites/RESEARCH/SGC—Scaffolding Graph Comprehension/SGC{-}X/ANALYSIS/MAIN"}
\CommentTok{\#setwd(imac)}

\CommentTok{\#SAVE FILES}
\FunctionTok{write.csv}\NormalTok{(df\_subjects,}\StringTok{"analysis/SGC4B/data/1{-}study{-}level/sgc4b\_participants.csv"}\NormalTok{, }\AttributeTok{row.names =} \ConstantTok{FALSE}\NormalTok{)}
\FunctionTok{write.csv}\NormalTok{(df\_items,}\StringTok{"analysis/SGC4B/data/1{-}study{-}level/sgc4b\_items.csv"}\NormalTok{, }\AttributeTok{row.names =} \ConstantTok{FALSE}\NormalTok{)}

\CommentTok{\#SAVE R Data Structures }
\CommentTok{\#export R DATA STRUCTURES (include codebook metadata)}
\NormalTok{rio}\SpecialCharTok{::}\FunctionTok{export}\NormalTok{(df\_subjects, }\StringTok{"analysis/SGC4B/data/1{-}study{-}level/sgc4b\_participants.rds"}\NormalTok{) }\CommentTok{\# to R data structure file}
\NormalTok{rio}\SpecialCharTok{::}\FunctionTok{export}\NormalTok{(df\_items, }\StringTok{"analysis/SGC4B/data/1{-}study{-}level/sgc4b\_items.rds"}\NormalTok{) }\CommentTok{\# to R data structure file}
\end{Highlighting}
\end{Shaded}

\hypertarget{resources-9}{%
\section{RESOURCES}\label{resources-9}}

\begin{Shaded}
\begin{Highlighting}[]
\FunctionTok{sessionInfo}\NormalTok{()}
\end{Highlighting}
\end{Shaded}

\begin{verbatim}
R version 4.2.1 (2022-06-23)
Platform: x86_64-apple-darwin17.0 (64-bit)
Running under: macOS Big Sur ... 10.16

Matrix products: default
BLAS:   /Library/Frameworks/R.framework/Versions/4.2/Resources/lib/libRblas.0.dylib
LAPACK: /Library/Frameworks/R.framework/Versions/4.2/Resources/lib/libRlapack.dylib

locale:
[1] en_US.UTF-8/en_US.UTF-8/en_US.UTF-8/C/en_US.UTF-8/en_US.UTF-8

attached base packages:
[1] stats     graphics  grDevices utils     datasets  methods   base     

other attached packages:
 [1] kableExtra_1.3.4 forcats_0.5.1    stringr_1.4.0    dplyr_1.0.9     
 [5] purrr_0.3.4      readr_2.1.2      tidyr_1.2.0      tibble_3.1.7    
 [9] ggplot2_3.3.6    tidyverse_1.3.1  codebook_0.9.2  

loaded via a namespace (and not attached):
 [1] Rcpp_1.0.8.3      svglite_2.1.0     lubridate_1.8.0   assertthat_0.2.1 
 [5] digest_0.6.29     utf8_1.2.2        R6_2.5.1          cellranger_1.1.0 
 [9] backports_1.4.1   reprex_2.0.1      labelled_2.9.1    evaluate_0.15    
[13] httr_1.4.3        pillar_1.7.0      rlang_1.0.3       curl_4.3.2       
[17] readxl_1.4.0      data.table_1.14.2 rstudioapi_0.13   rmarkdown_2.14   
[21] webshot_0.5.3     foreign_0.8-82    munsell_0.5.0     broom_0.8.0      
[25] compiler_4.2.1    modelr_0.1.8      xfun_0.31         pkgconfig_2.0.3  
[29] systemfonts_1.0.4 htmltools_0.5.2   tidyselect_1.1.2  rio_0.5.29       
[33] codetools_0.2-18  fansi_1.0.3       viridisLite_0.4.0 crayon_1.5.1     
[37] tzdb_0.3.0        dbplyr_2.2.1      withr_2.5.0       grid_4.2.1       
[41] jsonlite_1.8.0    gtable_0.3.0      lifecycle_1.0.1   DBI_1.1.3        
[45] magrittr_2.0.3    scales_1.2.0      zip_2.2.0         cli_3.3.0        
[49] stringi_1.7.6     fs_1.5.2          xml2_1.3.3        ellipsis_0.3.2   
[53] generics_0.1.2    vctrs_0.4.1       openxlsx_4.2.5    tools_4.2.1      
[57] glue_1.6.2        hms_1.1.1         fastmap_1.1.0     yaml_2.3.5       
[61] colorspace_2.0-3  rvest_1.0.2       knitr_1.39        haven_2.5.0      
\end{verbatim}

\newpage

\hypertarget{sec-SGC4B-scoring}{%
\chapter{Response Scoring}\label{sec-SGC4B-scoring}}

\emph{The purpose of this notebook is to score (assign a measure of
accuracy) to response data for the SGC4B study. This is required because
the question type on the graph comprehension task used a `Multiple
Response' (MR) question design. Here, we evaluate different approaches
to scoring multiple response questions, and transform raw item responses
(e.g.~boxes ABC are checked) to a measure of response accuracy.
(Warning: this notebook takes several minutes to execute.)} To review
the strategy behind Multiple Response scoring for the SGC project, refer
to section \textbf{?@sec-scoring}.

\begin{Shaded}
\begin{Highlighting}[]
\FunctionTok{options}\NormalTok{(}\AttributeTok{scipen=}\DecValTok{1}\NormalTok{, }\AttributeTok{digits=}\DecValTok{3}\NormalTok{)}

\FunctionTok{library}\NormalTok{(kableExtra) }\CommentTok{\#printing tables }
\FunctionTok{library}\NormalTok{(ggformula) }\CommentTok{\#quick graphs}
\FunctionTok{library}\NormalTok{(pbapply) }\CommentTok{\#progress bar and time estimate for *apply fns}
\FunctionTok{library}\NormalTok{(Hmisc) }\CommentTok{\# \%nin\% operator}
\FunctionTok{library}\NormalTok{(tidyverse) }\CommentTok{\#ALL THE THINGS}
\end{Highlighting}
\end{Shaded}

\hypertarget{score-sgc-data-3}{%
\section{SCORE SGC DATA}\label{score-sgc-data-3}}

To review the strategy behind Multiple Response scoring for the SGC
project, refer to section \textbf{?@sec-scoring}.

In SGC we are fundamentally interested in understanding how a
participant interprets the presented graph (stimulus). The \textbf{graph
comprehension task} asks them to select the data points in the graph
that meet the criteria posed in the question. To assess a participant's
performance, for each question (q=15) we will calculate the following
scores:

\emph{An overall, strict score:}\\
1. \textbf{Absolute Score} : using dichotomous scoring referencing true
(Triangular) answer. (see 1.2)

\emph{Sub-scores, for each alternative graph interpretation}\\
2. \textbf{Triangular Score} : using partial scoring {[}-1/q, +1/p{]}
referencing true (Triangular) answer key.

3. \textbf{Orthogonal Score} : using partial scoring {[}-1/q, +1/p{]}
referencing (incorrect Orthogonal) answer key.

Based on prior observational studies where we observed emergence of
other alternative interpretations (i.e.~transitional interpretations) we
also calculate subscores for these alternatives.

4. \textbf{Tversky Score} : using partial scoring {[}-1/q, +1/p{]}
referencing (incorrect connecting-lines strategy) answer key. 5.
\textbf{Satisficing Score} : using partial scoring {[}-1/q, +1/p{]}
referencing (incorrect satisficing strategy) answer key.

\hypertarget{sec-SGC4B-keys}{%
\subsection{Prepare Answer Keys}\label{sec-SGC4B-keys}}

We start by importing three answer keys: (1) Q1 - Q5 {[}control
condition{]}, (2) Q1-Q5 {[}impasse condition{]}, (3) Q6-15. Separate
answer keys by condition are required for Q1-Q5 because the stimuli for
each condition visualize a different underlying dataset (i.e.~the graphs
show datapoints in different positions). Q6-Q15 are identical across
conditions. Each answer key includes a row for each question, and a
column defining the subset of response options that correspond to
different graph interpretations.

\begin{Shaded}
\begin{Highlighting}[]
\CommentTok{\# \#HACK WD FOR LOCAL RUNNING?}
\CommentTok{\#imac = "/Users/amyraefox/Code/SGC{-}Scaffolding\_Graph\_Comprehension/SGC{-}X/ANALYSIS/MAIN"}
\CommentTok{\#setwd(imac)}

\CommentTok{\#SAVE KEYS FOR FUTURE USE}
\NormalTok{keys\_raw }\OtherTok{\textless{}{-}}  \FunctionTok{read\_csv}\NormalTok{(}\StringTok{"analysis/utils/keys/parsed\_keys/keys\_raw"}\NormalTok{)}
\NormalTok{keys\_orth }\OtherTok{\textless{}{-}}  \FunctionTok{read\_csv}\NormalTok{(}\StringTok{"analysis/utils/keys/parsed\_keys/keys\_orth"}\NormalTok{)}
\NormalTok{keys\_tri }\OtherTok{\textless{}{-}}  \FunctionTok{read\_csv}\NormalTok{(}\StringTok{"analysis/utils/keys/parsed\_keys/keys\_tri"}\NormalTok{)}
\NormalTok{keys\_satisfice\_left }\OtherTok{\textless{}{-}}  \FunctionTok{read\_csv}\NormalTok{(}\StringTok{"analysis/utils/keys/parsed\_keys/keys\_satisfice\_left"}\NormalTok{)}
\NormalTok{keys\_satisfice\_right }\OtherTok{\textless{}{-}}  \FunctionTok{read\_csv}\NormalTok{(}\StringTok{"analysis/utils/keys/parsed\_keys/keys\_satisfice\_right"}\NormalTok{)}
\NormalTok{keys\_tversky\_duration }\OtherTok{\textless{}{-}}  \FunctionTok{read\_csv}\NormalTok{(}\StringTok{"analysis/utils/keys/parsed\_keys/keys\_tversky\_duration"}\NormalTok{)}
\NormalTok{keys\_tversky\_end }\OtherTok{\textless{}{-}}  \FunctionTok{read\_csv}\NormalTok{(}\StringTok{"analysis/utils/keys/parsed\_keys/keys\_tversky\_end"}\NormalTok{)}
\NormalTok{keys\_tversky\_max }\OtherTok{\textless{}{-}}  \FunctionTok{read\_csv}\NormalTok{(}\StringTok{"analysis/utils/keys/parsed\_keys/keys\_tversky\_max"}\NormalTok{)}
\NormalTok{keys\_tversky\_start }\OtherTok{\textless{}{-}}  \FunctionTok{read\_csv}\NormalTok{(}\StringTok{"analysis/utils/keys/parsed\_keys/keys\_tversky\_start"}\NormalTok{)}
\end{Highlighting}
\end{Shaded}

\hypertarget{sec-SGC4B-subscores}{%
\subsection{Calculate Subscores}\label{sec-SGC4B-subscores}}

Next, we import the item-level response data. For each row in the item
level dataset (indicating the response to a single question-item for a
single subject), we compare the raw response
\texttt{df\_items\$response} with the answer keys in each interpretation
(e.g.~\texttt{keys\_orth}, \texttt{keys\_tri}, etc\ldots), then using
those sets, determine the number of correctly selected items(p) and
incorrectly selected items (q), which in turn are used to calculate
partial{[}-1/q, +1/p{]} scores for each interpretation. The resulting
scores are then stored on each item in \texttt{df\_items}, and can be
used to determine which graph interpretation the subject held.

Specifically, the following scores are calculated for each item:

\textbf{Interpretation Subscores}

\begin{itemize}
\tightlist
\item
  \texttt{score\_TRI} How consistent is the response with the
  \textbf{triangular}interpretation?
\item
  \texttt{score\_ORTH} How consistent is the response with the
  \textbf{orthogonal}interpretation?
\item
  \texttt{score\_SATISFICE} is calculated by taking the maximum value of
  :

  \begin{itemize}
  \tightlist
  \item
    \texttt{score\_SAT\_left} How consistent is the response with the
    \textbf{(left side) Satisficing} interpretation?
  \item
    \texttt{score\_SAT\_right} How consistent is the response with the
    \textbf{(right side) Satisficing} interpretation
  \end{itemize}
\item
  \texttt{score\_TVERSKY} is calculated by taking the maximum value of:

  \begin{itemize}
  \tightlist
  \item
    \texttt{score\_TV\_max} How consistent is the response with the
    \textbf{(maximal) Tversky} interpretation?
  \item
    \texttt{score\_TV\_start} How consistent is the response with the
    \textbf{(start-time) Tversky} interpretation?
  \item
    \texttt{score\_TV\_end} How consistent is the response with the
    \textbf{(end-time) Tversky} interpretation?
  \item
    \texttt{score\_TV\_duration} How consistent is the response with the
    \textbf{(duration) Tversky} interpretation?
  \end{itemize}
\item
  \texttt{score\_REF} Did the response select only the \textbf{reference
  point}?
\item
  \texttt{score\_BOTH} How consistent is the response with \textbf{both}
  the orthogonal and triangular interpretations?
\end{itemize}

\textbf{Absolute Scores}

\begin{itemize}
\tightlist
\item
  \texttt{score\_ABS} Is the response strictly correct? (triangular
  interpretation)
\item
  \texttt{score\_niceABS} Is the response strictly correct? (triangular
  interpretation, not penalizing ref points). This is a more generous
  version of the Absolute score that does not penalize the participant
  if in addition to the correct answer \emph{in addition to} they also
  select the data point referenced in the question.
\end{itemize}

\begin{Shaded}
\begin{Highlighting}[]
\CommentTok{\#HACK WD FOR LOCAL RUNNING?}
\NormalTok{imac }\OtherTok{=} \StringTok{"/Users/amyraefox/Code/SGC{-}Scaffolding\_Graph\_Comprehension/SGC{-}X/ANALYSIS/MAIN"}
\FunctionTok{setwd}\NormalTok{(imac)}

\CommentTok{\#backup \textless{}{-} read\_rds(\textquotesingle{}analysis/SGC4B/data/1{-}study{-}level/sgc4b\_items.rds\textquotesingle{}) \#for troubleshooting only}
\NormalTok{df\_items }\OtherTok{\textless{}{-}} \FunctionTok{read\_rds}\NormalTok{(}\StringTok{\textquotesingle{}analysis/SGC4B/data/1{-}study{-}level/sgc4b\_items.rds\textquotesingle{}}\NormalTok{)}
\end{Highlighting}
\end{Shaded}

\begin{Shaded}
\begin{Highlighting}[]
\CommentTok{\# \#HACK WD FOR LOCAL RUNNING?}
\CommentTok{\# imac = "/Users/amyraefox/Code/SGC{-}Scaffolding\_Graph\_Comprehension/SGC{-}X/ANALYSIS/MAIN"}
\CommentTok{\# setwd(imac)}

\FunctionTok{source}\NormalTok{(}\StringTok{"analysis/utils/scoring.R"}\NormalTok{)}
\end{Highlighting}
\end{Shaded}

\emph{note: this cell takes approximately 30 minutes to run on the full
df\_items dataframe with 4950 records}

\begin{Shaded}
\begin{Highlighting}[]
\CommentTok{\#RUN THIS \textless{}OR\textgreater{} THE CALCULATE{-}SCORES{-}FORLOOP [not both]}

\CommentTok{\#ALPHEBETIZE RESPONSE}
\NormalTok{df\_items}\SpecialCharTok{$}\NormalTok{response }\OtherTok{=} \FunctionTok{pbmapply}\NormalTok{(reorder\_inplace, df\_items}\SpecialCharTok{$}\NormalTok{response)}

\CommentTok{\#STRATEGY PARTIAL{-}SUBSCORES}
\NormalTok{df\_items}\SpecialCharTok{$}\NormalTok{score\_TRI }\OtherTok{=} \FunctionTok{pbmapply}\NormalTok{(calc\_subscore, df\_items}\SpecialCharTok{$}\NormalTok{q, df\_items}\SpecialCharTok{$}\NormalTok{condition, df\_items}\SpecialCharTok{$}\NormalTok{response, }\AttributeTok{MoreArgs =} \FunctionTok{list}\NormalTok{(}\AttributeTok{keyframe =}\NormalTok{ keys\_tri))}
\NormalTok{df\_items}\SpecialCharTok{$}\NormalTok{score\_ORTH }\OtherTok{=} \FunctionTok{pbmapply}\NormalTok{(calc\_subscore, df\_items}\SpecialCharTok{$}\NormalTok{q, df\_items}\SpecialCharTok{$}\NormalTok{condition, df\_items}\SpecialCharTok{$}\NormalTok{response, }\AttributeTok{MoreArgs =} \FunctionTok{list}\NormalTok{(}\AttributeTok{keyframe =}\NormalTok{ keys\_orth))}
\NormalTok{df\_items}\SpecialCharTok{$}\NormalTok{score\_SAT\_left }\OtherTok{=} \FunctionTok{pbmapply}\NormalTok{(calc\_subscore, df\_items}\SpecialCharTok{$}\NormalTok{q, df\_items}\SpecialCharTok{$}\NormalTok{condition, df\_items}\SpecialCharTok{$}\NormalTok{response, }\AttributeTok{MoreArgs =} \FunctionTok{list}\NormalTok{(}\AttributeTok{keyframe =}\NormalTok{ keys\_satisfice\_left))}
\NormalTok{df\_items}\SpecialCharTok{$}\NormalTok{score\_SAT\_right }\OtherTok{=} \FunctionTok{pbmapply}\NormalTok{(calc\_subscore, df\_items}\SpecialCharTok{$}\NormalTok{q, df\_items}\SpecialCharTok{$}\NormalTok{condition, df\_items}\SpecialCharTok{$}\NormalTok{response, }\AttributeTok{MoreArgs =} \FunctionTok{list}\NormalTok{(}\AttributeTok{keyframe =}\NormalTok{ keys\_satisfice\_right))}
\NormalTok{df\_items}\SpecialCharTok{$}\NormalTok{score\_TV\_max }\OtherTok{=} \FunctionTok{pbmapply}\NormalTok{(calc\_subscore, df\_items}\SpecialCharTok{$}\NormalTok{q, df\_items}\SpecialCharTok{$}\NormalTok{condition, df\_items}\SpecialCharTok{$}\NormalTok{response, }\AttributeTok{MoreArgs =} \FunctionTok{list}\NormalTok{(}\AttributeTok{keyframe =}\NormalTok{ keys\_tversky\_max))}
\NormalTok{df\_items}\SpecialCharTok{$}\NormalTok{score\_TV\_start }\OtherTok{=} \FunctionTok{pbmapply}\NormalTok{(calc\_subscore, df\_items}\SpecialCharTok{$}\NormalTok{q, df\_items}\SpecialCharTok{$}\NormalTok{condition, df\_items}\SpecialCharTok{$}\NormalTok{response, }\AttributeTok{MoreArgs =} \FunctionTok{list}\NormalTok{(}\AttributeTok{keyframe =}\NormalTok{ keys\_tversky\_start))}
\NormalTok{df\_items}\SpecialCharTok{$}\NormalTok{score\_TV\_end }\OtherTok{=} \FunctionTok{pbmapply}\NormalTok{(calc\_subscore, df\_items}\SpecialCharTok{$}\NormalTok{q, df\_items}\SpecialCharTok{$}\NormalTok{condition, df\_items}\SpecialCharTok{$}\NormalTok{response, }\AttributeTok{MoreArgs =} \FunctionTok{list}\NormalTok{(}\AttributeTok{keyframe =}\NormalTok{ keys\_tversky\_end))}
\NormalTok{df\_items}\SpecialCharTok{$}\NormalTok{score\_TV\_duration }\OtherTok{=} \FunctionTok{pbmapply}\NormalTok{(calc\_subscore, df\_items}\SpecialCharTok{$}\NormalTok{q, df\_items}\SpecialCharTok{$}\NormalTok{condition, df\_items}\SpecialCharTok{$}\NormalTok{response, }\AttributeTok{MoreArgs =} \FunctionTok{list}\NormalTok{(}\AttributeTok{keyframe =}\NormalTok{ keys\_tversky\_duration))}

\CommentTok{\#SPECIAL ABSOLUTE SUBSCORES}
\NormalTok{df\_items}\SpecialCharTok{$}\NormalTok{score\_REF }\OtherTok{=} \FunctionTok{pbmapply}\NormalTok{(calc\_refscore, df\_items}\SpecialCharTok{$}\NormalTok{q, df\_items}\SpecialCharTok{$}\NormalTok{response)}
\NormalTok{df\_items}\SpecialCharTok{$}\NormalTok{score\_BOTH }\OtherTok{=} \FunctionTok{as.integer}\NormalTok{((df\_items}\SpecialCharTok{$}\NormalTok{score\_TRI }\SpecialCharTok{==} \DecValTok{1}\NormalTok{) }\SpecialCharTok{\&}\NormalTok{ (df\_items}\SpecialCharTok{$}\NormalTok{score\_ORTH }\SpecialCharTok{==}\DecValTok{1}\NormalTok{))}

\CommentTok{\#ABSOLUTE SCORES}
\NormalTok{df\_items}\SpecialCharTok{$}\NormalTok{score\_ABS }\OtherTok{=} \FunctionTok{as.integer}\NormalTok{(df\_items}\SpecialCharTok{$}\NormalTok{correct) }
\NormalTok{df\_items}\SpecialCharTok{$}\NormalTok{score\_niceABS  }\OtherTok{\textless{}{-}} \FunctionTok{as.integer}\NormalTok{((df\_items}\SpecialCharTok{$}\NormalTok{score\_TRI }\SpecialCharTok{==} \DecValTok{1}\NormalTok{)) }\CommentTok{\#tri doesn\textquotesingle{}t penalize ref or ve{-}area}
\end{Highlighting}
\end{Shaded}

\hypertarget{sec-SGC4B-interpretation}{%
\subsection{Derive Interpretation}\label{sec-SGC4B-interpretation}}

Finally, we use the interpretation subscores to classify the response as
a particular interpretation. This classification algorithm : (1) First
decides if the response matches one or more `special' situations (blank
response, reference point response, both ORTH and TRI) (2) If response
doesn't match a special situation, it compares the individual subscores,
and subscores and decides if they are \emph{discriminant} (i.e.~are the
scores different enough to make a prediction). A discriminant threshold
of 0.5pts (on a scale from -1 to +1 is used) (2) If the variance in
subscores surpasses the threshold, the interpretation is classified
based on the highest subscore (TRIANGULAR, ORTHOGONAL, TVERSKY,
SATISFICE) (3) If the variance does not surpass the threshold, the
interpretation is labelled as ``?'', indicating it cannot be classified,
and is of an unknown interpretation.

The final output is called \texttt{interpretation}.

\begin{Shaded}
\begin{Highlighting}[]
\CommentTok{\#stoopid extra copying for troubleshooting safety}
\NormalTok{temp }\OtherTok{\textless{}{-}}\NormalTok{ df\_items }
\NormalTok{temp }\OtherTok{\textless{}{-}} \FunctionTok{derive\_interpretation}\NormalTok{(temp)}
\NormalTok{df\_items }\OtherTok{\textless{}{-}}\NormalTok{ temp }
\end{Highlighting}
\end{Shaded}

\hypertarget{sec-SGC4B-scaledScore}{%
\subsection{Derive Scaled Score}\label{sec-SGC4B-scaledScore}}

The \texttt{interpretation} response variable gives us the finest grain
indication of the reader's understanding of the graph for a particular
question. However, it is a categorical variable, which poses a challenge
for analyzing cumulative performance at the subject level. To address
this challenge, we derive a \emph{scaled\_score} that converts each
possible interpretation to a numeric value on a scale from -1 to +1.
This scaling takes advantage of the observation that each interpretation
can be positioned along a spectrum of understanding from completely
incorrect (orthogonal) to completely correct (triangular). Alternative
interpretations lay somewhere between.

Specifically, we assign the following values to each interpretation:

\begin{itemize}
\tightlist
\item
  (-1) : ORTHOGONAL, SATISFICE (satisfice represents an attempt at an
  orthogonal answer when none is available)
\item
  (-0.5): ? (some alternative that cannot be identified; but meaningful
  that it is not orthogonal)
\item
  (0): REFERENCE POINT, BLANK (indicates the individual thinks there is
  no answer, recognizes that ORTHOGONAL cannot be correct, but does not
  conceive of triangular)
\item
  (+0.5) TVERSKY, BOTH TRI + ORTH (indicates that they ``see'' a
  triangular response, but lack certainty and also select the ORTHOGONAL
  response)
\item
  (+1) TRIANGULAR +1
\end{itemize}

\begin{Shaded}
\begin{Highlighting}[]
\NormalTok{df\_items}\SpecialCharTok{$}\NormalTok{score\_SCALED }\OtherTok{\textless{}{-}} \FunctionTok{calc\_scaled}\NormalTok{(df\_items}\SpecialCharTok{$}\NormalTok{interpretation)}
\end{Highlighting}
\end{Shaded}

\hypertarget{summarize-by-subject-3}{%
\section{SUMMARIZE BY SUBJECT}\label{summarize-by-subject-3}}

Next, we summarize the item level scores by subject in order to
calculate cummulative subscores to be stored on the subject record.

\begin{Shaded}
\begin{Highlighting}[]
\CommentTok{\# \#HACK WD FOR LOCAL RUNNING?}
\NormalTok{imac }\OtherTok{=} \StringTok{"/Users/amyraefox/Code/SGC{-}Scaffolding\_Graph\_Comprehension/SGC{-}X/ANALYSIS/MAIN"}
\FunctionTok{setwd}\NormalTok{(imac)}

\CommentTok{\#import subjects}
\NormalTok{df\_subjects }\OtherTok{\textless{}{-}} \FunctionTok{read\_rds}\NormalTok{(}\StringTok{\textquotesingle{}analysis/SGC4B/data/1{-}study{-}level/sgc4b\_participants.rds\textquotesingle{}}\NormalTok{) }\SpecialCharTok{\%\textgreater{}\%} \FunctionTok{mutate}\NormalTok{(}\AttributeTok{subject =} \FunctionTok{as.character}\NormalTok{(subject)) }\SpecialCharTok{\%\textgreater{}\%} \FunctionTok{arrange}\NormalTok{(subject)}

\CommentTok{\#make temporary copies for testing safety}
\NormalTok{s }\OtherTok{=}\NormalTok{ df\_subjects}
\NormalTok{i }\OtherTok{=}\NormalTok{ df\_items }

\CommentTok{\#summarize}
\NormalTok{test\_subs }\OtherTok{\textless{}{-}} \FunctionTok{summarise\_bySubject}\NormalTok{(s,i)}
\end{Highlighting}
\end{Shaded}

\begin{verbatim}
`summarise()` has grouped output by 'subject'. You can override using the
`.groups` argument.
\end{verbatim}

\begin{verbatim}
[1] TRUE
[1] TRUE
[1] TRUE
[1] TRUE
[1] TRUE
[1] TRUE
[1] TRUE
\end{verbatim}

\begin{Shaded}
\begin{Highlighting}[]
\NormalTok{df\_subjects }\OtherTok{\textless{}{-}}\NormalTok{ test\_subs}
\end{Highlighting}
\end{Shaded}

We also summarize absolute and scaled score progress at each question in
the task, to explore cumulative performance over the task.

\begin{Shaded}
\begin{Highlighting}[]
\CommentTok{\#GET ABSOLUTE PROGRESS }
\NormalTok{df\_absolute\_progress }\OtherTok{\textless{}{-}} \FunctionTok{progress\_Absolute}\NormalTok{(df\_items)}

\CommentTok{\#GET SCALED PROGRESS}
\NormalTok{df\_scaled\_progress }\OtherTok{\textless{}{-}} \FunctionTok{progress\_Scaled}\NormalTok{(df\_items)}
\end{Highlighting}
\end{Shaded}

\hypertarget{explore-distributions-3}{%
\section{EXPLORE DISTRIBUTIONS}\label{explore-distributions-3}}

\begin{Shaded}
\begin{Highlighting}[]
\FunctionTok{options}\NormalTok{(}\AttributeTok{repr.plot.width =}\DecValTok{9}\NormalTok{, }\AttributeTok{repr.plot.height =}\DecValTok{12}\NormalTok{)}

\CommentTok{\#create temp data frame for visualizations}
\NormalTok{df }\OtherTok{=}\NormalTok{ df\_items }\SpecialCharTok{\%\textgreater{}\%} \FunctionTok{filter}\NormalTok{ (q }\SpecialCharTok{\%nin\%} \FunctionTok{c}\NormalTok{(}\DecValTok{6}\NormalTok{,}\DecValTok{9}\NormalTok{)) }\SpecialCharTok{\%\textgreater{}\%} \FunctionTok{mutate}\NormalTok{(}
  \AttributeTok{pretty\_condition =}\NormalTok{ pretty\_condition,}
  \AttributeTok{score\_niceABS =} \FunctionTok{as.factor}\NormalTok{(score\_niceABS),}
  \AttributeTok{pretty\_interpretation =} \FunctionTok{factor}\NormalTok{(interpretation,}
    \AttributeTok{levels =} \FunctionTok{c}\NormalTok{(}\StringTok{"Orthogonal"}\NormalTok{, }\StringTok{"Satisfice"}\NormalTok{, }
               \StringTok{"frenzy"}\NormalTok{,}\StringTok{"?"}\NormalTok{,}
                \StringTok{"reference"}\NormalTok{,}\StringTok{"blank"}\NormalTok{,}
                \StringTok{"Tversky"}\NormalTok{, }\StringTok{"both tri + orth"}\NormalTok{,}
               \StringTok{"Triangular"}\NormalTok{ ))}
\NormalTok{  )}
\end{Highlighting}
\end{Shaded}

\hypertarget{absolute-score-4}{%
\subsection{Absolute Score}\label{absolute-score-4}}

\begin{Shaded}
\begin{Highlighting}[]
\CommentTok{\#DISTRIBUTION ABSOLUTE SCORE FULL }\AlertTok{TASK}
\FunctionTok{gf\_props}\NormalTok{(}\SpecialCharTok{\textasciitilde{}}\NormalTok{score\_niceABS, }\AttributeTok{fill =} \SpecialCharTok{\textasciitilde{}}\NormalTok{pretty\_condition, }\AttributeTok{position =} \FunctionTok{position\_dodge}\NormalTok{(), }\AttributeTok{data =}\NormalTok{ df) }\SpecialCharTok{+}
  \FunctionTok{labs}\NormalTok{( }\AttributeTok{x =} \StringTok{"Absolute Score"}\NormalTok{, }
        \AttributeTok{title =} \StringTok{"Distribution of Absolute Score (all Items)"}\NormalTok{,}
        \AttributeTok{subtitle =} \FunctionTok{paste}\NormalTok{(}\StringTok{""}\NormalTok{),}
        \AttributeTok{y =} \StringTok{"Proportion of Items"}\NormalTok{) }\SpecialCharTok{+}
  \FunctionTok{scale\_fill\_discrete}\NormalTok{(}\AttributeTok{name =} \StringTok{"Condition"}\NormalTok{) }\SpecialCharTok{+}  
  \FunctionTok{theme\_minimal}\NormalTok{()}
\end{Highlighting}
\end{Shaded}

\begin{figure}[H]

{\centering \includegraphics{analysis/SGC4B/2_sgc4B_scoring_files/figure-pdf/DISTR-ABSCORE-1.pdf}

}

\end{figure}

\begin{Shaded}
\begin{Highlighting}[]
\CommentTok{\#DISTRIBUTION ABSOLUTE SCORE BY ITEM}
\FunctionTok{gf\_props}\NormalTok{(}\SpecialCharTok{\textasciitilde{}}\NormalTok{score\_niceABS, }\AttributeTok{fill =} \SpecialCharTok{\textasciitilde{}}\NormalTok{pretty\_condition, }\AttributeTok{position =} \FunctionTok{position\_dodge}\NormalTok{(), }\AttributeTok{data =}\NormalTok{ df)  }\SpecialCharTok{\%\textgreater{}\%}
  \FunctionTok{gf\_facet\_grid}\NormalTok{(pretty\_condition}\SpecialCharTok{\textasciitilde{}}\NormalTok{q) }\SpecialCharTok{+}
  \FunctionTok{labs}\NormalTok{( }\AttributeTok{x =} \StringTok{"Absolute Score"}\NormalTok{,}
        \AttributeTok{title =} \StringTok{"Distribution of Absolute Score (by Item)"}\NormalTok{,}
        \AttributeTok{subtitle =} \StringTok{""}\NormalTok{,}
        \AttributeTok{y =} \StringTok{"Proprition of Subjects"}\NormalTok{) }\SpecialCharTok{+}
  \FunctionTok{scale\_fill\_discrete}\NormalTok{(}\AttributeTok{name =} \StringTok{"Condition"}\NormalTok{) }\SpecialCharTok{+}
  \FunctionTok{theme\_minimal}\NormalTok{()}
\end{Highlighting}
\end{Shaded}

\begin{figure}[H]

{\centering \includegraphics{analysis/SGC4B/2_sgc4B_scoring_files/figure-pdf/DISTR-ABSCORE-2.pdf}

}

\end{figure}

\begin{Shaded}
\begin{Highlighting}[]
\CommentTok{\#DISTRIBUTION ABSOLUTE SCORE BY SUBJECT}
\FunctionTok{gf\_props}\NormalTok{(}\SpecialCharTok{\textasciitilde{}}\NormalTok{s\_NABS, }\AttributeTok{fill =} \SpecialCharTok{\textasciitilde{}}\NormalTok{pretty\_condition, }\AttributeTok{position =} \FunctionTok{position\_dodge}\NormalTok{(), }\AttributeTok{data =}\NormalTok{ df\_subjects) }\SpecialCharTok{\%\textgreater{}\%}
\FunctionTok{gf\_facet\_grid}\NormalTok{(pretty\_condition }\SpecialCharTok{\textasciitilde{}}\NormalTok{. )}\SpecialCharTok{+}
  \FunctionTok{labs}\NormalTok{( }\AttributeTok{x =} \StringTok{"Total Absolute Score"}\NormalTok{,}
        \AttributeTok{title =} \StringTok{"Distribution of Total Absolute Score (by Subject)"}\NormalTok{,}
        \AttributeTok{subtitle =} \StringTok{""}\NormalTok{,}
        \AttributeTok{y =} \StringTok{"Proportion of Subjects"}\NormalTok{) }\SpecialCharTok{+}
  \FunctionTok{scale\_fill\_discrete}\NormalTok{(}\AttributeTok{name =} \StringTok{"Condition"}\NormalTok{) }\SpecialCharTok{+}
  \FunctionTok{theme\_minimal}\NormalTok{() }\SpecialCharTok{+} \FunctionTok{theme}\NormalTok{(}\AttributeTok{legend.position =} \StringTok{"blank"}\NormalTok{)}
\end{Highlighting}
\end{Shaded}

\begin{figure}[H]

{\centering \includegraphics{analysis/SGC4B/2_sgc4B_scoring_files/figure-pdf/DISTR-ABSCORE-3.pdf}

}

\end{figure}

\hypertarget{scaled-score-4}{%
\subsection{Scaled Score}\label{scaled-score-4}}

\begin{Shaded}
\begin{Highlighting}[]
\FunctionTok{options}\NormalTok{(}\AttributeTok{repr.plot.width =}\DecValTok{9}\NormalTok{, }\AttributeTok{repr.plot.height =}\DecValTok{12}\NormalTok{)}

\CommentTok{\#DISTRIBUTION SCALED SCORE FULL }\AlertTok{TASK}
\FunctionTok{gf\_props}\NormalTok{(}\SpecialCharTok{\textasciitilde{}}\NormalTok{score\_SCALED, }\AttributeTok{fill =} \SpecialCharTok{\textasciitilde{}}\NormalTok{pretty\_condition, }\AttributeTok{position =} \FunctionTok{position\_dodge}\NormalTok{(), }\AttributeTok{data =}\NormalTok{ df) }\SpecialCharTok{+}
  \FunctionTok{labs}\NormalTok{( }\AttributeTok{x =} \StringTok{"Scaled Score"}\NormalTok{, }
        \AttributeTok{title =} \StringTok{"Distribution of Scaled Score (all Items)"}\NormalTok{,}
        \AttributeTok{subtitle =} \StringTok{""}\NormalTok{,}
        \AttributeTok{y =} \StringTok{"Proportion of Items"}\NormalTok{) }\SpecialCharTok{+}
  \FunctionTok{scale\_fill\_discrete}\NormalTok{(}\AttributeTok{name =} \StringTok{"Condition"}\NormalTok{) }\SpecialCharTok{+}  
  \FunctionTok{theme\_minimal}\NormalTok{()}
\end{Highlighting}
\end{Shaded}

\begin{figure}[H]

{\centering \includegraphics{analysis/SGC4B/2_sgc4B_scoring_files/figure-pdf/DISTR-SCALEDSCORE-1.pdf}

}

\end{figure}

\begin{Shaded}
\begin{Highlighting}[]
\CommentTok{\#DISTRIBUTION SCALED SCORE BY ITEM}
\FunctionTok{gf\_props}\NormalTok{(}\SpecialCharTok{\textasciitilde{}}\NormalTok{score\_SCALED, }\AttributeTok{fill =} \SpecialCharTok{\textasciitilde{}}\NormalTok{pretty\_condition, }\AttributeTok{position =} \FunctionTok{position\_dodge}\NormalTok{(), }\AttributeTok{data =}\NormalTok{ df)  }\SpecialCharTok{\%\textgreater{}\%}
  \FunctionTok{gf\_facet\_grid}\NormalTok{(q}\SpecialCharTok{\textasciitilde{}}\NormalTok{pretty\_condition) }\SpecialCharTok{+}
  \FunctionTok{labs}\NormalTok{( }\AttributeTok{x =} \StringTok{"Scaled Score"}\NormalTok{,}
        \AttributeTok{title =} \StringTok{"Distribution of Scaled Score (by Item)"}\NormalTok{,}
        \AttributeTok{subtitle =} \StringTok{""}\NormalTok{,}
        \AttributeTok{y =} \StringTok{"Proportion of Subjects"}\NormalTok{) }\SpecialCharTok{+}
  \FunctionTok{scale\_fill\_discrete}\NormalTok{(}\AttributeTok{name =} \StringTok{"Condition"}\NormalTok{) }\SpecialCharTok{+}  \FunctionTok{scale\_y\_continuous}\NormalTok{(}\AttributeTok{breaks=}\FunctionTok{c}\NormalTok{(}\DecValTok{0}\NormalTok{,}\FloatTok{0.5}\NormalTok{)) }\SpecialCharTok{+}
  \FunctionTok{theme\_minimal}\NormalTok{() }\SpecialCharTok{+} \FunctionTok{theme}\NormalTok{(}\AttributeTok{legend.position=}\StringTok{"blank"}\NormalTok{)}
\end{Highlighting}
\end{Shaded}

\begin{figure}[H]

{\centering \includegraphics{analysis/SGC4B/2_sgc4B_scoring_files/figure-pdf/DISTR-SCALEDSCORE-2.pdf}

}

\end{figure}

\begin{Shaded}
\begin{Highlighting}[]
\CommentTok{\#DISTRIBUTION SCALED SCORE BY SUBJECT}
\FunctionTok{gf\_props}\NormalTok{(}\SpecialCharTok{\textasciitilde{}}\NormalTok{s\_SCALED, }\AttributeTok{fill =} \SpecialCharTok{\textasciitilde{}}\NormalTok{pretty\_condition, }\AttributeTok{data =}\NormalTok{ df\_subjects)  }\SpecialCharTok{\%\textgreater{}\%}
  \FunctionTok{gf\_facet\_grid}\NormalTok{(pretty\_condition }\SpecialCharTok{\textasciitilde{}}\NormalTok{. )}\SpecialCharTok{+}
  \FunctionTok{labs}\NormalTok{( }\AttributeTok{x =} \StringTok{"Total Scaled Score"}\NormalTok{,}
        \AttributeTok{title =} \StringTok{"Distribution of Total Scaled Score (by Subject)"}\NormalTok{,}
        \AttributeTok{subtitle =} \StringTok{""}\NormalTok{,}
        \AttributeTok{y =} \StringTok{"Number of Subjects"}\NormalTok{) }\SpecialCharTok{+}
  \FunctionTok{scale\_fill\_discrete}\NormalTok{(}\AttributeTok{name =} \StringTok{"Condition"}\NormalTok{) }\SpecialCharTok{+}
  \FunctionTok{theme\_minimal}\NormalTok{()}
\end{Highlighting}
\end{Shaded}

\begin{figure}[H]

{\centering \includegraphics{analysis/SGC4B/2_sgc4B_scoring_files/figure-pdf/DISTR-SCALEDSCORE-3.pdf}

}

\end{figure}

\hypertarget{interpretations-3}{%
\subsection{Interpretations}\label{interpretations-3}}

\begin{Shaded}
\begin{Highlighting}[]
\CommentTok{\#DISTRIBUTION OF INTERPRETATION}
\FunctionTok{gf\_props}\NormalTok{(}\SpecialCharTok{\textasciitilde{}}\NormalTok{pretty\_interpretation, }\AttributeTok{fill =} \SpecialCharTok{\textasciitilde{}}\NormalTok{pretty\_condition, }\AttributeTok{data =}\NormalTok{ df) }\SpecialCharTok{\%\textgreater{}\%} 
  \FunctionTok{gf\_facet\_grid}\NormalTok{( pretty\_condition }\SpecialCharTok{\textasciitilde{}}\NormalTok{ ., }\AttributeTok{labeller =}\NormalTok{ label\_both) }\SpecialCharTok{+} 
  \FunctionTok{labs}\NormalTok{( }\AttributeTok{title =} \StringTok{"Distribution of Interpretations (across Task)"}\NormalTok{,}
        \AttributeTok{x =} \StringTok{"Graph Interpretation"}\NormalTok{,}
        \AttributeTok{y =} \StringTok{"Proportion of Responses"}\NormalTok{,}
        \AttributeTok{subtitle =} \StringTok{""}\NormalTok{) }\SpecialCharTok{+} 
  \FunctionTok{theme\_minimal}\NormalTok{() }\SpecialCharTok{+} \FunctionTok{theme}\NormalTok{(}\AttributeTok{legend.position =} \StringTok{"blank"}\NormalTok{)}
\end{Highlighting}
\end{Shaded}

\begin{figure}[H]

{\centering \includegraphics{analysis/SGC4B/2_sgc4B_scoring_files/figure-pdf/DISTR-INTERPRETATIONS-1.pdf}

}

\end{figure}

\begin{Shaded}
\begin{Highlighting}[]
\CommentTok{\#DISTRIBUTION OF INTERPRETATION ACROSS ITEMS}
\FunctionTok{gf\_propsh}\NormalTok{(}\SpecialCharTok{\textasciitilde{}}\NormalTok{ pretty\_interpretation, }\AttributeTok{fill =} \SpecialCharTok{\textasciitilde{}}\NormalTok{pretty\_condition, }\AttributeTok{data =}\NormalTok{ df) }\SpecialCharTok{\%\textgreater{}\%} 
  \FunctionTok{gf\_facet\_grid}\NormalTok{( pretty\_condition}\SpecialCharTok{\textasciitilde{}}\NormalTok{q) }\SpecialCharTok{+} 
  \FunctionTok{labs}\NormalTok{( }\AttributeTok{title =} \StringTok{"Distribution of Interpretations (by Item)"}\NormalTok{,}
        \AttributeTok{subtitle =} \StringTok{""}\NormalTok{,}
        \AttributeTok{y =} \StringTok{"Interpretation"}\NormalTok{, }\AttributeTok{x =} \StringTok{"Proportion of Subjects"}\NormalTok{) }\SpecialCharTok{+} \FunctionTok{theme\_minimal}\NormalTok{() }\SpecialCharTok{+} \FunctionTok{theme}\NormalTok{(}\AttributeTok{legend.position =} \StringTok{"blank"}\NormalTok{)}
\end{Highlighting}
\end{Shaded}

\begin{figure}[H]

{\centering \includegraphics{analysis/SGC4B/2_sgc4B_scoring_files/figure-pdf/DISTR-INTERPRETATIONS-2.pdf}

}

\end{figure}

\begin{Shaded}
\begin{Highlighting}[]
\CommentTok{\#DISTRIBUTION OF INTERPRETATION TYPE ACROSS ITEMS}
\FunctionTok{gf\_propsh}\NormalTok{(}\SpecialCharTok{\textasciitilde{}}\NormalTok{ high\_interpretation, }\AttributeTok{fill =} \SpecialCharTok{\textasciitilde{}}\NormalTok{pretty\_condition, }\AttributeTok{data =}\NormalTok{ df) }\SpecialCharTok{\%\textgreater{}\%} 
  \FunctionTok{gf\_facet\_grid}\NormalTok{( pretty\_condition}\SpecialCharTok{\textasciitilde{}}\NormalTok{q) }\SpecialCharTok{+} 
  \FunctionTok{labs}\NormalTok{( }\AttributeTok{title =} \StringTok{"Distribution of Interpretations (by Item)"}\NormalTok{,}
        \AttributeTok{subtitle =} \StringTok{""}\NormalTok{,}
        \AttributeTok{y =} \StringTok{"Interpretation"}\NormalTok{, }\AttributeTok{x =} \StringTok{"Proportion of Subjects"}\NormalTok{) }\SpecialCharTok{+} \FunctionTok{theme\_minimal}\NormalTok{() }\SpecialCharTok{+} \FunctionTok{theme}\NormalTok{(}\AttributeTok{legend.position =} \StringTok{"blank"}\NormalTok{)}
\end{Highlighting}
\end{Shaded}

\begin{figure}[H]

{\centering \includegraphics{analysis/SGC4B/2_sgc4B_scoring_files/figure-pdf/DISTR-INTERPRETATIONS-3.pdf}

}

\end{figure}

\hypertarget{progress-over-task-3}{%
\subsection{Progress over Task}\label{progress-over-task-3}}

\begin{Shaded}
\begin{Highlighting}[]
\CommentTok{\#VISUALIZE progress over time ABSOLUTE score }
\FunctionTok{ggplot}\NormalTok{(}\AttributeTok{data =}\NormalTok{ df\_absolute\_progress, }\FunctionTok{aes}\NormalTok{(}\AttributeTok{x =}\NormalTok{ question, }\AttributeTok{y =}\NormalTok{ score, }\AttributeTok{group =}\NormalTok{ subject, }\AttributeTok{alpha =} \FloatTok{0.01}\NormalTok{, }\AttributeTok{color =}\NormalTok{ pretty\_condition)) }\SpecialCharTok{+} 
 \FunctionTok{geom\_line}\NormalTok{(}\AttributeTok{position=}\FunctionTok{position\_jitter}\NormalTok{(}\AttributeTok{w=}\FloatTok{0.15}\NormalTok{, }\AttributeTok{h=}\FloatTok{0.15}\NormalTok{), }\AttributeTok{size=}\FloatTok{0.1}\NormalTok{) }\SpecialCharTok{+}
 \FunctionTok{facet\_wrap}\NormalTok{(}\SpecialCharTok{\textasciitilde{}}\NormalTok{pretty\_condition) }\SpecialCharTok{+} 
 \FunctionTok{labs}\NormalTok{ (}\AttributeTok{title =} \StringTok{"Cumulative Absolute Score over sequence of task"}\NormalTok{, }\AttributeTok{x =} \StringTok{"Question"}\NormalTok{ , }\AttributeTok{y =} \StringTok{"Cumulative Absolute Score"}\NormalTok{) }\SpecialCharTok{+} 
 \FunctionTok{scale\_x\_continuous}\NormalTok{(}\AttributeTok{breaks =} \FunctionTok{c}\NormalTok{(}\DecValTok{1}\NormalTok{,}\DecValTok{2}\NormalTok{,}\DecValTok{3}\NormalTok{,}\DecValTok{4}\NormalTok{,}\DecValTok{5}\NormalTok{,}\DecValTok{6}\NormalTok{,}\DecValTok{7}\NormalTok{,}\DecValTok{8}\NormalTok{,}\DecValTok{9}\NormalTok{,}\DecValTok{10}\NormalTok{,}\DecValTok{11}\NormalTok{,}\DecValTok{12}\NormalTok{,}\DecValTok{13}\NormalTok{)) }\SpecialCharTok{+}
 \FunctionTok{theme\_minimal}\NormalTok{() }\SpecialCharTok{+} \FunctionTok{theme}\NormalTok{(}\AttributeTok{legend.position =} \StringTok{"blank"}\NormalTok{)}
\end{Highlighting}
\end{Shaded}

\begin{figure}[H]

{\centering \includegraphics{analysis/SGC4B/2_sgc4B_scoring_files/figure-pdf/VIZ-PROGRESS-1.pdf}

}

\end{figure}

\begin{Shaded}
\begin{Highlighting}[]
\CommentTok{\#VISUALIZE progress over time SCALED score }
\FunctionTok{ggplot}\NormalTok{(}\AttributeTok{data =}\NormalTok{ df\_scaled\_progress, }\FunctionTok{aes}\NormalTok{(}\AttributeTok{x =}\NormalTok{ question, }\AttributeTok{y =}\NormalTok{ score, }\AttributeTok{group =}\NormalTok{ subject, }\AttributeTok{alpha =} \FloatTok{0.01}\NormalTok{, }\AttributeTok{color =}\NormalTok{ pretty\_condition)) }\SpecialCharTok{+} 
 \FunctionTok{geom\_line}\NormalTok{(}\AttributeTok{position=}\FunctionTok{position\_jitter}\NormalTok{(}\AttributeTok{w=}\FloatTok{0.15}\NormalTok{, }\AttributeTok{h=}\FloatTok{0.15}\NormalTok{), }\AttributeTok{size=}\FloatTok{0.1}\NormalTok{) }\SpecialCharTok{+}
 \FunctionTok{facet\_wrap}\NormalTok{(}\SpecialCharTok{\textasciitilde{}}\NormalTok{pretty\_condition) }\SpecialCharTok{+} 
 \FunctionTok{labs}\NormalTok{ (}\AttributeTok{title =} \StringTok{"Cumulative Scaled Score over sequence of task"}\NormalTok{, }\AttributeTok{x =} \StringTok{"Question"}\NormalTok{ , }\AttributeTok{y =} \StringTok{"Cumulative Scaled Score"}\NormalTok{) }\SpecialCharTok{+} 
 \FunctionTok{scale\_x\_continuous}\NormalTok{(}\AttributeTok{breaks =} \FunctionTok{c}\NormalTok{(}\DecValTok{1}\NormalTok{,}\DecValTok{2}\NormalTok{,}\DecValTok{3}\NormalTok{,}\DecValTok{4}\NormalTok{,}\DecValTok{5}\NormalTok{,}\DecValTok{6}\NormalTok{,}\DecValTok{7}\NormalTok{,}\DecValTok{8}\NormalTok{,}\DecValTok{9}\NormalTok{,}\DecValTok{10}\NormalTok{,}\DecValTok{11}\NormalTok{,}\DecValTok{12}\NormalTok{,}\DecValTok{13}\NormalTok{)) }\SpecialCharTok{+}
 \FunctionTok{theme\_minimal}\NormalTok{() }\SpecialCharTok{+} \FunctionTok{theme}\NormalTok{(}\AttributeTok{legend.position =} \StringTok{"blank"}\NormalTok{)}
\end{Highlighting}
\end{Shaded}

\begin{figure}[H]

{\centering \includegraphics{analysis/SGC4B/2_sgc4B_scoring_files/figure-pdf/VIZ-PROGRESS-2.pdf}

}

\end{figure}

\hypertarget{interpretation-subscores-3}{%
\subsection{Interpretation Subscores}\label{interpretation-subscores-3}}

\begin{Shaded}
\begin{Highlighting}[]
\FunctionTok{gf\_density}\NormalTok{(}\SpecialCharTok{\textasciitilde{}}\NormalTok{ s\_TRI, }\AttributeTok{fill =} \SpecialCharTok{\textasciitilde{}}\NormalTok{pretty\_condition, }\AttributeTok{data =}\NormalTok{ df\_subjects) }\SpecialCharTok{\%\textgreater{}\%}
  \FunctionTok{gf\_facet\_wrap}\NormalTok{( }\SpecialCharTok{\textasciitilde{}}\NormalTok{ pretty\_condition) }\SpecialCharTok{+}
  \FunctionTok{labs}\NormalTok{( }\AttributeTok{title =} \StringTok{"Distribution of Total Triangular Score"}\NormalTok{,}
        \AttributeTok{subtitle =} \StringTok{""}\NormalTok{,}
        \AttributeTok{x =} \StringTok{"Item Triangular Score"}\NormalTok{, }\AttributeTok{y =} \StringTok{"Proportion of Subjects"}\NormalTok{) }\SpecialCharTok{+}
        \FunctionTok{theme\_minimal}\NormalTok{() }\SpecialCharTok{+} \FunctionTok{theme}\NormalTok{(}\AttributeTok{legend.position =} \StringTok{"blank"}\NormalTok{)}
\end{Highlighting}
\end{Shaded}

\begin{figure}[H]

{\centering \includegraphics{analysis/SGC4B/2_sgc4B_scoring_files/figure-pdf/DIST-SUBSCORES-1.pdf}

}

\end{figure}

\begin{Shaded}
\begin{Highlighting}[]
\FunctionTok{gf\_density}\NormalTok{(}\SpecialCharTok{\textasciitilde{}}\NormalTok{ s\_ORTH, }\AttributeTok{fill =} \SpecialCharTok{\textasciitilde{}}\NormalTok{pretty\_condition, }\AttributeTok{data =}\NormalTok{ df\_subjects) }\SpecialCharTok{\%\textgreater{}\%}
  \FunctionTok{gf\_facet\_wrap}\NormalTok{( }\SpecialCharTok{\textasciitilde{}}\NormalTok{ pretty\_condition) }\SpecialCharTok{+}
  \FunctionTok{labs}\NormalTok{( }\AttributeTok{title =} \StringTok{"Distribution of Total Orthogonal Score"}\NormalTok{,}
        \AttributeTok{subtitle =} \StringTok{""}\NormalTok{,}
        \AttributeTok{x =} \StringTok{"Item Orthogonal Score"}\NormalTok{, }\AttributeTok{y =} \StringTok{"Proportion of Subjects"}\NormalTok{) }\SpecialCharTok{+}
        \FunctionTok{theme\_minimal}\NormalTok{() }\SpecialCharTok{+} \FunctionTok{theme}\NormalTok{(}\AttributeTok{legend.position =} \StringTok{"blank"}\NormalTok{)}
\end{Highlighting}
\end{Shaded}

\begin{figure}[H]

{\centering \includegraphics{analysis/SGC4B/2_sgc4B_scoring_files/figure-pdf/DIST-SUBSCORES-2.pdf}

}

\end{figure}

\begin{Shaded}
\begin{Highlighting}[]
\FunctionTok{gf\_density}\NormalTok{(}\SpecialCharTok{\textasciitilde{}}\NormalTok{ s\_TVERSKY, }\AttributeTok{fill =} \SpecialCharTok{\textasciitilde{}}\NormalTok{pretty\_condition, }\AttributeTok{data =}\NormalTok{ df\_subjects) }\SpecialCharTok{\%\textgreater{}\%}
  \FunctionTok{gf\_facet\_wrap}\NormalTok{( }\SpecialCharTok{\textasciitilde{}}\NormalTok{ pretty\_condition) }\SpecialCharTok{+}
  \FunctionTok{labs}\NormalTok{( }\AttributeTok{title =} \StringTok{"Distribution of Total Tversky Score"}\NormalTok{,}
        \AttributeTok{subtitle =} \StringTok{""}\NormalTok{,}
        \AttributeTok{x =} \StringTok{"Item Orthogonal Score"}\NormalTok{, }\AttributeTok{y =} \StringTok{"Proportion of Subjects"}\NormalTok{) }\SpecialCharTok{+}
        \FunctionTok{theme\_minimal}\NormalTok{() }\SpecialCharTok{+} \FunctionTok{theme}\NormalTok{(}\AttributeTok{legend.position =} \StringTok{"blank"}\NormalTok{)}
\end{Highlighting}
\end{Shaded}

\begin{figure}[H]

{\centering \includegraphics{analysis/SGC4B/2_sgc4B_scoring_files/figure-pdf/DIST-SUBSCORES-3.pdf}

}

\end{figure}

\begin{Shaded}
\begin{Highlighting}[]
\FunctionTok{gf\_histogram}\NormalTok{(}\SpecialCharTok{\textasciitilde{}}\NormalTok{ s\_SATISFICE, }\AttributeTok{fill =} \SpecialCharTok{\textasciitilde{}}\NormalTok{pretty\_condition, }\AttributeTok{data =}\NormalTok{ df\_subjects) }\SpecialCharTok{\%\textgreater{}\%}
  \FunctionTok{gf\_facet\_wrap}\NormalTok{( }\SpecialCharTok{\textasciitilde{}}\NormalTok{ pretty\_condition) }\SpecialCharTok{+}
  \FunctionTok{labs}\NormalTok{( }\AttributeTok{title =} \StringTok{"Distribution of Total Satisfice Score"}\NormalTok{,}
        \AttributeTok{subtitle =} \StringTok{""}\NormalTok{,}
        \AttributeTok{x =} \StringTok{"Item Orthogonal Score"}\NormalTok{, }\AttributeTok{y =} \StringTok{"Proportion of Subjects"}\NormalTok{) }\SpecialCharTok{+}
        \FunctionTok{theme\_minimal}\NormalTok{() }\SpecialCharTok{+} \FunctionTok{theme}\NormalTok{(}\AttributeTok{legend.position =} \StringTok{"blank"}\NormalTok{)}
\end{Highlighting}
\end{Shaded}

\begin{figure}[H]

{\centering \includegraphics{analysis/SGC4B/2_sgc4B_scoring_files/figure-pdf/DIST-SUBSCORES-4.pdf}

}

\end{figure}

\hypertarget{peeking-1}{%
\section{PEEKING}\label{peeking-1}}

\begin{Shaded}
\begin{Highlighting}[]
\FunctionTok{library}\NormalTok{(performance)}
\FunctionTok{library}\NormalTok{(report)}
\NormalTok{m1 }\OtherTok{\textless{}{-}} \FunctionTok{lm}\NormalTok{(s\_SCALED }\SpecialCharTok{\textasciitilde{}}\NormalTok{ pretty\_condition, }\AttributeTok{data =}\NormalTok{ df\_subjects)}
\FunctionTok{summary}\NormalTok{(m1)}
\end{Highlighting}
\end{Shaded}

\begin{verbatim}

Call:
lm(formula = s_SCALED ~ pretty_condition, data = df_subjects)

Residuals:
   Min     1Q Median     3Q    Max 
 -8.55  -5.55  -3.72   1.78  20.62 

Coefficients:
                      Estimate Std. Error t value Pr(>|t|)    
(Intercept)             -7.621      0.891   -8.55  9.2e-16 ***
pretty_conditionarrow    3.172      1.237    2.56    0.011 *  
pretty_conditioncross    1.344      1.290    1.04    0.299    
---
Signif. codes:  0 '***' 0.001 '**' 0.01 '*' 0.05 '.' 0.1 ' ' 1

Residual standard error: 8.5 on 269 degrees of freedom
Multiple R-squared:  0.0241,    Adjusted R-squared:  0.0168 
F-statistic: 3.32 on 2 and 269 DF,  p-value: 0.0376
\end{verbatim}

\begin{Shaded}
\begin{Highlighting}[]
\FunctionTok{anova}\NormalTok{(m1)}
\end{Highlighting}
\end{Shaded}

\begin{verbatim}
Analysis of Variance Table

Response: s_SCALED
                  Df Sum Sq Mean Sq F value Pr(>F)  
pretty_condition   2    480   240.0    3.32  0.038 *
Residuals        269  19434    72.2                 
---
Signif. codes:  0 '***' 0.001 '**' 0.01 '*' 0.05 '.' 0.1 ' ' 1
\end{verbatim}

\begin{Shaded}
\begin{Highlighting}[]
\FunctionTok{report}\NormalTok{(m1)}
\end{Highlighting}
\end{Shaded}

\begin{verbatim}
Warning: 'data_findcols()' is deprecated and will be removed in a future update.
  Its usage is discouraged. Please use 'data_find()' instead.

Warning: 'data_findcols()' is deprecated and will be removed in a future update.
  Its usage is discouraged. Please use 'data_find()' instead.

Warning: 'data_findcols()' is deprecated and will be removed in a future update.
  Its usage is discouraged. Please use 'data_find()' instead.
\end{verbatim}

\begin{verbatim}
We fitted a linear model (estimated using OLS) to predict s_SCALED with pretty_condition (formula: s_SCALED ~ pretty_condition). The model explains a statistically significant and weak proportion of variance (R2 = 0.02, F(2, 269) = 3.32, p = 0.038, adj. R2 = 0.02). The model's intercept, corresponding to pretty_condition = point, is at -7.62 (95% CI [-9.38, -5.87], t(269) = -8.55, p < .001). Within this model:

  - The effect of pretty condition [arrow] is statistically significant and positive (beta = 3.17, 95% CI [0.74, 5.61], t(269) = 2.56, p = 0.011; Std. beta = 0.37, 95% CI [0.09, 0.65])
  - The effect of pretty condition [cross] is statistically non-significant and positive (beta = 1.34, 95% CI [-1.20, 3.88], t(269) = 1.04, p = 0.299; Std. beta = 0.16, 95% CI [-0.14, 0.45])

Standardized parameters were obtained by fitting the model on a standardized version of the dataset. 95% Confidence Intervals (CIs) and p-values were computed using the Wald approximation.
\end{verbatim}

\hypertarget{export-7}{%
\section{EXPORT}\label{export-7}}

Finally, we export the scores for each item (\texttt{df\_items}) and
summarized over subjects (\texttt{df\_subjects}), as well as cumulative
progress dataframes (\texttt{df\_absolute\_progress},
\texttt{df\_scaled\_progress})

\begin{Shaded}
\begin{Highlighting}[]
\CommentTok{\# \#HACK WD FOR LOCAL RUNNING?}
\CommentTok{\# imac = "/Users/amyraefox/Code/SGC{-}Scaffolding\_Graph\_Comprehension/SGC{-}X/ANALYSIS/MAIN"}
\CommentTok{\# \# mbp = "/Users/amyfox/Sites/RESEARCH/SGC—Scaffolding Graph Comprehension/SGC{-}X/ANALYSIS/MAIN"}
\CommentTok{\# setwd(imac)}

\CommentTok{\#SAVE FILES}
\FunctionTok{write.csv}\NormalTok{(df\_subjects,}\StringTok{"analysis/SGC4B/data/2{-}scored{-}data/sgc4b\_scored\_participants.csv"}\NormalTok{, }\AttributeTok{row.names =} \ConstantTok{FALSE}\NormalTok{)}
\FunctionTok{write.csv}\NormalTok{(df\_items,}\StringTok{"analysis/SGC4B/data/2{-}scored{-}data/sgc4b\_scored\_items.csv"}\NormalTok{, }\AttributeTok{row.names =} \ConstantTok{FALSE}\NormalTok{)}
\FunctionTok{write.csv}\NormalTok{(df\_absolute\_progress,}\StringTok{"analysis/SGC4B/data/2{-}scored{-}data/sgc4b\_absolute\_progress.csv"}\NormalTok{, }\AttributeTok{row.names =} \ConstantTok{FALSE}\NormalTok{)}
\FunctionTok{write.csv}\NormalTok{(df\_scaled\_progress,}\StringTok{"analysis/SGC4B/data/2{-}scored{-}data/sgc4b\_scaled\_progress.csv"}\NormalTok{, }\AttributeTok{row.names =} \ConstantTok{FALSE}\NormalTok{)}

\CommentTok{\#SAVE R Data Structures }
\CommentTok{\#export R DATA STRUCTURES (include codebook metadata)}
\NormalTok{rio}\SpecialCharTok{::}\FunctionTok{export}\NormalTok{(df\_subjects, }\StringTok{"analysis/SGC4B/data/2{-}scored{-}data/sgc4b\_scored\_participants.rds"}\NormalTok{) }\CommentTok{\# to R data structure file}
\NormalTok{rio}\SpecialCharTok{::}\FunctionTok{export}\NormalTok{(df\_items, }\StringTok{"analysis/SGC4B/data/2{-}scored{-}data/sgc4b\_scored\_items.rds"}\NormalTok{) }\CommentTok{\# to R data structure file}
\end{Highlighting}
\end{Shaded}

\hypertarget{resources-10}{%
\section{RESOURCES}\label{resources-10}}

\begin{Shaded}
\begin{Highlighting}[]
\FunctionTok{sessionInfo}\NormalTok{()}
\end{Highlighting}
\end{Shaded}

\begin{verbatim}
R version 4.2.1 (2022-06-23)
Platform: x86_64-apple-darwin17.0 (64-bit)
Running under: macOS Big Sur ... 10.16

Matrix products: default
BLAS:   /Library/Frameworks/R.framework/Versions/4.2/Resources/lib/libRblas.0.dylib
LAPACK: /Library/Frameworks/R.framework/Versions/4.2/Resources/lib/libRlapack.dylib

locale:
[1] en_US.UTF-8/en_US.UTF-8/en_US.UTF-8/C/en_US.UTF-8/en_US.UTF-8

attached base packages:
[1] stats     graphics  grDevices utils     datasets  methods   base     

other attached packages:
 [1] report_0.5.1      performance_0.9.1 forcats_0.5.1     stringr_1.4.0    
 [5] dplyr_1.0.9       purrr_0.3.4       readr_2.1.2       tidyr_1.2.0      
 [9] tibble_3.1.7      tidyverse_1.3.1   Hmisc_4.7-0       Formula_1.2-4    
[13] survival_3.3-1    lattice_0.20-45   pbapply_1.5-0     ggformula_0.10.1 
[17] ggridges_0.5.3    scales_1.2.0      ggstance_0.3.5    ggplot2_3.3.6    
[21] kableExtra_1.3.4 

loaded via a namespace (and not attached):
 [1] colorspace_2.0-3    ellipsis_0.3.2      rio_0.5.29         
 [4] htmlTable_2.4.0     parameters_0.18.1   base64enc_0.1-3    
 [7] fs_1.5.2            rstudioapi_0.13     farver_2.1.0       
[10] bit64_4.0.5         fansi_1.0.3         lubridate_1.8.0    
[13] xml2_1.3.3          codetools_0.2-18    splines_4.2.1      
[16] knitr_1.39          polyclip_1.10-0     jsonlite_1.8.0     
[19] broom_0.8.0         cluster_2.1.3       dbplyr_2.2.1       
[22] png_0.1-7           ggforce_0.3.3       effectsize_0.7.0   
[25] compiler_4.2.1      httr_1.4.3          backports_1.4.1    
[28] assertthat_0.2.1    Matrix_1.4-1        fastmap_1.1.0      
[31] cli_3.3.0           tweenr_1.0.2        htmltools_0.5.2    
[34] tools_4.2.1         gtable_0.3.0        glue_1.6.2         
[37] Rcpp_1.0.8.3        cellranger_1.1.0    vctrs_0.4.1        
[40] svglite_2.1.0       insight_0.17.1      xfun_0.31          
[43] openxlsx_4.2.5      rvest_1.0.2         lifecycle_1.0.1    
[46] mosaicCore_0.9.0    MASS_7.3-57         vroom_1.5.7        
[49] hms_1.1.1           parallel_4.2.1      RColorBrewer_1.1-3 
[52] yaml_2.3.5          curl_4.3.2          gridExtra_2.3      
[55] labelled_2.9.1      rpart_4.1.16        latticeExtra_0.6-29
[58] stringi_1.7.6       bayestestR_0.12.1   checkmate_2.1.0    
[61] zip_2.2.0           rlang_1.0.3         pkgconfig_2.0.3    
[64] systemfonts_1.0.4   evaluate_0.15       htmlwidgets_1.5.4  
[67] labeling_0.4.2      bit_4.0.4           tidyselect_1.1.2   
[70] plyr_1.8.7          magrittr_2.0.3      R6_2.5.1           
[73] generics_0.1.2      DBI_1.1.3           pillar_1.7.0       
[76] haven_2.5.0         foreign_0.8-82      withr_2.5.0        
[79] datawizard_0.4.1    nnet_7.3-17         modelr_0.1.8       
[82] crayon_1.5.1        utf8_1.2.2          tzdb_0.3.0         
[85] rmarkdown_2.14      jpeg_0.1-9          grid_4.2.1         
[88] readxl_1.4.0        data.table_1.14.2   reprex_2.0.1       
[91] digest_0.6.29       webshot_0.5.3       munsell_0.5.0      
[94] viridisLite_0.4.0  
\end{verbatim}

\part{SGC5A}

\newpage

\hypertarget{sec-SGC5A-introduction}{%
\chapter{Introduction}\label{sec-SGC5A-introduction}}

\textbf{In Study 5A we explore the extent to which requiring
mouse-cursor interaction with the graph improves interpretation of the
underlying coordinate system. }

\begin{longtable}[]{@{}
  >{\raggedright\arraybackslash}p{(\columnwidth - 2\tabcolsep) * \real{0.2552}}
  >{\raggedright\arraybackslash}p{(\columnwidth - 2\tabcolsep) * \real{0.7448}}@{}}
\caption{SGC5A Study Conditions}\tabularnewline
\toprule()
\endhead
\includegraphics{analysis/utils/img/11111.png} &
\begin{minipage}[t]{\linewidth}\raggedright
\textbf{CONTROL (Check-box)}\\
Demo:
\href{https://limitless-plains-85018.herokuapp.com/?study=SGC5A\&condition=11111\&session=WEB-DEMO}{111}\strut
\end{minipage} \\
\includegraphics{analysis/utils/img/11115.png} &
\begin{minipage}[t]{\linewidth}\raggedright
\textbf{INTERACTION (Point-Click)}\\
Demo:
\href{https://limitless-plains-85018.herokuapp.com/?study=SGC5A\&condition=11115\&session=WEB-DEMO}{11115}\strut
\end{minipage} \\
\bottomrule()
\end{longtable}

\begin{Shaded}
\begin{Highlighting}[]
\FunctionTok{library}\NormalTok{(codebook) }\CommentTok{\#data dictionary}
\FunctionTok{library}\NormalTok{(tidyverse) }\CommentTok{\#ALL THE THINGS}
\FunctionTok{library}\NormalTok{(kableExtra) }\CommentTok{\#tables}

\CommentTok{\#set some output options}
\FunctionTok{library}\NormalTok{(dplyr, }\AttributeTok{warn.conflicts =} \ConstantTok{FALSE}\NormalTok{)}
\FunctionTok{options}\NormalTok{(}\AttributeTok{dplyr.summarise.inform =} \ConstantTok{FALSE}\NormalTok{)}
\FunctionTok{options}\NormalTok{(}\AttributeTok{scipen=}\DecValTok{1}\NormalTok{, }\AttributeTok{digits=}\DecValTok{3}\NormalTok{)}
\end{Highlighting}
\end{Shaded}

\begin{Shaded}
\begin{Highlighting}[]
\CommentTok{\# }\AlertTok{HACK}\CommentTok{ WD FOR LOCAL RUNNING?}
\CommentTok{\# imac = "/Users/amyraefox/Code/SGC{-}Scaffolding\_Graph\_Comprehension/SGC{-}X/ANALYSIS/MAIN"}
\CommentTok{\# mbp = "/Users/amyfox/Sites/RESEARCH/SGC—Scaffolding Graph Comprehension/SGC{-}X/ANALYSIS/MAIN"}
\CommentTok{\# setwd(imac)}

\CommentTok{\#IMPORT DATA }
\NormalTok{df\_subjects }\OtherTok{\textless{}{-}} \FunctionTok{read\_rds}\NormalTok{(}\StringTok{\textquotesingle{}analysis/SGC5A/data/0{-}study{-}level/sgc5\_participants.rds\textquotesingle{}}\NormalTok{) }
\end{Highlighting}
\end{Shaded}

\begin{Shaded}
\begin{Highlighting}[]
\NormalTok{title }\OtherTok{=} \StringTok{"Participants by Condition and Data Collection Period"}
\CommentTok{\# cols = c("Control Condition","Impasse Condition","Total for Period")}
\NormalTok{cont }\OtherTok{\textless{}{-}} \FunctionTok{table}\NormalTok{(df\_subjects}\SpecialCharTok{$}\NormalTok{term, df\_subjects}\SpecialCharTok{$}\NormalTok{condition)}
\NormalTok{cont }\SpecialCharTok{\%\textgreater{}\%} \FunctionTok{addmargins}\NormalTok{() }\SpecialCharTok{\%\textgreater{}\%} \FunctionTok{kbl}\NormalTok{(}\AttributeTok{caption =}\NormalTok{ title) }\SpecialCharTok{\%\textgreater{}\%}  \FunctionTok{kable\_classic}\NormalTok{()}
\end{Highlighting}
\end{Shaded}

\begin{table}

\caption{Participants by Condition and Data Collection Period}
\centering
\begin{tabular}[t]{l|r|r}
\hline
  & 11115 & Sum\\
\hline
winter22 & 115 & 115\\
\hline
Sum & 115 & 115\\
\hline
\end{tabular}
\end{table}

\hypertarget{hypotheses-3}{%
\subsection{Hypotheses}\label{hypotheses-3}}

\textbf{Experimental Hypothesis:} \emph{We hypothesize that requiring
interaction with the graph (by clicking on data points rather than
checkboxes to answer the questions) will improve interpretation of the
coordinate system.}

\begin{itemize}
\tightlist
\item
  H1 The INTERACTION condition should yield significantly better
  performance

  \begin{itemize}
  \tightlist
  \item
    H1A \textbar{} INTERACTION condition should yield higher total
    absolute score
  \item
    H1B \textbar{} INTERACTION condition should yield higher first
    question score
  \item
    H1C \textbar{} INTERACTION condition should yield lower response
    times on the first question (both overall, and specifically among
    participants who answer correctly)
  \end{itemize}
\end{itemize}

\textbf{Null Hypothesis:}\\
\emph{A null effect will be characterized by no significant differences
in performance between CONTROL and INTERACTION conditions.}

\textbf{Exploratory Questions}

\hypertarget{methods-4}{%
\section{METHODS}\label{methods-4}}

\hypertarget{design-4}{%
\subsection{Design}\label{design-4}}

We employed a mixed design with 1 between-subjects factor with 2 levels
(Scaffold: control, impasse) and 15 items (within-subjects factor).

Independent Variables:

\begin{itemize}
\tightlist
\item
  B-S
\item
  W-S (Item x 15)
\end{itemize}

Dependent Variables:

\begin{itemize}
\tightlist
\item
  Response Accuracy : Is the response triangular-correct?
\item
  Response Interpretation : (derived) With which interpretation of the
  graph is the subject's response on an individual question consistent?
\item
  Response Latency : Time from stimulus onset to clicking `Submit'
  button: time in (s)
\end{itemize}

\hypertarget{materials-4}{%
\subsection{Materials}\label{materials-4}}

Stimuli consisted of a series of 15 graph comprehension questions, each
testing a different combination of time interval relations, to be read
from a Triangular-Model graph. Figure~\ref{fig-sample}. The list of
questions can be found \href{static/stimuli/sgcx_questions.csv}{here}.

\begin{figure}

{\centering \includegraphics{analysis/SGC5A/static/stimuli/sample_task.png}

}

\caption{\label{fig-sample}Sample Question (Q=1) for Graph Comprehension
Task}

\end{figure}

Note that across both control and impasse conditions, both the question,
response options and graph structure were identical. The experimental
manipulation (posing a mental impasse) was accomplished by changing the
position of datapoints in the impasse-condition graph, such that for any
given question, there was no available response option if the reader
were to interpret the graph as cartesian (making an orthogonal rather
than diagonal projection from the x-axis.)

\emph{The green line indicates the ideal-scanpath to the correct
(triangular) answer to the first question, and the red line indicates
the (incorrect) orthogonal interpretation. In the IMPASSE figure (at
right), there are no data points that intersect the red line. We
hypothesize that this presents the reader with an obstacle, at which
point they are forced to confront their interpretation of the coordinate
system and (ideally) develop a new strategy.}

\begin{figure}

{\centering \includegraphics{analysis/SGC5A/static/stimuli/3A_conditions.png}

}

\caption{\label{fig-conditions}Sample Question (Q=1) graphs for each
condition}

\end{figure}

\hypertarget{procedure-4}{%
\subsection{Procedure}\label{procedure-4}}

Participants completed the study via a web-browser.

(1) Upon starting, they submitted informed consent, before reading task
instructions.

(2) Participants were introduced to a scenario in which they were to
play the role of a project manager, scheduling shifts for a group of
employees. The schedule of the employees was presented in a
TriangularModel (TM) graph, and they would be answering question about
the schedule.

(3) Then participants completed an experimental block of 15 items : the
Graph Comprehension Task

(4) Following the experimental block, participants answered a
free-response question about their strategy for reading the graph,
followed by a demographic questionnaire and debrief.

\hypertarget{sample-5}{%
\subsection{Sample}\label{sample-5}}

Data was collected by convenience sample \ldots{}

\hypertarget{analysis-4}{%
\section{ANALYSIS}\label{analysis-4}}

\hypertarget{sec-SGC5-harmonize}{%
\subsection{Data Preparation}\label{sec-SGC5-harmonize}}

Data were collected via a custom web application and stored in a NoSQL
database. The following exclusion criteria were applied during data
cleaning:

\begin{itemize}
\tightlist
\item
  completion status : ``success'' ; subject must have finished all parts
  of the study, including demographic questionnaire
\item
  session ID: {[}in list{]} ; subject must have been assigned to valid
  data collection session (discard testing and piloting data)
\item
  browser interaction violations \textless{} 3; subject must have fewer
  than 3 violations of non-allowed browser interactions (i.e.~resizing
  window, leaving browser tab or leaving fullscreen mode)
\item
  self-rated effort \textgreater{} 2; subjects who reported, ``not
  trying hard/rushing through questions'' or ``started out trying hard
  but giving up at some point'' were excluded from analysis.
\item
  attention check ==TRUE ; subjects who failed to answer a mid-study
  attention check question (Graph Comprehension Task Question \#6) are
  excluded
\end{itemize}

\begin{longtable}[]{@{}ll@{}}
\toprule()
Pre-Requisite & Followed By \\
\midrule()
\endhead
winter2022\_clean\_sgc5a.Rmd & 2\_sgc5\_scoring.qmd \\
\bottomrule()
\end{longtable}

The underlying data structure of the stimulus web application changed
across the data collection period, resulting in slightly different data
files (i.e.~columns are not named consistently). In this section, we
combine the files from each data collection period into a single
\emph{harmonized} data file for analysis (one for participants, one for
items).

\hypertarget{participants-5}{%
\subsubsection{Participants}\label{participants-5}}

First we import participant-level data, selecting only the columns
relevant for analysis. The result is a single data frame
\texttt{df\_subjects} containing one row for each subject (across all
periods). Note that we \emph{are not} discarding any \emph{response}
data. Rather, we discard columns that are automatically recorded by the
stimulus web application and help the application run.

\emph{Note that we discard some columns representing scores calculated
in the stimulus engine. These scores were calculated differently across
collection periods, and so we discard them and recalculate scores in the
next analysis notebook. No raw data (responses and response times) are
discarded, only algorithmically-derived scores for the responses.}

\begin{Shaded}
\begin{Highlighting}[]
\CommentTok{\#IMPORT PARTICIPANT DATA}

\CommentTok{\# }\AlertTok{HACK}\CommentTok{ WD FOR LOCAL RUNNING?}
\CommentTok{\# imac = "/Users/amyraefox/Code/SGC{-}Scaffolding\_Graph\_Comprehension/SGC{-}X/ANALYSIS/MAIN"}
\CommentTok{\# mbp = "/Users/amyfox/Sites/RESEARCH/SGC—Scaffolding Graph Comprehension/SGC{-}X/ANALYSIS/MAIN"}
\CommentTok{\# setwd(imac)}

\CommentTok{\#import file}
\NormalTok{df\_subjects }\OtherTok{\textless{}{-}} \FunctionTok{read\_rds}\NormalTok{(}\StringTok{"analysis/SGC5A/data/0{-}study{-}level/sgc5\_participants.rds"}\NormalTok{) }\CommentTok{\#use RDS file as it contains metadata}

\CommentTok{\#SAVE METADATA FROM WINTER, but no rows }
\CommentTok{\# df\_subjects \textless{}{-} df\_subjects\_winter22 \%\textgreater{}\% filter(condition==\textquotesingle{}X\textquotesingle{}) \%\textgreater{}\% select(}
\CommentTok{\#   subject,condition,term,mode,}
\CommentTok{\#   gender,age,language, schoolyear, country,}
\CommentTok{\#   effort,difficulty,confidence,enjoyment,other,}
\CommentTok{\#   totaltime\_m,absolute\_score}
\CommentTok{\# )}

\CommentTok{\#save \textquotesingle{}explanation\textquotesingle{} columns from winter22, which is actually a response to a free response item (Q16); was recorded with item\_level data in old webapp}
\NormalTok{df\_q16 }\OtherTok{\textless{}{-}}\NormalTok{ df\_subjects }\SpecialCharTok{\%\textgreater{}\%} 
\NormalTok{  dplyr}\SpecialCharTok{::}\FunctionTok{select}\NormalTok{(subject, condition, term , mode, explanation) }\SpecialCharTok{\%\textgreater{}\%} 
  \FunctionTok{mutate}\NormalTok{(}
    \AttributeTok{q =} \DecValTok{16}\NormalTok{,}
    \AttributeTok{response =}\NormalTok{ explanation}
\NormalTok{  ) }\SpecialCharTok{\%\textgreater{}\%}\NormalTok{ dplyr}\SpecialCharTok{::}\FunctionTok{select}\NormalTok{(}\SpecialCharTok{{-}}\NormalTok{explanation)}

\CommentTok{\#reduce data collected using NEW webapp to useful columns}
\NormalTok{df\_subjects }\OtherTok{\textless{}{-}}\NormalTok{ df\_subjects }\SpecialCharTok{\%\textgreater{}\%} 
  \CommentTok{\# mutate(score = absolute\_score) \%\textgreater{}\% }
  \CommentTok{\#select only columns we\textquotesingle{}ll be analyzing, discard others}
\NormalTok{  dplyr}\SpecialCharTok{::}\FunctionTok{select}\NormalTok{( subject, condition, pretty\_condition, term, mode, }
                 \CommentTok{\#demographics}
\NormalTok{                 gender, age, language, schoolyear, country,}
                 \CommentTok{\#effort survey}
\NormalTok{                 effort, difficulty, confidence, enjoyment, }
                 \CommentTok{\#explanations}
\NormalTok{                 other,disability,}
                 \CommentTok{\#response characteristics}
\NormalTok{                 totaltime\_m, }
                 \CommentTok{\#absolute\_score,\#drop absolute score as this is re{-}scored [though should be the same]}
                 \CommentTok{\#exploratory factors}
\NormalTok{                 violations, browser, width, height}
\NormalTok{                 )}


\NormalTok{effort\_labels }\OtherTok{\textless{}{-}} \FunctionTok{c}\NormalTok{(}\StringTok{"I tried my best on each question"}\NormalTok{, }\StringTok{"I tried my best on most questions"}\NormalTok{)}

\CommentTok{\#set factors}
\NormalTok{df\_subjects }\OtherTok{\textless{}{-}}\NormalTok{ df\_subjects }\SpecialCharTok{\%\textgreater{}\%} 
  \CommentTok{\#refactor factors}
  \FunctionTok{mutate}\NormalTok{ (}
    \AttributeTok{subject =} \FunctionTok{factor}\NormalTok{(subject),}
    \AttributeTok{condition =} \FunctionTok{factor}\NormalTok{(condition),}
    \AttributeTok{term =} \FunctionTok{factor}\NormalTok{(term),}
    \AttributeTok{mode =} \FunctionTok{factor}\NormalTok{(mode),}
    \AttributeTok{gender =} \FunctionTok{factor}\NormalTok{(gender),}
    \AttributeTok{schoolyear =} \FunctionTok{factor}\NormalTok{(schoolyear, }\AttributeTok{levels=}\FunctionTok{c}\NormalTok{(}\StringTok{"First"}\NormalTok{,}\StringTok{"Second"}\NormalTok{,}\StringTok{"Third"}\NormalTok{,}\StringTok{"Fourth"}\NormalTok{,}\StringTok{"Fifth"}\NormalTok{,}\StringTok{"Other"}\NormalTok{))}
\NormalTok{  )}
\end{Highlighting}
\end{Shaded}

\hypertarget{items-4}{%
\subsubsection{Items}\label{items-4}}

Next we import item-level data from each data collection period,
selecting only the columns relevant for analysis. The result is a single
data frame \texttt{df\_items} containing one row for each \emph{graph
comprehension task question} (qs=15) (across all periods). A second data
frame \texttt{df\_freeresponse} contains one row for each free response
strategy question (last question posed to participants in Winter2022)
Note that we \emph{do not} discard any \emph{response} data. Rather, we
\emph{do} discard several columns representing accuracy scores for
responses that were calculated in the stimulus engine. These scores were
calculated differently across collection periods, and so we discard them
and recalculate scores in the next analysis notebook. Original response
data are always preserved.

\begin{Shaded}
\begin{Highlighting}[]
\CommentTok{\# \#HACK WD FOR LOCAL RUNNING?}
\CommentTok{\# imac = "/Users/amyraefox/Code/SGC{-}Scaffolding\_Graph\_Comprehension/SGC{-}X/ANALYSIS/MAIN"}
\CommentTok{\# mbp = "/Users/amyfox/Sites/RESEARCH/SGC—Scaffolding Graph Comprehension/SGC{-}X/ANALYSIS/MAIN"}
\CommentTok{\# setwd(imac)}

\CommentTok{\#read datafiles}
\NormalTok{df\_items }\OtherTok{\textless{}{-}} \FunctionTok{read\_rds}\NormalTok{(}\StringTok{"analysis/SGC5A/data/0{-}study{-}level/sgc5\_items.rds"}\NormalTok{) }\CommentTok{\#use RDS file as it contains metadata}

\CommentTok{\#reduce data collected using new webapp}
\NormalTok{df\_items }\OtherTok{\textless{}{-}}\NormalTok{ df\_items }\SpecialCharTok{\%\textgreater{}\%} 
\NormalTok{  dplyr}\SpecialCharTok{::}\FunctionTok{select}\NormalTok{(subject, condition, pretty\_condition, term, mode, question, q, answer, correct, rt\_s) }\SpecialCharTok{\%\textgreater{}\%} \CommentTok{\#unfactor before combine}
  \FunctionTok{mutate}\NormalTok{(}
    \AttributeTok{subject =} \FunctionTok{as.character}\NormalTok{(subject),}
    \AttributeTok{condition =} \FunctionTok{as.character}\NormalTok{(condition),}
    \AttributeTok{term =} \FunctionTok{as.character}\NormalTok{(term),}
    \AttributeTok{mode =} \FunctionTok{as.character}\NormalTok{(mode),}
    \AttributeTok{q =} \FunctionTok{as.integer}\NormalTok{(q),}
    \AttributeTok{correct =} \FunctionTok{as.logical}\NormalTok{(correct)}
\NormalTok{  ) }\SpecialCharTok{\%\textgreater{}\%} 
  \FunctionTok{mutate}\NormalTok{(}
    \AttributeTok{response =} \FunctionTok{str\_remove\_all}\NormalTok{(}\FunctionTok{as.character}\NormalTok{(answer), }\StringTok{","}\NormalTok{),}
    \AttributeTok{num\_o =} \FunctionTok{str\_length}\NormalTok{(response)}
\NormalTok{  ) }\SpecialCharTok{\%\textgreater{}\%} 
  \CommentTok{\# handle NA values (why are some empty responses blank and others NA?) }
  \FunctionTok{mutate}\NormalTok{(}
    \AttributeTok{response =} \FunctionTok{replace\_na}\NormalTok{(response, }\StringTok{""}\NormalTok{),}
    \AttributeTok{num\_o =} \FunctionTok{replace\_na}\NormalTok{(num\_o, }\DecValTok{0}\NormalTok{)}
\NormalTok{  )}
\end{Highlighting}
\end{Shaded}

\hypertarget{validation-4}{%
\subsubsection{Validation}\label{validation-4}}

Next, we validate that we have the complete number of item-level records
based on the number of subject-level records

\begin{Shaded}
\begin{Highlighting}[]
\CommentTok{\#the number of items should be equal to 15 x the number of subjects}
\FunctionTok{nrow}\NormalTok{(df\_items) }\SpecialCharTok{==} \DecValTok{15}\SpecialCharTok{*} \FunctionTok{nrow}\NormalTok{(df\_subjects) }\CommentTok{\#TRUE}
\end{Highlighting}
\end{Shaded}

\begin{verbatim}
[1] TRUE
\end{verbatim}

\begin{Shaded}
\begin{Highlighting}[]
\CommentTok{\#each subject should have 15 items}
\NormalTok{df\_items }\SpecialCharTok{\%\textgreater{}\%} \FunctionTok{group\_by}\NormalTok{(subject) }\SpecialCharTok{\%\textgreater{}\%} \FunctionTok{summarise}\NormalTok{(}\AttributeTok{n =} \FunctionTok{n}\NormalTok{()) }\SpecialCharTok{\%\textgreater{}\%} \FunctionTok{filter}\NormalTok{(n }\SpecialCharTok{!=} \DecValTok{15}\NormalTok{) }\SpecialCharTok{\%\textgreater{}\%} \FunctionTok{nrow}\NormalTok{() }\SpecialCharTok{==} \DecValTok{0}
\end{Highlighting}
\end{Shaded}

\begin{verbatim}
[1] TRUE
\end{verbatim}

\hypertarget{export-8}{%
\subsubsection{Export}\label{export-8}}

Finally, we export the (session-harmonized) data for analysis, as CSVs,
and .RDS (includes metadata)

\begin{Shaded}
\begin{Highlighting}[]
\CommentTok{\# \#HACK WD FOR LOCAL RUNNING?}
\CommentTok{\# imac = "/Users/amyraefox/Code/SGC{-}Scaffolding\_Graph\_Comprehension/SGC{-}X/ANALYSIS/MAIN"}
\CommentTok{\# mbp = "/Users/amyfox/Sites/RESEARCH/SGC—Scaffolding Graph Comprehension/SGC{-}X/ANALYSIS/MAIN"}
\CommentTok{\# setwd(imac)}

\CommentTok{\#SAVE FILES}
\FunctionTok{write.csv}\NormalTok{(df\_subjects,}\StringTok{"analysis/SGC5A/data/1{-}study{-}level/sgc5\_participants.csv"}\NormalTok{, }\AttributeTok{row.names =} \ConstantTok{FALSE}\NormalTok{)}
\FunctionTok{write.csv}\NormalTok{(df\_items,}\StringTok{"analysis/SGC5A/data/1{-}study{-}level/sgc5\_items.csv"}\NormalTok{, }\AttributeTok{row.names =} \ConstantTok{FALSE}\NormalTok{)}
\FunctionTok{write.csv}\NormalTok{(df\_q16,}\StringTok{"analysis/SGC5A/data/1{-}study{-}level/sgc5\_freeresponse.csv"}\NormalTok{, }\AttributeTok{row.names =} \ConstantTok{FALSE}\NormalTok{)}

\CommentTok{\#SAVE R Data Structures }
\CommentTok{\#export R DATA STRUCTURES (include codebook metadata)}
\NormalTok{rio}\SpecialCharTok{::}\FunctionTok{export}\NormalTok{(df\_subjects, }\StringTok{"analysis/SGC5A/data/1{-}study{-}level/sgc5\_participants.rds"}\NormalTok{) }\CommentTok{\# to R data structure file}
\NormalTok{rio}\SpecialCharTok{::}\FunctionTok{export}\NormalTok{(df\_items, }\StringTok{"analysis/SGC5A/data/1{-}study{-}level/sgc5\_items.rds"}\NormalTok{) }\CommentTok{\# to R data structure file}
\end{Highlighting}
\end{Shaded}

\hypertarget{resources-11}{%
\section{RESOURCES}\label{resources-11}}

\begin{Shaded}
\begin{Highlighting}[]
\FunctionTok{sessionInfo}\NormalTok{()}
\end{Highlighting}
\end{Shaded}

\begin{verbatim}
R version 4.2.1 (2022-06-23)
Platform: x86_64-apple-darwin17.0 (64-bit)
Running under: macOS Big Sur ... 10.16

Matrix products: default
BLAS:   /Library/Frameworks/R.framework/Versions/4.2/Resources/lib/libRblas.0.dylib
LAPACK: /Library/Frameworks/R.framework/Versions/4.2/Resources/lib/libRlapack.dylib

locale:
[1] en_US.UTF-8/en_US.UTF-8/en_US.UTF-8/C/en_US.UTF-8/en_US.UTF-8

attached base packages:
[1] stats     graphics  grDevices utils     datasets  methods   base     

other attached packages:
 [1] kableExtra_1.3.4 forcats_0.5.1    stringr_1.4.0    dplyr_1.0.9     
 [5] purrr_0.3.4      readr_2.1.2      tidyr_1.2.0      tibble_3.1.7    
 [9] ggplot2_3.3.6    tidyverse_1.3.1  codebook_0.9.2  

loaded via a namespace (and not attached):
 [1] Rcpp_1.0.8.3      svglite_2.1.0     lubridate_1.8.0   assertthat_0.2.1 
 [5] digest_0.6.29     utf8_1.2.2        R6_2.5.1          cellranger_1.1.0 
 [9] backports_1.4.1   reprex_2.0.1      labelled_2.9.1    evaluate_0.15    
[13] httr_1.4.3        pillar_1.7.0      rlang_1.0.3       curl_4.3.2       
[17] readxl_1.4.0      data.table_1.14.2 rstudioapi_0.13   rmarkdown_2.14   
[21] webshot_0.5.3     foreign_0.8-82    munsell_0.5.0     broom_0.8.0      
[25] compiler_4.2.1    modelr_0.1.8      xfun_0.31         pkgconfig_2.0.3  
[29] systemfonts_1.0.4 htmltools_0.5.2   tidyselect_1.1.2  rio_0.5.29       
[33] codetools_0.2-18  fansi_1.0.3       viridisLite_0.4.0 crayon_1.5.1     
[37] tzdb_0.3.0        dbplyr_2.2.1      withr_2.5.0       grid_4.2.1       
[41] jsonlite_1.8.0    gtable_0.3.0      lifecycle_1.0.1   DBI_1.1.3        
[45] magrittr_2.0.3    scales_1.2.0      zip_2.2.0         cli_3.3.0        
[49] stringi_1.7.6     fs_1.5.2          xml2_1.3.3        ellipsis_0.3.2   
[53] generics_0.1.2    vctrs_0.4.1       openxlsx_4.2.5    tools_4.2.1      
[57] glue_1.6.2        hms_1.1.1         fastmap_1.1.0     yaml_2.3.5       
[61] colorspace_2.0-3  rvest_1.0.2       knitr_1.39        haven_2.5.0      
\end{verbatim}

\newpage

\hypertarget{sec-SGC5A-scoring}{%
\chapter{Response Scoring}\label{sec-SGC5A-scoring}}

\emph{The purpose of this notebook is to score (assign a measure of
accuracy) to response data for the SGC5 study. This is required because
the question type on the graph comprehension task used a `Multiple
Response' (MR) question design. Here, we evaluate different approaches
to scoring multiple response questions, and transform raw item responses
(e.g.~boxes ABC are checked) to a measure of response accuracy.
(Warning: this notebook takes several minutes to execute.)} To review
the strategy behind Multiple Response scoring for the SGC project, refer
to section \textbf{?@sec-scoring}.

\begin{Shaded}
\begin{Highlighting}[]
\FunctionTok{options}\NormalTok{(}\AttributeTok{scipen=}\DecValTok{1}\NormalTok{, }\AttributeTok{digits=}\DecValTok{3}\NormalTok{)}

\FunctionTok{library}\NormalTok{(kableExtra) }\CommentTok{\#printing tables }
\FunctionTok{library}\NormalTok{(ggformula) }\CommentTok{\#quick graphs}
\FunctionTok{library}\NormalTok{(pbapply) }\CommentTok{\#progress bar and time estimate for *apply fns}
\FunctionTok{library}\NormalTok{(Hmisc) }\CommentTok{\# \%nin\% operator}
\FunctionTok{library}\NormalTok{(tidyverse) }\CommentTok{\#ALL THE THINGS}
\end{Highlighting}
\end{Shaded}

\hypertarget{score-sgc-data-4}{%
\section{SCORE SGC DATA}\label{score-sgc-data-4}}

To review the strategy behind Multiple Response scoring for the SGC
project, refer to section \textbf{?@sec-scoring}.

In SGC we are fundamentally interested in understanding how a
participant interprets the presented graph (stimulus). The \textbf{graph
comprehension task} asks them to select the data points in the graph
that meet the criteria posed in the question. To assess a participant's
performance, for each question (q=15) we will calculate the following
scores:

\emph{An overall, strict score:}\\
1. \textbf{Absolute Score} : using dichotomous scoring referencing true
(Triangular) answer. (see 1.2)

\emph{Sub-scores, for each alternative graph interpretation}\\
2. \textbf{Triangular Score} : using partial scoring {[}-1/q, +1/p{]}
referencing true (Triangular) answer key.

3. \textbf{Orthogonal Score} : using partial scoring {[}-1/q, +1/p{]}
referencing (incorrect Orthogonal) answer key.

Based on prior observational studies where we observed emergence of
other alternative interpretations (i.e.~transitional interpretations) we
also calculate subscores for these alternatives.

4. \textbf{Tversky Score} : using partial scoring {[}-1/q, +1/p{]}
referencing (incorrect connecting-lines strategy) answer key. 5.
\textbf{Satisficing Score} : using partial scoring {[}-1/q, +1/p{]}
referencing (incorrect satisficing strategy) answer key.

\hypertarget{sec-SGC5A-keys}{%
\subsection{Prepare Answer Keys}\label{sec-SGC5A-keys}}

We start by importing three answer keys: (1) Q1 - Q5 {[}control
condition{]}, (2) Q1-Q5 {[}impasse condition{]}, (3) Q6-15. Separate
answer keys by condition are required for Q1-Q5 because the stimuli for
each condition visualize a different underlying dataset (i.e.~the graphs
show datapoints in different positions). Q6-Q15 are identical across
conditions. Each answer key includes a row for each question, and a
column defining the subset of response options that correspond to
different graph interpretations.

\begin{Shaded}
\begin{Highlighting}[]
\CommentTok{\# \#HACK WD FOR LOCAL RUNNING?}
\NormalTok{imac }\OtherTok{=} \StringTok{"/Users/amyraefox/Code/SGC{-}Scaffolding\_Graph\_Comprehension/SGC{-}X/ANALYSIS/MAIN"}
\FunctionTok{setwd}\NormalTok{(imac)}

\CommentTok{\#SAVE KEYS FOR FUTURE USE}
\NormalTok{keys\_raw }\OtherTok{\textless{}{-}}  \FunctionTok{read\_csv}\NormalTok{(}\StringTok{"analysis/utils/keys/parsed\_keys/keys\_raw"}\NormalTok{)}
\NormalTok{keys\_orth }\OtherTok{\textless{}{-}}  \FunctionTok{read\_csv}\NormalTok{(}\StringTok{"analysis/utils/keys/parsed\_keys/keys\_orth"}\NormalTok{)}
\NormalTok{keys\_tri }\OtherTok{\textless{}{-}}  \FunctionTok{read\_csv}\NormalTok{(}\StringTok{"analysis/utils/keys/parsed\_keys/keys\_tri"}\NormalTok{)}
\NormalTok{keys\_satisfice\_left }\OtherTok{\textless{}{-}}  \FunctionTok{read\_csv}\NormalTok{(}\StringTok{"analysis/utils/keys/parsed\_keys/keys\_satisfice\_left"}\NormalTok{)}
\NormalTok{keys\_satisfice\_right }\OtherTok{\textless{}{-}}  \FunctionTok{read\_csv}\NormalTok{(}\StringTok{"analysis/utils/keys/parsed\_keys/keys\_satisfice\_right"}\NormalTok{)}
\NormalTok{keys\_tversky\_duration }\OtherTok{\textless{}{-}}  \FunctionTok{read\_csv}\NormalTok{(}\StringTok{"analysis/utils/keys/parsed\_keys/keys\_tversky\_duration"}\NormalTok{)}
\NormalTok{keys\_tversky\_end }\OtherTok{\textless{}{-}}  \FunctionTok{read\_csv}\NormalTok{(}\StringTok{"analysis/utils/keys/parsed\_keys/keys\_tversky\_end"}\NormalTok{)}
\NormalTok{keys\_tversky\_max }\OtherTok{\textless{}{-}}  \FunctionTok{read\_csv}\NormalTok{(}\StringTok{"analysis/utils/keys/parsed\_keys/keys\_tversky\_max"}\NormalTok{)}
\NormalTok{keys\_tversky\_start }\OtherTok{\textless{}{-}}  \FunctionTok{read\_csv}\NormalTok{(}\StringTok{"analysis/utils/keys/parsed\_keys/keys\_tversky\_start"}\NormalTok{)}
\end{Highlighting}
\end{Shaded}

\hypertarget{sec-SGC5A-subscores}{%
\subsection{Calculate Subscores}\label{sec-SGC5A-subscores}}

Next, we import the item-level response data. For each row in the item
level dataset (indicating the response to a single question-item for a
single subject), we compare the raw response
\texttt{df\_items\$response} with the answer keys in each interpretation
(e.g.~\texttt{keys\_orth}, \texttt{keys\_tri}, etc\ldots), then using
those sets, determine the number of correctly selected items(p) and
incorrectly selected items (q), which in turn are used to calculate
partial{[}-1/q, +1/p{]} scores for each interpretation. The resulting
scores are then stored on each item in \texttt{df\_items}, and can be
used to determine which graph interpretation the subject held.

Specifically, the following scores are calculated for each item:

\textbf{Interpretation Subscores}

\begin{itemize}
\tightlist
\item
  \texttt{score\_TRI} How consistent is the response with the
  \textbf{triangular}interpretation?
\item
  \texttt{score\_ORTH} How consistent is the response with the
  \textbf{orthogonal}interpretation?
\item
  \texttt{score\_SATISFICE} is calculated by taking the maximum value of
  :

  \begin{itemize}
  \tightlist
  \item
    \texttt{score\_SAT\_left} How consistent is the response with the
    \textbf{(left side) Satisficing} interpretation?
  \item
    \texttt{score\_SAT\_right} How consistent is the response with the
    \textbf{(right side) Satisficing} interpretation
  \end{itemize}
\item
  \texttt{score\_TVERSKY} is calculated by taking the maximum value of:

  \begin{itemize}
  \tightlist
  \item
    \texttt{score\_TV\_max} How consistent is the response with the
    \textbf{(maximal) Tversky} interpretation?
  \item
    \texttt{score\_TV\_start} How consistent is the response with the
    \textbf{(start-time) Tversky} interpretation?
  \item
    \texttt{score\_TV\_end} How consistent is the response with the
    \textbf{(end-time) Tversky} interpretation?
  \item
    \texttt{score\_TV\_duration} How consistent is the response with the
    \textbf{(duration) Tversky} interpretation?
  \end{itemize}
\item
  \texttt{score\_REF} Did the response select only the \textbf{reference
  point}?
\item
  \texttt{score\_BOTH} How consistent is the response with \textbf{both}
  the orthogonal and triangular interpretations?
\end{itemize}

\textbf{Absolute Scores}

\begin{itemize}
\tightlist
\item
  \texttt{score\_ABS} Is the response strictly correct? (triangular
  interpretation)
\item
  \texttt{score\_niceABS} Is the response strictly correct? (triangular
  interpretation, not penalizing ref points). This is a more generous
  version of the Absolute score that does not penalize the participant
  if in addition to the correct answer \emph{in addition to} they also
  select the data point referenced in the question.
\end{itemize}

\begin{Shaded}
\begin{Highlighting}[]
\CommentTok{\#HACK WD FOR LOCAL RUNNING?}
\NormalTok{imac }\OtherTok{=} \StringTok{"/Users/amyraefox/Code/SGC{-}Scaffolding\_Graph\_Comprehension/SGC{-}X/ANALYSIS/MAIN"}
\FunctionTok{setwd}\NormalTok{(imac)}

\CommentTok{\#backup \textless{}{-} read\_rds(\textquotesingle{}analysis/SGC5A/data/1{-}study{-}level/sgc5\_items.rds\textquotesingle{}) \#for troubleshooting only}
\NormalTok{df\_items }\OtherTok{\textless{}{-}} \FunctionTok{read\_rds}\NormalTok{(}\StringTok{\textquotesingle{}analysis/SGC5A/data/1{-}study{-}level/sgc5\_items.rds\textquotesingle{}}\NormalTok{)}
\end{Highlighting}
\end{Shaded}

\begin{Shaded}
\begin{Highlighting}[]
\CommentTok{\# \#HACK WD FOR LOCAL RUNNING?}
\NormalTok{imac }\OtherTok{=} \StringTok{"/Users/amyraefox/Code/SGC{-}Scaffolding\_Graph\_Comprehension/SGC{-}X/ANALYSIS/MAIN"}
\FunctionTok{setwd}\NormalTok{(imac)}

\FunctionTok{source}\NormalTok{(}\StringTok{"analysis/utils/scoring.R"}\NormalTok{)}
\end{Highlighting}
\end{Shaded}

\emph{note: this cell takes approximately 30 minutes to run on the full
df\_items dataframe with 4950 records}

\begin{Shaded}
\begin{Highlighting}[]
\CommentTok{\#RUN THIS \textless{}OR\textgreater{} THE CALCULATE{-}SCORES{-}FORLOOP [not both]}

\CommentTok{\#ALPHEBETIZE RESPONSE}
\NormalTok{df\_items}\SpecialCharTok{$}\NormalTok{response }\OtherTok{=} \FunctionTok{pbmapply}\NormalTok{(reorder\_inplace, df\_items}\SpecialCharTok{$}\NormalTok{response)}

\CommentTok{\#STRATEGY PARTIAL{-}SUBSCORES}
\NormalTok{df\_items}\SpecialCharTok{$}\NormalTok{score\_TRI }\OtherTok{=} \FunctionTok{pbmapply}\NormalTok{(calc\_subscore, df\_items}\SpecialCharTok{$}\NormalTok{q, df\_items}\SpecialCharTok{$}\NormalTok{condition, df\_items}\SpecialCharTok{$}\NormalTok{response, }\AttributeTok{MoreArgs =} \FunctionTok{list}\NormalTok{(}\AttributeTok{keyframe =}\NormalTok{ keys\_tri))}
\NormalTok{df\_items}\SpecialCharTok{$}\NormalTok{score\_ORTH }\OtherTok{=} \FunctionTok{pbmapply}\NormalTok{(calc\_subscore, df\_items}\SpecialCharTok{$}\NormalTok{q, df\_items}\SpecialCharTok{$}\NormalTok{condition, df\_items}\SpecialCharTok{$}\NormalTok{response, }\AttributeTok{MoreArgs =} \FunctionTok{list}\NormalTok{(}\AttributeTok{keyframe =}\NormalTok{ keys\_orth))}
\NormalTok{df\_items}\SpecialCharTok{$}\NormalTok{score\_SAT\_left }\OtherTok{=} \FunctionTok{pbmapply}\NormalTok{(calc\_subscore, df\_items}\SpecialCharTok{$}\NormalTok{q, df\_items}\SpecialCharTok{$}\NormalTok{condition, df\_items}\SpecialCharTok{$}\NormalTok{response, }\AttributeTok{MoreArgs =} \FunctionTok{list}\NormalTok{(}\AttributeTok{keyframe =}\NormalTok{ keys\_satisfice\_left))}
\NormalTok{df\_items}\SpecialCharTok{$}\NormalTok{score\_SAT\_right }\OtherTok{=} \FunctionTok{pbmapply}\NormalTok{(calc\_subscore, df\_items}\SpecialCharTok{$}\NormalTok{q, df\_items}\SpecialCharTok{$}\NormalTok{condition, df\_items}\SpecialCharTok{$}\NormalTok{response, }\AttributeTok{MoreArgs =} \FunctionTok{list}\NormalTok{(}\AttributeTok{keyframe =}\NormalTok{ keys\_satisfice\_right))}
\NormalTok{df\_items}\SpecialCharTok{$}\NormalTok{score\_TV\_max }\OtherTok{=} \FunctionTok{pbmapply}\NormalTok{(calc\_subscore, df\_items}\SpecialCharTok{$}\NormalTok{q, df\_items}\SpecialCharTok{$}\NormalTok{condition, df\_items}\SpecialCharTok{$}\NormalTok{response, }\AttributeTok{MoreArgs =} \FunctionTok{list}\NormalTok{(}\AttributeTok{keyframe =}\NormalTok{ keys\_tversky\_max))}
\NormalTok{df\_items}\SpecialCharTok{$}\NormalTok{score\_TV\_start }\OtherTok{=} \FunctionTok{pbmapply}\NormalTok{(calc\_subscore, df\_items}\SpecialCharTok{$}\NormalTok{q, df\_items}\SpecialCharTok{$}\NormalTok{condition, df\_items}\SpecialCharTok{$}\NormalTok{response, }\AttributeTok{MoreArgs =} \FunctionTok{list}\NormalTok{(}\AttributeTok{keyframe =}\NormalTok{ keys\_tversky\_start))}
\NormalTok{df\_items}\SpecialCharTok{$}\NormalTok{score\_TV\_end }\OtherTok{=} \FunctionTok{pbmapply}\NormalTok{(calc\_subscore, df\_items}\SpecialCharTok{$}\NormalTok{q, df\_items}\SpecialCharTok{$}\NormalTok{condition, df\_items}\SpecialCharTok{$}\NormalTok{response, }\AttributeTok{MoreArgs =} \FunctionTok{list}\NormalTok{(}\AttributeTok{keyframe =}\NormalTok{ keys\_tversky\_end))}
\NormalTok{df\_items}\SpecialCharTok{$}\NormalTok{score\_TV\_duration }\OtherTok{=} \FunctionTok{pbmapply}\NormalTok{(calc\_subscore, df\_items}\SpecialCharTok{$}\NormalTok{q, df\_items}\SpecialCharTok{$}\NormalTok{condition, df\_items}\SpecialCharTok{$}\NormalTok{response, }\AttributeTok{MoreArgs =} \FunctionTok{list}\NormalTok{(}\AttributeTok{keyframe =}\NormalTok{ keys\_tversky\_duration))}

\CommentTok{\#SPECIAL ABSOLUTE SUBSCORES}
\NormalTok{df\_items}\SpecialCharTok{$}\NormalTok{score\_REF }\OtherTok{=} \FunctionTok{pbmapply}\NormalTok{(calc\_refscore, df\_items}\SpecialCharTok{$}\NormalTok{q, df\_items}\SpecialCharTok{$}\NormalTok{response)}
\NormalTok{df\_items}\SpecialCharTok{$}\NormalTok{score\_BOTH }\OtherTok{=} \FunctionTok{as.integer}\NormalTok{((df\_items}\SpecialCharTok{$}\NormalTok{score\_TRI }\SpecialCharTok{==} \DecValTok{1}\NormalTok{) }\SpecialCharTok{\&}\NormalTok{ (df\_items}\SpecialCharTok{$}\NormalTok{score\_ORTH }\SpecialCharTok{==}\DecValTok{1}\NormalTok{))}

\CommentTok{\#ABSOLUTE SCORES}
\NormalTok{df\_items}\SpecialCharTok{$}\NormalTok{score\_ABS }\OtherTok{=} \FunctionTok{as.integer}\NormalTok{(df\_items}\SpecialCharTok{$}\NormalTok{correct) }
\NormalTok{df\_items}\SpecialCharTok{$}\NormalTok{score\_niceABS  }\OtherTok{\textless{}{-}} \FunctionTok{as.integer}\NormalTok{((df\_items}\SpecialCharTok{$}\NormalTok{score\_TRI }\SpecialCharTok{==} \DecValTok{1}\NormalTok{)) }\CommentTok{\#tri doesn\textquotesingle{}t penalize ref or ve{-}area}
\end{Highlighting}
\end{Shaded}

\hypertarget{sec-SGC5A-interpretation}{%
\subsection{Derive Interpretation}\label{sec-SGC5A-interpretation}}

Finally, we use the interpretation subscores to classify the response as
a particular interpretation. This classification algorithm : (1) First
decides if the response matches one or more `special' situations (blank
response, reference point response, both ORTH and TRI) (2) If response
doesn't match a special situation, it compares the individual subscores,
and subscores and decides if they are \emph{discriminant} (i.e.~are the
scores different enough to make a prediction). A discriminant threshold
of 0.5pts (on a scale from -1 to +1 is used) (2) If the variance in
subscores surpasses the threshold, the interpretation is classified
based on the highest subscore (TRIANGULAR, ORTHOGONAL, TVERSKY,
SATISFICE) (3) If the variance does not surpass the threshold, the
interpretation is labelled as ``?'', indicating it cannot be classified,
and is of an unknown interpretation.

The final output is called \texttt{interpretation}.

\begin{Shaded}
\begin{Highlighting}[]
\CommentTok{\# extra copying for troubleshooting safety}
\NormalTok{temp }\OtherTok{\textless{}{-}}\NormalTok{ df\_items }
\NormalTok{temp }\OtherTok{\textless{}{-}} \FunctionTok{derive\_interpretation}\NormalTok{(temp)}
\NormalTok{df\_items }\OtherTok{\textless{}{-}}\NormalTok{ temp }
\end{Highlighting}
\end{Shaded}

\hypertarget{sec-SGC5A-scaledScore}{%
\subsection{Derive Scaled Score}\label{sec-SGC5A-scaledScore}}

The \texttt{interpretation} response variable gives us the finest grain
indication of the reader's understanding of the graph for a particular
question. However, it is a categorical variable, which poses a challenge
for analyzing cumulative performance at the subject level. To address
this challenge, we derive a \emph{scaled\_score} that converts each
possible interpretation to a numeric value on a scale from -1 to +1.
This scaling takes advantage of the observation that each interpretation
can be positioned along a spectrum of understanding from completely
incorrect (orthogonal) to completely correct (triangular). Alternative
interpretations lay somewhere between.

Specifically, we assign the following values to each interpretation:

\begin{itemize}
\tightlist
\item
  (-1) : ORTHOGONAL, SATISFICE (satisfice represents an attempt at an
  orthogonal answer when none is available)
\item
  (-0.5): ? (some alternative that cannot be identified; but meaningful
  that it is not orthogonal)
\item
  (0): REFERENCE POINT, BLANK (indicates the individual thinks there is
  no answer, recognizes that ORTHOGONAL cannot be correct, but does not
  conceive of triangular)
\item
  (+0.5) TVERSKY, BOTH TRI + ORTH (indicates that they ``see'' a
  triangular response, but lack certainty and also select the ORTHOGONAL
  response)
\item
  (+1) TRIANGULAR +1
\end{itemize}

\begin{Shaded}
\begin{Highlighting}[]
\NormalTok{df\_items}\SpecialCharTok{$}\NormalTok{score\_SCALED }\OtherTok{\textless{}{-}} \FunctionTok{calc\_scaled}\NormalTok{(df\_items}\SpecialCharTok{$}\NormalTok{interpretation)}
\end{Highlighting}
\end{Shaded}

\hypertarget{summarize-by-subject-4}{%
\section{SUMMARIZE BY SUBJECT}\label{summarize-by-subject-4}}

Next, we summarize the item level scores by subject in order to
calculate cummulative subscores to be stored on the subject record.

\begin{Shaded}
\begin{Highlighting}[]
\CommentTok{\# \#HACK WD FOR LOCAL RUNNING?}
\NormalTok{imac }\OtherTok{=} \StringTok{"/Users/amyraefox/Code/SGC{-}Scaffolding\_Graph\_Comprehension/SGC{-}X/ANALYSIS/MAIN"}
\FunctionTok{setwd}\NormalTok{(imac)}

\CommentTok{\#import subjects}
\NormalTok{df\_subjects }\OtherTok{\textless{}{-}} \FunctionTok{read\_rds}\NormalTok{(}\StringTok{\textquotesingle{}analysis/SGC5A/data/1{-}study{-}level/sgc5\_participants.rds\textquotesingle{}}\NormalTok{) }\SpecialCharTok{\%\textgreater{}\%} \FunctionTok{mutate}\NormalTok{(}\AttributeTok{subject =} \FunctionTok{as.character}\NormalTok{(subject)) }\SpecialCharTok{\%\textgreater{}\%} \FunctionTok{arrange}\NormalTok{(subject)}

\CommentTok{\#make temporary copies for testing safety}
\NormalTok{s }\OtherTok{=}\NormalTok{ df\_subjects}
\NormalTok{i }\OtherTok{=}\NormalTok{ df\_items }

\CommentTok{\#summarize}
\NormalTok{test\_subs }\OtherTok{\textless{}{-}} \FunctionTok{summarise\_bySubject}\NormalTok{(s,i)}
\end{Highlighting}
\end{Shaded}

\begin{verbatim}
`summarise()` has grouped output by 'subject'. You can override using the
`.groups` argument.
\end{verbatim}

\begin{verbatim}
[1] TRUE
[1] TRUE
[1] TRUE
[1] TRUE
[1] TRUE
[1] TRUE
[1] TRUE
\end{verbatim}

\begin{Shaded}
\begin{Highlighting}[]
\NormalTok{df\_subjects }\OtherTok{\textless{}{-}}\NormalTok{ test\_subs}
\end{Highlighting}
\end{Shaded}

We also summarize absolute and scaled score progress at each question in
the task, to explore cumulative performance over the task.

\begin{Shaded}
\begin{Highlighting}[]
\CommentTok{\#GET ABSOLUTE PROGRESS }
\NormalTok{df\_absolute\_progress }\OtherTok{\textless{}{-}} \FunctionTok{progress\_Absolute}\NormalTok{(df\_items)}

\CommentTok{\#GET SCALED PROGRESS}
\NormalTok{df\_scaled\_progress }\OtherTok{\textless{}{-}} \FunctionTok{progress\_Scaled}\NormalTok{(df\_items)}
\end{Highlighting}
\end{Shaded}

\hypertarget{explore-distributions-4}{%
\section{EXPLORE DISTRIBUTIONS}\label{explore-distributions-4}}

\begin{Shaded}
\begin{Highlighting}[]
\FunctionTok{options}\NormalTok{(}\AttributeTok{repr.plot.width =}\DecValTok{9}\NormalTok{, }\AttributeTok{repr.plot.height =}\DecValTok{12}\NormalTok{)}

\CommentTok{\#create temp data frame for visualizations}
\NormalTok{df }\OtherTok{=}\NormalTok{ df\_items }\SpecialCharTok{\%\textgreater{}\%} \FunctionTok{filter}\NormalTok{ (q }\SpecialCharTok{\%nin\%} \FunctionTok{c}\NormalTok{(}\DecValTok{6}\NormalTok{,}\DecValTok{9}\NormalTok{)) }\SpecialCharTok{\%\textgreater{}\%} \FunctionTok{mutate}\NormalTok{(}
  \AttributeTok{score\_niceABS =} \FunctionTok{as.factor}\NormalTok{(score\_niceABS),}
  \AttributeTok{pretty\_condition =} \FunctionTok{recode\_factor}\NormalTok{(condition, }\StringTok{"11115"} \OtherTok{=} \StringTok{"point{-}click"}\NormalTok{),}
  \AttributeTok{pretty\_interpretation =} \FunctionTok{factor}\NormalTok{(interpretation,}
    \AttributeTok{levels =} \FunctionTok{c}\NormalTok{(}\StringTok{"Orthogonal"}\NormalTok{, }\StringTok{"Satisfice"}\NormalTok{, }
               \StringTok{"frenzy"}\NormalTok{,}\StringTok{"?"}\NormalTok{,}
                \StringTok{"reference"}\NormalTok{,}\StringTok{"blank"}\NormalTok{,}
                \StringTok{"Tversky"}\NormalTok{, }\StringTok{"both tri + orth"}\NormalTok{,}
               \StringTok{"Triangular"}\NormalTok{ ))}
\NormalTok{  )}
\end{Highlighting}
\end{Shaded}

\hypertarget{absolute-score-5}{%
\subsection{Absolute Score}\label{absolute-score-5}}

\begin{Shaded}
\begin{Highlighting}[]
\CommentTok{\#DISTRIBUTION ABSOLUTE SCORE FULL }\AlertTok{TASK}
\FunctionTok{gf\_props}\NormalTok{(}\SpecialCharTok{\textasciitilde{}}\NormalTok{score\_niceABS, }\AttributeTok{fill =} \SpecialCharTok{\textasciitilde{}}\NormalTok{pretty\_condition, }\AttributeTok{position =} \FunctionTok{position\_dodge}\NormalTok{(), }\AttributeTok{data =}\NormalTok{ df) }\SpecialCharTok{+}
  \FunctionTok{labs}\NormalTok{( }\AttributeTok{x =} \StringTok{"Absolute Score"}\NormalTok{, }
        \AttributeTok{title =} \StringTok{"Distribution of Absolute Score (all Items)"}\NormalTok{,}
        \AttributeTok{subtitle =} \FunctionTok{paste}\NormalTok{(}\StringTok{""}\NormalTok{),}
        \AttributeTok{y =} \StringTok{"Proportion of Items"}\NormalTok{) }\SpecialCharTok{+}
  \FunctionTok{scale\_fill\_discrete}\NormalTok{(}\AttributeTok{name =} \StringTok{"Condition"}\NormalTok{) }\SpecialCharTok{+}  
  \FunctionTok{theme\_minimal}\NormalTok{()}
\end{Highlighting}
\end{Shaded}

\begin{figure}[H]

{\centering \includegraphics{analysis/SGC5A/2_sgc5A_scoring_files/figure-pdf/DISTR-ABSCORE-1.pdf}

}

\end{figure}

\begin{Shaded}
\begin{Highlighting}[]
\CommentTok{\#DISTRIBUTION ABSOLUTE SCORE BY ITEM}
\FunctionTok{gf\_props}\NormalTok{(}\SpecialCharTok{\textasciitilde{}}\NormalTok{score\_niceABS, }\AttributeTok{fill =} \SpecialCharTok{\textasciitilde{}}\NormalTok{pretty\_condition, }\AttributeTok{position =} \FunctionTok{position\_dodge}\NormalTok{(), }\AttributeTok{data =}\NormalTok{ df)  }\SpecialCharTok{\%\textgreater{}\%} 
  \FunctionTok{gf\_facet\_grid}\NormalTok{(pretty\_condition}\SpecialCharTok{\textasciitilde{}}\NormalTok{q) }\SpecialCharTok{+} 
  \FunctionTok{labs}\NormalTok{( }\AttributeTok{x =} \StringTok{"Absolute Score"}\NormalTok{, }
        \AttributeTok{title =} \StringTok{"Distribution of Absolute Score (by Item)"}\NormalTok{,}
        \AttributeTok{subtitle =} \StringTok{""}\NormalTok{,}
        \AttributeTok{y =} \StringTok{"Proprition of Subjects"}\NormalTok{) }\SpecialCharTok{+}
  \FunctionTok{scale\_fill\_discrete}\NormalTok{(}\AttributeTok{name =} \StringTok{"Condition"}\NormalTok{) }\SpecialCharTok{+}  
  \FunctionTok{theme\_minimal}\NormalTok{()}
\end{Highlighting}
\end{Shaded}

\begin{figure}[H]

{\centering \includegraphics{analysis/SGC5A/2_sgc5A_scoring_files/figure-pdf/DISTR-ABSCORE-2.pdf}

}

\end{figure}

\begin{Shaded}
\begin{Highlighting}[]
\CommentTok{\#DISTRIBUTION ABSOLUTE SCORE BY SUBJECT}
\FunctionTok{gf\_props}\NormalTok{(}\SpecialCharTok{\textasciitilde{}}\NormalTok{s\_NABS, }\AttributeTok{fill =} \SpecialCharTok{\textasciitilde{}}\NormalTok{pretty\_condition, }\AttributeTok{data =}\NormalTok{ df\_subjects) }\SpecialCharTok{\%\textgreater{}\%} 
   \FunctionTok{gf\_facet\_wrap}\NormalTok{(}\SpecialCharTok{\textasciitilde{}}\NormalTok{pretty\_condition) }\SpecialCharTok{+} 
  \FunctionTok{labs}\NormalTok{( }\AttributeTok{x =} \StringTok{"Total Absolute Score"}\NormalTok{, }
        \AttributeTok{title =} \StringTok{"Distribution of Total Absolute Score (by Subject)"}\NormalTok{,}
        \AttributeTok{subtitle =} \StringTok{""}\NormalTok{,}
        \AttributeTok{y =} \StringTok{"Proportion of Subjects"}\NormalTok{) }\SpecialCharTok{+}
  \FunctionTok{scale\_fill\_discrete}\NormalTok{(}\AttributeTok{name =} \StringTok{"Condition"}\NormalTok{) }\SpecialCharTok{+}  
  \FunctionTok{theme\_minimal}\NormalTok{() }\SpecialCharTok{+} \FunctionTok{theme}\NormalTok{(}\AttributeTok{legend.position =} \StringTok{"blank"}\NormalTok{)}
\end{Highlighting}
\end{Shaded}

\begin{figure}[H]

{\centering \includegraphics{analysis/SGC5A/2_sgc5A_scoring_files/figure-pdf/DISTR-ABSCORE-3.pdf}

}

\end{figure}

\hypertarget{scaled-score-5}{%
\subsection{Scaled Score}\label{scaled-score-5}}

\begin{Shaded}
\begin{Highlighting}[]
\FunctionTok{options}\NormalTok{(}\AttributeTok{repr.plot.width =}\DecValTok{9}\NormalTok{, }\AttributeTok{repr.plot.height =}\DecValTok{12}\NormalTok{)}

\CommentTok{\#DISTRIBUTION SCALED SCORE FULL }\AlertTok{TASK}
\FunctionTok{gf\_props}\NormalTok{(}\SpecialCharTok{\textasciitilde{}}\NormalTok{score\_SCALED, }\AttributeTok{fill =} \SpecialCharTok{\textasciitilde{}}\NormalTok{pretty\_condition, }\AttributeTok{position =} \FunctionTok{position\_dodge}\NormalTok{(), }\AttributeTok{data =}\NormalTok{ df) }\SpecialCharTok{+}
  \FunctionTok{labs}\NormalTok{( }\AttributeTok{x =} \StringTok{"Scaled Score"}\NormalTok{, }
        \AttributeTok{title =} \StringTok{"Distribution of Scaled Score (all Items)"}\NormalTok{,}
        \AttributeTok{subtitle =} \StringTok{""}\NormalTok{,}
        \AttributeTok{y =} \StringTok{"Proportion of Items"}\NormalTok{) }\SpecialCharTok{+}
  \FunctionTok{scale\_fill\_discrete}\NormalTok{(}\AttributeTok{name =} \StringTok{"Condition"}\NormalTok{) }\SpecialCharTok{+}  
  \FunctionTok{theme\_minimal}\NormalTok{()}
\end{Highlighting}
\end{Shaded}

\begin{figure}[H]

{\centering \includegraphics{analysis/SGC5A/2_sgc5A_scoring_files/figure-pdf/DISTR-SCALEDSCORE-1.pdf}

}

\end{figure}

\begin{Shaded}
\begin{Highlighting}[]
\CommentTok{\#DISTRIBUTION SCALED SCORE BY ITEM}
\FunctionTok{gf\_props}\NormalTok{(}\SpecialCharTok{\textasciitilde{}}\NormalTok{score\_SCALED, }\AttributeTok{fill =} \SpecialCharTok{\textasciitilde{}}\NormalTok{pretty\_condition, }\AttributeTok{position =} \FunctionTok{position\_dodge}\NormalTok{(), }\AttributeTok{data =}\NormalTok{ df)  }\SpecialCharTok{\%\textgreater{}\%} 
  \FunctionTok{gf\_facet\_grid}\NormalTok{(q}\SpecialCharTok{\textasciitilde{}}\NormalTok{pretty\_condition) }\SpecialCharTok{+} 
  \FunctionTok{labs}\NormalTok{( }\AttributeTok{x =} \StringTok{"Scaled Score"}\NormalTok{, }
        \AttributeTok{title =} \StringTok{"Distribution of Scaled Score (by Item)"}\NormalTok{,}
        \AttributeTok{subtitle =} \StringTok{""}\NormalTok{,}
        \AttributeTok{y =} \StringTok{"Proportion of Subjects"}\NormalTok{) }\SpecialCharTok{+}
  \FunctionTok{scale\_fill\_discrete}\NormalTok{(}\AttributeTok{name =} \StringTok{"Condition"}\NormalTok{) }\SpecialCharTok{+}  \FunctionTok{scale\_y\_continuous}\NormalTok{(}\AttributeTok{breaks=}\FunctionTok{c}\NormalTok{(}\DecValTok{0}\NormalTok{,}\FloatTok{0.5}\NormalTok{)) }\SpecialCharTok{+} 
  \FunctionTok{theme\_minimal}\NormalTok{() }\SpecialCharTok{+} \FunctionTok{theme}\NormalTok{(}\AttributeTok{legend.position=}\StringTok{"blank"}\NormalTok{)}
\end{Highlighting}
\end{Shaded}

\begin{figure}[H]

{\centering \includegraphics{analysis/SGC5A/2_sgc5A_scoring_files/figure-pdf/DISTR-SCALEDSCORE-2.pdf}

}

\end{figure}

\begin{Shaded}
\begin{Highlighting}[]
\CommentTok{\#DISTRIBUTION SCALED SCORE BY SUBJECT}
\FunctionTok{gf\_props}\NormalTok{(}\SpecialCharTok{\textasciitilde{}}\NormalTok{s\_SCALED, }\AttributeTok{fill =} \SpecialCharTok{\textasciitilde{}}\NormalTok{pretty\_condition, }\AttributeTok{data =}\NormalTok{ df\_subjects)  }\SpecialCharTok{\%\textgreater{}\%} 
  \FunctionTok{gf\_facet\_grid}\NormalTok{(pretty\_condition }\SpecialCharTok{\textasciitilde{}}\NormalTok{. )}\SpecialCharTok{+}
  \FunctionTok{labs}\NormalTok{( }\AttributeTok{x =} \StringTok{"Total Scaled Score"}\NormalTok{, }
        \AttributeTok{title =} \StringTok{"Distribution of Total Scaled Score (by Subject)"}\NormalTok{,}
        \AttributeTok{subtitle =} \StringTok{""}\NormalTok{,}
        \AttributeTok{y =} \StringTok{"Number of Subjects"}\NormalTok{) }\SpecialCharTok{+}
  \FunctionTok{scale\_fill\_discrete}\NormalTok{(}\AttributeTok{name =} \StringTok{"Condition"}\NormalTok{) }\SpecialCharTok{+}  
  \FunctionTok{theme\_minimal}\NormalTok{()}
\end{Highlighting}
\end{Shaded}

\begin{figure}[H]

{\centering \includegraphics{analysis/SGC5A/2_sgc5A_scoring_files/figure-pdf/DISTR-SCALEDSCORE-3.pdf}

}

\end{figure}

\hypertarget{interpretations-4}{%
\subsection{Interpretations}\label{interpretations-4}}

\begin{Shaded}
\begin{Highlighting}[]
\CommentTok{\#DISTRIBUTION OF INTERPRETATION}
\FunctionTok{gf\_props}\NormalTok{(}\SpecialCharTok{\textasciitilde{}}\NormalTok{pretty\_interpretation, }\AttributeTok{fill =} \SpecialCharTok{\textasciitilde{}}\NormalTok{pretty\_condition, }\AttributeTok{data =}\NormalTok{ df) }\SpecialCharTok{\%\textgreater{}\%} 
  \FunctionTok{gf\_facet\_grid}\NormalTok{( pretty\_condition }\SpecialCharTok{\textasciitilde{}}\NormalTok{ ., }\AttributeTok{labeller =}\NormalTok{ label\_both) }\SpecialCharTok{+} 
  \FunctionTok{labs}\NormalTok{( }\AttributeTok{title =} \StringTok{"Distribution of Interpretations (across Task)"}\NormalTok{,}
        \AttributeTok{x =} \StringTok{"Graph Interpretation"}\NormalTok{,}
        \AttributeTok{y =} \StringTok{"Proportion of Responses"}\NormalTok{,}
        \AttributeTok{subtitle =} \StringTok{""}\NormalTok{) }\SpecialCharTok{+} 
  \FunctionTok{theme\_minimal}\NormalTok{() }\SpecialCharTok{+} \FunctionTok{theme}\NormalTok{(}\AttributeTok{legend.position =} \StringTok{"blank"}\NormalTok{)}
\end{Highlighting}
\end{Shaded}

\begin{figure}[H]

{\centering \includegraphics{analysis/SGC5A/2_sgc5A_scoring_files/figure-pdf/DISTR-INTERPRETATIONS-1.pdf}

}

\end{figure}

\begin{Shaded}
\begin{Highlighting}[]
\CommentTok{\#DISTRIBUTION OF INTERPRETATION ACROSS ITEMS}
\FunctionTok{gf\_propsh}\NormalTok{(}\SpecialCharTok{\textasciitilde{}}\NormalTok{ pretty\_interpretation, }\AttributeTok{fill =} \SpecialCharTok{\textasciitilde{}}\NormalTok{pretty\_condition, }\AttributeTok{data =}\NormalTok{ df) }\SpecialCharTok{\%\textgreater{}\%} 
  \FunctionTok{gf\_facet\_grid}\NormalTok{( pretty\_condition}\SpecialCharTok{\textasciitilde{}}\NormalTok{q) }\SpecialCharTok{+} 
  \FunctionTok{labs}\NormalTok{( }\AttributeTok{title =} \StringTok{"Distribution of Interpretations (by Item)"}\NormalTok{,}
        \AttributeTok{subtitle =} \StringTok{""}\NormalTok{,}
        \AttributeTok{y =} \StringTok{"Interpretation"}\NormalTok{, }\AttributeTok{x =} \StringTok{"Proportion of Subjects"}\NormalTok{) }\SpecialCharTok{+} \FunctionTok{theme\_minimal}\NormalTok{() }\SpecialCharTok{+} \FunctionTok{theme}\NormalTok{(}\AttributeTok{legend.position =} \StringTok{"blank"}\NormalTok{)}
\end{Highlighting}
\end{Shaded}

\begin{figure}[H]

{\centering \includegraphics{analysis/SGC5A/2_sgc5A_scoring_files/figure-pdf/DISTR-INTERPRETATIONS-2.pdf}

}

\end{figure}

\begin{Shaded}
\begin{Highlighting}[]
\CommentTok{\#DISTRIBUTION OF INTERPRETATION TYPE ACROSS ITEMS}
\FunctionTok{gf\_propsh}\NormalTok{(}\SpecialCharTok{\textasciitilde{}}\NormalTok{ high\_interpretation, }\AttributeTok{fill =} \SpecialCharTok{\textasciitilde{}}\NormalTok{pretty\_condition, }\AttributeTok{data =}\NormalTok{ df) }\SpecialCharTok{\%\textgreater{}\%} 
  \FunctionTok{gf\_facet\_grid}\NormalTok{( pretty\_condition}\SpecialCharTok{\textasciitilde{}}\NormalTok{q) }\SpecialCharTok{+} 
  \FunctionTok{labs}\NormalTok{( }\AttributeTok{title =} \StringTok{"Distribution of Interpretations (by Item)"}\NormalTok{,}
        \AttributeTok{subtitle =} \StringTok{""}\NormalTok{,}
        \AttributeTok{y =} \StringTok{"Interpretation"}\NormalTok{, }\AttributeTok{x =} \StringTok{"Proportion of Subjects"}\NormalTok{) }\SpecialCharTok{+} \FunctionTok{theme\_minimal}\NormalTok{() }\SpecialCharTok{+} \FunctionTok{theme}\NormalTok{(}\AttributeTok{legend.position =} \StringTok{"blank"}\NormalTok{)}
\end{Highlighting}
\end{Shaded}

\begin{figure}[H]

{\centering \includegraphics{analysis/SGC5A/2_sgc5A_scoring_files/figure-pdf/DISTR-INTERPRETATIONS-3.pdf}

}

\end{figure}

\hypertarget{progress-over-task-4}{%
\subsection{Progress over Task}\label{progress-over-task-4}}

\begin{Shaded}
\begin{Highlighting}[]
\CommentTok{\#VISUALIZE progress over time ABSOLUTE score }
\FunctionTok{ggplot}\NormalTok{(}\AttributeTok{data =}\NormalTok{ df\_absolute\_progress, }\FunctionTok{aes}\NormalTok{(}\AttributeTok{x =}\NormalTok{ question, }\AttributeTok{y =}\NormalTok{ score, }\AttributeTok{group =}\NormalTok{ subject, }\AttributeTok{alpha =} \FloatTok{0.01}\NormalTok{)) }\SpecialCharTok{+} 
 \FunctionTok{geom\_line}\NormalTok{(}\AttributeTok{position=}\FunctionTok{position\_jitter}\NormalTok{(}\AttributeTok{w=}\FloatTok{0.15}\NormalTok{, }\AttributeTok{h=}\FloatTok{0.00}\NormalTok{), }\AttributeTok{size=}\FloatTok{0.1}\NormalTok{) }\SpecialCharTok{+}
 \FunctionTok{facet\_wrap}\NormalTok{( }\SpecialCharTok{\textasciitilde{}}\NormalTok{ pretty\_condition) }\SpecialCharTok{+} 
 \FunctionTok{labs}\NormalTok{ (}\AttributeTok{title =} \StringTok{"Cumulative Absolute Score over sequence of task"}\NormalTok{, }\AttributeTok{x =} \StringTok{"Question"}\NormalTok{ , }\AttributeTok{y =} \StringTok{"Cumulative Absolute Score"}\NormalTok{) }\SpecialCharTok{+} 
 \FunctionTok{scale\_x\_continuous}\NormalTok{(}\AttributeTok{breaks =} \FunctionTok{c}\NormalTok{(}\DecValTok{1}\NormalTok{,}\DecValTok{2}\NormalTok{,}\DecValTok{3}\NormalTok{,}\DecValTok{4}\NormalTok{,}\DecValTok{5}\NormalTok{,}\DecValTok{6}\NormalTok{,}\DecValTok{7}\NormalTok{,}\DecValTok{8}\NormalTok{,}\DecValTok{9}\NormalTok{,}\DecValTok{10}\NormalTok{,}\DecValTok{11}\NormalTok{,}\DecValTok{12}\NormalTok{,}\DecValTok{13}\NormalTok{)) }\SpecialCharTok{+}
 \FunctionTok{theme\_minimal}\NormalTok{() }\SpecialCharTok{+} \FunctionTok{theme}\NormalTok{(}\AttributeTok{legend.position =} \StringTok{"blank"}\NormalTok{)}
\end{Highlighting}
\end{Shaded}

\begin{figure}[H]

{\centering \includegraphics{analysis/SGC5A/2_sgc5A_scoring_files/figure-pdf/VIZ-PROGRESS-1.pdf}

}

\end{figure}

\begin{Shaded}
\begin{Highlighting}[]
\CommentTok{\#VISUALIZE progress over time SCALED score }
\FunctionTok{ggplot}\NormalTok{(}\AttributeTok{data =}\NormalTok{ df\_scaled\_progress, }\FunctionTok{aes}\NormalTok{(}\AttributeTok{x =}\NormalTok{ question, }\AttributeTok{y =}\NormalTok{ score, }\AttributeTok{group =}\NormalTok{ subject, }\AttributeTok{alpha =} \FloatTok{0.01}\NormalTok{)) }\SpecialCharTok{+} 
 \FunctionTok{geom\_line}\NormalTok{(}\AttributeTok{position=}\FunctionTok{position\_jitter}\NormalTok{(}\AttributeTok{w=}\FloatTok{0.15}\NormalTok{, }\AttributeTok{h=}\FloatTok{0.00}\NormalTok{), }\AttributeTok{size=}\FloatTok{0.1}\NormalTok{) }\SpecialCharTok{+}
 \FunctionTok{facet\_wrap}\NormalTok{( }\SpecialCharTok{\textasciitilde{}}\NormalTok{ pretty\_condition) }\SpecialCharTok{+} 
 \FunctionTok{labs}\NormalTok{ (}\AttributeTok{title =} \StringTok{"Cumulative Scaled Score over sequence of task"}\NormalTok{, }\AttributeTok{x =} \StringTok{"Question"}\NormalTok{ , }\AttributeTok{y =} \StringTok{"Cumulative Scaled Score"}\NormalTok{) }\SpecialCharTok{+} 
 \FunctionTok{scale\_x\_continuous}\NormalTok{(}\AttributeTok{breaks =} \FunctionTok{c}\NormalTok{(}\DecValTok{1}\NormalTok{,}\DecValTok{2}\NormalTok{,}\DecValTok{3}\NormalTok{,}\DecValTok{4}\NormalTok{,}\DecValTok{5}\NormalTok{,}\DecValTok{6}\NormalTok{,}\DecValTok{7}\NormalTok{,}\DecValTok{8}\NormalTok{,}\DecValTok{9}\NormalTok{,}\DecValTok{10}\NormalTok{,}\DecValTok{11}\NormalTok{,}\DecValTok{12}\NormalTok{,}\DecValTok{13}\NormalTok{)) }\SpecialCharTok{+}
 \FunctionTok{theme\_minimal}\NormalTok{() }\SpecialCharTok{+} \FunctionTok{theme}\NormalTok{(}\AttributeTok{legend.position =} \StringTok{"blank"}\NormalTok{)}
\end{Highlighting}
\end{Shaded}

\begin{figure}[H]

{\centering \includegraphics{analysis/SGC5A/2_sgc5A_scoring_files/figure-pdf/VIZ-PROGRESS-2.pdf}

}

\end{figure}

\hypertarget{interpretation-subscores-4}{%
\subsection{Interpretation Subscores}\label{interpretation-subscores-4}}

\begin{Shaded}
\begin{Highlighting}[]
\FunctionTok{gf\_histogram}\NormalTok{(}\SpecialCharTok{\textasciitilde{}}\NormalTok{ s\_TRI, }\AttributeTok{fill =} \SpecialCharTok{\textasciitilde{}}\NormalTok{pretty\_condition, }\AttributeTok{data =}\NormalTok{ df\_subjects) }\SpecialCharTok{\%\textgreater{}\%} 
  \FunctionTok{gf\_facet\_wrap}\NormalTok{( }\SpecialCharTok{\textasciitilde{}}\NormalTok{ pretty\_condition) }\SpecialCharTok{+} 
  \FunctionTok{labs}\NormalTok{( }\AttributeTok{title =} \StringTok{"Distribution of Total Triangular Score"}\NormalTok{,}
        \AttributeTok{subtitle =} \StringTok{""}\NormalTok{,}
        \AttributeTok{x =} \StringTok{"Item Triangular Score"}\NormalTok{, }\AttributeTok{y =} \StringTok{"Proportion of Subjects"}\NormalTok{) }\SpecialCharTok{+} 
        \FunctionTok{theme\_minimal}\NormalTok{() }\SpecialCharTok{+} \FunctionTok{theme}\NormalTok{(}\AttributeTok{legend.position =} \StringTok{"blank"}\NormalTok{)}
\end{Highlighting}
\end{Shaded}

\begin{figure}[H]

{\centering \includegraphics{analysis/SGC5A/2_sgc5A_scoring_files/figure-pdf/DIST-SUBSCORES-1.pdf}

}

\end{figure}

\begin{Shaded}
\begin{Highlighting}[]
\FunctionTok{gf\_histogram}\NormalTok{(}\SpecialCharTok{\textasciitilde{}}\NormalTok{ s\_ORTH, }\AttributeTok{fill =} \SpecialCharTok{\textasciitilde{}}\NormalTok{pretty\_condition, }\AttributeTok{data =}\NormalTok{ df\_subjects) }\SpecialCharTok{\%\textgreater{}\%} 
  \FunctionTok{gf\_facet\_wrap}\NormalTok{( }\SpecialCharTok{\textasciitilde{}}\NormalTok{ pretty\_condition) }\SpecialCharTok{+} 
  \FunctionTok{labs}\NormalTok{( }\AttributeTok{title =} \StringTok{"Distribution of Total Orthogonal Score"}\NormalTok{,}
        \AttributeTok{subtitle =} \StringTok{""}\NormalTok{,}
        \AttributeTok{x =} \StringTok{"Item Orthogonal Score"}\NormalTok{, }\AttributeTok{y =} \StringTok{"Proportion of Subjects"}\NormalTok{) }\SpecialCharTok{+} 
        \FunctionTok{theme\_minimal}\NormalTok{() }\SpecialCharTok{+} \FunctionTok{theme}\NormalTok{(}\AttributeTok{legend.position =} \StringTok{"blank"}\NormalTok{)}
\end{Highlighting}
\end{Shaded}

\begin{figure}[H]

{\centering \includegraphics{analysis/SGC5A/2_sgc5A_scoring_files/figure-pdf/DIST-SUBSCORES-2.pdf}

}

\end{figure}

\begin{Shaded}
\begin{Highlighting}[]
\FunctionTok{gf\_histogram}\NormalTok{(}\SpecialCharTok{\textasciitilde{}}\NormalTok{ s\_TVERSKY, }\AttributeTok{fill =} \SpecialCharTok{\textasciitilde{}}\NormalTok{pretty\_condition, }\AttributeTok{data =}\NormalTok{ df\_subjects) }\SpecialCharTok{\%\textgreater{}\%} 
  \FunctionTok{gf\_facet\_wrap}\NormalTok{( }\SpecialCharTok{\textasciitilde{}}\NormalTok{ pretty\_condition) }\SpecialCharTok{+} 
  \FunctionTok{labs}\NormalTok{( }\AttributeTok{title =} \StringTok{"Distribution of Total Tversky Score"}\NormalTok{,}
        \AttributeTok{subtitle =} \StringTok{"Impasse shifts density toward higher Tversky scores"}\NormalTok{,}
        \AttributeTok{x =} \StringTok{"Item Orthogonal Score"}\NormalTok{, }\AttributeTok{y =} \StringTok{"Proportion of Subjects"}\NormalTok{) }\SpecialCharTok{+} 
        \FunctionTok{theme\_minimal}\NormalTok{() }\SpecialCharTok{+} \FunctionTok{theme}\NormalTok{(}\AttributeTok{legend.position =} \StringTok{"blank"}\NormalTok{)}
\end{Highlighting}
\end{Shaded}

\begin{figure}[H]

{\centering \includegraphics{analysis/SGC5A/2_sgc5A_scoring_files/figure-pdf/DIST-SUBSCORES-3.pdf}

}

\end{figure}

\begin{Shaded}
\begin{Highlighting}[]
\FunctionTok{gf\_histogram}\NormalTok{(}\SpecialCharTok{\textasciitilde{}}\NormalTok{ s\_SATISFICE, }\AttributeTok{fill =} \SpecialCharTok{\textasciitilde{}}\NormalTok{pretty\_condition, }\AttributeTok{data =}\NormalTok{ df\_subjects) }\SpecialCharTok{\%\textgreater{}\%} 
  \FunctionTok{gf\_facet\_wrap}\NormalTok{( }\SpecialCharTok{\textasciitilde{}}\NormalTok{ pretty\_condition) }\SpecialCharTok{+} 
  \FunctionTok{labs}\NormalTok{( }\AttributeTok{title =} \StringTok{"Distribution of Total Satisfice Score"}\NormalTok{,}
        \AttributeTok{subtitle =} \StringTok{"Satisficing only occurs in impasse, when no orthogonal response is available"}\NormalTok{,}
        \AttributeTok{x =} \StringTok{"Item Orthogonal Score"}\NormalTok{, }\AttributeTok{y =} \StringTok{"Proportion of Subjects"}\NormalTok{) }\SpecialCharTok{+} 
        \FunctionTok{theme\_minimal}\NormalTok{() }\SpecialCharTok{+} \FunctionTok{theme}\NormalTok{(}\AttributeTok{legend.position =} \StringTok{"blank"}\NormalTok{)}
\end{Highlighting}
\end{Shaded}

\begin{figure}[H]

{\centering \includegraphics{analysis/SGC5A/2_sgc5A_scoring_files/figure-pdf/DIST-SUBSCORES-4.pdf}

}

\end{figure}

\hypertarget{peeking-2}{%
\section{PEEKING}\label{peeking-2}}

\begin{Shaded}
\begin{Highlighting}[]
\FunctionTok{library}\NormalTok{(performance)}
\FunctionTok{library}\NormalTok{(report)}

\NormalTok{sgc3a }\OtherTok{\textless{}{-}} \FunctionTok{read\_rds}\NormalTok{(}\StringTok{"analysis/SGC3A/data/2{-}scored{-}data/sgc3a\_scored\_participants.rds"}\NormalTok{) }\SpecialCharTok{\%\textgreater{}\%} \FunctionTok{filter}\NormalTok{(condition }\SpecialCharTok{==} \StringTok{"111"}\NormalTok{) }\SpecialCharTok{\%\textgreater{}\%}\NormalTok{ dplyr}\SpecialCharTok{::}\FunctionTok{select}\NormalTok{(}\SpecialCharTok{{-}}\NormalTok{pretty\_mode)}


\NormalTok{comb }\OtherTok{\textless{}{-}} \FunctionTok{rbind}\NormalTok{(sgc3a, df\_subjects)  }

\FunctionTok{gf\_histogram}\NormalTok{(}\SpecialCharTok{\textasciitilde{}}\NormalTok{s\_SCALED, }\AttributeTok{data =}\NormalTok{ comb) }\SpecialCharTok{\%\textgreater{}\%} 
  \FunctionTok{gf\_facet\_wrap}\NormalTok{(}\SpecialCharTok{\textasciitilde{}}\NormalTok{pretty\_condition)}
\end{Highlighting}
\end{Shaded}

\begin{figure}[H]

{\centering \includegraphics{analysis/SGC5A/2_sgc5A_scoring_files/figure-pdf/unnamed-chunk-16-1.pdf}

}

\end{figure}

\begin{Shaded}
\begin{Highlighting}[]
\NormalTok{m1 }\OtherTok{\textless{}{-}} \FunctionTok{lm}\NormalTok{(s\_SCALED }\SpecialCharTok{\textasciitilde{}}\NormalTok{ pretty\_condition, }\AttributeTok{data =}\NormalTok{ comb)}
\FunctionTok{summary}\NormalTok{(m1)}
\end{Highlighting}
\end{Shaded}

\begin{verbatim}

Call:
lm(formula = s_SCALED ~ pretty_condition, data = comb)

Residuals:
   Min     1Q Median     3Q    Max 
 -6.57  -5.49  -3.57   1.51  20.51 

Coefficients:
                            Estimate Std. Error t value Pr(>|t|)    
(Intercept)                   -6.427      0.647   -9.93   <2e-16 ***
pretty_conditionpoint-click   -1.086      0.997   -1.09     0.28    
---
Signif. codes:  0 '***' 0.001 '**' 0.01 '*' 0.05 '.' 0.1 ' ' 1

Residual standard error: 8.14 on 271 degrees of freedom
Multiple R-squared:  0.00436,   Adjusted R-squared:  0.000682 
F-statistic: 1.19 on 1 and 271 DF,  p-value: 0.277
\end{verbatim}

\begin{Shaded}
\begin{Highlighting}[]
\FunctionTok{anova}\NormalTok{(m1)}
\end{Highlighting}
\end{Shaded}

\begin{verbatim}
Analysis of Variance Table

Response: s_SCALED
                  Df Sum Sq Mean Sq F value Pr(>F)
pretty_condition   1     78    78.5    1.19   0.28
Residuals        271  17935    66.2               
\end{verbatim}

\begin{Shaded}
\begin{Highlighting}[]
\FunctionTok{report}\NormalTok{(m1)}
\end{Highlighting}
\end{Shaded}

\begin{verbatim}
Warning: 'data_findcols()' is deprecated and will be removed in a future update.
  Its usage is discouraged. Please use 'data_find()' instead.

Warning: 'data_findcols()' is deprecated and will be removed in a future update.
  Its usage is discouraged. Please use 'data_find()' instead.

Warning: 'data_findcols()' is deprecated and will be removed in a future update.
  Its usage is discouraged. Please use 'data_find()' instead.
\end{verbatim}

\begin{verbatim}
We fitted a linear model (estimated using OLS) to predict s_SCALED with pretty_condition (formula: s_SCALED ~ pretty_condition). The model explains a statistically not significant and very weak proportion of variance (R2 = 4.36e-03, F(1, 271) = 1.19, p = 0.277, adj. R2 = 6.82e-04). The model's intercept, corresponding to pretty_condition = control, is at -6.43 (95% CI [-7.70, -5.15], t(271) = -9.93, p < .001). Within this model:

  - The effect of pretty condition [point-click] is statistically non-significant and negative (beta = -1.09, 95% CI [-3.05, 0.88], t(271) = -1.09, p = 0.277; Std. beta = -0.13, 95% CI [-0.37, 0.11])

Standardized parameters were obtained by fitting the model on a standardized version of the dataset. 95% Confidence Intervals (CIs) and p-values were computed using the Wald approximation.
\end{verbatim}

\hypertarget{export-9}{%
\section{EXPORT}\label{export-9}}

Finally, we export the scores for each item (\texttt{df\_items}) and
summarized over subjects (\texttt{df\_subjects}), as well as cumulative
progress dataframes (\texttt{df\_absolute\_progress},
\texttt{df\_scaled\_progress})

\begin{Shaded}
\begin{Highlighting}[]
\CommentTok{\# \#HACK WD FOR LOCAL RUNNING?}
\NormalTok{imac }\OtherTok{=} \StringTok{"/Users/amyraefox/Code/SGC{-}Scaffolding\_Graph\_Comprehension/SGC{-}X/ANALYSIS/MAIN"}
\CommentTok{\# mbp = "/Users/amyfox/Sites/RESEARCH/SGC—Scaffolding Graph Comprehension/SGC{-}X/ANALYSIS/MAIN"}
\FunctionTok{setwd}\NormalTok{(imac)}

\CommentTok{\#SAVE FILES}
\FunctionTok{write.csv}\NormalTok{(df\_subjects,}\StringTok{"analysis/SGC5A/data/2{-}scored{-}data/sgc5a\_scored\_participants.csv"}\NormalTok{, }\AttributeTok{row.names =} \ConstantTok{FALSE}\NormalTok{)}
\FunctionTok{write.csv}\NormalTok{(df\_items,}\StringTok{"analysis/SGC5A/data/2{-}scored{-}data/sgc5a\_scored\_items.csv"}\NormalTok{, }\AttributeTok{row.names =} \ConstantTok{FALSE}\NormalTok{)}
\FunctionTok{write.csv}\NormalTok{(df\_absolute\_progress,}\StringTok{"analysis/SGC5A/data/2{-}scored{-}data/sgc5a\_absolute\_progress.csv"}\NormalTok{, }\AttributeTok{row.names =} \ConstantTok{FALSE}\NormalTok{)}
\FunctionTok{write.csv}\NormalTok{(df\_scaled\_progress,}\StringTok{"analysis/SGC5A/data/2{-}scored{-}data/sgc5a\_scaled\_progress.csv"}\NormalTok{, }\AttributeTok{row.names =} \ConstantTok{FALSE}\NormalTok{)}

\CommentTok{\#SAVE R Data Structures }
\CommentTok{\#export R DATA STRUCTURES (include codebook metadata)}
\NormalTok{rio}\SpecialCharTok{::}\FunctionTok{export}\NormalTok{(df\_subjects, }\StringTok{"analysis/SGC5A/data/2{-}scored{-}data/sgc5a\_scored\_participants.rds"}\NormalTok{) }\CommentTok{\# to R data structure file}
\NormalTok{rio}\SpecialCharTok{::}\FunctionTok{export}\NormalTok{(df\_items, }\StringTok{"analysis/SGC5A/data/2{-}scored{-}data/sgc5a\_scored\_items.rds"}\NormalTok{) }\CommentTok{\# to R data structure file}
\end{Highlighting}
\end{Shaded}

\hypertarget{resources-12}{%
\section{RESOURCES}\label{resources-12}}

\begin{Shaded}
\begin{Highlighting}[]
\FunctionTok{sessionInfo}\NormalTok{()}
\end{Highlighting}
\end{Shaded}

\begin{verbatim}
R version 4.2.1 (2022-06-23)
Platform: x86_64-apple-darwin17.0 (64-bit)
Running under: macOS Big Sur ... 10.16

Matrix products: default
BLAS:   /Library/Frameworks/R.framework/Versions/4.2/Resources/lib/libRblas.0.dylib
LAPACK: /Library/Frameworks/R.framework/Versions/4.2/Resources/lib/libRlapack.dylib

locale:
[1] en_US.UTF-8/en_US.UTF-8/en_US.UTF-8/C/en_US.UTF-8/en_US.UTF-8

attached base packages:
[1] stats     graphics  grDevices utils     datasets  methods   base     

other attached packages:
 [1] report_0.5.1      performance_0.9.1 forcats_0.5.1     stringr_1.4.0    
 [5] dplyr_1.0.9       purrr_0.3.4       readr_2.1.2       tidyr_1.2.0      
 [9] tibble_3.1.7      tidyverse_1.3.1   Hmisc_4.7-0       Formula_1.2-4    
[13] survival_3.3-1    lattice_0.20-45   pbapply_1.5-0     ggformula_0.10.1 
[17] ggridges_0.5.3    scales_1.2.0      ggstance_0.3.5    ggplot2_3.3.6    
[21] kableExtra_1.3.4 

loaded via a namespace (and not attached):
 [1] colorspace_2.0-3    ellipsis_0.3.2      rio_0.5.29         
 [4] htmlTable_2.4.0     parameters_0.18.1   base64enc_0.1-3    
 [7] fs_1.5.2            rstudioapi_0.13     farver_2.1.0       
[10] bit64_4.0.5         fansi_1.0.3         lubridate_1.8.0    
[13] xml2_1.3.3          codetools_0.2-18    splines_4.2.1      
[16] knitr_1.39          polyclip_1.10-0     jsonlite_1.8.0     
[19] broom_0.8.0         cluster_2.1.3       dbplyr_2.2.1       
[22] png_0.1-7           ggforce_0.3.3       effectsize_0.7.0   
[25] compiler_4.2.1      httr_1.4.3          backports_1.4.1    
[28] assertthat_0.2.1    Matrix_1.4-1        fastmap_1.1.0      
[31] cli_3.3.0           tweenr_1.0.2        htmltools_0.5.2    
[34] tools_4.2.1         gtable_0.3.0        glue_1.6.2         
[37] Rcpp_1.0.8.3        cellranger_1.1.0    vctrs_0.4.1        
[40] svglite_2.1.0       insight_0.17.1      xfun_0.31          
[43] openxlsx_4.2.5      rvest_1.0.2         lifecycle_1.0.1    
[46] mosaicCore_0.9.0    MASS_7.3-57         vroom_1.5.7        
[49] hms_1.1.1           parallel_4.2.1      RColorBrewer_1.1-3 
[52] yaml_2.3.5          curl_4.3.2          gridExtra_2.3      
[55] labelled_2.9.1      rpart_4.1.16        latticeExtra_0.6-29
[58] stringi_1.7.6       bayestestR_0.12.1   checkmate_2.1.0    
[61] zip_2.2.0           rlang_1.0.3         pkgconfig_2.0.3    
[64] systemfonts_1.0.4   evaluate_0.15       htmlwidgets_1.5.4  
[67] labeling_0.4.2      bit_4.0.4           tidyselect_1.1.2   
[70] plyr_1.8.7          magrittr_2.0.3      R6_2.5.1           
[73] generics_0.1.2      DBI_1.1.3           pillar_1.7.0       
[76] haven_2.5.0         foreign_0.8-82      withr_2.5.0        
[79] datawizard_0.4.1    nnet_7.3-17         modelr_0.1.8       
[82] crayon_1.5.1        utf8_1.2.2          tzdb_0.3.0         
[85] rmarkdown_2.14      jpeg_0.1-9          grid_4.2.1         
[88] readxl_1.4.0        data.table_1.14.2   reprex_2.0.1       
[91] digest_0.6.29       webshot_0.5.3       munsell_0.5.0      
[94] viridisLite_0.4.0  
\end{verbatim}

\hypertarget{references}{%
\chapter*{References}\label{references}}
\addcontentsline{toc}{chapter}{References}

\hypertarget{refs}{}
\begin{CSLReferences}{1}{0}
\leavevmode\vadjust pre{\hypertarget{ref-schmidtRelationExamineesTrue2021}{}}%
Schmidt, Dennis, Tobias Raupach, Annette Wiegand, Manfred Herrmann, and
Philipp Kanzow. 2021. {``Relation Between Examinees' True Knowledge and
Examination Scores: Systematic Review and Exemplary Calculations on
{Multiple}-{True}-{False} Items.''} \emph{Educational Research Review}
34 (November): 100409.
\url{https://doi.org/10.1016/j.edurev.2021.100409}.

\end{CSLReferences}



\end{document}
